\section{Programs and Recursive and RE Languages}
\label{ProgramsAndRecursiveAndRELanguages}

In this section, we introduce our functional programming language,
and then use it to define the recursive and recursively enumerable
languages.

\subsection{Programs}

Our programming language is statically scoped, i.e., nonlocal names
used by functions are interpreted relative to the environments in
which they are declared.  It is dynamically typed, in that all type
checking happens at runtime.  It is deterministic, in the sense that
every evaluation of a program has the same result.  It is functional,
not imperative (assignment-oriented), i.e., there is no mutable state.
And it is eager, not lazy, in the sense that a function's argument
must be completely evaluated before the function is called.

To say what programs are, we need some preliminary definitions:
\begin{itemize}
\item A \emph{variable} is a nonempty string of letters
  ($\mathsf{a,b,\ldots,z,A,B,\ldots,Z}$) and digits
  ($\mathsf{0,1,\ldots,9}$) that begins with a letter.  We write
  $\ProgVar$ for the set of all variables, and we order variables
  using the restriction of our total ordering on strings to
  $\ProgVar$.

\item A \emph{constant} is one of the strings $\progtrue$, $\progfalse$ and
  $\prognil$, and we write $\ProgConst$ for the set of all constants.

\item A \emph{program operator} is one of the strings
  $\progisNil$, $\progisInt$, $\progisNeg$, $\progisZero$,
  $\progisPos$, $\progisSym$, $\progisStr$, $\progisPair$,
  $\progisLam$, $\progplus$, $\progminus$, $\progcompare$, $\progfst$,
  $\progsnd$, $\progconsSym$, $\progdeconsSym$, $\progsymListToStr$ and
  $\progstrToSymList$, and we write $\ProgOper$ for the set of
  all operators.
\end{itemize}

Programs are trees (see Section~\ref{TreesAndInductiveDefinitions})
whose labels come from the set $\ProgLab$ of \emph{program labels},
which consists of the union of:
\begin{description}
\item[\quad(variable)] $\setof{\progvar(v)}{v\in\ProgVar}$;

\item[\quad(constant)] $\setof{\progconst(\progcon)}{\progcon\in\ProgConst}$;

\item[\quad(integer)] $\setof{\progint(n)}{n\in\ints}$;

\item[\quad(symbol)] $\setof{\progsym(a)}{a\in\Sym}$;

\item[\quad(string)] $\setof{\progstr(x)}{x\in\Str}$;

\item[\quad(pair)] $\{\progpair\}$;

\item[\quad(calculation)]
  $\setof{\progcalc(\progoper)}{\progoper\in\ProgOper}$;

\item[\quad(conditional)] $\{\progcond\}$;

\item[\quad(function application)] $\{\progapp\}$;

\item[\quad(anonymous function)] $\setof{\proglam(v)}{v\in\ProgVar}$
  ($\proglam$ stands for ``lambda'', recalling the notation for
  anonymous functions in the $\lambda$-calculus);

\item[\quad(simple let)] $\setof{\progletSimp(v)}{v\in\ProgVar}$; and

\item[\quad(recursive let)] $\setof{\progletRec(v_1,v_2)}{v_1,v_2\in\ProgVar}$.
\end{description}

Let the set $\Prog$ of \emph{programs} be the least subset
of $\Tree\,\ProgLab$ such that:
\begin{description}
\item[\quad(variable)] for all $v\in\ProgVar$,
  \begin{displaymath}
    \progvar(v)\in\Prog
  \end{displaymath}
  (a leaf);

\item[\quad(constant)] for all $\progcon\in\ProgConst$,
  \begin{displaymath}
   \progconst(\progcon)\in\Prog 
  \end{displaymath}
  (a leaf);

\item[\quad(integer)] for all $n\in\ints$, 
  \begin{displaymath}
    \progint(n)\in\Prog
  \end{displaymath}
  (a leaf);

\item[\quad(symbol)] for all $a\in\Sym$,
  \begin{displaymath}
    \progsym(a)\in\Prog
  \end{displaymath}
  (a leaf);

\item[\quad(string)] for all $x\in\Str$,
  \begin{displaymath}
    \progstr(x)\in\Prog    
  \end{displaymath}
  (a leaf);

\item[\quad(pair)] for all $\pr_1,\pr_2\in\Prog$,
  \begin{displaymath}
   \progpair(\pr_1,\pr_2)\in\Prog 
  \end{displaymath}
  (a node labeled $\progpair$ with two children);

\item[\quad(calculation)] for all $\progoper\in\ProgOper$ and
  $\pr\in\Prog$,
  \begin{displaymath}
    \progcalc(\progoper)(\pr)\in\Prog    
  \end{displaymath}
  (a node labeled $\progcalc(\progoper)$, with one child; we abbreviate it as
  $\progcalc(\progoper,\pr)$);

\item[\quad(conditional)] for all $\pr_1,\pr_2,\pr_3\in\Prog$,
  \begin{displaymath}
   \progcond(\pr_1,\pr_2,\pr_3)\in\Prog 
  \end{displaymath}
  (a node labeled $\progcond$, with three children);

\item[\quad(function application)] for all $\pr_1,\pr_2\in\Prog$,
  \begin{displaymath}
    \progapp(\pr_1, \pr_2)\in\Prog    
  \end{displaymath}
(a node labeled $\progapp$, with two children);

\item[\quad(anonymous function)] for all $v\in\ProgVar$ and
  $\pr\in\Prog$,
  \begin{displaymath}
    \proglam(v)(\pr)\in\Prog    
  \end{displaymath}
  (a node labeled $\proglam(v)$, with one child; we abbreviate it as
  $\proglam(v,\pr)$);

\item[\quad(simple let)] for all $v\in\ProgVar$ and
  $\pr_1,\pr_2\in\Prog$,
  \begin{displaymath}
    \progletSimp(v)(\pr_1,\pr_2)\in\Prog  
  \end{displaymath}
  (a node labeled $\progletSimp(v)$, with two children; we abbreviate
  it as $\progletSimp(v, \pr_1, \pr_2)$); and

\item[\quad(recursive let)] for all $v_1,v_2\in\ProgVar$ and
  $\pr_1,\pr_2\in\Prog$,
  \begin{displaymath}
    \progletRec(v_1,v_2)(\pr_1,\pr_2)\in\Prog
  \end{displaymath}
  (a node labeled $\progletRec(v_1,v_2)$, with two children; we
  abbreviate it as $\progletRec(v_1,v_2,\pr_1,\pr_2)$).
\end{description}
$\Prog$ is countably infinite.

Informally:
\begin{itemize}
\item A pair $\progpair(\pr_1,\pr_2)$ is evaluated by evaluating
  $\pr_1$ and then $\pr_2$.

\item A calculation $\progcalc(\progoper,\pr)$ applies the operator
  $\progoper$ to the result of evaluating $\pr$.

\item A conditional $\progcond(\pr_1,\pr_2,\pr_3)$ first evaluates
  $\pr_1$ to a boolean constant; if this constant is
  $\progconst(\progtrue)$, it evaluates $\pr_2$; and if this constant
  is $\progconst(\progfalse)$, it evaluates $\pr_3$.

\item A function application $\progapp(\pr_1, \pr_2)$ evaluates
  $\pr_1$ to an anonymous function, and then applies this function
  to the result of evaluating $\pr_2$.

\item If an anonymous function $\proglam(v,\pr)$ is applied to a fully
  evaluated argument, then $\pr$ is evaluated in an environment in
  which $v$ is bound to the argument.

\item A simple let $\progletSimp(v, \pr_1, \pr_2)$ is evaluated by
  evaluating $\pr_2$ in an environment in which $v$ is bound to the
  result of evaluating $\pr_1$.

\item A recursive let $\progletRec(v_1,v_2,\pr_1,\pr_2)$ is evaluated
  by evaluating $\pr_2$ in an envionment in which $v_1$ has been
  recursively declared by:
  \begin{displaymath}
    v_1 = \proglam(v_2, \pr_1) .
  \end{displaymath}
\end{itemize}

Lists are represented by pairs:
\begin{displaymath}
  \progpair(\pr_1,\progpair(\pr_2,\,\ldots \progpair(\pr_n,
  \progconst(\prognil))))
\end{displaymath}
represents the $n$-length list whose elements are $\pr_1$, $\pr_2$, \ldots,
$\pr_n$ (so $\progconst(\prognil)$ represents the empty list).

In the Forlan syntax for programs, the elements of $\Prog$ are
written using our standard syntax for trees, except that:
\begin{itemize}
\item The integer arguments to $\progint$ are written as base-10 numerals,
  preceded by \texttt{\tildesym} in the case of negative integers.  The
  integer $0$ is written as \texttt{0}; extra zeros aren't used/allowed.
  Positive integers are written without leading zeros. And negative
  integers are written written as \texttt{\tildesym} followed by a nonempty
  string of digits with no leading zeros.

\item The symbol arguments to $\progsym$, and the string arguments to
  $\progstr$ are written in abbreviated form, as usual.

\item All nodes are abbreviated.  E.g.,
  \begin{displaymath}
    \progletSimp(\mathsf{x})(\progint(-12),\progvar(\mathsf{x}))
  \end{displaymath}
\end{itemize}
is written as
\begin{verbatim}
letSimp(x, int(~12), var(x))
\end{verbatim}

Programs can also be described as strings over the
alphabet consisting of the letters and digits, plus the elements of
\begin{displaymath}
\{\commasym, \percsym, \mytildesym, \openparsym,
\closparsym, \lesssym, \greatsym\} .
\end{displaymath}
A program $\pr$ is described the string formed by writing $\pr$
in Forlan's syntax, using no whitespace, and then performing
the following substitutions:
\begin{itemize}
\item \texttt{,} is replaced by $\commasym$;

\item \texttt{\%} is replaced by $\percsym$;

\item \texttt{\tildesym} is replaced by $\mytildesym$;

\item \texttt{(} is replaced by $\openparsym$;

\item \texttt{)} is replaced by $\closparsym$;

\item \texttt{<} is replaced by $\lesssym$; and

\item \texttt{>} is replaced by $\greatsym$.
\end{itemize}

For example, the program
\begin{gather*}
\progcalc(\progplus,\progpair(\progint(4),\progint(-5)))
\end{gather*}
is described by the string
\begin{gather*}
\mathsf{calc\openparsym plus\commasym pair\openparsym int
\openparsym 4\closparsym} \\
\hspace{-1cm}
\mathsf{\commasym int\openparsym\mytildesym 5\closparsym\closparsym\closparsym} .
\end{gather*}
Every program is described by a unique string, and every
string describes at most one program.  (E.g., the string
$\mathsf{\commasym\closparsym}$ doesn't describe a program.)

The Forlan module \texttt{Var} defines the abstract type (in the top-level
environment) \texttt{var} of variables, along with functions, including:
\begin{verbatim}
val input   : string -> var
val output  : string * var -> unit
val compare : var * var -> order
val equal   : var * var -> bool
\end{verbatim}
The function \texttt{compare} implements our total ordering on
variables, and \texttt{equal} tests whether two
variables are equal.

The module \texttt{VarSet}
\index{VarSet@\texttt{VarSet}}%
defines various functions for processing
finite sets of variables (elements of type \texttt{var\;set}),
\index{set@\texttt{\primesym a~set}}%
\begin{verbatim}
val input    : string -> var set
val output   : string * var set -> unit
val fromList : var list -> var set
val memb     : var * var set -> bool
val subset   : var set * var set -> bool
val equal    : var set * var set -> bool
val union    : var set * var set -> var set
val inter    : var set * var set -> var set
val minus    : var set * var set -> var set
val genUnion : var set list -> var set
val genInter : var set list -> var set
\end{verbatim}
\index{VarSet@\texttt{VarSet}!input@\texttt{input}}%
\index{VarSet@\texttt{VarSet}!output@\texttt{output}}%
\index{VarSet@\texttt{VarSet}!fromList@\texttt{fromList}}%
\index{VarSet@\texttt{VarSet}!memb@\texttt{memb}}%
\index{VarSet@\texttt{VarSet}!subset@\texttt{subset}}%
\index{VarSet@\texttt{VarSet}!equal@\texttt{equal}}%
\index{VarSet@\texttt{VarSet}!union@\texttt{union}}%
\index{VarSet@\texttt{VarSet}!inter@\texttt{inter}}%
\index{VarSet@\texttt{VarSet}!minus@\texttt{minus}}%
\index{VarSet@\texttt{VarSet}!genUnion@\texttt{genUnion}}%
\index{VarSet@\texttt{VarSet}!genInter@\texttt{genInter}}%
The total ordering associated with sets of variables is our total
ordering on variables.  Sets of variables are expressed in Forlan's syntax
as sequences of variables, separated by commas.

The function \texttt{fromList} returns a set with the same elements of
the list of variables it is called with.  The function \texttt{memb}
tests whether a variable is a member (element) of a set of variables,
\texttt{subset} tests whether a first set of variables is a subset of a
second one, and \texttt{equal} tests whether two sets of variables are
equal.  The functions \texttt{union}, \texttt{inter} and
\texttt{minus} compute the union, intersection and difference of two
sets of variables.  The function \texttt{genUnion} computes the
generalized intersection of a list of sets of variables $\xss$,
returning the set of all variables appearing in at least one element of
$\xss$.  And, the function \texttt{genInter} computes the generalized
intersection of a nonempty list of sets of variables $\xss$, returning
the set of all variables appearing in all elements of $\xss$.

The Forlan module \texttt{Prog} defines the abstract type (in the top-level
environment) \texttt{prog} of programs, along with a number of types and
functions, including:
\begin{verbatim}
datatype const = True | False | Nil
datatype oper = IsNil | IsInt | IsNeg | IsZero | IsPos | IsSym
              | IsStr | IsPair | IsLam | Plus | Minus | Compare
              | Fst | Snd | ConsSym | DeconsSym | SymListToStr
              | StrToSymList
val var     : Var.var -> prog
val const   : const -> prog
val int     : IntInf.int -> prog
val sym     : sym -> prog
val str     : str -> prog
val pair    : prog * prog -> prog
val calc    : oper * prog -> prog
val cond    : prog * prog * prog -> prog
val app     : prog * prog -> prog
val lam     : var * prog -> prog
val letSimp : var * prog * prog -> prog
val letRec  : var * var * prog * prog -> prog
val input   : string -> prog
val output  : string * prog -> unit
val equal   : prog * prog -> bool
val height  : prog -> int
val size    : prog -> int
val fromStr : str -> prog
val toStr   : prog -> str
\end{verbatim}
\texttt{const} and \texttt{oper} are the datatypes of program constants
and operators.
The function \texttt{var} takes in a program variable $v$, and
returns the program $\progvar(v)$.
The function \texttt{const} takes in a program constant $\progcon$,
and returns the program $\progconst(\progcon)$.
The function \texttt{int} takes in a (infinite precision) integer $n$,
and returns the program $\progint(n)$.
The function \texttt{sym} takes in a symbol $a$,
and returns the program $\progsym(a)$.
The function \texttt{str} takes in a string $x$,
and returns the program $\progstr(x)$.
The function \texttt{pair} takes in a pair $(\pr_1,\pr_2)$ of programs,
and returns the program $\progpair(\pr_1,\pr_2)$.
The function \texttt{calc} takes in a pair $(\progoper,\pr)$ of a
program operator and a program, and returns the program
$\progcalc(\progoper,\pr)$.
The function \texttt{cond} takes in a triple $(\pr_1,\pr_2,\pr_3)$ of programs,
and returns the program $\progcond(\pr_1,\pr_2,\pr_3)$.
The function \texttt{app} takes in a pair $(\pr_1,\pr_2)$ of programs,
and returns the program $\progapp(\pr_1,\pr_2)$.
The function \texttt{lam} takes in a pair $(v,pr)$ of a variable $v$
and a program $\pr$, and returns the program
$\proglam(v,\pr)$.
The function \texttt{letSimp} takes in a triple $(v,\pr_1,\pr_2)$ of a
variable $v$ and programs $\pr_1$ and $\pr_2$, and returns the program
$\progletSimp(v,\pr_1,\pr_2)$.
The function \texttt{letRec} takes in a quadruple
$(v_1,v_2,pr_1,\pr_2)$ of variables $v_1$ and $v_2$ and programs
$\pr_1$ and $\pr_2$, and returns the program
$\progletRec(v_1,v_2,\pr_1,\pr_2)$.
The function \texttt{equal} tests two programs for equality.
The functions \texttt{height} and and \texttt{size} return the height
and size, respectively, of a program.
The function \texttt{fromStr} issues an error message if its argument
isn't a string over the alphabet (given above) for describing
programs, or if it is a string over this alphabet, but doesn't
describe a program; otherwise it returns the unique program described
by its argument.  And the function \texttt{toStr} converts a program
to the unique string describing it.

For example, we can proceed as follows:
\begin{list}{}
{\setlength{\leftmargin}{\leftmargini}
\setlength{\rightmargin}{0cm}
\setlength{\itemindent}{0cm}
\setlength{\listparindent}{0cm}
\setlength{\itemsep}{0cm}
\setlength{\parsep}{0cm}
\setlength{\labelsep}{0cm}
\setlength{\labelwidth}{0cm}
\catcode`\#=12
\catcode`\$=12
\catcode`\%=12
\catcode`\^=12
\catcode`\_=12
\catcode`\.=12
\catcode`\?=12
\catcode`\!=12
\catcode`\&=12
\ttfamily}
\small
\item[]\textsl{-\ }val\ pr\ =\ Prog.input\ "";
\item[]\textsl{@\ }calc(plus,\ pair(int(\symbol{'176}6),\ int(5)))
\item[]\textsl{@\ }.
\item[]\textsl{val\ pr\ =\ -\ :\ prog}
\item[]\textsl{-\ }Prog.height\ pr;
\item[]\textsl{val\ it\ =\ 2\ :\ int}
\item[]\textsl{-\ }Prog.size\ pr;
\item[]\textsl{val\ it\ =\ 4\ :\ int}
\item[]\textsl{-\ }val\ x\ =\ Prog.toStr\ pr;
\item[]\textsl{val\ x\ =}
\item[]\textsl{\ \ \symbol{'133}-,-,-,-,-,-,-,-,-,-,-,-,-,-,-,-,-,-,-,-,-,-,-,-,-,-,-,-,-,-,-\symbol{'135}}
\item[]\textsl{\ \ :\ str}
\item[]\textsl{-\ }Str.output("",\ x);
\item[]\textsl{calc<openPar>plus<comma>pair<openPar>int<openPar><tilde>6<closPar>}
\item[]\textsl{<comma>int<openPar>5<closPar><closPar><closPar>}
\item[]\textsl{val\ it\ =\ ()\ :\ unit}
\item[]\textsl{-\ }val\ pr'\ =\ Prog.fromStr\ x;
\item[]\textsl{val\ pr'\ =\ -\ :\ prog}
\item[]\textsl{-\ }Prog.equal(pr',\ pr);
\item[]\textsl{val\ it\ =\ true\ :\ bool}
\end{list}


The Java program JForlan, can be used to view and edit program trees.
It can be invoked directly, or run via Forlan.  See the Forlan website
for more information.

\subsection{Program Meaning}

Variables are \emph{bound} (declared) in programs, as follows:
\begin{itemize}
\item $\proglam(v, \pr)$ binds $v$ in $\pr$;

\item $\progletSimp(v, \pr_1, \pr_2)$ binds $v$ in $\pr_2$; and

\item $\progletSimp(v_1, v_2, \pr_1, \pr_2)$ binds $v_1$ in $\pr_1$ and
  $\pr_2$, and binds $v_2$ in $\pr_1$.
\end{itemize}
We say that an occurrence of a variable in a program is \emph{free} iff
it isn't bound.

Formally, we define a function $\free\in\Prog\fun
\setof{X\sub\ProgVar}{X\eqtxtl{is finite}}$ by structural recursion:
\begin{itemize}
\item $\free(\progvar(v)) = \emptyset$, for all $v\in\ProgVar$;

\item $\free(\progconst(\progcon)) = \emptyset$, for all
  $\progcon\in\ProgConst$;

\item $\free(\progint(n)) = \emptyset$, for all $n\in\ints$;

\item $\free(\progsym(a)) = \emptyset$, for all $a\in\Sym$;

\item $\free(\progstr(x)) = \emptyset$, for all $x\in\Str$;

\item $\free(\progpair(\pr_1,\pr_2)) = \free\,\pr_1\cup\free\,\pr_2$,
  for all $\pr_1,\pr_2\in\Prog$;

\item $\free(\progcalc(\progoper,\pr)) = \free\,\pr$,
  for all $\progoper\in\ProgOper$ and $\pr\in\Prog$;

\item $\free(\progcond(\pr_1,\pr_2,\pr_3)) =
\free\,\pr_1\cup\free\,\pr_2\cup\free\,\pr_3$,
  for all $\pr_1,\pr_2,\pr_3\in\Prog$;

\item $\free(\progapp(\pr_1,\pr_2)) = \free\,\pr_1\cup\free\,\pr_2$,
  for all $\pr_1,\pr_2\in\Prog$;

\item $\free(\proglam(v,\pr)) = \free\,\pr - \{v\}$,
  for all $v\in\ProgVar$ and $\pr\in\Prog$;

\item $\free(\progletSimp(v,\pr_1,\pr_2)) =
  \free\,\pr_1\cup(\free\,\pr_2 - \{v\})$,
  for all $v\in\ProgVar$ and $\pr_1,\pr_2\in\Prog$; and

\item $\free(\progletRec(v_1,v_2,\pr_1,\pr_2)) =
  (\free\,\pr_1 - \{v_1,v_2\})\cup (\free\,\pr_2 - \{v_1\})$,
  for all $v_1,v_2\in\ProgVar$ and $\pr_1,\pr_2\in\Prog$.
\end{itemize}
If $v\in\ProgVar$ and $pr\in\Prog$, we say that $v$ \emph{is free in}
$\pr$ iff $v\in\free\,pr$.

A program $\pr$ is \emph{closed} iff it has no free variables, i.e.,
$\free\,\pr = \emptyset$.  We write $\CP$ for the set of all closed
programs.

The module \texttt{Prog} also defines the following type and
functions:
\begin{verbatim}
val free : prog -> var set
type cp
val toClosed   : prog -> cp
val fromClosed : cp -> prog
\end{verbatim}
The function \texttt{free} returns the free variables of its argument.
The type \texttt{cp} (in the top-level environment) is the type of
closed programs.  The function \texttt{toClosed} issues an error message
if its argument isn't closed; otherwise, it returns its argument.
And the function \texttt{fromClosed} simply returns its argument.

For example, we can proceed as follows:
\begin{list}{}
{\setlength{\leftmargin}{\leftmargini}
\setlength{\rightmargin}{0cm}
\setlength{\itemindent}{0cm}
\setlength{\listparindent}{0cm}
\setlength{\itemsep}{0cm}
\setlength{\parsep}{0cm}
\setlength{\labelsep}{0cm}
\setlength{\labelwidth}{0cm}
\catcode`\#=12
\catcode`\$=12
\catcode`\%=12
\catcode`\^=12
\catcode`\_=12
\catcode`\.=12
\catcode`\?=12
\catcode`\!=12
\catcode`\&=12
\ttfamily}
\small
\item[]\textsl{-\ }val\ pr1\ =\ Prog.input\ "";
\item[]\textsl{@\ }lam(x,\ letSimp(y,\ var(x),\ app(var(z),\ var(w))))
\item[]\textsl{@\ }.
\item[]\textsl{val\ pr1\ =\ -\ :\ prog}
\item[]\textsl{-\ }VarSet.output("",\ Prog.free\ pr1);
\item[]\textsl{w,\ z}
\item[]\textsl{val\ it\ =\ ()\ :\ unit}
\item[]\textsl{-\ }val\ pr2\ =\ Prog.input\ "";
\item[]\textsl{@\ }letRec
\item[]\textsl{@\ }(x,\ y,
\item[]\textsl{@\ }\ app(var(x),\ app(var(y),\ var(z))),
\item[]\textsl{@\ }\ app(var(x),\ var(w)))
\item[]\textsl{@\ }.
\item[]\textsl{val\ pr2\ =\ -\ :\ prog}
\item[]\textsl{-\ }VarSet.output("",\ Prog.free\ pr2);
\item[]\textsl{w,\ z}
\item[]\textsl{val\ it\ =\ ()\ :\ unit}
\item[]\textsl{-\ }Prog.toClosed\ pr2;
\item[]\textsl{program\ has\ free\ variables:\ "w,\ z"}
\item[]
\item[]\textsl{uncaught\ exception\ Error}
\item[]\textsl{-\ }val\ pr'\ =\ Prog.input\ "";
\item[]\textsl{@\ }lam(x,\ letSimp(y,\ var(x),\ app(var(x),\ var(y))))
\item[]\textsl{@\ }.
\item[]\textsl{val\ pr'\ =\ -\ :\ prog}
\item[]\textsl{-\ }val\ cp\ =\ Prog.toClosed\ pr';
\item[]\textsl{val\ cp\ =\ -\ :\ cp}
\item[]\textsl{-\ }val\ pr''\ =\ Prog.fromClosed\ cp;
\item[]\textsl{val\ pr''\ =\ -\ :\ prog}
\item[]\textsl{-\ }Prog.equal(pr'',\ pr');
\item[]\textsl{val\ it\ =\ true\ :\ bool}
\end{list}


Next, we define a function
$\subst\in\CP\times\ProgVar\times\Prog\fun\Prog$ for substituting a
closed program for all of the free occurrences of a variable in a
program.  $\subst$ is defined by structural recursion on its third
argument, as follows:
\begin{itemize}
\item for all $\pr'\in\Prog$, $v'\in\ProgVar$ and $v\in\ProgVar$,
\begin{displaymath}
\subst(\pr',v',\progvar(v)) =
\casesdef{\pr',}{\eqtxtr{if} v'=v ,}%
{\progvar(v),}{\eqtxtr{if} v'\neq v ;}
\end{displaymath}

\item for all $\pr'\in\CP$, $v'\in\ProgVar$ and $\progcon\in\ProgConst$,
  \begin{displaymath}
    \subst(\pr',v',\progconst(\progcon)) = \progconst(\progcon) ;
  \end{displaymath}

\item for all $\pr'\in\CP$, $v'\in\ProgVar$ and $n\in\ints$,
  \begin{displaymath}
    \subst(\pr',v',\progint(n)) = \progint(n) ;
  \end{displaymath}

\item for all $\pr'\in\CP$, $v'\in\ProgVar$ and $a\in\Sym$,
  \begin{displaymath}
    \subst(\pr',v',\progsym(a)) = \progsym(a) ;
  \end{displaymath}

\item for all $\pr'\in\CP$, $v'\in\ProgVar$ and $x\in\Str$,
  \begin{displaymath}
    \subst(\pr',v',\progstr(x)) = \progstr(x) ;
  \end{displaymath}

\item for all $\pr'\in\CP$, $v'\in\ProgVar$ and $\pr_1,\pr_2\in\Prog$,
  \begin{multline*}
    \subst(\pr',v',\progpair(\pr_1,\pr_2)) \\
    = \progpair(\subst(\pr',v',\pr_1),\subst(\pr',v',\pr_2)) ;
  \end{multline*}

\item for all $\pr'\in\CP$, $v'\in\ProgVar$, $\progoper\in\ProgOper$ and
  $\pr\in\Prog$,
  \begin{displaymath}
    \subst(\pr',v',\progcalc(\progoper, \pr)) =
    \progcalc(\progoper,\subst(\pr',v',\pr)) ;
  \end{displaymath}

\item for all $\pr'\in\CP$, $v'\in\ProgVar$ and $\pr_1,\pr_2,\pr_3\in\Prog$,
  \begin{multline*}
    \subst(\pr',v',\progcond(\pr_1,\pr_2,\pr_3)) \\
    = \progcond(\subst(\pr',v',\pr_1),\subst(\pr',v',\pr_2),
      \subst(\pr',v',\pr_3)) ;
  \end{multline*}

\item for all $\pr'\in\CP$, $v'\in\ProgVar$ and $\pr_1,\pr_2\in\Prog$,
  \begin{multline*}
    \subst(\pr',v',\progapp(\pr_1,\pr_2)) \\
    = \progapp(\subst(\pr',v',\pr_1),\subst(\pr',v',\pr_2)) ;
  \end{multline*}

\item for all $\pr'\in\CP$, $v'\in\ProgVar$, $v\in\ProgVar$ and
  $\pr\in\Prog$,
  \begin{displaymath}
    \subst(\pr',v',\proglam(v,\pr)) =
    \casesdef{\proglam(v,\pr),}{\eqtxtr{if} v'=v ,}%
    {\proglam(v,\subst(\pr',v',\pr)),}{\eqtxtr{if} v'\neq v ;}
  \end{displaymath}

\item for all $\pr'\in\CP$, $v'\in\ProgVar$, $v\in\ProgVar$ and
  $\pr_1,\pr_2\in\Prog$,
  \begin{multline*}
    \subst(\pr',v',\progletSimp(v,\pr_1,\pr_2)) \\
    = \casesdef{\progletSimp(v,\subst(\pr',v',\pr_1),\pr_2),}%
    {\eqtxtr{if} v'=v,}%
    {\progletSimp(v,\subst(\pr',v',\pr_1),\subst(\pr',v',\pr_2)),}%
    {\eqtxtr{if} v'\neq v;}%
  \end{multline*}

\item for all $\pr'\in\CP$, $v'\in\ProgVar$, $v_1,v_2\in\ProgVar$ and
  $\pr_1,\pr_2\in\Prog$,
  \begin{multline*}
    \subst(\pr',v',\progletRec(v_1,v_2,\pr_1,\pr_2)) \\
    = \left\{ \begin{array}{l}
        \progletRec(v_1,v_2,\pr_1,\pr_2), \\
        \hspace*{.5cm}\eqtxtr{if} v'=v_1, \\
        \progletRec(v_1,v_2,\pr_1,\subst(\pr',v',\pr_2)), \\
        \hspace*{.5cm}\eqtxtr{if} v'\neq v_1\eqtxt{and} v'=v_2 , \\
        \progletRec(v_1,v_2,\subst(\pr',v',\pr_1),
          \subst(\pr',v',\pr_2)), \\
        \hspace*{.5cm}\eqtxtr{if} v'\neq v_1\eqtxt{and} v'\neq v_2 .
      \end{array} \right.
  \end{multline*}
\end{itemize}

The module \texttt{Prog} also defines the function
\begin{verbatim}
val subst : cp * var * prog -> prog
\end{verbatim}
corresponding to $\subst$.  Here are some examples of its use:
\begin{list}{}
{\setlength{\leftmargin}{\leftmargini}
\setlength{\rightmargin}{0cm}
\setlength{\itemindent}{0cm}
\setlength{\listparindent}{0cm}
\setlength{\itemsep}{0cm}
\setlength{\parsep}{0cm}
\setlength{\labelsep}{0cm}
\setlength{\labelwidth}{0cm}
\catcode`\#=12
\catcode`\$=12
\catcode`\%=12
\catcode`\^=12
\catcode`\_=12
\catcode`\.=12
\catcode`\?=12
\catcode`\!=12
\catcode`\&=12
\ttfamily}
\small
\item[]\textsl{-\ }val\ cp\ =\ Prog.toClosed(Prog.fromString\ "const(true)");
\item[]\textsl{val\ cp\ =\ -\ :\ cp}
\item[]\textsl{-\ }val\ pr1\ =\ Prog.input\ "";
\item[]\textsl{@\ }cond(var(x),\ var(y),\ int(4))
\item[]\textsl{@\ }.
\item[]\textsl{val\ pr1\ =\ -\ :\ prog}
\item[]\textsl{-\ }val\ pr1'\ =\ Prog.subst(cp,\ Var.fromString\ "y",\ pr1);
\item[]\textsl{val\ pr1'\ =\ -\ :\ prog}
\item[]\textsl{-\ }Prog.output("",\ pr1');
\item[]\textsl{cond(var(x),\ const(true),\ int(4))}
\item[]\textsl{val\ it\ =\ ()\ :\ unit}
\item[]\textsl{-\ }val\ pr2\ =\ Prog.input\ "";
\item[]\textsl{@\ }letSimp(x,\ var(x),\ pair(var(x),\ var(y)))
\item[]\textsl{@\ }.
\item[]\textsl{val\ pr2\ =\ -\ :\ prog}
\item[]\textsl{-\ }val\ pr2'\ =\ Prog.subst(cp,\ Var.fromString\ "x",\ pr2);
\item[]\textsl{val\ pr2'\ =\ -\ :\ prog}
\item[]\textsl{-\ }Prog.output("",\ pr2');
\item[]\textsl{letSimp(x,\ const(true),\ pair(var(x),\ var(y)))}
\item[]\textsl{val\ it\ =\ ()\ :\ unit}
\item[]\textsl{-\ }val\ pr3\ =\ Prog.input\ "";
\item[]\textsl{@\ }letSimp(x,\ var(y),\ pair(var(x),\ var(y)))
\item[]\textsl{@\ }.
\item[]\textsl{val\ pr3\ =\ -\ :\ prog}
\item[]\textsl{-\ }val\ pr3'\ =\ Prog.subst(cp,\ Var.fromString\ "y",\ pr3);
\item[]\textsl{val\ pr3'\ =\ -\ :\ prog}
\item[]\textsl{-\ }Prog.output("",\ pr3');
\item[]\textsl{letSimp(x,\ const(true),\ pair(var(x),\ const(true)))}
\item[]\textsl{val\ it\ =\ ()\ :\ unit}
\item[]\textsl{-\ }val\ pr4\ =\ Prog.input\ "";
\item[]\textsl{@\ }letRec
\item[]\textsl{@\ }(x,\ y,
\item[]\textsl{@\ }\ app(var(x),\ app(var(y),\ var(z))),
\item[]\textsl{@\ }\ app(var(x),\ app(var(y),\ var(z))))
\item[]\textsl{@\ }.
\item[]\textsl{val\ pr4\ =\ -\ :\ prog}
\item[]\textsl{-\ }val\ pr4'\ =\ Prog.subst(cp,\ Var.fromString\ "x",\ pr4);
\item[]\textsl{val\ pr4'\ =\ -\ :\ prog}
\item[]\textsl{-\ }Prog.output("",\ pr4');
\item[]\textsl{letRec(x,\ y,\ app(var(x),\ app(var(y),\ var(z))),}
\item[]\textsl{\ \ \ \ \ \ \ app(var(x),\ app(var(y),\ var(z))))}
\item[]\textsl{val\ it\ =\ ()\ :\ unit}
\item[]\textsl{-\ }val\ pr4''\ =\ Prog.subst(cp,\ Var.fromString\ "y",\ pr4);
\item[]\textsl{val\ pr4''\ =\ -\ :\ prog}
\item[]\textsl{-\ }Prog.output("",\ pr4'');
\item[]\textsl{letRec(x,\ y,\ app(var(x),\ app(var(y),\ var(z))),}
\item[]\textsl{\ \ \ \ \ \ \ app(var(x),\ app(const(true),\ var(z))))}
\item[]\textsl{val\ it\ =\ ()\ :\ unit}
\item[]\textsl{-\ }val\ pr4'''\ =\ Prog.subst(cp,\ Var.fromString\ "z",\ pr4);
\item[]\textsl{val\ pr4'''\ =\ -\ :\ prog}
\item[]\textsl{-\ }Prog.output("",\ pr4''');
\item[]\textsl{letRec(x,\ y,\ app(var(x),\ app(var(y),\ const(true))),}
\item[]\textsl{\ \ \ \ \ \ \ app(var(x),\ app(var(y),\ const(true))))}
\item[]\textsl{val\ it\ =\ ()\ :\ unit}
\end{list}


Next, we single out certain closed programs as completely evaluated,
or values.  Let the set $\Val$ of \emph{values} be the least subset of
$\CP$ such that:
\begin{description}
\item[\quad(constant)] for all $\progcon\in\ProgConst$,
  $\progconst(\progcon)\in\Val$;

\item[\quad(integer)] for all $n\in\ints$, $\progint(n)\in\Val$;

\item[\quad(symbol)] for all $a\in\Sym$, $\progsym(a)\in\Val$;

\item[\quad(string)] for all $x\in\Str$, $\progstr(x)\in\Val$;

\item[\quad(pair)] for all $\pr_1,\pr_2\in\Val$,
  $\progpair(\pr_1,\pr_2)\in\Val$; and

\item[\quad(anonymous function)] for all $v\in\ProgVar$ and
  $\pr\in\Prog$, if $\free\,\pr\sub\{v\}$, then
  $\proglam(v,\pr)\in\Val$.
\end{description}

The module \texttt{Prog} also defines a function
\begin{verbatim}
val isValue : cp -> bool
\end{verbatim}
that tests whether a closed program is a value.  Here are some
examples of its use:
\begin{list}{}
{\setlength{\leftmargin}{\leftmargini}
\setlength{\rightmargin}{0cm}
\setlength{\itemindent}{0cm}
\setlength{\listparindent}{0cm}
\setlength{\itemsep}{0cm}
\setlength{\parsep}{0cm}
\setlength{\labelsep}{0cm}
\setlength{\labelwidth}{0cm}
\catcode`\#=12
\catcode`\$=12
\catcode`\%=12
\catcode`\^=12
\catcode`\_=12
\catcode`\.=12
\catcode`\?=12
\catcode`\!=12
\catcode`\&=12
\ttfamily}
\small
\item[]\textsl{-\ }val\ pr1\ =\ Prog.input\ "";
\item[]\textsl{@\ }pair(int(4),\ pair(lam(x,\ var(x)),\ const(true)))
\item[]\textsl{@\ }.
\item[]\textsl{val\ pr1\ =\ -\ :\ prog}
\item[]\textsl{-\ }Prog.isValue(Prog.toClosed\ pr1);
\item[]\textsl{val\ it\ =\ true\ :\ bool}
\item[]\textsl{-\ }val\ pr2\ =\ Prog.input\ "";
\item[]\textsl{@\ }app(lam(x,\ var(x)),\ lam(x,\ var(x)))
\item[]\textsl{@\ }.
\item[]\textsl{val\ pr2\ =\ -\ :\ prog}
\item[]\textsl{-\ }Prog.isValue(Prog.toClosed\ pr2);
\item[]\textsl{val\ it\ =\ false\ :\ bool}
\end{list}


To explain the meaning of program operators, we define a function
$\calculate\in\ProgOper\times\Val\fun\Option\,\Val$, which returns
$\none$ to indicate an error, and returns $\some\,\pr$ when the
application of the operator produced the value $\pr$.  We proceed
as follows, using a case analysis on the form of the argument
value:
\begin{itemize}
\item ($(\progisNil, \progconst(\prognil))$)
  Return $\some(\progconst(\progtrue))$.

\item ($(\progisNil, \pr)$, where $\pr\not\in\{\progconst(\prognil)\}$)
  Return $\some(\progconst(\progfalse))$.

\item ($(\progisInt, \progint(n))$, where $n\in\ints$)
  Return $\some(\progconst(\progtrue))$.

\item ($(\progisInt, \pr)$, where $\pr\not\in\setof{\progint(n)}{n\in\ints}$)
  Return $\some(\progconst(\progfalse))$.

\item ($(\progisNeg, \progint(n))$, where $n\in\ints$)
  Return: $\some(\progconst(\progtrue))$, if $n < 0$;
  and $\some(\progconst(\progfalse))$, if $n\geq 0$.

\item ($(\progisZero, \progint(n))$, where $n\in\ints$)
  Return: $\some(\progconst(\progtrue))$, if $n = 0$;
  and $\some(\progconst(\progfalse))$, if $n\neq 0$.

\item ($(\progisPos, \progint(n))$, where $n\in\ints$)
  Return: $\some(\progconst(\progtrue))$, if $n > 0$;
  and $\some(\progconst(\progfalse))$, if $n\leq 0$.

\item ($(\progisSym, \progsym(a))$, where $a\in\Sym$)
  Return $\some(\progconst(\progtrue))$.

\item ($(\progisSym, \pr)$, where $\pr\not\in\setof{\progsym(a)}{a\in\Sym}$)
   Return $\some(\progconst(\progfalse))$.

\item ($(\progisStr, \progstr(x))$, where $x\in\Str$)
  Return $\some(\progconst(\progtrue))$.

\item ($(\progisStr, \pr)$, where $\pr\not\in\setof{\progstr(x)}{x\in\Str}$)
   Return $\some(\progconst(\progfalse))$.

\item ($(\progisPair, \progpair(\pr_1,\pr_2))$, where
  $\pr_1,\pr_2\in\Val$)
  Return $\some(\progconst(\progtrue))$.

\item ($(\progisPair, \pr)$, where $\pr\not\in
  \setof{\progpair(\pr_1,\pr_2)}{\pr_1,\pr_2\in\Val}$)
  Return $\some(\progconst(\progfalse))$.

\item ($(\progisLam, \proglam(v,\pr))$, where
  $v\in\ProgVar$, $\pr\in\Prog$ and $\free\,\pr\sub\{v\}$)
  Return $\some(\progconst(\progtrue))$.

\item ($(\progisLam, \pr)$, where $\pr\not\in
  \setof{\proglam(v,\pr)}{v\in\ProgVar\eqtxt{and}\pr\in\Prog\eqtxt{and}
    \free\,\pr\sub\{v\}}$)
  Return $\some(\progconst(\progfalse))$.

\item ($(\progplus, \progpair(\progint(m),\progint(n)))$, where
  $m,n\in\ints$)
  Return $\some(\progint(m + n))$.

\item ($(\progminus, \progpair(\progint(m),\progint(n)))$, where
  $m,n\in\ints$)
  Return $\some(\progint(m - n))$.

\item ($(\progcompare, \progpair(\progint(m),\progint(n)))$, where
  $m,n\in\ints$)
  Return: $\some\abr(\progint(-1))$, if $m < n$;
  $\some(\progint(0))$, if $m = n$; and
  $\some(\progint(1))$, if $m > n$.

\item ($(\progcompare, \progpair(\progsym(a),\progsym(b)))$, where
  $a,b\in\Sym$)
  Return: $\some\abr(\progint(-1))$, if $a < b$;
  $\some(\progint(0))$, if $a = b$; and
  $\some(\progint(1))$, if $a > b$.

\item ($(\progcompare, \progpair(\progstr(x),\progstr(y)))$, where
  $x,y\in\Str$)
  Return: $\some\abr(\progint(-1))$, if $x < y$;
  $\some(\progint(0))$, if $x = y$; and
  $\some(\progint(1))$, if $x > y$.

\item ($(\progfst, \progpair(\pr_1,\pr_2))$, where
  $\pr_1,\pr_2\in\Val$)
  Return $\some\,\pr_1$.

\item ($(\progsnd, \progpair(\pr_1,\pr_2))$, where
  $\pr_1,\pr_2\in\Val$)
  Return $\some\,\pr_2$.

\item ($(\progconsSym, \progint(n))$, where $n\in[1:62]$) Return
  $\some(\progsym(a))$, where $a$ is the $n$th (counting from $1$)
  element of the following sequence of symbols: $\mathsf{0, \ldots, 9,
  a, \ldots, z, A, \ldots, Z}$.

\item ($(\progconsSym, \progpair(pr_1,\, \ldots\,
  \progpair(\pr_n,\progconst(\prognil))\,\ldots\,))$, where $n\in\nats$
  and $\pr_1,\ldots,\pr_n\in \{\progconst(\prognil)\} \cup
  \setof{\progsym(a)}{a\in\Sym}$)
  Return
  \begin{displaymath}
    \some(\progsym([\,\langle\,] \myconcat f\,pr_1 \myconcat \cdots
    \myconcat f\,pr_n \myconcat [\,\rangle\,])) ,
  \end{displaymath}
  where $f$ is the function from $\{\progconst(\prognil)\} \cup
  \setof{\progsym(a)}{a\in\Sym}$ to $\{[\,,]\} \cup \Sym$ defined by:
  \begin{align*}
    f(\progconst(\prognil)) &= [\,,] , \\
    f(\progsym(a)) &= a , \eqtxt{for all} a\in \Sym .
  \end{align*}
  (Remember that symbols are lists; see
  Section~\ref{SymbolsStringsAlphabetsAndFormalLanguages}.
  If $n=0$, then $\some(\progsym(\langle\rangle))$ is returned.)

\item ($(\progdeconsSym, \progsym(a))$, where
  $a\in\{\mathsf{0,\ldots,9,a,\ldots,z,A,\ldots,Z}\}$)
  Return $\some(\progint(n))$, where $n$ is the position (counting from $1$)
  of $a$ in the following sequence of symbols:
  $\mathsf{0, \ldots, 9, a,\ldots,z, A, \ldots, Z}$.

\item ($(\progdeconsSym, \progsym([\,\langle\,] \myconcat x_1 \myconcat
  \cdots \myconcat x_n \myconcat [\,\rangle\,]))$, where $n\in\nats$
  and $x_1,\ldots,x_n\in \{[\,,]\} \cup \Sym$)
  Return
  \begin{displaymath}
    \some(\progpair(f\,x_1, \,\ldots\, \progpair(f\,x_n,
    \progconst(\prognil))\,\ldots\,)) ,
  \end{displaymath}
  where $f$ is the function from $\{[\,,]\} \cup \Sym$
  to $\{\progconst(\prognil)\} \cup \setof{\progsym(a)}{a\in\Sym}$ defined by
  \begin{align*}
    f[\,,] &= \progconst(\prognil) , \\
    f\,a &= \progsym(a) , \eqtxt{for all} a\in\Sym .
  \end{align*}
  (If $n = 0$, then $\some(\progconst(\prognil))$ is returned.)

\item ($(\progsymListToStr, \progconst(\prognil))$)
  Return $\some(\progstr(\%))$.

\item ($(\progsymListToStr, \progpair(\progsym(a_1),\, \ldots\,
  \progpair(\progsym(a_n),\progconst(\prognil))\,\ldots\,))$, where
  $n\in\nats-\{0\}$ and $a_1,\ldots,a_n\in\Sym$)
  Return $\some(\progstr(a_1\cdots a_n))$.

\item ($(\progstrToSymList, \progstr(\%))$)
  Return $\some(\progconst(\prognil))$.

\item ($(\progstrToSymList, \progstr(a_1\cdots a_n))$,
  where $n\in\nats-\{0\}$ and $a_1,\ldots,a_n\in\Sym$)
  Return
  \begin{displaymath}
    \some(\progpair(\progsym(a_1),\, \ldots\, \progpair(\progsym(a_n),
    \progconst(\prognil))\,\ldots\,)) .
  \end{displaymath}

\item (otherwise) Return $\none$.
\end{itemize}

Now we are able to say how closed programs evaluate.  Let
\begin{displaymath}
  \Step = \{\myvalue, \error\} \cup
  \setof{\next\,\pr}{\pr\in\CP} .
\end{displaymath}
We define a function $\step\in\CP\fun\Step$ using a case analysis
on the form of its argument, $\pr$.  It returns: $\myvalue$, if
$\pr$ is a value; $\error$, if $\pr$ isn't a value, but can't be
run a single step; and $\next\,\pr'$, for $\pr'\in\CP$, if
$\pr'$ is the result of running $\pr$ for one step.
We proceed as follows:
\begin{itemize}
\item ($\progconst(\progcon)$, where $\progcon\in\ProgConst$)\quad
  Return $\myvalue$.

\item ($\progint(n)$, where $n\in\ints$)\quad Return $\myvalue$.

\item ($\progsym(a)$, where $a\in\Sym$)\quad Return $\myvalue$.

\item ($\progstr(x)$, where $x\in\Str$)\quad Return $\myvalue$.

\item ($\progpair(\pr_1,\pr_2)$, where $\pr_1,\pr_2\in\CP$)\quad
  Use case analysis on the form of $\step\,pr_1$:
  \begin{itemize}
  \item ($\error$)\quad Return $\error$.

  \item ($\myvalue$)\quad Use case analysis on the form of $\step\,\pr_2$:
    \begin{itemize}
    \item ($\error$)\quad Return $\error$.

    \item ($\myvalue$)\quad Return $\myvalue$.

    \item ($\next\,\pr'_2$, where $\pr'_2\in\CP$)\quad
      Return $\progpair(pr_1,\pr'_2)$.
    \end{itemize}

  \item ($\next\,\pr'_1$, where $\pr'_1\in\CP$)\quad
    Return $\next(\progpair(\pr'_1,\pr_2))$.
  \end{itemize}

\item ($\progcalc(\progoper,\pr)$, where $\progoper\in\ProgOper$ and
  $\pr\in\CP$)\quad
  Use case analysis on the form of $\step\,\pr$:
  \begin{itemize}
  \item ($\error$)\quad Return $\error$.

  \item ($\myvalue)$\quad Use case analysis on the form of
    $\calculate(\progoper, \pr)$:
    \begin{itemize}
    \item ($\none$)\quad Return $\error$.

    \item ($\some\,\pr'$, where $\pr'\in\Val$)\quad
      Return $\next\,\pr'$.
    \end{itemize}

  \item ($\next\,\pr'$, where $\pr'\in\CP$)\quad Return
    $\next(\progcalc(\progoper,\pr'))$.
  \end{itemize}

\item ($\progcond(\pr_1,\pr_2,\pr_3)$, where
  $\pr_1,\pr_2,\pr_3\in\CP$)\quad Use case analysis on the form
  of $\step\,\pr_1$:
  \begin{itemize}
  \item ($\error$)\quad Return $\error$.

  \item ($\myvalue$)\quad Use case analysis on the form of $\pr_1$ (which
    is a value):
    \begin{itemize}
    \item ($\progconst(\progtrue)$)\quad Return $\next\,\pr_2$.

    \item ($\progconst(\progfalse)$)\quad Return $\next\,\pr_3$.

    \item (anything else)\quad Return $\error$.
    \end{itemize}

  \item ($\next\,\pr'_1$, where $\pr'_1\in\CP$)\quad Return
    $\next(\progcond(\pr'_1, \pr_2, \pr_3))$.
  \end{itemize}

\item ($\proglam(v,\pr)$, where $v\in\ProgVar$,
  $\pr\in\Prog$ and $\free\,\pr\sub\{v\}$)\quad Return $\myvalue$.

\item ($\progletSimp(v, \pr_1, \pr2)$, where $v\in\ProgVar$,
  $\pr_1\in\CP$ and $\free\,\pr_2\sub\{v\}$)\quad Use case analysis
  on the form of $\step\,\pr_1$:
  \begin{itemize}
  \item ($\error$)\quad Return $\error$.

  \item ($\myvalue$)\quad Return $\next(\subst(\pr_1,v,\pr_2))$.

  \item ($\next\,\pr'_1$)\quad Return $\progletSimp(v,\pr'_1,\pr_2)$.
  \end{itemize}

\item ($\progletRec(v_1,v_2,\pr_1,\pr_2)$, where $v_1,v_2\in\ProgVar$,
  $\pr_1,\pr_2\in\Prog$, $\free\,\pr_1\sub\{v_1,v_2\}$ and
  $\free\,\pr_2\sub\{v_1\}$)\quad
  Let $\pr\in\CP$ be
  \begin{displaymath}
    \subst
    (\progletRec(v_1, v_2, \pr_1, \progvar(v_1)),
     v_1,
     \proglam(v_2, \pr_1)) .
  \end{displaymath}
  Return $\next(\subst(\pr,v_1,\pr_2))$.
\end{itemize}

The module \texttt{Prog} defines the following datatype and function,
which correspond to the set $\Step$ and function $\step$:
\begin{verbatim}
datatype step = Value
              | Error
              | Next of cp
val step : cp -> step
\end{verbatim}
For example, here is an example of how a nonterminating recursive
function can be evaluated:
\begin{list}{}
{\setlength{\leftmargin}{\leftmargini}
\setlength{\rightmargin}{0cm}
\setlength{\itemindent}{0cm}
\setlength{\listparindent}{0cm}
\setlength{\itemsep}{0cm}
\setlength{\parsep}{0cm}
\setlength{\labelsep}{0cm}
\setlength{\labelwidth}{0cm}
\catcode`\#=12
\catcode`\$=12
\catcode`\%=12
\catcode`\^=12
\catcode`\_=12
\catcode`\.=12
\catcode`\?=12
\catcode`\!=12
\catcode`\&=12
\ttfamily}
\small
\item[]\textsl{-\ }val\ pr\ =\ Prog.input\ "";
\item[]\textsl{@\ }letRec(x,\ y,
\item[]\textsl{@\ }\ \ \ \ \ \ \ app(var(x),\ calc(plus,\ pair(var(y),\ int(1)))),
\item[]\textsl{@\ }\ \ \ \ \ \ \ app(var(x),\ int(0)))
\item[]\textsl{@\ }.
\item[]\textsl{val\ pr\ =\ -\ :\ prog}
\item[]\textsl{-\ }val\ pr1\ =\ Prog.toClosed\ pr;
\item[]\textsl{val\ pr1\ =\ -\ :\ cp}
\item[]\textsl{-\ }Prog.step\ pr1;
\item[]\textsl{val\ it\ =\ Next\ -\ :\ Prog.step}
\item[]\textsl{-\ }val\ Prog.Next\ pr2\ =\ it;
\item[]\textsl{val\ pr2\ =\ -\ :\ cp}
\item[]\textsl{-\ }Prog.output("",\ Prog.fromClosed\ pr2);
\item[]\textsl{app(lam(y,}
\item[]\textsl{\ \ \ \ \ \ \ \ app(letRec(x,\ y,}
\item[]\textsl{\ \ \ \ \ \ \ \ \ \ \ \ \ \ \ \ \ \ \ app(var(x),\ calc(plus,\ pair(var(y),\ int(1)))),}
\item[]\textsl{\ \ \ \ \ \ \ \ \ \ \ \ \ \ \ \ \ \ \ var(x)),}
\item[]\textsl{\ \ \ \ \ \ \ \ \ \ \ \ calc(plus,\ pair(var(y),\ int(1))))),}
\item[]\textsl{\ \ \ \ int(0))}
\item[]\textsl{val\ it\ =\ ()\ :\ unit}
\item[]\textsl{-\ }Prog.step\ pr2;
\item[]\textsl{val\ it\ =\ Next\ -\ :\ Prog.step}
\item[]\textsl{-\ }val\ Prog.Next\ pr3\ =\ it;
\item[]\textsl{val\ pr3\ =\ -\ :\ cp}
\item[]\textsl{-\ }Prog.output("",\ Prog.fromClosed\ pr3);
\item[]\textsl{app(letRec(x,\ y,\ app(var(x),\ calc(plus,\ pair(var(y),\ int(1)))),}
\item[]\textsl{\ \ \ \ \ \ \ \ \ \ \ var(x)),}
\item[]\textsl{\ \ \ \ calc(plus,\ pair(int(0),\ int(1))))}
\item[]\textsl{val\ it\ =\ ()\ :\ unit}
\item[]\textsl{-\ }Prog.step\ pr3;
\item[]\textsl{val\ it\ =\ Next\ -\ :\ Prog.step}
\item[]\textsl{-\ }val\ Prog.Next\ pr4\ =\ it;
\item[]\textsl{val\ pr4\ =\ -\ :\ cp}
\item[]\textsl{-\ }Prog.output("",\ Prog.fromClosed\ pr4);
\item[]\textsl{app(lam(y,}
\item[]\textsl{\ \ \ \ \ \ \ \ app(letRec(x,\ y,}
\item[]\textsl{\ \ \ \ \ \ \ \ \ \ \ \ \ \ \ \ \ \ \ app(var(x),\ calc(plus,\ pair(var(y),\ int(1)))),}
\item[]\textsl{\ \ \ \ \ \ \ \ \ \ \ \ \ \ \ \ \ \ \ var(x)),}
\item[]\textsl{\ \ \ \ \ \ \ \ \ \ \ \ calc(plus,\ pair(var(y),\ int(1))))),}
\item[]\textsl{\ \ \ \ calc(plus,\ pair(int(0),\ int(1))))}
\item[]\textsl{val\ it\ =\ ()\ :\ unit}
\item[]\textsl{-\ }Prog.step\ pr4;
\item[]\textsl{val\ it\ =\ Next\ -\ :\ Prog.step}
\item[]\textsl{-\ }val\ Prog.Next\ pr5\ =\ it;
\item[]\textsl{val\ pr5\ =\ -\ :\ cp}
\item[]\textsl{-\ }Prog.output("",\ Prog.fromClosed\ pr5);
\item[]\textsl{app(lam(y,}
\item[]\textsl{\ \ \ \ \ \ \ \ app(letRec(x,\ y,}
\item[]\textsl{\ \ \ \ \ \ \ \ \ \ \ \ \ \ \ \ \ \ \ app(var(x),\ calc(plus,\ pair(var(y),\ int(1)))),}
\item[]\textsl{\ \ \ \ \ \ \ \ \ \ \ \ \ \ \ \ \ \ \ var(x)),}
\item[]\textsl{\ \ \ \ \ \ \ \ \ \ \ \ calc(plus,\ pair(var(y),\ int(1))))),}
\item[]\textsl{\ \ \ \ int(1))}
\item[]\textsl{val\ it\ =\ ()\ :\ unit}
\end{list}


\begin{proposition}
For all $\pr\in\CP$, $\step\,\pr = \myvalue$ iff $\pr\in\Val$.
\end{proposition}

Let
\begin{align*}
  \Run &= \setof{\ans\,\pr}{\pr\in\Val} \\
  &\cup \;\setof{\fail\,\pr}{\pr\in\CP} \\
  &\cup \;\setof{\intermed\,\pr}{\pr\in\CP} ,
\end{align*}
and define a function $\run\in\CP\times\nats\fun\Run$ by
structural recursion on its second argument.  Given $\pr\in\CP$ and
$n\in\nats$, $\run(\pr,n)$ is computed using a case analysis of
the form of $n$:
\begin{itemize}
\item ($0$)\quad Return $\intermed\,\pr$.

\item ($n + 1$, where $n\in\nats$)\quad We proceed by a case analysis
  of the form of $\step\,\pr$:
  \begin{itemize}
  \item ($\error$)\quad Return $\fail\,\pr$.

  \item ($\myvalue$)\quad Return $\ans\,\pr$.

  \item ($\next\,\pr'$, where $\pr'\in\CP$)\quad
    Return $\run(\pr', n)$.
  \end{itemize}
\end{itemize}

The module \texttt{Prog} defines the following datatype and functions:
\begin{verbatim}
datatype run = Ans      of cp
             | Fail     of cp
             | Intermed of cp
val run      : cp * int -> run
val evaluate : prog * int -> unit
\end{verbatim}
The function \texttt{run} corresponds to $\run$, except that it issues
an error message when its second argument is negative.  The function
\texttt{evaluate} issues an error message if its first argument,
$\pr$, isn't closed, or its second argument, $n$, is negative.
Otherwise, \texttt{evaluate} explains what results from running
\begin{displaymath}
  \mathtt{run}(\mathtt{Prog.toClosed}~\pr,~n).
\end{displaymath}

Here are some examples of how \texttt{run} can be used:
\begin{list}{}
{\setlength{\leftmargin}{\leftmargini}
\setlength{\rightmargin}{0cm}
\setlength{\itemindent}{0cm}
\setlength{\listparindent}{0cm}
\setlength{\itemsep}{0cm}
\setlength{\parsep}{0cm}
\setlength{\labelsep}{0cm}
\setlength{\labelwidth}{0cm}
\catcode`\#=12
\catcode`\$=12
\catcode`\%=12
\catcode`\^=12
\catcode`\_=12
\catcode`\.=12
\catcode`\?=12
\catcode`\!=12
\catcode`\&=12
\ttfamily}
\small
\item[]\textsl{-\ }val\ pr\ =\ Prog.toClosed(Prog.input\ "");
\item[]\textsl{@\ }app(lam(x,\ calc(plus,\ pair(var(x),\ int(3)))),
\item[]\textsl{@\ }\ \ \ \ int(4))
\item[]\textsl{@\ }.
\item[]\textsl{val\ pr\ =\ -\ :\ cp}
\item[]\textsl{-\ }Prog.run(pr,\ 0);
\item[]\textsl{val\ it\ =\ Intermed\ -\ :\ Prog.run}
\item[]\textsl{-\ }val\ Prog.Intermed\ pr'\ =\ it;
\item[]\textsl{val\ pr'\ =\ -\ :\ cp}
\item[]\textsl{-\ }Prog.output("",\ Prog.fromClosed\ pr');
\item[]\textsl{app(lam(x,\ calc(plus,\ pair(var(x),\ int(3)))),\ int(4))}
\item[]\textsl{val\ it\ =\ ()\ :\ unit}
\item[]\textsl{-\ }Prog.run(pr,\ 1);
\item[]\textsl{val\ it\ =\ Intermed\ -\ :\ Prog.run}
\item[]\textsl{-\ }val\ Prog.Intermed\ pr'\ =\ it;
\item[]\textsl{val\ pr'\ =\ -\ :\ cp}
\item[]\textsl{-\ }Prog.output("",\ Prog.fromClosed\ pr');
\item[]\textsl{calc(plus,\ pair(int(4),\ int(3)))}
\item[]\textsl{val\ it\ =\ ()\ :\ unit}
\item[]\textsl{-\ }Prog.run(pr,\ 2);
\item[]\textsl{val\ it\ =\ Intermed\ -\ :\ Prog.run}
\item[]\textsl{-\ }val\ Prog.Intermed\ pr'\ =\ it;
\item[]\textsl{val\ pr'\ =\ -\ :\ cp}
\item[]\textsl{-\ }Prog.output("",\ Prog.fromClosed\ pr');
\item[]\textsl{int(7)}
\item[]\textsl{val\ it\ =\ ()\ :\ unit}
\item[]\textsl{-\ }Prog.run(pr,\ 3);
\item[]\textsl{val\ it\ =\ Ans\ -\ :\ Prog.run}
\item[]\textsl{-\ }val\ Prog.Ans\ pr'\ =\ it;
\item[]\textsl{val\ pr'\ =\ -\ :\ cp}
\item[]\textsl{-\ }Prog.output("",\ Prog.fromClosed\ pr');
\item[]\textsl{int(7)}
\item[]\textsl{val\ it\ =\ ()\ :\ unit}
\item[]\textsl{-\ }val\ pr\ =\ Prog.toClosed(Prog.fromString\ "calc(plus,\ int(4))");
\item[]\textsl{val\ pr\ =\ -\ :\ cp}
\item[]\textsl{-\ }Prog.run(pr,\ 1);
\item[]\textsl{val\ it\ =\ Fail\ -\ :\ Prog.run}
\item[]\textsl{-\ }val\ Prog.Fail\ pr'\ =\ it;
\item[]\textsl{val\ pr'\ =\ -\ :\ cp}
\item[]\textsl{-\ }Prog.output("",\ Prog.fromClosed\ pr');
\item[]\textsl{calc(plus,\ int(4))}
\item[]\textsl{val\ it\ =\ ()\ :\ unit}
\end{list}

Suppose that the file \texttt{even-prog} contains the following
definition of a function for testing whether a natural number is even:
\verbatiminput{even-prog}
Here are some examples of how we can test this function using
\texttt{evaluate}:
\begin{list}{}
{\setlength{\leftmargin}{\leftmargini}
\setlength{\rightmargin}{0cm}
\setlength{\itemindent}{0cm}
\setlength{\listparindent}{0cm}
\setlength{\itemsep}{0cm}
\setlength{\parsep}{0cm}
\setlength{\labelsep}{0cm}
\setlength{\labelwidth}{0cm}
\catcode`\#=12
\catcode`\$=12
\catcode`\%=12
\catcode`\^=12
\catcode`\_=12
\catcode`\.=12
\catcode`\?=12
\catcode`\!=12
\catcode`\&=12
\ttfamily}
\small
\item[]\textsl{-\ }val\ even\ =\ Prog.input\ "even-prog";
\item[]\textsl{val\ even\ =\ -\ :\ prog}
\item[]\textsl{-\ }fun\ test(n,\ m)\ =
\item[]\textsl{=\ }\ \ \ \ \ \ let\ val\ pr\ =\ Prog.app(even,\ Prog.int\ n)
\item[]\textsl{=\ }\ \ \ \ \ \ in\ Prog.evaluate(pr,\ m)\ end;
\item[]\textsl{val\ test\ =\ fn\ :\ IntInf.int\ \symbol{'052}\ int\ ->\ unit}
\item[]\textsl{-\ }test(3,\ 100);
\item[]\textsl{terminated\ with\ value\ "const(false)"}
\item[]\textsl{val\ it\ =\ ()\ :\ unit}
\item[]\textsl{-\ }test(4,\ 100);
\item[]\textsl{terminated\ with\ value\ "const(true)"}
\item[]\textsl{val\ it\ =\ ()\ :\ unit}
\item[]\textsl{-\ }test(5,\ 100);
\item[]\textsl{terminated\ with\ value\ "const(false)"}
\item[]\textsl{val\ it\ =\ ()\ :\ unit}
\item[]\textsl{-\ }test(6,\ 100);
\item[]\textsl{terminated\ with\ value\ "const(true)"}
\item[]\textsl{val\ it\ =\ ()\ :\ unit}
\end{list}

And here are some other uses of \texttt{evaluate}:
\begin{list}{}
{\setlength{\leftmargin}{\leftmargini}
\setlength{\rightmargin}{0cm}
\setlength{\itemindent}{0cm}
\setlength{\listparindent}{0cm}
\setlength{\itemsep}{0cm}
\setlength{\parsep}{0cm}
\setlength{\labelsep}{0cm}
\setlength{\labelwidth}{0cm}
\catcode`\#=12
\catcode`\$=12
\catcode`\%=12
\catcode`\^=12
\catcode`\_=12
\catcode`\.=12
\catcode`\?=12
\catcode`\!=12
\catcode`\&=12
\ttfamily}
\small
\item[]\textsl{-\ }val\ pr1\ =\ Prog.input\ "";
\item[]\textsl{@\ }calc(consSym,\ int(12))
\item[]\textsl{@\ }.
\item[]\textsl{val\ pr1\ =\ -\ :\ prog}
\item[]\textsl{-\ }Prog.evaluate(pr1,\ 2);
\item[]\textsl{terminated\ with\ value\ "sym(b)"}
\item[]\textsl{val\ it\ =\ ()\ :\ unit}
\item[]\textsl{-\ }val\ pr2\ =\ Prog.input\ "";
\item[]\textsl{@\ }calc(consSym,
\item[]\textsl{@\ }\ \ \ \ \ pair(sym(9),
\item[]\textsl{@\ }\ \ \ \ \ \ \ \ \ \ pair(const(nil),
\item[]\textsl{@\ }\ \ \ \ \ \ \ \ \ \ \ \ \ \ \ pair(sym(<hi>),\ const(nil)))))
\item[]\textsl{@\ }.
\item[]\textsl{val\ pr2\ =\ -\ :\ prog}
\item[]\textsl{-\ }Prog.evaluate(pr2,\ 2);
\item[]\textsl{terminated\ with\ value\ "sym(<9,<hi>>)"}
\item[]\textsl{val\ it\ =\ ()\ :\ unit}
\item[]\textsl{-\ }val\ pr3\ =\ Prog.input\ "";
\item[]\textsl{@\ }calc(deconsSym,\ sym(c))
\item[]\textsl{@\ }.
\item[]\textsl{val\ pr3\ =\ -\ :\ prog}
\item[]\textsl{-\ }Prog.evaluate(pr3,\ 2);
\item[]\textsl{terminated\ with\ value\ "int(13)"}
\item[]\textsl{val\ it\ =\ ()\ :\ unit}
\item[]\textsl{-\ }val\ pr4\ =\ Prog.input\ "";
\item[]\textsl{@\ }calc(deconsSym,\ sym(<on,,to>))
\item[]\textsl{@\ }.
\item[]\textsl{val\ pr4\ =\ -\ :\ prog}
\item[]\textsl{-\ }Prog.evaluate(pr4,\ 2);
\item[]\textsl{terminated\ with\ value}
\item[]\textsl{"pair(sym(o),}
\item[]\textsl{\ \ \ \ \ \ pair(sym(n),}
\item[]\textsl{\ \ \ \ \ \ \ \ \ \ \ pair(const(nil),}
\item[]\textsl{\ \ \ \ \ \ \ \ \ \ \ \ \ \ \ \ pair(const(nil),}
\item[]\textsl{\ \ \ \ \ \ \ \ \ \ \ \ \ \ \ \ \ \ \ \ \ pair(sym(t),\ pair(sym(o),\ const(nil)))))))"}
\item[]\textsl{val\ it\ =\ ()\ :\ unit}
\item[]\textsl{-\ }val\ pr5\ =\ Prog.input\ "";
\item[]\textsl{@\ }calc(symListToStr,
\item[]\textsl{@\ }\ \ \ \ \ pair(sym(9),
\item[]\textsl{@\ }\ \ \ \ \ \ \ \ \ \ pair(sym(<hi>),
\item[]\textsl{@\ }\ \ \ \ \ \ \ \ \ \ \ \ \ \ \ const(nil))))
\item[]\textsl{@\ }.
\item[]\textsl{val\ pr5\ =\ -\ :\ prog}
\item[]\textsl{-\ }Prog.evaluate(pr5,\ 2);
\item[]\textsl{terminated\ with\ value\ "str(9<hi>)"}
\item[]\textsl{val\ it\ =\ ()\ :\ unit}
\item[]\textsl{-\ }val\ pr6\ =\ Prog.input\ "";
\item[]\textsl{@\ }calc(strToSymList,\ str(ab<hi><>))
\item[]\textsl{@\ }.
\item[]\textsl{val\ pr6\ =\ -\ :\ prog}
\item[]\textsl{-\ }Prog.evaluate(pr6,\ 2);
\item[]\textsl{terminated\ with\ value}
\item[]\textsl{"pair(sym(a),}
\item[]\textsl{\ \ \ \ \ \ pair(sym(b),\ pair(sym(<hi>),\ pair(sym(<>),\ const(nil)))))"}
\item[]\textsl{val\ it\ =\ ()\ :\ unit}
\end{list}


\begin{proposition}
\begin{enumerate}
\item For all $\pr\in\CP$, $n\in\nats$ and $\pr'\in\Val$, if
  $\run(\pr,n) = \ans\,\pr'$, then, for all $m\in\nats$, if $m\geq n$,
  then $\run(\pr,m) = \ans\,\pr'$.

\item For all $\pr\in\CP$, $n\in\nats$ and $\pr'\in\CP$, if
  $\run(\pr,n) = \fail\,\pr'$, then, for all $m\in\nats$, if $m\geq n$,
  then $\run(\pr,m) = \fail\,\pr'$.
\end{enumerate}
\end{proposition}

Now we can define the mathematical meaning of closed programs.
Let
\begin{displaymath}
\Eval = \{\nonterm,\error\}\cup\setof{\norm\,\pr}{\pr\in\Val} .  
\end{displaymath}
We define a mathematical function---\emph{not} an
algorithm---$\eval\in\CP\fun\Eval$, as follows.
Suppose $\pr\in\CP$.  There are two main cases to consider:
\begin{itemize}
\item Suppose, for all $n\in\nats$, there is a $\pr'\in\CP$ such
  that $\run(\pr,n) = \intermed\,\pr'$.  Then $\eval\,\pr = \nonterm$.

\item Suppose there is an $n\in\nats$ such that there is no
  $\pr'\in\CP$ such that $\run(\pr,n) = \intermed\,\pr'$.  Let $n$ be
  the smallest natural number such that there is no $\pr'\in\CP$ such
  that $\run(\pr,n) = \intermed\,\pr'$.  There are two subcases to
  consider:
  \begin{itemize}
  \item Suppose $\run(\pr,n) = \ans\,\pr'$ for some $\pr'\in\Val$.
    Then $\eval\,\pr = \norm\,\pr'$.

  \item Suppose $\run(\pr,n) = \fail\,\pr'$ for some $\pr'\in\CP$.
    Then $\eval\,\pr = \error$.
  \end{itemize}
\end{itemize}

For example:
\begin{itemize}
\item Let $\pr$ be
  \begin{displaymath}
    \progletRec(\mathsf{x}, \mathsf{y},
                \progapp(\progvar(\mathsf{x}), \progvar(\mathsf{y})),
                \progapp(\progvar(\mathsf{x}), \progint(0))) .
  \end{displaymath}
  Then $\eval\,\pr = \nonterm$.

\item $\eval(\progapp(\progint(0), \progint(1))) = \error$.

\item $\eval(\progcalc(\progplus, \progpair(\progint(1),\progint(2)))) =
  \norm(\progint(3))$.
\end{itemize}

\subsection{Programs as Data}

In Section~\ref{DiagonalizationAndUndecidableProblems}, we will be
concerned with programs that process programs.  Because programs
are described by strings, these program processing programs could
work on strings.  But that would be complicated and cumbersome.  It's
far better to represent programs using pairs.

The set $\Rep$ of \emph{program representations} is the least subset of
$\Val$ such that:
\begin{description}
\item[\quad(variable)] for all $v\in\ProgVar$,
  $\progpair(\progstr(\mathsf{var}), \progstr(v)) \in \Rep$;

\item[\quad(constant)] for all $\progcon\in\ProgConst$,
  $\progpair(\progstr(\mathsf{const}), \progconst(\progcon))\in\Rep$;

\item[\quad(integer)] for all $n\in\ints$, 
  $\progpair(\progstr(\mathsf{int}), \progint(n))\in\Rep$;

\item[\quad(symbol)] for all $a\in\Sym$,
  $\progpair(\progstr(\mathsf{sym}), \progsym(a))\in\Rep$;

\item[\quad(string)] for all $x\in\Str$,
  $\progpair(\progstr(\mathsf{str}),\progstr(x))\in\Rep$;

\item[\quad(pair)] for all $\pr_1,\pr_2\in\Rep$,
  \begin{displaymath}
    \progpair(\progstr(\mathsf{pair}),
    \progpair(\pr_1, \pr_2)) \in \Rep ;
  \end{displaymath}

\item[\quad(calculation)] for all $\progoper\in\ProgOper$ and
  $\pr\in\Rep$,
  \begin{displaymath}
    \progpair(\progstr(\mathsf{calc}),
    \progpair(\progstr(\progoper), \pr)) \in \Rep ;
  \end{displaymath}

\item[\quad(conditional)] for all $\pr_1,\pr_2,\pr_3\in\Rep$,
  \begin{displaymath}
    \progpair(\progstr(\mathsf{cond}),
    \progpair(\pr_1, \progpair(\pr_2, \pr_3))) \in \Rep ;
  \end{displaymath}

\item[\quad(function application)] for all $\pr_1,\pr_2\in\Rep$,
  \begin{displaymath}
    \progpair(\progstr(\mathsf{app}),
    \progpair(\pr_1, \pr_2)) \in \Rep ;
  \end{displaymath}

\item[\quad(anonymous function)] for all $v\in\ProgVar$ and
  $\pr\in\Rep$,
  \begin{displaymath}
    \progpair(\progstr(\mathsf{lam}),
    \progpair(\progstr(v), \pr)) \in \Rep ;
  \end{displaymath}

\item[\quad(simple let)] for all $v\in\ProgVar$ and
  $\pr_1,\pr_2\in\Rep$,
  \begin{displaymath}
    \progpair(\progstr(\mathsf{letSimp}),
    \progpair(\progstr(v),
    \progpair(\pr_1, \pr_2))) \in \Rep ;
  \end{displaymath}

\item[\quad(recursive let)] for all $v_1,v_2\in\ProgVar$ and
  $\pr_1,\pr_2\in\Rep$,
  \begin{displaymath}
    \progpair(\progstr(\mathsf{letRec}),
    \progpair(\progstr(v_1),
    \progpair(\progstr(v_2),
    \progpair(\pr_1, \pr_2)))) \in \Rep .
  \end{displaymath}
\end{description}

We define a function $\overline{\;\cdot\;}\in\Prog\fun\Rep$
by structural recursion:
\begin{itemize}
\item for all $v\in\ProgVar$,
  $\overline{\progvar(v)} =
  \progpair(\progstr(\mathsf{var}), \progstr(v))$;

\item for all $\progcon\in\ProgConst$,
  $\overline{\progconst(\progcon)} =
  \progpair(\progstr(\mathsf{const}), \progconst(\progcon))$;

\item for all $n\in\ints$, $\overline{\progint(n)} =
  \progpair(\progstr(\mathsf{int}), \progint(n))$;

\item for all $a\in\Sym$, $\overline{\progsym(a)} =
  \progpair(\progstr(\mathsf{sym}), \progsym(a))$;

\item for all $x\in\Str$, $\overline{\progstr(x)} =
  \progpair(\progstr(\mathsf{str}), \progstr(x))$;

\item for all $\pr_1,\pr_2\in\Prog$,
  \begin{displaymath}
    \overline{\progpair(\pr_1,\pr_2)} =
    \progpair(\progstr(\mathsf{pair}),
    \progpair(\overline{\pr_1}, \overline{\pr_2})) ;
  \end{displaymath}

\item for all $\progoper\in\ProgOper$ and $\pr\in\Prog$,
  \begin{displaymath}
    \overline{\progcalc(\progoper,\pr)} =
    \progpair(\progstr(\mathsf{calc}),
    \progpair(\progstr(\progoper), \overline{\pr})) ;
  \end{displaymath}

\item for all $\pr_1,\pr_2,\pr_3\in\Prog$,
  \begin{displaymath}
    \overline{\progcond(\pr_1,\pr_2,\pr_3)} =
    \progpair(\progstr(\mathsf{cond}),
    \progpair(\overline{\pr_1},
    \progpair(\overline{\pr_2}, \overline{\pr_3}))) ;
  \end{displaymath}

\item for all $\pr_1,\pr_2\in\Prog$,
  \begin{displaymath}
    \overline{\progapp(\pr_1,\pr_2)} =
    \progpair(\progstr(\mathsf{app}),
    \progpair(\overline{\pr_1}, \overline{\pr_2})) ;
  \end{displaymath}

\item for all $v\in\ProgVar$ and $\pr\in\Prog$,
  \begin{displaymath}
  \overline{\proglam(v,\pr)} =
  \progpair(\progstr(\mathsf{lam}),
  \progpair(\progstr(v), \overline{\pr})) ;
  \end{displaymath}

\item for all $v\in\ProgVar$ and $\pr_1,\pr_2\in\Prog$,
  \begin{displaymath}
    \overline{\progletSimp(v,\pr_1,\pr_2)} =
    \progpair(\progstr(\mathsf{letSimp}),
    \progpair(\progstr(v),
    \progpair(\overline{\pr_1}, \overline{\pr_2}))) ;
  \end{displaymath}

\item for all $v_1,v_2\in\ProgVar$ and $\pr_1,\pr_2\in\Prog$,
  \begin{multline*}
  \overline{\progletRec(v_1,v_2,\pr_1,\pr_2)} \\
  = \progpair(\progstr(\mathsf{letRec}),
  \progpair(\progstr(v_1),
  \progpair(\progstr(v_2),
  \progpair(\overline{\pr_1}, \overline{\pr_2})))) .
  \end{multline*}
\end{itemize}
If $\pr\in\Prog$, we say that $\overline{\pr}$ \emph{represents} $\pr$,
and that $\pr$ \emph{is represented by} $\overline{\pr}$.

Each program is represented by a unique program representation, and
every program representation represents a unique program:

\begin{proposition}
$\overline{\;\cdot\;}$ is a bijection from $\Prog$ to $\Rep$.
\end{proposition}

The module \texttt{Prog} also defines the functions:
\begin{verbatim}
val toRep   : prog -> prog
val fromRep : prog -> prog
val isRep   : prog -> bool
\end{verbatim}
The function \texttt{toRep} converts a program to the program representation
that represents it.  The function \texttt{fromRep} issues an error message
if its argument isn't a program representation; otherwise it returns the
program represented by its argument.  And the function \texttt{isRep} tests
whether a program is a program representation.

For example:
\begin{list}{}
{\setlength{\leftmargin}{\leftmargini}
\setlength{\rightmargin}{0cm}
\setlength{\itemindent}{0cm}
\setlength{\listparindent}{0cm}
\setlength{\itemsep}{0cm}
\setlength{\parsep}{0cm}
\setlength{\labelsep}{0cm}
\setlength{\labelwidth}{0cm}
\catcode`\#=12
\catcode`\$=12
\catcode`\%=12
\catcode`\^=12
\catcode`\_=12
\catcode`\.=12
\catcode`\?=12
\catcode`\!=12
\catcode`\&=12
\ttfamily}
\small
\item[]\textsl{-\ }val\ pr\ =\ Prog.input\ "";
\item[]\textsl{@\ }app(lam(x,\ app(var(x),\ int(1))),
\item[]\textsl{@\ }\ \ \ \ lam(y,\ var(y)))
\item[]\textsl{@\ }.
\item[]\textsl{val\ pr\ =\ -\ :\ prog}
\item[]\textsl{-\ }val\ pr'\ =\ Prog.toRep\ pr;
\item[]\textsl{val\ pr'\ =\ -\ :\ prog}
\item[]\textsl{-\ }Prog.output("",\ pr');
\item[]\textsl{pair(str(app),}
\item[]\textsl{\ \ \ \ \ pair(pair(str(lam),}
\item[]\textsl{\ \ \ \ \ \ \ \ \ \ \ \ \ \ \ pair(str(x),}
\item[]\textsl{\ \ \ \ \ \ \ \ \ \ \ \ \ \ \ \ \ \ \ \ pair(str(app),}
\item[]\textsl{\ \ \ \ \ \ \ \ \ \ \ \ \ \ \ \ \ \ \ \ \ \ \ \ \ pair(pair(str(var),\ str(x)),}
\item[]\textsl{\ \ \ \ \ \ \ \ \ \ \ \ \ \ \ \ \ \ \ \ \ \ \ \ \ \ \ \ \ \ pair(str(int),\ int(1)))))),}
\item[]\textsl{\ \ \ \ \ \ \ \ \ \ pair(str(lam),\ pair(str(y),\ pair(str(var),\ str(y))))))}
\item[]\textsl{val\ it\ =\ ()\ :\ unit}
\item[]\textsl{-\ }val\ pr''\ =\ Prog.fromRep\ pr';
\item[]\textsl{val\ pr''\ =\ -\ :\ prog}
\item[]\textsl{-\ }Prog.equal(pr'',\ pr);
\item[]\textsl{val\ it\ =\ true\ :\ bool}
\item[]\textsl{-\ }Prog.isRep\ pr;
\item[]\textsl{val\ it\ =\ false\ :\ bool}
\item[]\textsl{-\ }Prog.isRep\ pr';
\item[]\textsl{val\ it\ =\ true\ :\ bool}
\item[]\textsl{-\ }Prog.isRep(Prog.fromString\ "pair(str(x),\ str(y))");
\item[]\textsl{val\ it\ =\ false\ :\ bool}
\end{list}


It is easy to write a program that tests whether a program
representation represents a closed program, as well to write a program
that tests whether a program representation represents a value.

It is possible to write a function in our programming language that
acts as an interpreter:
\begin{itemize}
\item It takes in a value $\overline{\pr}$, representing a closed
  program $\pr$.

\item It begins evaluating $\pr$, using the representation
  $\overline{\pr}$.

\item If this evaluation results in an error, then the interpreter
  returns $\progconst(\prognil)$.

\item Otherwise, if it results in a value $\overline{\pr'}$
  representing a value $\pr'$, then it returns $\overline{\pr'}$.

\item Otherwise, it runs forever.
\end{itemize}

E.g.,
  $\overline{\progcond(\progconst(\progtrue),\progconst(\progfalse),
    \progconst(\prognil))}$ evaluates to $\overline{\progconst(\progfalse)}$.

We can also write a function in our programming language that acts as
an \emph{incremental} interpreter:
\begin{itemize}
\item At each stage of its evaluation of a closed program, it carries
  out some fixed number of steps of the evaluation.

\item If during the execution of those steps, an error is detected,
  then it returns $\progconst(\prognil)$.

\item Otherwise, if a value $\overline{\pr'}$ representing a value
  $\pr'$ has been produced, then it returns this value.

\item But otherwise, it returns an anonymous function that when called
  will continue this process.
\end{itemize}

\subsection{Recursive and Recursively Enumerable Languages}

A \emph{string predicate program} $\pr$ is a closed program such that, for
all strings $w$, $\eval(\progapp(\pr,\progstr(w)))\in
\{\norm(\progconst(\progtrue)),\norm(\progconst(\progfalse))\}$.

A string $w$ is \emph{accepted by} a closed program $\pr$ iff
$\eval(\progapp(\pr,\progstr(w))) = \norm(\progconst(\progtrue))$.
We write $L(\pr)$ for the set of all strings accepted by a closed
program $\pr$.  When this set is a language, then we refer to $L(\pr)$
as the \emph{language accepted by} $\pr$.  (E.g., if
$\pr=\proglam(\mathsf{x},\progconst(\progtrue))$, then $L(\pr) =
\Str$, and so is not a language.)

The \texttt{Prog} module also includes:
\begin{verbatim}
val accepted : prog -> str * int -> unit
\end{verbatim}
The function \texttt{accepted} takes in a program $\pr$, and issues
and error message if $\pr$ is not closed.  Otherwise, it returns
a function $f$ that behaves as follows, when called with a pair
$(x, n)$.  If $n$ is negative, it issues an error message.  Otherwise,
it proceeds by a case analysis of the result of running
\begin{displaymath}
  \pr' = \mathtt{Prog.run}(\mathtt{Prog.app}(\pr,~\mathtt{Prog.str}~x), n) :
\end{displaymath}
\begin{itemize}
\item ($\ans(\progconst(\progtrue))$)\quad It explains that
  $x$ was accepted by $\pr$.

\item ($\ans(\progconst(\progfalse))$)\quad It explains that $x$ was
  rejected by $\pr$, because the application of $\pr$ to $\progstr(x)$
  resulted in $\progconst(\progfalse)$.

\item ($\ans\,\pr''$, where $\pr''\not\in\{\progconst(\progtrue),
  \progconst(\progfalse)\}$)\quad It explains that $x$ was rejected by
  $\pr$, since the application of $\pr$ to $\progstr(x)$ resulted in
  some value other than $\progconst(\progtrue)$ or
  $\progconst(\progfalse)$.

\item ($\fail\,\pr''$, where $\pr''\in\CP$)\quad It explains that $x$
  was rejected by $\pr$, since the application of $\pr$ to
  $\progstr(x)$ resulted in failure.

\item ($\intermed\,\pr''$, where $\pr''\in\CP$)\quad It explains that
  (based on running $\pr'$ for $n$ steps) it is unknown whether $x$ is
  accepted by $\pr$.
\end{itemize}

Suppose the file \texttt{equal-prog} contains the text:
\verbatiminput{equal-prog}
Suppose the file \texttt{succ-prog} contains the text:
\verbatiminput{equal-prog}
Suppose the file \texttt{count-prog} contains the text:
\verbatiminput{count-prog}
And suppose the file \texttt{zeros-ones-twos-prog} contains the
text:
\verbatiminput{zeros-ones-twos-prog}
We can construct---and experiment with---a program for testing whether
a string is an element of
$\setof{\zerosf^n\onesf^n\twosf^n}{n\in\nats}$, as follows:
\begin{list}{}
{\setlength{\leftmargin}{\leftmargini}
\setlength{\rightmargin}{0cm}
\setlength{\itemindent}{0cm}
\setlength{\listparindent}{0cm}
\setlength{\itemsep}{0cm}
\setlength{\parsep}{0cm}
\setlength{\labelsep}{0cm}
\setlength{\labelwidth}{0cm}
\catcode`\#=12
\catcode`\$=12
\catcode`\%=12
\catcode`\^=12
\catcode`\_=12
\catcode`\.=12
\catcode`\?=12
\catcode`\!=12
\catcode`\&=12
\ttfamily}
\small
\item[]\textsl{-\ }val\ equal\ =\ Prog.input\ "equal-prog";
\item[]\textsl{val\ equal\ =\ -\ :\ prog}
\item[]\textsl{-\ }val\ succ\ =\ Prog.input\ "succ-prog";
\item[]\textsl{val\ succ\ =\ -\ :\ prog}
\item[]\textsl{-\ }val\ count\ =
\item[]\textsl{=\ }\ \ \ \ \ \ Prog.app
\item[]\textsl{=\ }\ \ \ \ \ \ (Prog.app(Prog.input\ "count-prog",\ equal),
\item[]\textsl{=\ }\ \ \ \ \ \ \ succ);
\item[]\textsl{val\ count\ =\ -\ :\ prog}
\item[]\textsl{-\ }val\ zerosOnesTwos\ =
\item[]\textsl{=\ }\ \ \ \ \ \ Prog.app
\item[]\textsl{=\ }\ \ \ \ \ \ (Prog.app(Prog.input\ "zeros-ones-twos-prog",\ equal),
\item[]\textsl{=\ }\ \ \ \ \ \ \ count);
\item[]\textsl{val\ zerosOnesTwos\ =\ -\ :\ prog}
\item[]\textsl{-\ }val\ accepted\ =\ Prog.accepted\ zerosOnesTwos;
\item[]\textsl{val\ accepted\ =\ fn\ :\ str\ \symbol{'052}\ int\ ->\ unit}
\item[]\textsl{-\ }accepted(Str.fromString\ "%",\ 10);
\item[]\textsl{unknown\ if\ accepted\ or\ rejected}
\item[]\textsl{val\ it\ =\ ()\ :\ unit}
\item[]\textsl{-\ }accepted(Str.fromString\ "%",\ 100);
\item[]\textsl{accepted}
\item[]\textsl{val\ it\ =\ ()\ :\ unit}
\item[]\textsl{-\ }accepted(Str.fromString\ "012",\ 100);
\item[]\textsl{accepted}
\item[]\textsl{val\ it\ =\ ()\ :\ unit}
\item[]\textsl{-\ }accepted(Str.fromString\ "000111222",\ 1000);
\item[]\textsl{accepted}
\item[]\textsl{val\ it\ =\ ()\ :\ unit}
\item[]\textsl{-\ }accepted(Str.fromString\ "00011122",\ 1000);
\item[]\textsl{rejected\ with\ false}
\item[]\textsl{val\ it\ =\ ()\ :\ unit}
\item[]\textsl{-\ }accepted(Str.fromString\ "00111222",\ 1000);
\item[]\textsl{rejected\ with\ false}
\item[]\textsl{val\ it\ =\ ()\ :\ unit}
\item[]\textsl{-\ }accepted(Str.fromString\ "00011222",\ 1000);
\item[]\textsl{rejected\ with\ false}
\item[]\textsl{val\ it\ =\ ()\ :\ unit}
\item[]\textsl{-\ }accepted(Str.fromString\ "021",\ 100);
\item[]\textsl{rejected\ with\ false}
\item[]\textsl{val\ it\ =\ ()\ :\ unit}
\item[]\textsl{-\ }accepted(Str.fromString\ "112200",\ 1000);
\item[]\textsl{rejected\ with\ false}
\item[]\textsl{val\ it\ =\ ()\ :\ unit}
\end{list}


We say that a language $L$ is:
\begin{itemize}
\item \emph{recursive} iff $L=L(\pr)$, for some string predicate program $\pr$;
  and

\item \emph{recursively enumerable} (\emph{r.e.}) iff $L=L(\pr)$, for
  some closed program $\pr$.
\end{itemize}

We define
\begin{align*}
\RecLan &= \setof{L\in\Lan}{L\eqtxtl{is recursive}} , \eqtxtl{and} \\
\RELan &= \setof{L\in\Lan}{L\eqtxtl{is recursively enumerable}} .
\end{align*}
Hence $\RecLan\sub\RELan$.  Because $\CP$ is countably infinite, we
have that $\RecLan$ and $\RELan$ are countably infinite, so that
$\RELan\subsetneq\Lan$.  Later we will see that
$\RecLan\subsetneq\RELan$.

\begin{proposition}
\label{RecProp}
For all $L\in\Lan$, $L$ is recursive iff there is a closed program $\pr$ such
that, for all $w\in\Str$:
\begin{itemize}
\item if $w\in L$, then $\eval(\progapp(\pr,\progstr(w)))=
\norm(\progconst(\progtrue))$; and

\item if $w\not\in L$, then
$\eval(\progapp(\pr,\progstr(w)))=\norm(\progconst(\progfalse))$.
\end{itemize}
\end{proposition}

\begin{proof}
\begin{description}
\item[\quad(``only if'']
  Since $L$ is recursive, $L=L(\pr)$ for some
  string predicate program $\pr$.  Suppose $w\in\Str$.  There are two
  cases to show.
  \begin{itemize}
  \item Suppose $w\in L$.  Since $L=L(\pr)$, we have that
    $\eval(\progapp(\pr,\progstr(w)))=\norm(\progconst(\progtrue))$.

  \item Suppose $w\not\in L$.  Since $L=L(\pr)$, we have that
    $\eval(\progapp(\pr,\progstr(w)))\neq\norm(\progconst(\progtrue))$.
    But $\pr$ is a string predicate program, and thus

    $\eval(\progapp(\pr,\progstr(w)))=\norm(\progconst(\progfalse))$.
  \end{itemize}

\item[\quad(``if'')] To see that $\pr$ is a string predicate program,
  suppose $w\in\Str$.  Since $w\in L$ or $w\not\in L$, we have that
  $\eval(\progapp(\pr,\progstr(w)))\in
  \{\progconst(\progtrue),\progconst(\progfalse)\}$.  We will show that
  $L=L(\pr)$.
  \begin{itemize}
  \item Suppose $w\in L$.  Then $\eval(\progapp(\pr,\progstr(w)))=
    \norm(\progconst(\progtrue))$, so that $w\in L(\pr)$.

  \item Suppose $w\in L(\pr)$, so that
    $\eval(\progapp(\pr,\progstr(w)))= \norm\abr(\progconst(\progtrue))$.  If
    $w\not\in L$, then $\eval(\progapp(\pr,\progstr(w)))=
    \norm(\progconst(\progfalse))$---contradiction.  Thus $w\in L$.
\end{itemize}
\end{description}
\end{proof}

\begin{proposition}
\label{REProp}
For all $L\in\Lan$, $L$ is recursively enumerable iff there is a
closed program $\pr$ such that, for all $w\in\Str$,
\begin{gather*}
w\in L\;\eqtxt{iff}\;\eval(\progapp(\pr,\progstr(w)))=
\norm(\progconst(\progtrue)) .
\end{gather*}
\end{proposition}

\begin{proof}
\begin{description}
\item[\quad(``only if'')] Since $L$ is recursively enumerable,
  $L=L(\pr)$ for some closed program $\pr$.  Suppose $w\in\Str$.
  \begin{itemize}
  \item Suppose $w\in L$.  Since $L=L(\pr)$, we have that
    $\eval(\progapp(\pr,\progstr(w)))=\norm(\progconst(\progtrue))$.

  \item Suppose
    $\eval(\progapp(\pr,\progstr(w)))=\norm(\progconst(\progtrue))$.
    Thus $w\in L(\pr)=L$.
  \end{itemize}

\item[\quad(``if'')] It suffices to show that $L=L(\pr)$.
  \begin{itemize}
  \item Suppose $w\in L$.  Then
    $\eval(\progapp(\pr,\progstr(w)))=\norm(\progconst(\progtrue))$,
    so that $w\in L(\pr)$.

  \item Suppose $w\in L(\pr)$.  Then
    $\eval(\progapp(\pr,\progstr(w)))=\norm\abr(\progconst(\progtrue))$,
    so that $w\in L$.
  \end{itemize}
\end{description}
\end{proof}

\begin{theorem}
The context-free languages are a proper subset of the
recursive languages: $\CFLan\subsetneq\RecLan$.
\end{theorem}

\begin{proof}
To see that every context-free language is recursive, let $L$ be a
context-free language.  Thus there is a grammar $G$ such that
$L=L(G)$.  With some work, we can write and prove the correctness of a
string predicate program $\pr$ that implements our algorithm (see
Section~\ref{AParsingAlgorithm}) for checking whether a string is
generated by a grammar.  Thus $L$ is recursive.

To see that not every recursive language is context-free, let
$L=\setof{\zerosf^n\onesf^n\twosf^n}{n\in\nats}$.  In
Section~\ref{ThePumpingLemmaForContextFreeLanguages}, we
learned that $L$ is not context-free.  And in the preceding
subsection, we wrote a string predicate program $\pr$ that tests
whether a string is in $L$.  Thus $L$ is recursive.
\end{proof}

\subsection{Notes}

Neil Jones \cite{Jones97} pioneered the use of a programming language
with structured data as an alternative to Turing machines for studying
the limits of what is computable.  In contrast to Jones's approach,
however, our programming language is functional, not imperative
(assignment-oriented), and it has explicit support for the symbols and
strings of formal language theory.

%%% Local Variables: 
%%% mode: latex
%%% TeX-master: "book"
%%% End: 
