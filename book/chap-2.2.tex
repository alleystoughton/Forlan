\section{Using Induction to Prove Language Equalities}
\label{UsingInductionToProveLanguageEqualities}

In this section, we introduce three string induction principles, ways
of showing that every $w\in A^*$ has property $P(w)$, where $A$ is
some set of symbols.
Typically, $A$ will be an alphabet, i.e., a finite set of symbols.
But when we want to prove that all strings have some
property, we can let $A=\Sym$, so that $A^*=\Str$.
Each of these principles corresponds to an
instance of well-founded induction.  We also look at how different
kinds of induction can be used to show that two languages are equal.

\subsection{String Induction Principles}

Suppose $A$ is a set of symbols.  We define well-founded relations
$\myright_A$, $\myleft_A$ and $\strong_A$ on $A^*$ by:
\begin{itemize}
\item $x\myright_A y$ iff $y=ax$ for some $a\in A$
(it's called \emph{right} because the string $x$ is on the right
side of $ax$);
\item $x\myleft_A y$ iff $y=xa$ for some $a\in A$
(it's called \emph{left} because the string $x$ is on the left
side of $xa$);
\item $x\strong_A y$ iff $x$ is a proper substring of $y$.
\end{itemize}
Thus, for all $a\in A$ and $x\in A^*$, the only predecessor of $ax$ in
$\myright_A$ is $x$, and the only predecessor of $xa$ in
$\myleft_A$ is $x$.  And, for all $y\in A^*$, the predecessors
of $y$ in $\strong_A$ are the proper substrings of $y$.

The well-foundedness of $\myright_A$, $\myleft_A$ and $\strong_A$
follows by Proposition~\ref{WellFoundedSubset}, since each of these
relations is a subset of $\length_{\List\,A}$, which is a well-founded
relation on $\List\,A$.

We now introduce string induction principles corresponding to
well-founded induction on each of these relations.

\index{string induction!right|see {right string induction}}%
\index{right string induction}%
\begin{theorem}[Principle of Right String Induction]
Suppose $A\sub\Sym$ and $P(w)$ is a property of a string $w$.
If
\begin{description}
\item[\quad(basis step)]
\begin{gather*}
P(\%) \eqtxtl{and}
\end{gather*}
\item[\quad(inductive step)]
\begin{gather*}
\eqtxtr{for all}a\in A\eqtxt{and}w\in A^*,
\eqtxt{if}\eqtxt{\rm(\dag)} P(w),\eqtxt{then}P(aw),
\end{gather*}
\end{description}
then,
\begin{gather*}
\eqtxtr{for all}w\in A^*,\,P(w) .
\end{gather*}
\end{theorem}

We refer to the formula (\dag) as the \emph{inductive hypothesis}.
According to the induction principle, to show that every $w\in A^*$
has property $P$, we show that the empty string has property $P$, and
then assume that $a\in A$, $w\in A^*$ and that (the inductive
hypothesis) $w$ has property $P$, and show that $aw$ has property
$P$.

\begin{proof}
Equivalent to well-founded induction on $\myright_A$.
\end{proof}

By switching $aw$ to $wa$ in the inductive step, we get the principle
of left string induction.

\index{string induction!left|see {left string induction}}%
\index{left string induction}%
\begin{theorem}[Principle of Left String Induction]
Suppose $A\sub\Sym$ and $P(w)$ is a property of a string $w$.
If
\begin{description}
\item[\quad(basis step)]
\begin{gather*}
P(\%) \eqtxtl{and}
\end{gather*}
\item[\quad(inductive step)]
\begin{gather*}
\eqtxtr{for all}a\in A\eqtxt{and}w\in A^*,
\eqtxt{if} \eqtxt{\rm(\dag)} P(w),\eqtxt{then}P(wa),
\end{gather*}
\end{description}
then,
\begin{gather*}
\eqtxtr{for all}w\in A^*,\,P(w) .
\end{gather*}
\end{theorem}

We refer to the formula (\dag) as the \emph{inductive hypothesis}.

\begin{proof}
Equivalent to well-founded induction on $\myleft_A$.
\end{proof}

\index{strong string induction}%
\index{string induction!strong|see{strong string induction}}%
\begin{theorem}[Principle of Strong String Induction]
Suppose $A\sub\Sym$ and $P(w)$ is a property of a string $w$.
If
\begin{ctabbing}
for all $w\in A^*$, \\
if {\rm(\dag)} for all $x\in A^*$, if $x$ is a proper substring
of $w$, then $P(x)$, \\
then $P(w)$,
\end{ctabbing}
then,
\begin{gather*}
\eqtxtr{for all}w\in A^*,\,P(w) .
\end{gather*}
\end{theorem}

We refer to (\dag) as the inductive hypothesis.  It says that all the
proper substrings of $w$ have property $P$.  According to the
induction principle, to show that every $w\in A^*$ has property $P$,
we let $w\in A^*$, and assume (the inductive hypothesis) that every
proper substring of $w$ has property $P$.  Then we must show that $w$
has property $P$.

\begin{proof}
Equivalent to well-founded induction on $\strong_A$.
\end{proof}

The next subsection, on proving language equalities, contains two
examples of proofs by strong string induction.  Before moving on to
that subsection, we give an example proof by right string induction.

We define the
\emph{reversal}
\index{reversal!string}%
\index{string!reversal}%
$x^R\in\Str$
\index{ reversal@$\cdot^R$}%
\index{string! reversal@$\cdot^R$}%
\emph{of} a string $x$ by right recursion
\index{recursion!string}%
on strings:
\begin{align*}
\%^R &= \% ; \\
(ax)^R &= x^Ra, \eqtxt{for all} a\in\Sym \eqtxt{and} x\in\Str .
\end{align*}
E.g., we have that $\mathsf{(021)}^R=\mathsf{120}$.  And, an easy
calculation shows that, for all $a\in\Sym$, $a^R=a$.  We let the
reversal operation have higher precedence than string concatenation,
so that, e.g., $xx^R = x(x^R)$.

\begin{proposition}
\label{StrRevProp1}
For all $x,y\in\Str$, $(xy)^R = y^R x^R$.
\end{proposition}

As usual, we must start by figuring out which of $x$ and $y$
to do induction on, as well as what sort of induction to use.
Because we defined string reversal using right recursion,
it turns out that we should do right string induction
\index{right string induction}%
on $x$.

\begin{proof}
Suppose $y\in\Str$.  Since $\Sym^*=\Str$, it will suffice to
show that, for all $x\in\Sym^*$, $(xy)^R = y^R x^R$.  We proceed by
right string induction.
\begin{description}
\item[\quad(Basis Step)] We have that $(\%y)^R = y^R = y^R\% = y^R\%^R$.

\item[\quad(Inductive Step)] Suppose $a\in\Sym$ and $x\in\Sym^*$.  Assume
the inductive hypothesis: $(xy)^R = y^R x^R$.  Then,
\begin{alignat*}{2}
((ax)y)^R &= (a(xy))^R \\
&= (xy)^Ra && \by{definition of $(a(xy))^R$} \\
&= (y^Rx^R)a && \by{inductive hypothesis} \\
&= y^R(x^Ra) \\
&= y^R(ax)^R && \by{definition of $(ax)^R$}.
\end{alignat*}
\end{description}
\end{proof}

\begin{exercise}
\label{StrRevEx}
Use right string induction and Proposition~\ref{StrRevProp1} to prove
that, for all $x\in\Str$, $(x^R)^R=x$.
\end{exercise}

\begin{exercise}
In Section~\ref{SymbolsStringsAlphabetsAndFormalLanguages}, we used
right recursion to define the function
$\alphabet\in\Str\fun\Alp$.  Use right string induction to show
that,
\index{alphabet@$\alphabet$}%
\index{string!alphabet}%
\index{string!alphabet@$\alphabet$}%
for all $x,y\in\Str$, $\alphabet(xy)=\alphabet\,x\cup\alphabet\,y$.
\end{exercise}

\subsection{Proving Language Equalities}

In this subsection, we show two examples of how strong string induction
and induction over inductively defined languages can be used to
show that two languages are equal.

For the first example, let $X$ be the least subset of
$\{\mathsf{0,1}\}^*$ such that:
\begin{enumerate}[\quad(1)]
\item $\%\in X$; and

\item for all $a\in\{\mathsf{0,1}\}$ and $x\in X$, $axa\in X$.
\end{enumerate}
This is another example of an inductive definition:
\index{inductive definition}%
$X$ consists of just those strings of $\zerosf$'s and $\onesf$'s that
can be constructed using (1) and (2).  For example, by (1) and (2), we
have that $\zerosf\zerosf=\zerosf\%\zerosf\in X$.  Thus, by (2), we have
that $\mathsf{1001=1(00)1}\in X$.  In general, we have that $X$
contains the elements:
\begin{gather*}
\mathsf{\%, 00, 11, 0000, 0110, 1001, 1111,\,\ldots}
\end{gather*}

We will show that $X=Y$, where
\index{palindrome}%
\index{string!palindrome}%
$Y=\setof{w\in\{\mathsf{0,1}\}^*}{w\eqtxt{is a palindrome and}|w|
\eqtxtl{is even}}$.

\begin{lemma}
\label{SSIProp1Lem1}
$Y\sub X$.
\end{lemma}

\begin{proof}
Since $Y\sub\{\zerosf,\onesf\}^*$, it will suffice to show that,
for all $w\in\{\mathsf{0,1}\}^*$,
\begin{gather*}
\eqtxt{if}w\in Y, \eqtxt{then}w\in X.
\end{gather*}
We proceed by strong string induction.

Suppose $w\in\{\mathsf{0,1}\}^*$,
and assume the inductive hypothesis: for all $x\in\{\mathsf{0,1}\}^*$,
if $x$ is a proper substring of $w$, then
\begin{gather*}
\eqtxt{if}x\in Y, \eqtxt{then}x\in X .
\end{gather*}
We must show that
\begin{gather*}
\eqtxt{if}w\in Y, \eqtxt{then}w\in X .
\end{gather*}

Suppose $w\in Y$, so that $w\in\{\mathsf{0,1}\}^*$, $w$ is a
palindrome and $|w|$ is even.  It remains to show that $w\in X$.  If
$w=\%$, then $w=\%\in X$, by Part~(1) of the definition of $X$.  So,
suppose $w\neq\%$.  Since $|w|$ is even, we have that $|w|\geq 2$, and
thus that $w=axb$ for some $a,b\in\{\zerosf,\onesf\}$ and
$x\in\{\zerosf,\onesf\}^*$.  Because $|w|$ is even, it follows that
$|x|$ is even.  Furthermore, because $w$ is a palindrome, it follows
that $a=b$ and $x$ is a palindrome.  Thus $w=axa$ and $x\in Y$.  Since
$x$ is a proper substring of $w$, the inductive hypothesis tells us
that
\begin{gather*}
\eqtxt{if}x\in Y, \eqtxt{then}x\in X .
\end{gather*}
But $x\in Y$, and thus $x\in X$.  Thus, by Part~(2) of the definition
of $X$, we have that $w=axa\in X$.
\end{proof}

We could also prove $X\sub Y$ by strong string induction.
But an alternative approach is more elegant and generally applicable:
we use the induction principle that comes from the inductive definition
of $X$.

\index{inductive definition!induction principle}%
\begin{proposition}[Principle of Induction on $X$]
Suppose $P(w)$ is a property of a string $w$.
If
\begin{enumerate}[\quad(1)]
\item\quad
\begin{gather*}
P(\%), and
\end{gather*}
\item\quad
\begin{gather*}
\eqtxtr{for all}a\in\{\mathsf{0,1}\}\eqtxt{and}x\in X,
\eqtxt{if}\eqtxt{\rm(\dag)} P(x),\eqtxt{then} P(axa) ,
\end{gather*}
\end{enumerate}
then,
\begin{gather*}
\eqtxtr{for all}w\in X,\,P(w) .
\end{gather*}
\end{proposition}

We refer to (\dag) as the \emph{inductive hypothesis} of Part~(2).
By Part~(1) of the definition of $X$, $\%\in X$.  Thus Part~(1) of the
induction principle requires us to show $P(\%)$.  By Part~(2) of the
definition of $X$, if $a\in\{\zerosf,\onesf\}$ and $x\in X$, then
$axa\in X$.  Thus in Part~(2) of the induction principle, when proving
that the ``new'' element $axa$ has property $P$, we're allowed to
assume that the ``old'' element has property $P$.

\begin{lemma}
\label{SSIProp1Lem2}
$X\sub Y$.
\end{lemma}

\begin{proof}
We use induction on $X$ to show that, for all $w\in X$, $w\in Y$.

There are two steps to show.
\begin{enumerate}[\quad(1)]
\item Since $\%\in\{\mathsf{0,1}\}^*$, $\%$ is a palindrome and
  $|\%|=0$ is even, we have that $\%\in Y$.

\item Let $a\in\{\mathsf{0,1}\}$ and $x\in X$.  Assume the inductive
  hypothesis: $x\in Y$.  We must show that $axa\in Y$.  Since $x\in
  Y$, we have that $x\in\{\mathsf{0,1}\}^*$, $x$ is a palindrome and
  $|x|$ is even.  Because $a\in\{\mathsf{0,1}\}$ and
  $x\in\{\mathsf{0,1}\}^*$, it follows that
  $axa\in\{\mathsf{0,1}\}^*$.  Since $x$ is a palindrome, we have that
  $axa$ is also a palindrome.  And, because $|axa|=|x|+2$ and $|x|$ is
  even, it follows that $|axa|$ is even.  Thus $axa\in Y$, as
  required.
\end{enumerate}
\end{proof}

\begin{proposition}
\label{SSIProp1}
$X=Y$.
\end{proposition}

\begin{proof}
Follows immediately from Lemmas~\ref{SSIProp1Lem1} and \ref{SSIProp1Lem2}.
\end{proof}

We end this subsection by proving a more complex language equality.
One of the languages is defined using a ``difference'' function
\index{difference function}%
\index{string!difference function}%
on strings, which we will use a number of times in later chapters.
Define $\diff\in\{\mathsf{0,1}\}^*\fun\ints$ by:
\index{diff@$\diff$}%
\index{string!diff@$\diff$}%
for all $w\in\{\mathsf{0,1}\}^*$,
\begin{displaymath}
\diff\,w =
\eqtxtr{the number of $\mathsf{1}$'s in}w -
\eqtxtr{the number of $\mathsf{0}$'s in}w .
\end{displaymath}
Then:
\begin{itemize}
\item $\diff\,\% = 0$;

\item $\diff\,\mathsf{1} = 1$;

\item $\diff\,\mathsf{0} = -1$; and

\item for all $x,y\in\{\mathsf{0,1}\}^*$, $\diff(xy) = \diff\,x +
\diff\,y$.
\end{itemize}
Note that, for all $w\in\{\zerosf,\onesf\}^*$, $\diff\,w=0$
iff $w$ has an equal number of $\zerosf$'s and $\onesf$'s.
If we think of a $\onesf$ as representing the production of
one unit of some resource, and of a $\zerosf$ as representing
the consumption of one unit of that resource, then a string
will have a diff of $0$ iff it is balanced in terms of production
and consumption.  Note that such such a string may have prefixes
with negative diff's, i.e., it may temporarily go ``into the red''.

Let $X$ (forget the previous definition of $X$) be the least subset of
$\{\mathsf{0,1}\}^*$ such that:
\begin{enumerate}[\quad (1)]
\item $\%\in X$;

\item for all $x,y\in X$, $xy\in X$;

\item for all $x\in X$, $\zerosf x\onesf\in X$; and

\item for all $x\in X$, $\onesf x\zerosf\in X$.
\end{enumerate}
\index{inductive definition}%
Let $Y=\setof{w\in\{\mathsf{0,1}\}^*}{\diff\,w=0}$.

For example, since $\%\in X$, it follows, by (3) and (4) that
$\zerosf\onesf=\zerosf\%\onesf\in X$ and
$\onesf\zerosf=\onesf\%\zerosf\in X$.  Thus, by (2), we have that
$\mathsf{0110}=(\zerosf\onesf)(\onesf\zerosf)\in X$.
And, $Y$ consists of all strings of $\zerosf$'s and $\onesf$'s
with an equal number of $\zerosf$'s and $\onesf$'s.

Our goal is to prove that $X=Y$, i.e., that: (the easy direction)
every string that can be constructed using $X$'s rules has an equal
number of $\zerosf$'s and $\onesf$'s; and (the hard direction)
that every string of $\zerosf$'s and $\onesf$'s with an equal number
of $\zerosf$'s and $\onesf$'s can be constructed using $X$'s
rules.

Because $X$ was defined inductively, it gives rise to an
induction principle, which we will use to prove the following lemma.
\index{inductive definition!induction principle}%
(Because of Part~(2) of the definition of $X$, we wouldn't be
able to prove this lemma using strong string induction.)

\begin{lemma}
\label{SSIProp2Lem1}
$X\sub Y$.
\end{lemma}

\begin{proof}
We use induction on $X$ to show that, for all $w\in X$, $w\in Y$.
There are four steps to show, corresponding to the four rules
of $X$'s definition.
\begin{enumerate}[\quad(1)]
\item We must show $\%\in Y$.  Since $\%\in\{\mathsf{0,1}\}^*$ and
  $\diff\,\%=0$, we have that $\%\in Y$.

\item Suppose $x,y\in X$, and assume the inductive hypothesis: $x,y\in
  Y$.  We must show that $xy\in Y$.  Since $x,y\in Y$, we have that
  $xy\in\{\mathsf{0,1}\}^*$ and $\diff(xy)=\diff\,x+\diff\,y=0+0=0$.
  Thus $xy\in Y$.

\item Suppose $x\in X$, and assume the inductive hypothesis: $x\in Y$.
  We must show that $\zerosf x\onesf\in Y$.  Since $x\in Y$, we have
  that $\zerosf x\onesf\in \{\mathsf{0,1}\}^*$ and $\diff(\zerosf
  x\onesf)= \diff\,\zerosf+\diff\,x+\diff\,\onesf=-1+0+1=0$.  Thus
  $\zerosf x\onesf\in Y$.

\item Suppose $x\in X$, and assume the inductive hypothesis: $x\in Y$.
  We must show that $\onesf x\zerosf\in Y$.  Since $x\in Y$, we have
  that $\onesf x\zerosf\in\{\mathsf{0,1}\}^*$ and $\diff(\onesf
  x\zerosf)= \diff\,\onesf+\diff\,x+\diff\,\zerosf=1+0+-1=0$.  Thus
  $\onesf x\zerosf\in Y$.
\end{enumerate}
\end{proof}

\begin{lemma}
\label{SSIProp2Lem2}
$Y\sub X$.
\end{lemma}

\begin{proof}
Since $Y\sub\{\mathsf{0,1}\}^*$, it will suffice to show that, for all
$w\in\{\mathsf{0,1}\}^*$,
\begin{displaymath}
\eqtxtr{if}w\in Y,\eqtxt{then}w\in X .
\end{displaymath}
We proceed by strong string induction.
\index{strong string induction}%
Suppose
$w\in\{\mathsf{0,1}\}^*$, and assume the inductive hypothesis: for all
$x\in \{\mathsf{0,1}\}^*$, if $x$ is a proper substring of $w$, then
\begin{displaymath}
\eqtxtr{if}x\in Y,\eqtxt{then}x\in X .
\end{displaymath}
We must show that
\begin{displaymath}
\eqtxtr{if}w\in Y,\eqtxt{then}w\in X .
\end{displaymath}
Suppose $w\in Y$.  We must show that $w\in X$.  There are three cases
to consider.
\begin{itemize}
\item Suppose $w=\%$.  Then $w=\%\in X$, by Part~(1) of the definition of $X$.

\item Suppose $w=\zerosf t$ for some $t\in\{\mathsf{0,1}\}^*$. Since
  $w\in Y$, we have that
  $-1+\diff\,t=\diff\,\zerosf+\diff\,t=\diff(\zerosf t)=\diff\,w=0$,
  and thus that $\diff\,t=1$.

  Let $u$ be the shortest prefix of $t$ such that $\diff\,u\geq 1$.
  (Since $t$ is a prefix of itself and $\diff\,t=1\geq 1$, it follows
  that $u$ is well-defined.)  Let $z\in\{\mathsf{0,1}\}^*$ be such
  that $t=uz$.  Clearly, $u\neq\%$, and thus $u=yb$ for some
  $y\in\{\mathsf{0,1}\}^*$ and $b\in\{\mathsf{0,1}\}$.  Hence
  $t=uz=ybz$.  Since $y$ is a shorter prefix of $t$ than $u$, we have
  that $\diff\,y\leq 0$.

  Suppose, toward a contradiction, that $b=\zerosf$.  Then
  $\diff\,y+-1=\diff\,y+\diff\,\zerosf=\diff\,y+\diff\,b=\diff(yb)=\diff\,u\geq
  1$, so that $\diff\,y\geq 2$.  But $\diff\,y\leq 0$---contradiction.
  Hence $b=\onesf$.

  Summarizing, we have that $u=yb=y\onesf$, $t=uz=y\onesf z$ and
  $w=\zerosf t=\zerosf y\onesf z$.  Since
  $\diff\,y+1=\diff\,y+\diff\,\onesf=\diff(y\onesf)=\diff\,u\geq 1$,
  it follows that $\diff\,y\geq 0$.  But $\diff\,y\leq 0$, and thus
  $\diff\,y=0$.  Thus $y\in Y$.  Since
  $1+\diff\,z=0+1+\diff\,z=\diff\,y+\diff\,\onesf+\diff\,z=\diff(y\onesf
  z)=\diff\,t=1$, it follows that $\diff\,z = 0$.  Thus $z\in Y$.

  Because $y$ and $z$ are proper substrings of $w$, and $y,z\in Y$,
  the inductive hypothesis tells us that $y,z\in X$.  Thus, by
  Part~(3) of the definition of $X$, we have that $\zerosf y\onesf\in
  X$.  Hence, Part~(2) of the definition of $X$ tells us that
  $w=\zerosf y\onesf z= (\zerosf y\onesf)z\in X$.

\item Suppose $w=\onesf t$ for some $t\in\{\mathsf{0,1}\}^*$. Since
  $w\in Y$, we have that
  $1+\diff\,t=\diff\,\onesf+\diff\,t=\diff(\onesf t)=\diff\,w=0$, and
  thus that $\diff\,t=-1$.

  Let $u$ be the shortest prefix of $t$ such that $\diff\,u\leq -1$.
  (Since $t$ is a prefix of itself and $\diff\,t=-1\leq -1$, it
  follows that $u$ is well-defined.)  Let $z\in\{\mathsf{0,1}\}^*$ be
  such that $t=uz$.  Clearly, $u\neq\%$, and thus $u=yb$ for some
  $y\in\{\mathsf{0,1}\}^*$ and $b\in\{\mathsf{0,1}\}$.  Hence
  $t=uz=ybz$.  Since $y$ is a shorter prefix of $t$ than $u$, we have
  that $\diff\,y\geq 0$.

  Suppose, toward a contradiction, that $b=\onesf$.  Then
  $\diff\,y+1=\diff\,y+\diff\,\onesf=\diff\,y+\diff\,b=\diff(yb)=\diff\,u\leq
  -1$, so that $\diff\,y\leq -2$.  But $\diff\,y\geq
  0$---contradiction.  Hence $b=\zerosf$.

  Summarizing, we have that $u=yb=y\zerosf$, $t=uz=y\zerosf z$ and
  $w=\onesf t=\onesf y\zerosf z$.  Since
  $\diff\,y+-1=\diff\,y+\diff\,\zerosf=\diff(y\zerosf)=\diff\,u\leq
  -1$, it follows that $\diff\,y\leq 0$.  But $\diff\,y\geq 0$, and
  thus $\diff\,y=0$.  Thus $y\in Y$.  Since
  $-1+\diff\,z=0+-1+\diff\,z=\diff\,y+\diff\,\zerosf+\diff\,z=\diff(y\zerosf
  z)= \diff\,t=-1$, it follows that $\diff\,z = 0$.  Thus $z\in Y$.

  Because $y$ and $z$ are proper substrings of $w$, and $y,z\in Y$,
  the inductive hypothesis tells us that $y,z\in X$.  Thus, by
  Part~(4) of the definition of $X$, we have that $\onesf y\zerosf\in
  X$.  Hence, Part~(2) of the definition of $X$ tells us that
  $w=\onesf y\zerosf z= (\onesf y\zerosf)z\in X$.
\end{itemize}
\end{proof}

In the proof of the preceding lemma we made use of all four rules of
$X$'s definition.  If this had not been the case, we would have known
that the unused rules were redundant (or that we had made a mistake in
our proof!).

\begin{proposition}
\label{SSIProp2}
$X=Y$.
\end{proposition}

\begin{proof}
Follows immediately from Lemmas~\ref{SSIProp2Lem1} and \ref{SSIProp2Lem2}.
\end{proof}

\begin{exercise}
Define a function $\diff\in\{\mathsf{0,1}\}^*\fun\ints$ by:
for all $w\in\{\mathsf{0,1}\}^*$,
\begin{displaymath}
\diff\,w =
\eqtxtr{the number of $\mathsf{1}$'s in}w -
\eqtxtr{the number of $\mathsf{0}$'s in}w .
\end{displaymath}
Thus $\diff\,\%=0$, $\diff\,\zerosf = -1$, $\diff\,\onesf = 1$, and
for all $x,y\in\{\mathsf{0,1}\}^*$, $\diff(xy) = \diff\,x + \diff\,y$.
Furthermore, for all $w\in\{\mathsf{0,1}\}^*$, $\diff\,w=0$ iff $w$
has an equal number of $\zerosf$'s and $\onesf$'s.
Let $X$ be the least subset of $\{\mathsf{0,1}\}^*$ such that:
\begin{enumerate}[\quad(1)]
\item $\%\in X$;

\item for all $x\in X$, $\zerosf x\onesf\in X$; and

\item for all $x,y\in X$, $xy\in X$.
\end{enumerate}
Let $Y=\setof{w\in\{\mathsf{0,1}\}^*}{\diff\,w=0\eqtxt{and,
for all prefixes}v\eqtxt{of}w,\diff\,v\leq 0}$.
Prove that $X=Y$.
\end{exercise}

\begin{exercise}
Define a function $\diff\in\{\mathsf{0,1}\}^*\fun\ints$ by:
for all $w\in\{\mathsf{0,1}\}^*$,
\begin{displaymath}
\diff\,w =
\eqtxtr{the number of $\mathsf{1}$'s in}w -
2(\eqtxtr{the number of $\mathsf{0}$'s in}w) .
\end{displaymath}
Thus $\diff\,\%=0$,
$\diff\,\zerosf = -2$,
$\diff\,\onesf = 1$, and
for all $x,y\in\{\mathsf{0,1}\}^*$, $\diff(xy) = \diff\,x + \diff\,y$.
Furthermore, for all $w\in\{\mathsf{0,1}\}^*$,
$\diff\,w=0$ iff $w$ has twice as many $\onesf$'s as $\zerosf$'s.
Let $X$ be the least subset of $\{\mathsf{0,1}\}^*$ such that:
\begin{enumerate}[\quad(1)]
\item $\%\in X$;

\item $\onesf\in X$;

\item for all $x,y\in X$, $\onesf x\onesf y\zerosf\in X$; and

\item for all $x,y\in X$, $xy\in X$.
\end{enumerate}
Let $Y=\setof{w\in\{\mathsf{0,1}\}^*}{\eqtxtr{for all
prefixes}v\eqtxt{of}w,\diff\,v\geq 0}$.
Prove that $X=Y$.
\end{exercise}

\subsection{Notes}

A novel feature of this book is the introduction and use of explicit
string induction principles, as an alternative to doing proofs by
induction (mathematical or strong) on the length of strings.  Also
novel is our focus on languages defined using ``difference''
functions.

%%% Local Variables: 
%%% mode: latex
%%% TeX-master: "book"
%%% End: 
