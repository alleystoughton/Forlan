\section{Inductive Definitions and Recursion}
\label{TreesAndInductiveDefinitions}

In this section, we will introduce and study ordered trees of
arbitrary (finite) arity, whose nodes are labeled by elements of some
set.  In later chapters, we will define regular expressions (in
Chapter~\ref{RegularLanguages}), parse trees (in
Chapter~\ref{ContextFreeLanguages}) and programs (in
Chapter~\ref{RecursiveAndRecursivelyEnumerableLanguages}) as
restrictions of the trees we consider here.

The definition of the set of all trees over a set of labels is our
first example of an inductive definition---a definition in which we
collect together all of the values that can be constructed using a set
of rules.  We will see many examples of inductive definitions in the
book.  In this section, we will also see how to define functions by
recursion.

\subsection{Inductive Definition of Trees}

\index{tree|(}%

Suppose $X$ is a set.  The set $\Tree\,X$ of $X$-\emph{trees} is the
least set such that, for all $x\in X$ and $\trs\in\List(\Tree\,X)$,
$(x,\trs)\in\Tree\,X$.  Recall that saying $\trs\in\List(\Tree\,X)$
simply means that $\trs$ is a list all of whose elements come
from $\Tree\,X$.

Ignoring the adjective ``least'' for the moment, some example elements
of $\Tree\,\nats$ (the case when the set $X$ of tree labels is $\nats$)
are:
\begin{itemize}
\item Since $3\in \nats$ and $[\,]\in\List(\Tree\,\nats)$, we have
  that $(3,[\,])\in\Tree\,\nats$.  For similar reasons, $(1,[\,])$,
  $(6,[\,])$ and all pairs of the form $(n,[\,])$, for $n\in\nats$,
  are in $\Tree\,\nats$.  

\item Because $4\in \nats$, and $[(3,[\,]), (1,[\,]),
  (6,[\,])]\in\List(\Tree\,\nats)$, we have that $(4,[(3,[\,]),
  (1,[\,]), (6,[\,])])\in\Tree\,\nats$.

\item And we can continue like this forever.
\end{itemize}

Trees are often easier to comprehend if they are drawn.
We draw the $X$-tree $(x,[\tr_1,\ldots,\tr_n])$ as
\begin{center}
\input{chap-1.3-fig1.eepic}
\end{center}
For example,
\begin{center}
\input{chap-1.3-fig2.eepic}
\end{center}
is the drawing of the $\nats$-tree $(4,[(3,[\,]), (1,[\,]), (6,[\,])])$.
And
\begin{center}
\input{chap-1.3-fig3.eepic}
\end{center}
is the $\nats$-tree $(2,[(4,[(3,[\,]), (1,[\,]), (6,[\,])]),(9,[\,])])$.
Consider the tree
\begin{center}
\input{chap-1.3-fig1.eepic}
\end{center}
again.
We say that the \emph{root label} of this tree is $x$, and $\tr_1$ is the tree's
\emph{first child}, etc.
We write $\rootLabel\,\tr$ for the root label of $\tr$.
We often write a tree $(x,[\tr_1,\ldots,\tr_n])$ in a more
compact, linear syntax:
\begin{itemize}
\item $x(\tr_1,\ldots,\tr_n)$, when $n\geq 1$, and

\item $x$, when $n=0$.
\end{itemize}
Thus $(2,[(4,[(3,[\,]), (1,[\,]), (6,[\,])]),(9,[\,])])$
can be written as $2(4(3,1,6),9)$.

Consider the definition of $\Tree\,X$ again: the set $\Tree\,X$ of
$X$-\emph{trees} is the least set such that, (\dag) for all $x\in X$ and
$\trs\in\List(\Tree\,X)$, $(x,\trs)\in\Tree\,X$.
Let's call a set $U$ $X$-\emph{closed} iff it satisfies property (\dag),
where we have replaced $\Tree\,X$ by $U$:
for all $x\in X$ and $\trs\in\List\,U$, $(x,\trs)\in U$.
Thus the definition says that $\Tree\,X$ is the least $X$-closed set,
where we've yet to say just what ``least'' means.

First we address the concern that there might not be any $X$-closed sets.
An easy result of set theory says that:

\begin{lemma}
\label{ClosureLem}
For all sets $X$, there is a set $U$ such that
$X\sub U$, $U\times U\sub U$ and $\List\,U\sub U$.
\end{lemma}

In other words, the lemma says that, given any set $X$, there
exists a superset $U$ of $X$ such that every pair of elements of $U$
is already an element of U, and every list of elements of $U$ is
already an element of $U$.  Now, suppose $X$ is a set, and let $U$ be
as in the lemma.  To see that $U$ is $X$-closed, suppose $x\in
X$ and $\trs\in\List\,U$.  Thus $x\in U$ and $\trs\in U$, so that
$(x,\trs)\in U$, as required.

E.g., we know that there is an $\ints$-closed set $U$.  But
$\nats\sub\ints$, and thus $U$ is also $\nats$-closed.  But if
$\Tree\,\nats$ turned out to be $U$, it would have elements like
$(-5,[\,])$, which are not wanted.

To keep $\Tree\,X$ from having junk, we say that $\Tree\,X$ is
the set $U$ such that:
\begin{itemize}
\item $U$ is $X$-closed, and

\item for all $X$-closed sets $V$, $U\sub V$.
\end{itemize}
This is what we mean by saying that $\Tree\,X$ is the \emph{least}
$X$-closed set. It is our first example of an \emph{inductive
  definition}, the least (relative to $\sub$) set satisfying a given
set of rules saying that if some elements are already in the set, then
some other elements are also in the set.

To see that there is a unique least $X$-closed set, we first
prove the following lemma.

\begin{lemma}
\label{ClosureIntersect}
Suppose $X$ is a set and $\cal W$ is a nonempty set of $X$-closed sets.
Then $\bigcap{\cal W}$ is an $X$-closed set.
\end{lemma}

\begin{proof}
Suppose $X$ is a set and $\cal W$ is a nonempty set of $X$-closed
sets.  Because $\cal W$ is nonempty, $\bigcap\cal W$ is well-defined.
To see that $\bigcap{\cal W}$ is $X$-closed, suppose $x\in X$ and
$\trs\in\List\,\bigcap{\cal W}$.  We must show that $(x,trs)\in
\bigcap{\cal W}$, i.e., that $(x,trs)\in W$, for all $W\in\cal W$.
Suppose $W\in\cal W$.  We must show that $(x,\trs)\in W$.  Because
$W\in\cal W$, we have that $\bigcap{\cal W}\sub W$, so that
$\List\,\bigcap{\cal W}\sub\List\,W$.  Thus, since
$\trs\in\List\,\bigcap{\cal W}$, it follows that $\trs\in\List\,W$.
But $W$ is $X$-closed, and thus $(x,\trs)\in W$, as required.
\end{proof}

As explained above, we have that there is an $X$-closed set, $V$.  Let
$\cal W$ be the set of all subsets of $V$ that are $X$-closed.  Thus
$\cal W$ is a nonempty set of $X$-closed sets, since $V\in\cal W$.
Let $U=\bigcap{\cal W}$.  By Lemma~\ref{ClosureIntersect}, we have
that $U$ is $X$-closed.  To see that $U\sub T$ for all $X$-closed sets
$T$, suppose $T$ is $X$-closed.  By Lemma~\ref{ClosureIntersect}, we
have that $V\cap T=\bigcap\{V,T\}$ is $X$-closed.  And $V\cap T\sub
V$, so that $V\cap T\in\cal W$.  Hence $U=\bigcap{\cal W}\sub V\cap
T\sub T$, as required.  Finally, suppose that $U'$ is also an
$X$-closed set such that, for all $X$-closed sets $T$, $U'\sub T$.
Then $U\sub U'\sub U$, showing that $U'=U$.  Thus $U$ is \emph{the} least
$X$-closed set.

Suppose $X$ is a set, $x,x'\in X$ and $\trs,\trs'\in\List\,(\Tree\,X)$.
It is easy to see that $(x,\trs) = (x',\trs')$ iff $x=x'$, $|\trs|=|\trs'|$
and, for all $i\in[1:|\trs|]$, $trs\,i = \trs'\,i$.

Because trees are defined via an inductive definition, we
get an induction principle for trees almost for free:

\begin{theorem}[Principle of Induction on Trees]
Suppose $X$ is a set and $P(\tr)$ is a property of an element $\tr\in\Tree\,X$.
If
\begin{ctabbing}
for all $x\in X$ and $\trs\in\List(\Tree\,X)$, \\
if {\rm(\dag)} for all $i\in[1:|\trs|]$, $P(\trs\,i)$, \\
then $P((x,\trs))$,
\end{ctabbing}
then
\begin{gather*}
\eqtxtr{for all}\tr\in\Tree\,X,\,P(\tr) .
\end{gather*}
\end{theorem}

We refer to (\dag) as the inductive hypothesis.

\begin{proof}
Suppose $X$ is a set, $P(\tr)$ is a property of an element $\tr\in\Tree\,X$,
and
\begin{ctabbing}
(\ddag) \=\+ for all $x\in X$ and $\trs\in\List(\Tree\,X)$, \\
if for all $i\in[1:|\trs|]$, $P(\trs\,i)$, \\
then $P((x,\trs))$.
\end{ctabbing}
We must show that
\begin{gather*}
\eqtxtr{for all}\tr\in\Tree\,X,\,P(\tr) .
\end{gather*}

Let $U=\setof{\tr\in\Tree\,X}{P(\tr)}$.
We will show that $U$ is $X$-closed.
Suppose $x\in X$ and $\trs\in\List\,U$.  We must show
that $(x,\trs)\in U$.
It will suffice to show that $P((x,\trs))$.
By (\ddag), it will suffice to show that, for all $i\in[1:|\trs|]$,
$P(\trs\,i)$.
Suppose $i\in[1:|\trs|]$.  We must show that
$P(\trs\,i)$.
Because $\trs\in\List\,U$, we have that
$\trs\,i\in U$.
Hence $P(\trs\,i)$, as required.

Because $U$ is $X$-closed, we have that $\Tree\,X\sub U$, as
$\Tree\,X$ is the least $X$-closed set.  Hence, for all
$\tr\in\Tree\,X$, $\tr\in U$, so that, for all $\tr\in\Tree\,X$,
$P(\tr)$.
\end{proof}

Using our induction principle, we can now prove that every $X$-tree
can be ``destructed'' into an element of $X$ paired with a list of
$X$-trees:

\begin{proposition}
\label{TreeDestruct}
Suppose $X$ is a set.  For all $\tr\in\Tree\,X$, there are $x\in X$
and $\trs\in\List(\Tree\,X)$ such that $\tr = (x, \trs)$.
\end{proposition}

\begin{proof}
Suppose $X$ is a set.  We use induction on trees to prove that, for
all $\tr\in\Tree\,X$, there are $x\in X$ and $\trs\in\List(\Tree\,X)$
such that $\tr = (x, \trs)$.  Suppose $x\in X$,
$\trs\in\List(\Tree\,X)$, and assume the inductive hypothesis: for all
$i\in[1:|\trs|]$, there are $x'\in X$ and $\trs'\in\List(\Tree\,X)$
such that $\trs\,i = (x', \trs')$. We must show that there are $x'\in
X$ and $\trs'\in\List(\Tree\,X)$ such that $(x,\trs) =
(x',\trs')$. And this holds, since $x\in X$, $\trs\in\List(\Tree\,X)$
and $(x, \trs) = (x, \trs)$.
\end{proof}

Note that the preceding induction makes no use of its inductive
hypothesis, and yet the induction is still necessary.

Suppose $X$ is a set.  Let the predecessor relation $\pred_{\Tree\,X}$
on $\Tree\,X$ be the set of all pairs of $X$-trees $(\tr,\tr')$ such
that there are $x\in X$ and $\trs'\in\List(\Tree\,X)$ such that
$\tr'=(x,\trs')$ and $\trs'\,i=\tr$ for some $i\in[1:|\trs'|]$,
i.e., such that $\tr$ is one of the children of $\tr'$.
Thus the predecessors of a tree
$(x,[\tr_1,\ldots,\tr_n])$ are its children $\tr_1$, \ldots, $\tr_n$.

\begin{proposition}
If $X$ is a set, then $\pred_{\Tree\,X}$ is a well-founded relation
on $\Tree\,X$.
\end{proposition}

\begin{proof}
Suppose $X$ is a set and $Y$ is a nonempty subset of $\Tree\,X$.
Mimicking Proposition~1.2.5, we can use the principle of induction
on trees to prove that, for all $\tr\in\Tree\,X$, if $\tr\in Y$,
then $Y$ has a $\pred_{\Tree\,X}$-minimal element.  Because $Y$
is nonempty, we can conclude that $Y$ has a
$\pred_{\Tree\,X}$-minimal element.
\end{proof}

\begin{exercise}
Do the induction on trees used by the preceding proof.
\end{exercise}

\index{tree|)}%

\subsection{Recursion}

\index{recursion|(}%
Suppose $R$ is a well-founded relation on a set $A$.  We can define a
function $f$ from $A$ to a set $B$ by \emph{well-founded recursion on}
$R$.  The idea is simple: when $f$ is called with an element $x\in A$,
it may call itself recursively on as many of the predecessors of $x$ in
$R$ as it wants.  Typically, such a definition will be concrete enough
that we can regard it as defining an algorithm as well as a function.

If $R$ is a well-founded relation on a set $A$, and $B$ is a set,
then we write $\Rec_{A,R,B}$ for
\begin{gather*}
\setof{(x,f)}{x\in A\eqtxt{and}f\in\setof{y\in A}{y\mathrel{R}x}\fun B}
\fun B .
\end{gather*}
An element $F$ of $\Rec_{A,R,B}$ may only be called with a pair
$(x,f)$ such that $x\in A$ and $f$ is a function from the predecessors
of $x$ in $R$ to $B$.  Intutively, $F$'s job is to transform $x$
into an element of $B$, using $f$ to carry out recursive calls.

\begin{theorem}[Well-Founded Recursion]
\label{WellFoundedRecursion}
Suppose $R$ is a well-founded relation on a set $A$, $B$ is
a set, and $F\in\Rec_{A,R,B}$.  There is a unique $f\in A\fun B$
such that, for all $x\in A$,
\begin{gather*}
f\,x = F(x, f\restr\setof{y\in A}{y\mathrel{R}x}) .
\end{gather*}
\end{theorem}
 
The second argument to $F$ in the definition of $f$ is the restriction
of $f$ to the predecessors of $x$ in $R$, i.e., it's the subset of
$f$ whose domain is $\setof{y\in A}{y\mathrel{R}x}$.

If we can understand $F$ as an algorithm, then we can understand the
definition of $f$ as an algorithm.  If we call $f$ with an $x\in A$,
then $F$ may return an element of $B$ without consulting its second
argument.  Alternatively, it may call this second argument with a
predecessor of $x$ in $R$.  This is a recursive call of $f$, which must
complete before $F$ continues.  Once it does complete, $F$ may make
more recursive calls, but must eventually return an element of $B$.

\begin{proof}
We start with an inductive definition:
let $f$ be the least subset of $A\times B$ such that, for
all $x\in A$ and $g\in\setof{y\in A}{y\mathrel{R}x}\fun B$,
\begin{gather*}
\eqtxtr{if} g\sub f, \eqtxt{then} (x, F(x, g))\in f .
\end{gather*}
We say that a subset $U$ of $A\times B$ is \emph{closed} iff, for
all $x\in A$ and $g\in\setof{y\in A}{y\mathrel{R}x}\fun B$,
\begin{gather*}
\eqtxtr{if} g\sub U, \eqtxt{then} (x, F(x, g))\in U .
\end{gather*}
Thus, we are saying that $f$ is the least closed subset of
$A\times B$.
This definition is well-defined because $A\times B$ is closed,
and if $\cal W$ is a nonempty set of closed subsets
of $A\times B$, then $\bigcap\cal W$ is also closed.
Thus we can let $f$ be the intersection of
all closed subsets of $A\times B$.

Thus $f$ is a relation from $A$ to $B$.  An easy well-founded
induction on $R$ suffices to show that, for all $x\in A$,
$x\in\domain\,f$.  Suppose $x\in A$, and assume the inductive
hypothesis: for all $y\in A$, if $y\mathrel{R}x$, then $y\in\domain\,f$.
We must show that $x\in\domain\,f$.  By the inductive hypothesis,
we have that for all $y\in\setof{y\in A}{y\mathrel{R}x}$,
there is a $z\in B$ such that $(y,z)\in f$.  Thus there
is a subset $g$ of $f$ such that $g\in\setof{y\in A}{y\mathrel{R}x}\fun B$.
(Since we don't know at this point that $f$ is a function, we
are using the Axiom of Choice in this last step.)
Hence $(x,F(x,g))\in f$, showing that $x\in\domain\,f$.
Thus $\domain\,f=A$.

Next we show that $f$ is a function.  Let $h$ be
\begin{gather*}
\setof{(x,y)\in f}{\eqtxtr{for all}y'\in B,\eqtxt{if}(x,y')\in f,
\eqtxt{then} y = y'} .
\end{gather*}
If we can show that $h$ is closed, then we will have that $f\sub h$,
because $f$ is the least closed set, and thus we'll be able to
conclude that $f$ is a function.  To show that $h$ is closed, suppose
$x\in A$, $g\in\setof{y\in A}{y\mathrel{R}x}\fun B$ and $g\sub h$.  We
must show that $(x, F(x, g))\in h$.  It will suffice to show that, for
all $y'\in B$, if $(x,y')\in f$, then $F(x, g) = y'$.  Suppose $y'\in
B$ and $(x,y')\in f$.  We must show that $F(x, g) = y'$.  Because
$(x,y')\in f$, and $f$ is the least closed subset of $A\times B$,
there must be a $g'\in \setof{y\in
  A}{y\mathrel{R}x}\fun B$ such that $g'\sub f$ and $y'=F(x, g')$.
Thus it will suffice to show that $F(x,g) = F(x, g')$, which will follow
from showing that $g=g'$, i.e., for all $z\in\setof{y\in
  A}{y\mathrel{R}x}$, $g\,z=g'\,z$.  Suppose $z\in\setof{y\in
  A}{y\mathrel{R}x}$.  We must show that $g\,z=g'\,z$.  Since
$z\in\setof{y\in A}{y\mathrel{R}x}$, we have that $z\in A$ and
$z\mathrel{R}x$.  Because $(z,g\,z)\in g\sub h$, we have that
$(z,g\,z)\in h$.  Since $(z,g'\,z)\in g'\sub f$, we have that
$(z,g'\,z)\in f$.  Hence, by the definition of $h$, we have that
$g\,z=g'\,z$, as required.

Summarizing, we know that $f\in A\fun B$.  Next, we must show that,
for all $x\in A$,
\begin{gather*}
f\,x = F(x, f\restr\setof{y\in A}{y\mathrel{R}x}) .
\end{gather*}
Suppose $x\in A$.  Because $f\restr\setof{y\in A}{y\mathrel{R}x}\in
\setof{y\in A}{y\mathrel{R}x}\fun B$, we have that
$(x,F(x,f\restr\setof{y\in A}{y\mathrel{R}x}))\in f$, so that
$f\,x = F(x, f\restr\setof{y\in A}{y\mathrel{R}x})$.

Finally, suppose that $f'\in A\fun B$ and
for all $x\in A$,
\begin{gather*}
f'\,x = F(x, f'\restr\setof{y\in A}{y\mathrel{R}x}) .
\end{gather*}
To see that $f=f'$, it will suffice to show that, for all
$x\in A$, $f\,x=f'\,x$.  We proceed by well-founded induction
on $R$.  Suppose $x\in A$ and assume the inductive hypothesis:
for all $y\in A$, if $y\mathrel{R}x$, then $f\,y=f'\,y$.
We must show that $f\,x=f'\,x$.  By the inductive hypothesis,
we have that $f\restr\setof{y\in A}{y\mathrel{R}x} =
f'\restr\setof{y\in A}{y\mathrel{R}x}$.  
Thus
\begin{align*}
f\,x &= F(x, f\restr\setof{y\in A}{y\mathrel{R}x}) \\
&= F(x, f'\restr\setof{y\in A}{y\mathrel{R}x}) \\
&= f'\,x ,
\end{align*}
as required.
\end{proof}

Here are some examples of well-founded recursion:
\begin{itemize}
\item If we define $f\in\nats\fun B$ by well-founded recursion on $<$,
  then, when $f$ is called with $n\in\nats$, it may call itself
  recursively on any strictly smaller natural numbers.  In the case
  $n=0$, it can't make any recursive calls.

\item If we define $f\in\nats\fun B$ by well-founded recursion on
  the predecessor relation $\pred_\nats$, then when $f$ is called with
  $n\in\nats$, it may call itself recursively on $n-1$, in the
  case when $n\geq 1$, and may make no recursive calls, when $n=0$.

  Thus, if such a definition case-splits according to whether
  its input is $0$ or not, it can be split into two parts:
  \begin{itemize}
  \item $f\,0 = \cdots$;

  \item for all $n\in\nats$, $f(n + 1) = \cdots f\,n \cdots$.
  \end{itemize}
  We say that such a definition is by \emph{recursion on} $\nats$.

\item If we define $f\in\Tree\,X\fun B$ by well-founded recursion on
  the predecessor relation $\pred_{\Tree\,X}$, then when $f$ is called
  on an $X$-tree $(x,[\tr_1,\ldots,\tr_n])$, it may call itself
  recursively on any of $\tr_1$, \ldots, $\tr_n$.  When $n=0$, it may
  make no recursive calls. We say that such a definition is by
  \emph{structural recursion}.

\item We may define the \emph{size} of an $X$-tree
  $(x,[\tr_1,\ldots,\tr_n])$ by summing the recursively computed sizes
  of $\tr_1$, \ldots, $\tr_n$, and then adding $1$. (When $n=0$, the
  sum of no sizes is $0$, and so we get $1$.) Then, e.g., the size of
\begin{center}
\input{chap-1.3-fig3.eepic}
\end{center}
is $6$. This defines a function ${\size}\in\Tree\,X\fun\nats$.

\item We may define the \emph{number of leaves} of an $X$-tree
  $(x,[\tr_1,\ldots,\tr_n])$ as
  \begin{itemize}
  \item $1$, when $n=0$, and

  \item the sum of the recursively computed numbers of leaves of
    $\tr_1$, \ldots, $\tr_n$, when $n\geq 1$.
  \end{itemize}
  Then, e.g., the number of leaves of
  \begin{center}
    \input{chap-1.3-fig3.eepic}
  \end{center}
  is $4$. This defines a function ${\numLeaves}\in\Tree\,X\fun\nats$.

\item We may define the \emph{height} of an $X$-tree
  $(x,[\tr_1,\ldots,\tr_n])$ as
  \begin{itemize}
  \item $0$, when $n=0$, and

  \item $1$ plus the maximum of the recursively computed heights
    of $\tr_1$, \ldots, $\tr_n$, when $n\geq 1$.
  \end{itemize}
  E.g., the height of
  \begin{center}
    \input{chap-1.3-fig3.eepic}
  \end{center}
  is $2$. This defines a function ${\height}\in\Tree\,X\fun\nats$.

\item Given a set $X$, we can define a well-founded relation
  $\size_{\Tree\,X}$ on $\Tree\,X$ by: for all $\tr,\tr'\in\Tree\,X$,
  $\tr\mathrel{\size_{\Tree\,X}}\tr'$ iff
  $\size\,\tr < {\size}\,\tr'$.  (This is an application of
  Proposition~\ref{InverseImageWellFounded}.)

  If we define a function $f\in\Tree\,X\fun B$ by well-founded recursion
  on $\size_{\Tree\,X}$, when $f$ is called with an $X$-tree $\tr$,
  it may call itself recursively on any $X$-trees with strictly smaller sizes.
  
\item Given a set $X$, we can define a well-founded relation
  $\height_{\Tree\,X}$ on $\Tree\,X$ by: for all
  $\tr,\tr'\in\Tree\,X$, $\tr\mathrel{\height_{\Tree\,X}}\tr'$ iff
  $\height\,\tr < {\height}\,\tr'$.

  If we define a function $f\in\Tree\,X\fun B$ by well-founded
  recursion on $\height_{\Tree\,X}$, when $f$ is called with an
  $X$-tree $\tr$, it may call itself recursively on any $X$-trees with
  strictly smaller heights.

\item Given a set $X$, we can define a well-founded relation
  $\length_{\List\,X}$ on $\List\,X$ by: for all $\xs,\ys\in\List\,X$,
  $\xs\mathrel{\length_{\List\,X}}\ys$ iff $|\xs|<|\ys|$.

  If we define a function $f\in\List\,X\fun B$ by well-founded
  recursion on $\length_{\List\,X}$, when $f$ is called with an
  $X$-list $\xs$, it may call itself recursively on any $X$-lists with
  strictly smaller lengths.
\end{itemize}
\index{recursion|)}%

\subsection{Paths in Trees}

We can think of an $\nats-\{0\}$-list $[n_1, n_2, \ldots, n_m]$ as a
\emph{path} through an $X$-tree $\tr$: one starts with $\tr$ itself,
goes to the $n_1$-th child of $\tr$, selects the $n_2$-th child of
that tree, etc., stopping when the list is exhausted.

Consider the $\nats$-tree
\begin{center}
\input{chap-1.3-fig3.eepic}
\end{center}
Then:
\begin{itemize}
\item $[\,]$ takes us to the whole tree.

\item $[1]$ takes us to the tree $4(3,1,6)$.

\item $[1, 3]$ takes us to the tree $6$.

\item $[1, 4]$ takes us to no tree.
\end{itemize}

We define the valid paths of an $X$-tree via structural recursion.
For a set $X$, we define $\validPaths_X\in\Tree\,X\fun\List(\nats-\{0\})$
(we often drop the subscript $X$, when it's clear from the context)
by: for all $x\in X$ and $\trs\in\List(\Tree\,X)$,
$\validPaths(x, \trs)$ is
\begin{gather*}
\{[\,]\} \cup
\setof{[i]\myconcat\xs}{i\in[1:|\trs|]\eqtxt{and}\xs\in\validPaths(\trs\,i)} .
\end{gather*}
We say that $\xs\in\List(\nats-\{0\})$ \emph{is a valid path for}
an $X$-tree $\tr$ iff $\xs\in\validPaths\,\tr$.
For example, $\validPaths(3(4,1(7),6)) = \{[\,], [1], [2], [2,1], [3]\}$.
Thus, e.g.,  $[2,1]$ is a valid path for $3(4,1(7),6)$, whereas
$[2,2]$ and $[4]$ are not valid paths for this tree.

Now, we define a function $\select$ that takes in an $X$-tree $\tr$ and
a valid path $\xs$ for $\tr$, and returns the subtree of
$\tr$ pointed to by $\xs$.
Let $Y_X$ be
\begin{gather*}
\setof{(\tr,\xs)\in\Tree\,X\times\List(\nats-\{0\})}%
{\xs\eqtxt{is a valid path for}\tr} .
\end{gather*}
Let the relation
$R$ on $Y_X$ be
\begin{gather*}
\setof{((\tr,\xs),(\tr',\xs'))\in Y_X\times Y_X}{|\xs|<|\xs'|} .
\end{gather*}
By Proposition~\ref{InverseImageWellFounded}, we have that $R$ is
well-founded, and so we can use well-founded recursion on $R$ to
define a function $\select_X$ (we often drop the subscript $X$) from
$Y_X$ to $\Tree\,X$.  Suppose, we are given an input $((x,\trs),\xs)\in
Y$.  If $\xs$ is $[\,]$, then we return $(x,\trs)$.  Otherwise,
$\xs=[i]\myconcat\xs'$, for some $i\in\nats-\{0\}$ and
$\xs'\in\List(\nats-\{0\})$.  Because $((x,\trs),\xs)\in Y$, it
follows that $i\in[1:|\trs|]$ and $\xs'$ is a valid path for
$\trs\,i$.  Thus $(\trs\,i, \xs')$ is in $Y$, and is a predecessor of
$((x,\trs),\xs)$ in $R$, so that we can call ourselves recursively on
$(\trs\,i, \xs')$ and return the resulting tree.  For example
$\select(4(3(2,1(7)),6), [\,]) = 4(3(2,1(7)),6)$ and
$\select(4(3(2,1(7)),6),[1,2]) = 1(7)$.

We say that an $X$-tree $\tr'$ \emph{is a subtree of} an $X$-tree
$\tr$ iff there is a valid path $\xs$ for $\tr$ such that
$\tr'=\select(\tr,\xs)$.  And $\tr'$ \emph{is a leaf of} $\tr$
iff $\tr'$ is a subtree of $\tr$ and $\tr'$ has no children.

Finally, we can define a function that takes in an $X$-tree $\tr$,
a valid path $\xs$ for $\tr$, and an $X$-tree $\tr'$, and
returns the result of replacing the subtree of $\tr$ pointed
to by $\xs$ with $\tr'$.
Let $Y_X$ be
\begin{gather*}
\setof{(\tr,\xs,\tr')
\in\Tree\,X\times\List(\nats-\{0\})\times\Tree\,X}%
{\xs\eqtxt{is a valid path for}\tr} .
\end{gather*}
We use well-founded recursion on the size of the second components
(the paths) of the elements of $Y_X$ to define a function $\update_X$
(we often drop the subscript) from $Y_X$ to $\Tree\,X$.
Suppose, we are given an input $((x,\trs),\xs,\tr')\in Y$.  If
$\xs$ is $[\,]$, then we return $\tr'$.  Otherwise,
$\xs=[i]\myconcat\xs'$, for some $i\in\nats-\{0\}$ and
$\xs'\in\List(\nats-\{0\})$.  Because $((x,\trs),\xs,\tr')\in Y$,
it follows that $i\in[1:|\trs|]$ and $\xs'$ is a valid path
for $\trs\,i$.  Thus $(\trs\,i, \xs', \tr')$ is in $Y$, and
$|\xs'|<|\xs|$.  Hence, we can let $\tr''$ be the result of
calling ourselves recursively on $(\trs\,i, \xs', \tr')$.
Finally, we can return $(x,\trs')$, where $\trs'=
\trs[i\mapsto\tr'']$ (which is the same as $\trs$, excepts that
its $i$th element is $\tr''$.
For example $\update(4(3(2,1(7)),6), [\,], 3(7,8)) = 3(7,8)$ and
$\update(4(3(2,1(7)),6), [1,2], 3(7,8)) =
4(3(2,3(7,8)),6)$.

\subsection{Notes}

Our treatment of trees, inductive definition, and well-founded
recursion is more formal than what one finds in typical books on
formal language theory.  But those with a background in set theory
will find nothing surprising in this section, and our foundational
approach will serve the reader well in later chapters.

%%% Local Variables: 
%%% mode: latex
%%% TeX-master: "book"
%%% End: 
