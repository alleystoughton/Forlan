\section{Nondeterministic Finite Automata}
\label{NondeterministicFiniteAutomata}

\index{finite automaton!nondeterministic|(}%
\index{nondeterministic finite automaton|(}%
\index{NFA|(}%

In this section, we study the second of our more restricted kinds of
finite automata: nondeterministic finite automata.

\subsection{Definition of NFAs}

A \emph{nondeterministic finite automaton} (NFA) $M$ is a finite
automaton such that
\begin{gather*}
T_M\sub\setof{q,x\fun r}{q,r\in\Sym\eqtxt{and}x\in\Str\eqtxt{and}|x|=1}.
\end{gather*}
In other words, an FA is an NFA iff every string of every transition
of the FA has a single symbol.
For example, $\Asf,\onesf\fun \Bsf$ is a legal NFA transition, but
$\Asf, \%\fun \Bsf$ and $\Asf,\onesf\onesf\fun \Bsf$ are not legal.
We write $\NFA$ for the set of all nondeterministic finite automata.
\index{NFA@$\NFA$}%
\index{finite automaton!NFA@$\NFA$}%
\index{nondeterministic finite automaton!NFA@$\NFA$}%
Thus $\NFA\subsetneq\EFA\subsetneq\FA$.

The following proposition obviously holds.

\begin{proposition}
Suppose $M$ is an NFA.
\begin{itemize}
\item For all $N\in\FA$, if $M\iso N$, then $N$ is an NFA.

\item For all bijections $f$ from $Q_M$ to some set of symbols,
$\renameStates(M, f)$ is an NFA.

\item $\renameStatesCanonically\,M$ is an NFA.

\item $\simplify\,M$ is an NFA.
\end{itemize}
\end{proposition}

\subsection{Converting EFAs to NFAs}

\index{empty-string finite automaton!converting EFA to NFA}%
\index{nondeterministic finite automaton!converting EFA to NFA}%

Suppose $M$ is the EFA
\begin{center}
\input{chap-3.10-fig2.eepic}
\end{center}
To convert $M$ into an equivalent NFA, we will have to:
\begin{itemize}
\item replace the transitions $\Asf,\%\fun\Bsf$ and $\Bsf,\%\fun\Csf$
  with legal transitions (for example, because of the valid labeled
  path
  \begin{gather*}
    \Asf\lparr{\%}\Bsf\lparr{\onesf}\Bsf\lparr{\%}\Csf,
  \end{gather*}
  we will add the transition $\Asf,\onesf\fun\Csf$);

\item make (at least) $\Asf$ be an accepting state (so that $\%$ is
  accepted by the NFA).
\end{itemize}

Before defining a general procedure for converting EFAs to NFAs, we
first say what we mean by the empty-closure of a set of states.
Suppose $M$ is a finite automaton and $P\sub Q_M$.  The
\emph{empty-closure} of $P$ ($\emptyClose_M\,P$) is the least subset
$X$ of $Q_M$ such that
\begin{itemize}
\item $P\sub X$; and

\item for all $q,r\in Q_M$, if $q\in X$ and $q,\%\fun r\in T_M$, then
$r\in X$.
\end{itemize}
We sometimes abbreviate $\emptyClose_M\,P$ to $\emptyClose\,P$, when
$M$ is clear from the context.
For example, if $M$ is our example EFA and $P=\{\Asf\}$, then:
\begin{itemize}
\item $\Asf\in X$;

\item $\Bsf\in X$, since $\Asf\in X$ and $\Asf,\%\fun\Bsf\in T_M$;

\item $\Csf\in X$, since $\Bsf\in X$ and $\Bsf,\%\fun\Csf\in T_M$.
\end{itemize}
Thus $\emptyClose\,P=\{\Asf,\Bsf,\Csf\}$.

Suppose $M$ is a finite automaton and $P\sub Q_M$.  The \emph{backwards
empty-closure} of $P$ ($\emptyCloseBackwards_M\,P$) is the least
subset $X$ of $Q_M$ such that
\begin{itemize}
\item $P\sub X$; and

\item for all $q,r\in Q_M$, if $r\in X$ and $q,\%\fun r\in T_M$, then
$q\in X$.
\end{itemize}
We sometimes drop the $M$ from $\emptyCloseBackwards_M$, when
it's clear from the context.
For example, if $M$ is our example EFA and $P=\{\Csf\}$, then:
\begin{itemize}
\item $\Csf\in X$;

\item $\Bsf\in X$, since $\Csf\in X$ and $\Bsf,\%\fun\Csf\in T_M$;

\item $\Asf\in X$, since $\Bsf\in X$ and $\Asf,\%\fun\Bsf\in T_M$.
\end{itemize}
Thus $\emptyCloseBackwards\,P=\{\Asf,\Bsf,\Csf\}$.
  
\begin{proposition}
Suppose $M$ is a finite automaton.  For all $P\sub Q_M$,
\begin{displaymath}
\emptyClose_M\,P=\Delta_M(P,\%) .  
\end{displaymath}
\end{proposition}

In other words, $\emptyClose_M\,P$ is all of the states that can be
reached from elements of $P$ by sequences of $\%$-transitions.

\begin{proposition}
Suppose $M$ is a finite automaton.  For all $P\sub Q_M$,
\begin{displaymath}
\emptyCloseBackwards_M\,P=
\setof{q\in Q_M}{\Delta_M(\{q\},\%)\cap P\neq\emptyset} .
\end{displaymath}
\end{proposition}

In other words, $\emptyCloseBackwards_M\,P$ is all of the states from
which it is possible to reach elements of $P$ by sequences of
$\%$-transitions.

We define a function/algorithm $\efaToNFA\in\EFA\fun\NFA$ that
\index{empty-string finite automaton!efaToNFA@$\efaToNFA$}%
\index{nondeterministic finite automaton!efaToNFA@$\efaToNFA$}%
converts EFAs into NFAs by saying that $\efaToNFA\,M$ is the NFA $N$
such that:
\begin{itemize}
\item $Q_N=Q_M$;

\item $s_N=s_M$;

\item $A_N=\emptyCloseBackwards\,A_M$; and

\item $T_N$ is the set of all transitions $q', a\fun r'$ such that
  $q',r'\in Q_M$, $a\in\Sym$, and there are $q,r\in Q_M$ such that:
\begin{itemize}
\item $q,a\fun r\in T_M$;
\item $q'\in{\emptyCloseBackwards\,\{q\}}$; and
\item $r'\in{\emptyClose\,\{r\}}$.
\end{itemize}
\end{itemize}

To compute the set $T_N$, we process each transition $q,x\fun r$ of
$M$ as follows.  If $x=\%$, then we generate no transitions.
Otherwise, our transition is $q, a\fun r$ for some symbol $a$.  We
then compute the backwards empty-closure of $\{q\}$, and call the
result $X$, and compute the (forwards) empty-closure of $\{r\}$, and
call the result $Y$.  We then add all of the elements of
\begin{gather*}
\setof{q', a\fun r'}{q'\in X\eqtxt{and}r'\in Y}
\end{gather*}
to $T_N$.

Because the algorithm defines $A_N$ to be $\emptyCloseBackwards\,A_M$,
it could let $T_N$ be the set of all transitions $q',a\fun r$
such that $q',r\in Q_M$, $a\in\Sym$, and there is a $q\in Q_M$ such
that:
\begin{itemize}
\item $q,a\fun r\in T_M$; and
\item $q'\in{\emptyCloseBackwards\,\{q\}}$.
\end{itemize}
This would mean that $N$ would have fewer transitions.  However, for
aesthetic reasons, we'll stick with the symmetric definition of $T_N$.

Let $M$ be our example EFA
\begin{center}
\input{chap-3.10-fig2.eepic}
\end{center}
and let $N=\efaToNFA\,M$.  Then
\begin{itemize}
\item $Q_N=Q_M=\{\Asf,\Bsf,\Csf\}$;

\item $s_N=s_M=\Asf$;

\item $A_N=\emptyCloseBackwards\,A_M=\emptyCloseBackwards\,\{\Csf\}=
  \{\Asf,\Bsf,\Csf\}$.
\end{itemize}
Now, let's work out what $T_N$ is, by processing each of $M$'s transitions.
\begin{itemize}
\item From the transitions $\Asf,\%\fun\Bsf$ and $\Bsf,\%\fun\Csf$, we
  get no elements of $T_N$.

\item Consider the transition $\Asf,\zerosf\fun\Asf$.  Since
  $\emptyCloseBackwards\,\{\Asf\}=\{\Asf\}$ and
  $\emptyClose\,\{\Asf\}=\{\Asf,\Bsf,\Csf\}$, we add
  $\Asf,\zerosf\fun\Asf$, $\Asf,\zerosf\fun\Bsf$ and
  $\Asf,\zerosf\fun\Csf$ to $T_N$.

\item Consider the transition $\Bsf,\onesf\fun\Bsf$.  Since
  $\emptyCloseBackwards\,\{\Bsf\}=\{\Asf,\Bsf\}$ and
  $\emptyClose\,\{\Bsf\}=\{\Bsf,\Csf\}$, we add
  $\Asf,\onesf\fun\Bsf$, $\Asf,\onesf\fun\Csf$, $\Bsf,\onesf\fun\Bsf$
  and $\Bsf,\onesf\fun\Csf$ to $T_N$.

\item Consider the transition $\Csf,\twosf\fun\Csf$.  Since
  $\emptyCloseBackwards\,\{\Csf\}=\{\Asf,\Bsf,\Csf\}$ and
  $\emptyClose\,\{\Csf\}=\{\Csf\}$, we add
  $\Asf,\twosf\fun\Csf$, $\Bsf,\twosf\fun\Csf$ and $\Csf,\twosf\fun\Csf$
  to $T_N$.
\end{itemize}

Thus our NFA $N$ is
\begin{center}
\input{chap-3.10-fig3.eepic}
\end{center}

\begin{theorem}
\label{EFAToNFATheorem}
For all $M\in\EFA$:
\begin{itemize}
\item $\efaToNFA\,M\approx M$; and

\item $\alphabet(\efaToNFA\,M) = \alphabet\,M$.
\end{itemize}
\end{theorem}

\subsection{Converting EFAs to NFAs, and
Processing NFAs in Forlan}

The Forlan module \texttt{FA} defines the following functions
for computing forwards and backwards empty-closures:
\begin{verbatim}
val emptyClose          : fa -> sym set -> sym set
val emptyCloseBackwards : fa -> sym set -> sym set
\end{verbatim}
\index{FA@\texttt{FA}!emptyClose@\texttt{emptyClose}}%
\index{FA@\texttt{FA}!emptyCloseBackwards@\texttt{emptyCloseBackwards}}%

For example, if \texttt{fa} is bound to the finite automaton
\begin{center}
\input{chap-3.10-fig2.eepic}
\end{center}
then we can compute the empty-closure of $\{\Asf\}$ as follows:
\begin{list}{}
{\setlength{\leftmargin}{\leftmargini}
\setlength{\rightmargin}{0cm}
\setlength{\itemindent}{0cm}
\setlength{\listparindent}{0cm}
\setlength{\itemsep}{0cm}
\setlength{\parsep}{0cm}
\setlength{\labelsep}{0cm}
\setlength{\labelwidth}{0cm}
\catcode`\#=12
\catcode`\$=12
\catcode`\%=12
\catcode`\^=12
\catcode`\_=12
\catcode`\.=12
\catcode`\?=12
\catcode`\!=12
\catcode`\&=12
\ttfamily}
\small
\item[]\textsl{-\ }SymSet.output("",\ FA.emptyClose\ fa\ (SymSet.input\ ""));
\item[]\textsl{@\ }A
\item[]\textsl{@\ }.
\item[]\textsl{A,\ B,\ C}
\item[]\textsl{val\ it\ =\ ()\ :\ unit}
\end{list}


The Forlan module \texttt{NFA} defines an abstract type \texttt{nfa}
\index{NFA@\texttt{NFA}}%
\index{NFA@\texttt{NFA}!nfa@\texttt{nfa}}%
(in the top-level environment) of nondeterministic finite automata,
along with various functions for processing NFAs.  Values of type
\texttt{nfa} are implemented as values of type \texttt{fa}, and the
module NFA provides the following injection and projection functions:
\begin{verbatim}
val injToFA     : nfa -> fa
val injToEFA    : nfa -> efa
val projFromFA  : fa -> nfa
val projFromEFA : efa -> nfa
\end{verbatim}
\index{NFA@\texttt{NFA}!injToFA@\texttt{injToFA}}%
\index{iNFA@\texttt{iNFA}!injToEFA@\texttt{injToEFA}}%
\index{proNFA@\texttt{proNFA}!projFromFA@\texttt{projFromFA}}%
\index{projNFA@\texttt{projNFA}!projFromEFA@\texttt{projFromEFA}}%
The functions \texttt{injToFA}, \texttt{injToEFA}, \texttt{projFromFA} and
\texttt{projFromEFA} are available in the top-level environment as
\texttt{injNFAToFA}, \texttt{injNFAToEFA}, \texttt{projFAToNFA} and
\texttt{projEFAToNFA}, respectively.
\index{nondeterministic finite automaton!injNFAToEFA@\texttt{injNFAToEFA}}%
\index{nondeterministic finite automaton!projFAToNFA@\texttt{projFAToNFA}}%
\index{nondeterministic finite automaton!projEFAToNFA@\texttt{projEFAToNFA}}%

The module \texttt{NFA} also defines the functions:
\begin{verbatim}
val input   : string -> nfa
val fromEFA : efa -> nfa
\end{verbatim}
\index{NFA@\texttt{NFA}!input@\texttt{input}}%
\index{NFA@\texttt{NFA}!fromEFA@\texttt{fromEFA}}%

The function \texttt{input} is used to input an NFA, and
the function \texttt{fromEFA} corresponds to our
conversion function $\efaToNFA$, and is available in the top-level
environment with that name:
\begin{verbatim}
val efaToNFA : efa -> nfa
\end{verbatim}
\index{nondeterministic finite automaton!efaToNFA@\texttt{efaToNFA}}%

Most of the functions for processing FAs that were introduced
in previous sections are inherited by \texttt{NFA}:
\begin{verbatim}
val output                  : string * nfa -> unit 
val numStates               : nfa -> int
val numTransitions          : nfa -> int
val alphabet                : nfa -> sym set
val equal                   : nfa * nfa -> bool
val checkLP                 : nfa -> lp -> unit
val validLP                 : nfa -> lp -> bool
val isomorphism             : nfa * nfa * sym_rel -> bool
val findIsomorphism         : nfa * nfa -> sym_rel
val isomorphic              : nfa * nfa -> bool
val renameStates            : nfa * sym_rel -> nfa
val renameStatesCanonically : nfa -> nfa
val processStr              : nfa -> sym set * str -> sym set
val accepted                : nfa -> str -> bool
val findLP                  : nfa -> sym set * str * sym set -> lp
val findAcceptingLP         : nfa -> str -> lp
val simplified              : nfa -> bool
val simplify                : nfa -> nfa
\end{verbatim}
\index{NFA@\texttt{NFA}!output@\texttt{output}}%
\index{NFA@\texttt{NFA}!numStates@\texttt{numStates}}%
\index{NFA@\texttt{NFA}!numTransitions@\texttt{numTransitions}}%
\index{NFA@\texttt{NFA}!alphabet@\texttt{alphabet}}%
\index{NFA@\texttt{NFA}!equal@\texttt{equal}}%
\index{NFA@\texttt{NFA}!checkLP@\texttt{checkLP}}%
\index{NFA@\texttt{NFA}!validLP@\texttt{validLP}}%
\index{NFA@\texttt{NFA}!isomorphism@\texttt{isomorphism}}%
\index{NFA@\texttt{NFA}!findIsomorphism@\texttt{findIsomorphism}}%
\index{NFA@\texttt{NFA}!isomorphic@\texttt{isomorphic}}%
\index{NFA@\texttt{NFA}!renameStates@\texttt{renameStates}}%
\index{NFA@\texttt{NFA}!renameStatesCanonically@\texttt{renameStatesCanonically}}%
\index{NFA@\texttt{NFA}!processStr@\texttt{processStr}}%
\index{NFA@\texttt{NFA}!accepted@\texttt{accepted}}%
\index{NFA@\texttt{NFA}!findLP@\texttt{findLP}}%
\index{NFA@\texttt{NFA}!findAcceptingLP@\texttt{findAcceptingLP}}%
\index{NFA@\texttt{NFA}!simplified@\texttt{simplified}}%
\index{NFA@\texttt{NFA}!simplify@\texttt{simplify}}%
Finally, the functions for computing forwards and backwards
empty-closures are inherited by the EFA module
\begin{verbatim}
val emptyClose          : efa -> sym set -> sym set
val emptyCloseBackwards : efa -> sym set -> sym set
\end{verbatim}
\index{EFA@\texttt{EFA}!emptyClose@\texttt{emptyClose}}%
\index{EFA@\texttt{EFA}!emptyCloseBackwards@\texttt{emptyCloseBackwards}}%

Suppose that \texttt{efa} is the \texttt{efa}
\begin{center}
\input{chap-3.10-fig2.eepic}
\end{center}
Here are some example uses of a few of the above functions:
\begin{list}{}
{\setlength{\leftmargin}{\leftmargini}
\setlength{\rightmargin}{0cm}
\setlength{\itemindent}{0cm}
\setlength{\listparindent}{0cm}
\setlength{\itemsep}{0cm}
\setlength{\parsep}{0cm}
\setlength{\labelsep}{0cm}
\setlength{\labelwidth}{0cm}
\catcode`\#=12
\catcode`\$=12
\catcode`\%=12
\catcode`\^=12
\catcode`\_=12
\catcode`\.=12
\catcode`\?=12
\catcode`\!=12
\catcode`\&=12
\ttfamily}
\small
\item[]\textsl{-\ }projEFAToNFA\ efa;
\item[]\textsl{invalid\ label\ in\ transition:\ "%"}
\item[]
\item[]\textsl{uncaught\ exception\ Error}
\item[]\textsl{-\ }val\ nfa\ =\ efaToNFA\ efa;
\item[]\textsl{val\ nfa\ =\ -\ :\ nfa}
\item[]\textsl{-\ }NFA.output("",\ nfa);
\item[]\textsl{\symbol{'173}states\symbol{'175}\ A,\ B,\ C\ \symbol{'173}start\ state\symbol{'175}\ A\ \symbol{'173}accepting\ states\symbol{'175}\ A,\ B,\ C}
\item[]\textsl{\symbol{'173}transitions\symbol{'175}}
\item[]\textsl{A,\ 0\ ->\ A\ |\ B\ |\ C;\ A,\ 1\ ->\ B\ |\ C;\ A,\ 2\ ->\ C;\ B,\ 1\ ->\ B\ |\ C;}
\item[]\textsl{B,\ 2\ ->\ C;\ C,\ 2\ ->\ C}
\item[]\textsl{val\ it\ =\ ()\ :\ unit}
\item[]\textsl{-\ }LP.output("",\ EFA.findAcceptingLP\ efa\ (Str.input\ ""));
\item[]\textsl{@\ }012
\item[]\textsl{@\ }.
\item[]\textsl{A,\ 0\ =>\ A,\ %\ =>\ B,\ 1\ =>\ B,\ %\ =>\ C,\ 2\ =>\ C}
\item[]\textsl{val\ it\ =\ ()\ :\ unit}
\item[]\textsl{-\ }LP.output("",\ NFA.findAcceptingLP\ nfa\ (Str.input\ ""));
\item[]\textsl{@\ }012
\item[]\textsl{@\ }.
\item[]\textsl{A,\ 0\ =>\ A,\ 1\ =>\ B,\ 2\ =>\ C}
\item[]\textsl{val\ it\ =\ ()\ :\ unit}
\end{list}


\subsection{Notes}

Because we have defined the meaning of finite automata via labeled
paths instead of transition functions, our EFA to NFA conversion algorithm
is easy to understand and prove correct.

\index{finite automaton!nondeterministic|)}%
\index{nondeterministic finite automaton|)}%
\index{NFA|)}%

%%% Local Variables: 
%%% mode: latex
%%% TeX-master: "book"
%%% End: 
