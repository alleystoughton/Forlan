\chapter{Regular Languages}
\label{RegularLanguages}

In this chapter, we study our most restrictive set of languages, the
regular languages.  We begin by introducing regular expressions, and
saying that a language is regular iff it is generated by a regular
expression. We study regular expression equivalance, look at how
regular expressions can be synthesized and proved correct, and study
several algorithms for regular expression simplification.

We go on to study five kinds of finite automata, culminating in finite
automata whose transitions are labeled by regular expressions.  We
introduce methods for synthesizing and proving the correctness of
finite automata, and study numerous algorithms for processing and
converting between regular expressions and finite automata.  Because
of these conversions, the set of languages accepted by the finite
automata is exactly the regular languages.  The chapter concludes by
considering the application of regular expressions and finite automata
to searching in text files, lexical analysis, and the design of
finite state systems.

\section{Regular Expressions and Languages}
\label{RegularExpressionsAndLanguages}

In this section, we define several operations on languages, say what
regular expressions are, what they mean, and what regular languages
are, and begin to show how regular expressions can be processed by Forlan.

\subsection{Operations on Languages}

\index{union!language}%
\index{intersection!language}%
\index{set difference!language}%
The union, intersection and set-difference operations on sets are also
operations on languages, i.e., if $L_1,L_2\in\Lan$, then $L_1\cup
L_2$, $L_1\cap L_2$ and $L_1-L_2$ are all languages.
(Since $L_1,L_2\in\Lan$, we have that $L_1\sub\Sigma_1^*$
and $L_2\sub\Sigma_2^*$, for alphabets $\Sigma_1$ and $\Sigma_2$.
Let $\Sigma=\Sigma_1\cup\Sigma_2$, so that $\Sigma$ is an alphabet,
$L_1\sub\Sigma^*$ and $L_2\sub\Sigma^*$.
Thus $L_1\cup L_2$, $L_1\cap L_2$ and $L_1-L_2$ are all subsets
of $\Sigma^*$, and so are all languages.)

\index{language!concatenation}%
\index{concatenation!language}%
The first new operation on languages is language concatenation.
The \emph{concatenation} of languages $L_1$ and $L_2$ ($L_1\myconcat
L_2$)
\index{ at@$\myconcat$}%
\index{language! at@$\myconcat$}%
is the language
\begin{gather*}
\setof{x_1\myconcat x_2}{x_1\in L_1\eqtxt{and}x_2\in L_2} .
\end{gather*}
I.e., $L_1\myconcat L_2$ consists of all strings that can be formed by
concatenating an element of $L_1$ with an element of $L_2$.
For example,
\begin{align*}
\mathsf{\{ab,abc\}\myconcat\{cd,d\}} &=
\mathsf{\{(ab)(cd), (ab)(d), (abc)(cd), (abc)(d)\}} \\
&= \mathsf{\{abcd, abd, abccd\}}.
\end{align*}
Note that, if $L_1,L_2\sub\Sigma^*$, for an alphabet $\Sigma$,
then $L_1\myconcat L_2\sub\Sigma^*$.

Concatenation of languages is associative: for all $L_1,L_2,L_3\in\Lan$,
\index{associative!language concatenation}%
\index{language!concatenation!associative}%
\index{concatenation!language!associative}%
\begin{gather*}
(L_1\myconcat L_2)\myconcat L_3 = L_1\myconcat(L_2\myconcat L_3) .
\end{gather*}
And, $\{\%\}$ is the identity for concatenation:
\index{identity!language concatenation}%
\index{language!concatenation!identity}%
\index{concatenation!language!identity}%
for all $L\in\Lan$,
\begin{gather*}
{\{\%\}}\myconcat L=L\myconcat{\{\%\}}=L.
\end{gather*}
Furthermore, $\emptyset$ is the zero for concatenation:
\index{zero!language concatenation}%
\index{language!concatenation!zero}%
\index{concatenation!language!zero}%
for all $L\in\Lan$,
\begin{gather*}
\emptyset\myconcat L=L\myconcat\emptyset={\emptyset} .
\end{gather*}
We often abbreviate $L_1\myconcat L_2$ to $L_1L_2$.

Now that we know what language concatenation is, we can say what it
means to raise a language to a power.  We define the \emph{language}
\index{concatenation!language!power}%
\index{language!concatenation!power}%
\index{concatenation!language!exponentiation}%
\index{language!concatenation!exponentiation}%
$L^n$ \emph{formed by raising} a language $L$ \emph{to the
  power} $n\in\nats$
\index{ power@$\cdot^\cdot$}%
\index{language! power@$\cdot^\cdot$}%
by recursion on $n$:
\begin{align*}
L^0      &= {\{\%\}} , \eqtxt{for all}L\in\Lan ; \eqtxt{and} \\
L^{n + 1} &= LL^n ,\eqtxt{for all}L\in\Lan\eqtxt{and}n\in\nats .
\end{align*}
We assign this exponentiation operation higher precedence than
concatenation, so that $LL^n$ means $L(L^n)$ in the above definition.
Note that, if $L\sub\Sigma^*$, for an alphabet $\Sigma$, then
$L^n\sub\Sigma^*$, for all $n\in\nats$.

For example, we have that
\begin{align*}
\mathsf{\{a,b\}}^2 &= 
\mathsf{\{a,b\}}\mathsf{\{a,b\}}^1 =
\mathsf{\{a,b\}}\mathsf{\{a,b\}}\mathsf{\{a,b\}}^0 \\
&=\mathsf{\{a,b\}}\mathsf{\{a,b\}}\{\%\} =
\mathsf{\{a,b\}}\mathsf{\{a,b\}} \\
&=\mathsf{\{aa, ab, ba, bb\}}.
\end{align*}

\begin{proposition}
\label{LangExponProp1}
For all $L\in\Lan$ and $n,m\in\nats$, $L^{n+m}=L^nL^m$.
\end{proposition}

\begin{proof}
An easy mathematical induction on $n$.  The language $L$ and the natural
number $m$ can be fixed at the beginning of the proof.
\end{proof}

Thus, if $L\in\Lan$ and $n\in\nats$, then
\begin{alignat*}{2}
L^{n+1} &= LL^n && \by{definition}, \\
\intertext{and}
L^{n+1} &= L^nL^1 = L^nL && \by{Proposition~\ref{LangExponProp1}} .
\end{alignat*}

Another useful fact about language exponentiation is:

\begin{proposition}
\label{LangExponProp2}
For all $w\in\Str$ and $n\in\nats$, $\{w\}^n = \{w^n\}$.
\end{proposition}

\begin{proof}
By mathematical induction on $n$.
\end{proof}

For example, we have that $\{\mathsf{01}\}^4=\{(\mathsf{01})^4\}=
\{\mathsf{01010101}\}$.

Now we consider a language operation that is named after
Stephen Cole Kleene, one of the founders of formal language theory.
\index{closure!language}%
\index{Kleene closure|see{closure}}%
The \emph{Kleene closure} (or just \emph{closure}) of a language
$L$ ($L^*$) is the language
\begin{gather*}
\bigcup\setof{L^n}{n\in\nats}.
\end{gather*}
Thus, for all $w$,
\begin{align*}
w\in L^*&\myiff w\in A,\eqtxt{for some}A\in\setof{L^n}{n\in\nats} \\
&\myiff w\in L^n\eqtxt{for some}n\in\nats .
\end{align*}
Or, in other words:
\begin{itemize}
\item $L^*=L^0\cup L^1\cup L^2\cup{\cdots}$; and

\item $L^*$ consists of all strings that can be formed by concatenating
together some number (maybe none) of elements of $L$ (the same element of
$L$ can be used as many times as is desired).
\end{itemize}
For example,
\begin{align*}
\mathsf{\{a,ba\}^*} &=
\mathsf{\{a,ba\}^0\cup\{a,ba\}^1\cup\{a,ba\}^2\cup\cdots} \\
&= \mathsf{\{\%\}\cup
   \{a,ba\}\cup
   \{aa,aba,baa,baba\}\cup\cdots}
\end{align*}

If $L$ is a language, then $L\sub\Sigma^*$ for some alphabet $\Sigma$,
and thus $L^*$ is also a subset of $\Sigma^*$---showing that $L^*$
is a language, not just a set of strings.

Suppose $w\in\Str$.
By Proposition~\ref{LangExponProp2}, we have
that, for all $x$,
\begin{align*}
x\in\{w\}^* &\myiff x\in\{w\}^n,\eqtxt{for some}n\in\nats , \\
&\myiff x\in\{w^n\},\eqtxt{for some}n\in\nats , \\
&\myiff x=w^n, \eqtxt{for some} n\in\nats .
\end{align*}

If we write $\mathsf{\{0,1\}}^*$, then this could mean:
\begin{itemize}
\item
all strings over the alphabet $\mathsf{\{0,1\}}$
(Section~\ref{SymbolsStringsAlphabetsAndFormalLanguages}); or

\item
the closure of the language $\mathsf{\{0,1\}}$.
\end{itemize}
Fortunately, these languages are equal (both are all strings of
$\zerosf$'s and $\onesf$'s), and this kind of ambiguity is harmless.

\index{language!operation precedence}%
We assign our operations on languages relative precedences as follows:
\begin{description}
\item[\quad Highest:] closure ($(\cdot)^*$) and raising to a power
($(\cdot)^n$);

\item[\quad Intermediate:] concatenation ($\myconcat$, or just
  juxtapositioning); and

\item[\quad Lowest:] union ($\cup$), intersection ($\cap$) and difference ($-$).
\end{description}
For example, if $n\in\nats$ and $A,B,C\in\Lan$, then $A^*BC^n\cup B$
abbreviates $((A^*)B(C^n))\cup B$.  The language $((A\cup B)C)^*$
can't be abbreviated, since removing either pair of parentheses will
change its meaning.  If we removed the outer pair, then we would have
$(A\cup B)(C^*)$, and removing the inner pair would yield $(A\cup
(BC))^*$.

\index{alphabet@$\alphabet$}%
\index{language!alphabet@$\alphabet$}%
\index{alphabet!language}%
\index{language!alphabet}%
Suppose $L$, $L_1$ and $L_2$ are languages, and $n\in\nats$.  It
is easy to see that $\alphabet(L_1\cup L_2)=\alphabet(L_1)\cup\alphabet(L_2)$.
And, if $L_1$ and $L_2$ are both nonempty, then
$\alphabet(L_1L_2)=\alphabet(L_1)\cup\alphabet(L_2)$, and otherwise,
$\alphabet(L_1L_2)=\emptyset$.
Furthermore, if $n\geq 1$, then $\alphabet(L^n)=\alphabet(L)$; otherwise,
$\alphabet(L^n)=\emptyset$.
Finally, we have that $\alphabet(L^*)=\alphabet(L)$.

In Section~\ref{IntroductionToForlan}, we introduced the
Forlan module \texttt{StrSet},
which defines various functions for processing finite sets of strings,
i.e., finite languages.  This module also defines the
functions
\index{StrSet@\texttt{StrSet}!concat@\texttt{concat}}%
\index{StrSet@\texttt{StrSet}!power@\texttt{power}}%
\begin{verbatim}
val concat : str set * str set -> str set
val power  : str set * int -> str set
\end{verbatim}
which implement our concatenation and exponentiation operations
on finite languages.  Here are some examples of how these functions
can be used:
\begin{list}{}
{\setlength{\leftmargin}{\leftmargini}
\setlength{\rightmargin}{0cm}
\setlength{\itemindent}{0cm}
\setlength{\listparindent}{0cm}
\setlength{\itemsep}{0cm}
\setlength{\parsep}{0cm}
\setlength{\labelsep}{0cm}
\setlength{\labelwidth}{0cm}
\catcode`\#=12
\catcode`\$=12
\catcode`\%=12
\catcode`\^=12
\catcode`\_=12
\catcode`\.=12
\catcode`\?=12
\catcode`\!=12
\catcode`\&=12
\ttfamily}
\small
\item[]\textsl{-\ }val\ xs\ =\ StrSet.fromString\ "ab,\ cd";
\item[]\textsl{val\ xs\ =\ -\ :\ str\ set}
\item[]\textsl{-\ }val\ ys\ =\ StrSet.fromString\ "uv,\ wx";
\item[]\textsl{val\ ys\ =\ -\ :\ str\ set}
\item[]\textsl{-\ }StrSet.output("",\ StrSet.concat(xs,\ ys));
\item[]\textsl{abuv,\ abwx,\ cduv,\ cdwx}
\item[]\textsl{val\ it\ =\ ()\ :\ unit}
\item[]\textsl{-\ }StrSet.output("",\ StrSet.power(xs,\ 0));
\item[]\textsl{%}
\item[]\textsl{val\ it\ =\ ()\ :\ unit}
\item[]\textsl{-\ }StrSet.output("",\ StrSet.power(xs,\ 1));
\item[]\textsl{ab,\ cd}
\item[]\textsl{val\ it\ =\ ()\ :\ unit}
\item[]\textsl{-\ }StrSet.output("",\ StrSet.power(xs,\ 2));
\item[]\textsl{abab,\ abcd,\ cdab,\ cdcd}
\item[]\textsl{val\ it\ =\ ()\ :\ unit}
\item[]\textsl{-\ }StrSet.output("",\ StrSet.power(xs,\ 3));
\item[]\textsl{ababab,\ ababcd,\ abcdab,\ abcdcd,\ cdabab,\ cdabcd,\ cdcdab,\ cdcdcd}
\item[]\textsl{val\ it\ =\ ()\ :\ unit}
\end{list}


\subsection{Regular Expressions}

\index{regular expression|(}%
\index{regular expression!label}%
\index{RegLab@$\RegLab$}%
Next, we define the set of all regular expressions.  Let the set
$\RegLab$ of \emph{regular expression labels} be
\begin{gather*}
\Sym\cup\{\%,\$,{*},@,{+}\} .
\end{gather*}
\index{Reg@$\Reg$}%
Let the set $\Reg$ of
\emph{regular expressions} be the least subset of
$\Tree_\RegLab$ such that:
\index{Tree sub X@$\Tree_X$}%
\index{tree}%
\index{tree!Tree sub X@$\Tree_X$}%
\begin{description}
\item[\quad(empty string)] $\%\in\Reg$;

\item[\quad(empty set)] $\$\in\Reg$;

\item[\quad(symbol)] for all $a\in\Sym$, $a\in\Reg$;

\item[\quad(closure)] for all $\alpha\in\Reg$, ${*}(\alpha)\in\Reg$;
\index{closure!regular expression}%
\index{regular expression!closure}%

\item[\quad(concatenation)] for all $\alpha,\beta\in\Reg$,
$@(\alpha,\beta)\in\Reg$; and
\index{concatenation!regular expression}%
\index{regular expression!concatenation}%

\item[\quad(union)] for all $\alpha,\beta\in\Reg$, ${+}(\alpha,\beta)\in\Reg$.
\index{union!regular expression}%
\index{regular expression!union}%
\end{description}
This is yet another example of an inductive definition.  The elements
of $\Reg$ are precisely those $\RegLab$-trees
(trees (See Section~\ref{TreesAndInductiveDefinitions})
whose labels come from $\RegLab$) that can be built using these six
rules.

Whenever possible, we will use the mathematical variables
$\alpha$, $\beta$ and $\gamma$
\index{alpha, beta, gamma@$\alpha$, $\beta$, $\gamma$}%
\index{regular expression!alpha, beta, gamma@$\alpha$, $\beta$, $\gamma$}%
to name regular expressions.  Since
regular expressions are $\RegLab$-trees, we may talk of their sizes
and heights.

For example,
\begin{gather*}
\mathsf{{+}(@({*}(0),@(1,{*}(0))),\%)} ,
\end{gather*}
i.e.,
\begin{center}
\input{chap-3.1-fig1.eepic}
\end{center}
is a regular expression.  On the other hand, the $\RegLab$-tree
$*(*,*)$ is \emph{not} a regular expression, since it can't be
built using our six rules.

We order the elements of $\RegLab$ as follows:
\begin{gather*}
\% < \$ < \eqtxtn{symbols in order} < {*} < @ < {+} .
\end{gather*}
\index{regular expression!order}%
It is important that ${+}$ be the greatest element of $\RegLab$;
if this were not so, then the definition of weakly simplified regular
expressions (see Section~\ref{SimplificationOfRegularExpressions})
would have to be altered.

We order regular expressions first by their root labels, and then,
recursively, by their children, working from left to right.
For example, we have that
\begin{gather*}
\% < {*}(\%)
   < {*}(@(\$,{*}(\$)))
   < {*}(@(\asf,\%))
   < @(\%,\$) .
\end{gather*}

Because $\Reg$ is defined inductively, it gives rise to an induction
principle.

\begin{theorem}[Principle of Induction on Regular Expressions]
Suppose $P(\alpha)$ is a property of a regular expression $\alpha$.
If
\begin{itemize}
\item $P(\%)$,

\item $P(\$)$,

\item for all $a\in\Sym$, $P(a)$,

\item for all $\alpha\in\Reg$, if $P(\alpha)$, then
$P({*}(\alpha))$,

\item for all $\alpha,\beta\in\Reg$, if $P(\alpha)$ and
$P(\beta)$, then $P(@(\alpha,\beta))$, and

\item for all $\alpha,\beta\in\Reg$, if $P(\alpha)$ and
$P(\beta)$, then $P({+}(\alpha,\beta))$,
\end{itemize}
then
\begin{gather*}
\eqtxtr{for all}\alpha\in\Reg,\,P(\alpha) .
\end{gather*}
\end{theorem}

To increase readability, we use infix and postfix notation, abbreviating:
\begin{itemize}
\item ${*}(\alpha)$ to $\alpha^*$ or $\alpha{*}$;

\item $@(\alpha,\beta)$ to $\alpha\myconcat \beta$; and

\item $+(\alpha,\beta)$ to $\alpha+\beta$.
\end{itemize}
\index{regular expression!abbreviated notation}%
\index{regular expression!operator precedence}%
\index{regular expression!operator associativity}%
We assign the operators $(\cdot)^*$, $\myconcat$ and $+$ the following
precedences and associativities:
\begin{description}
\item[\quad Highest:] $(\cdot)^*$;

\item[\quad Intermediate:] $\myconcat$ (right associative); and

\item[\quad Lowest:] $+$ (right associative).
\end{description}
We parenthesize regular expressions when we need to override the
default precedences and associativities, and for reasons of clarity.
Furthermore, we often abbreviate $\alpha\myconcat\beta$ to $\alpha\beta$.

For example, we can abbreviate the regular expression
\begin{displaymath}
\mathsf{{+}(@({*}(0),@(1,{*}(0))),\%)}
\end{displaymath}
to $\mathsf{0^*\myconcat 1\myconcat 0^*+\%}$ or $\mathsf{0^*10^*+\%}$.  On
the other hand, the regular expression $\mathsf{((0+1)2)^*}$ can't be
further abbreviated, since removing either pair of parentheses would
result in a different regular expression.  Removing the outer pair
would result in $\mathsf{(0+1)(2^*)}=\mathsf{(0+1)2^*}$, and removing
the inner pair would yield $\mathsf{(0+(12))^*}=\mathsf{(0+12)^*}$.

Now we can say what regular expressions mean, using some of our
language operations.  The \emph{language generated by} a regular
expression $\alpha$ ($L(\alpha)$) is defined by recursion:
\index{regular expression!meaning}%
\index{regular expression!language generated by}%
\index{L(.)@$L(\cdot)$}%
\index{regular expression!L@$L(\cdot)$}%}%
\begin{align*}
L(\%) &= \{\%\} ; \\
L(\$) &= \emptyset ; \\
L(a) &= \{[a]\} = \{a\}, \eqtxt{for all}a\in\Sym ; \\
L({*}(\alpha)) &= L(\alpha)^* , \eqtxt{for all}\alpha\in\Reg ; \\
L(@(\alpha,\beta)) &= L(\alpha)\myconcat L(\beta) ,
\eqtxt{for all}\alpha,\beta\in\Reg ; \eqtxt{and} \\
L({+}(\alpha,\beta)) &= L(\alpha)\cup L(\beta) ,
\eqtxt{for all}\alpha,\beta\in\Reg .
\end{align*}
This is a good definition since, if $L$ is a language, then so is
$L^*$, and, if $L_1$ and $L_2$ are languages, then so are $L_1L_2$ and
$L_1\cup L_2$.  We say that $w$ \emph{is generated by} $\alpha$ iff
$w\in L(\alpha)$.

For example,
\begin{align*}
L(\mathsf{0^*10^*+\%}) &=
L(\mathsf{{+}(@({*}(0),@(1,{*}(0))),\%)}) \\
&= L(\mathsf{@({*}(0),@(1,{*}(0)))})\cup L(\%) \\
&= L(\mathsf{{*}(0)})L(\mathsf{@(1,{*}(0))})\cup \{\%\} \\
&= L(\mathsf{0})^* L(\mathsf{1}) L(\mathsf{{*}(0)}) \cup \{\%\} \\
&= \mathsf{\{0\}}^* \mathsf{\{1\}} L(\mathsf{0})^*
  \cup \{\%\} \\
&= \mathsf{\{0\}}^* \mathsf{\{1\}} \mathsf{\{0\}}^* 
  \cup \{\%\} \\
&= \setof{{\mathsf 0}^n{\mathsf 1}{\mathsf 0}^m}{n,m\in\nats} \cup
\{\%\} .
\end{align*}
E.g., $\mathsf{0001000}$, $\mathsf{10}$,
$\mathsf{001}$ and $\%$ are generated by $\mathsf{0^*10^*+\%}$.

We define functions $\symToReg\in\Sym\fun\Reg$ and
$\strToReg\in\Str\fun\Reg$, as follows.  Given a symbol $a\in\Sym$,
$\symToReg\,a = a$. And, given a string $x$, $\strToReg\,x$ is the
\emph{canonical regular expression for} $x$: $\%$, if $x = \%$, and
${@}(a_1, {@}(a_2, \ldots a_n \ldots)) = a_1a_2\ldots a_n$, if $x =
a_1a_2\ldots a_n$, for symbols $a_1,a_2,\ldots,a_n$ and $n\geq 1$.  It
is easy to see that, for all $a\in\Sym$, $L(\symToReg\,a)=\{a\}$, and,
for all $x\in\Str$, $L(\strToReg\,x)=\{x\}$.

We define the \emph{regular expression} $\alpha^n$ \emph{formed by
raising} a regular expression $\alpha$ \emph{to the power} $n\in\nats$ by
recursion on $n$:
\index{regular expression!power}%
\index{regular expression!exponentiation}%
\begin{align*}
\alpha^0       &= \%, \eqtxt{for all}\alpha\in\Reg; \\
\alpha^1       &= \alpha, \eqtxt{for all}\alpha\in\Reg; \eqtxt{and} \\
\alpha^{n + 1} &= \alpha\alpha^n,\eqtxt{for all}\alpha\in\Reg\eqtxt{and}
n\in\nats-\{0\} .
\end{align*}
We assign this operation the same precedence as closure, so that
$\alpha\alpha^n$ means $\alpha(\alpha^n)$ in the above definition.
Note that, in contrast to the definitions of $x^n$ and $L^n$,
we have split the case $n+1$ into two subcases, depending upon
whether $n=0$ or $n\geq 1$. Thus
$\alpha^1$ is $\alpha$, not $\alpha\%$.
For example, $\mathsf{(0+1)^3}=\mathsf{(0+1)(0+1)(0+1)}$.

\begin{proposition}
\label{RegExponProp}
For all $\alpha\in\Reg$ and $n\in\nats$, $L(\alpha^n)=L(\alpha)^n$.
\end{proposition}

\begin{proof}
By mathematical induction on $n$, case-splitting in the inductive
step. $\alpha$ may be fixed at the beginning of the proof.
\end{proof}

An example consequence of the proposition is that
$L(\mathsf{(0+1)^3})=L(\mathsf{0+1})^3=\{\mathsf{0,1}\}^3$, the
set of all strings of $\zerosf$'s and $\onesf$'s of length $3$.

We define $\alphabet\in\Reg\fun\Alp$ by recursion:
\index{regular expression!alphabet}%
\index{alphabet@$\alphabet$}%
\begin{align*}
\alphabet\,\% &= \emptyset; \\
\alphabet\,\$ &= \emptyset; \\
\alphabet\,a &= \{a\}, \eqtxt{for all}a\in\Sym; \\
\alphabet({*}(\alpha)) &= \alphabet\,\alpha, \eqtxt{for all}\alpha\in\Reg;\\
\alphabet(@(\alpha,\beta)) &= \alphabet\,\alpha\cup\alphabet\,\beta ,
\eqtxt{for all}\alpha,\beta\in\Reg; \eqtxt{and} \\
\alphabet({+}(\alpha,\beta)) &= \alphabet\,\alpha\cup\alphabet\,\beta,
\eqtxt{for all}\alpha,\beta\in\Reg .
\end{align*}
This is a good definition, since the union of two alphabets is
an alphabet. For example,
$\alphabet(\mathsf{0^*10^*+\%}) = \mathsf{\{0,1\}}$.
We say that $\alphabet\,\alpha$ is \emph{the alphabet of} a regular
expression $\alpha$.

Because the alphabet of regular expressions is defined by structural
recursion, we can prove the following proposition by induction
on regular expressions:
\begin{proposition}
\label{RegAlphabetSubtreeSubstitute}
\begin{enumerate}[\quad(1)]
\item Suppose $\alpha,\beta,\beta'\in\Reg$,
  $\alphabet\,\beta' = \alphabet\,\beta$, $\pat\in\Path$ is valid for
  $\alpha$, and $\beta$ is the subtree of $\alpha$ at position $\pat$.
  Let $\alpha'$ be the result of replacing the subtree at position
  $\pat$ in $\alpha$ by $\beta'$. Then
  $\alphabet\,\alpha' = \alphabet\,\alpha$.

\item Suppose $\alpha,\beta,\beta'\in\Reg$,
  $\alphabet\,\beta' \sub \alphabet\,\beta$, $\pat\in\Path$ is valid for
  $\alpha$, and $\beta$ is the subtree of $\alpha$ at position $\pat$.
  Let $\alpha'$ be the result of replacing the subtree at position
  $\pat$ in $\alpha$ by $\beta'$. Then
  $\alphabet\,\alpha' \sub \alphabet\,\alpha$.
\end{enumerate}
\end{proposition}

\begin{exercise}
\label{RegExactExpOverAlphabetFinite}
Suppose $\Sigma$ is an alphabet.
For $n\in\nats$, define
$X_n = \setof{\alpha\in\Reg}{\alphabet\,\alpha\sub\Sigma
  \eqtxt{and} \size\,\alpha = n}$. Prove that, for all
$n\in\nats$, $X_n$ is finite. Hint: use strong induction
on $n$.
\end{exercise}

\begin{proposition}
\label{AlphabetRegMeaning}
For all $\alpha\in\Reg$, $\alphabet(L(\alpha))\sub\alphabet\,\alpha$.
\end{proposition}

In other words, the proposition says that every symbol of every string in
$L(\alpha)$  comes from $\alphabet\,\alpha$.

\begin{proof}
An easy induction on regular expressions.
\end{proof}

For example, since $L(\onesf\$) = \{\onesf\}\emptyset = \emptyset$,
we have that
\begin{align*}
\alphabet(L(\mathsf{0^*+1\$})) &=
\alphabet(\{\zerosf\}^*) \\
&= \{\zerosf\} \\
&\sub \{\mathsf{0,1}\} \\
&= \alphabet(\mathsf{0^*+1\$}) .
\end{align*}

Next, we define some useful auxiliary functions on regular expressions.
The \emph{generalized concatenation} function
$\genConcat\in\List\,\Reg\fun\Reg$ is defined by right recursion:
\begin{align*}
\genConcat\,[\,] &= \% , \\
\genConcat\,[\alpha] &= \alpha , \eqtxt{and} \\
\genConcat([\alpha] \myconcat \bar{\alpha}) &=
{@}(\alpha, \genConcat\,\bar{\alpha}) , \eqtxt{if} \bar{\alpha}\neq [\,] .
\end{align*}
And the \emph{generalized union} function
$\genUnion\in\List\,\Reg\fun\Reg$ is defined by right recursion:
\begin{align*}
\genUnion\,[\,] &= \% , \\
\genUnion\,[\alpha] &= \alpha , \eqtxt{and} \\
\genUnion([\alpha] \myconcat \bar{\alpha}) &=
{+}(\alpha, \genUnion\,\bar{\alpha}) , \eqtxt{if} \bar{\alpha}\neq [\,] .
\end{align*}
E.g., $\genConcat[\mathsf{1, 0, 12, 3+4}] = \mathsf{10(12)(3+4)}$ and
$\genUnion[\mathsf{1, 0, 12, 3+4}] = \mathsf{1 + 0 + (12) + 3 + 4}$.

$\rightConcat\in\Reg\times\Reg\fun\Reg$ is defined by
structural recursion on its first argument:
\begin{align*}
\rightConcat({@}(\alpha_1, \alpha_2), \beta) &=
{@}(\alpha_1, \rightConcat(\alpha_2, \beta)) , \eqtxt{and}\\
\rightConcat(\alpha, \beta) &= {@}(\alpha, \beta),
\eqtxt{if} \alpha \eqtxtl{is not a concatenation}.
\end{align*}
And $\rightUnion\in\Reg\times\Reg\fun\Reg$ is defined by
structural recursion on its first argument:
\begin{align*}
\rightUnion({+}(\alpha_1, \alpha_2), \beta) &=
{+}(\alpha_1, \rightUnion(\alpha_2, \beta)) , \eqtxt{and} \\
\rightUnion(\alpha, \beta) &= {+}(\alpha, \beta),
\eqtxt{if} \alpha \eqtxtl{is not a union}.
\end{align*}
E.g., $\rightConcat(\mathsf{012}, \mathsf{345}) =
\mathsf{012345}$ and $\rightUnion(\mathsf{0 + 1 + 2}, \mathsf{1 + 2 + 3}) =
\mathsf{0 + 1 + 2 + 1 + 2 + 3}$.

$\concatsToList\in\Reg\fun\List\,\Reg$ is defined by structural
recursion:
\begin{align*}
\concatsToList({@}(\alpha, \beta)) &=
[\alpha] \myconcat \concatsToList\,\beta, \eqtxt{and} \\
\concatsToList\,\alpha &= [\alpha],
\eqtxt{if} \alpha \eqtxtl{is not a concatenation}.
\end{align*}
And $\unionsToList\in\Reg\fun\List\,\Reg$ is defined by structural
recursion:
\begin{align*}
\unionsToList({+}(\alpha, \beta)) &=
[\alpha] \myconcat \unionsToList\,\beta, \eqtxt{and} \\
\unionsToList\,\alpha &= [\alpha],
\eqtxt{if} \alpha \eqtxtl{is not a union}.
\end{align*}
E.g., $\concatsToList(\mathsf{(12)34}) = [\mathsf{12, 3, 4}]$
and $\unionsToList(\mathsf{(0+1)+2+3}) = [\mathsf{0 + 1, 2, 3}]$.

\begin{lemma}
\label{ConcatsToListRightConcat}
For all $\alpha,\beta\in\Reg$,
$\concatsToList(\rightConcat(\alpha,\beta)) = \concatsToList\,\alpha
\myconcat \concatsToList\,\beta$.
\end{lemma}

\begin{lemma}
\label{UnionsToListRightUnion}
For all $\alpha,\beta\in\Reg$,
$\unionsToList(\rightUnion(\alpha,\beta)) = \unionsToList\,\alpha
  \myconcat \unionsToList\,\beta$.
\end{lemma}

Finally, $\sortUnions\in\Reg\fun\Reg$ is defined by:
\begin{displaymath}
\sortUnions\,\alpha = \genUnion\,\bar{\beta} ,
\end{displaymath}
where $\bar{\beta}$ is the result of sorting the elements of
$\unionsToList\,\alpha$ into strictly ascending order (without
duplicates), according to our total ordering on regular expressions.
E.g., $\sortUnions(\mathsf{1 + 0 + 23 + 1}) = \mathsf{0 + 1 + 23}$.

We define functions $\allSym\in\Alp\fun\Reg$ and $\allStr\in
\Alp\fun\Reg$ as follows.  Given an alphabet $\Sigma$, $\allSym\,\Sigma$
is the \emph{all symbols regular expression} for $\Sigma$:
$a_1+\cdots+a_n$, where $a_1,\ldots,a_n$
  are the elements of $\Sigma$, listed in order and without repetition
  (when $n=0$, we use $\$$, and when $n=1$, we use $a_1$).
And, given an alphabet $\Sigma$, $\allStr\,\Sigma$ is
the \emph{all strings regular expression} for $\Sigma$:
$(\allSym\,\Sigma)^*$.
For example, 
\begin{align*}
\allSym\,\{\mathsf{0,1,2}\} &= \mathsf{0 + 1 + 2} , \eqtxt{and} \\
\allStr\,\{\mathsf{0,1,2}\} &= (\mathsf{0 + 1 + 2})^* .
\end{align*}
Thus, for all $\Sigma\in\Alp$,
\begin{align*}
L(\allSym\,\Sigma) &= \setof{[a]}{a\in\Sigma} = \setof{a}{a\in\Sigma},
\eqtxt{and} \\
L(\allStr\,\Sigma) &= \Sigma^* .
\end{align*}

\index{language!regular|see{regular language}}%
\index{regular language}%
Now we are able to say what it means for a language to be regular:
a language $L$ is \emph{regular} iff $L=L(\alpha)$ for some
$\alpha\in\Reg$.  We define
\index{RegLan@$\RegLan$}%
\begin{align*}
\RegLan &= \setof{L(\alpha)}{\alpha\in\Reg} \\
&= \setof{L\in\Lan}{L\eqtxtl{is regular}} .
\end{align*}

Since every regular expression can be described, e.g., in fully
parenthesized form, by a finite sequence of ASCII characters, we can
enumerate the regular expressions, and consequently we have that
$\Reg$ is countably infinite.  Since $\{\mathsf{0}^0\}$,
$\{\mathsf{0}^1\}$, $\{\mathsf{0}^2\}$, \ldots, are all regular
languages, we have that $\RegLan$ is infinite.  Furthermore, we can
establish an injection $h$ from $\RegLan$ to $\Reg$: $h\,L$ is the
first (in our enumeration of regular expressions) $\alpha$ such that
$L(\alpha)=L$.  Because $\Reg$ is countably infinite, it follows that
there is an injection from $\Reg$ to $\nats$.  Composing these
injections, gives us an injection from $\RegLan$ to $\nats$.  And when
observing that $\RegLan$ is infinite, we implicity gave an injection
from $\nats$ to $\RegLan$. Thus, by the Schr\"oder-Bernstein Theorem,
we have that $\RegLan$ and $\nats$ have the same size, so that
$\RegLan$ is countably infinite.

Since $\RegLan$ is countably infinite but $\Lan$ is uncountable, it
follows that $\RegLan\subsetneq\Lan$, i.e., there are non-regular
languages.
In Section~\ref{ThePumpingLemmaForRegularLanguages}, we will see a concrete
example of a non-regular language.

\subsection{Processing Regular Expressions in Forlan}

Now, we turn to the Forlan implementation of regular expressions.  The
Forlan module \texttt{Reg}
\index{Reg@\texttt{Reg}}%
defines the abstract type \texttt{reg}
\index{reg@\texttt{reg}}%
\index{Reg@\texttt{Reg}!reg@\texttt{reg}}%
(in the top-level environment) of regular expressions, as well as various
functions and constants for processing regular expressions, including:
\index{Reg@\texttt{Reg}!input@\texttt{input}}%
\index{Reg@\texttt{Reg}!output@\texttt{output}}%
\index{Reg@\texttt{Reg}!size@\texttt{size}}%
\index{Reg@\texttt{Reg}!numLeaves@\texttt{numLeaves}}%
\index{Reg@\texttt{Reg}!height@\texttt{height}}%
\index{Reg@\texttt{Reg}!emptyStr@\texttt{emptyStr}}%
\index{Reg@\texttt{Reg}!emptySet@\texttt{emptySet}}%
\index{Reg@\texttt{Reg}!fromSym@\texttt{fromSym}}%
\index{Reg@\texttt{Reg}!closure@\texttt{closure}}%
\index{Reg@\texttt{Reg}!concat@\texttt{concat}}%
\index{Reg@\texttt{Reg}!union@\texttt{union}}%
\index{Reg@\texttt{Reg}!compare@\texttt{compare}}%
\index{Reg@\texttt{Reg}!equal@\texttt{equal}}%
\index{Reg@\texttt{Reg}!fromStr@\texttt{fromStr}}%
\index{Reg@\texttt{Reg}!power@\texttt{power}}%
\index{Reg@\texttt{Reg}!alphabet@\texttt{alphabet}}%
\index{Reg@\texttt{Reg}!genConcat@\texttt{genConcat}}%
\index{Reg@\texttt{Reg}!genUnion@\texttt{genUnion}}%
\index{Reg@\texttt{Reg}!rightConcat@\texttt{rightConcat}}%
\index{Reg@\texttt{Reg}!rightUnion@\texttt{rightUnion}}%
\index{Reg@\texttt{Reg}!concatsToList@\texttt{concatsToList}}%
\index{Reg@\texttt{Reg}!unionsToList@\texttt{unionsToList}}%
\index{Reg@\texttt{Reg}!sortUnions@\texttt{sortUnions}}%
\index{Reg@\texttt{Reg}!allSym@\texttt{allSym}}%
\index{Reg@\texttt{Reg}!allStr@\texttt{allStr}}%
\index{Reg@\texttt{Reg}!fromStrSet@\texttt{fromStrSet}}%
\begin{verbatim}
val input         : string -> reg
val output        : string * reg -> unit
val size          : reg -> int
val numLeaves     : reg -> int
val height        : reg -> int
val emptyStr      : reg
val emptySet      : reg
val fromSym       : sym -> reg
val closure       : reg -> reg
val concat        : reg * reg -> reg
val union         : reg * reg -> reg
val compare       : reg * reg -> order
val equal         : reg * reg -> bool
val fromStr       : str -> reg
val power         : reg * int -> reg
val alphabet      : reg -> sym set
val genConcat     : reg list -> reg
val genUnion      : reg list -> reg
val rightConcat   : reg * reg -> reg
val rightUnion    : reg * reg -> reg
val concatsToList : reg -> reg list
val unionsToList  : reg -> reg list
val sortUnions    : reg -> reg
val allSym        : sym set -> reg
val allStr        : sym set -> reg
val fromStrSet    : str set -> reg
\end{verbatim}

The Forlan syntax for regular expressions is the infix/postfix one
\index{Forlan!regular expression syntax}%
\index{regular expression!Forlan syntax}%
introduced in the preceding subsection, where $\alpha\myconcat\beta$
is always written as $\alpha\beta$, and we use parentheses
to override default precedences/associativities, or simply
for clarity.
For example, $\mathsf{0^*10^*+\%}$ and $\mathsf{(0^*(1(0^*)))+\%}$
are the same regular expression.  And, $\mathsf{((0^*)1)0^*+\%}$
is a different regular expression, but one with the same meaning.
Furthermore, $\mathsf{0^*1(0^*+\%)}$ is not only different from
the two preceding regular expressions, but it has a different
meaning (it fails to generate $\%$.)
When regular expressions are outputted, as few parentheses as
possible are used.

The functions \texttt{size}, \texttt{numLeaves}
and \texttt{height} return the size, number of leaves and
height, respectively, of a regular expression.
The values \texttt{emptyStr} and \texttt{emptySet} are
$\%$ and $\$$, respectively.  The function \texttt{fromSym}
takes in a symbol $a$ and returns the regular expression $a$.
It is available in the
top-level environment as \texttt{symToReg}.
\index{symToReg@\texttt{symToReg}}%
The function \texttt{closure} takes in a regular expression $\alpha$ and
returns ${*}(\alpha)$.  The function \texttt{concat} takes a pair $(\alpha,
\beta)$ of regular expressions and returns ${@}(\alpha,\beta)$.
The function \texttt{union} takes a pair $(\alpha,
\beta)$ of regular expressions and returns ${+}(\alpha,\beta)$.
The function \texttt{compare} implements our total ordering on
regular expressions, and \texttt{equal} tests whether two
regular expressions are equal.
The function \texttt{fromStr} implements the function $\strToReg$,
and is also available in the top-level environment as \texttt{strToReg}.
\index{strToReg@\texttt{strToReg}}%
The function \texttt{power} raises a regular
expression to a power, and the
function \texttt{alphabet} returns the alphabet of a regular
expression.
Finally, the functions \texttt{genConcat}, \texttt{genUnion},
\texttt{rightConcat}, \texttt{rightUnion}, \texttt{concatsToList},
\texttt{unionsToList}, \texttt{sortUnions}, \texttt{allSym} and
\texttt{allStr} implement the functions with the same names.
The function \texttt{fromStrSet} returns $\$$, if called with
the empty set. Otherwise, it returns $\mathtt{fromStr}\,x_1 + \cdots +
\mathtt{fromStr}\,x_n$, where $x_1,\ldots,x_n$ are the elements
of its argument, listed in strictly ascending order.

Here are some example uses of the functions of \texttt{Reg}:
\begin{list}{}
{\setlength{\leftmargin}{\leftmargini}
\setlength{\rightmargin}{0cm}
\setlength{\itemindent}{0cm}
\setlength{\listparindent}{0cm}
\setlength{\itemsep}{0cm}
\setlength{\parsep}{0cm}
\setlength{\labelsep}{0cm}
\setlength{\labelwidth}{0cm}
\catcode`\#=12
\catcode`\$=12
\catcode`\%=12
\catcode`\^=12
\catcode`\_=12
\catcode`\.=12
\catcode`\?=12
\catcode`\!=12
\catcode`\&=12
\ttfamily}
\small
\item[]\textsl{-\ }val\ reg\ =\ Reg.input\ "";
\item[]\textsl{@\ }0\symbol{'052}10\symbol{'052}\ +\ %
\item[]\textsl{@\ }.
\item[]\textsl{val\ reg\ =\ -\ :\ reg}
\item[]\textsl{-\ }Reg.size\ reg;
\item[]\textsl{val\ it\ =\ 9\ :\ int}
\item[]\textsl{-\ }Reg.numLeaves\ reg;
\item[]\textsl{val\ it\ =\ 4\ :\ int}
\item[]\textsl{-\ }val\ reg'\ =\ Reg.fromStr(Str.power(Str.input\ "",\ 3));
\item[]\textsl{@\ }01
\item[]\textsl{@\ }.
\item[]\textsl{val\ reg'\ =\ -\ :\ reg}
\item[]\textsl{-\ }Reg.output("",\ reg');
\item[]\textsl{010101}
\item[]\textsl{val\ it\ =\ ()\ :\ unit}
\item[]\textsl{-\ }Reg.size\ reg';
\item[]\textsl{val\ it\ =\ 11\ :\ int}
\item[]\textsl{-\ }Reg.numLeaves\ reg';
\item[]\textsl{val\ it\ =\ 6\ :\ int}
\item[]\textsl{-\ }Reg.compare(reg,\ reg');
\item[]\textsl{val\ it\ =\ GREATER\ :\ order}
\item[]\textsl{-\ }val\ reg''\ =\ Reg.concat(Reg.closure\ reg,\ reg');
\item[]\textsl{val\ reg''\ =\ -\ :\ reg}
\item[]\textsl{-\ }Reg.output("",\ reg'');
\item[]\textsl{(0\symbol{'052}10\symbol{'052}\ +\ %)\symbol{'052}010101}
\item[]\textsl{val\ it\ =\ ()\ :\ unit}
\item[]\textsl{-\ }SymSet.output("",\ Reg.alphabet\ reg'');
\item[]\textsl{0,\ 1}
\item[]\textsl{val\ it\ =\ ()\ :\ unit}
\item[]\textsl{-\ }val\ reg'''\ =\ Reg.power(reg,\ 3);
\item[]\textsl{val\ reg'''\ =\ -\ :\ reg}
\item[]\textsl{-\ }Reg.output("",\ reg''');
\item[]\textsl{(0\symbol{'052}10\symbol{'052}\ +\ %)(0\symbol{'052}10\symbol{'052}\ +\ %)(0\symbol{'052}10\symbol{'052}\ +\ %)}
\item[]\textsl{val\ it\ =\ ()\ :\ unit}
\item[]\textsl{-\ }Reg.size\ reg''';
\item[]\textsl{val\ it\ =\ 29\ :\ int}
\item[]\textsl{-\ }Reg.numLeaves\ reg''';
\item[]\textsl{val\ it\ =\ 12\ :\ int}
\item[]\textsl{-\ }Reg.output("",\ Reg.fromString\ "(0\symbol{'052}(1(0\symbol{'052})))\ +\ %");
\item[]\textsl{0\symbol{'052}10\symbol{'052}\ +\ %}
\item[]\textsl{val\ it\ =\ ()\ :\ unit}
\item[]\textsl{-\ }Reg.output("",\ Reg.fromString\ "(0\symbol{'052}1)0\symbol{'052}\ +\ %");
\item[]\textsl{(0\symbol{'052}1)0\symbol{'052}\ +\ %}
\item[]\textsl{val\ it\ =\ ()\ :\ unit}
\item[]\textsl{-\ }Reg.output("",\ Reg.fromString\ "0\symbol{'052}1(0\symbol{'052}\ +\ %)");
\item[]\textsl{0\symbol{'052}1(0\symbol{'052}\ +\ %)}
\item[]\textsl{val\ it\ =\ ()\ :\ unit}
\item[]\textsl{-\ }Reg.equal
\item[]\textsl{=\ }(Reg.fromString\ "0\symbol{'052}10\symbol{'052}\ +\ %",
\item[]\textsl{=\ }\ Reg.fromString\ "0\symbol{'052}1(0\symbol{'052}\ +\ %)");
\item[]\textsl{val\ it\ =\ false\ :\ bool}
\end{list}

We can use the functions \texttt{genConcat}, \texttt{genUnion},
\texttt{rightConcat}, \texttt{rightUnion}, \texttt{concatsToList},
\texttt{unionsToList} and \texttt{sortUnions} as follows:
\begin{list}{}
{\setlength{\leftmargin}{\leftmargini}
\setlength{\rightmargin}{0cm}
\setlength{\itemindent}{0cm}
\setlength{\listparindent}{0cm}
\setlength{\itemsep}{0cm}
\setlength{\parsep}{0cm}
\setlength{\labelsep}{0cm}
\setlength{\labelwidth}{0cm}
\catcode`\#=12
\catcode`\$=12
\catcode`\%=12
\catcode`\^=12
\catcode`\_=12
\catcode`\.=12
\catcode`\?=12
\catcode`\!=12
\catcode`\&=12
\ttfamily}
\small
\item[]\textsl{-\ }Reg.output("",\ Reg.genConcat\ nil);
\item[]\textsl{%}
\item[]\textsl{val\ it\ =\ ()\ :\ unit}
\item[]\textsl{-\ }Reg.output("",\ Reg.genUnion\ nil);
\item[]\textsl{$}
\item[]\textsl{val\ it\ =\ ()\ :\ unit}
\item[]\textsl{-\ }val\ regs\ =
\item[]\textsl{=\ }\ \ \ \ \ \ \symbol{'133}Reg.fromString\ "01",\ Reg.fromString\ "01\ +\ 12",
\item[]\textsl{=\ }\ \ \ \ \ \ \ Reg.fromString\ "(1\ +\ 2)\symbol{'052}",\ Reg.fromString\ "3\ +\ 4"\symbol{'135};
\item[]\textsl{val\ regs\ =\ \symbol{'133}-,-,-,-\symbol{'135}\ :\ reg\ list}
\item[]\textsl{-\ }Reg.output("",\ Reg.genConcat\ regs);
\item[]\textsl{(01)(01\ +\ 12)(1\ +\ 2)\symbol{'052}(3\ +\ 4)}
\item[]\textsl{val\ it\ =\ ()\ :\ unit}
\item[]\textsl{-\ }Reg.output("",\ Reg.genUnion\ regs);
\item[]\textsl{01\ +\ (01\ +\ 12)\ +\ (1\ +\ 2)\symbol{'052}\ +\ 3\ +\ 4}
\item[]\textsl{val\ it\ =\ ()\ :\ unit}
\item[]\textsl{-\ }Reg.output
\item[]\textsl{=\ }("",
\item[]\textsl{=\ }\ Reg.rightConcat
\item[]\textsl{=\ }\ (Reg.fromString\ "0123",\ Reg.fromString\ "4567"));
\item[]\textsl{01234567}
\item[]\textsl{val\ it\ =\ ()\ :\ unit}
\item[]\textsl{-\ }Reg.output
\item[]\textsl{=\ }("",
\item[]\textsl{=\ }\ Reg.rightUnion
\item[]\textsl{=\ }\ (Reg.fromString\ "0\ +\ 1\ +\ 2",\ Reg.fromString\ "1\ +\ 2\ +\ 3"));
\item[]\textsl{0\ +\ 1\ +\ 2\ +\ 1\ +\ 2\ +\ 3}
\item[]\textsl{val\ it\ =\ ()\ :\ unit}
\item[]\textsl{-\ }map
\item[]\textsl{=\ }Reg.toString
\item[]\textsl{=\ }(Reg.concatsToList(Reg.fromString\ "0(12)34"));
\item[]\textsl{val\ it\ =\ \symbol{'133}"0","12","3","4"\symbol{'135}\ :\ string\ list}
\item[]\textsl{-\ }map
\item[]\textsl{=\ }Reg.toString
\item[]\textsl{=\ }(Reg.unionsToList(Reg.fromString\ "0\ +\ (1\ +\ 2)\ +\ 3\ +\ 0"));
\item[]\textsl{val\ it\ =\ \symbol{'133}"0","1\ +\ 2","3","0"\symbol{'135}\ :\ string\ list}
\item[]\textsl{-\ }Reg.output
\item[]\textsl{=\ }("",\ Reg.sortUnions(Reg.fromString\ "12\ +\ 0\ +\ 3\ +\ 0"));
\item[]\textsl{0\ +\ 3\ +\ 12}
\item[]\textsl{val\ it\ =\ ()\ :\ unit}
\end{list}

We can use the functions \texttt{allSym}, \texttt{allStr} and
\texttt{fromStrSet} like this:
\begin{list}{}
{\setlength{\leftmargin}{\leftmargini}
\setlength{\rightmargin}{0cm}
\setlength{\itemindent}{0cm}
\setlength{\listparindent}{0cm}
\setlength{\itemsep}{0cm}
\setlength{\parsep}{0cm}
\setlength{\labelsep}{0cm}
\setlength{\labelwidth}{0cm}
\catcode`\#=12
\catcode`\$=12
\catcode`\%=12
\catcode`\^=12
\catcode`\_=12
\catcode`\.=12
\catcode`\?=12
\catcode`\!=12
\catcode`\&=12
\ttfamily}
\small
\item[]\textsl{-\ }Reg.output("",\ Reg.allSym(SymSet.fromString\ ""));
\item[]\textsl{$}
\item[]\textsl{val\ it\ =\ ()\ :\ unit}
\item[]\textsl{-\ }Reg.output("",\ Reg.allSym(SymSet.fromString\ "0"));
\item[]\textsl{0}
\item[]\textsl{val\ it\ =\ ()\ :\ unit}
\item[]\textsl{-\ }Reg.output("",\ Reg.allSym(SymSet.fromString\ "0,\ 1,\ 2"));
\item[]\textsl{0\ +\ 1\ +\ 2}
\item[]\textsl{val\ it\ =\ ()\ :\ unit}
\item[]\textsl{-\ }Reg.output("",\ Reg.allStr(SymSet.fromString\ ""));
\item[]\textsl{$\symbol{'052}}
\item[]\textsl{val\ it\ =\ ()\ :\ unit}
\item[]\textsl{-\ }Reg.output("",\ Reg.allStr(SymSet.fromString\ "0"));
\item[]\textsl{0\symbol{'052}}
\item[]\textsl{val\ it\ =\ ()\ :\ unit}
\item[]\textsl{-\ }Reg.output("",\ Reg.allStr(SymSet.fromString\ "2,\ 1,\ 0"));
\item[]\textsl{(0\ +\ 1\ +\ 2)\symbol{'052}}
\item[]\textsl{val\ it\ =\ ()\ :\ unit}
\item[]\textsl{-\ }Reg.output
\item[]\textsl{=\ }("",
\item[]\textsl{=\ }\ Reg.fromStrSet(StrSet.fromString\ "one,\ two,\ three,\ four"));
\item[]\textsl{one\ +\ two\ +\ four\ +\ three}
\item[]\textsl{val\ it\ =\ ()\ :\ unit}
\end{list}


\subsection{JForlan}

The Java program JForlan
\index{JForlan}%
can be used to view and edit regular expression trees.  It can be
invoked directly, or run via Forlan.  See the Forlan website for more
information.

\subsection{Notes}

A novel feature of this book is that regular expressions are trees, so
that our linear syntax for regular expressions is derived rather than
primary.  Thus regular expression equality is just tree equality,
and it's easy to explain when parentheses are necessary in
a linear description of a regular expression.  Furthermore,
tree-oriented concepts, notation and operations automatically
apply to regular expressions, letting us, e.g., give definitions
by structural recursion.
\index{regular expression|)}%

%%% Local Variables: 
%%% mode: latex
%%% TeX-master: "book"
%%% End: 

\section{Equivalence and  Correctness of Regular  Expressions}
\label{EquivalenceOfRegularExpressions}

\index{regular expression|(}%
In this section, we say what it means for regular expressions to be
equivalent, show a series of results about regular expression
equivalence, and consider how regular expressions may be designed and
proved correct.

\subsection{Equivalence of Regular Expressions}

Regular expressions $\alpha$ and $\beta$ are
\index{regular expression!equivalence|(}%
\index{ equiv@$\approx$}%
\index{regular expression! equiv@$\approx$}%
\emph{equivalent} iff $L(\alpha) = L(\beta)$.  In other words, $\alpha$
and $\beta$ are equivalent iff $\alpha$ and $\beta$ generate the same
language.  We define a relation $\approx$ on $\Reg$ by:
$\alpha\approx\beta$ iff $\alpha$ and $\beta$ are equivalent.
For example, $L(\mathsf{(00)}^*+\%)=L(\mathsf{{(00)}^*})$,
and thus $\mathsf{(00)}^*+\%\approx\mathsf{(00)^*}$.

One approach to showing that $\alpha\approx\beta$ is to show that
$L(\alpha)\sub L(\beta)$ and $L(\beta)\sub L(\alpha)$.  The
\index{inclusion}%
\index{set!inclusion}%
following proposition is useful for showing language inclusions, not just
ones involving regular languages.

\begin{proposition}
\label{Inclusions}

\begin{enumerate}[(1)]
\item For all $A_1,A_2,B_1,B_2\in\Lan$, if
$A_1\sub B_1$ and $A_2\sub B_2$, then $A_1\cup A_2\sub B_1\cup B_2$.

\item For all $A_1,A_2,B_1,B_2\in\Lan$, if
$A_1\sub B_1$ and $A_2\sub B_2$, then $A_1\cap A_2\sub B_1\cap B_2$.

\item For all $A_1,A_2,B_1,B_2\in\Lan$, if
$A_1\sub B_1$ and $B_2\sub A_2$, then $A_1-A_2\sub B_1-B_2$.

\item For all $A_1,A_2,B_1,B_2\in\Lan$, if
$A_1\sub B_1$ and $A_2\sub B_2$, then $A_1A_2\sub B_1B_2$.

\item For all $A,B\in\Lan$ and $n\in\nats$,
if $A\sub B$, then $A^n\sub B^n$.

\item For all $A,B\in\Lan$, if $A\sub B$, then $A^*\sub B^*$.
\end{enumerate}
\end{proposition}

In Part~(3), note that the second part of the sufficient condition for
knowing $A_1-A_2\sub B_1-B_2$ is $B_2\sub A_2$, not $A_2\sub B_2$.

\begin{proof}
(1) and (2) are straightforward.  We show (3) as an example, below.
(4) is easy.  (5) is proved by mathematical induction, using (4).  (6)
is proved using (5).

For (3), suppose that $A_1,A_2,B_1,B_2\in\Lan$, $A_1\sub B_1$ and
$B_2\sub A_2$.  To show that $A_1-A_2\sub B_1-B_2$, suppose $w\in
A_1-A_2$.  We must show that $w\in B_1-B_2$.  It will suffice to show
that $w\in B_1$ and $w\not\in B_2$.

Since $w\in A_1-A_2$, we have that $w\in A_1$ and
$w\not\in A_2$.  Since $A_1\sub B_1$, it follows that $w\in B_1$.
Thus, it remains to show that $w\not\in B_2$.

Suppose, toward a contradiction, that $w\in B_2$.  Since
$B_2\sub A_2$, it follows that $w\in A_2$---contradiction.  Thus we
have that $w\not\in B_2$.
\end{proof}

Next we show that our relation $\approx$ has some of the familiar
properties of equality.
\index{reflexive on set! equiv@$\approx$}%
\index{symmetric! equiv@$\approx$}%
\index{transitive! equiv@$\approx$}%
\index{regular expression! equiv@$\approx$!reflexive}%
\index{regular expression! equiv@$\approx$!symmetric}%
\index{regular expression! equiv@$\approx$!transitive}%
\begin{proposition}
\label{RegEquivalence}
\begin{enumerate}[(1)]
\item $\approx$ is reflexive on $\Reg$, symmetric and transitive.

\item For all $\alpha,\beta\in\Reg$,
if $\alpha\approx\beta$, then $\alpha^*\approx{\beta}^*$.

\item For all $\alpha_1,\alpha_2,\beta_1,\beta_2\in\Reg$,
if $\alpha_1\approx\beta_1$ and $\alpha_2\approx\beta_2$,
then $\alpha_1\alpha_2\approx \beta_1\beta_2$.

\item For all $\alpha_1,\alpha_2,\beta_1,\beta_2\in\Reg$,
if $\alpha_1\approx\beta_1$ and $\alpha_2\approx\beta_2$,
then $\alpha_1+\alpha_2\approx \beta_1+\beta_2$.
\end{enumerate}
\end{proposition}

\begin{proof}
Follows from the properties of $=$.  As an example, we show
Part~(4).

Suppose $\alpha_1,\alpha_2,\beta_1,\beta_2\in\Reg$, and assume that
$\alpha_1\approx\beta_1$ and $\alpha_2\approx\beta_2$.
Then $L(\alpha_1)=L(\beta_1)$ and $L(\alpha_2)=L(\beta_2)$, so that
\begin{align*}
L(\alpha_1+\alpha_2) &= L(\alpha_1)\cup L(\alpha_2) =
L(\beta_1)\cup L(\beta_2) \\
&= L(\beta_1+\beta_2) .
\end{align*}
Thus $\alpha_1+\alpha_2\approx \beta_1+\beta_2$.
\end{proof}

A consequence of Proposition~\ref{RegEquivalence} is the following
proposition, which says that, if we replace a subtree of a regular
expression $\alpha$ by an equivalent regular expression, that the
resulting regular expression is equivalent to $\alpha$.

\begin{proposition}
\label{RegContext}
Suppose $\alpha,\beta,\beta'\in\Reg$, $\beta\approx\beta'$,
$\pat\in\Path$ is valid for $\alpha$, and $\beta$ is
the subtree of $\alpha$ at position $\pat$.
Let $\alpha'$ be the result of replacing the subtree at
position $\pat$ in $\alpha$ by $\beta'$.  Then $\alpha\approx\alpha'$.
\end{proposition}

\begin{proof}
By induction on $\alpha$.
\end{proof}

Next, we state and prove some equivalences involving union.

\begin{proposition}
\label{RegUnion}

\begin{enumerate}[(1)]
\item For all $\alpha,\beta\in\Reg$,
$\alpha + \beta\approx\beta + \alpha$.

\item For all $\alpha,\beta,\gamma\in\Reg$,
$(\alpha + \beta) + \gamma\approx\alpha + (\beta + \gamma)$.

\item For all $\alpha\in\Reg$, $\$ +
\alpha\approx\alpha$.

\item For all $\alpha\in\Reg$, $\alpha +
\alpha\approx\alpha$.

\item If $L(\alpha)\sub L(\beta)$, then $\alpha + \beta\approx
\beta$.
\end{enumerate}
\end{proposition}

\begin{proof}
\begin{enumerate}[(1)]
\item Follows from the commutativity of $\cup$.

\item Follows from the associativity of $\cup$.

\item Follows since $\emptyset$ is the identity for $\cup$.

\item Follows since $\cup$ is idempotent: $A\cup A=A$,
for all sets $A$.

\item Follows since, if $L_1\sub L_2$, then $L_1\cup L_2=L_2$.
\end{enumerate}
\end{proof}

Next, we consider equivalences for concatenation.
\index{language!concatenation}%

\begin{proposition}
\label{RegConcat}
\begin{enumerate}[(1)]
\item For all $\alpha,\beta,\gamma\in\Reg$,
$(\alpha\beta)\gamma\approx\alpha(\beta\gamma)$.

\item For all $\alpha\in\Reg$, $\%\alpha\approx\alpha\approx\alpha\%$.

\item For all $\alpha\in\Reg$, $\$\alpha\approx\$ \approx
\alpha\$$.
\end{enumerate}
\end{proposition}

\begin{proof}
\begin{enumerate}[(1)]
\item Follows from the associativity of language concatenation.

\item Follows since $\{\%\}$ is the identity for language concatenation.

\item Follows since $\emptyset$ is the zero for language
concatenation.
\end{enumerate}
\end{proof}

Next we consider the distributivity of concatenation over union.
First, we prove a proposition concerning languages.  Then,
we use this proposition to show the corresponding proposition
for regular expressions.

\begin{proposition}
\label{DistribLan}
\begin{enumerate}[(1)]
\item For all $L_1,L_2,L_3\in\Lan$,
$L_1(L_2\cup L_3) = L_1L_2\cup L_1L_3$.

\item For all $L_1,L_2,L_3\in\Lan$,
$(L_1\cup L_2)L_3 = L_1L_3\cup L_2L_3$.
\end{enumerate}
\end{proposition}

\begin{proof}
We show the proof of Part~(1); the proof of the other part is
similar.  Suppose $L_1,L_2,L_3\in\Lan$.  It will suffice to show that
\begin{gather*}
L_1(L_2\cup L_3)\sub L_1L_2\cup L_1L_3\sub L_1(L_2\cup L_3).
\end{gather*}

To see that
$L_1(L_2\cup L_3)\sub L_1L_2\cup L_1L_3$, suppose
$w\in L_1(L_2\cup L_3)$.  We must show that $w\in L_1L_2\cup L_1L_3$.
By our assumption, $w=xy$ for some $x\in L_1$ and $y\in L_2\cup L_3$.
There are two cases to consider.
\begin{itemize}
\item Suppose $y\in L_2$.  Then $w=xy\in L_1L_2\sub L_1L_2\cup L_1L_3$.

\item Suppose $y\in L_3$.  Then $w=xy\in L_1L_3\sub L_1L_2\cup L_1L_3$.
\end{itemize}

To see that $L_1L_2\cup L_1L_3\sub L_1(L_2\cup L_3)$, suppose
$w\in L_1L_2\cup L_1L_3$.  We must show that $w\in L_1(L_2\cup L_3)$.
There are two cases to consider.
\begin{itemize}
\item Suppose $w\in L_1L_2$.  Then $w=xy$ for some $x\in L_1$ and $y\in L_2$.
Thus $y\in L_2\cup L_3$, so that $w=xy\in L_1(L_2\cup L_3)$.

\item Suppose $w\in L_1L_3$.  Then $w=xy$ for some $x\in L_1$ and $y\in L_3$.
Thus $y\in L_2\cup L_3$, so that $w=xy\in L_1(L_2\cup L_3)$.
\end{itemize}
\end{proof}

\begin{proposition}
\label{DistribReg}
\begin{enumerate}[(1)]
\item For all $\alpha,\beta,\gamma\in\Reg$,
$\alpha(\beta+\gamma)\approx \alpha\beta+\alpha\gamma$.

\item For all $\alpha,\beta,\gamma\in\Reg$,
$(\alpha+\beta)\gamma\approx \alpha\gamma+\beta\gamma$.
\end{enumerate}
\end{proposition}

\begin{proof}
Follows from Proposition~\ref{DistribLan}.  Consider, e.g., the proof
of Part~(1).  By Proposition~\ref{DistribLan}(1), we have that
\begin{align*}
L(\alpha(\beta+\gamma)) &= L(\alpha)L(\beta+\gamma) \\
&= L(\alpha)(L(\beta)\cup L(\gamma))  \\
&= L(\alpha)L(\beta)\cup L(\alpha)L(\gamma) \\
&= L(\alpha\beta)\cup L(\alpha\gamma) \\
&= L(\alpha\beta + \alpha\gamma)
\end{align*}
Thus $\alpha(\beta+\gamma)\approx \alpha\beta+\alpha\gamma$.
\end{proof}

Finally, we turn our attention to equivalences for Kleene closure,
\index{language!closure}%
first stating and proving some results for languages, and then stating
and proving the corresponding results for regular expressions.

\begin{proposition}
\label{ClosureInclusions}
\begin{itemize}
\item For all $L\in\Lan$, $LL^*\sub L^*$.

\item For all $L\in\Lan$, $L^*L\sub L^*$.
\end{itemize}
\end{proposition}

\begin{proof}
E.g., to see that $LL^*\sub L^*$, suppose $w\in LL^*$.
Then $w=xy$ for some $x\in L$ and $y\in L^*$.  Hence
$y\in L^n$ for some $n\in\nats$.  Thus $w=xy\in LL^n=L^{n+1}\sub L^*$.
\end{proof}

\begin{proposition}
\label{ClosureLan}
\begin{enumerate}[(1)]
\item $\emptyset^* = \{\%\}$.

\item $\{\%\}^* = \{\%\}$.

\item For all $L\in\Lan$, $L^*L = LL^*$.

\item For all $L\in\Lan$, $L^*L^* = L^*$.

\item For all $L\in\Lan$, $(L^*)^* = L^*$.

\item For all $L_1L_2\in\Lan$, $(L_1L_2)^*L_1 = L_1(L_2L_1)^*$.
\end{enumerate}
\end{proposition}

\begin{proof}
The six parts can be proven in order using
Proposition~\ref{Inclusions}.  All parts but (2), (5) and (6) can be
proved without using induction.

As an example, we show the proof of (5).  To show that
$(L^*)^*=L^*$, it will suffice to show that $(L^*)^*\sub L^*\sub
(L^*)^*$.

To see that $(L^*)^*\sub L^*$, we use mathematical induction
to show that, for all $n\in\nats$,
$(L^*)^n\sub L^*$.
\begin{description}
\item[\quad(Basis Step)] We have that $(L^*)^0=\{\%\}=L^0\sub L^*$.

\item[\quad(Inductive Step)] Suppose $n\in\nats$, and assume the inductive
hypothesis: $(L^*)^n\sub L^*$.  We must show that $(L^*)^{n+1}\sub
L^*$.  By the inductive hypothesis, Proposition~\ref{Inclusions}(4) and
Part~(4), we have that $(L^*)^{n+1}=L^*(L^*)^n\sub L^*L^*=L^*$.
\end{description}

Now, we use the result of the induction to prove that
$(L^*)^*\sub L^*$.  Suppose $w\in (L^*)^*$.  We must show that
$w\in L^*$.  Since $w\in (L^*)^*$, we have that $w\in 
(L^*)^n$ for some $n\in\nats$.  Thus, by the result of the
induction, $w\in (L^*)^n\sub L^*$.

Finally, for the other inclusion, we have that $L^*=(L^*)^1\sub
(L^*)^*$.
\end{proof}

\begin{exercise}
Prove Proposition~\ref{ClosureLan}(6), i.e., for all
$L_1L_2\in\Lan$, $(L_1L_2)^*L_1 = L_1(L_2L_1)^*$.
\end{exercise}

\begin{proposition}
\label{ClosureReg}
\begin{enumerate}[(1)]
\item $\$^*\approx\%$.

\item $\%^*\approx\%$.

\item For all $\alpha\in\Reg$, $\alpha^*\alpha \approx \alpha\,\alpha^*$.

\item For all $\alpha\in\Reg$, $\alpha^*\alpha^* \approx \alpha^*$.

\item For all $\alpha\in\Reg$, $(\alpha^*)^* \approx \alpha^*$.

\item For all $\alpha,\beta\in\Reg$,
$(\alpha\beta)^*\alpha \approx \alpha(\beta\alpha)^*$.
\end{enumerate}
\end{proposition}

\begin{proof}
Follows from Proposition~\ref{ClosureLan}.  Consider, e.g., the proof
of Part~(5).  By Proposition~\ref{ClosureLan}(5), we have that
\begin{gather*}
L((\alpha^*)^*) = L(\alpha^*)^* = (L(\alpha)^*)^* = L(\alpha)^* =
L(\alpha^*) .
\end{gather*}
Thus $(\alpha^*)^* \approx \alpha^*$.
\end{proof}
\index{regular expression!equivalence|)}%

\subsection{Proving the Correctness of Regular Expressions}
\label{ProvingTheCorrectnessOfRegularExpressions}

\index{regular expression!design(}%
\index{regular expression!proof of correctness(}%
In this subsection, we use two examples to show how regular
expressions can be designed and proved correct.

For our first example, define a function
$\zeros\in\{\mathsf{0,1}\}^*\fun\nats$ by recursion:
\begin{align*}
\zeros\,\% &= 0 , \\
\zeros(\zerosf w) &= \zeros\,w + 1,
\eqtxtr{for all} w\in\{\mathsf{0,1}\}^* , \eqtxt{and}\\
\zeros(\onesf w) &= \zeros\,w, \eqtxtr{for all} w\in\{\mathsf{0,1}\}^* .
\end{align*}
Thus $\zeros\,w$ is the number of occurrences of $\zerosf$ in $w$.
It is easy to show that:
\begin{itemize}
\item $\zeros\,\zerosf = 1$;

\item $\zeros\,\onesf = 0$;

\item for all $x,y\in\{\mathsf{0,1}\}^*$, $\zeros(xy) = \zeros\,x +
\zeros\,y$;

\item for all $n\in\nats$, $\zeros(\zerosf^n) = n$; and

\item for all $n\in\nats$, $\zeros(\onesf^n) = 0$.
\end{itemize}
Let
\begin{displaymath}
X=\setof{w\in\{\mathsf{0,1}\}^*}{\zeros\,w\eqtxtl{is even}} ,
\end{displaymath}
so that $X$ is all strings of $\zerosf$'s and $\onesf$'s with an even
number of $\zerosf$'s.  Clearly, $\%\in X$ and $\{\onesf\}^*\sub X$.

Let's consider the problem of finding a regular expression that
generates $X$.  A string with this property would begin with some
number of $\onesf$'s (possibly none).  After this, the string would
have some number of parts (possibly none), each consisting of a
$\zerosf$, followed by some number of $\onesf$'s, followed by a
$\zerosf$, followed by some number of $\onesf$'s.  The above
considerations lead us to the regular expression
\begin{displaymath}
\alpha=\onesf^*(\zerosf\onesf^*\zerosf\onesf^*)^* .
\end{displaymath}

To prove $L(\alpha) = X$, it's helpful to give names
to the meanings of two parts of $\alpha$.  Let
\begin{displaymath}
Y = \{\zerosf\}\{\onesf\}^*\{\zerosf\}\{\onesf\}^*
\quad \eqtxt{and} \quad
Z = \{\onesf\}^* Y^* ,
\end{displaymath}
so that $L(\zerosf\onesf^*\zerosf\onesf^*) = Y$ and
$L(\alpha) = Z$.  Thus it will suffice to prove that $Z = X$,
and we do this by showing $Z\sub X\sub Z$.  We begin by showing
$Z\sub X$.

\begin{lemma}
\label{RegSyn1Lem1}
\begin{enumerate}[\quad(1)]
\item $Y\sub X$.

\item $XX\sub X$.
\end{enumerate}
\end{lemma}

\begin{proof}
\begin{enumerate}[\quad(1)]
\item Suppose $w\in Y$, so that $w=\zerosf x\zerosf y$ for some
$x,y\in\{\onesf\}^*$.  Thus $\zeros\,w = \zeros(\zerosf x\zerosf y) =
\zeros\,\zerosf + \zeros\,x + \zeros\,\zerosf + \zeros\,y =
1 + 0 + 1 + 0 = 2$ is even, so that $w\in X$.

\item Suppose $w\in XX$, so that $w=xy$ for some $x,y\in X$.  Then
$\zeros\,x$ and $\zeros\,y$ are even, so that $\zeros\,w =
\zeros\,x + \zeros\,y$ is even.  Hence $w\in X$.
\end{enumerate}
\end{proof}

\begin{lemma}
\label{RegSyn1Lem2}
$Y^*\sub X$.
\end{lemma}

\begin{proof}
It will suffice to show that, for all $n\in\nats$, $Y^n\sub X$, and
we show this using mathematical induction.
\begin{description}
\item[\quad(Basis Step)] We have that $Y^0 = \{\%\}\sub X$.

\item[\quad(Inductive Step)] Suppose $n\in\nats$, and assume the
inductive hypothesis: $Y^n\sub X$. Then $Y^{n+1} = YY^n\sub XX\sub X$,
by Lemma~\ref{RegSyn1Lem1}
\end{description}
\end{proof}

\begin{lemma}
\label{RegSyn1Lem3}
$Z\sub X$.
\end{lemma}

\begin{proof}
By Lemmas~\ref{RegSyn1Lem1} and \ref{RegSyn1Lem2}, we have that
$Z = \{\onesf\}^*Y^* \sub XX \sub X$.
\end{proof}

To prove $X\sub Z$, it's helpful to define another language:
\begin{displaymath}
  U = \setof{w\in X}{\onesf\eqtxt{is not a prefix of} w} .
\end{displaymath}

\begin{lemma}
\label{RegSyn1Lem4}
$U\sub Y^*$.
\end{lemma}

\begin{proof}
Because $U\sub\{\mathsf{0,1}\}^*$, it will suffice to show that,
for all $w\in\{\mathsf{0,1}\}^*$,
\begin{displaymath}
  \eqtxtr{if} w\in U, \eqtxt{then} w\in Y^* .
\end{displaymath}
We proceed by strong string induction.  Suppose $w\in\{\mathsf{0,1}\}^*$,
and assume the inductive hypothesis: for all $x\in\{\mathsf{0,1}\}^*$,
if $x$ is a proper substring of $w$, then
\begin{displaymath}
  \eqtxtr{if} x\in U, \eqtxt{then} x\in Y^* .
\end{displaymath}
We must show that
\begin{displaymath}
  \eqtxtr{if} w\in U, \eqtxt{then} w\in Y^* .
\end{displaymath}
Suppose $w\in U$, so that $\zeros\,w$ is even and $\onesf$ is not a
prefix of $w$.  We must show that $w\in Y^*$.  If $w=\%$, then $w\in
Y^*$.  Otherwise, $w=\zerosf x$ for some $x\in\{\mathsf{0,1}\}^*$.
Since $1 + \zeros\,x = \zeros\,\zerosf + \zeros\,x = \zeros(\zerosf x)
= \zeros\,w$ is even, we have that $\zeros\,x$ is odd. Let $n$ be the
largest element of $\nats$ such that $\onesf^n$ is a prefix of
$x$. ($n$ is well-defined, since $\onesf^0=\%$ is a prefix of $x$.)
Thus $x = \onesf^ny$ for some $y\in\{\mathsf{0,1}\}^*$.  Since
$\zeros\, y = 0 + \zeros\,y = \zeros\,\onesf^n + \zeros\,y =
\zeros(\onesf^n y) = \zeros\,x$ is odd, we have that $y\neq\%$.  And,
by the definition of $n$, $\onesf$ is not a prefix of $y$.  Hence
$y=\zerosf z$ for some $z\in\{\mathsf{0,1}\}^*$.  Since $1 + \zeros\,z
= \zeros\,\zerosf + \zeros\,z = \zeros(\zerosf z) = \zeros\,y$ is odd,
we have that $\zeros\,z$ is even.  Let $m$ be the largest element of
$\nats$ such that $\onesf^m$ is a prefix of $z$.  Thus $z=\onesf^m u$
for some $u\in\{\mathsf{0,1}\}^*$, and $\onesf$ is not a prefix of
$u$.  Since $\zeros\,u = 0 + \zeros\,u = \zeros\,\onesf^m + \zeros\,u
= \zeros(\onesf^m u) = \zeros\,z$ is even, it follows that $u\in U$.
Summarizing, we have that $w = \zerosf x = \zerosf \onesf^n y =
\zerosf \onesf^n \zerosf z = \zerosf \onesf^n \zerosf \onesf^m u$ and
$u\in U$.  Since $u$ is a proper substring of $w$, the inductive
hypothesis tells us that $u\in Y^*$.  Hence $w = \zerosf \onesf^n
\zerosf \onesf^m u = (\zerosf \onesf^n \zerosf \onesf^m) u \in
YY^*\sub Y^*$.
\end{proof}

\begin{lemma}
\label{RegSyn1Lem5}
$X\sub Z$.
\end{lemma}

\begin{proof}
Suppose $w\in X$.  Let $n$ be the largest element of $\nats$ such that
$\onesf^n$ is a prefix of $w$.  Thus $w=\onesf^n x$ for some
$x\in\{\mathsf{0,1}\}^*$. Since $w\in X$, we have that $\zeros\,x = 0
+ \zeros\,x = \zeros\,\onesf^n + \zeros\,x = \zeros\,w$ is even, so
that $x\in X$.  By the definition of $n$, we have that $\onesf$
is not a prefix of $x$, and thus $x\in U$.  Hence $w=\onesf^n x\in
\{\onesf\}^* U\sub \{\onesf\}^*Y^* = Z$, by Lemma~\ref{RegSyn1Lem4}.
\end{proof}

By Lemmas~\ref{RegSyn1Lem3} and \ref{RegSyn1Lem5}, we have
that $Z\sub X\sub Z$, completing our proof that $\alpha$ is correct.

Our second example of regular expression design and proof
of correction involves the languages
\begin{align*}
A &=\{\mathsf{001, 011, 101, 111}\} , \eqtxt{and} \\
B &=\{\,w\in\{\mathsf{0,1}\}^* \mid \eqtxtr{for all}x,y\in\{\mathsf{0,1}\}^*,
\eqtxt{if}w=x\zerosf y, \eqtxtr{then there is a} z\in A \\
&\quad\;\;\; \eqtxt{such that} z\eqtxt{is a prefix of} y\,\}.
\end{align*}
The elements of $A$ can be thought of as the odd numbers between $1$
and $7$, expressed in binary, and $B$ consists of those strings
of $\zerosf$'s and $\onesf$'s in which every occurrence of $\zerosf$
is immediately followed by an element of $A$.

E.g., $\%\in B$, since the empty string
has no occurrences of $\zerosf$, and $\mathsf{00111}$ is in $B$, since
its first $\zerosf$ is followed by $\mathsf{011}$ and its second
$\zerosf$ is followed by $\mathsf{111}$.  But $\mathsf{0000111}$ is
not in $B$, since its first $\zerosf$ is followed by $\mathsf{000}$,
which is not in $A$.  And $\mathsf{011}$ is not in $B$, since
$|11|<3$.

Note that, for all $x,y\in B$, $xy\in B$, i.e., $BB\sub B$.  This
holds, since: each occurrence of $\zerosf$ in $x$ is followed by an
element of $A$ in $x$, and is thus followed by the same element of $A$
in $xy$; and each occurrence of $\zerosf$ in $y$ is followed by an
element of $A$ in $y$, and is thus followed by the same element of $A$
in $xy$.

Furthermore, for all strings $x,y$, if $xy\in B$, then $y$ is in $B$,
i.e., every suffix of an element of $B$ is also in $B$.  This holds
since if there was an occurrence of $\zerosf$ in $y$ that wasn't
followed by an element of $A$, then this same occurrence of $\zerosf$
in the suffix $y$ of $xy$ would also not be followed by an element of
$A$, contradicting $xy\in B$.

How should we go about finding a regular expression $\alpha$ such that
$L(\alpha)=B$?  Because $\%\in B$, for all $x,y\in B$, $xy\in B$, and
for all strings $x,y$, if $xy\in B$ then $y\in B$,
our regular expression can have the form $\beta^*$, where $\beta$
generates all the strings that are \emph{basic} in the sense that they
are nonempty elements of $B$ with no non-empty proper prefixes that
are in $B$.

Let's try to understand what the basic strings look like.  Clearly,
$\onesf$ is basic, so there will be no more basic strings that begin
with $\onesf$.  But what about the basic strings beginning with
$\zerosf$?  No sequence of $\zerosf$'s is basic, and any string that
begins with four or more $\zerosf$'s will not be basic.  It is easy to
see that $\mathsf{000111}$ is basic.  In fact, it is the only basic
string of the form $\mathsf{000}u$.  (The first $\zerosf$ forces $u$
to begin with $\onesf$, the second $\zerosf$ forces $u$ to continue
with $\mathsf{1}$, and the third forces $u$ to continue with
$\mathsf{1}$.  And, if $|u|>3$, then the overall string would have a
nonempty, proper prefix in $B$, and so wouldn't be basic.)  Similarly,
$\mathsf{00111}$ is the only basic string beginning with
$\mathsf{001}$.  But what about the basic strings beginning with
$\mathsf{01}$?  It's not hard to see that there are infinitely many
such strings: $\mathsf{0111}$, $\mathsf{010111}$, $\mathsf{01010111}$,
$\mathsf{0101010111}$, etc.  Fortunately, there is a simple pattern
here: we have all strings of the form
$\zerosf(\mathsf{10})^n\mathsf{111}$ for $n\in\nats$.

By the above considerations, it seems that we can let our regular
expression be
\begin{displaymath}
(\mathsf{1} + \zerosf(\mathsf{10})^*\mathsf{111} + \mathsf{00111} +
\mathsf{000111})^* .
\end{displaymath}
But, using some of the equivalences we learned about above,
we can turn this regular expression into
\begin{displaymath}
\mathsf{(1 + 0(0 + 00 + (10)^*)111)^*} ,
\end{displaymath}
which we take as our $\alpha$.  Now, we prove that $L(\alpha)=B$.

Let
\begin{displaymath}
X = \mathsf{\{0\} \cup \{00\} \cup \{10\}^*} \quad \eqtxt{and} \quad
Y = \{\onesf\}\cup\{\zerosf\}X\{\mathsf{111}\} .
\end{displaymath}
Then, we have that
\begin{align*}
X &= L(\mathsf{0 + 00 + (10)^*}) , \\
Y &= L(\mathsf{1 + 0(0 + 00 + (10)^*)111}) , \eqtxt{and} \\
Y^* &= L(\mathsf{(1 + 0(0 + 00 + (10)^*)111)^*}) = L(\alpha) .
\end{align*}
Thus, it will suffice to show that $Y^* = B$.  We will show that
$Y^*\sub B\sub Y^*$.

To begin with, we would like to use mathematical induction to prove
that, for all $n\in\nats$,
$\{\zerosf\}\{\mathsf{10}\}^n\{\mathsf{111}\}\sub B$.  But in order
for the inductive step to succeed, we must prove something stronger.
Let
\begin{displaymath}
C = \setof{w\in B}{\mathsf{01}\eqtxt{is a prefix of}w} .
\end{displaymath}

\begin{lemma}
\label{RegSyn2Lem1}
For all $n\in\nats$, $\{\zerosf\}\{\mathsf{10}\}^n\{\mathsf{111}\}\sub
C$.
\end{lemma}

\begin{proof}
We proceed by mathematical induction.
\begin{description}
\item[\quad(Basis Step)] Since $\mathsf{01}$ is a prefix of $\mathsf{0111}$,
and $\mathsf{0111}\in B$, we have that $\mathsf{0111}\in C$.  Hence
  ${\{\zerosf\}\{\mathsf{10}\}^0\{\mathsf{111}\}} =
  \mathsf{\{0\}\{\%\}\{111\}} = \mathsf{\{0\}\{111\}} =
  \mathsf{\{0111\}}\sub C$.

\item[\quad(Inductive Step)] Suppose $n\in\nats$, and assume the
  inductive hypothesis:
  $\{\zerosf\}\{\mathsf{10}\}^n\{\mathsf{111}\}\sub C$.  We must show
  that ${\{\mathsf{0}\}\{\mathsf{10}\}^{n+1}\{\mathsf{111}\}\sub C}$.
  Since
  \begin{alignat*}{2}
    \{\mathsf{0}\}\{\mathsf{10}\}^{n+1}\{\mathsf{111}\} &=
    \{\mathsf{0}\}\{\mathsf{10}\}\{\mathsf{10}\}^n\{\mathsf{111}\} \\
    &= \{\mathsf{01}\}\{\mathsf{0}\}\{\mathsf{10}\}^n\{\mathsf{111}\} \\
    &\sub \mathsf{\{01\}}C &&\by{inductive hypothesis} ,
  \end{alignat*}
  it will suffice to show that $\mathsf{\{01\}}C\sub C$.  Suppose
  $w\in\mathsf{\{01\}}C$.  We must show that $w\in C$.  We have that
  $w=\mathsf{01}x$ for some $x\in C$.  Thus $w$ begins with
  $\mathsf{01}$.  It remains to show that $w\in B$.  Since $x\in C$,
  we have that $x$ begins with $\mathsf{01}$.  Thus the first
  occurrence of $\zerosf$ in $w=\mathsf{01}x$ is followed by
  $\mathsf{101}\in A$.  Furthermore, any other occurrence of
  $\zerosf$ in $w=\mathsf{01}x$ is within $x$, and so is followed by
  an element of $A$ because $x\in C\sub B$.  Thus $w\in B$.
\end{description}
\end{proof}

\begin{lemma}
\label{RegSyn2Lem2}
$Y\sub B$.
\end{lemma}

\begin{proof}
Suppose $w\in Y$.  We must show that $w\in B$.  If $w=\onesf$, then
$w\in B$.  Otherwise, we have that $w=\zerosf x\mathsf{111}$ for some
$x\in X$.  There are three cases to consider.
\begin{itemize}
\item Suppose $x=\zerosf$.  Then $w=\mathsf{00111}$ is in $B$.

\item Suppose $x=\mathsf{00}$.  Then $w=\mathsf{000111}$ is in $B$.

\item Suppose $x\in\{\mathsf{10}\}^*$.  Then $x\in \{\mathsf{10}\}^n$
for some $n\in\nats$.  By Lemma~\ref{RegSyn2Lem1},
we have that $w=\zerosf x\mathsf{111}\in
\{\zerosf\}\{\mathsf{10}\}^n\{\mathsf{111}\}\sub C\sub B$.
\end{itemize}
\end{proof}

\begin{lemma}
\label{RegSyn2Lem3}
$Y^*\sub B$.
\end{lemma}

\begin{proof}
It will suffice to show that, for all $n\in\nats$, $Y^n\sub B$, and
we proceed by mathematical induction.
\begin{description}
\item[\quad(Basis Step)] Since $\%\in B$, we have that $Y^0=\{\%\}\sub B$.

\item[\quad(Inductive Step)] Suppose $n\in\nats$, and assume the
  inductive hypothesis: $Y^n\sub B$.  Then $Y^{n+1} = YY^n \sub BB
  \sub B$, by Lemma~\ref{RegSyn2Lem2} and the inductive hypothesis.
\end{description}
\end{proof}

\begin{lemma}
\label{RegSyn2Lem5}
$B\sub Y^*$.
\end{lemma}

\begin{proof}
Since $B\sub\{\mathsf{0,1}\}^*$, it will suffice to show that, for
all $w\in\{\mathsf{0,1}\}^*$,
\begin{displaymath}
\eqtxtr{if}w\in B,\eqtxt{then}w\in Y^* .
\end{displaymath}
We proceed by strong string induction.  Suppose
$w\in\{\mathsf{0,1}\}^*$, and assume the inductive hypothesis: for all
$x\in \{\mathsf{0,1}\}^*$, if $x$ is a proper substring of $w$, then
\begin{displaymath}
\eqtxtr{if}x\in B,\eqtxt{then}x\in Y^* .
\end{displaymath}
We must show that
\begin{displaymath}
\eqtxtr{if}w\in B,\eqtxt{then}w\in Y^* .
\end{displaymath}
Suppose $w\in B$.  We must show that $w\in Y^*$.
There are three main cases to consider.
\begin{itemize}
\item Suppose $w=\%$.  Then $w\in\{\%\}=Y^0\sub Y^*$.

\item Suppose $w=\zerosf x$ for some $x\in\mathsf{\{0,1\}}^*$.  Since
  $w\in B$, the first $\zerosf$ of $w$ must be followed by an element
  of $A$.  Hence $x\neq\%$, so that there are two cases to consider.
  \begin{itemize}
  \item Suppose $x=\zerosf y$ for some $y\in\mathsf{\{0,1\}}^*$.  Thus
    $w=\zerosf x=\mathsf{00}y$.  Since $\mathsf{00}y=w\in B$, we have
    that $y\neq\%$.  Thus, there are two cases to consider.
    \begin{itemize}
    \item Suppose $y=\zerosf z$ for some $z\in\mathsf{\{0,1\}}^*$.
      Thus $w=\mathsf{00}y=\mathsf{000}z$.  Since the first $\zerosf$
      in $\mathsf{000}z=w$ is followed by an element of $A$, and the
      only element of $A$ that begins with $\mathsf{00}$ is
      $\mathsf{001}$, we have that $z=\onesf u$ for some
      $u\in\mathsf{\{0,1\}}^*$.  Thus $w=\mathsf{0001}u$.  Since the
      second $\zerosf$ in $\mathsf{0001}u=w$ is followed by an element
      of $A$, and $\mathsf{011}$ is the only element of $A$ that
      begins with $\mathsf{01}$, we have that $u=\onesf v$ for some
      $v\in\mathsf{\{0,1\}}^*$.  Thus $w=\mathsf{00011}v$.  Since the
      third $\zerosf$ in $\mathsf{00011}v=w$ is followed by an element
      of $A$, and $\mathsf{111}$ is the only element of $A$ that
      begins with $\mathsf{11}$, we have that $v=\onesf t$ for some
      $t\in\mathsf{\{0,1\}}^*$.  Thus $w=\mathsf{000111}t$.  Since
      $\mathsf{00}\in X$, we have that
      $\mathsf{000111}=\mathsf{(0)(00)(111)}\in\{\zerosf\}X\{\mathsf{111}\}\sub
      Y$.  Because $t$ is a suffix of $w$, it follows that $t\in B$.
      Thus the inductive hypothesis tells us that $t\in Y^*$.  Hence
      $w=\mathsf{(000111)}t\in YY^*\sub Y^*$.
    \item Suppose $y=\onesf z$ for some $z\in\mathsf{\{0,1\}}^*$.
      Thus $w=\mathsf{00}y=\mathsf{001}z$.  Since the first $\zerosf$
      in $\mathsf{001}z=w$ is followed by an element of $A$, and the
      only element of $A$ that begins with $\mathsf{01}$ is
      $\mathsf{011}$, we have that $z=\onesf u$ for some
      $u\in\mathsf{\{0,1\}}^*$.  Thus $w=\mathsf{0011}u$.  Since the
      second $\zerosf$ in $\mathsf{0011}u=w$ is followed by an element
      of $A$, and $\mathsf{111}$ is the only element of $A$ that
      begins with $\mathsf{11}$, we have that $u=\onesf v$ for some
      $v\in\mathsf{\{0,1\}}^*$.  Thus $w=\mathsf{00111}v$.  Since
      $\mathsf{0}\in X$, we have that
      $\mathsf{00111}=\mathsf{(0)(0)(111)}\in\{\zerosf\}X\{\mathsf{111}\}\sub
      Y$.  Because $v$ is a suffix of $w$, it follows that $v\in B$.
      Thus the inductive hypothesis tells us that $v\in Y^*$.  Hence
      $w=\mathsf{(00111)}v\in YY^*\sub Y^*$.
    \end{itemize}
  \item Suppose $x=\onesf y$ for some $y\in\mathsf{\{0,1\}}^*$.  Thus
    $w=\zerosf x=\mathsf{01}y$.  Since $w\in B$, we have that
    $y\neq\%$.  Thus, there are two cases to consider.
    \begin{itemize}
    \item Suppose $y=\zerosf z$ for some $z\in\mathsf{\{0,1\}}^*$.
      Thus $w=\mathsf{010}z$.  Let $u$ be the longest prefix of $z$
      that is in $\{\mathsf{10}\}^*$.  (Since $\%$ is a prefix of $z$
      and is in $\{\mathsf{10}\}^*$, it follows that $u$ is
      well-defined.)  Let $v\in\mathsf{\{0,1\}}^*$ be such that
      $z=uv$.  Thus $w=\mathsf{010}z=\mathsf{010}uv$.

      Suppose, toward a contradiction, that $v$ begins with
      $\mathsf{10}$.  Then $u\mathsf{10}$ is a prefix of $z=uv$ that
      is longer than $u$.  Furthermore
      $u\mathsf{10}\in\{\mathsf{10}\}^*\{\mathsf{10}\}\sub
      \{\mathsf{10}\}^*$, contradicting the definition of $u$.  Thus
      we have that $v$ does not begin with $\mathsf{10}$.

      Next, we show that $\mathsf{010}u$ ends with $\mathsf{010}$.
      Since $u\in\{\mathsf{10}\}^*$, we have that
      $u\in\{\mathsf{10}\}^n$ for some $n\in\nats$.  There are three
      cases to consider.
      \begin{itemize}
      \item Suppose $n=0$.  Since $u\in\{\mathsf{10}\}^0=\{\%\}$, we
        have that $u=\%$.  Thus $\mathsf{010}u=\mathsf{010}$ ends with
        $\mathsf{010}$.
      \item Suppose $n=1$.  Since
        $u\in\{\mathsf{10}\}^1=\{\mathsf{10}\}$, we have that
        $u=\mathsf{10}$.  Hence $\mathsf{010}u= \mathsf{01010}$ ends
        with $\mathsf{010}$.
      \item Suppose $n\geq 2$.  Then $n-2\geq 0$, so that
        $u\in\{\mathsf{10}\}^{(n-2)+2}=\{\mathsf{10}\}^{n-2}\{\mathsf{10}\}^2$.
        Hence $u$ ends with $\mathsf{1010}$, showing that
        $\mathsf{010}u$ ends with $\mathsf{010}$.
      \end{itemize}
      Summarizing, we have that $w=\mathsf{010}uv$,
      $u\in\{\mathsf{10}\}^*$, $\mathsf{010}u$ ends with
      $\mathsf{010}$, and $v$ does not begin with $\mathsf{10}$.
      Since the second-to-last $\zerosf$ in $\mathsf{010}u$ is
      followed in $w$ by an element of $A$, and $\mathsf{101}$ is the
      only element of $A$ that begins with $\mathsf{10}$, we have that
      $v=\onesf s$ for some $s\in\mathsf{\{0,1\}}^*$.  Thus
      $w=\mathsf{010}u\onesf s$, and $\mathsf{010}u\onesf$ ends with
      $\mathsf{0101}$.  Since the second-to-last symbol of
      $\mathsf{010}u\onesf$ is a $\zerosf$, we have that $s\neq\%$.
      Furthermore, $s$ does not begin with $\zerosf$, since, if it
      did, then $v=\onesf s$ would begin with $\mathsf{10}$.  Thus we
      have that $s=\onesf t$ for some $t\in\mathsf{\{0,1\}}^*$.  Hence
      $w=\mathsf{010}u\mathsf{11}t$.  Since $\mathsf{010}u\mathsf{11}$
      ends with $\mathsf{011}$, it follows that the last $\zerosf$ in
      $\mathsf{010}u\mathsf{11}$ must be followed in $w$ by an element
      of $A$.  Because $\mathsf{111}$ is the only element of $A$ that
      begins with $\mathsf{11}$, we have that $t=\onesf r$ for some
      $r\in \mathsf{\{0,1\}}^*$.  Thus $w=\mathsf{010}u\mathsf{111}r$.
      Since $(\mathsf{10})u\in\{\mathsf{10}\}\{\mathsf{10}\}^*\sub
      \{\mathsf{10}\}^*\sub X$, we have that
      $\mathsf{010}u\mathsf{111}=(\zerosf)((\mathsf{10})u)\mathsf{111}\in
      \{\zerosf\}X\{\mathsf{111}\}\sub Y$.  Since $r$ is a suffix of
      $w$, it follows that $r\in B$.  Thus, the inductive hypothesis
      tells us that $r\in Y^*$.  Hence
      $w=(\mathsf{010}u\mathsf{111})r\in YY^*\sub Y^*$.
    \item Suppose $y=\onesf z$ for some $z\in\mathsf{\{0,1\}}^*$.
      Thus $w=\mathsf{011}z$.  Since the first $\zerosf$ of $w$ is
      followed by an element of $A$, and $\mathsf{111}$ is the only
      element of $A$ that begins with $\mathsf{11}$, we have that
      $z=\onesf u$ for some $u\in\mathsf{\{0,1\}}^*$.  Thus
      $w=\mathsf{0111}u$.  Since $\%\in\{\mathsf{10}\}^*\sub X$, we
      have that
      $\mathsf{0111}=\mathsf{(0)(\%)(111)}\in\{\zerosf\}X\{\mathsf{111}\}\sub
      Y$.  Because $u$ is a suffix of $w$, it follows that $u\in B$.
      Thus, since $u$ is a proper substring of $w$, the inductive
      hypothesis tells us that $u\in Y^*$.  Hence
      $w=\mathsf{(0111)}u\in YY^*\sub Y^*$.
    \end{itemize}
  \end{itemize}

\item Suppose $w=\onesf x$ for some $x\in\mathsf{\{0,1\}}^*$.  Since
  $x$ is a suffix of $w$, we have that $x\in B$.  Because $x$ is a
  proper substring of $w$, the inductive hypothesis tells us that
  $x\in Y^*$.  Thus $w=\onesf x\in YY^*\sub Y^*$.
\end{itemize}
\end{proof}

By Lemmas~\ref{RegSyn2Lem3} and \ref{RegSyn2Lem5}, we have that $Y^*\sub
B\sub Y^*$, so that $Y^*=B$.  This completes our regular expression
design and proof of correctness example.

\begin{exercise}
Let $X=\setof{w\in\{\mathsf{0,1}\}^*}{\mathsf{010}\eqtxt{is not a
    substring of}w}$.  Find a regular expression $\alpha$ such that
$L(\alpha)=X$, and prove that your answer is correct.
\end{exercise}

\begin{exercise}
Define $\diff\in\{\mathsf{0,1}\}^*\fun\ints$ as in
Section~\ref{UsingInductionToProveLanguageEqualities}, so
that, for all $w\in\{\mathsf{0,1}\}^*$,
\begin{displaymath}
\diff\,w =
\eqtxtr{the number of $\mathsf{1}$'s in}w -
\eqtxtr{the number of $\mathsf{0}$'s in}w .
\end{displaymath}
Thus $\diff\,\%=0$, $\diff\,\zerosf = -1$, $\diff\,\onesf = 1$, and
for all $x,y\in\{\mathsf{0,1}\}^*$, $\diff(xy) = \diff\,x + \diff\,y$.
Let $X=\setof{w\in\{\mathsf{0,1}\}^*}{\diff\,w=3m,\eqtxt{for some}m\in\ints}$.
Find a regular expression $\alpha$ such that $L(\alpha)=X$, and
prove that your answer is correct.
\end{exercise}

\begin{exercise}
Define a function $\diff\in\{\mathsf{0,1}\}^*\fun\ints$ by:
for all $w\in\{\mathsf{0,1}\}^*$,
\begin{displaymath}
\diff\,w =
\eqtxtr{the number of $\mathsf{0}$'s in}w -
2(\eqtxtr{the number of $\mathsf{1}$'s in}w) .
\end{displaymath}
Thus $\diff\,w = 0$ iff $w$ has twice as many $\onesf$'s as
$\zerosf$'s.  Furthermore $\diff\,\%=0$, $\diff\,\zerosf = 1$,
$\diff\,\onesf = -2$, and, for all $x,y\in\{\mathsf{0,1}\}^*$,
$\diff(xy) = \diff\,x + \diff\,y$.  Let
$X=\setof{w\in\{\mathsf{0,1}\}^*}{\diff\,w=0\eqtxt{and, for all
    prefixes} v\eqtxt{of}w,\,0\leq\diff\,v\leq 3}$.  Find a regular
expression $\alpha$ such that $L(\alpha)=X$, and prove that your
answer is correct.
\end{exercise}

\index{regular expression!design)}%
\index{regular expression!proof of correctness)}%

\subsection{Notes}

Our approach in this section is somewhat more formal than is common, but
is otherwise standard.

%%% Local Variables: 
%%% mode: latex
%%% TeX-master: "book"
%%% End: 

\section{Simplification of Regular Expressions}
\label{SimplificationOfRegularExpressions}

\index{regular expression!simplification|(}%
\index{simplification!regular expression|(}%
In this section, we give three algorithms---of increasing power, but
decreasing efficiency---for regular expression simplification.  The
first algorithm---weak simplification---is defined via a
straightforward structural recursion, and is sufficient for many
purposes.  The remaining two algorithms---local simplification and
global simplification---are based on a set of simplification rules
that is still incomplete and evolving.

\subsection{Regular Expression Complexity}

To begin with, let's consider how we might measure the
complexity/simplicity of regular expressions.  The most obvious
criterion is size (remember that regular expressions are trees).
But consider this pair of equivalent regular expressions:
\begin{align*}
\alpha &= \mathsf{(00^*11^*)^*} , \eqtxt{and} \\
\beta &= \mathsf{\% + 0(0 + 11^*0)^*11^*} .
\end{align*}
Although the size of $\beta$ ($18$) is strictly greater than the
size of $\alpha$ ($10$), $\beta$ has only one closure inside another
closure, whereas $\alpha$ has two closures inside its outer closure,
and thus there is a sense in which $\beta$ is easier to understand
than $\alpha$.

The standard measure of the closure-related complexity
of a regular expression is its \emph{star-height}: the maximum number
$n\in\nats$ such that there is a path from the root of the regular expression
to one of its leaves that passes through $n$ closures.  But $\alpha$ and
$\beta$ both have star-heights of $2$.  Furthermore, star-height
isn't respected by the ways of forming regular expressions.  E.g.,
if $\gamma_1$ has strictly smaller star-height than $\gamma_2$,
we can't conclude that $\gamma_1\gamma'$ has strictly smaller star-height
than $\gamma_2\gamma'$, as the star height of $\gamma'$ may be
greater than the star-height of $\gamma_2$.

So, we need a better measure of the closure-related complexity of
regular expressions than star-height.  Toward that end, let's define a
\emph{closure complexity} to be a nonempty list $\ns$ of natural
\index{closure complexity}%
\index{regular expression!closure complexity}%
numbers that is (not-necessarily strictly) descending: for all
$i\in[1:|\ns|-1]$, $\ns\,i\geq \ns(i+1)$.  This is a way of
representing nonempty multisets of natural numbers that makes it easy
to define the usual ordering on multisets. We write $\CC$ for the set
of all closure complexities. E.g., $[3, 2, 2, 1]$ is a closure
complexity, but $[3, 2, 3]$ and $[\,]$ are not.  For all $n\in\nats$,
$[n]$ is a \emph{singleton} closure complexity.  The \emph{union} of
closure complexities $\ns$ and $\ms$ ($\ns\cup\ms$) is the closure
complexity that results from putting $\ns\myconcat\ms$ in descending
order, keeping any duplicate elements.  (Here we are overloading the
term union and the operation $\cup$, but the set-theoretic union isn't
an operation on closure complexities, and so no confusion should
result.)  E.g., $[3,2,2,1]\cup[4,2,1,0] = [4,3,2,2,2,1,1,0]$.  The
\emph{successor} $\overline{\ns}$ of a closure complexity $\ns$ is the
closure complexity formed by adding one to each element of $\ns$,
maintaining the order of the elements.  E.g.,
$\overline{[3,2,2,1]} = [4,3,3,2]$.

It is easy to see that $\cup$ is commutative and associative on $\CC$,
and that the successor operation on $\CC$ preserves union:

\begin{proposition}
\begin{enumerate}[\quad(1)]
\item For all $\ns,\ms\in\CC$, $\ns\cup\ms = \ms\cup\ns$.

\item For all $\ns,\ms,\ls\in\CC$, $(\ns\cup\ms)\cup\ls = \ns\cup(\ms\cup\ls)$.

\item For all $\ns,\ms\in\CC$, $\overline{\ns\cup\ms} =
  \overline{\ns}\cup\overline{\ms}$.
\end{enumerate}
\end{proposition}

\begin{proposition}
\begin{enumerate}[\quad(1)]
\label{CCEquivContext}
\item For all $\ns,\ms\in\CC$, $\overline{\ns} = \overline{\ms}$ iff
  $\ns=\ms$.

\item For all $\ns,\ms,\ls\in\CC$, $\ns\cup\ls = \ms\cup\ls$ iff
  $\ns = \ms$.
\end{enumerate}
\end{proposition}

We define a relation $\ltcc$ on $\CC$ by: for all $\ns,\ms\in\CC$,
$\ns\ltcc\ms$ iff either:
\begin{itemize}
\item $\ms = \ns \myconcat \ls$ for some $\ls\in\CC$; or

\item there is an $i\in\nats - \{0\}$ such that
  \begin{itemize}
  \item $i\leq|\ns|$ and $i\leq|\ms|$,

  \item for all $j\in[1:i-1]$, $\ns\,j = \ms\,j$, and

  \item $\ns\,i < \ms\,i$.
  \end{itemize}
\end{itemize}
In other words, $ns\ltcc\ms$ iff either $\ms$ consists of the
result of appending a nonempty list at the end of $\ns$, or
$\ns$ and $\ms$ agree up to some point, at which $\ns$'s value
is strictly smaller than $\ms$'s value.  E.g.,
$[2, 2]\leqcc[2, 2, 1]$ and $[2,1,1,0,0]\ltcc[2, 2, 1]$.

\begin{proposition}
\label{CCLTContext}
\begin{enumerate}[\quad(1)]
\item For all $\ns,\ms\in\CC$, $\overline{\ns} \ltcc \overline{\ms}$ iff
  $\ns\ltcc\ms$.

\item For all $\ns,\ms,\ls\in\CC$, $\ns\cup\ls \ltcc \ms\cup\ls$ iff
  $\ns \ltcc \ms$.

\item For all $\ns,\ms\in\CC$, $\ns\ltcc\ns\cup\ms$.
\end{enumerate}
\end{proposition}

\begin{proposition}
$\ltcc$ is a strict total ordering on $\CC$.
\end{proposition}

\begin{proposition}
\label{LTCCWellFounded}
$\ltcc$ is a well-founded relation on $\CC$.
\end{proposition}

Now we can define the closure complexity of a regular expression.
Define the function $\cc\in\Reg\fun\CC$ by structural recursion:
\begin{align*}
\cc\,\% &= [0]; \\
\cc\,\$ &= [0]; \\
\cc\,a &= [0], \eqtxt{for all}a\in\Sym; \\
\cc({*}(\alpha)) &= \overline{\cc\,\alpha}, \eqtxt{for all}\alpha\in\Reg;\\
\cc(@(\alpha,\beta)) &= \cc\,\alpha\cup\cc\,\beta ,
\eqtxt{for all}\alpha,\beta\in\Reg; \eqtxt{and} \\
\cc({+}(\alpha,\beta)) &= \cc\,\alpha\cup\cc\,\beta,
\eqtxt{for all}\alpha,\beta\in\Reg .
\end{align*}
We say that $\cc\,\alpha$ is \emph{the closure complexity of} $\alpha$.
E.g.,
\begin{align*}
  \cc(\mathsf{(12^*)^*})
  &= \overline{\cc(\mathsf{12^*})} =
  \overline{\cc\,\mathsf{1} \cup \cc(\mathsf{2^*})} =
  \overline{[0] \cup \overline{\cc\,\twosf}} \\
  &= \overline{[0] \cup \overline{[0]}} =
  \overline{[0] \cup [1]} =
  \overline{[1, 0]} =
  [2, 1] .
\end{align*}
In other words, the $\cc\,\alpha$ can be computed by first collecting
together all the paths through $\alpha$ that terminate in leafs, then
counting the numbers of closures visited when following each of these
paths, and finally putting those sums in descending order.

Returning to our initial examples, we have that
$\cc(\mathsf{(00^*11^*)^*}) = [2,2,1,1]$ and $\cc(\% + \mathsf{0(0 +
  11^*0)^*11^*}) = [2,1,1,1,1,0,0,0]$.  Since
$[2,1,1,1,1,0,0,0]\ltcc[2,2,1,1]$, the closure
complexity of $\% + \mathsf{0(0 + 11^*0)^*11^*}$ is strictly smaller
than the closure complexity of $\mathsf{(00^*11^*)^*}$.

\begin{proposition}
For all $\alpha\in\Reg$, $|\cc\,\alpha| = \numLeaves\,\alpha$.
\end{proposition}

\begin{proof}
An easy induction on regular expressions.
\end{proof}

\begin{exercise}
Find regular expressions $\alpha$ and $\beta$ such that
$\cc\,\alpha = \cc\,\beta$ but $\mysize\,\alpha\neq\mysize\,\beta$.
\end{exercise}

In contrast to star-height, closure complexity is compatible with the
ways of forming regular expressions.  In fact, we can prove even
stronger results.

\begin{proposition}
\label{RegCCEquiv}
\begin{enumerate}[\quad(1)]
\item For all $\alpha\in\Reg$, $\cc\,\alpha = \cc\,\beta$
  iff $\cc(\alpha^*) = \cc(\beta^*)$.

\item For all $\alpha,\beta,\gamma\in\Reg$, $\cc\,\alpha = \cc\,\beta$
  iff $\cc(\alpha\gamma) = \cc(\beta\gamma)$.

\item For all $\alpha,\beta,\gamma\in\Reg$, $\cc\,\alpha = \cc\,\beta$
  iff $\cc(\gamma\alpha) = \cc(\gamma\beta)$.

\item For all $\alpha,\beta,\gamma\in\Reg$, $\cc\,\alpha = \cc\,\beta$
  iff $\cc(\alpha + \gamma) = \cc(\beta + \gamma)$.

\item For all $\alpha,\beta,\gamma\in\Reg$, $\cc\,\alpha = \cc\,\beta$
  iff $\cc(\gamma + \alpha) = \cc(\gamma + \beta)$.
\end{enumerate}
\end{proposition}

\begin{proof}
Follows by Proposition~\ref{CCEquivContext}.
\end{proof}

The following proposition says that if we replace a subtree of
a regular expression by a regular expression with the same closure
complexity, then the closure complexity of the
resulting, whole regular expression will be unchanged.

\begin{proposition}
\label{RegCCEquivSubstituteSubtree}

Suppose $\alpha,\beta,\beta'\in\Reg$, $\cc\,\beta = \cc\,\beta'$,
$\pat\in\Path$ is valid for $\alpha$, and $\beta$ is
the subtree of $\alpha$ at position $\pat$.
Let $\alpha'$ be the result of replacing the subtree at
position $\pat$ in $\alpha$ by $\beta'$. Then $\cc\,\alpha =
\cc\,\alpha'$.
\end{proposition}

\begin{proof}
By induction on $\alpha$ using Proposition~\ref{RegCCEquiv}.
\end{proof}

\begin{proposition}
\label{RegCCLTContext}
\begin{enumerate}[\quad(1)]
\item For all $\alpha\in\Reg$, $\cc\,\alpha\ltcc\cc\,\beta$
  iff $\cc(\alpha^*)\ltcc\cc(\beta^*)$.

\item For all $\alpha,\beta,\gamma\in\Reg$, $\cc\,\alpha\ltcc\cc\,\beta$
  iff $\cc(\alpha\gamma)\ltcc\cc(\beta\gamma)$.

\item For all $\alpha,\beta,\gamma\in\Reg$, $\cc\,\alpha\ltcc\cc\,\beta$
  iff $\cc(\gamma\alpha)\ltcc\cc(\gamma\beta)$.

\item For all $\alpha,\beta,\gamma\in\Reg$, $\cc\,\alpha\ltcc\cc\,\beta$
  iff $\cc(\alpha + \gamma)\ltcc\cc(\beta + \gamma)$.

\item For all $\alpha,\beta,\gamma\in\Reg$, $\cc\,\alpha\ltcc\cc\,\beta$
  iff $\cc(\gamma + \alpha)\ltcc\cc(\gamma + \beta)$.
\end{enumerate}
\end{proposition}

\begin{proof}
Follows by Proposition~\ref{CCLTContext}.
\end{proof}

The following proposition says that if we replace a subtree of a regular
expression by a regular expression with strictly smaller closure
complexity, that the resulting, whole regular expression
will have strictly smaller closure complexity than the original
regular expression.

\begin{proposition}
\label{RegCCLTSubstituteSubtree}

Suppose $\alpha,\beta,\beta'\in\Reg$, $\cc\,\beta'\ltcc\cc\,\beta$,
$\pat\in\Path$ is valid for $\alpha$, and $\beta$ is
the subtree of $\alpha$ at position $\pat$.
Let $\alpha'$ be the result of replacing the subtree at
position $\pat$ in $\alpha$ by $\beta'$. Then $\cc\,\alpha'\ltcc
\cc\,\alpha$.
\end{proposition}

\begin{proof}
By induction on $\alpha$, using Proposition~\ref{RegCCLTContext}.
\end{proof}

When judging the relative complexity of regular expressions $\alpha$
and $\beta$, we will first look at how their closure complexities are
related.  And, when their closure complexities are equal, we will
look at how their sizes are related.  To finish explaining how
we will judge the relative complexity of regular expressions, we
need three definitions.

The function
\index{numConcats@$\numConcats$}%
\index{regular expression!numConcats@$\numConcats$}%
\index{regular expression!number of concatenations}%
\begin{gather*}
\numConcats\in\Reg\fun\nats
\end{gather*}
is defined by recursion:
\begin{align*}
\numConcats\,\% &= 0 ; \\
\numConcats\,\$ &= 0 ; \\
\numConcats\,a &= 0, \eqtxt{for all}a\in\Sym ; \\
\numConcats(\alpha^*) &= \numConcats\,\alpha , \eqtxt{for all} \alpha\in\Reg ; \\
\numConcats(\alpha\beta) &= 1 + \numConcats\,\alpha + \numConcats\,\beta ;
\eqtxt{and}\\
\numConcats(\alpha + \beta) &= \numConcats\,\alpha + \numConcats\,\beta .
\end{align*}
Thus $\numConcats\,\alpha$ is the number of concatenations in
$\alpha$, i.e., the number of subtrees of $\alpha$ that are
concatenations, where a given concatenation may occur (and will be
counted) multiple times.  E.g.,
$\numConcats(\mathsf{((01)^*(01))^*}) = 3$.  The function
\index{numSyms@$\numSyms$}%
\index{regular expression!numSyms@$\numSyms$}%
\index{regular expression!number of symbols}%
\begin{gather*}
\numSyms\in\Reg\fun\nats
\end{gather*}
is defined by structural recursion:
\begin{align*}
\numSyms\,\% &= 0 ; \\
\numSyms\,\$ &= 0 ; \\
\numSyms\,a &= 1, \eqtxt{for all}a\in\Sym ; \\
\numSyms(\alpha^*) &= \numSyms\,\alpha , \eqtxt{for all} \alpha\in\Reg ; \\
\numSyms(\alpha\beta) &= \numSyms\,\alpha + \numSyms\,\beta ; \eqtxt{and}\\
\numSyms(\alpha + \beta) &= \numSyms\,\alpha + \numSyms\,\beta .
\end{align*}
Thus $\numSyms\,\alpha$ is the number of occurrences of symbols in $\alpha$,
where a given symbol may occur (and will be counted) more than once.
E.g., $\numSyms(\mathsf{(0^*1)+0}) = 3$.

Finally, we say that a regular expression $\alpha$ is \emph{standardized}
\index{standardized}%
\index{regular expression!standardized}%
iff none of $\alpha$'s subtrees have any of the following forms:
\begin{itemize}
\item $(\beta_1+\beta_2)+\beta_3$ (we can avoid needing parentheses,
  and make a regular expression easier to understand/process from
  left-to-right, by grouping unions to the right);

\item $\beta_1+\beta_2$, where $\beta_1>\beta_2$, or
  $\beta_1+(\beta_2+\beta_3)$, where $\beta_1>\beta_2$ (it's pleasing
  if the regular expressions appear in order (recall that unions
  are greater than all other kinds of regular expressions));

\item $(\beta_1\beta_2)\beta_3$ (we can avoid needing parentheses, and
  make a regular expression easier to understand/process from
  left-to-right, by grouping concatenations to the right); and

\item $\beta^*\beta$, $\beta^*(\beta\gamma)$,
  $(\beta_1\beta_2)^*\beta_1$ or $(\beta_1\beta_2)^*\beta_1\gamma$
  (moving closures to the right makes a regular expression easier to
  understand/process from left-to-right).
\end{itemize}
Thus every subtree of a standardized regular expression will be standardized.

Returning to our assessment of regular expression complexity, suppose
that $\alpha$ and $\beta$ are regular expressions generating $\%$.
Then $(\alpha\beta)^*$ and $(\alpha+\beta)^*$ are equivalent, but will
will prefer the latter over the former, because unions are generally
more amenable to understanding and processing than concatenations.
Consequently, when two regular expression have the same closure
complexity and size, we will judge their relative complexity
according to their numbers of concatenations.

Next, consider the regular expressions $\mathsf{0 + 01}$ and
$\mathsf{0(\% + 1)}$.  These regular expressions have the same closure
complexity $[0,0,0]$, size ($5$) and number of concatenations ($1$).
We would like to consider the latter to be simpler than the former,
since in general we would like to prefer $\alpha(\%+\beta)$ over
$\alpha + \alpha\beta$.  And we can base this preference on the fact
that the number of symbols of $\mathsf{0(\% + 1)}$ ($2$) is one less
than the number of symbols of $\mathsf{0 + 01}$.  When regular
expressions have the same closure complexity, size and number of
concatenations, the one with fewer symbols is likely to be easier to
understand and process.  Thus, when regular expressions have
identical closure complexity, size and number of concatenations, we
will use their relative numbers of symbols to judge their relative
complexity.

Finally, when regular expressions have the same closure complexity,
size, number of concatenations, and number of symbols, we will judge
their relative complexity according to whether they are standardized,
thinking that a standardized regular expression is simpler than one
that is not standardized.

We define a relation $\ltsimp$ on $\Reg$ by, for all $\alpha,\beta\in\Reg$,
$\alpha\ltsimp\beta$ iff:
\begin{itemize}
\item $\cc\,\alpha \ltcc \cc\,\beta$; or

\item $\cc\,\alpha = \cc\,\beta$ but $\mysize\,\alpha < \mysize\,\beta$; or

\item $\cc\,\alpha = \cc\,\beta$ and $\mysize\,\alpha = \mysize\,\beta$,
  but $\numConcats\,\alpha < \numConcats\,\beta$; or

\item $\cc\,\alpha = \cc\,\beta$, $\mysize\,\alpha = \mysize\,\beta$
  and $\numConcats\,\alpha = \numConcats\,\beta$, but
  $\numSyms\,\alpha < \numSyms\,\beta$; or

\item $\cc\,\alpha = \cc\,\beta$, $\mysize\,\alpha = \mysize\,\beta$,
  $\numConcats\,\alpha = \numConcats\,\beta$ and $\numSyms\,\alpha =
  \numSyms\,\beta$, but $\alpha$ is standardized and $\beta$ is
  not standardized.
\end{itemize}

We read $\alpha\ltsimp\beta$ as $\alpha$ is \emph{simpler} (less
\emph{complex}) than $\beta$.  We define a relation $\equivsimp$ on
$\Reg$ by, for all $\alpha,\beta\in\Reg$, $\alpha\equivsimp\beta$ iff
$\alpha$ and $\beta$ have the same closure complexity, size, numbers
of concatenations, numbers of symbols, and status of being (or not
being) standardized.  We read $\alpha\equivsimp\beta$ as $\alpha$ and
$\beta$ have the \emph{same complexity}.  Finally, we define a
relation $\leqsimp$ on $\Reg$ by, for all $\alpha,\beta\in\Reg$,
$\alpha\leqsimp\beta$ iff $\alpha\ltsimp\beta$ or
$\alpha\equivsimp\beta$.  We read $\alpha\leqsimp\beta$ as $\alpha$ is
at least as simpler as (no more complex) than $\beta$

For example, the following regular expressions are equivalent and have
the same complexity:
\begin{displaymath}
\mathsf{1(01 + 10) + (\% + 01)1} \quad\eqtxt{and}\quad
\mathsf{011 + 1(\% + 01 + 10)} .  
\end{displaymath}

\begin{proposition}
\begin{enumerate}[\quad(1)]
\item $\ltsimp$ is transitive.

\item $\equivsimp$ is reflexive on $\Reg$, transitive and symmetric.

\item For all $\alpha,\beta\in\Reg$, exactly one of the following holds:
$\alpha\ltsimp\beta$, $\beta\ltsimp\alpha$ or $\alpha\equivsimp\beta$.

\item $\leqsimp$ is transitive, and, for all $\alpha,\beta\in\Reg$,
$\alpha\equivsimp\beta$ iff $\alpha\leq\beta$ and $\beta\leq\alpha$.
\end{enumerate}
\end{proposition}

The Forlan module \texttt{Reg} defines the abstract type \texttt{cc}
of closure complexities, along with these functions:
\begin{verbatim}
val ccToList  : cc -> int list
val singCC    : int -> cc
val unionCC   : cc * cc -> cc
val succCC    : cc -> cc
val cc        : reg -> cc
val compareCC : cc * cc -> order
\end{verbatim}
The function \texttt{ccToList} is the identity function on closure
complexities: all that changes is the type.  $\mathtt{singCC\,n}$
returns the singleton closure complexity $[n]$, if $n$ is nonnegative;
otherwise it issues an error message.  The functions \texttt{unionCC}
and \texttt{succCC} implement the union and successor operations on
closure complexities.  The function \texttt{cc} corresponds to $\cc$, and
\texttt{compareCC} implements $\ltcc$.

Here are some examples of how these functions can be used:
\begin{list}{}
{\setlength{\leftmargin}{\leftmargini}
\setlength{\rightmargin}{0cm}
\setlength{\itemindent}{0cm}
\setlength{\listparindent}{0cm}
\setlength{\itemsep}{0cm}
\setlength{\parsep}{0cm}
\setlength{\labelsep}{0cm}
\setlength{\labelwidth}{0cm}
\catcode`\#=12
\catcode`\$=12
\catcode`\%=12
\catcode`\^=12
\catcode`\_=12
\catcode`\.=12
\catcode`\?=12
\catcode`\!=12
\catcode`\&=12
\ttfamily}
\small
\item[]\textsl{-\ }val\ ns\ =\ Reg.succCC(Reg.unionCC(Reg.singCC\ 1,\ Reg.singCC\ 1));
\item[]\textsl{val\ ns\ =\ -\ :\ Reg.cc}
\item[]\textsl{-\ }Reg.ccToList\ ns;
\item[]\textsl{val\ it\ =\ \symbol{'133}2,2\symbol{'135}\ :\ int\ list}
\item[]\textsl{-\ }val\ ms\ =\ Reg.unionCC(ns,\ Reg.succCC\ ns);
\item[]\textsl{val\ ms\ =\ -\ :\ Reg.cc}
\item[]\textsl{-\ }Reg.ccToList\ ms;
\item[]\textsl{val\ it\ =\ \symbol{'133}3,3,2,2\symbol{'135}\ :\ int\ list}
\item[]\textsl{-\ }Reg.ccToList(Reg.cc(Reg.fromString\ "(00\symbol{'052}11\symbol{'052})\symbol{'052}"));
\item[]\textsl{val\ it\ =\ \symbol{'133}2,2,1,1\symbol{'135}\ :\ int\ list}
\item[]\textsl{-\ }Reg.ccToList(Reg.cc(Reg.fromString\ "%\ +\ 0(0\ +\ 11\symbol{'052}0)\symbol{'052}11\symbol{'052}"));
\item[]\textsl{val\ it\ =\ \symbol{'133}2,1,1,1,1,0,0,0\symbol{'135}\ :\ int\ list}
\item[]\textsl{-\ }Reg.compareCC
\item[]\textsl{=\ }(Reg.cc(Reg.fromString\ "(00\symbol{'052}11\symbol{'052})\symbol{'052}"),
\item[]\textsl{=\ }\ Reg.cc(Reg.fromString\ "%\ +\ 0(0\ +\ 11\symbol{'052}0)\symbol{'052}11\symbol{'052}"));
\item[]\textsl{val\ it\ =\ GREATER\ :\ order}
\item[]\textsl{-\ }Reg.compareCC
\item[]\textsl{=\ }(Reg.cc(Reg.fromString\ "(00\symbol{'052}11\symbol{'052})\symbol{'052}"),
\item[]\textsl{=\ }\ Reg.cc(Reg.fromString\ "(1\symbol{'052}10\symbol{'052}0)\symbol{'052}"));
\item[]\textsl{val\ it\ =\ EQUAL\ :\ order}
\end{list}


The module \texttt{Reg} also includes these functions:
\begin{verbatim}
val numConcats             : reg -> int
val numSyms                : reg -> int
val standardized           : reg -> bool
val compareComplexity      : reg * reg -> order
val compareComplexityTotal : reg * reg -> order
\end{verbatim}
The first two functions implement the functions with the same names.
The function \texttt{standardized} tests whether a regular expression
is standardized, and the function \texttt{compareComplexity} implements
$\leqsimp$/$\equivsimp$. Finally, \texttt{compareComplexityTotal} is like
\texttt{compareComplexity}, but falls back on \texttt{Reg.compare}
(our total ordering on regular expressions) to order regular expressions
with the same complexity.  Thus \texttt{compareComplexityTotal} is
a total ordering.

Here are some examples of how these functions can be used:
\begin{list}{}
{\setlength{\leftmargin}{\leftmargini}
\setlength{\rightmargin}{0cm}
\setlength{\itemindent}{0cm}
\setlength{\listparindent}{0cm}
\setlength{\itemsep}{0cm}
\setlength{\parsep}{0cm}
\setlength{\labelsep}{0cm}
\setlength{\labelwidth}{0cm}
\catcode`\#=12
\catcode`\$=12
\catcode`\%=12
\catcode`\^=12
\catcode`\_=12
\catcode`\.=12
\catcode`\?=12
\catcode`\!=12
\catcode`\&=12
\ttfamily}
\small
\item[]\textsl{-\ }Reg.numConcats(Reg.fromString\ "(01)\symbol{'052}(10)\symbol{'052}");
\item[]\textsl{val\ it\ =\ 3\ :\ int}
\item[]\textsl{-\ }Reg.numSyms(Reg.fromString\ "(01)\symbol{'052}(10)\symbol{'052}");
\item[]\textsl{val\ it\ =\ 4\ :\ int}
\item[]\textsl{-\ }Reg.standardized(Reg.fromString\ "00\symbol{'052}1");
\item[]\textsl{val\ it\ =\ true\ :\ bool}
\item[]\textsl{-\ }Reg.standardized(Reg.fromString\ "00\symbol{'052}0");
\item[]\textsl{val\ it\ =\ false\ :\ bool}
\item[]\textsl{-\ }Reg.compareComplexity
\item[]\textsl{=\ }(Reg.fromString\ "(00\symbol{'052}11\symbol{'052})\symbol{'052}",
\item[]\textsl{=\ }\ Reg.fromString\ "%\ +\ 0(0\ +\ 11\symbol{'052}0)\symbol{'052}11\symbol{'052}");
\item[]\textsl{val\ it\ =\ GREATER\ :\ order}
\item[]\textsl{-\ }Reg.compareComplexity
\item[]\textsl{=\ }(Reg.fromString\ "0\symbol{'052}\symbol{'052}1\symbol{'052}\symbol{'052}",\ \ Reg.fromString\ "(01)\symbol{'052}\symbol{'052}");
\item[]\textsl{val\ it\ =\ GREATER\ :\ order}
\item[]\textsl{-\ }Reg.compareComplexity
\item[]\textsl{=\ }(Reg.fromString\ "(0\symbol{'052}1\symbol{'052})\symbol{'052}",\ \ Reg.fromString\ "(0\symbol{'052}+1\symbol{'052})\symbol{'052}");
\item[]\textsl{val\ it\ =\ GREATER\ :\ order}
\item[]\textsl{-\ }Reg.compareComplexity
\item[]\textsl{=\ }(Reg.fromString\ "0+01",\ \ Reg.fromString\ "0(%+1)");
\item[]\textsl{val\ it\ =\ GREATER\ :\ order}
\item[]\textsl{-\ }Reg.compareComplexity
\item[]\textsl{=\ }(Reg.fromString\ "(01)2",\ \ Reg.fromString\ "012");
\item[]\textsl{val\ it\ =\ GREATER\ :\ order}
\item[]\textsl{-\ }Reg.compareComplexity
\item[]\textsl{=\ }(Reg.fromString\ "1(01+10)+(%+01)1",
\item[]\textsl{=\ }\ Reg.fromString\ "011+1(%+01+10)");
\item[]\textsl{val\ it\ =\ EQUAL\ :\ order}
\end{list}


\subsection{Weak Simplification}

In this subsection, we give our first simplification algorithm: weak
simplification.  We say that a regular expression $\alpha$ is
\emph{weakly simplified}
\index{weakly simplified}%
\index{regular expression!weakly simplified}%
\index{simplification!regular expression!weakly simplified}%
iff $\alpha$ is standardized and none of $\alpha$'s subtrees have
any of the following forms:
\begin{itemize}
\item $\$+\beta$ or $\beta+\$$ (the $\$$ is redundant);

\item $\beta+\beta$ or $\beta+(\beta+\gamma)$ (the duplicate occurrence
  of $\beta$ is redundant);

\item $\%\beta$ or $\beta\%$ (the $\%$ is redundant);

\item $\$\beta$ or $\beta\$$ (both are equivalent to $\$$); and

\item $\%^*$ or $\$^*$ or $(\beta^*)^*$ (the first two can be replaced
  by $\%$, and the extra closure can be omitted in the third case).
\end{itemize}
Thus, if a regular expression $\alpha$ is weakly simplified,
then each of its subtrees will also be weakly simplified.

Weakly simplified regular expressions have some pleasing properties.

\begin{proposition}
\label{WeakSimpProp3}
\begin{enumerate}[\quad(1)]
\item For all $\alpha\in\Reg$, if $\alpha$ is weakly simplified and
  $L(\alpha)=\emptyset$, then $\alpha=\$$.

\item For all $\alpha\in\Reg$, if $\alpha$ is weakly simplified and
  $L(\alpha)=\{\%\}$, then $\alpha=\%$.

\item For all $\alpha\in\Reg$, for all $a\in\Sym$, if $\alpha$ is
  weakly simplified and $L(\alpha)=\{a\}$, then $\alpha=a$.
\end{enumerate}
\end{proposition}

E.g., part~(2) of the proposition says that, if $\alpha$
is weakly simplified and $L(\alpha)$ is the language whose only
string is $\%$, then $\alpha$ is $\%$.

\begin{proof}
The three parts are proved in order, using induction on regular
expressions.  We will show the concatenation case of part~(3).
Suppose $\alpha,\beta\in\Reg$ and assume the inductive hypothesis: for
all $a\in\Sym$, if $\alpha$ is weakly simplified and
$L(\alpha)=\{a\}$, then $\alpha=a$, and for all $a\in\Sym$, if $\beta$
is weakly simplified and $L(\beta)=\{a\}$, then $\beta=a$.  Suppose
$a\in\Sym$, and assume that $\alpha\beta$ is weakly simplified and
$L(\alpha\beta)=\{a\}$.  We must show that $\alpha\beta=a$.
Because $\alpha\beta$ is weakly simplified, so are $\alpha$ and $\beta$.

Since $L(\alpha)L(\beta)=L(\alpha\beta)=\{a\}$, there are two cases to
consider.
\begin{itemize}
\item Suppose $L(\alpha)=\{a\}$ and $L(\beta)=\{\%\}$.  Since $\beta$
  is weakly simplified and $L(\beta)=\{\%\}$, part~(2) tells us that
  $\beta=\%$.  But this means that $\alpha\beta=\alpha\%$ is not
  weakly simplified after all---contradiction.  Thus we can conclude
  that $\alpha\beta=a$.

\item Suppose $L(\alpha)=\{\%\}$ and $L(\beta)=\{a\}$.  The proof of
  this case is similar to that of the other one.
\end{itemize}
\end{proof}

\begin{proposition}
\label{WeakSimpProp2}
For all $\alpha\in\Reg$, if $\alpha$ is weakly simplified, then
$\alphabet(L(\alpha)) = \alphabet\,\alpha$.
\end{proposition}

\begin{proof}
By Proposition~\ref{AlphabetRegMeaning}, it suffices to show that,
for all $\alpha\in\Reg$, if $\alpha$ is weakly simplified, then
$\alphabet\,\alpha\sub\alphabet(L(\alpha))$.  And this
follows by an easy induction on $\alpha$, using
Proposition~\ref{WeakSimpProp3}(2).
\end{proof}

The next proposition says that $\$$ need only be used at the 
top-level of a regular expression.

\begin{proposition}
\label{WeakSimpProp4}
For all $\alpha\in\Reg$, if $\alpha$ is weakly simplified and $\alpha$ has
one or more occurrences of $\$$, then $\alpha=\$$.
\end{proposition}

\begin{proof}
An easy induction on regular expressions.
\end{proof}

Finally, we have that weakly simplified regular expressions with
closures generate infinite languages:

\begin{proposition}
For all $\alpha\in\Reg$, if $\alpha$ is weakly simplified and
$\alpha$ has one or more closures, then $L(\alpha)$ is
infinite.
\end{proposition}

\begin{proof}
An easy induction on regular expressions.
\end{proof}

Next, we see how we can test whether a regular expression is weakly
simplified via a simple structural recursion.  Define
$\weaklySimplified\in\Reg\fun\Bool$ by structural recursion, as
follows.  Given a regular expression $\alpha$, it proceeds as follows:
\begin{itemize}
\item Suppose $\alpha$ is $\%$, $\$$ or a symbol.  Then it returns $\true$.

\item Suppose $\alpha$ has the form $\beta^*$.  Then it checks that:
  \begin{itemize}
  \item $\beta$ is weakly simplified (this is done using recursion); and

  \item $\beta$ is neither $\%$, nor $\$$, nor a closure.
  \end{itemize}

\item Suppose $\alpha$ has the form $\alpha_1\,\alpha_2$.  Then it
  checks that:
  \begin{itemize}
  \item $\alpha_1$ and $\alpha_2$ are weakly simplified; and

  \item $\alpha_1$ is neither $\%$ nor $\$$ nor a concatenation; and

  \item $\alpha_2$ is neither $\%$ nor $\$$; and

  \item $\alpha$ has none of the following forms: $\beta^*\beta$,
    $\beta^*(\beta\gamma)$, $(\beta_1\beta_2)^*\beta_1$ or
    $(\beta_1\beta_2)^*\beta_1\gamma$.
  \end{itemize}
  
\item Suppose $\alpha$ has the form $\alpha_1 + \alpha_2$.  Then it
  checks that:
  \begin{itemize}
  \item $\alpha_1$ and $\alpha_2$ are weakly simplified; and

  \item $\alpha_1$ is neither $\$$ nor a union; and

  \item $\alpha_2$ is not $\$$;

  \item if $\alpha_2$ has the form $\beta_1 + \beta_2$, then
    $\alpha_1<\beta_1$; and

  \item if $\alpha_2$ is not a union, then $\alpha_1<\alpha_2$.
  \end{itemize}
\end{itemize}

\begin{proposition}
For all $\alpha\in\Reg$, $\alpha$ is weakly simplified iff
$\weaklySimplified\,\alpha = \true$.
\end{proposition}

\begin{proof}
By induction on regular expressions.
\end{proof}

In preparation for giving our weak simplification algorithm, we need
to define some auxiliary functions.  We say that a regular expression
$\alpha$ is \emph{almost weakly simplified} iff either:
\begin{itemize}
\item $w\in\{\%,\$\}$; or

\item all elements of $\concatsToList\,\alpha$ are weakly simplified,
  and are not $\%$, $\$$ or concatenations.
\end{itemize}

For example, $\mathsf{0^*0(1+2)^*(1+2)} =
\mathsf{0^*(0((1+2)^*(1+2)))}$ is almost weakly simplified, even
though it's not weakly simplified.  On the other hand:
$\mathsf{(\$+1)1}$ isn't almost weakly simplified, because
$\mathsf{\$+1}$ isn't weakly simplified; $\mathsf{1\%}$ isn't weakly
simplified, because of the location of $\%$; and $\mathsf{(01)1}$
isn't almost weakly simplified, because of the location of the
concatenation $\mathsf{01}$.

Let
\begin{align*}
  \WS &= \setof{\alpha\in\Reg}{\alpha\eqtxtl{is weakly simplified}},
  \eqtxt{and} \\
  \AWS &= \setof{\alpha\in\Reg}{\alpha\eqtxtl{is almost weakly simplified}} .
\end{align*}

We define a function $\shiftClosuresRight\in\AWS\fun\WS$ by recursion.
Given $\alpha\in\AWS$, $\shiftClosuresRight$ proceeds as follows.  If
$\alpha$ is not a concatenation, than it returns $\alpha$.
Otherwise, $\alpha = \alpha_1\alpha_2$ for some $\alpha_1,\alpha_2\in\Reg$.
Since $\alpha$ is almost weakly simplified, so is $\alpha_2$.  So it lets
$\alpha'_2\in\WS$ be the result of calling $\shiftClosuresRight$ on
$\alpha_2$.
\begin{itemize}
\item If $\alpha_1\alpha'_2$ has the form $\beta^*\beta$, for some
  $\beta\in\Reg$, then $\shiftClosuresRight$ returns
  \begin{displaymath}
    \shiftClosuresRight(\rightConcat(\beta, \beta^*)) .
  \end{displaymath}

\item Otherwise, if $\alpha_1\alpha'_2$ has the form
  $\beta^*\beta\gamma$, for some $\beta,\gamma\in\Reg$,
  then $\shiftClosuresRight$ returns
  \begin{displaymath}
   \shiftClosuresRight(\beta\beta^*\gamma)) . 
  \end{displaymath}

\item Otherwise, if $\alpha_1\alpha'_2$ has the form
  $(\beta_1\beta_2)^*\beta_1$, for some
  $\beta_1,\beta_2\in\Reg$, then $\shiftClosuresRight$ returns
  \begin{displaymath}
   \shiftClosuresRight(\beta_1(\rightConcat(\beta_2,\beta_1))^*) . 
  \end{displaymath}

\item Otherwise, if $\alpha_1\alpha'_2$ has the form
  $(\beta_1\beta_2)^*\beta_1\gamma$, for some
  $\beta_1,\beta_2,\gamma\in\Reg$, then $\shiftClosuresRight$ returns
  \begin{displaymath}
   \shiftClosuresRight(\beta_1(\rightConcat(\beta_2,\beta_1))^*\gamma) . 
  \end{displaymath}

\item Otherwise, $\shiftClosuresRight$ returns $\alpha_1\alpha'_2$.
\end{itemize}

(The work needed to justify the kind of well-founded recursion used in
$\shiftClosuresRight$'s definition will be added in a subsequent
revision.)

\begin{proposition}
\label{ShiftClosuresRightLem}
For all $\alpha\in\AWS$, $\shiftClosuresRight\,\alpha$ is equivalent to
$\alpha$ and has the same closure complexity, size, number of
concatenations and number of symbols as $\alpha$.
\end{proposition}

Define a function $\deepClosure\in\WS\fun\WS$ as follows.  For all
$\alpha\in\WS$:
\begin{align*}
  \deepClosure\,\% &= \% , \\
  \deepClosure\,\$ &= \% , \\
  \deepClosure\,({*}(\alpha)) &= \alpha^* , \eqtxt{and} \\
  \deepClosure\,\alpha &= \alpha^*, \eqtxt{if}
  \alpha\not\in\{\%,\$\} \eqtxt{and} \alpha \eqtxtl{is not a closure.}
\end{align*}

\begin{lemma}
\label{DeepClosureLem}
For all $\alpha\in\WS$, $\deepClosure\,\alpha$ is equivalent to
$\alpha^*$, has the same alphabet as $\alpha^*$, has a closure
complexity that is no bigger than that of $\alpha^*$, has a size that
is no bigger than that of $\alpha^*$, has the same number of
concatenations as $\alpha^*$, and has the same number of symbols
$\alpha^*$.
\end{lemma}

Define a function $\deepConcat\in\WS\times\WS\fun\WS$ as follows.  For all
$\alpha,\beta\in\WS$:
\begin{align*}
  \deepConcat(\alpha, \$) &= \$ , \\
  \deepConcat(\$, \alpha) &= \$ , \eqtxt{if} \alpha\neq\$ , \\
  \deepConcat(\alpha, \%) &= \alpha, \eqtxt{if} \alpha\neq\$ , \\
  \deepConcat(\%, \alpha) &= \alpha, \eqtxt{if} \alpha\not\in\{\$,\%\} ,
  \eqtxt{and} \\
  \deepConcat(\alpha,\beta) &=
    \shiftClosuresRight(\rightConcat(\alpha, \beta)), \\
    &{} \hspace*{1cm}\eqtxt{if}\alpha,\beta\not\in\{\$,\%\} .
\end{align*}
To see that the last clause of this definition is proper, suppose
that $\alpha,\beta\in\WS-\{\%,\$\}$.  Thus all the elements of
$\concatsToList\,\alpha$ and $\concatsToList\,\beta$ are weakly
simplified, and are not $\%$, $\$$ or concatenations.  Hence
\begin{displaymath}
\concatsToList(\rightConcat(\alpha,\beta)) =  
\concatsToList\,\alpha \myconcat \concatsToList\,\beta
\end{displaymath}
also has this property, showing that $\rightConcat(\alpha,\beta)$
is almost weakly simplified, which is what $\shiftClosuresRight$
needs to deliver a weakly simplified result.

\begin{lemma}
\label{DeepConcatLem}
For all $\alpha,\beta\in\WS$, $\deepConcat(\alpha,\beta)$
is equivalent to $\alpha\beta$, has an alphabet that is a
subset of the alphabet of $\alpha\beta$, has a closure complexity that
is no bigger than that of $\alpha\beta$, has a size that is no bigger
than that of $\alpha\beta$, has no more concatenations than
$\alpha\beta$, and has no more symbols than $\alpha\beta$.
\end{lemma}

Define a function $\deepUnion\in\WS\times\WS\fun\WS$ as follows.  For all
$\alpha,\beta\in\WS$:
\begin{align*}
  \deepUnion(\alpha, \$) &= \alpha , \\
  \deepUnion(\$, \alpha) &= \alpha , \eqtxt{if} \alpha\neq\$ , \eqtxt{and} \\
  \deepUnion(\alpha, \beta) &=
    \sortUnions(\rightUnion(\alpha, \beta)), \eqtxt{if} \alpha\neq\$
    \eqtxt{and} \beta\neq\$ .
\end{align*}
To see that the last clause of this definition is proper, suppose
$\alpha,\beta\in\WS-\{\$\}$.  Then all the elements of
$\unionsToList(\rightUnion(\alpha,\beta))$ will be weakly simplified,
and won't be $\$$ or unions.  Consequently, $\sortUnions$ will
deliver a weakly simplified result.

\begin{lemma}
\label{DeepUnionLem}
For all $\alpha,\beta\in\WS$, $\deepUnion(\alpha,\beta)$
is equivalent to $\alpha+\beta$, has an alphabet that is a
subset of the alphabet of $\alpha+\beta$, has a closure complexity
that is no bigger than that of $\alpha+\beta$, has a size that is no
bigger than that of $\alpha+\beta$, has no more concatenations than
$\alpha+\beta$, and has no more symbols than $\alpha+\beta$.
\end{lemma}

Now, we can define our weak simplification function/algorithm.
Define $\weaklySimplify\in\Reg\fun\WS$ by structural recursion:
\begin{itemize}
\item $\weaklySimplify\,\% = \%$;

\item $\weaklySimplify\,\$ = \$$;

\item $\weaklySimplify\,a = a$, for all $a\in\Sym$;

\item $\weaklySimplify({*}(\alpha))$
  \begin{displaymath}
    {} = \deepClosure(\weaklySimplify\,\alpha) ,
  \end{displaymath}
  for all $\alpha\in\Reg$;

\item $\weaklySimplify(@(\alpha,\beta))$
  \begin{displaymath}
    {} = \deepConcat(\weaklySimplify\,\alpha, \weaklySimplify\,\beta) ,
  \end{displaymath}
  for all $\alpha,\beta\in\Reg$; and

\item $\weaklySimplify({+}(\alpha,\beta))$
  \begin{displaymath}
    {} = \deepUnion(\weaklySimplify\,\alpha, \weaklySimplify\,\beta) ,
  \end{displaymath}
  for all $\alpha,\beta\in\Reg$.
\end{itemize}

\begin{proposition}
\label{WeakSimpProp1}
For all $\alpha\in\Reg$:
\begin{enumerate}[\quad(1)]
\item $\weaklySimplify\,\alpha\approx\alpha$;

\item $\alphabet(\weaklySimplify(\alpha))\sub\alphabet\,\alpha$;

\item $\cc(\weaklySimplify\,\alpha)\leqcc\cc\,\alpha$;

\item $\mysize(\weaklySimplify\,\alpha)\leq\mysize\,\alpha$;

\item $\numSyms(\weaklySimplify\,\alpha)\leq\numSyms\,\alpha$; and

\item $\numConcats(\weaklySimplify\,\alpha)\leq\numConcats\,\alpha$.
\end{enumerate}
\end{proposition}

\begin{proof}
By induction on regular expressions.
\end{proof}

\begin{exercise}
\label{WeakSimpExercise}
Prove Proposition~\ref{WeakSimpProp1}.
\end{exercise}

\begin{proposition}
For all $\alpha\in\Reg$, if $\alpha$ is weakly simplified, then
$\weaklySimplify(\alpha) = \alpha$.
\end{proposition}

\begin{proof}
By induction on regular expressions.
\end{proof}

Using our weak simplification algorithm, we can define an algorithm
for calculating the language generated by a regular expression, when
this language is finite, and for announcing that this language is
infinite, otherwise.  First, we weakly simplify our regular
expression, $\alpha$, and call the resulting regular expression
$\beta$.  If $\beta$ contains no closures, then we compute its meaning
in the usual way.  But, if $\beta$ contains one or more closures, then
its language will be infinite, and thus we can output a message saying
that $L(\alpha)$ is infinite.

The Forlan module \texttt{Reg} defines the following functions relating
to weak simplification:
\begin{verbatim}
val weaklySimplified : reg -> bool
val weaklySimplify   : reg -> reg
val toStrSet         : reg -> str set
\end{verbatim}
The function \texttt{weaklySimplified} tests whether its argument is
weakly simplified, and \texttt{weaklySimplify} implements
$\weaklySimplify$.  Finally, the function \texttt{toStrSet} implements our
algorithm for calculating the language generated by a regular expression,
if that language is finite, and for announcing the this language is
infinite, otherwise.

Here are some examples of how these functions can be used:
\begin{list}{}
{\setlength{\leftmargin}{\leftmargini}
\setlength{\rightmargin}{0cm}
\setlength{\itemindent}{0cm}
\setlength{\listparindent}{0cm}
\setlength{\itemsep}{0cm}
\setlength{\parsep}{0cm}
\setlength{\labelsep}{0cm}
\setlength{\labelwidth}{0cm}
\catcode`\#=12
\catcode`\$=12
\catcode`\%=12
\catcode`\^=12
\catcode`\_=12
\catcode`\.=12
\catcode`\?=12
\catcode`\!=12
\catcode`\&=12
\ttfamily}
\small
\item[]\textsl{-\ }val\ reg\ =\ Reg.input\ "";
\item[]\textsl{@\ }(%\ +\ $0)(%\ +\ 00\symbol{'052}0\ +\ 0\symbol{'052}\symbol{'052})\symbol{'052}
\item[]\textsl{@\ }.
\item[]\textsl{val\ reg\ =\ -\ :\ reg}
\item[]\textsl{-\ }Reg.output("",\ Reg.weaklySimplify\ reg);
\item[]\textsl{(%\ +\ 0\symbol{'052}\ +\ 000\symbol{'052})\symbol{'052}}
\item[]\textsl{val\ it\ =\ ()\ :\ unit}
\item[]\textsl{-\ }Reg.toStrSet\ reg;
\item[]\textsl{language\ is\ infinite}
\item[]
\item[]\textsl{uncaught\ exception\ Error}
\item[]\textsl{-\ }val\ reg'\ =\ Reg.input\ "";
\item[]\textsl{@\ }(1\ +\ %)(2\ +\ $)(3\ +\ %\symbol{'052})(4\ +\ $\symbol{'052})
\item[]\textsl{@\ }.
\item[]\textsl{val\ reg'\ =\ -\ :\ reg}
\item[]\textsl{-\ }StrSet.output("",\ Reg.toStrSet\ reg');
\item[]\textsl{2,\ 12,\ 23,\ 24,\ 123,\ 124,\ 234,\ 1234}
\item[]\textsl{val\ it\ =\ ()\ :\ unit}
\item[]\textsl{-\ }Reg.output("",\ Reg.weaklySimplify\ reg');
\item[]\textsl{(%\ +\ 1)2(%\ +\ 3)(%\ +\ 4)}
\item[]\textsl{val\ it\ =\ ()\ :\ unit}
\item[]\textsl{-\ }Reg.output
\item[]\textsl{=\ }("",
\item[]\textsl{=\ }\ Reg.weaklySimplify(Reg.fromString\ "(00\symbol{'052}11\symbol{'052})\symbol{'052}"));
\item[]\textsl{(00\symbol{'052}11\symbol{'052})\symbol{'052}}
\item[]\textsl{val\ it\ =\ ()\ :\ unit}
\end{list}


\subsection{Local and Global Simplification}

In preparation for the definition of our local and global simplification
algorithms, we must define some auxiliary functions.
First, we show how we can recursively test whether $\%\in L(\alpha)$, for
a regular expression $\alpha$.  We define a function
\index{hasEmp@$\hasEmp$}%
\index{regular expression!hasEmp@$\hasEmp$}%
\index{regular expression!testing for membership of empty string}%
\begin{gather*}
\hasEmp\in\Reg\fun\Bool
\end{gather*}
 by recursion:
\begin{align*}
\hasEmp\,\% &= \true ; \\
\hasEmp\,\$ &= \false ; \\
\hasEmp\,a &= \false, \eqtxt{for all}a\in\Sym ; \\
\hasEmp(\alpha^*) &= \true , \eqtxt{for all} \alpha\in\Reg ; \\
\hasEmp(\alpha\beta) &=
\hasEmp\,\alpha\myand\hasEmp\,\beta , \eqtxt{for all} \alpha,\beta\in\Reg ;
  \eqtxt{and} \\
\hasEmp(\alpha+\beta) &=
\hasEmp\,\alpha\myor\hasEmp\,\beta , \eqtxt{for all} \alpha,\beta\in\Reg .
\end{align*}

\begin{proposition}
\label{HasEmpProp}
For all $\alpha\in\Reg$, $\%\in L(\alpha)$ iff $\hasEmp\,\alpha =
\true$.
\end{proposition}

\begin{proof}
By induction on regular expressions.
\end{proof}

Next, we show how we can recursively test whether $a\in L(\alpha)$, for
a symbol $a$ and a regular expression $\alpha$.  We define a function
\index{hasSym@$\hasSym$}%
\index{regular expression!hasSym@$\hasSym$}%
\index{regular expression!testing for membership of symbol}%
\begin{gather*}
\hasSym\in\Sym\times\Reg\fun\Bool
\end{gather*}
 by recursion:
\begin{align*}
\hasSym(a, \%) &= \false , \eqtxt{for all} a\in\Sym; \\
\hasSym(a, \$) &= \false , \eqtxt{for all} a\in\Sym; \\
\hasSym(a, b) &= a = b , \eqtxt{for all} a,b\in\Sym; \\
\hasSym(a, \alpha^*) &= \hasSym(a, \alpha) , \eqtxt{for all}
a\in\Sym\eqtxt{and}\alpha\in\Reg ; \\
\hasSym(a, \alpha\beta) &=
(\hasSym(a, \alpha)\myand \hasEmp(\beta)) \myor {} \\
&\quad\;\, (\hasEmp(\alpha)\myand \hasSym(a, \beta)) , \\
&\quad\, \eqtxt{for all} a\in\Sym\eqtxt{and}\alpha,\beta\in\Reg ; \eqtxt{and} \\
\hasSym(a, \alpha+\beta) &=
\hasSym(a, \alpha)\myor\hasSym(a, \beta), \\
&\quad\, \eqtxt{for all} a\in\Sym\eqtxt{and}\alpha,\beta\in\Reg .
\end{align*}

\begin{proposition}
\label{HasSymProp}
For all $a\in\Sym$ and $\alpha\in\Reg$,
$a\in L(\alpha)$ iff $\hasSym(a,\alpha) = \true$.
\end{proposition}

\begin{proof}
By induction on regular expressions, using Proposition~\ref{HasEmpProp}.
\end{proof}

Finally, we define a function/algorithm
\index{obviousSubset@$\obviousSubset$}%
\index{regular expression!obviousSubset@$\obviousSubset$}%
\index{regular expression!conservative subset test}%
\begin{displaymath}
\obviousSubset\in\Reg\times\Reg\fun\{\true,\false\}
\end{displaymath}
meeting the following specification: for all $\alpha,\beta\in\Reg$,
\begin{displaymath}
\eqtxtr{if} \obviousSubset(\alpha,\beta)=\true,
\eqtxt{then} L(\alpha)\sub L(\beta) .
\end{displaymath}
I.e., this function is a \emph{conservative approximation to subset testing}.
\index{regular expression!conservative approximation to subset testing}%
\index{conservative approximation to subset testing}%
The function that always returns $\false$
would meet this specification, but our function will do much better
than this, and will be reasonably efficient. In
Section~\ref{EquivalenceTestingAndMinimizationOfDFAs},
we will learn of a less efficient algorithm that will provide a complete test
for $L(\alpha)\sub L(\beta)$.

Given $\alpha,\beta\in\Reg$, $\obviousSubset(\alpha,\beta)$ proceeds
as follows.  First, it lets $\alpha'=\weaklySimplify\,\alpha$ and
$\beta'=\weaklySimplify\,\beta$.  Then it returns
$\obviSub(\alpha',\beta')$, where
\begin{displaymath}
\obviSub\in\WS\times\WS\fun\Bool
\end{displaymath}
is the function defined below.

$\obviSub$ is defined by well-founded recursion on the sum of the
sizes of its arguments.  If $\alpha=\beta$, then it returns $\true$;
otherwise, it considers the possible forms of $\alpha$.
\begin{itemize}
\item Suppose $\alpha = \%$. It returns $\hasEmp\,\beta$.

\item Suppose $\alpha = \$$. It returns $\true$.

\item Suppose $\alpha = a$, for some $a\in\Sym$. It returns $\hasSym(a,
  \beta)$.

\item Suppose $\alpha = {\alpha_1}^*$, for some $\alpha_1\in\Reg$.
Here it looks at the form of $\beta$.
\begin{itemize}
\item Suppose $\beta = \%$. It returns $\false$.  (Because
$\alpha$ will be weakly simplified, and so $\alpha$ won't generate
$\{\%\}$.)

\item Suppose $\beta = \$$. It returns $\false$.

\item Suppose $\beta = a$, for some $a\in\Sym$. It returns $\false$.

\item Suppose $\beta$ is a closure.  It returns $\obviSub(\alpha_1, \beta)$.

\item Suppose $\beta=\beta_1\beta_2$, for some $\beta_1,\beta_2\in\Reg$.
If $\hasEmp\,\beta_1=\true$ and $\obviSub(\alpha,\beta_2)$,
then it returns $\true$.
Otherwise, if $\hasEmp\,\beta_2=\true$ and $\obviSub(\alpha,\beta_1)$,
then it returns $\true$.
Otherwise, it returns $\false$ (even though the answer sometimes
should be $\true$).

\item Suppose $\beta = \beta_1 + \beta_2$, for some $\beta_1,\beta_2\in\Reg$.
It returns
\begin{gather*}
\obviSub(\alpha, \beta_1) \myor \obviSub(\alpha, \beta_2)
\end{gather*}
(even though this is $\false$ too often).
\end{itemize}

\item Suppose $\alpha = \alpha_1\alpha_2$, for some $\alpha_1,\alpha_2\in\Reg$.
Here it looks at the form of $\beta$.
\begin{itemize}
\item Suppose $\beta = \%$.  It returns $\false$.  (Because $\alpha$
  is weakly simplified, $\alpha$ won't generate $\{\%\}$.)

\item Suppose $\beta = \$$. It returns $\false$.  (Because $\alpha$ is
  weakly simplified, $\alpha$ won't generate $\emptyset$.)

\item Suppose $\beta = a$, for some $a\in\Sym$. It returns $\false$.
  (Because $\alpha$ is weakly simplified, $\alpha$ won't generate
  $\{a\}$.)

\item Suppose $\beta={\beta_1}^*$, for some $\beta_1\in\Reg$. It returns
\begin{gather*}
\obviSub(\alpha, \beta_1) \\
\myor \\
(\obviSub(\alpha_1, \beta) \myand \obviSub(\alpha_2, \beta))
\end{gather*}
(even though this returns $\false$ too often).

\item Suppose $\beta = \beta_1\beta_2$, for some $\beta_1,\beta_2\in\Reg$.
If $\obviSub(\alpha_1, \beta_1)=\true$ and $\obviSub(\alpha_2,
\beta_2)=\true$, then it returns $\true$.
Otherwise, if $\hasEmp\,\beta_1=\true$ and $\obviSub(\alpha,\beta_2)=\true$,
then it returns $\true$.
Otherwise, if $\hasEmp\,\beta_2=\true$ and $\obviSub(\alpha,\beta_1)=\true$,
then it returns $\true$.
Otherwise, if $\beta_1$ is a closure but $\beta_2$ is not a closure,
then it returns
\begin{gather*}
\obviSub(\alpha_1, \beta_1) \myand \obviSub(\alpha_2, \beta)
\end{gather*}
(even though this returns $\false$ too often).
Otherwise, if $\beta_2$ is a closure but $\beta_1$ is not a closure,
then it returns
\begin{gather*}
\obviSub(\alpha_1, \beta) \myand \obviSub(\alpha_2, \beta_2)
\end{gather*}
(even though this returns $\false$ too often).
Otherwise, if $\beta_1$ and $\beta_2$ are closures, then it
returns
\begin{gather*}
(\obviSub(\alpha_1, \beta_1) \myand \obviSub(\alpha_2, \beta)) \\
\myor \\
(\obviSub(\alpha_1, \beta) \myand \obviSub(\alpha_2, \beta_2))
\end{gather*}
(even though this returns $\false$ too often).
Otherwise, it returns $\false$, even though sometimes we would like
the answer to be $\true$).

\item Suppose $\beta = \beta_1 + \beta_2$, for some $\beta_1,\beta_2\in\Reg$.
It returns
\begin{gather*}
\obviSub(\alpha, \beta_1) \myor \obviSub(\alpha, \beta_2)
\end{gather*}
(even though this is $\false$ too often).
\end{itemize}

\item Suppose $\alpha=\alpha_1+\alpha_2$. It returns
\begin{gather*}
\obviSub(\alpha_1, \beta) \myand \obviSub(\alpha_2, \beta) .
\end{gather*}
\end{itemize}

We say that $\alpha$ is \emph{obviously a subset of} $\beta$ iff
$\obviousSubset(\alpha,\beta)=\true$. On the positive side, we have
that, e.g.,
$\obviousSubset(\mathsf{0^*011^*1},\mathsf{0^*1^*})=\true$.  On the
other hand, $\obviousSubset(\mathsf{(01)^*},\mathsf{(\% + 0)(10)^*(\%
  + 1)})=\false$, even though $L(\mathsf{(01)^*})\sub L(\mathsf{(\% +
  0)(10)^*(\% + 1)})$.

\begin{proposition}
\label{WeakSubProp}
For all $\alpha,\beta\in\Reg$, if $\obviousSubset(\alpha,\beta)=\true$,
then $L(\alpha)\sub L(\beta)$.
\end{proposition}

\begin{proof}
First, we use induction on the sum of the sizes of $\alpha$ and
$\beta$ to show that, for all $\alpha,\beta\in\Reg$, if
$\obviSub(\alpha,\beta)=\true$, then $L(\alpha)\sub L(\beta)$.
The result then follows by Proposition~\ref{WeakSimpProp1}.
\end{proof}

The Forlan module \texttt{Reg} provides the following functions
corresponding to the auxiliary functions $\hasEmp$, $\hasSym$ and
$\obviousSubset$:
\begin{verbatim}
val hasEmp        : reg -> bool
val hasSym        : sym * reg -> bool
val obviousSubset : reg * reg -> bool
\end{verbatim}
Here are some examples of how they can be used:
\begin{list}{}
{\setlength{\leftmargin}{\leftmargini}
\setlength{\rightmargin}{0cm}
\setlength{\itemindent}{0cm}
\setlength{\listparindent}{0cm}
\setlength{\itemsep}{0cm}
\setlength{\parsep}{0cm}
\setlength{\labelsep}{0cm}
\setlength{\labelwidth}{0cm}
\catcode`\#=12
\catcode`\$=12
\catcode`\%=12
\catcode`\^=12
\catcode`\_=12
\catcode`\.=12
\catcode`\?=12
\catcode`\!=12
\catcode`\&=12
\ttfamily}
\small
\item[]\textsl{-\ }Reg.hasEmp(Reg.fromString\ "0\symbol{'052}1\symbol{'052}");
\item[]\textsl{val\ it\ =\ true\ :\ bool}
\item[]\textsl{-\ }Reg.hasEmp(Reg.fromString\ "01\symbol{'052}");
\item[]\textsl{val\ it\ =\ false\ :\ bool}
\item[]\textsl{-\ }Reg.hasSym(Sym.fromString\ "0",\ Reg.fromString\ "0\symbol{'052}1\symbol{'052}");
\item[]\textsl{val\ it\ =\ true\ :\ bool}
\item[]\textsl{-\ }Reg.hasSym(Sym.fromString\ "1",\ Reg.fromString\ "0\symbol{'052}1\symbol{'052}");
\item[]\textsl{val\ it\ =\ true\ :\ bool}
\item[]\textsl{-\ }Reg.hasSym(Sym.fromString\ "0",\ Reg.fromString\ "0\symbol{'052}$");
\item[]\textsl{val\ it\ =\ false\ :\ bool}
\item[]\textsl{-\ }Reg.obviousSubset
\item[]\textsl{=\ }(Reg.fromString\ "(0\ +\ 1)\symbol{'052}",
\item[]\textsl{=\ }\ Reg.fromString\ "0\symbol{'052}(0\ +\ 1)\symbol{'052}1\symbol{'052}");
\item[]\textsl{val\ it\ =\ true\ :\ bool}
\item[]\textsl{-\ }Reg.obviousSubset
\item[]\textsl{=\ }(Reg.fromString\ "0\symbol{'052}(0\ +\ 1)\symbol{'052}1\symbol{'052}",
\item[]\textsl{=\ }\ Reg.fromString\ "(0\ +\ 1)\symbol{'052}");
\item[]\textsl{val\ it\ =\ true\ :\ bool}
\item[]\textsl{-\ }Reg.obviousSubset
\item[]\textsl{=\ }(Reg.fromString\ "0\symbol{'052}011\symbol{'052}1",
\item[]\textsl{=\ }\ Reg.fromString\ "0\symbol{'052}1\symbol{'052}");
\item[]\textsl{val\ it\ =\ true\ :\ bool}
\item[]\textsl{-\ }Reg.obviousSubset
\item[]\textsl{=\ }(Reg.fromString\ "(01\ +\ 011)1\symbol{'052}",
\item[]\textsl{=\ }\ Reg.fromString\ "01\symbol{'052}");
\item[]\textsl{val\ it\ =\ true\ :\ bool}
\item[]\textsl{-\ }Reg.obviousSubset
\item[]\textsl{=\ }(Reg.fromString\ "(01)\symbol{'052}",
\item[]\textsl{=\ }\ Reg.fromString\ "(%\ +\ 0)(10)\symbol{'052}(%\ +\ 1)");
\item[]\textsl{val\ it\ =\ false\ :\ bool}
\end{list}


Our local and global simplification algorithms make use of
simplification rules, which may be applied to arbitrary subtrees of
regular expressions.  There are three kinds of rules: structural
rules, distributive rules and reduction rules.

There are nine \emph{structural rules},
\index{structural rule}%
\index{regular expression!structural rule}%
\index{simplification!regular expression!structural rule}%
which preserve the alphabet, closure complexity, size, number of
concatenations and number of symbols of a regular expression:
\begin{enumerate}[\quad(1)]
\item $(\alpha + \beta) + \gamma \fun \alpha + (\beta + \gamma)$.

\item $\alpha + (\beta + \gamma) \fun (\alpha + \beta) + \gamma$.

\item $\alpha(\beta\gamma) \fun (\alpha\beta)\gamma$.

\item $(\alpha\beta)\gamma \fun \alpha(\beta\gamma)$.

\item $\alpha + \beta \fun \beta + \alpha$.

\item $\alpha^*\alpha \fun \alpha\alpha^*$.

\item $\alpha\alpha^* \fun \alpha^*\alpha$.

\item $\alpha(\beta\alpha)^* \fun (\alpha\beta)^*\alpha$.

\item $(\alpha\beta)^*\alpha \fun \alpha(\beta\alpha)^*$.
\end{enumerate}

There are two \emph{distributive rules}, which preserve the
alphabet of a regular expression:
\begin{enumerate}[\quad(1)]
\item $\alpha(\beta_1 + \beta_2) \fun \alpha \beta_1 + \alpha \beta_2$.

\item $(\alpha_1 + \alpha_2)\beta \fun \alpha_1 \beta + \alpha_2 \beta$.
\end{enumerate}

Finally, there are $26$ \emph{reduction rules}, some of which make use
of a conservative approximation $\Sub$ to subset testing.  When
$\alpha\fun\beta$ because of a reduction rule, we have that
$\alphabet\,\beta\sub\alphabet\,\alpha$ and $\beta\simp\alpha$, where
$\simp$ is the well-founded relation on $\Reg$ that is defined below.

We define the relation $\simp$ on $\Reg$ by: for all $\alpha,\beta\in
\Reg$, $\alpha\simp\beta$ iff either:
\begin{itemize}
\item $\cc\,\alpha\ltcc\cc\,\beta$; or

\item $\cc\,\alpha = \cc\,\beta$, but $\mysize\,\alpha < \mysize\,\beta$; or

\item $\cc\,\alpha = \cc\,\beta$ and $\mysize\,\alpha = \mysize\,\beta$,
  but $\numConcats\,\alpha < \numConcats\,\beta$; or

\item $\cc\,\alpha = \cc\,\beta$ and $\mysize\,\alpha =
  \mysize\,\beta$, and $\numConcats\,\alpha = \numConcats\,\beta$, but
  $\numSyms\,\alpha < \numSyms\,\beta$.
\end{itemize}
Note that this is almost the same definition as that of
$\ltsimp$---the difference being that $\simp$ doesn't have
the final step involving standardization.

\begin{proposition}
$\simp$ is a well-founded relation on $\Reg$.
\end{proposition}

\begin{proof}
Follows by Propositions~\ref{LTCCWellFounded}, \ref{LexWellFounded}
and \ref{InverseImageWellFounded}, plus the fact that $<$ is
well-founded on $\nats$ (Proposition~\ref{NonemptyLeastProp}).
\end{proof}

Our reduction rules follow.  In the rules, we abbreviate
$\hasEmp\,\alpha=\true$ and $\Sub(\alpha,\beta)=\true$ to
$\hasEmp\,\alpha$ and $\Sub(\alpha,\beta)$, respectively.  Most of
the rules strictly decrease a regular expression's closure complexity
and size.  The exceptions are labeled ``cc'' (for when the closure
complexity strictly decreases, but the size strictly increases),
``concatenations'' (for when the closure complexity and size are
preserved, but the number of concatenations strictly decreases) or
``symbols'' (for when the closure complexity and size normally strictly
decrease, but occasionally they and the number of concatenations stay
they same, but the number of symbols strictly decreases).

\begin{enumerate}[\quad(1)]
%1
\item If $\Sub(\alpha,\beta)$, then $\alpha + \beta \fun \beta$.

%2
\item $\alpha\beta_1 + \alpha\beta_2 \fun \alpha(\beta_1+\beta_2)$.

%3
\item $\alpha_1\beta + \alpha_2\beta \fun (\alpha_1+\alpha_2)\beta$.

%4
\item If $\hasEmp\,\alpha$ and $\Sub(\alpha,\beta^*)$, then
  $\alpha\beta^* \fun \beta^*$.

%5
\item If $\hasEmp\,\beta$ and $\Sub(\beta,\alpha^*)$, then
  $\alpha^*\beta \fun \alpha^*$.

%6
\item If $\Sub(\alpha,\beta^*)$, then $(\alpha+\beta)^* \fun \beta^*$.

%7
\item $(\alpha^* + \beta)^* \fun (\alpha+\beta)^*$.

%8
\item (concatenations) If $\hasEmp\,\alpha$ and $\hasEmp\,\beta$, then
  $(\alpha\beta)^* \fun (\alpha+\beta)^*$.

%9
\item (concatenations) If $\hasEmp\,\alpha$ and $\hasEmp\,\beta$, then
  $(\alpha\beta + \gamma)^* \fun (\alpha+\beta+\gamma)^*$.
 
%10
\item If $\hasEmp\,\alpha$ and $\Sub(\alpha,\beta^*)$, then
  $(\alpha\beta)^* \fun \beta^*$.

%11
\item If $\hasEmp\,\beta$ and $\Sub(\beta,\alpha^*)$, then
  $(\alpha\beta)^* \fun \alpha^*$.

%12
\item If $\hasEmp\,\alpha$ and $\Sub(\alpha,(\beta + \gamma)^*)$, then
  $(\alpha\beta + \gamma)^* \fun (\beta + \gamma)^*$.

%13
\item If $\hasEmp\,\beta$ and $\Sub(\beta,(\alpha + \gamma)^*)$, then
  $(\alpha\beta + \gamma)^* \fun (\alpha + \gamma)^*$.

%14
\item (cc) If $\mynot(\hasEmp\,\alpha)$ and $\cc\,\alpha \cup
  \overline{\cc\,\beta} \ltcc \overline{\overline{\cc\,\beta}}$, then
  $(\alpha\beta^*)^* \fun \% + \alpha(\alpha+\beta)^*$.

%15
\item (cc) If $\mynot(\hasEmp\,\beta)$ and $\overline{\cc\,\alpha} \cup
  \cc\,\beta \ltcc \overline{\overline{\cc\,\alpha}}$, then
  $(\alpha^*\beta)^* \fun \% + (\alpha+\beta)^*\beta$.

%16
\item (cc) If $\mynot(\hasEmp\,\alpha)$ or $\mynot(\hasEmp\,\gamma)$, and
  $\cc\,\alpha \cup \overline{\cc\,\beta} \cup \cc\,\gamma \ltcc
  \overline{\overline{\cc,\beta}}$, then $(\alpha\beta^*\gamma)^*
  \fun \% + \alpha(\beta + \gamma\alpha)^*\gamma$.

%17
\item If $\Sub(\alpha\alpha^*,\beta)$, then $\alpha^*+\beta \fun \% +
  \beta$.

%18
\item If $\hasEmp\,\beta$ and $\Sub(\alpha\alpha\alpha^*, \beta)$,
  then $\alpha^* + \beta \fun \alpha + \beta$.

%19
\item (symbols) If $\alpha\not\in\{\%,\$\}$ and $\Sub(\alpha^n, \beta)$, then
  $\alpha^{n+1}\alpha^* + \beta \fun \alpha^n\alpha^* + \beta$.

%20
\item If $n\geq 2$, $l\geq 0$ and $2n - 1 < m_1 < \cdots < m_l$, then
  $(\alpha^n + \alpha^{n+1} + \cdots + \alpha^{2n - 1} + \alpha^{m_1}
  + \cdots + \alpha^{m_l})^* \fun \% + \alpha^n\alpha^*$.

%21
\item (symbols) If $\alpha\not\in\{\%,\$\}$, then $\alpha+\alpha\beta
  \fun \alpha(\%+\beta)$.

%22
\item (symbols) If $\alpha\not\in\{\%,\$\}$, then $\alpha+\beta\alpha
  \fun (\%+\beta)\alpha$.

%23
\item $\alpha^*(\% + \beta(\alpha+\beta)^*) \fun (\alpha+\beta)^*$.

%24
\item $(\% + (\alpha+\beta)^*\alpha)\beta^* \fun (\alpha+\beta)^*$.

%25
\item If $\Sub(\alpha, \beta^*)$ and $\Sub(\beta, \alpha)$, then
   $\% + \alpha\beta^* \fun \beta^*$.

%26
\item If $\Sub(\beta, \alpha^*)$ and $\Sub(\alpha, \beta)$, then
   $\% + \alpha^* \beta \fun \alpha^*$.
\end{enumerate}

In rules (14)-(16), the preconditions involving $\cc$ are necessary
and sufficient conditions for the right-hand side to have strictly
smaller closure complexity than the left-hand side.

Consider, e.g., reduction rule (4).  Suppose $\hasEmp\,\alpha=\true$
and $\Sub(\alpha,\beta^*)=\true$, so that that $\%\in L(\alpha)$ and
$L(\alpha)\sub L(\beta^*)$.  We need that $\alpha\beta^* \approx
\beta^*$, $\alphabet(\beta^*)\sub\alphabet(\alpha\beta^*)$ and
$\beta^*\simp\alpha\beta^*$.  The alphabet of $\beta^*$ is clearly
a subset of that of $\alpha\beta^*$.

To obtain $\alpha\beta^* \approx \beta^*$, it will suffice to show
that, for all $A,B\in\Lan$, if $\%\in A$ and $A\sub B^*$, then
$AB^*=B^*$.  Suppose $A,B\in\Lan$, $\%\in A$ and $A\sub B^*$.
We show that $AB^*\sub B^*\sub AB^*$.  Suppose $w\in AB^*$, so
that $w=xy$, for some $x\in A$ and $y\in B^*$.  Since $A\sub B^*$,
it follows that $w=xy\in B^*B^*=B^*$.  Suppose $w\in B^*$.
Then $w=\%w\in AB^*$.

And, to see that $\beta^*\ltcc\alpha\beta^*$, it will suffice to
show that $\cc(\beta^*)\ltcc\cc(\alpha\beta^*)$.  And we
have that
\begin{displaymath}
 \cc(\beta^*) = \overline{\cc\,\beta} \ltcc
 \cc\,\alpha \cup \overline{\cc\,\beta} = \cc(\alpha\beta^*) .
\end{displaymath}

Because the structural rules preserve the size and alphabet of regular
expressions, if we start with a regular expression $\alpha$, there are
only finitely many regular expressions that we can transform $\alpha$
into using structural rules (we can apply one of the rules to some
subtree of $\alpha$, giving us $\beta_1$, apply a rule to one of the
subtrees of $\beta_2$, giving us $\beta_2$, etc.).

Suppose $\Sub$ is a conservative approximation to subset testing.
We say that a regular expression $\alpha$ is \emph{locally simplified with
respect to} $\Sub$:
iff
\begin{itemize}
\item $\alpha$ is weakly simplified, and

\item $\alpha$ can't be transformed by our structural rules into
a regular expression to which one of our reduction rules
applies.
\end{itemize}

The \emph{local simplification of} a regular expression $\alpha$
\emph{with respect to} a conservative approximation to subset testing
$\Sub$ proceeds as follows.  It calls its main function with the weak
simplification, $\beta$ of $\alpha$.  The closure complexity, size,
number of concatenations, and number of symbols of $\beta$ are no
bigger than those of $\alpha$, and
$\alphabet\,\beta\sub\alphabet\,\alpha$.

The main function is defined by well-founded recursion $\simp$.  It
works as follows, when called with a weakly simplified argument,
$\alpha$.
\begin{itemize}
\item It generates the set $X$ of all regular expressions
  $\weaklySimplify\,\gamma$, such that $\alpha$ can be reorganized
  using the structural rules into a regular expression $\beta$,
  which can be transformed by a single application of one of our
  reduction rules into $\gamma$.

\item If $X$ is empty, then it returns $\alpha$.

\item Otherwise, it calls itself recursively on the simplest element,
  $\gamma$ of $X$ (when $X$ doesn't have a unique simplest element,
  the smallest of the simplest elements---in our total ordering on regular
  expressions---is selected).  Because
  \begin{itemize}
  \item the structural rules preserve closure complexity, size, number
    of concatenations, and number of symbols,

  \item the reduction rules produce $\simp$-predecessors, and

  \item and weak simplification doesn't increase closure complexity,
    size, numbers of concatenations, or numbers of symbols,
  \end{itemize}
  we have that $\gamma\simp\alpha$, so that this recursive call
  is legal.  Furthermore, weak simplification, and all of the rules,
  either preserve or decrease (via $\sub$) the alphabet of
  regular expressions.  Thus $\alphabet\,\gamma\sub\alphabet\,\alpha$.
\end{itemize}

The algorithm is referred to as ``local'', because at each recursive
call of its main function, $\gamma$ is chosen using the best local
knowledge.  This strategy is reasonably efficient, but there is no
guarantee that another local choice wouldn't result in a simpler
global answer.

We define a function/algorithm
\begin{displaymath}
\locallySimplify\in(\Reg\times\Reg\fun\Bool)\fun\Reg\fun\Reg
\end{displaymath}
by: for all conservative approximations to subset testing $\Sub$, and
$\alpha\in\Reg$, $\locallySimplify\,\Sub\,\alpha$ is the result of
running our local simplification algorithm on $\alpha$, using $\Sub$
as the conservative approximation to subset testing.

\begin{theorem}
For all conservative approximations to subset testing $\Sub$,
and $\alpha\in\Reg$:
\begin{itemize}
\item $\locallySimplify\,\Sub\,\alpha$ is locally simplified with
  respect to $\Sub$;

\item $\locallySimplify\,\Sub\,\alpha$ is equivalent to $\alpha$;

\item $\alphabet(\locallySimplify\,\Sub\,\alpha)\sub\alphabet\,\alpha$;
  and

\item $\locallySimplify\,\Sub\,\alpha\leqsimp\alpha$.
\end{itemize}
\end{theorem}

The Forlan module \texttt{Reg} provides the following functions
relating to local simplification:
\begin{verbatim}
val locallySimplified    :
      (reg * reg -> bool) -> reg -> bool
val locallySimplify      :
      int option * (reg * reg -> bool) -> reg -> bool * reg
val locallySimplifyTrace :
      int option * (reg * reg -> bool) -> reg -> bool * reg
\end{verbatim}
The function \texttt{locallySimplified} takes in a conservative
approximation to subset testing $\Sub$ and returns a function that
tests whether a regular expression is $\Sub$-locally simplified.  The
function \texttt{locallySimplifyTrace} implements $\locallySimplify$.
It emits tracing messages explaining its operation, takes in an extra
argument of type \texttt{int~option}, and produces an extra result of
type \texttt{bool}.  If this extra argument is \texttt{NONE}, then it
runs as does $\locallySimplify$, and its boolean result is always
$\true$.  But if it is $\mathtt{SOME}\;n$, for $n\geq 1$, then at each
recursive call of the algorithm's function, no more than $n$ ways of
reorganizing the function's argument will be considered, and the
boolean part of the result will be $\false$ iff, in the final
recursive call, $n$ was not sufficient to explore all structural
reorganizations, so that the regular expression returned may not be
locally simplified with respect to $\Sub$.  The function
\texttt{locallySimplify} works identically, except it doesn't issue
tracing messages.

Here are some examples of how these functions can be used.
\begin{list}{}
{\setlength{\leftmargin}{\leftmargini}
\setlength{\rightmargin}{0cm}
\setlength{\itemindent}{0cm}
\setlength{\listparindent}{0cm}
\setlength{\itemsep}{0cm}
\setlength{\parsep}{0cm}
\setlength{\labelsep}{0cm}
\setlength{\labelwidth}{0cm}
\catcode`\#=12
\catcode`\$=12
\catcode`\%=12
\catcode`\^=12
\catcode`\_=12
\catcode`\.=12
\catcode`\?=12
\catcode`\!=12
\catcode`\&=12
\ttfamily}
\small
\item[]\textsl{-\ }val\ locSimped\ =\ Reg.locallySimplified\ Reg.obviousSubset;
\item[]\textsl{val\ locSimped\ =\ fn\ :\ reg\ ->\ bool}
\item[]\textsl{-\ }locSimped(Reg.fromString\ "(1\ +\ 00\symbol{'052}1)\symbol{'052}00\symbol{'052}");
\item[]\textsl{val\ it\ =\ false\ :\ bool}
\item[]\textsl{-\ }locSimped(Reg.fromString\ "(0\ +\ 1)\symbol{'052}0");
\item[]\textsl{val\ it\ =\ true\ :\ bool}
\item[]\textsl{-\ }fun\ locSimp\ nOpt\ =
\item[]\textsl{=\ }\ \ \ \ \ \ Reg.locallySimplify(nOpt,\ Reg.obviousSubset);
\item[]\textsl{val\ locSimp\ =\ fn\ :\ int\ option\ ->\ reg\ ->\ bool\ \symbol{'052}\ reg}
\item[]\textsl{-\ }locSimp\ NONE\ (Reg.fromString\ "%\ +\ 0\symbol{'052}0(0\ +\ 1)\symbol{'052}\ +\ 1\symbol{'052}1(0\ +\ 1)\symbol{'052}");
\item[]\textsl{val\ it\ =\ (true,-)\ :\ bool\ \symbol{'052}\ reg}
\item[]\textsl{-\ }Reg.output("",\ #2\ it);
\item[]\textsl{(0\ +\ 1)\symbol{'052}}
\item[]\textsl{val\ it\ =\ ()\ :\ unit}
\item[]\textsl{-\ }locSimp\ NONE\ (Reg.fromString\ "%\ +\ 1\symbol{'052}0(0\ +\ 1)\symbol{'052}\ +\ 0\symbol{'052}1(0\ +\ 1)\symbol{'052}");
\item[]\textsl{val\ it\ =\ (true,-)\ :\ bool\ \symbol{'052}\ reg}
\item[]\textsl{-\ }Reg.output("",\ #2\ it);
\item[]\textsl{(0\ +\ 1)\symbol{'052}}
\item[]\textsl{val\ it\ =\ ()\ :\ unit}
\item[]\textsl{-\ }locSimp\ NONE\ (Reg.fromString\ "(1\ +\ 00\symbol{'052}1)\symbol{'052}00\symbol{'052}");
\item[]\textsl{val\ it\ =\ (true,-)\ :\ bool\ \symbol{'052}\ reg}
\item[]\textsl{-\ }Reg.output("",\ #2\ it);
\item[]\textsl{(0\ +\ 1)\symbol{'052}0}
\item[]\textsl{val\ it\ =\ ()\ :\ unit}
\item[]\textsl{-\ }Reg.locallySimplifyTrace
\item[]\textsl{=\ }(NONE,\ Reg.obviousSubset)
\item[]\textsl{=\ }(Reg.fromString\ "1\symbol{'052}(01\symbol{'052}01\symbol{'052})\symbol{'052}");
\item[]\textsl{considered\ all\ 10\ structural\ reorganizations\ of\ 1\symbol{'052}(01\symbol{'052}01\symbol{'052})\symbol{'052}}
\item[]\textsl{1\symbol{'052}(01\symbol{'052}01\symbol{'052})\symbol{'052}\ transformed\ by\ structural\ rule\ 4\ at\ position\ \symbol{'133}2,\ 1\symbol{'135}\ to}
\item[]\textsl{1\symbol{'052}((01\symbol{'052})01\symbol{'052})\symbol{'052}\ transformed\ by\ structural\ rule\ 4\ at\ position\ \symbol{'133}2,\ 1\symbol{'135}}
\item[]\textsl{to\ 1\symbol{'052}(((01\symbol{'052})0)1\symbol{'052})\symbol{'052}\ transformed\ by\ reduction\ rule\ 14\ at\ position}
\item[]\textsl{\symbol{'133}2\symbol{'135}\ to\ 1\symbol{'052}(%\ +\ ((01\symbol{'052})0)((01\symbol{'052})0\ +\ 1)\symbol{'052})\ weakly\ simplifies\ to}
\item[]\textsl{1\symbol{'052}(%\ +\ 01\symbol{'052}0(1\ +\ 01\symbol{'052}0)\symbol{'052})}
\item[]\textsl{considered\ all\ 40\ structural\ reorganizations\ of}
\item[]\textsl{1\symbol{'052}(%\ +\ 01\symbol{'052}0(1\ +\ 01\symbol{'052}0)\symbol{'052})}
\item[]\textsl{1\symbol{'052}(%\ +\ 01\symbol{'052}0(1\ +\ 01\symbol{'052}0)\symbol{'052})\ transformed\ by\ structural\ rule\ 4\ at}
\item[]\textsl{position\ \symbol{'133}2,\ 2,\ 2\symbol{'135}\ to\ 1\symbol{'052}(%\ +\ 0(1\symbol{'052}0)(1\ +\ 01\symbol{'052}0)\symbol{'052})\ transformed\ by}
\item[]\textsl{structural\ rule\ 4\ at\ position\ \symbol{'133}2,\ 2\symbol{'135}\ to\ 1\symbol{'052}(%\ +\ (01\symbol{'052}0)(1\ +\ 01\symbol{'052}0)\symbol{'052})}
\item[]\textsl{transformed\ by\ reduction\ rule\ 23\ at\ position\ \symbol{'133}\symbol{'135}\ to\ (1\ +\ 01\symbol{'052}0)\symbol{'052}}
\item[]\textsl{considered\ all\ 4\ structural\ reorganizations\ of\ (1\ +\ 01\symbol{'052}0)\symbol{'052}}
\item[]\textsl{(1\ +\ 01\symbol{'052}0)\symbol{'052}\ is\ locally\ simplified}
\item[]\textsl{val\ it\ =\ (true,-)\ :\ bool\ \symbol{'052}\ reg}
\end{list}

For even fairly small regular expressions, running through all the
structural reorganizations can take prohibitively long.  So, one
often has to bound the number of such reorganizations, as in:
\begin{list}{}
{\setlength{\leftmargin}{\leftmargini}
\setlength{\rightmargin}{0cm}
\setlength{\itemindent}{0cm}
\setlength{\listparindent}{0cm}
\setlength{\itemsep}{0cm}
\setlength{\parsep}{0cm}
\setlength{\labelsep}{0cm}
\setlength{\labelwidth}{0cm}
\catcode`\#=12
\catcode`\$=12
\catcode`\%=12
\catcode`\^=12
\catcode`\_=12
\catcode`\.=12
\catcode`\?=12
\catcode`\!=12
\catcode`\&=12
\ttfamily}
\small
\item[]\textsl{-\ }val\ reg\ =\ Reg.input\ "";
\item[]\textsl{@\ }1\ +\ (%\ +\ 0\ +\ 2)(%\ +\ 0\ +\ 2)\symbol{'052}1\ +
\item[]\textsl{@\ }(1\ +\ (%\ +\ 0\ +\ 2)(%\ +\ 0\ +\ 2)\symbol{'052}1)
\item[]\textsl{@\ }(%\ +\ 0\ +\ 2\ +\ 1(%\ +\ 0\ +\ 2)\symbol{'052}1)
\item[]\textsl{@\ }(%\ +\ 0\ +\ 2\ +\ 1(%\ +\ 0\ +\ 2)\symbol{'052}1)\symbol{'052}
\item[]\textsl{@\ }.
\item[]\textsl{val\ reg\ =\ -\ :\ reg}
\item[]\textsl{-\ }Reg.equal(Reg.weaklySimplify\ reg,\ reg);
\item[]\textsl{val\ it\ =\ true\ :\ bool}
\item[]\textsl{-\ }val\ (b',\ reg')\ =\ locSimp\ (SOME\ 10)\ reg;
\item[]\textsl{val\ b'\ =\ false\ :\ bool}
\item[]\textsl{val\ reg'\ =\ -\ :\ reg}
\item[]\textsl{-\ }Reg.output("",\ reg');
\item[]\textsl{(0\ +\ 2)\symbol{'052}1(0\ +\ 2\ +\ 1(0\ +\ 2)\symbol{'052}1)\symbol{'052}}
\item[]\textsl{val\ it\ =\ ()\ :\ unit}
\item[]\textsl{-\ }val\ (b'',\ reg'')\ =\ locSimp\ (SOME\ 1000)\ reg';
\item[]\textsl{val\ b''\ =\ true\ :\ bool}
\item[]\textsl{val\ reg''\ =\ -\ :\ reg}
\item[]\textsl{-\ }Reg.output("",\ reg'');
\item[]\textsl{(0\ +\ 2)\symbol{'052}1(0\ +\ 2\ +\ 1(0\ +\ 2)\symbol{'052}1)\symbol{'052}}
\item[]\textsl{val\ it\ =\ ()\ :\ unit}
\end{list}

Note that, in this transcript, \texttt{reg'} turns out to be
locally simplified, despite the fact that \texttt{b'} is \texttt{false}.

Our global simplification algorithm comes in two variants, a
non-distributive one, which doesn't use the distributive rules, and a
distributive one, which does.  Given a boolean $b$, a conservative
approximation to subset testing $\Sub$, and a regular expression
$\alpha$, we say that $\alpha$ is \emph{globally simplified with
  respect to} $b$ and $\Sub$ iff no strictly simpler regular
expression can be found by an arbitrary number of applications of weak
simplification, structural rules, reduction rules and---if
$b=\true$---distributive rules.

The \emph{global simplification of} a regular expression $\alpha$
\emph{with respect to} a boolean $b$ and conservative approximation to
subset testing $\Sub$ consists of generating the set $X$ of all
regular expressions $\beta$ that can formed from $\alpha$ by an
arbitrary number of applications of weak simplification, the
structural rules, reduction rules, and---in the case of the
distributive variant---the distributive ones.  The simplest element of
$X$ is then selected (when there isn't a unique simplest element, the
smallest of the simplest elements---in our total ordering on regular
expressions---is selected).  (A proof that the generation of $X$
terminates even in the distributive case is not yet complete.)

Of course, this algorithm is much less efficient than the local
one, but by revisiting choices, it is capable of producing simpler
answers.

We define a function/algorithm
\begin{displaymath}
\globallySimplify\in\Bool\times(\Reg\times\Reg\fun\Bool)\fun\Reg\fun\Reg 
\end{displaymath}
by: for all $b\in\Bool$, conservative approximation to subset testing
$\Sub$, and $\alpha\in\Reg$, $\globallySimplify\,(b,\Sub)\,\alpha$ is
the result of running our global simplification algorithm on $\alpha$,
including the distributive rules iff $b=\true$, and using $\Sub$ as
our conservative approximation to subset testing.

\begin{theorem}
For all $b\in\Bool$, conservative approximations to subset testing $\Sub$, and
$\alpha\in\Reg$:
\begin{itemize}
\item $\globallySimplify\,(b,\Sub)\,\alpha$ is globally simplified with
  respect to $b$ and $\Sub$;

\item $\globallySimplify\,(b,\Sub)\,\alpha$ is equivalent to $\alpha$;

\item $\alphabet(\globallySimplify\,(b,\Sub)\,\alpha)\sub\alphabet\,\alpha$;
  and

\item $\globallySimplify\,(b,\Sub)\,\alpha\leqsimp\alpha$.
\end{itemize}
\end{theorem}

The Forlan module \texttt{Reg} provides the following functions
relating to global simplification:
\begin{verbatim}
val globallySimplified    :
      bool * (reg * reg -> bool) -> reg -> bool
val globallySimplifyTrace :
      int option * bool * (reg * reg -> bool) -> reg -> bool * reg
val globallySimplify      :
      int option * bool * (reg * reg -> bool) -> reg -> bool * reg
\end{verbatim}
The function \texttt{globallySimplified} takes in a boolean $b$ and a
conservative approximation to subset testing $\Sub$, and returns a
function that tests whether a regular expression is globally
simplified with respect to $b$ and $\Sub$.  The function
\texttt{globallySimplifyTrace} implements $\globallySimplify$.  It
emits tracing messages explaining its operation, and takes in an extra
argument of type \texttt{int~option}, and produces an extra result of
type \texttt{bool}.  If this argument is \texttt{NONE}, then it runs
as does $\globallySimplify$, and the boolean result is always $\true$.
But if it is $\mathtt{SOME}\;n$, for $n\geq 1$, then at most $n$
elements of the set $X$ are generated, before picking the simplest
one, and the boolean result is $\false$ if this $n$ isn't enough to
generate all of $X$.  The function \texttt{globallySimplify} works
identically, except it doesn't issue tracing messages.

For even quite small regular expressions,
\texttt{globallySimplified} will fail to run to completion in an
acceptable time-frame, and one will have to bound the size of the set
$X$ in order for \texttt{globallySimplify} and
\texttt{globallySimplifyTrace} to run to completion in an
acceptable time-frame.

Here are some examples of how these functions can be used.
\begin{list}{}
{\setlength{\leftmargin}{\leftmargini}
\setlength{\rightmargin}{0cm}
\setlength{\itemindent}{0cm}
\setlength{\listparindent}{0cm}
\setlength{\itemsep}{0cm}
\setlength{\parsep}{0cm}
\setlength{\labelsep}{0cm}
\setlength{\labelwidth}{0cm}
\catcode`\#=12
\catcode`\$=12
\catcode`\%=12
\catcode`\^=12
\catcode`\_=12
\catcode`\.=12
\catcode`\?=12
\catcode`\!=12
\catcode`\&=12
\ttfamily}
\small
\item[]\textsl{-\ }Reg.size(Reg.fromString\ "(00\symbol{'052}1)\symbol{'052}");
\item[]\textsl{val\ it\ =\ 7\ :\ int}
\item[]\textsl{-\ }Reg.globallySimplifyTrace\ (NONE,\ Reg.obviousSubset)
\item[]\textsl{=\ }(Reg.fromString\ "(00\symbol{'052}1)\symbol{'052}");
\item[]\textsl{considering\ candidates\ with\ explanations\ of\ length\ 0}
\item[]\textsl{simplest\ result\ now:\ (00\symbol{'052}1)\symbol{'052}}
\item[]\textsl{considering\ candidates\ with\ explanations\ of\ length\ 1}
\item[]\textsl{simplest\ result\ now:\ (00\symbol{'052}1)\symbol{'052}\ transformed\ by\ reduction\ rule\ 16\ at}
\item[]\textsl{position\ \symbol{'133}\symbol{'135}\ to\ %\ +\ 0(0\ +\ 10)\symbol{'052}1}
\item[]\textsl{considering\ candidates\ with\ explanations\ of\ length\ 2}
\item[]\textsl{simplest\ result\ now:\ (00\symbol{'052}1)\symbol{'052}\ transformed\ by\ reduction\ rule\ 16\ at}
\item[]\textsl{position\ \symbol{'133}\symbol{'135}\ to\ %\ +\ 0(0\ +\ 10)\symbol{'052}1\ transformed\ by\ reduction\ rule\ 22\ at}
\item[]\textsl{position\ \symbol{'133}2,\ 2,\ 1,\ 1\symbol{'135}\ to\ %\ +\ 0((%\ +\ 1)0)\symbol{'052}1}
\item[]\textsl{considering\ candidates\ with\ explanations\ of\ length\ 3}
\item[]\textsl{considering\ candidates\ with\ explanations\ of\ length\ 4}
\item[]\textsl{considering\ candidates\ with\ explanations\ of\ length\ 5}
\item[]\textsl{considering\ candidates\ with\ explanations\ of\ length\ 6}
\item[]\textsl{considering\ candidates\ with\ explanations\ of\ length\ 7}
\item[]\textsl{search\ completed\ after\ considering\ 36\ candidates\ with\ maximum\ size}
\item[]\textsl{12}
\item[]\textsl{(00\symbol{'052}1)\symbol{'052}\ transformed\ by\ reduction\ rule\ 16\ at\ position\ \symbol{'133}\symbol{'135}\ to}
\item[]\textsl{%\ +\ 0(0\ +\ 10)\symbol{'052}1\ transformed\ by\ reduction\ rule\ 22\ at\ position}
\item[]\textsl{\symbol{'133}2,\ 2,\ 1,\ 1\symbol{'135}\ to\ %\ +\ 0((%\ +\ 1)0)\symbol{'052}1\ is\ globally\ simplified}
\item[]\textsl{val\ it\ =\ (true,-)\ :\ bool\ \symbol{'052}\ reg}
\item[]\textsl{-\ }Reg.size(#2\ it);
\item[]\textsl{val\ it\ =\ 12\ :\ int}
\item[]\textsl{-\ }locSimp\ NONE\ (Reg.fromString\ "(00\symbol{'052}11\symbol{'052})\symbol{'052}");
\item[]\textsl{val\ it\ =\ (true,-)\ :\ bool\ \symbol{'052}\ reg}
\item[]\textsl{-\ }Reg.output("",\ #2\ it);
\item[]\textsl{%\ +\ 00\symbol{'052}1(%\ +\ (0\ +\ 1)\symbol{'052}1)}
\item[]\textsl{val\ it\ =\ ()\ :\ unit}
\item[]\textsl{-\ }Reg.globallySimplify\ (NONE,\ Reg.obviousSubset)
\item[]\textsl{=\ }(Reg.fromString\ "(00\symbol{'052}11\symbol{'052})\symbol{'052}");
\item[]\textsl{val\ it\ =\ (true,-)\ :\ bool\ \symbol{'052}\ reg}
\item[]\textsl{-\ }Reg.output("",\ #2\ it);
\item[]\textsl{%\ +\ 0(0\ +\ 1)\symbol{'052}1}
\item[]\textsl{val\ it\ =\ ()\ :\ unit}
\end{list}

Finally, here are two examples showing how using the distributive
rules can make a difference:
\begin{list}{}
{\setlength{\leftmargin}{\leftmargini}
\setlength{\rightmargin}{0cm}
\setlength{\itemindent}{0cm}
\setlength{\listparindent}{0cm}
\setlength{\itemsep}{0cm}
\setlength{\parsep}{0cm}
\setlength{\labelsep}{0cm}
\setlength{\labelwidth}{0cm}
\catcode`\#=12
\catcode`\$=12
\catcode`\%=12
\catcode`\^=12
\catcode`\_=12
\catcode`\.=12
\catcode`\?=12
\catcode`\!=12
\catcode`\&=12
\ttfamily}
\small
\item[]\textsl{-\ }globSimp\ (NONE,\ false)\ (Reg.fromString\ "%\ +\ 0\symbol{'052}(0\ +\ 1)");
\item[]\textsl{val\ it\ =\ (true,-)\ :\ bool\ \symbol{'052}\ reg}
\item[]\textsl{-\ }Reg.output("",\ #2\ it);
\item[]\textsl{%\ +\ 0\symbol{'052}(0\ +\ 1)}
\item[]\textsl{val\ it\ =\ ()\ :\ unit}
\item[]\textsl{-\ }Reg.globallySimplifyTrace
\item[]\textsl{=\ }(NONE,\ true,\ Reg.obviousSubset)
\item[]\textsl{=\ }(Reg.fromString\ "%\ +\ 0\symbol{'052}(0\ +\ 1)");
\item[]\textsl{considering\ candidates\ with\ explanations\ of\ length\ 0}
\item[]\textsl{simplest\ result\ now:\ %\ +\ 0\symbol{'052}(0\ +\ 1)}
\item[]\textsl{considering\ candidates\ with\ explanations\ of\ length\ 1}
\item[]\textsl{considering\ candidates\ with\ explanations\ of\ length\ 2}
\item[]\textsl{considering\ candidates\ with\ explanations\ of\ length\ 3}
\item[]\textsl{considering\ candidates\ with\ explanations\ of\ length\ 4}
\item[]\textsl{simplest\ result\ now:\ %\ +\ 0\symbol{'052}(0\ +\ 1)\ transformed\ by\ distributive}
\item[]\textsl{rule\ 1\ at\ position\ \symbol{'133}2\symbol{'135}\ to\ %\ +\ 0\symbol{'052}0\ +\ 0\symbol{'052}1\ transformed\ by\ structural}
\item[]\textsl{rule\ 2\ at\ position\ \symbol{'133}\symbol{'135}\ to\ (%\ +\ 0\symbol{'052}0)\ +\ 0\symbol{'052}1\ transformed\ by\ reduction}
\item[]\textsl{rule\ 26\ at\ position\ \symbol{'133}1\symbol{'135}\ to\ 0\symbol{'052}\ +\ 0\symbol{'052}1\ transformed\ by\ reduction\ rule}
\item[]\textsl{21\ at\ position\ \symbol{'133}\symbol{'135}\ to\ 0\symbol{'052}(%\ +\ 1)}
\item[]\textsl{considering\ candidates\ with\ explanations\ of\ length\ 5}
\item[]\textsl{considering\ candidates\ with\ explanations\ of\ length\ 6}
\item[]\textsl{considering\ candidates\ with\ explanations\ of\ length\ 7}
\item[]\textsl{considering\ candidates\ with\ explanations\ of\ length\ 8}
\item[]\textsl{considering\ candidates\ with\ explanations\ of\ length\ 9}
\item[]\textsl{considering\ candidates\ with\ explanations\ of\ length\ 10}
\item[]\textsl{considering\ candidates\ with\ explanations\ of\ length\ 11}
\item[]\textsl{search\ completed\ after\ considering\ 76\ candidates\ with\ maximum\ size}
\item[]\textsl{11}
\item[]\textsl{%\ +\ 0\symbol{'052}(0\ +\ 1)\ transformed\ by\ distributive\ rule\ 1\ at\ position\ \symbol{'133}2\symbol{'135}}
\item[]\textsl{to\ %\ +\ 0\symbol{'052}0\ +\ 0\symbol{'052}1\ transformed\ by\ structural\ rule\ 2\ at\ position\ \symbol{'133}\symbol{'135}}
\item[]\textsl{to\ (%\ +\ 0\symbol{'052}0)\ +\ 0\symbol{'052}1\ transformed\ by\ reduction\ rule\ 26\ at\ position}
\item[]\textsl{\symbol{'133}1\symbol{'135}\ to\ 0\symbol{'052}\ +\ 0\symbol{'052}1\ transformed\ by\ reduction\ rule\ 21\ at\ position\ \symbol{'133}\symbol{'135}\ to}
\item[]\textsl{0\symbol{'052}(%\ +\ 1)\ is\ globally\ simplified}
\item[]\textsl{val\ it\ =\ (true,-)\ :\ bool\ \symbol{'052}\ reg}
\item[]\textsl{-\ }globSimp\ (NONE,\ false)\ (Reg.fromString\ "(0(0(0\ +\ 1))\symbol{'052})\symbol{'052}");
\item[]\textsl{val\ it\ =\ (true,-)\ :\ bool\ \symbol{'052}\ reg}
\item[]\textsl{-\ }Reg.output("",\ #2\ it);
\item[]\textsl{%\ +\ 0(0(%\ +\ 0\ +\ 1))\symbol{'052}}
\item[]\textsl{val\ it\ =\ ()\ :\ unit}
\item[]\textsl{-\ }globSimp\ (NONE,\ true)\ (Reg.fromString\ "(0(0(0\ +\ 1))\symbol{'052})\symbol{'052}");
\item[]\textsl{val\ it\ =\ (true,-)\ :\ bool\ \symbol{'052}\ reg}
\item[]\textsl{-\ }Reg.output("",\ #2\ it);
\item[]\textsl{%\ +\ 0(0(%\ +\ 1))\symbol{'052}}
\item[]\textsl{val\ it\ =\ ()\ :\ unit}
\end{list}


\subsection{Notes}

Although books on formal language theory usually study various regular
expression equivalences, we have gone much further, giving three at
least partly novel algorithms for regular expression simplification.
Although many of the simplification and structural rules used in the
simplification algorithms are well-known, some were invented, as was
the concept of closure complexity.

\index{regular expression!simplification|)}%
\index{simplification!regular expression|)}%
\index{regular expression|)}%

%%% Local Variables: 
%%% mode: latex
%%% TeX-master: "book"
%%% End: 

\section{Finite Automata and Labeled Paths}
\label{FiniteAutomataAndLabeledPaths}

\index{finite automaton|(}
\index{FA|(}

In this section, we: say what finite automata (FA) are, and show how
they can be processed using Forlan; say what labeled paths are, and
show how they can be processed using Forlan; and use the notion of
labeled path to say what finite automata mean.

\subsection{Finite Automata}

A \emph{finite automaton} (FA) $M$ consists of:
\begin{itemize}
\item a finite set $Q_M$ of symbols (we call the elements of $Q_M$
the \emph{states} of $M$);

\item an element $s_M$ of $Q_M$ (we call $s_M$ the \emph{start state}
of $M$);

\item a subset $A_M$ of $Q_M$ (we call the elements of $A_M$ the
\emph{accepting states} of $M$);

\item a finite subset $T_M$ of $\setof{(q,x,r)}{q,r\in Q_M\eqtxt{and}
x\in\Str}$ (we call the elements of $T_M$ the \emph{transitions} of
$M$, and we often write $(q, x, r)$ as
\begin{gather*}
q\tranarr{x}r
\end{gather*}
or $q,x\fun r$).
\end{itemize}
\index{finite automaton!states}%
\index{finite automaton!start state}%
\index{finite automaton!accepting states}%
\index{finite automaton!transitions}%

We order transitions first by their left-hand sides, then by their
middles, and then by their right-hand sides, using our total orderings
on symbols and strings.  This gives us a total ordering on
transitions.

We often abbreviate $Q_M$, $s_M$, $A_M$ and $T_M$ to $Q$, $s$, $A$ and
$T$, when it's clear which FA we are working with.  Whenever possible,
we will use the mathematical variables $p$, $q$ and $r$ to name
states.  We write $\FA$ for the set of all finite automata, which is a
\index{finite automaton!FA@$\FA$}%
countably infinite set.

As an example, we can define an FA $M$ as follows:
\begin{itemize}
\item $Q_M=\{\mathsf{A,B,C}\}$;

\item $s_M=\Asf$;

\item $A_M=\{\Asf, \Csf\}$; and

\item $T_M=\{\mathsf{(A,1,A), (B,11,B), (C,111,C),
  (A,0,B),(A,2,B),(A,0,C),(A,2,C),}\abr\mathsf{(B,0,C), (B,2,C)}\}$.
\end{itemize}

Finite automata are \emph{nondeterministic} machines that take strings as
inputs.  When a machine is run on a given input, it begins in
its start state.

If, after some number of steps, the machine is in state $p$, the
machine's remaining input begins with $x$, and one of the machine's
transitions is $p,x\fun q$, then the machine \emph{may} read $x$ from
its input and switch to state $q$.  If $p, y\fun r$ is also a
transition, and the remaining input begins with $y$, then consuming
$y$ and switching to state $r$ will also be possible, etc.  The case
when $x=\%$, i.e., when we have a $\%$-\emph{transition}, is
interesting: a state switch can happen without reading anything.

If \emph{at least one} execution sequence consumes all of the machine's input
and takes it to one of its accepting states, then we say that the
input is \emph{accepted} by the machine; otherwise, we say that the
input is \emph{rejected}.  The meaning of a machine is the language
consisting of all strings that it accepts.

\index{finite automaton!Forlan syntax}%
Here is how our example FA $M$ can be expressed in Forlan's syntax:
\verbatiminput{3.4-fa}
Since whitespace characters are ignored by Forlan's input routines,
the preceding description of $M$ could have been formatted in many other
ways.  States are separated by commas, and transitions are separated
by semicolons.  The order of states and transitions is irrelevant.

Transitions that only differ in their right-hand states can be
merged into single transition families.  E.g., we can merge
\begin{verbatim}
A, 0 -> B
\end{verbatim}
and
\begin{verbatim}
A, 0 -> C
\end{verbatim}
into the transition family
\begin{verbatim}
A, 0 -> B | C
\end{verbatim}

The Forlan module \texttt{FA} defines an abstract type \texttt{fa} (in
\index{FA@\texttt{FA}}%
\index{FA@\texttt{FA}!fa@\texttt{fa}}%
the top-level environment) of finite automata, as well as a large
number of functions and constants for processing FAs, including:
\begin{verbatim}
val input  : string -> fa
val output : string * fa -> unit 
\end{verbatim}
\index{FA@\texttt{FA}!input@\texttt{input}}%
\index{FA@\texttt{FA}!output@\texttt{output}}%
Remember that it's possible to read input from a file, and to write
output to a file.  During printing, Forlan merges transitions into
transition families whenever possible.

Suppose that our example FA is in the file \texttt{3.4-fa}.
We can input this FA into Forlan, and then output it to the
standard output, as follows:
\begin{list}{}
{\setlength{\leftmargin}{\leftmargini}
\setlength{\rightmargin}{0cm}
\setlength{\itemindent}{0cm}
\setlength{\listparindent}{0cm}
\setlength{\itemsep}{0cm}
\setlength{\parsep}{0cm}
\setlength{\labelsep}{0cm}
\setlength{\labelwidth}{0cm}
\catcode`\#=12
\catcode`\$=12
\catcode`\%=12
\catcode`\^=12
\catcode`\_=12
\catcode`\.=12
\catcode`\?=12
\catcode`\!=12
\catcode`\&=12
\ttfamily}
\small
\item[]\textsl{-\ }val\ fa\ =\ FA.input\ "3.4-fa";
\item[]\textsl{val\ fa\ =\ -\ :\ fa}
\item[]\textsl{-\ }FA.output("",\ fa);
\item[]\textsl{\symbol{'173}states\symbol{'175}\ A,\ B,\ C\ \symbol{'173}start\ state\symbol{'175}\ A\ \symbol{'173}accepting\ states\symbol{'175}\ A,\ C}
\item[]\textsl{\symbol{'173}transitions\symbol{'175}}
\item[]\textsl{A,\ 0\ ->\ B\ |\ C;\ A,\ 1\ ->\ A;\ A,\ 2\ ->\ B\ |\ C;\ B,\ 0\ ->\ C;\ B,\ 2\ ->\ C;}
\item[]\textsl{B,\ 11\ ->\ B;\ C,\ 111\ ->\ C}
\item[]\textsl{val\ it\ =\ ()\ :\ unit}
\end{list}


We also make use of graphical notation for finite automata.
Each of the states of a machine is circled,
and its accepting states are double-circled.  The machine's start state is
pointed to by an arrow coming from ``Start'', and
each transition $p, x\fun q$ is drawn as an arrow from state $p$
to state $q$ that is labeled by the string $x$.  Multiple labeled
arrows from one state to another can be abbreviated to a single
arrow, whose label consists of the comma-separated list of the
labels of the original arrows.

Here is how our FA $M$ can be described graphically:
\begin{center}
\input{chap-3.4-fig1.eepic}
\end{center}

The Java program JForlan, can be used to view and edit finite
automata.  It can be invoked directly, or run via Forlan.  See the
Forlan website for more information.

\index{finite automaton!alphabet}%
\index{finite automaton!alphabet@$\alphabet$}%
We define a function $\alphabet\in\FA\fun\Alp$ by: for all $M\in\FA$,
$\alphabet\,M$ is $\setof{a\in\Sym}{\eqtxt{there are}q,x,r\eqtxt{such
    that} q,x\fun r\in T_M\eqtxt{and}a\in\alphabet\,x}$.  I.e.,
$\alphabet\,M$ is all of the symbols appearing in the strings of $M$'s
transitions.  We say that $\alphabet\,M$ is \emph{the alphabet of}
$M$.  For example, the alphabet of our example FA $M$ is
$\{\mathsf{0,1,2}\}$.

\index{finite automaton!sub-FA}%
We say that an FA $M$ is a \emph{sub-FA} of an FA $N$ iff:
\begin{itemize}
\item $Q_M\sub Q_N$;
\item $s_M = s_N$;
\item $A_M\sub A_N$; and
\item $T_M\sub T_N$.
\end{itemize}
Thus $M=N$ iff $M$ is a sub-FA of $N$ and $N$ is a sub-FA of $M$.

The Forlan module \texttt{FA} contains the functions
\begin{verbatim}
val equal          : fa * fa -> bool
val numStates      : fa -> int
val numTransitions : fa -> int
val alphabet       : fa -> sym set
val sub            : fa * fa -> bool
\end{verbatim}
\index{FA@\texttt{FA}!equal@\texttt{equal}}%
\index{FA@\texttt{FA}!numStates@\texttt{numStates}}%
\index{FA@\texttt{FA}!numTransitions@\texttt{numTransitions}}%
\index{FA@\texttt{FA}!alphabet@\texttt{alphabet}}%
\index{FA@\texttt{FA}!sub@\texttt{sub}}%
The function \texttt{equal} tests whether two FAs are equal, i.e.,
whether they have the same states, start states, accepting states
and transitions.
The functions \texttt{numStates} and \texttt{numTransitions} return
the numbers of states and transitions, respectively, of an FA.
The function \texttt{alphabet} returns the alphabet of an FA.  
And the function \texttt{sub} tests whether a first FA is a sub-FA of
a second FA.

For example, we can continue out Forlan session as follows:
\begin{list}{}
{\setlength{\leftmargin}{\leftmargini}
\setlength{\rightmargin}{0cm}
\setlength{\itemindent}{0cm}
\setlength{\listparindent}{0cm}
\setlength{\itemsep}{0cm}
\setlength{\parsep}{0cm}
\setlength{\labelsep}{0cm}
\setlength{\labelwidth}{0cm}
\catcode`\#=12
\catcode`\$=12
\catcode`\%=12
\catcode`\^=12
\catcode`\_=12
\catcode`\.=12
\catcode`\?=12
\catcode`\!=12
\catcode`\&=12
\ttfamily}
\small
\item[]\textsl{-\ }val\ fa'\ =\ FA.input\ "";
\item[]\textsl{@\ }\symbol{'173}states\symbol{'175}\ A,\ B,\ C
\item[]\textsl{@\ }\symbol{'173}start\ state\symbol{'175}\ A
\item[]\textsl{@\ }\symbol{'173}accepting\ states\symbol{'175}\ C
\item[]\textsl{@\ }\symbol{'173}transitions\symbol{'175}
\item[]\textsl{@\ }A,\ 0\ ->\ B;\ A,\ 2\ ->\ B;\ A,\ 0\ ->\ C;\ A,\ 2\ ->\ C;
\item[]\textsl{@\ }B,\ 0\ ->\ C;\ B,\ 2\ ->\ C
\item[]\textsl{@\ }.
\item[]\textsl{val\ fa'\ =\ -\ :\ fa}
\item[]\textsl{-\ }FA.equal(fa',\ fa);
\item[]\textsl{val\ it\ =\ false\ :\ bool}
\item[]\textsl{-\ }FA.sub(fa',\ fa);
\item[]\textsl{val\ it\ =\ true\ :\ bool}
\item[]\textsl{-\ }FA.sub(fa,\ fa');
\item[]\textsl{val\ it\ =\ false\ :\ bool}
\item[]\textsl{-\ }FA.numStates\ fa;
\item[]\textsl{val\ it\ =\ 3\ :\ int}
\item[]\textsl{-\ }FA.numTransitions\ fa;
\item[]\textsl{val\ it\ =\ 9\ :\ int}
\item[]\textsl{-\ }SymSet.output("",\ FA.alphabet\ fa);
\item[]\textsl{0,\ 1,\ 2}
\item[]\textsl{val\ it\ =\ ()\ :\ unit}
\end{list}


\subsection{Labeled Paths and FA Meaning}

We will formally explain when strings are accepted by finite automata
\index{finite automaton!labeled path|see{labeled path}}%
\index{labeled path}%
\index{labeled path!LP@$\LP$}%
using the notion of a labeled path.  A \emph{labeled path} consists of
a pair $(\xs,q)$, where $\xs\in\List(\Sym\times\Str)$ and $q\in\Sym$,
and the set $\LP$ of labeled paths is
$\List(\Sym\times\Str)\times\Sym$.  Clearly, $\LP$ is countably
infinite.  We typically write
$([(q_1,x_1),(q_2,x_2)\ldots,(q_n,x_n)],q_n+1)\in\LP$ as:
\begin{gather*}
q_1\lparr{x_1}q_2\lparr{x_2}\cdots\,q_n\lparr{x_n}q_{n+1}
\end{gather*}
or
\begin{gather*}
q_1,x_1\Rightarrow q_2,x_2\Rightarrow\cdots\,q_n,x_n\Rightarrow q_{n+1} .
\end{gather*}
This path describes a way of getting from state $q_1$ to state $q_{n+1}$
in some unspecified machine, by reading the strings
$x_1,\,\ldots,x_n$ from the machine's input.  We start out in
state $q_1$, make use of the transition $q_1,x_1\fun q_2$ to read
$x_1$ from the input and switch to state $q_2$, etc.

Let $\lp=(\xs,q)\in\LP$.
We say that:
\begin{itemize}
\item the \emph{start state} of $\lp$ ($\startState\,\lp$) is
  the left-hand side of the first element of $\xs$, if $\xs$ is nonempty,
  and is $q$, if $\xs$ is empty;

\item the \emph{end state} of $\lp$ ($\myendState\,\lp$) is $q$;

\item the \emph{length} of $\lp$ ($|\lp|$) is $|\xs|$; and

\item the \emph{label} of $\lp$ ($\mylabel\,\lp$) is the result of
  concatenating the right-hand sides of $\xs$ ($\%$, if $\xs$ is
  empty).
\end{itemize}
\index{labeled path!startState@$\startState$}%
\index{labeled path!endState@$\myendState$}%
\index{labeled path!length}%
\index{labeled path!label@$\mylabel$}%

This defines functions $\startState\in\LP\fun\Sym$,
$\myendState\in\LP\fun\Sym$ and $\mylabel\in\LP\fun\Str$.
For example $\Asf = ([\,],\Asf)$
is a labeled path whose start and end states are both $\Asf$, whose
length is $0$, and whose label is $\%$.  And
\begin{gather*}
\Asf\lparr{\mathsf{0}}\Bsf\lparr{\mathsf{11}}\Bsf\lparr{\twosf}\Csf
\end{gather*}
is a labeled path whose start state is $\Asf$, end state is $\Csf$,
length is $3$, and label is $\mathsf{(0)(11)(2)}=\mathsf{0112}$.

We can join compatible paths together.  Let $\JOIN=
\setof{(\lp_1,\lp_2)\in\LP\times\LP}{\myendState\,\lp_1 = \startState\,\lp_2}$,
and define $\join\in\JOIN\fun\LP$ by: for all $\xs,\ys\in\List(\Sym\times\Str)$
and $q,r\in\Sym$, if $((\xs,q),(\ys,r))\in\JOIN$, then
$\join((\xs,q), (\ys,r)) = (\xs\myconcat\ys, r)$.  E.g., if
$\lp_1$ and $\lp_2$ are defined by
\begin{gather*}
\lp_1 = 
\Asf\lparr{\mathsf{0}}\Bsf\lparr{\mathsf{11}}\Bsf , \quad\eqtxt{and}\quad
\lp_2 =
\Bsf\lparr{\mathsf{11}}\Bsf\lparr{\mathsf{2}}\Csf ,
\end{gather*}
then $\join(\lp_1,\lp_2)$ is
\begin{gather*}
\Asf\lparr{\mathsf{0}}\Bsf\lparr{\mathsf{11}}\Bsf
\lparr{\mathsf{11}}\Bsf\lparr{\mathsf{2}}\Csf .
\end{gather*}

A labeled path $(\xs,q)\in\LP$ is \emph{valid for} an FA $M$ iff
\begin{itemize}
\item for all $i\in[1:|\xs|-1]$, $\hash{1}(\xs\,i),\hash{2}(\xs\,i)\fun
  \hash{1}(\xs(i+1))$;

\item if $\xs$ is nonempty, then $\hash{1}(\xs\,|\xs|),\hash{2}(\xs\,|\xs|)\fun
  q$; and

\item $q\in Q_M$.
\end{itemize}
(The last of these conditions is redundant whenever $\xs$ is nonempty.)

Recall our example FA $M$:
\begin{center}
\input{chap-3.4-fig1.eepic}
\end{center}
The labeled path $\Asf=([\,],\Asf)$ is valid for $M$,
since $A\in Q_M$.  And
\begin{gather*}
\Asf\lparr{\mathsf{0}}\Bsf\lparr{\mathsf{11}}\Bsf\lparr{\twosf}\Csf
\end{gather*}
is valid for $M$, since $\Asf\tranarr{\zerosf}\Bsf$,
$\Bsf\tranarr{\mathsf{11}}\Bsf$ and $\Bsf\tranarr{\twosf}\Csf$ are in
$T_M$ (and $\Csf\in Q_M$). But the labeled path
\begin{gather*}
\Asf\lparr{\%}\Asf
\end{gather*}
is not valid for $M$, since $\Asf,\%\fun\Asf\not\in T_M$.

\index{finite automaton!accepted by}%
A string $w$ is \emph{accepted by} a finite automaton $M$ iff
there is a labeled path $\lp$ such that
\begin{itemize}
\item $\lp$ is valid for $M$;

\item the label of $\lp$ is $w$;

\item the start state of $\lp$ is the start state of $M$; and

\item the end state of $\lp$ is an accepting state of $M$.
\end{itemize}
For example, $\mathsf{0112}$ is accepted by
$M$ because of the labeled path
\begin{gather*}
\Asf\lparr{\mathsf{0}}\Bsf\lparr{\mathsf{11}}\Bsf\lparr{\twosf}\Csf ,
\end{gather*}
since this labeled path is valid for $M$, is labeled by $\mathsf{0112} =
\mathsf{(0)(11)(2)}$, has a start state ($\Asf$) that is $M$'s start state,
and has an end state ($\Csf$) that is one of $M$'s accepting states.

\index{finite automaton!meaning}%
\index{finite automaton!language accepted by}%
\index{L(@$L(\cdot)$}%
\index{finite automaton!L(@$L(\cdot)$}%}%
Clearly, if $w$ is accepted by $M$, then
$\alphabet\,w\sub\alphabet\,M$.  Thus $\setof{w\in\Str}{w\eqtxt{is
    accepted by}M}\sub(\alphabet\,M)^*$, so we may define the
\emph{language accepted by} a finite automaton $M$ ($L(M)$) to be
\begin{gather*}
\setof{w\in\Str}{w\eqtxt{is accepted by}M}.
\end{gather*}
Furthermore:

\begin{proposition}
Suppose $M$ is a finite automaton.  Then $\alphabet(L(M))\sub\alphabet\,M$.
\end{proposition}

In other words, every symbol of every string that is accepted by $M$
comes from the alphabet of $M$, i.e., appears in the label of one of
$M$'s transitions.

Going back to our example, we have that
\begin{align*}
L(M) &= \{\onesf\}^* \cup {}\\
     &\quad\;\,\mathsf{\{1\}^*\{0,2\}\{11\}^*\{0,2\}\{111\}^*} \cup {} \\
     &\quad\;\,\mathsf{\{1\}^*\{0,2\}\{111\}^*} .
\end{align*}
For example, $\%$, $\mathsf{11}$, $\mathsf{110112111}$ and
$\mathsf{2111111}$ are accepted by $M$.  But $\mathsf{21112}$ and
$\mathsf{2211}$ are not accepted by $M$.

Suppose that $M$ is a sub-FA of $N$.  Then any labeled path that is
valid for $M$ will also be valid for $N$.  Furthermore, we have that:

\begin{proposition}
If $M$ is a sub-FA of $N$, then $L(M)\sub L(N)$.
\end{proposition}

\index{regular expression!equivalence|(}%
\index{ equiv@$\approx$}%
\index{regular expression! equiv@$\approx$}%
We say that finite automata $M$ and $N$ are \emph{equivalent} iff
$L(M) = L(N)$.  In other words, $M$ and $N$ are equivalent iff $M$ and
$N$ accept the same language.  We define a relation $\approx$ on $\FA$
by: $M\approx N$ iff $M$ and $N$ are equivalent.  It is easy to see
that $\approx$ is reflexive on $\FA$, symmetric and transitive.

The Forlan module \texttt{LP} defines an abstract type \texttt{lp} (in the
top-level environment) of labeled paths, as well as various functions
for processing labeled paths, including:
\begin{verbatim}
val input      : string -> lp
val output     : string * lp -> unit
val equal      : lp * lp -> bool
val startState : lp -> sym
val endState   : lp -> sym
val label      : lp -> str
val length     : lp -> int
val join       : lp * lp -> lp
\end{verbatim}
\index{LP@\texttt{LP}}%
\index{LP@\texttt{LP}!input@\texttt{input}}%
\index{LP@\texttt{LP}!output@\texttt{output}}%
\index{LP@\texttt{LP}!startState@\texttt{startState}}%
\index{LP@\texttt{LP}!endState@\texttt{endState}}%
\index{LP@\texttt{LP}!label@\texttt{label}}%
\index{LP@\texttt{LP}!length@\texttt{length}}%
\index{LP@\texttt{LP}!join@\texttt{join}}%
The function \texttt{equal} tests whether two labeled paths are
equal.  The functions \texttt{startState}, \texttt{endState},
\texttt{label} and \texttt{length} return the start state, end
state, label and length, respectively, of a labeled path.  And
the function \texttt{join} joins two compatible paths, and issues
an error message when given paths that are incompatible.

The module \texttt{FA} also defines the functions
\begin{verbatim}
val checkLP : fa -> lp -> unit
val validLP : fa -> lp -> bool
\end{verbatim}
\index{FA@\texttt{FA}!checkLP@\texttt{checkLP}}%
\index{FA@\texttt{FA}!validLP@\texttt{validLP}}%
for checking whether a labeled path is valid in a finite automaton.
These are curried functions---functions that return functions
as their results.
The function \texttt{checkLP} takes in an FA $M$ and returns a function
that checks whether a labeled path $\lp$ is valid for $M$.  When
$\lp$ is not valid for $M$, the function explains why it isn't;
otherwise, it prints nothing. 
And, the function \texttt{validLP} takes in an FA $M$ and returns a function
that tests whether a labeled path $\lp$ is valid for $M$, silently
returning \texttt{true}, if it is, and silently returning \texttt{false},
otherwise.

Here are some examples of labeled path and FA processing
(\texttt{fa} is still our example FA):
\begin{list}{}
{\setlength{\leftmargin}{\leftmargini}
\setlength{\rightmargin}{0cm}
\setlength{\itemindent}{0cm}
\setlength{\listparindent}{0cm}
\setlength{\itemsep}{0cm}
\setlength{\parsep}{0cm}
\setlength{\labelsep}{0cm}
\setlength{\labelwidth}{0cm}
\catcode`\#=12
\catcode`\$=12
\catcode`\%=12
\catcode`\^=12
\catcode`\_=12
\catcode`\.=12
\catcode`\?=12
\catcode`\!=12
\catcode`\&=12
\ttfamily}
\small
\item[]\textsl{-\ }val\ lp\ =\ LP.input\ "";
\item[]\textsl{@\ }A,\ 1\ =>\ A,\ 0\ =>\ B,\ 11\ =>\ B,\ 2\ =>\ C,\ 111\ =>\ C
\item[]\textsl{@\ }.
\item[]\textsl{val\ lp\ =\ -\ :\ lp}
\item[]\textsl{-\ }Sym.output("",\ LP.startState\ lp);
\item[]\textsl{A}
\item[]\textsl{val\ it\ =\ ()\ :\ unit}
\item[]\textsl{-\ }Sym.output("",\ LP.endState\ lp);
\item[]\textsl{C}
\item[]\textsl{val\ it\ =\ ()\ :\ unit}
\item[]\textsl{-\ }LP.length\ lp;
\item[]\textsl{val\ it\ =\ 5\ :\ int}
\item[]\textsl{-\ }Str.output("",\ LP.label\ lp);
\item[]\textsl{10112111}
\item[]\textsl{val\ it\ =\ ()\ :\ unit}
\item[]\textsl{-\ }val\ checkLP\ =\ FA.checkLP\ fa;
\item[]\textsl{val\ checkLP\ =\ fn\ :\ lp\ ->\ unit}
\item[]\textsl{-\ }checkLP\ lp;
\item[]\textsl{val\ it\ =\ ()\ :\ unit}
\item[]\textsl{-\ }val\ lp'\ =\ LP.fromString\ "A";
\item[]\textsl{val\ lp'\ =\ -\ :\ lp}
\item[]\textsl{-\ }LP.length\ lp';
\item[]\textsl{val\ it\ =\ 0\ :\ int}
\item[]\textsl{-\ }Str.output("",\ LP.label\ lp');
\item[]\textsl{%}
\item[]\textsl{val\ it\ =\ ()\ :\ unit}
\item[]\textsl{-\ }checkLP\ lp';
\item[]\textsl{val\ it\ =\ ()\ :\ unit}
\item[]\textsl{-\ }val\ lp''\ =\ LP.input\ "";
\item[]\textsl{@\ }A,\ %\ =>\ A,\ 1\ =>\ B
\item[]\textsl{@\ }.
\item[]\textsl{val\ lp''\ =\ -\ :\ lp}
\item[]\textsl{-\ }checkLP\ lp'';
\item[]\textsl{invalid\ transition:\ "A,\ %\ ->\ A"}
\item[]
\item[]\textsl{uncaught\ exception\ Error}
\item[]\textsl{-\ }val\ lp'''\ =\ LP.input\ "";
\item[]\textsl{@\ }B,\ 2\ =>\ C,\ 34\ =>\ D
\item[]\textsl{@\ }.
\item[]\textsl{val\ lp'''\ =\ -\ :\ lp}
\item[]\textsl{-\ }LP.output("",\ LP.join(lp'',\ lp'''));
\item[]\textsl{A,\ %\ =>\ A,\ 1\ =>\ B,\ 2\ =>\ C,\ 34\ =>\ D}
\item[]\textsl{val\ it\ =\ ()\ :\ unit}
\item[]\textsl{-\ }LP.output("",\ LP.join(lp''',\ lp''));
\item[]\textsl{incompatible\ labeled\ paths}
\item[]
\item[]\textsl{uncaught\ exception\ Error}
\end{list}


\subsection{Design of Finite Automata}

\index{finite automaton!design|(}%
In this subsection, we give two examples of finite automata
design.  First, let's find a finite automaton that accepts the set
of all strings of $\zerosf$'s and $\onesf$'s with an even number of
$\zerosf$'s.  Thus, we should be looking for an FA whose alphabet is
$\{\mathsf{0,1}\}^*$.  It seems reasonable that our machine have two
states: an accepting state $\Asf$ corresponding to the strings of
$\zerosf$'s and $\onesf$'s with an even number of zeros, and a state
$\Bsf$ corresponding to the strings of $\zerosf$'s and $\onesf$'s with
an odd number of zeros.  Processing a $\onesf$ in either state should
cause us to stay in that state, but processing a $\zerosf$ in one of
the states should cause us to switch to the other state.  The above
considerations lead us to the FA
\begin{center}
\input{chap-3.4-fig2.eepic}
\end{center}

For the second example, let's find an FA that accepts the language
$X=\setof{w\in\mathsf{\{0,1,2\}^*}}{\eqtxtr{for all substrings}x
  \eqtxt{of}w,\eqtxt{if}|x|=2,\eqtxt{then}x\in\{\mathsf{01,12,20}\}}$.
We have that $\mathsf{0}$, $\mathsf{01}$, $\mathsf{012}$,
$\mathsf{0120}$, etc., are in $X$, and so are $\mathsf{1}$,
$\mathsf{12}$, $\mathsf{120}$, $\mathsf{1201}$, etc., and
$\mathsf{2}$, $\mathsf{20}$, $\mathsf{201}$, $\mathsf{2012}$, etc.  On
the other hand, no string containing $\mathsf{00}$, $\mathsf{02}$,
$\mathsf{11}$, $\mathsf{10}$, $\mathsf{22}$ or $\mathsf{20}$ is in
$X$.  The above observations suggest that part of our machine should
look like:
\begin{center}
\input{chap-3.4-fig4.eepic}
\end{center}
But how should the machine get started?  The simplest approach is to
make use of $\%$-transitions from the start state, giving us
the FA
\begin{center}
\input{chap-3.4-fig3.eepic}
\end{center}

\begin{exercise}
Let $X=\setof{w\in\{\mathsf{0,1}\}^*}{\mathsf{010}\eqtxt{is not a
    substring of}w}$.  Find a finite automaton $M$ such that $L(M)=X$.
\end{exercise}

\begin{exercise}
Let
\begin{align*}
A &=\{\mathsf{001, 011, 101, 111}\} , \eqtxt{and} \\
B &=\{\,w\in\{\mathsf{0,1}\}^* \mid \eqtxtr{for all}x,y\in\{\mathsf{0,1}\}^*,
\eqtxt{if}w=x\zerosf y, \eqtxtr{then there is a} z\in A \\
&\quad\;\;\; \eqtxt{such that} z\eqtxt{is a prefix of} y\,\}.
\end{align*}
Find a finite automaton $M$ such that $L(M)=B$.  Hint: see the second
example of Subsection~\ref{ProvingTheCorrectnessOfRegularExpressions}.
\end{exercise}

\index{finite automaton!design|)}%

\subsection{Notes}

Finite automata are normally defined via transition functions,
$\delta$, which is simple to do for deterministric finite automata,
but increasingly complicated as one adds degrees of nontermininism.
Furthermore, this approach means that a deterministic finite automaton
(DFA) is not a nondeterministic finite automaton (NFA), and that an
NFA is not an $\epsilon$-NFA, because the transition function of a DFA
is not one for an NFA, and the transition function for an NFA is not
one of an $\epsilon$-NFA.  And formalizing our FAs, whose transition
labels can be strings of length greater than one, is very messy if
done via transition functions, which probably accounts for why such
machines are not normally considered.  Furthermore, in the standard
approach, to say when strings are accepted by finite automata, one must
first extend the different kinds of transition functions to work on
strings.

In contrast, our approach is very simple.  Instead of transition
functions, we work with finite sets of transitions, enabling us to
define the deterministric finite automata, nondeterministic finite
automata, and nondeterministic finite automata with $\%$-moves (which
we call empty-string finite automata) as restrictions on finite
automata.  Furthermore, using labeled paths to say when---and
how---strings are accepted by finite automata is simple, natural and
diagrammatic.  It's the analogue of using parse trees to say
when---and how---strings are generated by grammars.

\index{finite automaton|)}
\index{FA|)}

%%% Local Variables: 
%%% mode: latex
%%% TeX-master: "book"
%%% End: 

\section{Isomorphism of Finite Automata}
\label{IsomorphismOfFiniteAutomata}

\subsection{Definition and Algorithm}

\index{isomorphism!finite automaton|(}%
\index{finite automaton!isomorphism|(}%
Let $M$ and $N$ be the finite automata
\begin{center}
\input{chap-3.5-fig1.eepic}
\end{center}
How are $M$ and $N$ related?
Although they are not equal, they do have the same
``structure'', in that $M$ can be turned into $N$ by replacing
$\Asf$, $\Bsf$ and $\Csf$ by $\Asf$, $\Csf$ and $\Bsf$, respectively.
When FAs have the same structure, we will say they are ``isomorphic''.

In order to say more formally what it means for two FAs to be
isomorphic, we define the notion of an isomorphism from one FA to
another.  An {\em isomorphism}
\index{finite automaton!isomorphism from FA to FA}%
\index{isomorphism!finite automaton!isomorphism from FA to FA}%
$h$ from an FA $M$ to an FA $N$ is a bijection from $Q_M$ to $Q_N$
such that:
\begin{itemize}
\item $h\,s_M = s_N$;

\item $\setof{h\,q}{q\in A_M} = A_N$; and

\item $\setof{(h\,q), x\fun (h\,r)}{q,x\fun r\in T_M} =
T_N$.
\end{itemize}
We define a relation $\iso$ on $\FA$ by: $M\iso N$
\index{iso@$\iso$}%
\index{finite automaton!iso@$\iso$}%
\index{isomorphism!finite automaton!iso@$\iso$}%
iff there
is an isomorphism from $M$ to $N$.  We say that $M$ and $N$ are
{\em isomorphic\/}
\index{isomorphic!finite automaton}%
\index{finite automaton!isomorphic}%
\index{isomorphism!finite automaton!isomorphic}%
iff $M\iso N$.

Consider our example FAs $M$ and $N$, and
let $h$ be the function
\begin{gather*}
\mathsf{\{(A, A), (B, C), (C, B)\}}.
\end{gather*}
Then it is easy to check that $h$ is an isomorphism from $M$ to $N$.
Hence $M\iso N$.

\begin{proposition}
\label{IsoEquivProp}
The relation $\iso$ is reflexive on $\FA$, symmetric and transitive.
\index{reflexive on set!iso@$\iso$}%
\index{symmetry!iso@$\iso$}%
\index{transitive!iso@$\iso$}%
\index{iso@$\iso$!reflexive}%
\index{iso@$\iso$!symmetric}%
\index{iso@$\iso$!transitive}%
\end{proposition}

\begin{proof}
If $M$ is an FA, then $\id_M$ is an isomorphism from $M$ to $M$.

If $M,N$ are FAs, and $h$ is a isomorphism from $M$ to $N$, then
$h^{-1}$ is an isomorphism from $N$ to $M$.

If $M_1,M_2,M_3$ are FAs, $f$ is an isomorphism from $M_1$ to $M_2$,
and $g$ is an isomorphism from $M_2$ to $M_3$, then $g\circ f$ an
isomorphism from $M_1$ to $M_3$.
\end{proof}

Next, we see that, if $M$ and $N$ are isomorphic, then
every string accepted by $M$ is also accepted by $N$.

\begin{proposition}
\label{IsoSubProp}
Suppose $M$ and $N$ are isomorphic FAs.  Then $L(M)\sub L(N)$.
\end{proposition}

\begin{proof}
Let $h$ be an isomorphism from $M$ to $N$.  Suppose $w\in L(M)$.
Then, there is a labeled path
\begin{gather*}
\lp = q_1\lparr{x_1}q_2\lparr{x_2}\cdots\,q_n\lparr{x_n}q_{n+1} ,
\end{gather*}
such that $w=x_1x_2\cdots x_n$, $\lp$ is valid for $M$,
$q_1 = s_M$ and $q_{n+1}\in A_M$.  Let
\begin{gather*}
\lp' = h\,q_1\lparr{x_1}h\,q_2\lparr{x_2}\cdots\,h\,q_n
\lparr{x_n}h\,q_{n+1}.
\end{gather*}
Then the label of $\lp'$ is $w$, $\lp'$ is valid for $N$,
$h\,q_1=h\,s_M=s_N$ and $h\,q_{n+1}\in A_N$, showing that $w\in L(N)$.
\end{proof}

A consequence of the two preceding propositions is that
isomorphic FAs are equivalent.  Of course, the converse is
not true, in general, since there are many FAs that accept the
same language and yet don't have the same structure.

\begin{proposition}
Suppose $M$ and $N$ are isomorphic FAs. Then $M\approx N$.
\end{proposition}

\begin{proof}
Since $M\iso N$, we have that $N\iso M$, by Proposition~\ref{IsoEquivProp}.
Thus, by Proposition~\ref{IsoSubProp}, we have that $L(M)\sub L(N)\sub L(M)$.
Hence $L(M)=L(N)$, i.e., $M\approx N$.
\end{proof}

The function $\renameStates$ takes in a pair $(M,f)$, where $M\in\FA$
\index{finite automaton!renaming states}%
\index{renameStates@$\renameStates$}%
\index{finite automaton!renameStates@$\renameStates$}%
and $f$ is a bijection from $Q_M$ to some set of symbols, and returns
the $\FA$ produced from $M$ by renaming $M$'s states using the
bijection $f$.

\begin{proposition}
Suppose $M$ is an FA and $f$ is a bijection from $Q_M$ to some set of
symbols.  Then $\renameStates(M,f)\iso M$.
\end{proposition}

The following function is a special case of $\renameStates$.
The function $\renameStatesCanonically\in\FA\fun\FA$ renames the
\index{renameStatesCanonically@$\renameStatesCanonically$}%
\index{finite automaton!renameStatesCanonically@$\renameStatesCanonically$}%
states of an FA $M$ to:
\begin{itemize}
\item $\mathsf{A}$, $\mathsf{B}$, etc., when the automaton has no more
than 26 states (the smallest state of $M$ will be renamed to
$\mathsf{A}$, the next smallest one to $\mathsf{B}$, etc.); or

\item $\mathsf{\langle 1\rangle}$, $\mathsf{\langle 2\rangle}$, etc.,
otherwise.
\end{itemize}
Of course, the resulting automaton will always be isomorphic to the original
one.

Next, we consider an algorithm that finds an isomorphism from an FA
\index{finite automaton!isomorphism!checking whether FAs are isomorphic}%
\index{isomorphism!finite automaton!checking whether FAs are isomorphic}%
$M$ to an FA $N$, if one exists, and that indicates that no such
isomorphism exists, otherwise.

Our algorithm is based on the following lemma.
\begin{lemma}
\label{IsoLem}
Suppose that $h$ is a bijection from $Q_M$ to $Q_N$.  Then
\begin{gather*}
\setof{h\,q \tranarr{x} h\,r}{q \tranarr{x} r\in T_M} = T_N
\end{gather*}
iff, for all $(q, r)\in h$ and
$x\in\Str$, there is a subset of $h$ that is a bijection from
\begin{gather*}
\setof{p\in Q_M}{q \tranarr{x} p\in T_M}
\end{gather*}
to
\begin{gather*}
\setof{p\in Q_N}{r \tranarr{x} p\in T_N} .
\end{gather*}
\end{lemma}

If any of the following conditions are true, then the algorithm reports
that there is no isomorphism from $M$ to $N$:
\begin{itemize}
\item $|Q_M|\neq|Q_N|$;

\item $|A_M|\neq|A_N|$;

\item $|T_M|\neq|T_N|$;

\item $s_M\in A_M$, but $s_N\not\in A_N$; and

\item $s_N\in A_N$, but $s_M\not\in A_M$.
\end{itemize}
Otherwise, it calls its main function, $\findIso$, which is
defined by well-founded recursion.

The function $\findIso$ is called with an argument $(f,
[C_1,\ldots,C_n])$, where:
\begin{itemize}
\item $f$ is a bijection from a subset of $Q_M$ to a subset of $Q_N$, and

\item the $C_i$ are \emph{constraints} of the form $(X,Y)$, where
  $X\sub Q_M$, $Y\sub Q_N$ and $|X|=|Y|$.
\end{itemize}
It returns an element of $\Option\,X$, where $X$ is the set of
bijections from $Q_M$ to $Q_N$.  $\none$ is returned to indicate
failure, and $\some\,h$ is returned when it has produced the bijection $h$.
We say that a bijection \emph{satisfies} a constraint $(X,Y)$ iff it
has a subset that is a bijection from $X$ to $Y$.

We say that the \emph{weight} of a constraint $(X,Y)$ is $3^{|X|}$.
Thus, we have the following facts:
\begin{itemize}
\item If $(X,Y)$ is a constraint, then its weight is at least $3^0=1$.

\item If $(\{p\}\cup X,\{q\}\cup Y)$ is a constraint, $p\not\in X$,
  $q\not\in Y$ and $|X|\geq 1$, then the weight of $(\{p\}\cup
  X,\{q\}\cup Y)$ is $3^{1+|X|}=3\cdot 3^{|X|}$, the weight of
  $(\{p\},\{q\})$ is $3^1=3$, and the weight of $(X,Y)$ is $3^{|X|}$.
  Because $|X|\geq 1$, it follows that the sum of the weights of
  $(\{p\},\{q\})$ and $(X,Y)$ ($3 + 3^{|X|}$) is strictly less than
  the weight of $(\{p\}\cup X,\{q\}\cup Y)$.
\end{itemize}

Each argument to a recursive call of $\findIso$ will be strictly
smaller than the argument to the original call in the well-founded
\emph{termination relation} in which argument $(f, [C_1,\ldots,C_n])$
is less than argument $(f', [C'_1,\ldots,C'_m])$ iff either:
\begin{itemize}
\item $|f|>|f'|$ (remember that $|f|\leq|Q_M|=|Q_N|$); or

\item $|f|=|f'|$ but the sum of the weights of the constraints
  $C_1,\,\ldots,C_n$ is strictly less than the sum of the weights of
  the constraints $C'_1, \,\ldots,C'_m$.
\end{itemize}

When $\findIso$ is called with argument $(f, [C_1,\ldots,C_n])$, the
following property, which we call (\dag), will hold: for all bijections
$h$ from a subset of $Q_M$ to a subset of $Q_N$, if $h\supseteq f$ and $h$
satisfies all of the $C_i$'s, then:
\begin{itemize}
\item $h$ is a bijection from $Q_M$ to $Q_N$;

\item $h\,s_M = s_N$;

\item $\setof{h\,q}{q\in A_M} = A_N$; and

\item for all $(q, r)\in f$ and $x\in\Str$, there is a subset of $h$
  that is a bijection from $\setof{p\in Q_M}{q,x\fun p\in T_M}$ to
  $\setof{p\in Q_N}{r,x\fun p\in T_N}$.
\end{itemize}
Thus, if $\findIso$ is called with a bijection $f$ and an empty list
of constraints, it will follow, by Lemma~\ref{IsoLem},
that $f$ is an isomorphism from $M$ to $N$.

Initially, the algorithm calls $\findIso$ with the \emph{initial argument}:
\begin{displaymath}
(\emptyset, [(\{s_M\}, \{s_N\}), (U, V), (X, Y)]) ,
\end{displaymath}
where $U = A_M - \{s_M\}$, $V = A_N - \{s_N\}$, $X = (Q_M - A_M) -
\{s_M\}$ and $Y= (Q_N - A_N) - \{s_N\}$. Clearly the initial
argument satisfies (\dag).

If $\findIso$ is called with argument $(f, [\,])$, then it returns
$\some\,f$.

Otherwise, if $\findIso$ is called with argument $(f, [(\emptyset,
\emptyset), C_2, \ldots, C_n])$, then it calls itself recursively with
argument $(f, [C_2, \ldots, C_n])$.  (The size of the bijection has
been preserved, but the sum of the weights of the constraints has gone
down by one.)

Otherwise, if $\findIso$ is called with argument $(f, [(\{q\},\{r\}),
C_2,\ldots,C_n])$, then it proceeds as follows:
\begin{itemize}
\item If $(q,r)\in f$, then it calls itself recursively with argument
  $(f,[C_2,\ldots,C_n])$ and returns what the recursive call returns.
  (The size of the bijection has been preserved, but the sum of the
  weights of the constraints has gone down by three.)

\item Otherwise, if $q\in\domain\,f$ or $r\in\range\,f$, then
  $\findIso$ returns $\none$.

\item Otherwise, it works its way through the strings appearing in the
  transitions of $M$ and $N$, forming a list of new constraints,
  $C'_1,\,\ldots,C'_m$.  Given such a string, $x$, it lets
  $A_x=\setof{p\in Q_M}{q,x\fun p\in T_M}$ and $B_x=\setof{p\in
    Q_N}{r,x\fun p\in T_N}$.  If $|A_x|\neq|B_x|$, then it returns
  $\none$.  Otherwise, it adds the constraint $(A_x,B_x)$ to our
  list of new constraints.  When all such strings have been exhausted,
  it calls itself recursively with argument $(f \cup \{(q,r)\},
  [C'_1,\ldots,C'_m,C_2,\ldots,C_n])$ and returns what this recursive
  call returns.  (The size of the bijection has been increased by
  one.)
\end{itemize}

Otherwise, $\findIso$ has been called with argument $(f, [(A, A'),
C_2, \ldots, C_n])$, where $|A|>1$, and it proceeds as follows.  It
picks the smallest symbol $q\in A$, and lets $B=A-\{q\}$.  Then,
it works its way through the elements of $A'$.  Given $r\in A'$, it
lets $B'=A'-\{r\}$.  Then, it tries calling itself recursively with
argument $(f, [(\{q\},\{r\}), (B,B'), C_2,\ldots, C_n])$.  (The
size of the bijection has been preserved, but the sum of the sizes of
the weights of the constraints has gone down by $2\cdot
3^{|B_1|}-3\geq 3$.)  If this call returns a result of form $\some\,h$,
then it returns this to its caller.  But if the call returns $\none$, it
tries the next element of $A'$.  If it exhausts the elements of $A'$,
then it returns $\none$.

\begin{lemma}
\label{FindIsoLem}
If $\findIso$ is called with an argument $(f,[C_1,\,\ldots,C_n])$ satisfying
property (\dag), then it returns $\none$, if there is no isomorphism
from $M$ to $N$ that is a superset of $f$ and satisfies the constaints
$C_1,\ldots,C_n$, and returns $\some\,h$ where $h$ is such an isomorphism,
otherwise.
\end{lemma}

\begin{proof}
By well-founded induction on our termination relation.  I.e., when
proving the result for $(f,[C_1,\ldots,C_n])$, we may assume that the
result holds for all arguments $(f',[C'_1,\ldots,C'_m])$ that are
strictly smaller in our termination ordering.
\end{proof}

\begin{theorem}
If $\findIso$ is called with its initial argument, then it returns
$\none$, if there is no isomorphism from $M$ to $N$, and returns
$\some\,h$ where $h$ is an isomorphism from $M$ to $N$, otherwise.
\end{theorem}

\begin{proof}
Follows easily from Lemma~\ref{FindIsoLem}.
\end{proof}

\subsection{Isomorphism Finding/Checking in Forlan}

The Forlan module \texttt{FA}
\index{FA@\texttt{FA}}%
also defines the functions
\begin{verbatim}
val isomorphism             : fa * fa * sym_rel -> bool
val findIsomorphism         : fa * fa -> sym_rel
val isomorphic              : fa * fa -> bool
val renameStates            : fa * sym_rel -> fa
val renameStatesCanonically : fa -> fa
\end{verbatim}
\index{FA@\texttt{FA}!isomorphism@\texttt{isomorphism}}%
\index{FA@\texttt{FA}!findIsomorphism@\texttt{findIsomorphism}}%
\index{FA@\texttt{FA}!isomorphic@\texttt{isomorphic}}%
\index{FA@\texttt{FA}!renameStates@\texttt{renameStates}}%
\index{FA@\texttt{FA}!renameStatesCanonically@\texttt{renameStatesCanonically}}%
The function \texttt{isomorphism} checks whether a relation on symbols
is an isomorphism from one FA to another.  The function
\texttt{findIsomorphism} tries to find an isomorphism from one FA to
another; it issues an error message if there isn't one.  The
function \texttt{isomorphic} checks whether two FAs are isomorphic.
The function \texttt{renameStates} issues an error message
if the supplied relation isn't a bijection from the set
of states of the supplied FA to some set; otherwise, it returns the
result of $\renameStates$.  And the function
\texttt{renameStatesCanonically} acts like $\renameStatesCanonically$.

Suppose \texttt{fa1} and \texttt{fa2} have been bound to our example
our example finite automata $M$ and $N$:
\begin{center}
\input{chap-3.5-fig1.eepic}
\end{center}
Then, here are some example uses of the above functions:
\begin{list}{}
{\setlength{\leftmargin}{\leftmargini}
\setlength{\rightmargin}{0cm}
\setlength{\itemindent}{0cm}
\setlength{\listparindent}{0cm}
\setlength{\itemsep}{0cm}
\setlength{\parsep}{0cm}
\setlength{\labelsep}{0cm}
\setlength{\labelwidth}{0cm}
\catcode`\#=12
\catcode`\$=12
\catcode`\%=12
\catcode`\^=12
\catcode`\_=12
\catcode`\.=12
\catcode`\?=12
\catcode`\!=12
\catcode`\&=12
\ttfamily}
\small
\item[]\textsl{-\ }val\ rel\ =\ FA.findIsomorphism(fa1,\ fa2);
\item[]\textsl{val\ rel\ =\ -\ :\ sym_rel}
\item[]\textsl{-\ }SymRel.output("",\ rel);
\item[]\textsl{(A,\ A),\ (B,\ C),\ (C,\ B)}
\item[]\textsl{val\ it\ =\ ()\ :\ unit}
\item[]\textsl{-\ }FA.isomorphism(fa1,\ fa2,\ rel);
\item[]\textsl{val\ it\ =\ true\ :\ bool}
\item[]\textsl{-\ }FA.isomorphic(fa1,\ fa2);
\item[]\textsl{val\ it\ =\ true\ :\ bool}
\item[]\textsl{-\ }val\ rel'\ =\ FA.findIsomorphism(fa1,\ fa1);
\item[]\textsl{val\ rel'\ =\ -\ :\ sym_rel}
\item[]\textsl{-\ }SymRel.output("",\ rel');
\item[]\textsl{(A,\ A),\ (B,\ B),\ (C,\ C)}
\item[]\textsl{val\ it\ =\ ()\ :\ unit}
\item[]\textsl{-\ }FA.isomorphism(fa1,\ fa1,\ rel');
\item[]\textsl{val\ it\ =\ true\ :\ bool}
\item[]\textsl{-\ }FA.isomorphism(fa1,\ fa2,\ rel');
\item[]\textsl{val\ it\ =\ false\ :\ bool}
\item[]\textsl{-\ }val\ rel''\ =\ SymRel.input\ "";
\item[]\textsl{@\ }(A,\ 2),\ (B,\ 1),\ (C,\ 0)
\item[]\textsl{@\ }.
\item[]\textsl{val\ rel''\ =\ -\ :\ sym_rel}
\item[]\textsl{-\ }val\ fa3\ =\ FA.renameStates(fa1,\ rel'');
\item[]\textsl{val\ fa3\ =\ -\ :\ fa}
\item[]\textsl{-\ }FA.output("",\ fa3);
\item[]\textsl{\symbol{'173}states\symbol{'175}\ 0,\ 1,\ 2\ \symbol{'173}start\ state\symbol{'175}\ 2\ \symbol{'173}accepting\ states\symbol{'175}\ 0,\ 1,\ 2}
\item[]\textsl{\symbol{'173}transitions\symbol{'175}\ 0,\ 1\ ->\ 1;\ 2,\ 0\ ->\ 1\ |\ 2;\ 2,\ 1\ ->\ 0}
\item[]\textsl{val\ it\ =\ ()\ :\ unit}
\item[]\textsl{-\ }val\ fa4\ =\ FA.renameStatesCanonically\ fa3;
\item[]\textsl{val\ fa4\ =\ -\ :\ fa}
\item[]\textsl{-\ }FA.output("",\ fa4);
\item[]\textsl{\symbol{'173}states\symbol{'175}\ A,\ B,\ C\ \symbol{'173}start\ state\symbol{'175}\ C\ \symbol{'173}accepting\ states\symbol{'175}\ A,\ B,\ C}
\item[]\textsl{\symbol{'173}transitions\symbol{'175}\ A,\ 1\ ->\ B;\ C,\ 0\ ->\ B\ |\ C;\ C,\ 1\ ->\ A}
\item[]\textsl{val\ it\ =\ ()\ :\ unit}
\item[]\textsl{-\ }FA.equal(fa4,\ fa1);
\item[]\textsl{val\ it\ =\ false\ :\ bool}
\item[]\textsl{-\ }FA.isomorphic(fa4,\ fa1);
\item[]\textsl{val\ it\ =\ true\ :\ bool}
\end{list}

\index{isomorphism!finite automaton|)}%
\index{finite automaton!isomorphism|)}%

\subsection{Notes}

Books on formal language theory rarely formalize the isomorphism
of finite automata, although most note or prove that the minimization of
deterministic finite automata yields a result that is unique up
to the renaming of states.  Our algorithm for trying to find an
isomorphism between finite automata will be unsurprising to those
familiar with graph algorithms.

%%% Local Variables: 
%%% mode: latex
%%% TeX-master: "book"
%%% End: 

\section{Checking Acceptance and Finding Accepting Paths}
\label{CheckingAcceptanceAndFindingAcceptingPaths}

In this section we study algorithms for checking whether a string is
\index{finite automaton!checking for string acceptance|(}%
\index{finite automaton!searching for labeled paths|(}%
accepted by a finite automaton, and for finding a labeled path that
explains why a string is accepted by a finite automaton.

\subsection{Processing a String from a Set of States}

Suppose $M$ is a finite automaton.  We define a function
$\Delta_M\in\powset\,Q_M\times\Str\fun\powset\,Q_M$ by:
\index{Delta@$\Delta_\cdot(\cdot,\cdot)$}%
\index{finite automaton!Delta@$\Delta_\cdot(\cdot,\cdot)$}%
$\Delta_M(P,w)$ is the set of all $r\in Q_M$ such that there is
an $\lp\in\LP$ such that
\begin{itemize}
\item $w$ is the label of $\lp$;

\item $\lp$ is valid for $M$;

\item the start state of $\lp$ is in $P$; and

\item $r$ is the end state of $\lp$.
\end{itemize}
In other words, $\Delta_M(P,w)$ consists of all of the states that can
be reached from elements of $P$ by labeled paths that are labeled by
$w$ and valid for $M$.  When the FA $M$ is clear from the context, we
sometimes abbreviate $\Delta_M$ to $\Delta$.

Suppose $M$ is the finite automaton
\begin{center}
\input{chap-3.6-fig1.eepic}
\end{center}
Then, $\Delta_M(\{\Asf\},\mathsf{12111111}) =
\{\Bsf,\Csf\}$, since
\begin{gather*}
\mathsf{A\lparr{1}A\lparr{2}B\lparr{11}B\lparr{11}B\lparr{11}B}
\quad\eqtxt{and}\quad
\mathsf{A\lparr{1}A\lparr{2}C\lparr{111}C\lparr{111}C}
\end{gather*}
are all of the labeled paths that are labeled by $\mathsf{12111111}$,
valid for $M$ and whose start states are $\Asf$.
Furthermore, $\Delta_M(\{\Asf,\Bsf,\Csf\},\mathsf{11}) =
\{\Asf,\Bsf\}$, since
\begin{gather*}
\mathsf{A\lparr{1}A\lparr{1}A}
\quad\eqtxt{and}\quad
\mathsf{B\lparr{11}B}
\end{gather*}
are all of the labeled paths that are labeled by $\mathsf{11}$ and
valid for $M$.

Suppose $M$ is a finite automaton, $P\sub Q_M$ and $w\in Str$.
We can calculate $\Delta_M(P,w)$ as follows.
\index{Delta@$\Delta_\cdot$}%
\index{finite automaton!Delta@$\Delta_\cdot$}%
\index{finite automaton!Delta calculating@$\Delta_\cdot(\cdot,\cdot)$!calculating}%

Let $S$ be the set of all suffixes of $w$.  Given $y\in S$, we write
$\pre\,y$ for the unique $x$ such that $w=xy$.

First, we generate the least subset $X$ of $Q_M\times S$ such that:
\begin{enumerate}[(1)]
\item for all $p\in P$, $(p,w)\in X$; and

\item for all $q,r\in Q_M$ and $x,y\in\Str$, if
$(q,xy)\in X$ and $q,x\fun r\in T_M$, then
$(r,y)\in X$.
\end{enumerate}
We start by using rule (1), adding $(p,w)$ to $X$, whenever $p\in P$.
Then $X$ (and any superset of $X$) will satisfy property (1).
Then, rule (2) is used repeatedly to add more pairs to $X$.  Since
$Q_M\times S$ is a finite set, eventually $X$ will satisfy property (2).

If $M$ is our example finite automaton, then here are the elements of
$X$, when $P=\{\Asf\}$ and $w=\mathsf{2111}$:
\begin{itemize}
\item $(\Asf,\mathsf{2111})$;

\item $(\Bsf,\mathsf{111})$, because of $(\Asf,\mathsf{2111})$ and the
  transition $\Asf,\twosf\fun\Bsf$;

\item $(\Csf, \mathsf{111})$, because of $(\Asf,\mathsf{2111})$ and
  the transition $\Asf,\twosf\fun\Csf$ (now, we're done with
  $(\Asf,\mathsf{2111})$);

\item $(\Bsf,\mathsf{1})$, because of $(\Bsf,\mathsf{111})$ and the
  transition $\Bsf,\mathsf{11}\fun\Bsf$ (now, we're done with
  $(\Bsf,\mathsf{111})$);

\item $(\Csf, \%)$, because of $(\Csf, \mathsf{111})$ and the
  transition $\Csf,\mathsf{111}\fun\Csf$ (now, we're done with $(\Csf,
  \mathsf{111})$); and

\item nothing can be added using $(\Bsf,\mathsf{1})$ and $(\Csf, \%)$,
and so we've found all the elements of $X$.
\end{itemize}

The following lemma explains when pairs show up in $X$.

\begin{lemma}
\label{AcceptLem1}
For all $q\in Q_M$ and $y\in S$,
\begin{gather*}
(q,y)\in X\quad\eqtxt{iff}\quad q\in\Delta_M(P,\pre\,y) .
\end{gather*}
\end{lemma}

\begin{proof}
The ``only if'' (left-to-right) direction is by induction on $X$:
we show that, for all $(q,y)\in X$, $q\in\Delta_M(P,\pre\,y)$.
\begin{enumerate}[(1)]
\item Suppose $p\in P$.  Then $p\in\Delta_M(P,\%)$.  But
$\pre\,w=\%$, so that $p\in\Delta_M(P,\pre\,w)$.

\item Suppose $q,r\in Q_M$, $x,y\in\Str$, $(q,xy)\in X$ and
$(q,x,r)\in T_M$.  Assume the inductive hypothesis:
$q\in\Delta_M(P,\pre(xy))$.
Thus there is an $\lp\in\LP$
such that $\pre(xy)$ is the label of $\lp$, $\lp$ is valid for $M$,
the start state of $\lp$ is in $P$, and $q$ is the end state of $\lp$.
Let $lp'\in\LP$ be the result of adding the step $q,x\Rightarrow r$ at
the end of $\lp$.  Thus $\pre\,y$ is the label of $\lp'$, $\lp'$ is
valid for $M$, the start state of $\lp'$ is in $P$, and $r$ is the end
state of $\lp'$, showing that $r\in\Delta_M(P,\pre\,y)$.
\end{enumerate}

For the `if'' (right-to-left) direction, we have that there is a
labeled path
\begin{gather*}
q_1\lparr{x_1}q_2\lparr{x_2}\cdots\,q_{n-1}\lparr{x_{n-1}}q_n ,
\end{gather*}
that is valid for $M$ and where $\pre\,y=x_1x_2\cdots x_{n-1}$,
$q_1\in P$ and $q_n=q$.  Since $q_1\in P$ and
$w=(\pre\,y)y=x_1x_2\cdots x_{n-1}y$, we have that
$(q_1, x_1x_2\cdots x_{n-1}y)=(q_1,w)\in X$, by (1).  But
$(q_1,x_1,q_2)\in T_M$, and thus $(q_2, x_2\cdots x_{n-1}y)\in X$,
by (2). Continuing on in this way (we could do this by mathematical
induction), we finally get that $(q,y)=(q_n,y)\in X$.
\end{proof}

\begin{lemma}
\label{AcceptLem2}
For all $q\in Q_M$, $(q,\%)\in X$ iff $q\in\Delta_M(P,w)$.
\end{lemma}

\begin{proof}
Suppose $(q,\%)\in X$.  Lemma~\ref{AcceptLem1} tells us that
$q\in\Delta_M(P,\pre\,\%)$.  But $\pre\,\%=w$, and thus
$q\in\Delta_M(P,w)$.

Suppose $q\in\Delta_M(P,w)$.  Since $w=\pre\,\%$, we have that
$q\in\Delta_M(P,\pre\,\%)$.  Lemma~\ref{AcceptLem1} tells us that
$(q,\%)\in X$.
\end{proof}

By Lemma~\ref{AcceptLem2}, we have that
\begin{gather*}
\Delta_M(P,w)=\setof{q\in Q_M}{(q,\%)\in X} .
\end{gather*}
Thus, we return the set of all states $q$ that are paired with $\%$ in
$X$.

\subsection{Checking String Acceptance and Finding Accepting Paths}

\begin{proposition}
\label{AcceptLem3}
Suppose $M$ is a finite automaton.  Then
\begin{gather*}
L(M) = \setof{w\in\Str}{\Delta_M(\{s_M\}, w)
\index{L(.)@$L(\cdot)$}%
\index{finite automaton!L(.)@$L(\cdot)$}%
\index{finite automaton!L characterizing@$L(\cdot)$!characterizing}%
\mathbin{\cap}A_M\mathrel{\neq}\emptyset} .
\end{gather*}
\end{proposition}

\begin{proof}
Suppose $w\in L(M)$.  Then $w$ is the label of a labeled path $\lp$
such that $\lp$ is valid for $M$, the start state of $\lp$ is $s_M$ and
the end state of $\lp$ is in $A_M$.  Let $q$ be the end state of
$\lp$.  Thus $q\in\Delta_M(\{s_M\}, w)$ and $q\in A_M$, showing that
$\Delta_M(\{s_M\}, w)\cap A_M\neq\emptyset$.

Suppose $\Delta_M(\{s_M\}, w)\cap A_M\neq\emptyset$, so that
there is a $q$ such that $q\in\Delta_M(\{s_M\}, w)$ and
$q\in A_M$.  Thus $w$ is the label of a labeled path $\lp$ such that
$\lp$ is valid for $M$, the start state of $\lp$ is $s_M$,
and the end state of $\lp$ is $q\in A_M$.  Thus $w\in L(M)$.
\end{proof}

According to Proposition~\ref{AcceptLem3}, to check if a string $w$ is
accepted by a finite automaton $M$, we simply 
use our algorithm to generate $\Delta_M(\{s_M\}, w)$,
and then check if this set contains at least one accepting state.

Given a finite automaton $M$, subsets $P,R$ of $Q_M$ and a string $w$,
\index{finite automaton!searching for labeled paths}%
how do we search for a labeled path that is labeled by $w$, valid
for $M$, starts from an element of $P$, and ends with an element of $R$?
What we need to do is associate with each pair
\begin{gather*}
(q,y)
\end{gather*}
of the set $X$ that we generate when computing $\Delta_M(P,w)$ a
labeled path $\lp$ such that $\lp$ is labeled by $\pre(y)$, $\lp$ is
valid for $M$, the start state of $\lp$ is an element of $P$, and the
end state of $\lp$ is $q$.  If we process the elements of $X$ in a
breadth-first (rather than depth-first) manner, this will ensure that
these labeled paths are as short as possible.  As we generate the
elements of $X$, we look for a pair of the form $(q,\%)$, where $q\in
R$.  Our answer will then be the labeled path associated with this
pair.

The Forlan module \texttt{FA}
\index{FA@\texttt{FA}}%
also contains the following functions
for processing strings and checking string acceptance:
\begin{verbatim}
val processStr          : fa -> sym set * str -> sym set
val accepted            : fa -> str -> bool
\end{verbatim}
\index{FA@\texttt{FA}!processStr@\texttt{processStr}}%
\index{FA@\texttt{FA}!accepted@\texttt{accepted}}%
The function \texttt{processStr} takes in a finite automaton $M$,
and returns a function that takes in a pair $(P, w)$ and returns
$\Delta_M(P, w)$.
And, the function \texttt{accepted} takes in a finite automaton $M$,
and returns a function that checks whether a string $x$ is
accepted by $M$.

The Forlan module \texttt{FA}
\index{FA@\texttt{FA}}%
also contains the following functions for finding labeled paths:
\begin{verbatim}
val findLP          : fa -> sym set * str * sym set -> lp
val findAcceptingLP : fa -> str -> lp
\end{verbatim}
\index{FA@\texttt{FA}!findLP@\texttt{findLP}}%
\index{FA@\texttt{FA}!findAcceptingLP@\texttt{findAcceptingLP}}%
The function \texttt{findLP} takes in a finite automaton $M$, and
returns a function that takes in a triple $(P,w,R)$ and tries
to find a labeled path $\lp$ that is labeled by $w$, valid for $M$,
starts out with an element of $P$, and ends up at an element of $R$.
It issues an error message when there is no such labeled path.
The function \texttt{findAcceptingLP} takes in a finite automaton $M$,
and returns a function that looks for a labeled path $\lp$ that
explains why a string $w$ is accepted by $M$.  It issues an error
message when there is no such labeled path.  The labeled paths
returned by these functions are always of minimal length.

Suppose \texttt{fa} is the finite automaton
\begin{center}
\input{chap-3.6-fig1.eepic}
\end{center}
We begin by applying our five functions to \texttt{fa}, and giving names
to the resulting functions:
\begin{list}{}
{\setlength{\leftmargin}{\leftmargini}
\setlength{\rightmargin}{0cm}
\setlength{\itemindent}{0cm}
\setlength{\listparindent}{0cm}
\setlength{\itemsep}{0cm}
\setlength{\parsep}{0cm}
\setlength{\labelsep}{0cm}
\setlength{\labelwidth}{0cm}
\catcode`\#=12
\catcode`\$=12
\catcode`\%=12
\catcode`\^=12
\catcode`\_=12
\catcode`\.=12
\catcode`\?=12
\catcode`\!=12
\catcode`\&=12
\ttfamily}
\small
\item[]\textsl{-\ }val\ processStr\ =\ FA.processStr\ fa;
\item[]\textsl{val\ processStr\ =\ fn\ :\ sym\ set\ \symbol{'052}\ str\ ->\ sym\ set}
\item[]\textsl{-\ }val\ accepted\ =\ FA.accepted\ fa;
\item[]\textsl{val\ accepted\ =\ fn\ :\ str\ ->\ bool}
\item[]\textsl{-\ }val\ findLP\ =\ FA.findLP\ fa;
\item[]\textsl{val\ findLP\ =\ fn\ :\ sym\ set\ \symbol{'052}\ str\ \symbol{'052}\ sym\ set\ ->\ lp}
\item[]\textsl{-\ }val\ findAcceptingLP\ =\ FA.findAcceptingLP\ fa;
\item[]\textsl{val\ findAcceptingLP\ =\ fn\ :\ str\ ->\ lp}
\end{list}

Next, we'll define a set of states and a string to use later:
\begin{list}{}
{\setlength{\leftmargin}{\leftmargini}
\setlength{\rightmargin}{0cm}
\setlength{\itemindent}{0cm}
\setlength{\listparindent}{0cm}
\setlength{\itemsep}{0cm}
\setlength{\parsep}{0cm}
\setlength{\labelsep}{0cm}
\setlength{\labelwidth}{0cm}
\catcode`\#=12
\catcode`\$=12
\catcode`\%=12
\catcode`\^=12
\catcode`\_=12
\catcode`\.=12
\catcode`\?=12
\catcode`\!=12
\catcode`\&=12
\ttfamily}
\small
\item[]\textsl{-\ }val\ bs\ =\ SymSet.input\ "";
\item[]\textsl{@\ }A,\ B,\ C
\item[]\textsl{@\ }.
\item[]\textsl{val\ bs\ =\ -\ :\ sym\ set}
\item[]\textsl{-\ }val\ x\ =\ Str.input\ "";
\item[]\textsl{@\ }11
\item[]\textsl{@\ }.
\item[]\textsl{val\ x\ =\ \symbol{'133}-,-\symbol{'135}\ :\ str}
\end{list}

Here are some example uses of our functions:
\begin{list}{}
{\setlength{\leftmargin}{\leftmargini}
\setlength{\rightmargin}{0cm}
\setlength{\itemindent}{0cm}
\setlength{\listparindent}{0cm}
\setlength{\itemsep}{0cm}
\setlength{\parsep}{0cm}
\setlength{\labelsep}{0cm}
\setlength{\labelwidth}{0cm}
\catcode`\#=12
\catcode`\$=12
\catcode`\%=12
\catcode`\^=12
\catcode`\_=12
\catcode`\.=12
\catcode`\?=12
\catcode`\!=12
\catcode`\&=12
\ttfamily}
\small
\item[]\textsl{-\ }SymSet.output("",\ processStr(bs,\ x));
\item[]\textsl{A,\ B}
\item[]\textsl{val\ it\ =\ ()\ :\ unit}
\item[]\textsl{-\ }accepted(Str.input\ "");
\item[]\textsl{@\ }12111111
\item[]\textsl{@\ }.
\item[]\textsl{val\ it\ =\ true\ :\ bool}
\item[]\textsl{-\ }accepted(Str.input\ "");
\item[]\textsl{@\ }1211
\item[]\textsl{@\ }.
\item[]\textsl{val\ it\ =\ false\ :\ bool}
\item[]\textsl{-\ }LP.output("",\ findLP(bs,\ x,\ bs));
\item[]\textsl{B,\ 11\ =>\ B}
\item[]\textsl{val\ it\ =\ ()\ :\ unit}
\item[]\textsl{-\ }LP.output("",\ findAcceptingLP(Str.input\ ""));
\item[]\textsl{@\ }12111111
\item[]\textsl{@\ }.
\item[]\textsl{A,\ 1\ =>\ A,\ 2\ =>\ C,\ 111\ =>\ C,\ 111\ =>\ C}
\item[]\textsl{val\ it\ =\ ()\ :\ unit}
\item[]\textsl{-\ }LP.output("",\ findAcceptingLP(Str.input\ ""));
\item[]\textsl{@\ }222
\item[]\textsl{@\ }.
\item[]\textsl{no\ such\ labeled\ path\ exists}
\item[]
\item[]\textsl{uncaught\ exception\ Error}
\end{list}


\subsection{Notes}

The material in this section is original.  Our definition of the
meaning of FAs via labeled paths allows us not simply to test \emph{whether}
an FA accepts a string $w$, but to ask for evidence---in the form of a
labeled path---for \emph{why} FA accepts $w$.

\index{finite automaton!checking for string acceptance|)}%
\index{finite automaton!searching for labeled paths|)}%

%%% Local Variables: 
%%% mode: latex
%%% TeX-master: "book"
%%% End: 

\section{Simplification of Finite Automata}
\label{SimplificationOfFiniteAutomata}

\index{simplification!finite automaton|(}%
\index{finite automaton!simplification|(}%
In this section, we: say what it means for a finite automaton to be
simplified; study an algorithm for simplifying finite automata; and
see how finite automata can be simplified in Forlan.

Suppose $M$ is the finite automaton
\begin{center}
\input{chap-3.7-fig1.eepic}
\end{center}
$M$ is odd for two distinct reasons.  First, there are no valid
labeled paths from the start state to $\Dsf$ and $\Esf$, and so these
states are redundant.  Second, there are no valid labeled paths from
$\Csf$ to an accepting state, and so it is also redundant.  We will
say that $\Csf$ is not ``live'' ($\Csf$ is ``dead''), and that $\Dsf$
and $\Esf$ are not ``reachable''.

Suppose $M$ is a finite automaton.  We say that a state $q\in Q_M$ is:
\begin{itemize}
\item \emph{reachable in} $M$ iff there is a labeled path $\lp$ such that
\index{reachable state}%
\index{finite automaton!reachable state}%
$\lp$ is valid for $M$, the start state of $\lp$ is $s_M$, and
the end state of $\lp$ is $q$;

\item \emph{live in} $M$ iff there is a labeled path $\lp$ such that
\index{live state}%
\index{finite automaton!live state}%
$\lp$ is valid for $M$, the start state of $\lp$ is $q$, and
the end state of $\lp$ is in $A_M$;

\item \emph{dead in} $M$ iff $q$ is not live in $M$; and
\index{dead state}%
\index{finite automaton!dead state}%

\item \emph{useful in} $M$ iff $q$ is both reachable and live in $M$.
\index{useful state}%
\index{finite automaton!useful state}%
\end{itemize}

Let $M$ be our example finite automaton.
The reachable states of $M$ are: $\Asf$, $\Bsf$ and $\Csf$.
The live states of $M$ are: $\Asf$, $\Bsf$, $\Dsf$ and $\Esf$.
And, the useful states of $M$ are: $\Asf$ and $\Bsf$.

There is a simple algorithm for generating the set of reachable states
of a finite automaton $M$.  We generate the least subset $X$ of $Q_M$
such that:
\begin{itemize}
\item $s_M\in X$; and

\item for all $q,r\in Q_M$ and $x\in\Str$, if $q\in X$ and
$(q,x,r)\in T_M$, then $r\in X$.
\end{itemize}
The start state of $M$ is added to $X$, since $s_M$ is always
reachable, by the zero-length labeled path $s_M$.  Then, if $q$ is
reachable, and $(q,x,r)$ is a transition of $M$, then $r$ is clearly
reachable.  Thus all of the elements of $X$ are indeed reachable.
And, it's not hard to show that every reachable state will be added to
$X$.

Similarly, there is a simple algorithm for generating the
set of live states of a finite automaton $M$.  We generate the least
subset $Y$ of $Q_M$ such that:
\begin{itemize}
\item $A_M\sub Y$; and

\item for all $q,r\in Q_M$ and $x\in\Str$, if $r\in Y$ and
$(q,x,r)\in T_M$, then $q\in Y$.
\end{itemize}
This time it's the accepting states of $M$ that initially added
to our set, since each accepting state is trivially live.
Then, if $r$ is live, and $(q,x,r)$ is a transition
of $M$, then $q$ is clearly live.

Thus, we can generate the set of useful states of an FA by generating
the set of reachable states, generating the set of live states, and
intersecting those sets of states.

Now, suppose $N$ is the FA
\begin{center}
  \input{chap-3.7-fig5.eepic}
\end{center}
Here, the transitions $(\Asf,\zerosf,\Bsf)$ and $(\Asf,\onesf,\Bsf)$
are redundant, in the sense that if $N'$ is the result of removing these
transitions from $N$, we still have that $\Bsf\in\Delta_{N'}(\{\Asf\},\zerosf)$
and $\Bsf\in\Delta_{N'}(\{\Asf\},\onesf)$.

Given an FA $M$ and a finite subset $U$ of $\setof{(q,x,r)}{q,r\in
  Q_M\eqtxt{and} x\in\Str}$, we write $M/U$ for the FA that is
identical to $M$ except that its set of transitions is $U$.  If $M$ is
an FA and $(p,x,q)\in T_M$, we say that:
\begin{itemize}
\item $(p,x,q)$ \emph{is redundant in} $M$ iff
  $q\in\Delta_{N}(\{p\},x)$, where $N=M/(T_M-\{(p,x,q)\})$; and

\item $(p,x,q)$ \emph{is irredundant in} $M$ iff $(p,x,q)$ is not
  redundant in $M$.
\end{itemize}

We say that a finite automaton $M$ is \emph{simplified} iff either
\index{finite automaton!simplified}%
\index{simplified!finite automaton}%
\index{simplification!finite automaton!simplified}%
\begin{itemize}
\item every state of $M$ is useful, and every transition of $M$
  is irredundant; or

\item $|Q_M|=1$ and $A_M = T_M = \emptyset$.
\end{itemize}
Thus the FA
\begin{center}
\input{chap-3.7-fig2.eepic}
\end{center}
is simplified, even though its start state is not live, and is
thus not useful.

\begin{proposition}
\label{AlphabetSimplifiedFA}
If $M$ is a simplified finite automaton, then
$\alphabet\,M=\alphabet(L(M))$.
\end{proposition}

We always have that $\alphabet(L(M))\sub\alphabet\,M$.  But, when $M$
is simplified, we also have that $\alphabet\,M\sub\alphabet(L(M))$,
i.e., that every symbol appearing in a string of one of $M$'s
transitions also appears in one of the strings accepted by $M$.

To give our simplification algorithm for finite automta, we need an
auxiliary function for removing redundant transitions from an FA.
Given an FA $M$, $p,q\in Q_M$ and $x\in\Str$, we say that $(p,x,q)$
\emph{is implicit in} $M$ iff $q\in\Delta_M(\{p\},x)$.

Given an FA $M$, we define a function
$\remRedun_M\in\powset\,T_M\times\powset\,T_M\fun\powset\,T_M$ (we
often drop the $M$ when it's clear from the context) by well-founded
recursion on the size of its second argument.  For $U,V\sub T_M$,
$\remRedun(U, V)$ proceeds as follows:
\begin{itemize}
\item If $V=\emptyset$, then it returns $U$.

\item Otherwise, let $v$ be the greatest element of $V$, and $V' = V -
  \{v\}$.  If $v$ is implicit in $M/(U\cup V')$, then $\remRedun$
  returns the result of evaluating $\remRedun(U, V')$.  Otherwise, it
  returns the result of evaluating $\remRedun(U \cup \{v\}, V')$.
\end{itemize}

In general, there are multiple---incompatible---ways of removing
redundant transitions from an FA.  $\remRedun$ is defined so as to
favor removing transitions that are larger in our total ordering on
transitions.

\begin{proposition}
\label{RemRedunFA}
Suppose $M$ is a finite automaton.  For all $U,V\sub T_M$, if all the
elements of $U$ are irredundant in $M/(U\cup V)$, and, for all $p,q\in Q_M$
and $x\in\Str$, $(p,x,q)$ is implicit in $M$ iff $(p,x,q)$ is implicit
in $M/(U\cup V)$, then all the elements of $\remRedun(U,V)$ are
irredundant in $M/\remRedun(U,V)$, and, for all $p,q\in Q_M$ and
$x\in\Str$, $(p,x,q)$ is implicit in $M$ iff $(p,x,q)$ is implicit in
$M/\remRedun(U,V)$.
\end{proposition}

\begin{proof}
By well-founded induction on the size of the second argument to
$\remRedun$.
\end{proof}

Now we can give an algorithm for simplifying finite automata.
\index{simplification!finite automaton!algorithm}%
\index{finite automaton!simplification algorithm}%
We define a function $\simplify\in\FA\fun\FA$ by: $\simplify\,M$ is
\index{simplify@$\simplify$}%
\index{finite automaton!simplify@$\simplify$}%
\index{simplification!finite automaton!simplify@$\simplify$}%
the finite automaton $N$ produced by the following process.
\begin{itemize}
\item First, the useful states are $M$ are determined.

\item If $s_M$ is not useful in $M$, the $N$ is defined by:
\begin{itemize}
\item $Q_N=\{s_M\}$;

\item $s_N = s_M$;

\item $A_N=\emptyset$; and

\item $T_N=\emptyset$.
\end{itemize}

\item And, if $s_M$ is useful in $M$, then $N$ is defined by:
\begin{itemize}
\item $Q_N=\setof{q\in Q_M}{q\eqtxtl{is useful in $M$}}$;

\item $s_N = s_M$;

\item $A_N=A_M\cap Q_N=\setof{q\in A_M}{q\in Q_N}$; and

\item $T_N=\remRedun(\emptyset, \setof{(q, x, r)\in T_M}{q,r\in Q_N})$.
\end{itemize}
\end{itemize}

\begin{proposition}
Suppose $M$ is a finite automaton.
Then:
\begin{enumerate}[\quad(1)]
\item $\simplify\,M$ is simplified;

\item $\simplify\,M\approx M$; and

\item $\alphabet(\simplify\,M) = \alphabet(L(M)) \sub\alphabet\,M$.
\end{enumerate}
\end{proposition}

\begin{proof}
Follows easily using Propositions~\ref{AlphabetSimplifiedFA} and
\ref{RemRedunFA}.
\end{proof}

If $M$ is the finite automaton
\begin{center}
\input{chap-3.7-fig1.eepic}
\end{center}
then $\simplify\,M$ is the finite automaton
\begin{center}
\input{chap-3.7-fig4.eepic}
\end{center}
And if $N$ is the finite automaton
\begin{center}
\input{chap-3.7-fig5.eepic}
\end{center}
then $\simplify\,N$ is the finite automaton
\begin{center}
\input{chap-3.7-fig7.eepic}
\end{center}

The Forlan module \texttt{FA}
\index{FA@\texttt{FA}}%
includes the following functions relating to the simplification of
finite automata:
\begin{verbatim}
val simplify   : fa -> fa
val simplified : fa -> bool
\end{verbatim}
\index{FA@\texttt{FA}!simplified@\texttt{simplified}}%
\index{FA@\texttt{FA}!simplify@\texttt{simplify}}%
The function \texttt{simplify} corresponds to $\simplify$, and
\texttt{simplified} tests whether an FA is simplified.

In the following, suppose \texttt{fa1} is the finite automaton
\begin{center}
\input{chap-3.7-fig1.eepic}
\end{center}
\texttt{fa2} is the finite automaton
\begin{center}
\input{chap-3.7-fig5.eepic}
\end{center}
and \texttt{fa3} is the finita automaton
\begin{center}
\input{chap-3.7-fig6.eepic}
\end{center}
Here are some example uses of \texttt{simplify} and \texttt{simplified}:
\begin{list}{}
{\setlength{\leftmargin}{\leftmargini}
\setlength{\rightmargin}{0cm}
\setlength{\itemindent}{0cm}
\setlength{\listparindent}{0cm}
\setlength{\itemsep}{0cm}
\setlength{\parsep}{0cm}
\setlength{\labelsep}{0cm}
\setlength{\labelwidth}{0cm}
\catcode`\#=12
\catcode`\$=12
\catcode`\%=12
\catcode`\^=12
\catcode`\_=12
\catcode`\.=12
\catcode`\?=12
\catcode`\!=12
\catcode`\&=12
\ttfamily}
\small
\item[]\textsl{-\ }FA.simplified\ fa1;
\item[]\textsl{val\ it\ =\ false\ :\ bool}
\item[]\textsl{-\ }val\ fa1'\ =\ FA.simplify\ fa1;
\item[]\textsl{val\ fa1'\ =\ -\ :\ fa}
\item[]\textsl{-\ }FA.output("",\ fa1');
\item[]\textsl{\symbol{'173}states\symbol{'175}\ A,\ B\ \symbol{'173}start\ state\symbol{'175}\ A\ \symbol{'173}accepting\ states\symbol{'175}\ B}
\item[]\textsl{\symbol{'173}transitions\symbol{'175}\ A,\ %\ ->\ B;\ A,\ 0\ ->\ A;\ B,\ 1\ ->\ B}
\item[]\textsl{val\ it\ =\ ()\ :\ unit}
\item[]\textsl{-\ }FA.simplified\ fa1';
\item[]\textsl{val\ it\ =\ true\ :\ bool}
\item[]\textsl{-\ }val\ fa2'\ =\ FA.simplify\ fa2;
\item[]\textsl{val\ fa2'\ =\ -\ :\ fa}
\item[]\textsl{-\ }FA.output("",\ fa2');
\item[]\textsl{\symbol{'173}states\symbol{'175}\ A,\ B\ \symbol{'173}start\ state\symbol{'175}\ A\ \symbol{'173}accepting\ states\symbol{'175}\ B}
\item[]\textsl{\symbol{'173}transitions\symbol{'175}\ A,\ %\ ->\ B;\ A,\ 0\ ->\ A;\ B,\ 1\ ->\ B}
\item[]\textsl{val\ it\ =\ ()\ :\ unit}
\item[]\textsl{-\ }val\ fa3'\ =\ FA.simplify\ fa3;
\item[]\textsl{val\ fa3'\ =\ -\ :\ fa}
\item[]\textsl{-\ }FA.output("",\ fa3');
\item[]\textsl{\symbol{'173}states\symbol{'175}\ A,\ B,\ C\ \symbol{'173}start\ state\symbol{'175}\ A\ \symbol{'173}accepting\ states\symbol{'175}\ A}
\item[]\textsl{\symbol{'173}transitions\symbol{'175}\ A,\ %\ ->\ B\ |\ C;\ B,\ %\ ->\ A;\ C,\ %\ ->\ A}
\item[]\textsl{val\ it\ =\ ()\ :\ unit}
\end{list}

Thus the simplification of \texttt{fa3} resulted in the removal of
the $\%$-transitions between $\Bsf$ and $\Csf$.

\begin{exercise}
In the simplification of \texttt{fa3}, if transitions had been
considered for removal due to being redundant in other orders,
what FAs could have resulted.
\end{exercise}

\index{simplification!finite automaton|)}%
\index{finite automaton!simplification|)}%

\subsection{Notes}

The removal of useless states is analogous to the standard approach to
ridding grammars of useless variables.  The idea of removing redundant
transitions, though, seems to be novel.

%%% Local Variables: 
%%% mode: latex
%%% TeX-master: "book"
%%% End: 

\section{Proving the Correctness of Finite Automata}
\label{ProvingTheCorrectnessOfFiniteAutomata}

In this section, we consider techniques for proving the correctness
of finite automata, i.e., for proving that finite automata accept
the languages we want them to.

We begin by defining an indexed family of languages, $\Lambda$.

\subsection{Definition of $\Lambda$}

\begin{proposition}
\label{DeltaProp1}
Suppose $M$ is a finite automaton.
\begin{enumerate}[\quad(1)]
\item For all $q\in Q_M$, ${q}\in\Delta_M(\{q\},\%)$.

\item For all $q,r\in Q_M$ and $w\in\Str$, if $q,w\fun r\in T_M$,
then ${r}\in\Delta_M(\{q\},w)$.

\item For all $p,q,r\in Q_M$ and $x,y\in\Str$, if $q\in\Delta_M(\{p\},x)$
and $r\in\Delta_M(\{q\},y)$, then
${r}\in\Delta_M(\{p\},xy)$.
\end{enumerate}
\end{proposition}

Suppose $M$ is a finite automaton and $q\in Q_M$.
Then we define
\begin{displaymath}
\Lambda_{M,q} = \setof{w\in\Str}{q\in\Delta_M(\{s_M\},w)}.
\end{displaymath}
In other words, $\Lambda_{M,q}$ is the labels of all of the valid
labeled paths for $M$ that start at state $s_M$ and end at $q$, i.e.,
it's all strings that can take us to state $q$ when processed by $M$.
Clearly, $\Lambda_{M,q}\sub(\alphabet\,M)^*$, for all FAs $M$ and
$q\in Q_M$.  If it's clear which FA we are talking about, we sometimes
abbreviate $\Lambda_{M,q}$ to $\Lambda_q$.

Let our example FA, $M$, be
\begin{center}
\input{chap-3.8-fig1.eepic}
\end{center}
Then:
\begin{itemize}
\item $\mathsf{01101}\in\Lambda_\Asf$, because of the labeled path
  \begin{gather*}
    \mathsf{A\lparr{0}B\lparr{1}B\lparr{1}B\lparr{0}A\lparr{1}A} ,
  \end{gather*}

\item $\mathsf{01100}\in\Lambda_\Bsf$, because of the labeled path
  \begin{gather*}
    \mathsf{A\lparr{0}B\lparr{1}B\lparr{1}B\lparr{0}A\lparr{0}B} .
  \end{gather*}
\end{itemize}

\begin{proposition}
\label{LambdaProp1}
Suppose $M$ is an FA.  Then $L(M) = \bigcup\setof{\Lambda_{M,q}}{q\in
  A_M}$, i.e., for all $w$, $w\in L(M)$ iff $w\in\Lambda_{M,q}$ for
some $q\in A_M$.
\end{proposition}

\begin{proof}
\begin{description}
\item[\quad(only if)] Suppose $w\in L(M)$.  By Proposition~3.5.3,
we have that $\Delta_M({s_M},w)\cap A_M\neq\emptyset$, so that there
is a $q\in A_M$ such that $q\in\Delta_M({s_M},w)$.  Thus
$w\in\Lambda_{M,q}$.

\item[\quad(if)] Suppose $w\in\Lambda_{M,q}$ for some $q\in A_M$.
Thus $q\in\Delta_M({s_M},w)$ and $q\in A_M$, so that
$\Delta_M({s_M},w)\cap A_M\neq\emptyset$.  Hence $w\in L(M)$, by
Proposition~3.5.3.
\end{description}
\end{proof}

\begin{proposition}
\label{LambdaProp2}
Suppose $M$ is a finite automaton.
\begin{enumerate}[\quad(1)]
\item $\%\in\Lambda_{M,s_M}$.

\item For all $q,r\in Q_M$ and $w,x\in\Str$.  If $w\in\Lambda_{M,q}$
  and ${q,x\fun r}\in T_M$, then $wx\in\Lambda_{M,r}$.
\end{enumerate}
\end{proposition}

\begin{proof}
\begin{enumerate}[(1)]
\item By Proposition~\ref{DeltaProp1}(1), we have
  that $s_M\in\Delta(\{s_M\},\%)$, so that $\%\in\Lambda_{M,s_M}$.

\item Suppose $q,r\in Q_M$, $w,x\in\Str$, $w\in\Lambda_{M,q}$ and
  $q,x\fun r\in T_M$.  Thus $q\in\Delta(\{s_M\},w)$.  Because
  ${q,x\fun r}\in T_M$, Proposition~\ref{DeltaProp1}(2) tells us that
  $r\in\Delta(\{q\},x)$.  Hence by Proposition~\ref{DeltaProp1}(3), we
  have that ${r}\in\Delta(\{s_M\},wx)$, so that
  ${wx}\in\Lambda_{M,{r}}$.
\end{enumerate}
\end{proof}

Our main example will be the FA, $M$:
\begin{center}
\input{chap-3.8-fig1.eepic}
\end{center}
Let
\begin{align*}
X &= \setof{w\in\{\mathsf{0,1}\}^*}{w\eqtxt{has an even number of}
\zerosf\eqtxtn{'s}} , \eqtxt{and} \\
Y &= \setof{w\in\{\mathsf{0,1}\}^*}{w\eqtxt{has an odd number of}
\zerosf\eqtxtn{'s}} .
\end{align*}

We want to prove that $L(M) = X$.  Because $A_M=\{\Asf\}$,
Proposition~\ref{LambdaProp1} tells us that $L(M)=\Lambda_{M,\Asf}$.
Thus it will suffice to show that $\Lambda_{M,\Asf} = X$.  But our
approach will also involve showing $\Lambda_{M,\Bsf} = Y$.  We would
cope with more states analogously, having one language per state.

\subsection{Proving that Enough is Accepted}

First, we study techniques for showing that everything we want an
automaton to accept is really accepted.

Since $X,Y\sub\{\mathsf{0,1}\}^*$, to prove that
$X\sub\Lambda_{M,\Asf}$ and $Y\sub\Lambda_{M,\Bsf}$, it will suffice
to use strong string induction to show that, for all
$w\in\mathsf{\{0,1\}^*}$:
\begin{enumerate}[\quad(A)]
\item if $w\in X$, then $w\in\Lambda_{M,\Asf}$; and

\item if $w\in Y$, then $w\in\Lambda_{M,\Bsf}$.
\end{enumerate}

We proceed by strong string induction.  Suppose
$w\in\{\mathsf{0,1}\}^*$, and assume the inductive hypothesis:
for all $x\in\{\mathsf{0,1}\}^*$, if $x$ is a proper substring of
$w$, then:
\begin{enumerate}[\quad(A)]
\item if $x\in X$, then $x\in\Lambda_\Asf$; and

\item if $x\in Y$, then $x\in\Lambda_\Bsf$.
\end{enumerate}
We must prove that:
\begin{enumerate}[\quad(A)]
\item if $w\in X$, then $w\in\Lambda_\Asf$; and

\item if $w\in Y$, then $w\in\Lambda_\Bsf$.
\end{enumerate}
There are two parts to show.
\begin{enumerate}[\quad(A)]
\item Suppose $w\in X$, so that $w$ has an even number of $\zerosf$'s.
  We must show that
  $w\in\Lambda_\Asf$.  There are three cases to consider.
  \begin{itemize}
  \item Suppose $w=\%$.  By Proposition~\ref{LambdaProp2}(1), we have
    that $w=\%\in\Lambda_\Asf$.
  
  \item Suppose $w=x\zerosf$, for some $x\in\{\mathsf{0,1}\}^*$.  Thus
    $x$ has an odd number of $\zerosf$'s, so that $x\in Y$.
    Because $x$ is a proper substring of $w$,  part~(B) of
    the inductive hypothesis tells us that $x\in\Lambda_\Bsf$.
    Furthermore, $\Bsf,\zerosf\fun\Asf\in T$, so that
    ${w=x\zerosf}\in\Lambda_\Asf$, by
    Proposition~\ref{LambdaProp2}(2).
  
  \item Suppose $w=x\onesf$, for some $x\in\{\mathsf{0,1}\}^*$.  Thus
    $x$ has an even number of $\zerosf$'s, so that $x\in X$.
    Because $x$ is a proper
    substring of $w$, part~(A) of the inductive hypothesis tells us
    that $x\in\Lambda_\Asf$.  Furthermore, $\Asf,\onesf\fun\Asf\in
    T$, so that $w=x\onesf\in\Lambda_\Asf$, by
    Proposition~\ref{LambdaProp2}(2).
  \end{itemize}

\item Suppose $w\in Y$, so that $w$ has an odd number of $\zerosf$'s.
  We must show that
  $w\in\Lambda_\Bsf$.  There are three cases to consider.
  \begin{itemize}
  \item Suppose $w=\%$.  But the number of $\zerosf$'s in
    $\%$ is $0$, which is even---contradiction.  Thus
    $w\in\Lambda_\Bsf$.
  
  \item Suppose $w=x\zerosf$, for some $x\in\{\mathsf{0,1}\}^*$.  Thus
    $x$ has an even number of $\zerosf$'s, so that $x\in X$.
    Because $x$ is a proper
    substring of $w$, part (A) of the inductive hypothesis tells us
    that $x\in\Lambda_\Asf$.  Furthermore, $\Asf,\zerosf\fun\Bsf\in
    T$, so that $w=x\zerosf\in\Lambda_\Bsf$, by
    Proposition~\ref{LambdaProp2}(2).
  
  \item Suppose $w=x\onesf$, for some $x\in\{\mathsf{0,1}\}^*$.  Thus
    $x$ has an odd number of $\zerosf$'s, so that $x\in Y$.
    Because $x$ is a proper
    substring of $w$, part (B) of the inductive hypothesis tells us
    that $x\in\Lambda_\Bsf$.  Furthermore, $\Bsf,\onesf\fun\Bsf\in
    T$, so that $w=x\onesf\in\Lambda_\Bsf$, by
    Proposition~\ref{LambdaProp2}(2).
  \end{itemize}
\end{enumerate}

Let $N$ be the finite automaton
\begin{center}
\input{chap-3.8-fig2.eepic}
\end{center}
Here we hope that $\Lambda_{N,\Asf} = {\{\zerosf\}^*}$ and $L(N) =
\Lambda_{N,\Bsf} = \{\zerosf\}^*\{\onesf\onesf\}^*$, but if we try to
prove that
\begin{align*}
  \{\zerosf\}^*&\sub\Lambda_{N,\Asf} , \eqtxt{and}\\
  \{\zerosf\}^*\{\onesf\onesf\}^*&\sub\Lambda_{N,\Bsf}
\end{align*}
using our standard technique, there is a complication related to the
$\%$-transition.

We use strong string induction to show that, for all
$w\in\{\mathsf{0,1}\}^*$:
\begin{enumerate}[\quad(A)]
\item if $w\in\{\zerosf\}^*$, then $w\in\Lambda_\Asf$; \eqtxt{and}

\item if $w\in\{\zerosf\}^*\{\onesf\onesf\}^*$, then
  $w\in\Lambda_\Bsf$.
\end{enumerate}

In part~(B), we assume that $w\in\{\zerosf\}^*\{\onesf\onesf\}^*$, so
that $w=\zerosf^n(\onesf\onesf)^m$ for some $n,m\in\nats$.  We must show that
$w\in\Lambda_\Bsf$.  We consider two cases: $m=0$ and $m\geq 1$.
The second of these is straightforward, so let's focus on the first.
Then $w=\zerosf^n\in\{\zerosf\}^*$.  We want to use part~(A) of
the inductive hypothesis to conclude that $\zerosf^n\in\Lambda_\Asf$,
but there is a problem: $\zerosf^n$ is not a proper substring of
$\zerosf^n=w$.

So, we must consider two subcases, when $n=0$ and $n\geq 1$.
In the first subcase, because $\%\in\Lambda_\Asf$ and $\Asf,\%\fun\Bsf\in T$,
we have that $w=\%=\%\%\in\Lambda_\Bsf$.

In the second subcase, we have that $w=\zerosf^{n-1}\zerosf$.  By
part~(A) of the inductive hypothesis, we have that
$\zerosf^{n-1}\in\Lambda_\Asf$.  Thus, because
$\Asf,\zerosf\fun\Asf\in T$ and $\Asf,\%\fun\Bsf\in T$, we can conclude
$w=\zerosf^n=\zerosf^{n-1}\zerosf\%\in\Lambda_\Bsf$.

Because there are no transitions from $\Bsf$ back to $\Asf$, we
could first prove that, for all $w\in\{\mathsf{0,1}\}^*$,
\begin{enumerate}[\quad(A)]
\item if $w\in\{\zerosf\}^*$, then $w\in\Lambda_\Asf$,
\end{enumerate}
and then use (A) to prove that for all $w\in\{\mathsf{0,1}\}^*$,
\begin{enumerate}[\quad(A)]
\setcounter{enumi}{1}
\item if $w\in\{\zerosf\}^*\{\onesf\onesf\}^*$, then
$w\in\Lambda_\Bsf$.
\end{enumerate}

This works whenever one part of a machine has transitions to
another part, but there are no transitions from that second part
back to the first part, i.e., when the two parts are not mutually
recursive.

In the case of $N$, we could use mathematical induction instead of
strong string induction:
\begin{enumerate}[\quad(A)]
\item for all $n\in\nats$, $\zerosf^n\in\Lambda_\Asf$, and

\item for all $n,m\in\nats$, $\zerosf^n(\onesf\onesf)^m\in\Lambda_\Bsf$
(do induction on $m$, fixing $n$).
\end{enumerate}

\subsection{Proving that Everything Accepted is Wanted}

It's tempting to try to prove that everything accepted by a finite
automaton is wanted using string string induction, with implications
like
\begin{enumerate}[\quad(A)]
\item if $w\in\Lambda_\Asf$, then $w\in X$.
\end{enumerate}
Unfortunately, this doesn't work when a finite automaton contains
$\%$-transitions.  Instead, we do such proofs using a new induction
principle that we call induction on $\Lambda$.

\begin{theorem}[Principle of Induction on $\Lambda$]
Suppose $M$ is a finite automaton, and $P_q(w)$ is a property
of a $w\in\Lambda_{M,q}$, for all $q\in Q_M$.
If
\begin{itemize}
\item $P_{s_M}(\%)$ and

\item for all $q,r\in Q_M$, $x\in\Str$ and $w\in\Lambda_{M,q}$,\\
if $q, x\fun r\in T_M$ and (\dag) $P_q(w)$, then  $P_r(wx)$,
\end{itemize}
then
\begin{gather*}
\eqtxtr{for all}q\in Q_M,\eqtxt{for all}w\in\Lambda_{M,q}, P_q(w) .
\end{gather*}
\end{theorem}

We refer to (\dag) as the inductive hypothesis.

\begin{proof}
It suffices to show that, for all $\lp\in\LP$, for all $q\in Q_M$,
if $\lp$ is valid for $M$, $\startState\,\lp = s_M$ and
$\myendState\,\lp = q$, then $P_q(\mylabel\,\lp)$.  We prove this by
well-founded induction on the length of $\lp$.
\end{proof}

In the case of our example FA, $M$, we can let $P_\Asf(w)$ and $P_\Bsf(w)$ be
$w\in X$ and $w\in Y$, respectively, where, as before,
\begin{align*}
X &= \setof{w\in\{\mathsf{0,1}\}^*}{w\eqtxt{has an even number of}
\zerosf\eqtxtn{'s}} , \eqtxt{and}\\
Y &= \setof{w\in\{\mathsf{0,1}\}^*}{w\eqtxt{has an odd number of}
\zerosf\eqtxtn{'s}} .
\end{align*}

Then the principle of induction on $\Lambda$ tells us that
\begin{enumerate}[\quad(A)]
\item for all $w\in\Lambda_\Asf$, $w\in X$, and

\item for all $w\in\Lambda_\Bsf$, $w\in Y$,
\end{enumerate}
follows from showing
\begin{description}
\item[\quad(empty string)] $\%\in X$,

\item[\quad($\Asf, \zerosf\fun\Bsf$)] for all $w\in\Lambda_\Asf$, if
  (\dag) $w\in X$, then $w\zerosf\in Y$,

\item[\quad($\Asf, \onesf\fun\Asf$)] for all $w\in\Lambda_\Asf$, if
  (\dag) $w\in X$, then $w\onesf\in X$,

\item[\quad($\Bsf, \zerosf\fun\Asf$)] for all $w\in\Lambda_\Bsf$, if
  (\dag) $w\in Y$, then $w\zerosf\in X$, and

\item[\quad($\Bsf, \onesf\fun\Bsf$)] for all $w\in\Lambda_\Bsf$, if
  (\dag) $w\in Y$, then $w\onesf\in Y$.
\end{description}
We refer to (\dag) as the inductive hypothesis.

In fact, when setting this proof up, instead of explicitly mentioning
$P_\Asf$ and $P_\Bsf$, we can simply say that we are proving
\begin{enumerate}[\quad(A)]
\item for all $w\in\Lambda_\Asf$, $w\in X$, and

\item for all $w\in\Lambda_\Bsf$, $w\in Y$,
\end{enumerate}
by induction on $\Lambda$.

There are five steps to show.
\begin{description}
\item[\quad(empty string)] Because $\%\in\{\mathsf{0,1}\}^*$ and $\%$
has no $\zerosf$'s, we have that $\%\in X$.

\item[\quad($\Asf, \zerosf\fun\Bsf$)] Suppose $w\in\Lambda_\Asf$, and
  assume the inductive hypothesis: $w\in X$.  Hence
  $w\in\{\mathsf{0,1}\}^*$ and $w$ has an even number of $\zerosf$'s.
  Thus $w\zerosf\in\{\mathsf{0,1}\}^*$ and $w\zerosf$ has an odd
  number of $\zerosf$'s, so that $w\zerosf\in Y$.
\end{description}

\begin{description}
\item[\quad($\Asf, \onesf\fun\Asf$)] Suppose $w\in\Lambda_\Asf$, and
  assume the inductive hypothesis: $w\in X$.  Then $w\onesf\in X$.

\item[\quad($\Bsf, \zerosf\fun\Asf$)] Suppose $w\in\Lambda_\Bsf$, and
  assume the inductive hypothesis: $w\in Y$.  Then $w\zerosf\in X$.

\item[\quad($\Bsf, \onesf\fun\Bsf$)] Suppose $w\in\Lambda_\Bsf$, and
  assume the inductive hypothesis: $w\in Y$.  Then $w\onesf\in Y$.
\end{description}

Because of
\begin{enumerate}[\quad(A)]
\item for all $w\in\Lambda_\Asf$, $w\in X$, and

\item for all $w\in\Lambda_\Bsf$, $w\in Y$,
\end{enumerate}
we have that
$\Lambda_\Asf \sub X$ and $\Lambda_\Bsf \sub Y$.

Because $X \sub \Lambda_\Asf$ and $Y \sub \Lambda_\Bsf$, we can
conclude that $L(M) = \Lambda_\Asf = X$ and $\Lambda_\Bsf = Y$.

Consider our second example, $N$, again:
\begin{center}
\input{chap-3.8-fig2.eepic}
\end{center}

We can use induction on $\Lambda$ to prove that
\begin{enumerate}[\quad(A)]
\item for all $w\in\Lambda_\Asf$, $w\in\{\zerosf\}^*$, and

\item for all $w\in\Lambda_\Bsf$, $w\in\{\zerosf\}^*\{\mathsf{11}\}^*$.
\end{enumerate}
Thus $\Lambda_\Asf\sub\{\zerosf\}^*$ and
$\Lambda_\Bsf\sub\{\zerosf\}^*\{\mathsf{11}\}^*$.  Because
$\{\zerosf\}^*\sub\Lambda_\Asf$ and
$\{\zerosf\}^*\{\mathsf{11}\}^*\sub\Lambda_\Bsf$, we can conclude that
$\Lambda_\Asf=\{\zerosf\}^*$ and
$L(N)=\Lambda_\Bsf=\{\zerosf\}^*\{\mathsf{11}\}^*$.

\subsection{Notes}

Books on formal language theory typically give short shrift to the
proof of correctness of finite automata, carrying out one or two
correctness proofs using induction on the length of strings.  In
contrast, we have introduced and applied elegant techniques for
proving the correctness of FAs.  Of particular note is our principle
of induction on $\Lambda$.

%%% Local Variables: 
%%% mode: latex
%%% TeX-master: "book"
%%% End: 

\section{Empty-string Finite Automata}
\label{EmptyStringFiniteAutomata}

\index{finite automaton!empty string|(}%
\index{empty string finite automaton|(}%
\index{EFA|(}%

In this and the following two sections, we will study three
progressively more restricted kinds of finite automata:
\begin{itemize}
\item empty-string finite automata (EFAs);

\item nondeterministic finite automata (NFAs); and

\item deterministic finite automata (DFAs).
\end{itemize}
Every DFA will be an NFA; every NFA will be an EFA; and every EFA will
be an FA.  Thus, $L(M)$ will be well-defined, if $M$ is a DFA, NFA or
EFA.

The more restricted kinds of automata will be easier to process on the
computer than the more general kinds; they will also have nicer
reasoning principles than the more general kinds.  We will give
algorithms for converting the more general kinds of automata into the
more restricted kinds.  Thus even the deterministic finite automata
will accept the same set of languages as the finite automata.  On the
other hand, it will sometimes be easier to find one of the more
general kinds of automata that accepts a given language rather than
one of the more restricted kinds accepting the language.  And, there
are languages where the smallest DFA accepting the language is
exponentially bigger than the smallest FA accepting the language.

\subsection{Definition of EFAs}

An \emph{empty-string finite automaton} (EFA) $M$ is
a finite automaton such that
\begin{gather*}
T_M\sub\setof{q, x\fun r}{q,r\in\Sym\eqtxt{and}x\in\Str\eqtxt{and}|x|\leq 1}.
\end{gather*}
In other words, an FA is an EFA iff every string of every transition
of the FA is either $\%$ or has a single symbol.

For example, $\Asf,\%\fun\Bsf$ and $\Asf,\onesf\fun\Bsf$ are legal EFA
transitions, but $\Asf,\onesf\onesf\fun\Bsf$ is not legal.  We write
\index{EFA@$\EFA$}%
\index{finite automaton!EFA@$\EFA$}%
\index{empty string finite automaton!EFA@$\EFA$}%
$\EFA$ for the set of all empty-string finite automata.  Thus
$\EFA\subsetneq\FA$.

The following proposition obviously holds.

\begin{proposition}
Suppose $M$ is an EFA.
\begin{itemize}
\item For all $N\in\FA$, if $M\iso N$, then $N$ is an EFA.

\item For all bijections $f$ from $Q_M$ to some set of symbols,
  $\renameStates(M, f)$ is an EFA.

\item $\renameStatesCanonically\,M$ is an EFA.

\item $\simplify\,M$ is an EFA.
\end{itemize}
\end{proposition}

\subsection{Converting FAs to EFAs}

\index{finite automaton!converting FA to EFA}%
\index{empty string finite automaton!converting FA to EFA}%
If we want to convert an FA into an equivalent EFA, we can proceed as
follows.  Every state of the FA will be a state of the EFA, the start
and accepting states are unchanged, and every transition of the FA
that is a legal EFA transition will be a transition
of the EFA.  If our FA has a transition
\begin{gather*}
p, b_1b_2\cdots b_n\fun r,
\end{gather*}
where $n\geq 2$ and the $b_i$ are symbols, then we replace this
transition with the $n$ transitions
\begin{gather*}
p\tranarr{b_1}q_1, q_1\tranarr{b_2}q_2,\,\ldots, q_{n-1}\tranarr{b_n}r,
\end{gather*}
where $q_1,\,\ldots,q_{n-1}$ are $n-1$ new, non-accepting, states.

For example, we can convert the FA
\begin{center}
\input{chap-3.9-fig1.eepic}
\end{center}
into the EFA
\begin{center}
\input{chap-3.9-fig2.eepic}
\end{center}

We have to be careful how we choose our new states.  The
symbols we choose can't be states of the original machine, and we
can't choose the same symbol twice.  Instead of making a series of
random choices, we will use structured symbols in such a way that
one will be able to look at a resulting EFA and tell what the
original FA was.

First, the algorithm renames each old state $q$ to
$\langle 1,q\rangle$.  Then it can replace a transition
\begin{gather*}
p\tranarr{b_1b_2\cdots b_n}r,
\end{gather*}
where $n\geq 2$ and the $b_i$ are symbols, with the transitions
\begin{gather*}
\langle 1,p\rangle\tranarr{b_1}
\langle 2,\langle p,b_1,b_2\cdots b_n,r\rangle\rangle, \\
\langle 2,\langle p,b_1,b_2\cdots b_n,r\rangle\rangle\tranarr{b_2}
\langle 2,\langle p,b_1b_2,b_3\cdots b_n,r\rangle\rangle, \\
\ldots,\\
\langle 2,\langle p,b_1b_2\cdots b_{n-1},b_n,r\rangle\rangle\tranarr{b_n},
\langle 1,r\rangle.
\end{gather*}

\index{finite automaton!faToEFA@$\faToEFA$}%
\index{empty string finite automaton!faToEFA@$\faToEFA$}%
We define a function $\faToEFA\in\FA\fun\EFA$ that converts FAs into
EFAs by saying that $\faToEFA\,M$ is the result of running the above
algorithm on input $M$.

\begin{theorem}
For all $M\in\FA$:
\begin{itemize}
\item $\faToEFA\,M\approx M$; and

\item $\alphabet(\faToEFA\,M) = \alphabet\,M$.
\end{itemize}
\end{theorem}

\subsection{Processing EFAs in Forlan}

The Forlan module \texttt{EFA} defines an abstract type \texttt{efa}
\index{EFA@\texttt{EFA}}%
\index{EFA@\texttt{EFA}!efa@\texttt{efa}}%
(in the top-level environment) of empty-string finite automata,
along with various functions for processing EFAs.
Values of type \texttt{efa} are implemented as values of type \texttt{fa}, and
the module EFA provides functions
\begin{verbatim}
val injToFA    : efa -> fa
val projFromFA : fa -> efa
\end{verbatim}
\index{EFA@\texttt{EFA}!injToFA@\texttt{injToFA}}%
\index{EFA@\texttt{EFA}!projFromFA@\texttt{projFromFA}}%
for making a value of type \texttt{efa} have type \texttt{fa}, i.e.,
``injecting'' an \texttt{efa} into type \texttt{fa}, and for
making a value of type \texttt{fa} that is an EFA have type
\texttt{efa}, i.e., ``projecting'' an \texttt{fa} that is an EFA to
type \texttt{efa}.  If one tries to project an \texttt{fa} that is not
an EFA to type \texttt{efa}, an error is signaled.  The functions
\texttt{injToFA} and \texttt{projFromFA} are available in the top-level
environment as \texttt{injEFAToFA} and \texttt{projFAToEFA}, respectively.
\index{empty string finite automaton!injEFAToFA@\texttt{injEFAToFA}}%
\index{empty string finite automaton!projFAToEFA@\texttt{projFAToEFA}}%

The module \texttt{EFA} also defines the functions:
\begin{verbatim}
val input  : string -> efa
val fromFA : fa -> efa
\end{verbatim}
\index{EFA@\texttt{EFA}!input@\texttt{input}}%
\index{EFA@\texttt{EFA}!fromFA@\texttt{fromFA}}%
The function \texttt{input} is used to input an EFA, i.e., to input a
value of type \texttt{fa} using \texttt{FA.input}, and then attempt to
project it to type \texttt{efa}.  The function \texttt{fromFA}
corresponds to our conversion function $\faToEFA$, and is available in
the top-level environment with that name:
\begin{verbatim}
val faToEFA : fa -> efa
\end{verbatim}
\index{empty string finite automaton!faToEFA@\texttt{faToEFA}}%

Finally, most of the functions for processing FAs that were introduced
in previous sections are inherited by \texttt{EFA}:
\begin{verbatim}
val output                  : string * efa -> unit 
val numStates               : efa -> int
val numTransitions          : efa -> int
val equal                   : efa * efa -> bool
val alphabet                : efa -> sym set
val checkLP                 : efa -> lp -> unit
val validLP                 : efa -> lp -> bool
val isomorphism             : efa * efa * sym_rel -> bool
val findIsomorphism         : efa * efa -> sym_rel
val isomorphic              : efa * efa -> bool
val renameStates            : efa * sym_rel -> efa
val renameStatesCanonically : efa -> efa
val processStr              : efa -> sym set * str -> sym set
val accepted                : efa -> str -> bool
val findLP                  : efa -> sym set * str * sym set -> lp
val findAcceptingLP         : efa -> str -> lp
val simplified              : efa -> bool
val simplify                : efa -> efa
\end{verbatim}
\index{EFA@\texttt{EFA}!output@\texttt{output}}%
\index{EFA@\texttt{EFA}!numStates@\texttt{numStates}}%
\index{EFA@\texttt{EFA}!numTransitions@\texttt{numTransitions}}%
\index{EFA@\texttt{EFA}!equal@\texttt{equal}}%
\index{EFA@\texttt{EFA}!alphabet@\texttt{alphabet}}%
\index{EFA@\texttt{EFA}!checkLP@\texttt{checkLP}}%
\index{EFA@\texttt{EFA}!validLP@\texttt{validLP}}%
\index{EFA@\texttt{EFA}!isomorphism@\texttt{isomorphism}}%
\index{EFA@\texttt{EFA}!findIsomorphism@\texttt{findIsomorphism}}%
\index{EFA@\texttt{EFA}!isomorphic@\texttt{isomorphic}}%
\index{EFA@\texttt{EFA}!renameStates@\texttt{renameStates}}%
\index{EFA@\texttt{EFA}!renameStatesCanonically@\texttt{renameStatesCanonically}}%
\index{EFA@\texttt{EFA}!processStr@\texttt{processStr}}%
\index{EFA@\texttt{EFA}!accepted@\texttt{accepted}}%
\index{EFA@\texttt{EFA}!findLP@\texttt{findLP}}%
\index{EFA@\texttt{EFA}!findAcceptingLP@\texttt{findAcceptingLP}}%
\index{EFA@\texttt{EFA}!simplified@\texttt{simplified}}%
\index{EFA@\texttt{EFA}!simplify@\texttt{simplify}}%

Suppose that \texttt{fa} is the finite automaton
\begin{center}
\input{chap-3.9-fig1.eepic}
\end{center}
Here are some example uses of a few of the above functions:
\begin{list}{}
{\setlength{\leftmargin}{\leftmargini}
\setlength{\rightmargin}{0cm}
\setlength{\itemindent}{0cm}
\setlength{\listparindent}{0cm}
\setlength{\itemsep}{0cm}
\setlength{\parsep}{0cm}
\setlength{\labelsep}{0cm}
\setlength{\labelwidth}{0cm}
\catcode`\#=12
\catcode`\$=12
\catcode`\%=12
\catcode`\^=12
\catcode`\_=12
\catcode`\.=12
\catcode`\?=12
\catcode`\!=12
\catcode`\&=12
\ttfamily}
\small
\item[]\textsl{-\ }projFAToEFA\ fa;
\item[]\textsl{invalid\ label\ in\ transition:\ "12"}
\item[]
\item[]\textsl{uncaught\ exception\ Error}
\item[]\textsl{-\ }val\ efa\ =\ faToEFA\ fa;
\item[]\textsl{val\ efa\ =\ -\ :\ efa}
\item[]\textsl{-\ }EFA.output("",\ efa);
\item[]\textsl{\symbol{'173}states\symbol{'175}}
\item[]\textsl{<1,A>,\ <1,B>,\ <2,<A,1,2,B>>,\ <2,<B,3,45,B>>,\ <2,<B,34,5,B>>}
\item[]\textsl{\symbol{'173}start\ state\symbol{'175}\ <1,A>\ \symbol{'173}accepting\ states\symbol{'175}\ <1,B>}
\item[]\textsl{\symbol{'173}transitions\symbol{'175}}
\item[]\textsl{<1,A>,\ 0\ ->\ <1,A>;\ <1,A>,\ 1\ ->\ <2,<A,1,2,B>>;}
\item[]\textsl{<1,B>,\ 3\ ->\ <2,<B,3,45,B>>;\ <2,<A,1,2,B>>,\ 2\ ->\ <1,B>;}
\item[]\textsl{<2,<B,3,45,B>>,\ 4\ ->\ <2,<B,34,5,B>>;\ <2,<B,34,5,B>>,\ 5\ ->\ <1,B>}
\item[]\textsl{val\ it\ =\ ()\ :\ unit}
\item[]\textsl{-\ }val\ efa'\ =\ EFA.renameStatesCanonically\ efa;
\item[]\textsl{val\ efa'\ =\ -\ :\ efa}
\item[]\textsl{-\ }EFA.output("",\ efa');
\item[]\textsl{\symbol{'173}states\symbol{'175}\ A,\ B,\ C,\ D,\ E\ \symbol{'173}start\ state\symbol{'175}\ A\ \symbol{'173}accepting\ states\symbol{'175}\ B}
\item[]\textsl{\symbol{'173}transitions\symbol{'175}}
\item[]\textsl{A,\ 0\ ->\ A;\ A,\ 1\ ->\ C;\ B,\ 3\ ->\ D;\ C,\ 2\ ->\ B;\ D,\ 4\ ->\ E;\ E,\ 5\ ->\ B}
\item[]\textsl{val\ it\ =\ ()\ :\ unit}
\item[]\textsl{-\ }val\ rel\ =\ EFA.findIsomorphism(efa,\ efa');
\item[]\textsl{val\ rel\ =\ -\ :\ sym_rel}
\item[]\textsl{-\ }SymRel.output("",\ rel);
\item[]\textsl{(<1,A>,\ A),\ (<1,B>,\ B),\ (<2,<A,1,2,B>>,\ C),\ (<2,<B,3,45,B>>,\ D),}
\item[]\textsl{(<2,<B,34,5,B>>,\ E)}
\item[]\textsl{val\ it\ =\ ()\ :\ unit}
\item[]\textsl{-\ }LP.output("",\ FA.findAcceptingLP\ fa\ (Str.input\ ""));
\item[]\textsl{@\ }012345
\item[]\textsl{@\ }.
\item[]\textsl{A,\ 0\ =>\ A,\ 12\ =>\ B,\ 345\ =>\ B}
\item[]\textsl{val\ it\ =\ ()\ :\ unit}
\item[]\textsl{-\ }LP.output("",\ EFA.findAcceptingLP\ efa'\ (Str.input\ ""));
\item[]\textsl{@\ }012345
\item[]\textsl{@\ }.
\item[]\textsl{A,\ 0\ =>\ A,\ 1\ =>\ C,\ 2\ =>\ B,\ 3\ =>\ D,\ 4\ =>\ E,\ 5\ =>\ B}
\item[]\textsl{val\ it\ =\ ()\ :\ unit}
\end{list}


\subsection{Notes}

The algorithm for converting FAs to EFAs is obvious, but our use of
structured state names so as to make the resulting EFAs
self-documenting is novel.

\index{finite automaton!empty string|)}%
\index{empty string finite automaton|)}%
\index{EFA|)}%

%%% Local Variables: 
%%% mode: latex
%%% TeX-master: "book"
%%% End: 

\section{Nondeterministic Finite Automata}
\label{NondeterministicFiniteAutomata}

\index{finite automaton!nondeterministic|(}%
\index{nondeterministic finite automaton|(}%
\index{NFA|(}%

In this section, we study the second of our more restricted kinds of
finite automata: nondeterministic finite automata.

\subsection{Definition of NFAs}

A \emph{nondeterministic finite automaton} (NFA) $M$ is a finite
automaton such that
\begin{gather*}
T_M\sub\setof{q,x\fun r}{q,r\in\Sym\eqtxt{and}x\in\Str\eqtxt{and}|x|=1}.
\end{gather*}
In other words, an FA is an NFA iff every string of every transition
of the FA has a single symbol.
For example, $\Asf,\onesf\fun \Bsf$ is a legal NFA transition, but
$\Asf, \%\fun \Bsf$ and $\Asf,\onesf\onesf\fun \Bsf$ are not legal.
We write $\NFA$ for the set of all nondeterministic finite automata.
\index{NFA@$\NFA$}%
\index{finite automaton!NFA@$\NFA$}%
\index{nondeterministic finite automaton!NFA@$\NFA$}%
Thus $\NFA\subsetneq\EFA\subsetneq\FA$.

The following proposition obviously holds.

\begin{proposition}
Suppose $M$ is an NFA.
\begin{itemize}
\item For all $N\in\FA$, if $M\iso N$, then $N$ is an NFA.

\item For all bijections $f$ from $Q_M$ to some set of symbols,
$\renameStates(M, f)$ is an NFA.

\item $\renameStatesCanonically\,M$ is an NFA.

\item $\simplify\,M$ is an NFA.
\end{itemize}
\end{proposition}

\subsection{Converting EFAs to NFAs}

\index{empty string finite automaton!converting EFA to NFA}%
\index{nondeterministic finite automaton!converting EFA to NFA}%

Suppose $M$ is the EFA
\begin{center}
\input{chap-3.10-fig2.eepic}
\end{center}
To convert $M$ into an equivalent NFA, we will have to:
\begin{itemize}
\item replace the transitions $\Asf,\%\fun\Bsf$ and $\Bsf,\%\fun\Csf$
  with legal transitions (for example, because of the valid labeled
  path
  \begin{gather*}
    \Asf\lparr{\%}\Bsf\lparr{\onesf}\Bsf\lparr{\%}\Csf,
  \end{gather*}
  we will add the transition $\Asf,\onesf\fun\Csf$);

\item make (at least) $\Asf$ be an accepting state (so that $\%$ is
  accepted by the NFA).
\end{itemize}

Before defining a general procedure for converting EFAs to NFAs, we
first say what we mean by the empty-closure of a set of states.
Suppose $M$ is a finite automaton and $P\sub Q_M$.  The
\emph{empty-closure} of $P$ ($\emptyClose_M\,P$) is the least subset
$X$ of $Q_M$ such that
\begin{itemize}
\item $P\sub X$; and

\item for all $q,r\in Q_M$, if $q\in X$ and $q,\%\fun r\in T_M$, then
$r\in X$.
\end{itemize}
We sometimes abbreviate $\emptyClose_M\,P$ to $\emptyClose\,P$, when
$M$ is clear from the context.
For example, if $M$ is our example EFA and $P=\{\Asf\}$, then:
\begin{itemize}
\item $\Asf\in X$;

\item $\Bsf\in X$, since $\Asf\in X$ and $\Asf,\%\fun\Bsf\in T_M$;

\item $\Csf\in X$, since $\Bsf\in X$ and $\Bsf,\%\fun\Csf\in T_M$.
\end{itemize}
Thus $\emptyClose\,P=\{\Asf,\Bsf,\Csf\}$.

Suppose $M$ is a finite automaton and $P\sub Q_M$.  The \emph{backwards
empty-closure} of $P$ ($\emptyCloseBackwards_M\,P$) is the least
subset $X$ of $Q_M$ such that
\begin{itemize}
\item $P\sub X$; and

\item for all $q,r\in Q_M$, if $r\in X$ and $q,\%\fun r\in T_M$, then
$q\in X$.
\end{itemize}
We sometimes drop the $M$ from $\emptyCloseBackwards_M$, when
it's clear from the context.
For example, if $M$ is our example EFA and $P=\{\Csf\}$, then:
\begin{itemize}
\item $\Csf\in X$;

\item $\Bsf\in X$, since $\Csf\in X$ and $\Bsf,\%\fun\Csf\in T_M$;

\item $\Asf\in X$, since $\Bsf\in X$ and $\Asf,\%\fun\Bsf\in T_M$.
\end{itemize}
Thus $\emptyCloseBackwards\,P=\{\Asf,\Bsf,\Csf\}$.
  
\begin{proposition}
Suppose $M$ is a finite automaton.  For all $P\sub Q_M$,
\begin{displaymath}
\emptyClose_M\,P=\Delta_M(P,\%) .  
\end{displaymath}
\end{proposition}

In other words, $\emptyClose_M\,P$ is all of the states that can be
reached from elements of $P$ by sequences of $\%$-transitions.

\begin{proposition}
Suppose $M$ is a finite automaton.  For all $P\sub Q_M$,
\begin{displaymath}
\emptyCloseBackwards_M\,P=
\setof{q\in Q_M}{\Delta_M(\{q\},\%)\cap P\neq\emptyset} .
\end{displaymath}
\end{proposition}

In other words, $\emptyCloseBackwards_M\,P$ is all of the states from
which it is possible to reach elements of $P$ by sequences of
$\%$-transitions.

We define a function/algorithm $\efaToNFA\in\EFA\fun\NFA$ that
\index{empty string finite automaton!efaToNFA@$\efaToNFA$}%
\index{nondeterministic finite automaton!efaToNFA@$\efaToNFA$}%
converts EFAs into NFAs by saying that $\efaToNFA\,M$ is the NFA $N$
such that:
\begin{itemize}
\item $Q_N=Q_M$;

\item $s_N=s_M$;

\item $A_N=\emptyCloseBackwards\,A_M$; and

\item $T_N$ is the set of all transitions $q', a\fun r'$ such that
  $q',r'\in Q_M$, $a\in\Sym$, and there are $q,r\in Q_M$ such that:
\begin{itemize}
\item $q,a\fun r\in T_M$;
\item $q'\in{\emptyCloseBackwards\,\{q\}}$; and
\item $r'\in{\emptyClose\,\{r\}}$.
\end{itemize}
\end{itemize}

To compute the set $T_N$, we process each transition $q,x\fun r$ of
$M$ as follows.  If $x=\%$, then we generate no transitions.
Otherwise, our transition is $q, a\fun r$ for some symbol $a$.  We
then compute the backwards empty-closure of $\{q\}$, and call the
result $X$, and compute the (forwards) empty-closure of $\{r\}$, and
call the result $Y$.  We then add all of the elements of
\begin{gather*}
\setof{q', a\fun r'}{q'\in X\eqtxt{and}r'\in Y}
\end{gather*}
to $T_N$.

Because the algorithm defines $A_N$ to be $\emptyCloseBackwards\,A_M$,
it could let $T_N$ be the set of all transitions $q',a\fun r$
such that $q',r\in Q_M$, $a\in\Sym$, and there is a $q\in Q_M$ such
that:
\begin{itemize}
\item $q,a\fun r\in T_M$; and
\item $q'\in{\emptyCloseBackwards\,\{q\}}$.
\end{itemize}
This would mean that $N$ would have fewer transitions.  However, for
aesthetic reasons, we'll stick with the symmetric definition of $T_N$.

Let $M$ be our example EFA
\begin{center}
\input{chap-3.10-fig2.eepic}
\end{center}
and let $N=\efaToNFA\,M$.  Then
\begin{itemize}
\item $Q_N=Q_M=\{\Asf,\Bsf,\Csf\}$;

\item $s_N=s_M=\Asf$;

\item $A_N=\emptyCloseBackwards\,A_M=\emptyCloseBackwards\,\{\Csf\}=
  \{\Asf,\Bsf,\Csf\}$.
\end{itemize}
Now, let's work out what $T_N$ is, by processing each of $M$'s transitions.
\begin{itemize}
\item From the transitions $\Asf,\%\fun\Bsf$ and $\Bsf,\%\fun\Csf$, we
  get no elements of $T_N$.

\item Consider the transition $\Asf,\zerosf\fun\Asf$.  Since
  $\emptyCloseBackwards\,\{\Asf\}=\{\Asf\}$ and
  $\emptyClose\,\{\Asf\}=\{\Asf,\Bsf,\Csf\}$, we add
  $\Asf,\zerosf\fun\Asf$, $\Asf,\zerosf\fun\Bsf$ and
  $\Asf,\zerosf\fun\Csf$ to $T_N$.

\item Consider the transition $\Bsf,\onesf\fun\Bsf$.  Since
  $\emptyCloseBackwards\,\{\Bsf\}=\{\Asf,\Bsf\}$ and
  $\emptyClose\,\{\Bsf\}=\{\Bsf,\Csf\}$, we add
  $\Asf,\onesf\fun\Bsf$, $\Asf,\onesf\fun\Csf$, $\Bsf,\onesf\fun\Bsf$
  and $\Bsf,\onesf\fun\Csf$ to $T_N$.

\item Consider the transition $\Csf,\twosf\fun\Csf$.  Since
  $\emptyCloseBackwards\,\{\Csf\}=\{\Asf,\Bsf,\Csf\}$ and
  $\emptyClose\,\{\Csf\}=\{\Csf\}$, we add
  $\Asf,\twosf\fun\Csf$, $\Bsf,\twosf\fun\Csf$ and $\Csf,\twosf\fun\Csf$
  to $T_N$.
\end{itemize}

Thus our NFA $N$ is
\begin{center}
\input{chap-3.10-fig3.eepic}
\end{center}

\begin{theorem}
\label{EFAToNFATheorem}
For all $M\in\EFA$:
\begin{itemize}
\item $\efaToNFA\,M\approx M$; and

\item $\alphabet(\efaToNFA\,M) = \alphabet\,M$.
\end{itemize}
\end{theorem}

\subsection{Converting EFAs to NFAs, and
Processing NFAs in Forlan}

The Forlan module \texttt{FA} defines the following functions
for computing forwards and backwards empty-closures:
\begin{verbatim}
val emptyClose          : fa -> sym set -> sym set
val emptyCloseBackwards : fa -> sym set -> sym set
\end{verbatim}
\index{FA@\texttt{FA}!emptyClose@\texttt{emptyClose}}%
\index{FA@\texttt{FA}!emptyCloseBackwards@\texttt{emptyCloseBackwards}}%

For example, if \texttt{fa} is bound to the finite automaton
\begin{center}
\input{chap-3.10-fig2.eepic}
\end{center}
then we can compute the empty-closure of $\{\Asf\}$ as follows:
\begin{list}{}
{\setlength{\leftmargin}{\leftmargini}
\setlength{\rightmargin}{0cm}
\setlength{\itemindent}{0cm}
\setlength{\listparindent}{0cm}
\setlength{\itemsep}{0cm}
\setlength{\parsep}{0cm}
\setlength{\labelsep}{0cm}
\setlength{\labelwidth}{0cm}
\catcode`\#=12
\catcode`\$=12
\catcode`\%=12
\catcode`\^=12
\catcode`\_=12
\catcode`\.=12
\catcode`\?=12
\catcode`\!=12
\catcode`\&=12
\ttfamily}
\small
\item[]\textsl{-\ }SymSet.output("",\ FA.emptyClose\ fa\ (SymSet.input\ ""));
\item[]\textsl{@\ }A
\item[]\textsl{@\ }.
\item[]\textsl{A,\ B,\ C}
\item[]\textsl{val\ it\ =\ ()\ :\ unit}
\end{list}


The Forlan module \texttt{NFA} defines an abstract type \texttt{nfa}
\index{NFA@\texttt{NFA}}%
\index{NFA@\texttt{NFA}!nfa@\texttt{nfa}}%
(in the top-level environment) of nondeterministic finite automata,
along with various functions for processing NFAs.  Values of type
\texttt{nfa} are implemented as values of type \texttt{fa}, and the
module NFA provides the following injection and projection functions:
\begin{verbatim}
val injToFA     : nfa -> fa
val injToEFA    : nfa -> efa
val projFromFA  : fa -> nfa
val projFromEFA : efa -> nfa
\end{verbatim}
\index{NFA@\texttt{NFA}!injToFA@\texttt{injToFA}}%
\index{iNFA@\texttt{iNFA}!injToEFA@\texttt{injToEFA}}%
\index{proNFA@\texttt{proNFA}!projFromFA@\texttt{projFromFA}}%
\index{projNFA@\texttt{projNFA}!projFromEFA@\texttt{projFromEFA}}%
The functions \texttt{injToFA}, \texttt{injToEFA}, \texttt{projFromFA} and
\texttt{projFromEFA} are available in the top-level environment as
\texttt{injNFAToFA}, \texttt{injNFAToEFA}, \texttt{projFAToNFA} and
\texttt{projEFAToNFA}, respectively.
\index{nondeterministic finite automaton!injNFAToEFA@\texttt{injNFAToEFA}}%
\index{nondeterministic finite automaton!projFAToNFA@\texttt{projFAToNFA}}%
\index{nondeterministic finite automaton!projEFAToNFA@\texttt{projEFAToNFA}}%

The module \texttt{NFA} also defines the functions:
\begin{verbatim}
val input   : string -> nfa
val fromEFA : efa -> nfa
\end{verbatim}
\index{NFA@\texttt{NFA}!input@\texttt{input}}%
\index{NFA@\texttt{NFA}!fromEFA@\texttt{fromEFA}}%

The function \texttt{input} is used to input an NFA, and
the function \texttt{fromEFA} corresponds to our
conversion function $\efaToNFA$, and is available in the top-level
environment with that name:
\begin{verbatim}
val efaToNFA : efa -> nfa
\end{verbatim}
\index{nondeterministic finite automaton!efaToNFA@\texttt{efaToNFA}}%

Most of the functions for processing FAs that were introduced
in previous sections are inherited by \texttt{NFA}:
\begin{verbatim}
val output                  : string * nfa -> unit 
val numStates               : nfa -> int
val numTransitions          : nfa -> int
val alphabet                : nfa -> sym set
val equal                   : nfa * nfa -> bool
val checkLP                 : nfa -> lp -> unit
val validLP                 : nfa -> lp -> bool
val isomorphism             : nfa * nfa * sym_rel -> bool
val findIsomorphism         : nfa * nfa -> sym_rel
val isomorphic              : nfa * nfa -> bool
val renameStates            : nfa * sym_rel -> nfa
val renameStatesCanonically : nfa -> nfa
val processStr              : nfa -> sym set * str -> sym set
val accepted                : nfa -> str -> bool
val findLP                  : nfa -> sym set * str * sym set -> lp
val findAcceptingLP         : nfa -> str -> lp
val simplified              : nfa -> bool
val simplify                : nfa -> nfa
\end{verbatim}
\index{NFA@\texttt{NFA}!output@\texttt{output}}%
\index{NFA@\texttt{NFA}!numStates@\texttt{numStates}}%
\index{NFA@\texttt{NFA}!numTransitions@\texttt{numTransitions}}%
\index{NFA@\texttt{NFA}!alphabet@\texttt{alphabet}}%
\index{NFA@\texttt{NFA}!equal@\texttt{equal}}%
\index{NFA@\texttt{NFA}!checkLP@\texttt{checkLP}}%
\index{NFA@\texttt{NFA}!validLP@\texttt{validLP}}%
\index{NFA@\texttt{NFA}!isomorphism@\texttt{isomorphism}}%
\index{NFA@\texttt{NFA}!findIsomorphism@\texttt{findIsomorphism}}%
\index{NFA@\texttt{NFA}!isomorphic@\texttt{isomorphic}}%
\index{NFA@\texttt{NFA}!renameStates@\texttt{renameStates}}%
\index{NFA@\texttt{NFA}!renameStatesCanonically@\texttt{renameStatesCanonically}}%
\index{NFA@\texttt{NFA}!processStr@\texttt{processStr}}%
\index{NFA@\texttt{NFA}!accepted@\texttt{accepted}}%
\index{NFA@\texttt{NFA}!findLP@\texttt{findLP}}%
\index{NFA@\texttt{NFA}!findAcceptingLP@\texttt{findAcceptingLP}}%
\index{NFA@\texttt{NFA}!simplified@\texttt{simplified}}%
\index{NFA@\texttt{NFA}!simplify@\texttt{simplify}}%
Finally, the functions for computing forwards and backwards
empty-closures are inherited by the EFA module
\begin{verbatim}
val emptyClose          : efa -> sym set -> sym set
val emptyCloseBackwards : efa -> sym set -> sym set
\end{verbatim}
\index{EFA@\texttt{EFA}!emptyClose@\texttt{emptyClose}}%
\index{EFA@\texttt{EFA}!emptyCloseBackwards@\texttt{emptyCloseBackwards}}%

Suppose that \texttt{efa} is the \texttt{efa}
\begin{center}
\input{chap-3.10-fig2.eepic}
\end{center}
Here are some example uses of a few of the above functions:
\begin{list}{}
{\setlength{\leftmargin}{\leftmargini}
\setlength{\rightmargin}{0cm}
\setlength{\itemindent}{0cm}
\setlength{\listparindent}{0cm}
\setlength{\itemsep}{0cm}
\setlength{\parsep}{0cm}
\setlength{\labelsep}{0cm}
\setlength{\labelwidth}{0cm}
\catcode`\#=12
\catcode`\$=12
\catcode`\%=12
\catcode`\^=12
\catcode`\_=12
\catcode`\.=12
\catcode`\?=12
\catcode`\!=12
\catcode`\&=12
\ttfamily}
\small
\item[]\textsl{-\ }projEFAToNFA\ efa;
\item[]\textsl{invalid\ label\ in\ transition:\ "%"}
\item[]
\item[]\textsl{uncaught\ exception\ Error}
\item[]\textsl{-\ }val\ nfa\ =\ efaToNFA\ efa;
\item[]\textsl{val\ nfa\ =\ -\ :\ nfa}
\item[]\textsl{-\ }NFA.output("",\ nfa);
\item[]\textsl{\symbol{'173}states\symbol{'175}\ A,\ B,\ C\ \symbol{'173}start\ state\symbol{'175}\ A\ \symbol{'173}accepting\ states\symbol{'175}\ A,\ B,\ C}
\item[]\textsl{\symbol{'173}transitions\symbol{'175}}
\item[]\textsl{A,\ 0\ ->\ A\ |\ B\ |\ C;\ A,\ 1\ ->\ B\ |\ C;\ A,\ 2\ ->\ C;\ B,\ 1\ ->\ B\ |\ C;}
\item[]\textsl{B,\ 2\ ->\ C;\ C,\ 2\ ->\ C}
\item[]\textsl{val\ it\ =\ ()\ :\ unit}
\item[]\textsl{-\ }LP.output("",\ EFA.findAcceptingLP\ efa\ (Str.input\ ""));
\item[]\textsl{@\ }012
\item[]\textsl{@\ }.
\item[]\textsl{A,\ 0\ =>\ A,\ %\ =>\ B,\ 1\ =>\ B,\ %\ =>\ C,\ 2\ =>\ C}
\item[]\textsl{val\ it\ =\ ()\ :\ unit}
\item[]\textsl{-\ }LP.output("",\ NFA.findAcceptingLP\ nfa\ (Str.input\ ""));
\item[]\textsl{@\ }012
\item[]\textsl{@\ }.
\item[]\textsl{A,\ 0\ =>\ A,\ 1\ =>\ B,\ 2\ =>\ C}
\item[]\textsl{val\ it\ =\ ()\ :\ unit}
\end{list}


\subsection{Notes}

Because we have defined the meaning of finite automata via labeled
paths instead of transition functions, our EFA to NFA conversion algorithm
is easy to understand and prove correct.

\index{finite automaton!nondeterministic|)}%
\index{nondeterministic finite automaton|)}%
\index{NFA|)}%

%%% Local Variables: 
%%% mode: latex
%%% TeX-master: "book"
%%% End: 

\section{Deterministic Finite Automata}
\label{DeterministicFiniteAutomata}

\index{finite automaton!deterministic|(}%
\index{deterministic finite automaton|(}%
\index{DFA|(}%

In this section, we study the third of our more restricted kinds of
finite automata: deterministic finite automata.

\subsection{Definition of DFAs}

A \emph{deterministic finite automaton} (DFA) $M$ is a finite
automaton such that:
\begin{itemize}
\item $T_M\sub\setof{q,x\fun r}{q,r\in\Sym\eqtxt{and}x\in\Str\eqtxt{and}|x|=1}$;
  and

\item for all $q\in Q_M$ and $a\in\alphabet\,M$, there is a unique $r\in Q_M$
such that $q,a\fun r\in T_M$.
\end{itemize}
In other words, an FA is a DFA iff it is an NFA and, for every state
$q$ of the automaton and every symbol $a$ of the automaton's alphabet,
there is exactly one state that can be entered from state $q$ by
\index{finite automaton!DFA@$\DFA$}%
\index{deterministic finite automaton!DFA@$\DFA$}%
reading $a$ from the automaton's input.  We write $\DFA$ for the set
of all deterministic finite automata.  Thus
$\DFA\subsetneq\NFA\subsetneq\EFA\subsetneq\FA$.

Let $M$ be the finite automaton
\begin{center}
\input{chap-3.11-fig1.eepic}
\end{center}
It turns out that $L(M)=
\setof{w\in\{\mathsf{0,1}\}^*}{\mathsf{000}\eqtxt{is not a substring
    of}w}$.  Although $M$ is an NFA, it's not a DFA, since
$\zerosf\in\alphabet\,M$ but there is no transition of the form
$\Csf,\zerosf\fun r$.  However, we can make $M$ into a DFA by adding a dead
state $\Dsf$:
\begin{center}
\input{chap-3.11-fig2.eepic}
\end{center}
We will never need more than one dead state in a DFA.

The following proposition obviously holds.

\begin{proposition}
Suppose $M$ is a DFA.
\begin{itemize}
\item For all $N\in\FA$, if $M\iso N$, then $N$ is a DFA.

\item For all bijections $f$ from $Q_M$ to some set of symbols,
$\renameStates(M, f)$ is a DFA.

\item $\renameStatesCanonically\,M$ is a DFA.
\end{itemize}
\end{proposition}

Now, we prove a proposition that doesn't hold for arbitary NFAs.

\begin{proposition}
\label{DetermProp1}

Suppose $M$ is a DFA.  For all $q\in Q_M$ and $w\in(\alphabet\,M)^*$,
$|\Delta_M(\{q\},w)|=1${.}
\end{proposition}

\begin{proof}
An easy left string induction on $w$.
\end{proof}

\index{deterministic finite automaton!transition function}%
\index{deterministic finite automaton!delta@$\delta$}%
Suppose $M$ is a DFA.  Because of Proposition~\ref{DetermProp1}, we
can define \emph{the transition function} $\delta_M$ \emph{for} $M$,
$\delta_M\in Q_M\times(\alphabet\,M)^*\fun Q_M$, by:
\begin{gather*}
\delta_M(q,w) = \eqtxtr{the unique} r\in Q_M\eqtxt{such that}
r\in\Delta_M(\{q\},w) .
\end{gather*}
In other words, $\delta_M(q,w)$ is the unique state $r$ of $M$ that
is the end of a valid labeled path for $M$ that starts at $q$ and
is labeled by $w$.
Thus, for all $q,r\in Q_M$ and $w\in(\alphabet\,M)^*$,
\begin{gather*}
\delta_M(q,w)=r\quad\myiff\quad r\in\Delta_M(\{q\},w) .
\end{gather*}
We sometimes abbreviate $\delta_M(q,w)$ to $\delta(q,w)$.

For example, if $M$ is the DFA
\begin{center}
\input{chap-3.11-fig2.eepic}
\end{center}
then $\delta(\Asf, \%)={\Asf}$, $\delta(\Asf, \mathsf{0100})={\Csf}$
and $\delta(\Bsf, \mathsf{000100})={\Dsf}$.

Having defined the $\delta$ function, we can study its properties.

\begin{proposition}
\label{DetermProp2}
Suppose $M$ is a DFA.
\begin{enumerate}[\quad(1)]
\item For all $q\in Q_M$, $\delta_M(q,\%)={q}$.

\item For all $q\in Q_M$ and $a\in\alphabet\,M$,
$\delta_M(q,a)=\eqtxtr{the unique}r\in Q_M\eqtxt{such that}
{q,a\fun r\in T_M}$.

\item For all $q\in Q_M$ and $x,y\in(\alphabet\,M)^*$,
$\delta_M(q,xy)={\delta_M(\delta_M(q,x),y)}$.
\end{enumerate}
\end{proposition}

Suppose $M$ is a DFA.  By part~(2) of the proposition, we have that,
for all $q,r\in Q_M$ and $a\in\alphabet\,M$,
\begin{gather*}
\delta_M(q,a)=r\quad\myiff\quad q,a\fun r\in T_M .
\end{gather*}

Now we can use the $\delta$ function to explain when a string is
accepted by an FA.

\begin{proposition}
\label{DetermProp3}
Suppose $M$ is a DFA.
$L(M)=\setof{w\in(\alphabet\,M)^*}{\delta_M(s_M,w)\in A_M}$.
\end{proposition}
\begin{proof}
To see that the left-hand side is a subset of the right-hand side,
suppose $w\in L(M)$.  Then $w\in(\alphabet\,M)^*$ and there is a $q\in A_M$
such that $q\in\Delta_M(\{s_M\},w)$.  Thus $\delta_M(s_M,w)=q\in A_M$.

To see that the right-hand side is a subset of the left-hand side,
suppose $w\in(\alphabet\,M)^*$ and $\delta_M(s_M,w)\in A_M$.  Then
$\delta_M(s_M,w)\in\Delta_M(\{s_M\},w)$, and thus $w\in L(M)$.
\end{proof}

The preceding propositions give us an efficient algorithm for checking
whether a string is accepted by a DFA.  For example, suppose
$M$ is the DFA
\begin{center}
\input{chap-3.11-fig2.eepic}
\end{center}
To check whether $\mathsf{0100}$ is accepted by $M$, we need
to determine whether $\delta(\Asf,\mathsf{0100})\in\{\Asf,\Bsf,\Csf\}$.

For instance, we have that:
\begin{align*}
\delta(\Asf,\mathsf{0100})
&= \delta(\delta(\Asf, \zerosf),\mathsf{100}) \\
&= \delta(\Bsf,\mathsf{100}) \\
&= \delta(\delta(\Bsf,\onesf),\mathsf{00}) \\
&= \delta(\Asf,\mathsf{00}) \\
&= \delta(\delta(\Asf,\zerosf),\zerosf) \\
&= \delta(\Bsf,\zerosf) \\
&= \Csf \\
&\in \{\Asf,\Bsf,\Csf\}.
\end{align*}
Thus $\mathsf{0100}$ is accepted by $M$.

It is easy to see that, for all DFAs $M$ and $q\in Q_M$:
\begin{itemize}
\item $q$ is reachable in $M$ iff there is a $w\in(\alphabet\, M)^*$,
  $\delta_M(s_M, w) = q$;
\index{reachable state}%
\index{deterministic finite automaton!reachable state}%
and

\item $q$ is live in $M$ iff there is a $w\in(\alphabet\, M)^*$,
  $\delta_M(q, w) = r$, for some $r\in A_M$.
\index{live state}%
\index{deterministic finite automaton!live state}%
\end{itemize}

\subsection{Proving the Correctness of DFAs}

\index{deterministic finite automaton!proving correctness}%
Since every DFA is an FA, we could prove the correctness of DFAs
using the techniques that we have already studied.
But it turns out that giving a separate proof that enough is accepted
by a DFA is unnecessary---it will follow from the proof that
everything accepted is wanted.

\begin{proposition}
Suppose $M$ is a DFA.  Then, for all $w\in(\alphabet\,M)^*$ and $q\in Q_M$,
\begin{gather*}
w\in\Lambda_{M,q} \quad\myiff\quad \delta_M(s_M,w) = q .
\end{gather*}
\end{proposition}

\begin{proof}
Suppose $w\in(\alphabet\,M)^*$ and $q\in Q_M$.
\begin{description}
\item[\quad(only if)] Suppose $w\in\Lambda_q$.  Then
  $q\in\Delta(\{s\},w)$.  Thus $\delta(s,w) = q$.

\item[\quad(if)] Suppose $\delta(s,w) = q$.  Then
  $q\in\Delta(\{s\},w)$, so that $w\in\Lambda_q$.
\end{description}
\end{proof}

We already know that, if $M$ is an FA, then
$L(M)=\bigcup\setof{\Lambda_q}{q\in A_M}$.

\begin{proposition}
Suppose $M$ is a DFA.
\begin{enumerate}[\quad(1)]
\item $(\alphabet\,M)^* = \bigcup\setof{\Lambda_q}{q\in Q_M}$.

\item For all $q,r\in Q_M$, if $q\neq r$, then
  $\Lambda_q\cap\Lambda_r=\emptyset$.
\end{enumerate}
\end{proposition}

Suppose $M$ is the DFA
\begin{center}
\input{chap-3.11-fig2.eepic}
\end{center}
and let $X=\setof{w\in\{\mathsf{0,1}\}^*}{\mathsf{000}\eqtxt{is not a
    substring of}w}$.  We will show that $L(M)=X$.  Note that, for all
$w\in\{\mathsf{0,1}\}^*$:
\begin{itemize}
\item $w\in X$ iff $\mathsf{000}$ is not a substring of $w$.

\item $w\not\in X$ iff $\mathsf{000}$ is a substring of $w$.
\end{itemize}

First, we use induction on $\Lambda$, to prove that:
\begin{enumerate}[\quad(A)]
\item for all $w\in\Lambda_\Asf$, $w\in X$ and $\zerosf$ is not a
  suffix of $w$;

\item for all $w\in\Lambda_\Bsf$, $w\in X$ and $\zerosf$, but not
  $\zerosf\zerosf$, is a suffix of $w$;

\item for all $w\in\Lambda_\Csf$, $w\in X$ and $\zerosf\zerosf$ is a
  suffix of $w$; and

\item for all $w\in\Lambda_\Dsf$, $w\not\in X$.
\end{enumerate}

There are nine steps (1 + the number of transitions) to show.
\begin{description}
\item[\quad(empty string)] We must show that  $\%\in X$ and
  $\zerosf$ is not a suffix of $\%$.  This follows since $\%$ has
  no $\zerosf$'s.

\item[\quad($\Asf, \zerosf\fun\Bsf$)] Suppose $w\in\Lambda_\Asf$, and
  assume the inductive hypothesis: $w\in X$ and $\zerosf$ is not a
  suffix of $w$.  We must show that $w\zerosf\in X$ and
  $\zerosf$, but not $\zerosf\zerosf$, is a suffix of
  $w\zerosf$. Because $w\in X$ and $\zerosf$ is not a suffix of $w$,
  we have that $w\zerosf\in X$.  Clearly, $\zerosf$ is a suffix of
  $w\zerosf$.  And, since $\zerosf$ is not a suffix of $w$, we have
  that $\zerosf\zerosf$ is not a suffix of $w\zerosf$.

\item[\quad($\Asf, \onesf\fun\Asf$)] Suppose $w\in\Lambda_\Asf$, and
  assume the inductive hypothesis: $w\in X$ and $\zerosf$ is not a
  suffix of $w$.  We must show that $w\onesf\in X$ and $\zerosf$ is
  not a suffix of $w\onesf$.  Since $w\in X$, we have that $w\onesf\in
  X$.  And, $\zerosf$ is not a suffix of $w\onesf$.

\item[\quad($\Bsf, \zerosf\fun\Csf$)] Suppose $w\in\Lambda_\Bsf$, and
  assume the inductive hypothesis: $w\in X$ and $\zerosf$, but not
  $\zerosf\zerosf$, is a suffix of $w$.  We must show that
  $w\zerosf\in X$ and $\zerosf\zerosf$ is a suffix of $w\zerosf$.
  Because $w\in X$ and $\zerosf\zerosf$ is not suffix of $w$, we have
  that $w\zerosf\in X$.  And since $\zerosf$ is a suffix of $w$, it
  follows that $\zerosf\zerosf$ is a suffix of $w\zerosf$.

\item[\quad($\Bsf, \onesf\fun\Asf$)] Suppose $w\in\Lambda_\Bsf$, and
  assume the inductive hypothesis: $w\in X$ and $\zerosf$, but not
  $\zerosf\zerosf$, is a suffix of $w$.  We must show that $w\onesf\in
  X$ and $\zerosf$ is not a suffix of $w\onesf$.  Because $w\in X$, we
  have that $w\onesf\in X$.  And, $\zerosf$ is not a suffix of
  $w\onesf$.

\item[\quad($\Csf, \zerosf\fun\Dsf$)] Suppose $w\in\Lambda_\Csf$, and
  assume the inductive hypothesis: $w\in X$ and $\zerosf\zerosf$ is a
  suffix of $w$.  We must show that $w\zerosf\not\in X$.  Because
  $\zerosf\zerosf$ is a suffix of $w$, we have that $\mathsf{000}$ is
  a suffix of $w\zerosf$.  Thus $w\zerosf\not\in X$.

\item[\quad($\Csf, \onesf\fun\Asf$)] Suppose $w\in\Lambda_\Csf$, and
  assume the inductive hypothesis: $w\in X$ and $\zerosf\zerosf$ is a
  suffix of $w$.  We must show that $w\onesf\in X$ and $\zerosf$ is
  not a suffix of $w\onesf$.  Because $w\in X$, we have that
  $w\onesf\in X$.  And, $\zerosf$ is not a suffix of $w\onesf$.

\item[\quad($\Dsf, \zerosf\fun\Dsf$)] Suppose $w\in\Lambda_\Dsf$, and
  assume the inductive hypothesis: $w\not\in X$.  We must show that
  $w\zerosf\not\in X$.  Because $w\not\in X$, we have that
  $w\zerosf\not\in X$.

\item[\quad($\Dsf, \onesf\fun\Dsf$)] Suppose $w\in\Lambda_\Dsf$, and
  assume the inductive hypothesis: $w\not\in X$.  We must show that
  $w\onesf\not\in X$.   Because $w\not\in X$, we have that
  $w\onesf\not\in X$.
\end{description}

Now, we use the result of our induction on $\Lambda$ to show that
$L(M)=X$.

\begin{description}
\item[\quad($L(M)\sub X$)] Suppose $w\in L(M)$.  Because
  $A_M=\{\Asf,\Bsf,\Csf\}$, we have that $w\in L(M) = \Lambda_\Asf
  \cup \Lambda_\Bsf \cup \Lambda_\Csf$.  Thus, by parts~(A)--(C), we
  have that $w\in X$.

\item[\quad($X\sub L(M)$)] Suppose $w\in X$.  Since
  $X\sub\{\zerosf,\onesf\}^*$, we have that
  $w\in\{\zerosf,\onesf\}^*$.  Suppose, toward a contradiction, that
  $w\not\in L(M)$.  Because $w\not\in L(M) = \Lambda_\Asf \cup
  \Lambda_\Bsf \cup \Lambda_\Csf$ and $w\in\{\zerosf,\onesf\}^* =
  (\alphabet\,M)^* = \Lambda_\Asf \cup \Lambda_\Bsf \cup \Lambda_\Csf
  \cup \Lambda_\Dsf$, we must have that $w\in\Lambda_\Dsf$.  But then
  part~(D) tells us that $w\not\in X$---contradiction.  Thus $w\in
  L(M)$.
\end{description}

For the above approach to work, when proving $L(M)=X$ for a DFA $M$
and language $X$, we simply need that:
\begin{itemize}
\item the property associated with each accepting state implies
  being in $X$; and

\item the property associated with each non-accepting state implies
  not being in $X$.
\end{itemize}

\begin{exercise}
\label{AllGoodDFACorrLem}
Define $\diff\in\{\mathsf{0,1}\}^*\fun\ints$ by:
\index{diff@$\diff$}%
\index{string!diff@$\diff$}%
\index{difference function}%
\index{string!difference function}%
for all $w\in\{\mathsf{0,1}\}^*$,
\begin{displaymath}
\diff\,w =
\eqtxtr{the number of $\mathsf{1}$'s in}w -
\eqtxtr{the number of $\mathsf{0}$'s in}w .
\end{displaymath}
Define $\AllPrefixGood$ to be the language
$\setof{w\in\{\mathsf{0,1}\}^*}{\eqtxtr{for all prefixes} v\eqtxt{of}
  w, |\diff\,v|\leq 2}$.  Find a DFA $\allPrefixGoodDFA$ such that
$L(\allPrefixGoodDFA) = \AllPrefixGood$, and prove that your solution
is correct.
\end{exercise}

\subsection{Simplification of DFAs}

\index{deterministic finite automaton!simplification}%
\index{simplification!deterministic finite automaton}%
Let $M$ be our example DFA
\begin{center}
\input{chap-3.11-fig2.eepic}
\end{center}
Then $M$ is not simplified, since the state $\Dsf$ is dead.  But if we
get rid of $\Dsf$, then we won't have a DFA anymore.  Thus, we will
need:
\begin{itemize}
\item a notion of when a DFA is simplified that is more liberal than
  our standard notion; and

\item a corresponding simplification procedure for DFAs.
\end{itemize}
\index{deterministically simplified}%
\index{deterministic finite automaton!deterministically simplified}%
We say that a DFA $M$ is \emph{deterministically simplified} iff
\begin{itemize}
\item every element of $Q_M$ is reachable; and

\item at most one element of $Q_M$ is dead.
\end{itemize}
For example, both of the following DFAs, which accept $\emptyset$,
are deterministically simplified:
\begin{center}
\input{chap-3.11-fig3.eepic}
\end{center}

We define a simplification algorithm for DFAs that takes in
\begin{itemize}
\item a DFA $M$ and
\item an alphabet $\Sigma$
\end{itemize}
and returns a DFA $N$ such that
\begin{itemize}
\item $N$ is deterministically simplified,

\item $N\approx M$,

\item $\alphabet\,N = \alphabet(L(M))\cup\Sigma$, and

\item if $\Sigma\sub\alphabet(L(M))$, then $|Q_N| \leq |Q_M|$.
\end{itemize}
Thus, the alphabet of $N$ will consist of all symbols that
either appear in strings that are accepted by $M$ or are in $\Sigma$.

The algorithm begins by letting the FA $M'$ be $\simplify\,M$, i.e.,
the result of running our simplification algorithm for FAs on $M$.
$M'$ will have the following properties:
\begin{itemize}
\item $Q_{M'}\sub Q_M$ and $T_{M'}\sub T_M$;

\item $M'$ is simplified;

\item $M'\approx M$;

\item $\alphabet\,M'=\alphabet(L(M'))=\alphabet(L(M))$; and

\item for all $q\in Q_{M'}$ and $a\in\alphabet\,M'$, there is at most
  one $r\in Q_{M'}$ such that $q,a\fun r\in T_{M'}$ (this property
  holds since $M$ is a DFA, $Q_{M'}\sub Q_M$ and $T_{M'}\sub T_M$).
\end{itemize}

Let $\Sigma'=\alphabet\,M'\cup\Sigma=\alphabet(L(M))\cup\Sigma$.  If
$M'$ is a DFA and $\alphabet\,M'=\Sigma'$, the algorithm returns $M'$
as its DFA, $N$. Because $M'$ is simplified, all states of $M'$ are
reachable, and either $M'$ has no dead states, or it consists
of a single dead state (the start state). In either case, $M'$
is deterministically simplified. Because $Q_{M'}\sub Q_M$, we
have $|Q_N| \leq |Q_M|$.

Otherwise, it must turn $M'$ into a DFA whose alphabet is $\Sigma'$.
We have that
\begin{itemize}
\item $\alphabet\,M'\sub\Sigma'$; and

\item for all $q\in Q_{M'}$ and $a\in\Sigma'$, there is at most one
  $r\in Q_{M'}$ such that $q,a\fun r\in T_{M'}$.
\end{itemize}

Since $M'$ is simplified, there are two cases to consider.  If $M'$
has no accepting states, then $s_{M'}$ is the only state of $M'$ and
$M'$ has no transitions.  Thus the algorithm can return the DFA $N$
defined by:
\begin{itemize}
\item $Q_N=Q_{M'}=\{s_{M'}\}$;

\item $s_N=s_{M'}$;

\item $A_N=A_{M'}=\emptyset$; and

\item $T_N=\setof{s_{M'},a\fun s_{M'}}{a\in\Sigma'}$.
\end{itemize}
In this case, we have that $|Q_N| \leq |Q_M|$.

Alternatively, $M'$ has at least one accepting state, so that $M'$ has
no dead states. (Consider the case when $\Sigma\sub\alphabet(L(M))$,
so that $\Sigma' = \alphabet(L(M)) = \alphabet\,M'$.  Suppose, toward
a contradiction, that $Q_{M'} = Q_M$, so that all elements of $Q_M$
are useful.  Then $s_{M'} = s_M$ and $A_{M'} = A_M$.  And
$T_{M'} = T_{M}$, since no transitions of a DFA are redundant. Hence
$M' = M$, so that $M'$ is a DFA with alphabet $\Sigma'$---a
contradiction. Thus $Q_{M'}\subsetneq Q_M$.)

Thus the DFA $N$ returned by the algorithm is defined by:
\begin{itemize}
\item $Q_N = Q_{M'}\cup\{\dead\}$ (we put enough brackets around
$\dead$ so that it's not in $Q_{M'}$);

\item $s_N=s_{M'}$;

\item $A_N=A_{M'}$; and

\item $T_N=T_{M'}\cup T'$, where $T'$ is the set of all transitions
  $q,a\fun\dead$ such that either
\begin{itemize}
\item $q\in Q_{M'}$ and $a\in\Sigma'$, but there is no $r\in Q_{M'}$
  such that $q,a\fun r\in T_{M'}$; or

\item $q=\dead$ and $a\in\Sigma'$.
\end{itemize}
\end{itemize}
(If $\Sigma\sub\alphabet(L(M))$, then $|Q_N|\leq |Q_M|$.)

\index{deterministic finite automaton!determSimplify@$\determSimplify$}
We define a function $\determSimplify\in\DFA\times\Alp\fun\DFA$
by: $\determSimplify(M,\Sigma)$ is the result of
running the above algorithm on $M$ and $\Sigma$.

\begin{theorem}
For all $M\in\DFA$ and $\Sigma\in\Alp$:
\begin{itemize}
\item $\determSimplify(M,\Sigma)$ is deterministically simplified;

\item $\determSimplify(M,\Sigma)\approx M$;

\item $\alphabet(\determSimplify(M,\Sigma)) = \alphabet(L(M))\cup\Sigma$; and

\item if $\Sigma\sub\alphabet(L(M))$, then 
  $|Q_{\determSimplify(M, \Sigma)}| \leq |Q_M|$.
\end{itemize}
\end{theorem}

For example, suppose $M$ is the DFA
\begin{center}
\input{chap-3.11-fig4.eepic}
\end{center}
Then $\determSimplify(M,\{\twosf\})$ is the DFA
\begin{center}
\input{chap-3.11-fig5.eepic}
\end{center}

\begin{exercise}
Find a DFA $M$ and alphabet $\Sigma$ such that
$\determSimplify(M,\Sigma)$ has one more state than does $M$.
\end{exercise}

\subsection{Converting NFAs to DFAs}

\index{nondeterministic finite automaton!converting NFA to DFA}%
\index{deterministic finite automaton!converting NFA to DFA}%
Suppose $M$ is the NFA
\begin{center}
\input{chap-3.11-fig6.eepic}
\end{center}
Our algorithm for converting NFAs to DFAs will convert $M$ into a DFA
$N$ whose states represent the elements of the set
\begin{gather*}
\setof{\Delta_M(\{\Asf\},w)}{w\in\{\zerosf,\onesf\}^*} .
\end{gather*}
For example, one the states of $N$ will be $\langle\Asf,\Bsf\rangle$,
which represents $\{\Asf,\Bsf\}=\Delta_M(\{\Asf\},\mathsf{1})$.
This is the state that our DFA will be in after processing $\onesf$
from the start state.

Before describing our conversion algorithm, we first state a
proposition concerning the $\Delta$ function for NFAs, and say how we
will represent finite sets of symbols as symbols.

\begin{proposition}
\label{NFADeltaProp}
Suppose $M$ is an NFA.
\begin{enumerate}[\quad(1)]
\item For all $P\sub Q_M$, $\Delta_M(P,\%)=P$.

\item For all $P\sub Q_M$ and $a\in\alphabet\,M$,
  \begin{displaymath}
    \Delta_M(P,a)=\setof{r\in Q_M}{p,a\fun r\in T_M,\eqtxt{for some}p\in P}.
  \end{displaymath}

\item For all $P\sub Q_M$ and $x,y\in(\alphabet\,M)^*$,
  \begin{displaymath}
    \Delta_M(P, xy)=\Delta_M(\Delta_M(P,x),y) .
  \end{displaymath}
\end{enumerate}
\end{proposition}

Given a finite set of symbols $P$, we write $\overline{P}$ for
the symbol
\begin{gather*}
\langle a_1,\,\ldots,a_n\rangle,
\end{gather*}
where $a_1,\,\ldots,a_n$ are all of the elements of $P$, in order
according to our total ordering on $\Sym$, and without repetition.  For
example, $\overline{\{\Bsf,\Asf\}}=\langle\Asf,\Bsf\rangle$ and
$\overline{\emptyset}=\langle\rangle$.
It is easy to see that, if $P$ and $R$ are finite sets of symbols, then
$\overline{P}=\overline{R}$ iff $P=R$.

We convert an NFA $M$ into a DFA $N$ as follows.  First,
we generate the least subset $X$ of $\powset\,Q_M$ such that:
\begin{itemize}
\item $\{s_M\}\in X$; and

\item for all $P\in X$ and $a\in\alphabet\,M$,
$\Delta_M(P,a)\in X$.
\end{itemize}
Thus $|X|\leq 2^{|Q_M|}$.  Then we define the DFA $N$ as follows:
\begin{itemize}
\item $Q_N=\setof{\overline{P}}{P\in X}$;

\item $s_N=\overline{\{s_M\}}=\langle s_M\rangle$;

\item $A_N=\setof{\overline{P}}{P\in X\eqtxt{and}P\cap A_M\neq\emptyset}$;
  and

\item $T_N=\setof{(\overline{P},a,
\overline{\Delta_M(P,a)})}{P\in X\eqtxt{and}a\in\alphabet\,M}$.
\end{itemize}
Then $N$ is a DFA with alphabet $\alphabet\,M$ and, for all $P\in X$ and
$a\in\alphabet\,M$, $\delta_N(\overline{P},a)=\overline{\Delta_M(P,a)}$.

Suppose $M$ is the NFA
\begin{center}
\input{chap-3.11-fig6.eepic}
\end{center}
Let's work out what the DFA $N$ is.
\begin{itemize}
\item To begin with, $\{\Asf\}\in X$, so that $\langle\Asf\rangle\in
  Q_N$.  And $\langle\Asf\rangle$ is the start state of $N$.  It is
  not an accepting state, since $\Asf\not\in A_M$.

\item Since $\{\Asf\}\in X$, and $\Delta(\{\Asf\},\zerosf)=\emptyset$,
  we add $\emptyset$ to $X$, $\langle\rangle$ to $Q_N$ and
  $\langle\Asf\rangle,\zerosf\fun\langle\rangle$ to $T_N$.

  Since $\{\Asf\}\in X$, and $\Delta(\{\Asf\},\onesf)=\{\Asf,\Bsf\}$,
  we add $\{\Asf,\Bsf\}$ to $X$, $\langle\Asf,\Bsf\rangle$ to $Q_N$
  and $\langle\Asf\rangle,\onesf\fun\langle\Asf,\Bsf\rangle$ to $T_N$.

\item Since $\emptyset\in X$, $\Delta(\emptyset,\zerosf)=\emptyset$
  and $\emptyset\in X$, we don't have to add anything to $X$ or $Q_N$,
  but we add $\langle\rangle,\zerosf\fun\langle\rangle$ to $T_N$.

  Since $\emptyset\in X$, $\Delta(\emptyset,\onesf)=\emptyset$ and
  $\emptyset\in X$, we don't have to add anything to $X$ or $Q_N$, but
  we add $\langle\rangle,\onesf\fun\langle\rangle$ to $T_N$.

\item Since $\{\Asf,\Bsf\}\in X$,
  $\Delta(\{\Asf,\Bsf\},\zerosf)=\emptyset$ and $\emptyset\in X$, we
  don't have to add anything to $X$ or $Q_N$, but we add
  $\langle\Asf,\Bsf\rangle,\zerosf\fun\langle\rangle$ to $T_N$.

  Since $\{\Asf,\Bsf\}\in X$,
  $\Delta(\{\Asf,\Bsf\},\onesf)=\{\Asf,\Bsf\}\cup\{\Csf\}=
  \{\Asf,\Bsf,\Csf\}$, we add $\{\Asf,\Bsf,\Csf\}$ to $X$,
  $\langle\Asf,\Bsf,\Csf\rangle$ to $Q_N$, and
  $\langle\Asf,\Bsf\rangle,\onesf\fun\langle\Asf,\Bsf,\Csf\rangle$ to
  $T_N$.  Since $\{\Asf,\Bsf,\Csf\}$ contains (the only) one of $M$'s
  accepting states, we add $\langle\Asf,\Bsf,\Csf\rangle$ to $A_N$.

\item Since $\{\Asf,\Bsf,\Csf\}\in X$ and
  $\Delta(\{\Asf,\Bsf,\Csf\},\zerosf)=\emptyset\cup\emptyset\cup\{\Csf\}=
  \{\Csf\}$, we add $\{\Csf\}$ to $X$, $\langle\Csf\rangle$ to $Q_N$
  and $\langle\Asf,\Bsf,\Csf\rangle,\zerosf\fun\langle\Csf\rangle$ to
  $T_N$.  Since $\{\Csf\}$ contains one of $M$'s accepting states, we
  add $\langle\Csf\rangle$ to $A_N$.

  Since $\{\Asf,\Bsf,\Csf\}\in X$,
  $\Delta(\{\Asf,\Bsf,\Csf\},\onesf)=\{\Asf,\Bsf\}\cup\{\Csf\}\cup\emptyset=
  \{\Asf,\Bsf,\Csf\}$ and $\{\Asf,\Bsf,\Csf\}\in X$, we don't have to
  add anything to $X$ or $Q_N$, but we add
  $\langle\Asf,\Bsf,\Csf\rangle,\onesf\fun\langle\Asf,\Bsf,\Csf\rangle$
  to $T_N$.

\item Since $\{\Csf\}\in X$, $\Delta(\{\Csf\},\zerosf)=\{\Csf\}$ and
  $\{\Csf\}\in X$, we don't have to add anything to $X$ or $Q_N$, but
  we add $\langle\Csf\rangle,\zerosf\fun\langle\Csf\rangle$ to $T_N$.

  Since $\{\Csf\}\in X$, $\Delta(\{\Csf\},\onesf)=\emptyset$ and
  $\emptyset\in X$, we don't have to add anything to $X$ or $Q_N$, but
  we add $\langle\Csf\rangle,\onesf\fun\langle\rangle$ to $T_N$.
\end{itemize}
Since there are no more elements to add to $X$, we are done.
Thus, the DFA $N$ is
\begin{center}
\input{chap-3.11-fig7.eepic}
\end{center}

The following two lemmas show why our conversion process is correct.

\begin{lemma}
\label{NFAToDFAConvLemma}
For all $w\in(\alphabet\,M)^*$:
\begin{itemize}
\item $\Delta_M(\{s_M\},w)\in X$; and
\item $\delta_N(s_N,w)=\overline{\Delta_M(\{s_M\},w)}$.
\end{itemize}
\end{lemma}

\begin{proof}
By left string induction.
\begin{description}
\item[\quad(Basis Step)] We have that $\Delta_M(\{s_M\},\%) = \{s_M\}\in X$
and $\delta_N(s_N,\%)= {s_N}= {\overline{\{s_M\}}}=
{\overline{\Delta_M(\{s_M\},\%)}}$.

\item[\quad(Inductive Step)] Suppose $a\in\alphabet\,M$ and
  $w\in(\alphabet\,M)^*$.  Assume the inductive hypothesis:
  $\Delta_M(\{s_M\},w)\in X$ and
  $\delta_N(s_N,w)=\overline{\Delta_M(\{s_M\},w)}$.  Since
  $\Delta_M(\{s_M\},w)\in X$ and $a\in\alphabet\,M$, we have that
  $\Delta_M(\{s_M\},wa) = \Delta_M(\Delta_M(\{s_M\},w),a)\in X$.  Thus
  \begin{alignat*}{2}
    \delta_N(s_N,wa) &= \delta_N(\delta_N(s_N,w),a) \\
    &= \delta_N(\overline{\Delta_M(\{s_M\},w)},a) &&
    \by{ind.\ hyp.} \\
    &= \overline{\Delta_M(\Delta_M(\{s_M\},w),a)} \\
    &= \overline{\Delta_M(\{s_M\},wa)} .
  \end{alignat*}
\end{description}
\end{proof}

\begin{lemma}
$L(N)=L(M)$.
\end{lemma}

\begin{proof}
\begin{description}
\item[\quad($L(M)\sub L(N)$)] Suppose $w\in L(M)$, so that
  $w\in(\alphabet\,M)^*=(\alphabet\,N)^*$ and $\Delta_M(\{s_M\},w)\cap
  A_M\neq\emptyset$.  By Lemma~\ref{NFAToDFAConvLemma}, we have that
  $\Delta_M(\{s_M\},w)\in X$ and
  $\delta_N(s_N,w)=\overline{\Delta_M(\{s_M\},w)}$.  Since
  $\Delta_M(\{s_M\},w)\in X$ and $\Delta_M(\{s_M\},w)\cap
  A_M\neq\emptyset$, it follows that
  $\delta_N(s_N,w)=\overline{\Delta_M(\{s_M\},w)} \in A_N$.  Thus
  $w\in L(N)$.

\item[\quad($L(N)\sub L(M)$)] Suppose $w\in L(N)$, so that
  $w\in(\alphabet\,N)^*=(\alphabet\,M)^*$ and $\delta_N(s_N,w)\in
  A_N$.  By Lemma~\ref{NFAToDFAConvLemma}, we have that
  $\delta_N(s_N,w)= \overline{\Delta_M(\{s_M\},w)}$.  Thus
  $\overline{\Delta_M(\{s_M\},w)}\in A_N$, so that
  $\Delta_M(\{s_M\},w)\cap A_M\neq\emptyset$.  Thus $w\in L(M)$.
\end{description}
\end{proof}

\index{deterministic finite automaton!nfaToDFA@$\nfaToDFA$}%
\index{nondeterministic finite automaton!nfaToDFA@$\nfaToDFA$}%
We define a function $\nfaToDFA\in\NFA\fun\DFA$ by: $\nfaToDFA\,M$ is
the result of running the preceding algorithm with input $M$.

\begin{theorem}
\label{NFAToDFATheorem}
For all $M\in\NFA$:
\begin{itemize}
\item $\nfaToDFA\,M\approx M$; and

\item $\alphabet(\nfaToDFA\,M)=\alphabet\,M$.
\end{itemize}
\end{theorem}

\subsection{Processing DFAs in Forlan}

The Forlan module \texttt{DFA} defines an abstract type \texttt{dfa}
\index{DFA@\texttt{DFA}}%
\index{DFA@\texttt{DFA}!dfa@\texttt{dfa}}%
(in the top-level environment) of deterministic finite automata,
along with various functions for processing DFAs.
Values of type \texttt{dfa} are implemented as values of type \texttt{fa}, and
the module DFA provides the following injection and projection functions
\begin{verbatim}
val injToFA     : dfa -> fa
val injToEFA    : dfa -> efa
val injToNFA    : dfa -> nfa
val projFromFA  : fa -> dfa
val projFromEFA : efa -> dfa
val projFromNFA : nfa -> dfa
\end{verbatim}
\index{DFA@\texttt{DFA}!injToFA@\texttt{injToFA}}%
\index{DFA@\texttt{DFA}!injToEFA@\texttt{injToEFA}}%
\index{DFA@\texttt{DFA}!injToNFA@\texttt{injToNFA}}%
\index{DFA@\texttt{DFA}!projFromFA@\texttt{projFromFA}}%
\index{DFA@\texttt{DFA}!projFromEFA@\texttt{projFromEFA}}%
\index{DFA@\texttt{DFA}!projFromNFA@\texttt{projFromNFA}}%
These functions are available in the top-level environment with the
names \texttt{injDFAToFA}, \texttt{injDFAToEFA}, \texttt{injDFAToNFA},
\texttt{projFAToDFA}, \texttt{projEFAToDFA} and \texttt{projNFAToDFA}.
\index{deterministic finite automaton!injDFAToFA@\texttt{injDFAToFA}}%
\index{deterministic finite automaton!injDFAToEFA@\texttt{injDFAToEFA}}%
\index{deterministic finite automaton!injDFAToNFA@\texttt{injDFAToNFA}}%
\index{deterministic finite automaton!projFAToDFA@\texttt{projFAToDFA}}%
\index{deterministic finite automaton!projEFAToDFA@\texttt{projEFAToDFA}}%
\index{deterministic finite automaton!projNFAToDFA@\texttt{projNFAToDFA}}%

The module \texttt{DFA} also defines the functions:
\begin{verbatim}
val input            : string -> dfa
val determProcStr    : dfa -> sym * str -> sym
val determAccepted   : dfa -> str -> bool
val determSimplified : dfa -> bool
val determSimplify   : dfa * sym set -> dfa
val fromNFA          : nfa -> dfa
\end{verbatim}
\index{DFA@\texttt{DFA}!input@\texttt{input}}%
\index{DFA@\texttt{DFA}!determProcStr@\texttt{determProcStr}}%
\index{DFA@\texttt{DFA}!determAccepted@\texttt{determAccepted}}%
\index{DFA@\texttt{DFA}!determSimplified@\texttt{determSimplified}}%
\index{DFA@\texttt{DFA}!determSimplify@\texttt{determSimplify}}%
\index{DFA@\texttt{DFA}!fromNFA@\texttt{fromNFA}}%
The function \texttt{input} is used to input a DFA.  The function
\texttt{determProcStr} is used to compute $\delta_M(q,w)$ for a DFA
$M$, using the properties of $\delta_M$.  The function
\texttt{determAccepted} uses \texttt{determProcStr} to check whether a
string is accepted by a DFA.  The function \texttt{determSimplified}
tests whether a DFA is deterministically simplified, and
the function \texttt{determSimplify}
corresponds to $\determSimplify$.  The function \texttt{fromNFA}
corresponds to our conversion function $\nfaToDFA$, and is available
in the top-level environment with that name:
\begin{verbatim}
val nfaToDFA : nfa -> dfa
\end{verbatim}
\index{DFA@\texttt{DFA}!nfaToDFA@\texttt{nfaToDFA}}%

Most of the functions for processing FAs that were introduced
in previous sections are inherited by \texttt{DFA}:
\begin{verbatim}
val output                  : string * dfa -> unit 
val numStates               : dfa -> int
val numTransitions          : dfa -> int
val alphabet                : dfa -> sym set
val equal                   : dfa * dfa -> bool
val checkLP                 : dfa -> lp -> unit
val validLP                 : dfa -> lp -> bool
val isomorphism             : dfa * dfa * sym_rel -> bool
val findIsomorphism         : dfa * dfa -> sym_rel
val isomorphic              : dfa * dfa -> bool
val renameStates            : dfa * sym_rel -> dfa
val renameStatesCanonically : dfa -> dfa
val processStr              : dfa -> sym set * str -> sym set
val accepted                : dfa -> str -> bool
val findLP                  : dfa -> sym set * str * sym set -> lp
val findAcceptingLP         : dfa -> str -> lp
\end{verbatim}
\index{DFA@\texttt{DFA}!output@\texttt{output}}%
\index{DFA@\texttt{DFA}!numStates@\texttt{numStates}}%
\index{DFA@\texttt{DFA}!numTransitions@\texttt{numTransitions}}%
\index{DFA@\texttt{DFA}!alphabet@\texttt{alphabet}}%
\index{DFA@\texttt{DFA}!equal@\texttt{equal}}%
\index{DFA@\texttt{DFA}!checkLP@\texttt{checkLP}}%
\index{DFA@\texttt{DFA}!validLP@\texttt{validLP}}%
\index{DFA@\texttt{DFA}!isomorphism@\texttt{isomorphism}}%
\index{DFA@\texttt{DFA}!findIsomorphism@\texttt{findIsomorphism}}%
\index{DFA@\texttt{DFA}!isomorphic@\texttt{isomorphic}}%
\index{DFA@\texttt{DFA}!renameStates@\texttt{renameStates}}%
\index{DFA@\texttt{DFA}!renameStatesCanonically@\texttt{renameStatesCanonically}}%
\index{DFA@\texttt{DFA}!processStr@\texttt{processStr}}%
\index{DFA@\texttt{DFA}!accepted@\texttt{accepted}}%
\index{DFA@\texttt{DFA}!findLP@\texttt{findLP}}%
\index{DFA@\texttt{DFA}!findAcceptingLP@\texttt{findAcceptingLP}}%

Suppose \texttt{dfa} is the DFA
\begin{center}
\input{chap-3.11-fig2.eepic}
\end{center}
We can turn \texttt{dfa} into an equivalent deterministically simplified
DFA whose alphabet is the union of the alphabet of the language
of \texttt{dfa} and $\{\twosf\}$, i.e., whose alphabet is
$\{\zerosf,\onesf,\twosf\}$, as follows:
\begin{list}{}
{\setlength{\leftmargin}{\leftmargini}
\setlength{\rightmargin}{0cm}
\setlength{\itemindent}{0cm}
\setlength{\listparindent}{0cm}
\setlength{\itemsep}{0cm}
\setlength{\parsep}{0cm}
\setlength{\labelsep}{0cm}
\setlength{\labelwidth}{0cm}
\catcode`\#=12
\catcode`\$=12
\catcode`\%=12
\catcode`\^=12
\catcode`\_=12
\catcode`\.=12
\catcode`\?=12
\catcode`\!=12
\catcode`\&=12
\ttfamily}
\small
\item[]\textsl{-\ }val\ dfa'\ =\ DFA.determSimplify(dfa,\ SymSet.input\ "");
\item[]\textsl{@\ }2
\item[]\textsl{@\ }.
\item[]\textsl{val\ dfa'\ =\ -\ :\ dfa}
\item[]\textsl{-\ }DFA.output("",\ dfa');
\item[]\textsl{\symbol{'173}states\symbol{'175}\ A,\ B,\ C,\ <dead>\ \symbol{'173}start\ state\symbol{'175}\ A}
\item[]\textsl{\symbol{'173}accepting\ states\symbol{'175}\ A,\ B,\ C}
\item[]\textsl{\symbol{'173}transitions\symbol{'175}}
\item[]\textsl{A,\ 0\ ->\ B;\ A,\ 1\ ->\ A;\ A,\ 2\ ->\ <dead>;\ B,\ 0\ ->\ C;\ B,\ 1\ ->\ A;}
\item[]\textsl{B,\ 2\ ->\ <dead>;\ C,\ 0\ ->\ <dead>;\ C,\ 1\ ->\ A;\ C,\ 2\ ->\ <dead>;}
\item[]\textsl{<dead>,\ 0\ ->\ <dead>;\ <dead>,\ 1\ ->\ <dead>;\ <dead>,\ 2\ ->\ <dead>}
\item[]\textsl{val\ it\ =\ ()\ :\ unit}
\end{list}

Thus \texttt{dfa'} is
\begin{center}
\input{chap-3.11-fig8.eepic}
\end{center}

Suppose that \texttt{nfa} is the NFA
\begin{center}
\input{chap-3.11-fig6.eepic}
\end{center}
We can convert \texttt{nfa} to a DFA as follows:
\begin{list}{}
{\setlength{\leftmargin}{\leftmargini}
\setlength{\rightmargin}{0cm}
\setlength{\itemindent}{0cm}
\setlength{\listparindent}{0cm}
\setlength{\itemsep}{0cm}
\setlength{\parsep}{0cm}
\setlength{\labelsep}{0cm}
\setlength{\labelwidth}{0cm}
\catcode`\#=12
\catcode`\$=12
\catcode`\%=12
\catcode`\^=12
\catcode`\_=12
\catcode`\.=12
\catcode`\?=12
\catcode`\!=12
\catcode`\&=12
\ttfamily}
\small
\item[]\textsl{-\ }val\ dfa\ =\ nfaToDFA\ nfa;
\item[]\textsl{val\ dfa\ =\ -\ :\ dfa}
\item[]\textsl{-\ }DFA.output("",\ dfa);
\item[]\textsl{\symbol{'173}states\symbol{'175}\ <>,\ <A>,\ <C>,\ <A,B>,\ <A,B,C>\ \symbol{'173}start\ state\symbol{'175}\ <A>}
\item[]\textsl{\symbol{'173}accepting\ states\symbol{'175}\ <C>,\ <A,B,C>}
\item[]\textsl{\symbol{'173}transitions\symbol{'175}}
\item[]\textsl{<>,\ 0\ ->\ <>;\ <>,\ 1\ ->\ <>;\ <A>,\ 0\ ->\ <>;\ <A>,\ 1\ ->\ <A,B>;}
\item[]\textsl{<C>,\ 0\ ->\ <C>;\ <C>,\ 1\ ->\ <>;\ <A,B>,\ 0\ ->\ <>;\ <A,B>,\ 1\ ->\ <A,B,C>;}
\item[]\textsl{<A,B,C>,\ 0\ ->\ <C>;\ <A,B,C>,\ 1\ ->\ <A,B,C>}
\item[]\textsl{val\ it\ =\ ()\ :\ unit}
\end{list}

Thus \texttt{dfa} is
\begin{center}
\input{chap-3.11-fig7.eepic}
\end{center}
And we can see why \texttt{nfa} and \texttt{dfa} accept
$\mathsf{111100}$, as follows:
\begin{list}{}
{\setlength{\leftmargin}{\leftmargini}
\setlength{\rightmargin}{0cm}
\setlength{\itemindent}{0cm}
\setlength{\listparindent}{0cm}
\setlength{\itemsep}{0cm}
\setlength{\parsep}{0cm}
\setlength{\labelsep}{0cm}
\setlength{\labelwidth}{0cm}
\catcode`\#=12
\catcode`\$=12
\catcode`\%=12
\catcode`\^=12
\catcode`\_=12
\catcode`\.=12
\catcode`\?=12
\catcode`\!=12
\catcode`\&=12
\ttfamily}
\small
\item[]\textsl{-\ }LP.output
\item[]\textsl{=\ }("",
\item[]\textsl{=\ }\ NFA.findAcceptingLP\ nfa\ (Str.fromString\ "111100"));
\item[]\textsl{A,\ 1\ =>\ A,\ 1\ =>\ A,\ 1\ =>\ B,\ 1\ =>\ C,\ 0\ =>\ C,\ 0\ =>\ C}
\item[]\textsl{val\ it\ =\ ()\ :\ unit}
\item[]\textsl{-\ }LP.output
\item[]\textsl{=\ }("",
\item[]\textsl{=\ }\ DFA.findAcceptingLP\ dfa\ (Str.fromString\ "111100"));
\item[]\textsl{<A>,\ 1\ =>\ <A,B>,\ 1\ =>\ <A,B,C>,\ 1\ =>\ <A,B,C>,\ 1\ =>\ <A,B,C>,\ 0\ =>}
\item[]\textsl{<C>,\ 0\ =>\ <C>}
\item[]\textsl{val\ it\ =\ ()\ :\ unit}
\end{list}


Finally, we see an example in which an NFA with $4$ states
is converted to a DFA with $2^4=16$ states; there is a state of the
DFA corresponding to every element of the power set of
the set of states of the NFA.  Suppose \texttt{nfa'} is
the NFA
\begin{center}
\input{chap-3.11-fig9.eepic}
\end{center}
Then we can convert \texttt{nfa'} into a DFA, as follows:
\begin{list}{}
{\setlength{\leftmargin}{\leftmargini}
\setlength{\rightmargin}{0cm}
\setlength{\itemindent}{0cm}
\setlength{\listparindent}{0cm}
\setlength{\itemsep}{0cm}
\setlength{\parsep}{0cm}
\setlength{\labelsep}{0cm}
\setlength{\labelwidth}{0cm}
\catcode`\#=12
\catcode`\$=12
\catcode`\%=12
\catcode`\^=12
\catcode`\_=12
\catcode`\.=12
\catcode`\?=12
\catcode`\!=12
\catcode`\&=12
\ttfamily}
\small
\item[]\textsl{-\ }val\ dfa'\ =\ nfaToDFA\ nfa';
\item[]\textsl{val\ dfa'\ =\ -\ :\ dfa}
\item[]\textsl{-\ }DFA.numStates\ dfa';
\item[]\textsl{val\ it\ =\ 16\ :\ int}
\item[]\textsl{-\ }DFA.output("",\ dfa');
\item[]\textsl{\symbol{'173}states\symbol{'175}}
\item[]\textsl{<>,\ <A>,\ <B>,\ <C>,\ <D>,\ <A,B>,\ <A,C>,\ <A,D>,\ <B,C>,\ <B,D>,\ <C,D>,}
\item[]\textsl{<A,B,C>,\ <A,B,D>,\ <A,C,D>,\ <B,C,D>,\ <A,B,C,D>}
\item[]\textsl{\symbol{'173}start\ state\symbol{'175}\ <A>}
\item[]\textsl{\symbol{'173}accepting\ states\symbol{'175}}
\item[]\textsl{<D>,\ <A,D>,\ <B,D>,\ <C,D>,\ <A,B,D>,\ <A,C,D>,\ <B,C,D>,\ <A,B,C,D>}
\item[]\textsl{\symbol{'173}transitions\symbol{'175}}
\item[]\textsl{<>,\ 0\ ->\ <>;\ <>,\ 1\ ->\ <>;\ <>,\ 2\ ->\ <>;\ <A>,\ 0\ ->\ <A>;}
\item[]\textsl{<A>,\ 1\ ->\ <A,B>;\ <A>,\ 2\ ->\ <>;\ <B>,\ 0\ ->\ <C>;\ <B>,\ 1\ ->\ <C>;}
\item[]\textsl{<B>,\ 2\ ->\ <B>;\ <C>,\ 0\ ->\ <D>;\ <C>,\ 1\ ->\ <D>;\ <C>,\ 2\ ->\ <C>;}
\item[]\textsl{<D>,\ 0\ ->\ <>;\ <D>,\ 1\ ->\ <>;\ <D>,\ 2\ ->\ <D>;\ <A,B>,\ 0\ ->\ <A,C>;}
\item[]\textsl{<A,B>,\ 1\ ->\ <A,B,C>;\ <A,B>,\ 2\ ->\ <B>;\ <A,C>,\ 0\ ->\ <A,D>;}
\item[]\textsl{<A,C>,\ 1\ ->\ <A,B,D>;\ <A,C>,\ 2\ ->\ <C>;\ <A,D>,\ 0\ ->\ <A>;}
\item[]\textsl{<A,D>,\ 1\ ->\ <A,B>;\ <A,D>,\ 2\ ->\ <D>;\ <B,C>,\ 0\ ->\ <C,D>;}
\item[]\textsl{<B,C>,\ 1\ ->\ <C,D>;\ <B,C>,\ 2\ ->\ <B,C>;\ <B,D>,\ 0\ ->\ <C>;}
\item[]\textsl{<B,D>,\ 1\ ->\ <C>;\ <B,D>,\ 2\ ->\ <B,D>;\ <C,D>,\ 0\ ->\ <D>;}
\item[]\textsl{<C,D>,\ 1\ ->\ <D>;\ <C,D>,\ 2\ ->\ <C,D>;\ <A,B,C>,\ 0\ ->\ <A,C,D>;}
\item[]\textsl{<A,B,C>,\ 1\ ->\ <A,B,C,D>;\ <A,B,C>,\ 2\ ->\ <B,C>;\ <A,B,D>,\ 0\ ->\ <A,C>;}
\item[]\textsl{<A,B,D>,\ 1\ ->\ <A,B,C>;\ <A,B,D>,\ 2\ ->\ <B,D>;\ <A,C,D>,\ 0\ ->\ <A,D>;}
\item[]\textsl{<A,C,D>,\ 1\ ->\ <A,B,D>;\ <A,C,D>,\ 2\ ->\ <C,D>;\ <B,C,D>,\ 0\ ->\ <C,D>;}
\item[]\textsl{<B,C,D>,\ 1\ ->\ <C,D>;\ <B,C,D>,\ 2\ ->\ <B,C,D>;}
\item[]\textsl{<A,B,C,D>,\ 0\ ->\ <A,C,D>;\ <A,B,C,D>,\ 1\ ->\ <A,B,C,D>;}
\item[]\textsl{<A,B,C,D>,\ 2\ ->\ <B,C,D>}
\item[]\textsl{val\ it\ =\ ()\ :\ unit}
\end{list}

In Section~\ref{EquivalenceTestingAndMinimizationOfDFAs},
we will use Forlan to show that there is no
DFA with fewer than $16$ states that accepts the language accepted
\index{deterministic finite automaton!exponential blowup}%
by \texttt{nfa'} and \texttt{dfa'}.

\subsection{Notes}

In contrast to the standard approach, the transition function $\delta$
for a DFA $M$ is not part of the definition of $M$, but is derived from
the definition.  Our approach to proving the correctness of DFAs,
using induction on $\Lambda$ plus proof by contradiction, is novel,
simple and elegant.  The material on deterministic simplification is
original, but straightforward.  And the algorithm for converting NFAs
to DFAs is standard.

\index{finite automaton!deterministic|)}%
\index{deterministic finite automaton|)}%
\index{DFA|)}%

%%% Local Variables: 
%%% mode: latex
%%% TeX-master: "book"
%%% End: 

\section{Closure Properties of Regular Languages}
\label{ClosurePropertiesOfRegularLanguages}

\index{regular languages!closure properties|(}%

In this section, we show how to convert regular expressions to finite
automata, as well as how to convert finite automata to regular
expressions.  As a result, we will be able to conclude that the
following statements about a language $L$ are equivalent:
\begin{itemize}
\item $L$ is regular;

\item $L$ is generated by a regular expression;

\item $L$ is accepted by a finite automaton;

\item $L$ is accepted by an EFA;

\item $L$ is accepted by an NFA; and

\item $L$ is accepted by a DFA.
\end{itemize}

Also, we will introduce:
\begin{itemize}
\item operations on FAs corresponding to union, concatenation and
  closure;

\item an operation on EFAs corresponding to intersection; and

\item an operation on DFAs corresponding to set difference.
\end{itemize}
As a result, we will have that the set $\RegLan$ of regular languages
is closed under union, concatenation, closure, intersection and set
difference.  I.e., we will have that, if $L,L_1,L_2\in\RegLan$, then
$L_1\cup L_2$, $L_1L_2$, $L^*$, $L_1\cap L_2$ and $L_1-L_2$ are in
$\RegLan$.

We will also show several additional closure properties of regular
languages, in addition to giving the corresponding operations on regular
expressions and automata.

\subsection{Converting Regular Expressions to FAs}

\index{regular expression!converting to FA}%
\index{finite automaton!converting from regular expression}%

In order to give an algorithm for converting regular expressions to
finite automata, we must first define several constants and operations
on FAs.

\index{finite automaton!emptyStr@$\emptyStr$}%
We write $\emptyStr$ for the \emph{canonical finite
automaton for} $\%$,
\begin{center}
\input{chap-3.12-fig1.eepic}
\end{center}
\index{finite automaton!emptySet@$\emptySet$}%
And we write $\emptySet$ for \emph{canonical finite automaton for} $\emptyset$,
\begin{center}
\input{chap-3.12-fig2.eepic}
\end{center}
Thus, we have that $L(\emptyStr)=\{\%\}$ and $L(\emptySet)=\emptyset$.
Furthermore both $\emptyStr$ and $\emptySet$ are DFAs, so that they
are also NFAs and EFAs.  Thus, we also refer to $\emptyStr$ as the
\emph{canonical DFA/NFA/EFA/FA for} $\%$, and $\emptySet$ as the
\emph{canonical DFA/NFA/EFA/FA for} $\emptyset$.

\index{finite automaton!strToFA@$\strToFA$}%
Next, we define a function $\strToFA\in\Str\fun\FA$ by:
$\strToFA\,x$ is the \emph{canonical finite automaton for} $x$,
\begin{center}
\input{chap-3.12-fig10.eepic}
\end{center}
Thus, for all $x\in\Str$, $L(\strToFA\,x)=\{x\}$.
\index{nondeterministic finite automaton!symToNFA@$\symToNFA$}%
It is also convenient to define a function $\symToNFA\in\Sym\fun\NFA$
by: $\symToNFA\,a=\strToFA\,a$.  Then, for all $a\in\Sym$,
$L(\symToNFA\,a)=\{a\}$.  Of course, $\symToNFA$ is also an element of
$\Sym\fun\EFA$ and $\Sym\fun\FA$, and we say that $\symToNFA\,a$ is
the \emph{canonical NFA/EFA/FA for} $a$.

\index{language!union}%
\index{finite automaton!union}%
\index{finite automaton!union@$\union$}%
\index{union!finite automaton}%
Next, we define a function/algorithm $\union\in\FA\times\FA\fun\FA$
such that $L(\union(M_1,M_2))=L(M_1)\cup L(M_2)$, for all
$M_1,M_2\in\FA$.  If $M_1,M_2\in\FA$, then $\union(M_1,M_2)$,
\emph{the union of} $M_1$ \emph{and} $M_2$, is the FA $N$ such that:
\begin{itemize}
\item $Q_N= \{\Asf\}\cup\setof{\langle\onesf,q\rangle}{q\in
    Q_{M_1}}\cup \setof{\langle\twosf,q\rangle}{q\in Q_{M_2}}$;

\item $s_N=\Asf$;

\item $A_N=\setof{\langle\onesf,q\rangle}{q\in A_{M_1}}\cup
  \setof{\langle\twosf,q\rangle}{q\in A_{M_2}}$; and

\item $T_N={}$

\begin{align*}
  &\quad
  \,\{\Asf,\%\fun \langle\onesf,s_{M_1}\rangle\} \\
  &\cup \{\Asf,\%\fun\langle\twosf,s_{M_2}\rangle\} \\
  &\cup \setof{\langle\onesf,q\rangle,a\fun\langle\onesf,r\rangle}{q,
    a\fun r\in T_{M_1}}
  \\
  &\cup \setof{\langle\twosf,q\rangle,a\fun\langle\twosf,r\rangle}{q,
    a\fun r \in T_{M_2}}.
\end{align*}
\end{itemize}

For example, if $M_1$ and $M_2$ are the FAs
\begin{center}
\input{chap-3.12-fig3.eepic}
\input{chap-3.12-fig4.eepic}
\end{center}
then $\union(M_1,M_2)$ is the FA
\begin{center}
\input{chap-3.12-fig5.eepic}
\end{center}

\begin{proposition}
For all $M_1,M_2\in\FA$:
\begin{itemize}
\item $L(\union(M_1,M_2))=L(M_1)\cup L(M_2)$; and

\item $\alphabet(\union(M_1,M_2))=\alphabet\,M_1\cup\alphabet\,M_2$.
\end{itemize}
\end{proposition}

\begin{proposition}
For all $M_1,M_2\in\EFA$, $\union(M_1,M_2)\in\EFA$.
\end{proposition}

\index{language!concatenation}%
\index{finite automaton!concatenation}%
\index{finite automaton!concat@$\concat$}%
\index{concatenation!finite automaton}%
Next, we define a function/algorithm $\concat\in\FA\times\FA\fun\FA$
such that $L(\concat(M_1,M_2))=L(M_1)L(M_2)$, for all
$M_1,M_2\in\FA$.  If $M_1,M_2\in\FA$, then $\concat(M_1,M_2)$, \emph{the
concatenation of} $M_1$ \emph{and} $M_2$, is the
FA $N$ such that:
\begin{itemize}
\item $Q_N=\setof{\langle\onesf,q\rangle}{q\in Q_{M_1}}\cup
\setof{\langle\twosf,q\rangle}{q\in Q_{M_2}}$;

\item $s_N=\langle\onesf,s_{M_1}\rangle$;

\item $A_N=\setof{\langle\twosf,q\rangle}{q\in
A_{M_2}}$; and

\item $T_N={}$
  \begin{align*}
    &\quad\,
    \setof{\langle\onesf,q\rangle,\%\fun\langle\twosf,s_{M_2}\rangle}{q\in
      A_{M_1}}
    \\
    &\cup
    \setof{\langle\onesf,q\rangle,a\fun\langle\onesf,r\rangle}{q,
      a\fun r\in T_{M_1}}
    \\
    &\cup
    \setof{\langle\twosf,q\rangle,a\fun\langle\twosf,r\rangle}{q,
      a\fun r\in T_{M_2}}.
  \end{align*}
\end{itemize}

For example, if $M_1$ and $M_2$ are the FAs
\begin{center}
\input{chap-3.12-fig11.eepic}
\input{chap-3.12-fig12.eepic}
\end{center}
then $\concat(M_1,M_2)$ is the FA
\begin{center}
\input{chap-3.12-fig6.eepic}
\end{center}

\begin{proposition}
For all $M_1,M_2\in\FA$:
\begin{itemize}
\item $L(\concat(M_1,M_2))=L(M_1)L(M_2)$; and

\item $\alphabet(\concat(M_1,M_2))=\alphabet\,M_1\cup\alphabet\,M_2$.
\end{itemize}
\end{proposition}

\begin{proposition}
For all $M_1,M_2\in\EFA$, $\concat(M_1,M_2)\in\EFA$.
\end{proposition}

\index{language!closure}%
\index{finite automaton!closure}%
\index{finite automaton!closure@$\closure$}%
\index{closure!finite automaton}%
Next, we define a function/algorithm $\closure\in\FA\fun\FA$ such that
$L(\closure\,M)=L(M)^*$, for all $M\in\FA$.  If $M\in\FA$, then
$\closure\,M$, \emph{the closure of} $M$, is the FA $N$ such that:
\begin{itemize}
\item $Q_N=\{\Asf\}\cup\setof{\langle q\rangle}{q\in Q_M}$;

\item $s_N=\Asf$;

\item $A_N=\{\Asf\}$; and

\item $T_N={}$
  \begin{align*}
    &\quad\, \{\Asf,\%\fun\langle s_M\rangle\} \\
    &\cup \setof{\langle q\rangle,\%\fun\Asf}{q\in A_M} \\
    &\cup \setof{\langle q\rangle,a\fun\langle r\rangle}{q,a\fun r\in
      T_M}.
  \end{align*}
\end{itemize}

For example, if $M$ is the FA
\begin{center}
\input{chap-3.12-fig7.eepic}
\end{center}
then $\closure\,M$ is the FA
\begin{center}
\input{chap-3.12-fig8.eepic}
\end{center}

\begin{proposition}
For all $M\in\FA$,
\begin{itemize}
\item $L(\closure\,M)=L(M)^*$; and

\item $\alphabet(\closure\,M)=\alphabet\,M$.
\end{itemize}
\end{proposition}

\begin{proposition}
For all $M\in\EFA$, $\closure\,M\in\EFA$.
\end{proposition}

\index{regular expression!regToFA@$\regToFA$}%
\index{finite automaton!regToFA@$\regToFA$}%
We define a function/algorithm $\regToFA\in\Reg\fun\FA$ by well-founded
recursion on the height of regular expressions, as follows.
The goal is for $L(\regToFA\,\alpha)$ to be equal to $L(\alpha)$, for
all regular expressions $\alpha$.
\begin{itemize}
\item $\regToFA\,\% = \emptyStr$;

\item $\regToFA\,\$ = \emptySet$;

\item for all $\alpha\in\Reg$, $\regToFA(\alpha^*) =
  \closure(\regToFA\,\alpha)$;

\item for all $\alpha,\beta\in\Reg$, $\regToFA(\alpha+\beta) =
  \union(\regToFA\,\alpha,\regToFA\,\beta)$;

\item for all $n\in\nats-\{0\}$ and $a_1,\,\ldots,a_n\in\Sym$,
  $\regToFA(a_1\cdots a_n)=\strToFA(a_1\cdots a_n)$;

\item for all $n\in\nats-\{0\}$, $a_1,\,\ldots,a_n\in\Sym$ and
  $\alpha\in\Reg$, if $\alpha$ doesn't consist of a single symbol, and
  doesn't have the form $b\,\beta$ for some $b\in\Sym$ and
  $\beta\in\Reg$, then $\regToFA(a_1\,\cdots\,a_n\,\alpha)=
  \concat(\strToFA(a_1\cdots a_n),\regToFA\,\alpha)$; and

\item for all $\alpha,\beta\in\Reg$, if $\alpha$ doesn't consist of a
  single symbol, then $\regToFA(\alpha\beta)=
  \concat(\regToFA\,\alpha,\regToFA\,\beta)$.
\end{itemize}

For example, $\regToFA(\mathsf{0101^*})=
\concat(\strToFA(010),\regToFA(1^*))$.

\begin{theorem}
For all $\alpha\in\Reg$:
\begin{itemize}
\item $L(\regToFA\,\alpha)=L(\alpha)$; and

\item $\alphabet(\regToFA\,\alpha)=\alphabet\,\alpha$.
\end{itemize}
\end{theorem}

\begin{proof}
Because of the form of recursion used, the proof uses well-founded
induction on the height of $\alpha$.
\end{proof}

For example, $\regToFA(\mathsf{0^*11+001^*})$ is isomorphic to
the FA
\begin{center}
\input{chap-3.12-fig9.eepic}
\end{center}

The Forlan module \texttt{FA} includes these constants and functions
for building finite automata and converting regular expressions to
finite automata:
\begin{verbatim}
val emptyStr : fa
val emptySet : fa
val fromStr  : str -> fa
val fromSym  : sym -> fa
val union    : fa * fa -> fa
val concat   : fa * fa -> fa
val closure  : fa -> fa
val fromReg  : reg -> fa
\end{verbatim}
\index{FA@\texttt{FA}!emptyStr@\texttt{emptyStr}}%
\index{FA@\texttt{FA}!emptySet@\texttt{emptySet}}%
\index{FA@\texttt{FA}!fromStr@\texttt{fromStr}}%
\index{FA@\texttt{FA}!fromSym@\texttt{fromSym}}%
\index{FA@\texttt{FA}!union@\texttt{union}}%
\index{FA@\texttt{FA}!concat@\texttt{concat}}%
\index{FA@\texttt{FA}!closure@\texttt{closure}}%
\index{FA@\texttt{FA}!fromReg@\texttt{fromReg}}%
\texttt{emptyStr} and \texttt{emptySet} correspond to $\emptyStr$ and
$\emptySet$, respectively.
The functions \texttt{fromStr} and \texttt{fromSym} correspond to
$\strToFA$ and $\symToNFA$, and are also available in the top-level
environment with the names
\begin{verbatim}
val strToFA : str -> fa
val symToFA : sym -> fa
\end{verbatim}
\index{finite automaton!strToFA@\texttt{strToFA}}%
\index{finite automaton!symToFA@\texttt{symToFA}}%
\texttt{union} and \texttt{concat} and \texttt{closure} correspond to
$\union$, $\concat$ and $\closure$, respectively.
The function \texttt{fromReg} corresponds to $\regToFA$ and is
available in the top-level environment with that name:
\begin{verbatim}
val regToFA : reg -> fa
\end{verbatim}
\index{finite automaton!regToFA@\texttt{regToFA}}%
The constants \texttt{emptyStr} and \texttt{emptySet} are inherited by
the modules \texttt{DFA}, \texttt{NFA} and \texttt{EFA}.
\index{DFA@\texttt{DFA}!emptyStr@\texttt{emptyStr}}%
\index{DFA@\texttt{DFA}!emptySet@\texttt{emptySet}}%
\index{NFA@\texttt{NFA}!emptyStr@\texttt{emptyStr}}%
\index{NFA@\texttt{NFA}!emptySet@\texttt{emptySet}}%
\index{EFA@\texttt{EFA}!emptyStr@\texttt{emptyStr}}%
\index{EFA@\texttt{EFA}!emptySet@\texttt{emptySet}}%
The function \texttt{fromSym} is inherited by
the modules \texttt{NFA} and \texttt{EFA}, and is available in
\index{NFA@\texttt{NFA}!fromSym@\texttt{fromSym}}%
\index{EFA@\texttt{EFA}!fromSym@\texttt{fromSym}}%
the top-level environment with the names
\begin{verbatim}
val symToNFA : sym -> nfa
val symToEFA : sym -> efa
\end{verbatim}
\index{nondeterministic finite automaton!symToNFA@\texttt{symToNFA}}%
\index{empty-string finite automaton!symToEFA@\texttt{symToEFA}}%
The functions \texttt{union}, \texttt{concat} and \texttt{closure} are
inherited by the module \texttt{EFA}.
\index{EFA@\texttt{EFA}!union@\texttt{union}}%
\index{EFA@\texttt{EFA}!concat@\texttt{concat}}%
\index{EFA@\texttt{EFA}!closure@\texttt{closure}}%

Here is how the regular expression $\mathsf{0^*11+001^*}$ can
be converted to an FA in Forlan:
\begin{list}{}
{\setlength{\leftmargin}{\leftmargini}
\setlength{\rightmargin}{0cm}
\setlength{\itemindent}{0cm}
\setlength{\listparindent}{0cm}
\setlength{\itemsep}{0cm}
\setlength{\parsep}{0cm}
\setlength{\labelsep}{0cm}
\setlength{\labelwidth}{0cm}
\catcode`\#=12
\catcode`\$=12
\catcode`\%=12
\catcode`\^=12
\catcode`\_=12
\catcode`\.=12
\catcode`\?=12
\catcode`\!=12
\catcode`\&=12
\ttfamily}
\small
\item[]\textsl{-\ }val\ reg\ =\ Reg.input\ "";
\item[]\textsl{@\ }0\symbol{'052}11\ +\ 001\symbol{'052}
\item[]\textsl{@\ }.
\item[]\textsl{val\ reg\ =\ -\ :\ reg}
\item[]\textsl{-\ }val\ fa\ =\ regToFA\ reg;
\item[]\textsl{val\ fa\ =\ -\ :\ fa}
\item[]\textsl{-\ }FA.output("",\ fa);
\item[]\textsl{\symbol{'173}states\symbol{'175}}
\item[]\textsl{A,\ <1,<1,A>>,\ <1,<2,A>>,\ <1,<2,B>>,\ <2,<1,A>>,\ <2,<1,B>>,}
\item[]\textsl{<2,<2,A>>,\ <1,<1,<A>>>,\ <1,<1,<B>>>,\ <2,<2,<A>>>,\ <2,<2,<B>>>}
\item[]\textsl{\symbol{'173}start\ state\symbol{'175}\ A\ \symbol{'173}accepting\ states\symbol{'175}\ <1,<2,B>>,\ <2,<2,A>>}
\item[]\textsl{\symbol{'173}transitions\symbol{'175}}
\item[]\textsl{A,\ %\ ->\ <1,<1,A>>\ |\ <2,<1,A>>;}
\item[]\textsl{<1,<1,A>>,\ %\ ->\ <1,<2,A>>\ |\ <1,<1,<A>>>;}
\item[]\textsl{<1,<2,A>>,\ 11\ ->\ <1,<2,B>>;\ <2,<1,A>>,\ 00\ ->\ <2,<1,B>>;}
\item[]\textsl{<2,<1,B>>,\ %\ ->\ <2,<2,A>>;\ <2,<2,A>>,\ %\ ->\ <2,<2,<A>>>;}
\item[]\textsl{<1,<1,<A>>>,\ 0\ ->\ <1,<1,<B>>>;\ <1,<1,<B>>>,\ %\ ->\ <1,<1,A>>;}
\item[]\textsl{<2,<2,<A>>>,\ 1\ ->\ <2,<2,<B>>>;\ <2,<2,<B>>>,\ %\ ->\ <2,<2,A>>}
\item[]\textsl{val\ it\ =\ ()\ :\ unit}
\item[]\textsl{-\ }val\ fa'\ =\ FA.renameStatesCanonically\ fa;\ 
\item[]\textsl{val\ fa'\ =\ -\ :\ fa}
\item[]\textsl{-\ }FA.output("",\ fa');
\item[]\textsl{\symbol{'173}states\symbol{'175}\ A,\ B,\ C,\ D,\ E,\ F,\ G,\ H,\ I,\ J,\ K\ \symbol{'173}start\ state\symbol{'175}\ A}
\item[]\textsl{\symbol{'173}accepting\ states\symbol{'175}\ D,\ G}
\item[]\textsl{\symbol{'173}transitions\symbol{'175}}
\item[]\textsl{A,\ %\ ->\ B\ |\ E;\ B,\ %\ ->\ C\ |\ H;\ C,\ 11\ ->\ D;\ E,\ 00\ ->\ F;\ F,\ %\ ->\ G;}
\item[]\textsl{G,\ %\ ->\ J;\ H,\ 0\ ->\ I;\ I,\ %\ ->\ B;\ J,\ 1\ ->\ K;\ K,\ %\ ->\ G}
\item[]\textsl{val\ it\ =\ ()\ :\ unit}
\end{list}

Thus \texttt{fa'} is the finite automaton
\begin{center}
\input{chap-3.12-fig9.eepic}
\end{center}

Putting together our algorithm for converting regular expressions to finite
automata with our algorithm for checking whether strings are accepted by
finite automata, we are now able to check whether strings are
generated by regular expressions:
\begin{list}{}
{\setlength{\leftmargin}{\leftmargini}
\setlength{\rightmargin}{0cm}
\setlength{\itemindent}{0cm}
\setlength{\listparindent}{0cm}
\setlength{\itemsep}{0cm}
\setlength{\parsep}{0cm}
\setlength{\labelsep}{0cm}
\setlength{\labelwidth}{0cm}
\catcode`\#=12
\catcode`\$=12
\catcode`\%=12
\catcode`\^=12
\catcode`\_=12
\catcode`\.=12
\catcode`\?=12
\catcode`\!=12
\catcode`\&=12
\ttfamily}
\small
\item[]\textsl{-\ }fun\ generated\ reg\ =
\item[]\textsl{=\ }\ \ \ \ \ \ let\ val\ fa\ =\ FA.renameStatesCanonically(regToFA\ reg)
\item[]\textsl{=\ }\ \ \ \ \ \ in\ FA.accepted\ fa\ end;
\item[]\textsl{val\ generated\ =\ fn\ :\ reg\ ->\ str\ ->\ bool}
\item[]\textsl{-\ }val\ generated\ =\ generated\ reg;
\item[]\textsl{val\ generated\ =\ fn\ :\ str\ ->\ bool}
\item[]\textsl{-\ }generated(Str.fromString\ "000011");
\item[]\textsl{val\ it\ =\ true\ :\ bool}
\item[]\textsl{-\ }generated(Str.fromString\ "001111");
\item[]\textsl{val\ it\ =\ true\ :\ bool}
\item[]\textsl{-\ }generated(Str.fromString\ "000111");
\item[]\textsl{val\ it\ =\ false\ :\ bool}
\end{list}


\subsection{Converting FAs to Regular Expressions}

\index{regular expression!converting from FA}%
\index{finite automaton!converting to regular expression}%

Our algorithm for converting FAs to regular expressions makes
use of a more general kind of finite automata that we call
regular expression finite automata.

\index{regular expression finite automaton}%
\index{finite automaton!regular expression}%
\index{RFA}%
A \emph{regular expression finite automaton} (RFA) $M$ consists of:
\begin{itemize}
\item a finite set $Q_M$ of symbols;

\item an element $s_M$ of $Q_M$;

\item a subset $A_M$ of $Q_M$; and

\item a finite subset $T_M$ of
$\setof{(q,{\alpha},r)}{q,r\in Q_M\eqtxt{and}
{\alpha}\in{\Reg}}$ such that,
for all $q,r\in Q_M$, there is at most one $\alpha\in\Reg$
such that $(q,\alpha,r)\in T_M$.
\end{itemize}
As usual $Q_M$ consists of $M$'s \emph{states}, $s_M$ is $M$'s
\emph{start state}, $A_M$ consists of $M$'s \emph{accepting states},
and $T_M$ consists of $M$'s \emph{transitions}.  We often write
a transition $(q,\alpha,r)$ as
\begin{gather*}
q\tranarr{\alpha}r
\end{gather*}
or $q,\alpha\fun r$.
We write $\RFA$ for the set of all RFAs, which is a countably infinite
set.  RFAs are drawn analogously to FAs, and the Forlan syntax for
RFAs is analogous to that of FAs.

For example, the RFA $M$ whose states are $\Asf$ and $\Bsf$, start
state is $\Asf$, only accepting state is $\Bsf$, and
transitions are $(\Asf,\twosf, \Asf)$, $(\Asf,\zerosf\zerosf^*,\Bsf)$,
$(\Bsf,\threesf,\Bsf)$ and $(\Bsf,\onesf\onesf^*,\Asf)$ can
be drawn as
\begin{center}
\input{chap-3.12-fig13.eepic}  
\end{center}
and expressed in Forlan as
\begin{verbatim}
{states} A, B {start state} A {accepting states} B
{transitions} A, 2 -> A; A, 00* -> B; B, 3 -> B; B, 11* -> A
\end{verbatim}

\index{regular expression finite automaton!alphabet}%
\index{regular expression finite automaton!alphabet@$\alphabet$}
We define a function $\alphabet\in\RFA\fun\Alp$ by: for all
$M\in\RFA$, $\alphabet\,M$ is $\setof{a\in\Sym}{\eqtxt{there
    are}q,\alpha,r\eqtxt{such that} q,\alpha\fun r\in
  T_M\eqtxt{and}a\in\alphabet\,\alpha}$.  I.e., $\alphabet\,M$ is the
union of the alphabets of all of the regular expressions appearing in
$M$'s transitions.  We say that $\alphabet\,M$ is \emph{the alphabet
  of} $M$.  For example, the alphabet of our example FA $M$ is
$\{\mathsf{0,1,2}\}$.

The Forlan module \texttt{RFA} defines an abstract type \texttt{rfa}
\index{RFA@\texttt{RFA}}%
\index{RFA@\texttt{RFA}!rfa@\texttt{rfa}}%
(in the top-level environment) of regular expression finite automata,
as well as some functions for processing RFAs including:
\begin{verbatim}
val input          : string -> rfa
val output         : string * rfa -> unit 
val alphabet       : rfa -> sym set
val numStates      : rfa -> int
val numTransitions : rfa -> int
val equal          : rfa * rfa -> bool
\end{verbatim}
\index{RFA@\texttt{RFA}!input@\texttt{input}}%
\index{RFA@\texttt{RFA}!output@\texttt{output}}%
\index{RFA@\texttt{RFA}!alphabet@\texttt{alphabet}}%
\index{RFA@\texttt{RFA}!numStates@\texttt{numStates}}%
\index{RFA@\texttt{RFA}!numTransitions@\texttt{numTransitions}}%
\index{RFA@\texttt{RFA}!equal@\texttt{equal}}%
\index{JForlan}
JForlan can be used to view and edit regular expression finite
automata.  It can be invoked directly, or run via Forlan.  See the
Forlan website for more information.

The isomorphism relation between RFAs is defined in an analogous way
to this relation for FAs.  And the functions $\renameStates$
and $\renameStatesCanonically$ are also defined analogously, and
have analogous properties. The \texttt{RFA} module has the functions
\begin{verbatim}
val renameStates            : rfa * sym_rel -> rfa
val renameStatesCanonically : rfa -> rfa
\end{verbatim}
\index{RFA@\texttt{RFA}!renameStates@\texttt{renameStates}}%
\index{RFA@\texttt{RFA}!renameStatesCanonically@\texttt{renameStatesCanonically}}%

\index{regular expression finite automaton!labeled path}%
A labeled path
\begin{gather*}
q_1\lparr{x_1}q_2\lparr{x_2}\cdots\,q_n\lparr{x_n}q_{n+1} ,
\end{gather*}
is \emph{valid for} an RFA $M$ iff, for all $i\in[1:n]$,
\begin{gather*}
x_i\in L(\alpha),\eqtxt{for some}\alpha\in\Reg\eqtxt{such that}
q_i,\alpha\fun q_{i+1} ,
\end{gather*}
and $q_{n+1}\in Q_M$.
For example, the labeled path
\begin{gather*}
\Asf\lparr{\mathsf{000}}\Bsf\lparr{3}\Bsf
\end{gather*}
is valid for our example FA $M$, because
\begin{itemize}
\item $\mathsf{000}\in L(\mathsf{00^*})$ and
  $\Asf,\mathsf{00^*}\fun\Bsf\in T$, and

\item $\mathsf{3}\in L(\mathsf{3})$ and $\Bsf,\mathsf{3}\fun\Bsf\in
  T$.
\end{itemize}

The \texttt{RFA} module contains the functions
\begin{verbatim}
val checkLP : (str * reg -> bool) * rfa -> lp -> unit
val validLP : (str * reg -> bool) * rfa -> lp -> bool
\end{verbatim}
\index{RFA@\texttt{RFA}!checkLP@\texttt{checkLP}}%
\index{RFA@\texttt{RFA}!validLP@\texttt{validLP}}%
which are analogous to the identically named functions provided by
\texttt{FA}, except that they take a first argument whose job is
to test whether a string is generated by a regular expression.
For example, we can proceed as follows:
\begin{list}{}
{\setlength{\leftmargin}{\leftmargini}
\setlength{\rightmargin}{0cm}
\setlength{\itemindent}{0cm}
\setlength{\listparindent}{0cm}
\setlength{\itemsep}{0cm}
\setlength{\parsep}{0cm}
\setlength{\labelsep}{0cm}
\setlength{\labelwidth}{0cm}
\catcode`\#=12
\catcode`\$=12
\catcode`\%=12
\catcode`\^=12
\catcode`\_=12
\catcode`\.=12
\catcode`\?=12
\catcode`\!=12
\catcode`\&=12
\ttfamily}
\small
\item[]\textsl{-\ }val\ rfa\ =\ RFA.input\ "";
\item[]\textsl{@\ }\symbol{'173}states\symbol{'175}\ A,\ B\ \symbol{'173}start\ state\symbol{'175}\ A\ \symbol{'173}accepting\ states\symbol{'175}\ B
\item[]\textsl{@\ }\symbol{'173}transitions\symbol{'175}\ A,\ 2\ ->\ A;\ A,\ 00\symbol{'052}\ ->\ B;\ B,\ 3\ ->\ B;\ B,\ 11\symbol{'052}\ ->\ A
\item[]\textsl{@\ }.
\item[]\textsl{val\ rfa\ =\ -\ :\ rfa}
\item[]\textsl{-\ }fun\ memb(x,\ reg)\ =\ FA.accepted\ (regToFA\ reg)\ x;
\item[]\textsl{val\ memb\ =\ fn\ :\ str\ \symbol{'052}\ reg\ ->\ bool}
\item[]\textsl{-\ }val\ lp\ =\ LP.input\ "";
\item[]\textsl{@\ }A,\ 000\ =>\ B,\ 3\ =>\ B
\item[]\textsl{@\ }.
\item[]\textsl{val\ lp\ =\ -\ :\ lp}
\item[]\textsl{-\ }RFA.validLP\ (memb,\ rfa)\ lp;
\item[]\textsl{val\ it\ =\ true\ :\ bool}
\item[]\textsl{-\ }val\ lp'\ =\ LP.input\ "";
\item[]\textsl{@\ }A,\ 0\ =>\ B,\ 0\ =>\ A
\item[]\textsl{@\ }.
\item[]\textsl{val\ lp'\ =\ -\ :\ lp}
\item[]\textsl{-\ }RFA.checkLP\ (memb,\ rfa)\ lp';
\item[]\textsl{transition\ from\ "B"\ to\ "A"\ has\ regular\ expression\ that\ doesn't}
\item[]\textsl{generate\ "0"}
\item[]
\item[]\textsl{uncaught\ exception\ Error}
\end{list}


\index{regular expression finite automaton!accepted by}%
A string $w$ is \emph{accepted by} an RFA $M$ iff
there is a labeled path $\lp$ such that
\begin{itemize}
\item the label of $\lp$ is $w$;

\item $\lp$ is valid for $M$;

\item the start state of $\lp$ is the start state of $M$; and

\item the end state of $\lp$ is an accepting state of $M$.
\end{itemize}
We have that, if $w$ is accepted by $M$, then
$\alphabet\,w\sub\alphabet\,M$.
\index{regular expression finite automaton!meaning}%
\index{regular expression finite automaton!language accepted by}%
\index{L(@$L(\cdot)$}%
\index{regular expression finite automaton!L(@$L(\cdot)$}%}%
The \emph{language accepted by} an RFA $M$ ($L(M)$) is
\begin{gather*}
\setof{w\in\Str}{w\eqtxt{is accepted by}M}.
\end{gather*}

Consider our example RFA $M$:
\begin{center}
\input{chap-3.12-fig13.eepic}
\end{center}
We have that $\mathsf{20}$ and $\mathsf{0000111103}$ are
accepted by $M$, but that $\mathsf{23}$ and $\mathsf{122}$ are
not accepted by $M$.

We define a function $\combineTrans$ that takes in a pair $(\Simp, U)$
such that
\begin{itemize}
\item $\Simp\in\Reg\fun\Reg$ and

\item $U$ is a finite subset of
  $\setof{p,\alpha\fun q}{p,q\in\Sym\eqtxt{and}\alpha\in\Reg}$,
\end{itemize}
and returns a finite subset $V$ of $\setof{p,\alpha\fun
  q}{p,q\in\Sym\eqtxt{and}\alpha\in\Reg}$ with the property that, for
all $p,q\in\Sym$, there is at most one $\beta$ such that $p,\beta\fun
q\in V$.
Given such a pair $(\Simp, U)$, $\combineTrans$ returns the set of all
transitions $p,\alpha\fun q$ such that $\setof{\beta}{p,\beta\fun q\in U}$
is nonempty, and $\alpha = \Simp(\beta_1+\cdots+\beta_n)$,
where $\beta_1,\ldots,\beta_n$ are all of the elements of this set,
listed in increasing order and without repetition.

Now, we define a function/algorithm
\index{regular expression!faToRFA@$\faToRFA$}%
\index{regular expression finite automaton!faToRFA@$\faToRFA$}%
\begin{gather*}
\faToRFA\in(\Reg\fun\Reg)\fun\FA\fun\RFA .  
\end{gather*}
$\faToRFA$ takes in $\Simp\in\Reg\fun\Reg$, and returns a
function that takes in $M\in\FA$, and returns the RFA
$N$ such that:
\begin{itemize}
\item $Q_N = Q_M$;

\item $s_N = s_M$;

\item $A_N = A_M$; and

\item $T_N = \combineTrans(\Simp,
  \setof{p, \strToReg\,x\fun q}{p,x\fun q\in T_M})$.
\end{itemize}

For example, if the FA $M$ is
\begin{center}
  \input{chap-3.12-fig25.eepic}
\end{center}
and $\Simp$ is $\locallySimplify\,\obviousSubset$,
then $\faToRFA\,\Simp\,M$ is the RFA
\begin{center}
  \input{chap-3.12-fig26.eepic}
\end{center}

\begin{proposition}
Suppose $\Simp\in\Reg\fun\Reg$ and $M\in\FA$.  If, for all
$\alpha\in\Reg$, $L(\Simp\,\alpha)=L(\alpha)$ and
$\alphabet(\Simp\,\alpha)\sub\alphabet\,\alpha$, then
\begin{enumerate}[\quad(1)]
\item $L(\faToRFA\,\Simp\,M) = L(M)$, and

\item $\alphabet(\faToRFA\,\Simp\,M) = \alphabet\,M$.
\end{enumerate}
\end{proposition}

The \texttt{RFA} module has a function
\begin{verbatim}
val fromFA : (reg -> reg) -> fa -> rfa
\end{verbatim}
\index{RFA@\texttt{RFA}!fromFA@\texttt{fromFA}}%
that corresponds to $\faToRFA$.
Here is how our conversion example can be carried out in Forlan:
\begin{list}{}
{\setlength{\leftmargin}{\leftmargini}
\setlength{\rightmargin}{0cm}
\setlength{\itemindent}{0cm}
\setlength{\listparindent}{0cm}
\setlength{\itemsep}{0cm}
\setlength{\parsep}{0cm}
\setlength{\labelsep}{0cm}
\setlength{\labelwidth}{0cm}
\catcode`\#=12
\catcode`\$=12
\catcode`\%=12
\catcode`\^=12
\catcode`\_=12
\catcode`\.=12
\catcode`\?=12
\catcode`\!=12
\catcode`\&=12
\ttfamily}
\small
\item[]\textsl{-\ }val\ simp\ =\ #2\ o\ Reg.locallySimplify(NONE,\ Reg.obviousSubset);
\item[]\textsl{val\ simp\ =\ fn\ :\ reg\ ->\ reg}
\item[]\textsl{-\ }val\ fa\ =\ FA.input\ "";
\item[]\textsl{@\ }\symbol{'173}states\symbol{'175}\ A,\ B\ \symbol{'173}start\ state\symbol{'175}\ A\ \symbol{'173}accepting\ states\symbol{'175}\ B
\item[]\textsl{@\ }\symbol{'173}transitions\symbol{'175}
\item[]\textsl{@\ }A,\ 0\ ->\ A;\ A,\ 1\ ->\ B;\ A,\ 2\ ->\ B;
\item[]\textsl{@\ }B,\ 3\ ->\ B;\ B,\ 34\ ->\ B
\item[]\textsl{@\ }.
\item[]\textsl{val\ fa\ =\ -\ :\ fa}
\item[]\textsl{-\ }val\ rfa\ =\ RFA.fromFA\ simp\ fa;
\item[]\textsl{val\ rfa\ =\ -\ :\ rfa}
\item[]\textsl{-\ }RFA.output("",\ rfa);
\item[]\textsl{\symbol{'173}states\symbol{'175}\ A,\ B\ \symbol{'173}start\ state\symbol{'175}\ A\ \symbol{'173}accepting\ states\symbol{'175}\ B}
\item[]\textsl{\symbol{'173}transitions\symbol{'175}\ A,\ 0\ ->\ A;\ A,\ 1\ +\ 2\ ->\ B;\ B,\ 3(%\ +\ 4)\ ->\ B}
\item[]\textsl{val\ it\ =\ ()\ :\ unit}
\end{list}


We say that an RFA $M$ is \emph{standard} iff
\begin{itemize}
\item $M$'s start state is not an accepting state, and there are no
  transitions \emph{into} $M$'s start state (even from $s_M$ to
  itself); and

\item $M$ has a single accepting state, and there are no transitions
  \emph{from} that state (even from the accepting state to itself).
\end{itemize}

\index{regular expression finite automaton!standard}%
\begin{proposition}
Suppose $M$ is a standard RFA with only two states,
and that $q$ is $M's$ accepting state.
\begin{enumerate}[\quad(1)]
\item For all $\alpha\in\Reg$, if $s_M,\alpha\fun q$, then
  $L(M) = L(\alpha)$.

\item If there is no $\alpha\in\Reg$ such that $s_M,\alpha\fun q$,
  then $L(M) = \emptyset$.
\end{enumerate}
\end{proposition}

\index{regular expression finite automaton!standardize@$\standardize$}%
We define a function $\standardize\in\RFA\fun\RFA$ that standardizes
an RFA, as follows.  Given an argument $M$, it returns the RFA $N$
such that:
\begin{itemize}
\item $Q_N = \setof{\langle q\rangle}{q\in Q_M}\cup\{\Asf,\Bsf\}$;

\item $s_N = \Asf$;

\item $A_N = \{\Bsf\}$; and

\item $T_N$
  \begin{align*}
    &= \{\Asf,\%\fun\langle s_M\rangle\} \\
    &\cup \;\setof{\langle q\rangle,\%\fun\Bsf}{q\in A_M} \\
    &\cup \;\setof{\langle q\rangle,\alpha\fun\langle r\rangle}%
            {q,\alpha\fun r\in T_M}.
  \end{align*}
\end{itemize}

For example, if $M$ is the RFA
\begin{center}
  \input{chap-3.12-fig26.eepic}
\end{center}
then $\standardize\,M$ is the RFA
\begin{center}
  \input{chap-3.12-fig32.eepic}
\end{center}

\begin{proposition}
Suppose $M$ is an RFA.  Then:
\begin{itemize}
\item $\standardize\,M$ is standard;

\item $L(\standardize\,M) = L(M)$; and

\item $\alphabet(\standardize\,M) = \alphabet\,M$.
\end{itemize}
\end{proposition}

The \texttt{RFA} module has functions
\begin{verbatim}
val standard    : rfa -> bool
val standardize : rfa -> rfa
\end{verbatim}
\index{RFA@\texttt{RFA}!standard@\texttt{standard}}%
\index{RFA@\texttt{RFA}!standardize@\texttt{standardize}}%
The function \texttt{standard} tests whether an RFA is standard,
and the function \texttt{standardize} corresponds to
$\standardize$.

Here is how the above example can be carried out in Forlan:
\begin{list}{}
{\setlength{\leftmargin}{\leftmargini}
\setlength{\rightmargin}{0cm}
\setlength{\itemindent}{0cm}
\setlength{\listparindent}{0cm}
\setlength{\itemsep}{0cm}
\setlength{\parsep}{0cm}
\setlength{\labelsep}{0cm}
\setlength{\labelwidth}{0cm}
\catcode`\#=12
\catcode`\$=12
\catcode`\%=12
\catcode`\^=12
\catcode`\_=12
\catcode`\.=12
\catcode`\?=12
\catcode`\!=12
\catcode`\&=12
\ttfamily}
\small
\item[]\textsl{-\ }RFA.standard\ rfa;
\item[]\textsl{val\ it\ =\ false\ :\ bool}
\item[]\textsl{-\ }val\ rfa'\ =\ RFA.standardize\ rfa;
\item[]\textsl{val\ rfa'\ =\ -\ :\ rfa}
\item[]\textsl{-\ }RFA.output("",\ rfa');
\item[]\textsl{\symbol{'173}states\symbol{'175}\ A,\ B,\ <A>,\ <B>\ \symbol{'173}start\ state\symbol{'175}\ A\ \symbol{'173}accepting\ states\symbol{'175}\ B}
\item[]\textsl{\symbol{'173}transitions\symbol{'175}}
\item[]\textsl{A,\ %\ ->\ <A>;\ <A>,\ 0\ ->\ <A>;\ <A>,\ 1\ +\ 2\ ->\ <B>;\ <B>,\ %\ ->\ B;}
\item[]\textsl{<B>,\ 3(%\ +\ 4)\ ->\ <B>}
\item[]\textsl{val\ it\ =\ ()\ :\ unit}
\item[]\textsl{-\ }RFA.standard\ rfa';
\item[]\textsl{val\ it\ =\ true\ :\ bool}
\end{list}


\index{regular expression finite automaton!eliminateState@$\eliminateState$}%
Next, we define a function $\eliminateState$ that takes in a function
$\Simp\in\Reg\fun\Reg$, and returns a function that takes in a pair
$(M, q)$, where $M$ is an RFA and $q\in Q_M-(\{s_M\}\cup A_M)$, and
returns an RFA.  When called with such a $\Simp$ and $(M,q)$,
$\eliminateState$ returns the RFA $N$ such that:
\begin{itemize}
\item $Q_N = Q_M - \{q\}$;

\item $s_N = s_M$;

\item $A_N = A_M$; and

\item $T_N = \combineTrans(\Simp, U\cup V)$, where
  \begin{itemize}
  \item $U = \setof{p,\alpha\fun r\in T_M}{p\neq q\eqtxt{and}r\neq
      q}$,

  \item $V = \setof{p,\Simp(\alpha\beta^*\gamma)\fun r}{p\neq q, r\neq
      q, p,\alpha\fun q\in T_M \eqtxt{and} q,\gamma\fun r\in T_M}$, and

  \item $\beta$ is the unique $\alpha\in\Reg$ such that
    $q,\alpha\fun q\in T_M$, if such an $\alpha$ exists, and is $\%$,
    otherwise.
  \end{itemize}
\end{itemize}

Suppose $\Simp$ is $\locallySimplify\,\obviousSubset$ and $M$ is the FA
\begin{center}
\input{chap-3.12-fig27.eepic}
\end{center}
Then $\eliminateState\,\Simp\,(M, \Bsf)$ is
\begin{center}
\input{chap-3.12-fig28.eepic}
\end{center}
And, we can eliminate $\Csf$ from this RFA, yielding
\begin{center}
\input{chap-3.12-fig31.eepic}
\end{center}
Alternatively, we could eliminate $\Csf$ from
\begin{center}
\input{chap-3.12-fig27.eepic}
\end{center}
yielding
\begin{center}
\input{chap-3.12-fig29.eepic}
\end{center}
And could then eliminate $\Bsf$ from this RFA, yielding
\begin{center}
\input{chap-3.12-fig31.eepic}
\end{center}
($\Simp(\mathsf{0(13^*2)^*(13^*4)}) = \mathsf{01(3+21)^*4}$.)

If we had an efficient regular expression simplifier that produced
optimal results, then the order in which we eliminated states would be
irrelevant.  But using our existing simplifiers, it turns out that
eliminating states in some orders produces much better results than
doing so in other orders.  Instead of eliminating first $\Csf$ and then
$\Bsf$, we could have renamed $M$'s states using the bijection
\begin{gather*}
\{\mathsf{(A, A), (B, C), (C, B), (D, D)}\}  
\end{gather*}
and then have eliminated states in ascending order, according
to our usual ordering on symbols: first $\Bsf$ and then $\Csf$.
This is the approach we'll use when looking for alternative answers.

\begin{proposition}
Suppose $\Simp\in\Reg\fun\Reg$, $M$ is an RFA and $q\in
Q_M-(\{s_M\}\cup A_M)$.  Then:
\begin{enumerate}[\quad(1)]
\item $\eliminateState\,\Simp\,(M,q)$ has one less state than $M$.

\item If $M$ is standard, then $\eliminateState\,\Simp\,(M,q)$ is standard.

\item If, for all $\alpha\in\Reg$, $L(\Simp\,\alpha) = L(\alpha)$,
  then
  \begin{displaymath}
    L(\eliminateState\,\Simp\,(M,q)) = L(M) .  
  \end{displaymath}

\item If, for all $\alpha\in\Reg$, $\alphabet(\Simp\,\alpha)\sub
  \alphabet\,\alpha$, then
  \begin{displaymath}
    \alphabet(\eliminateState\,\Simp\,(M,q)) \sub \alphabet\,M .
  \end{displaymath}
\end{enumerate}
\end{proposition}

The \texttt{RFA} module has a function
\begin{verbatim}
val eliminateState : (reg -> reg) -> rfa * sym -> rfa
\end{verbatim}
\index{RFA@\texttt{RFA}!eliminateState@\texttt{eliminateState}}%
that corresponds to $\eliminateState$.
Here is how our state-elimination examples can be carried out in Forlan:
\begin{list}{}
{\setlength{\leftmargin}{\leftmargini}
\setlength{\rightmargin}{0cm}
\setlength{\itemindent}{0cm}
\setlength{\listparindent}{0cm}
\setlength{\itemsep}{0cm}
\setlength{\parsep}{0cm}
\setlength{\labelsep}{0cm}
\setlength{\labelwidth}{0cm}
\catcode`\#=12
\catcode`\$=12
\catcode`\%=12
\catcode`\^=12
\catcode`\_=12
\catcode`\.=12
\catcode`\?=12
\catcode`\!=12
\catcode`\&=12
\ttfamily}
\small
\item[]\textsl{-\ }val\ rfa\ =\ RFA.input\ "";
\item[]\textsl{@\ }\symbol{'173}states\symbol{'175}\ A,\ B,\ C,\ D\ \symbol{'173}start\ state\symbol{'175}\ A\ \symbol{'173}accepting\ states\symbol{'175}\ D
\item[]\textsl{@\ }\symbol{'173}transitions\symbol{'175}
\item[]\textsl{@\ }A,\ 0\ ->\ B;\ B,\ 1\ ->\ C;\ C,\ 2\ ->\ B;\ C,\ 3\ ->\ C;\ C,\ 4\ ->\ D
\item[]\textsl{@\ }.
\item[]\textsl{val\ rfa\ =\ -\ :\ rfa}
\item[]\textsl{-\ }val\ eliminateState\ =\ RFA.eliminateState\ simp;
\item[]\textsl{val\ eliminateState\ =\ fn\ :\ rfa\ \symbol{'052}\ sym\ ->\ rfa}
\item[]\textsl{-\ }val\ rfa'\ =\ eliminateState(rfa,\ Sym.fromString\ "B");
\item[]\textsl{val\ rfa'\ =\ -\ :\ rfa}
\item[]\textsl{-\ }RFA.output("",\ rfa');
\item[]\textsl{\symbol{'173}states\symbol{'175}\ A,\ C,\ D\ \symbol{'173}start\ state\symbol{'175}\ A\ \symbol{'173}accepting\ states\symbol{'175}\ D}
\item[]\textsl{\symbol{'173}transitions\symbol{'175}\ A,\ 01\ ->\ C;\ C,\ 4\ ->\ D;\ C,\ 3\ +\ 21\ ->\ C}
\item[]\textsl{val\ it\ =\ ()\ :\ unit}
\item[]\textsl{-\ }val\ rfa''\ =\ eliminateState(rfa',\ Sym.fromString\ "C");
\item[]\textsl{val\ rfa''\ =\ -\ :\ rfa}
\item[]\textsl{-\ }RFA.output("",\ rfa'');
\item[]\textsl{\symbol{'173}states\symbol{'175}\ A,\ D\ \symbol{'173}start\ state\symbol{'175}\ A\ \symbol{'173}accepting\ states\symbol{'175}\ D}
\item[]\textsl{\symbol{'173}transitions\symbol{'175}\ A,\ 01(3\ +\ 21)\symbol{'052}4\ ->\ D}
\item[]\textsl{val\ it\ =\ ()\ :\ unit}
\item[]\textsl{-\ }val\ rfa'''\ =\ eliminateState(rfa,\ Sym.fromString\ "C");
\item[]\textsl{val\ rfa'''\ =\ -\ :\ rfa}
\item[]\textsl{-\ }RFA.output("",\ rfa''');
\item[]\textsl{\symbol{'173}states\symbol{'175}\ A,\ B,\ D\ \symbol{'173}start\ state\symbol{'175}\ A\ \symbol{'173}accepting\ states\symbol{'175}\ D}
\item[]\textsl{\symbol{'173}transitions\symbol{'175}\ A,\ 0\ ->\ B;\ B,\ 13\symbol{'052}2\ ->\ B;\ B,\ 13\symbol{'052}4\ ->\ D}
\item[]\textsl{val\ it\ =\ ()\ :\ unit}
\item[]\textsl{-\ }val\ rfa''''\ =\ eliminateState(rfa''',\ Sym.fromString\ "B");
\item[]\textsl{val\ rfa''''\ =\ -\ :\ rfa}
\item[]\textsl{-\ }RFA.output("",\ rfa'''');
\item[]\textsl{\symbol{'173}states\symbol{'175}\ A,\ D\ \symbol{'173}start\ state\symbol{'175}\ A\ \symbol{'173}accepting\ states\symbol{'175}\ D}
\item[]\textsl{\symbol{'173}transitions\symbol{'175}\ A,\ 01(3\ +\ 21)\symbol{'052}4\ ->\ D}
\item[]\textsl{val\ it\ =\ ()\ :\ unit}
\end{list}

And \texttt{eliminateState} stops us from eliminating a start state or
an accepting state:
\begin{list}{}
{\setlength{\leftmargin}{\leftmargini}
\setlength{\rightmargin}{0cm}
\setlength{\itemindent}{0cm}
\setlength{\listparindent}{0cm}
\setlength{\itemsep}{0cm}
\setlength{\parsep}{0cm}
\setlength{\labelsep}{0cm}
\setlength{\labelwidth}{0cm}
\catcode`\#=12
\catcode`\$=12
\catcode`\%=12
\catcode`\^=12
\catcode`\_=12
\catcode`\.=12
\catcode`\?=12
\catcode`\!=12
\catcode`\&=12
\ttfamily}
\small
\item[]\textsl{-\ }eliminateState(rfa,\ Sym.fromString\ "A");
\item[]\textsl{cannot\ eliminate\ start\ state:\ "A"}
\item[]
\item[]\textsl{uncaught\ exception\ Error}
\item[]\textsl{-\ }eliminateState(rfa,\ Sym.fromString\ "D");
\item[]\textsl{cannot\ eliminate\ accepting\ state:\ "D"}
\item[]
\item[]\textsl{uncaught\ exception\ Error}
\end{list}


\index{regular expression finite automaton!rfaToReg@$\rfaToReg$}%
Now, we use $\eliminateState$ to define a function/algorithm
\begin{displaymath}
  \rfaToReg \in (\Reg\fun\Reg) \fun \RFA \fun \Reg .
\end{displaymath}
It takes elements $\Simp\in\Reg\fun\Reg$ and $M\in\RFA$, and
returns
\begin{displaymath}
  f(\standardize\,M) ,
\end{displaymath}
where $f$ is the function from standard RFAs to regular expressions
that is defined by well-founded recursion on the number of states of
its input, $M$, as follows:
\begin{itemize}
\item If $M$ has only two states,  then $f$ returns the
  label of the transition from $s_M$ to $M$'s accepting state, if such
  a transition exists, and returns $\$$, otherwise.

\item Otherwise,  $f$ calls itself recursively on
  $\eliminateState\,\Simp\,(M,q)$, where $q$ is the least element (in the
  standard ordering on symbols) of $Q_M-(\{s_M\}\cup A_M)$.
\end{itemize}

\begin{proposition}
Suppose $M$ is an RFA and
$\Simp\in\Reg\fun\Reg$ has the property that,
for all $\alpha\in\Reg$, $L(\Simp\,\alpha) = L(\alpha)$ and
$\alphabet(\Simp\,\alpha)\sub\alphabet\,\alpha$.
Then:
\begin{enumerate}[\quad(1)]
\item $L(\rfaToReg\,\Simp\,M) = L(M)$; and 

\item $\alphabet(\rfaToReg\,\Simp\,M) \sub \alphabet\,M$.
\end{enumerate}
\end{proposition}

Finally, we define our RFA to regular expression conversion algorithm/function:
\index{regular expression finite automaton!rfaToReg@$\faToReg$}%
\begin{gather*}
\faToReg\in(\Reg\fun\Reg)\fun\FA\fun\Reg .  
\end{gather*}
$\faToReg$ takes in $\Simp\in\Reg\fun\Reg$, and returns
\begin{displaymath}
\rfaToReg\,\Simp \circ \faToRFA\,\Simp .
\end{displaymath}

\begin{proposition}
Suppose $M$ is an FA and
$\Simp\in\Reg\fun\Reg$ has the property that,
for all $\alpha\in\Reg$, $L(\Simp\,\alpha) = L(\alpha)$ and
$\alphabet(\Simp\,\alpha)\sub\alphabet\,\alpha$.
Then:
\begin{enumerate}[\quad(1)]
\item $L(\faToReg\,\Simp\,M) = L(M)$; and 

\item $\alphabet(\faToReg\,\Simp\,M) \sub \alphabet\,M$.
\end{enumerate}
\end{proposition}

The Forlan module \texttt{RFA} includes functions
\begin{verbatim}
val toReg             : (reg -> reg) -> rfa -> reg
val faToReg           : (reg -> reg) -> fa -> reg
val faToRegPerms      : int option * (reg -> reg) -> fa -> reg
val faToRegPermsTrace : int option * (reg -> reg) -> fa -> reg
\end{verbatim}
\index{RFA@\texttt{RFA}!toReg@\texttt{toReg}}%
\index{RFA@\texttt{RFA}!faToReg@\texttt{faToReg}}%
\index{RFA@\texttt{RFA}!faToRegPerms@\texttt{faToRegPerms}}%
\index{RFA@\texttt{RFA}!faToRegPermsTrace@\texttt{faToRegPermsTrace}}%
The function \texttt{toReg} corresponds to $\rfaToReg$.
The function \texttt{faToReg} is the implementation of $\faToReg$.  If
$\Simp$ is a simplification function and $M$ is an FA, then
$\mathtt{faToRegPerms}\,(\mathtt{NONE},\Simp)\,M$ applies $\faToReg$
to all of the FAs $N$ that can be formed by renaming $M$'s states
using bijections (i.e., permutations) from $Q_M$ to $Q_M$, and returns
the simplest answer found (ties in complexity are broken by selecting
the smallest regular expression in our total ordering on regular
expressions.)
$\mathtt{faToRegPerms}\,(\mathtt{SOME}\,n,\Simp)\,M$, for $n\geq 1$,
works similarly, except that only $n$ ways of renaming $M$'s state are
considered.  And \texttt{faToRegPermsTrace} is like
\texttt{faToRegPerms} except that it explains what it's doing.  The
functions \texttt{faToReg}, \texttt{faToRegPerms} and
\texttt{faToRegTrace} are also available in the top-level environment
with those names:
\begin{verbatim}
val faToReg           : (reg -> reg) -> fa -> reg
val faToRegPerms      : int option * (reg -> reg) -> fa -> reg
val faToRegPermsTrace : int option * (reg -> reg) -> fa -> reg
\end{verbatim}
\index{regular expression!faToReg@\texttt{faToReg}}%
\index{regular expression!faToRegPerms@\texttt{faToRegPerms}}%
\index{regular expression!faToRegPermsTrace@\texttt{faToRegPermsTrace}}%
\index{finite automaton!faToReg@\texttt{faToReg}}%
\index{finite automaton!faToRegPerms@\texttt{faToRegPerms}}%
\index{finite automaton!faToRegPermsTrace@\texttt{faToRegPermsTrace}}%

Suppose \texttt{fa} is the FA
\begin{center}
\input{chap-3.12-fig30.eepic}
\end{center}
which accepts $\setof{w\in\{\mathsf{0,1}\}^*}{w\eqtxt{has
an even number of}\zerosf\eqtxt{and}\onesf\eqtxtn{'s}}$.
converting \texttt{fa} into a regular expression using
\texttt{faToReg} and $\weaklySimplify$ yields a fairly complicated answer:
\begin{list}{}
{\setlength{\leftmargin}{\leftmargini}
\setlength{\rightmargin}{0cm}
\setlength{\itemindent}{0cm}
\setlength{\listparindent}{0cm}
\setlength{\itemsep}{0cm}
\setlength{\parsep}{0cm}
\setlength{\labelsep}{0cm}
\setlength{\labelwidth}{0cm}
\catcode`\#=12
\catcode`\$=12
\catcode`\%=12
\catcode`\^=12
\catcode`\_=12
\catcode`\.=12
\catcode`\?=12
\catcode`\!=12
\catcode`\&=12
\ttfamily}
\small
\item[]\textsl{-\ }val\ reg\ =\ faToReg\ Reg.weaklySimplify\ fa;
\item[]\textsl{val\ reg\ =\ -\ :\ reg}
\item[]\textsl{-\ }Reg.output("",\ reg);
\item[]\textsl{%\ +\ 00(00)\symbol{'052}\ +\ (1\ +\ 00(00)\symbol{'052}1)(11\ +\ 100(00)\symbol{'052}1)\symbol{'052}(1\ +\ 100(00)\symbol{'052})\ +}
\item[]\textsl{(0(00)\symbol{'052}1\ +\ (1\ +\ 00(00)\symbol{'052}1)(11\ +\ 100(00)\symbol{'052}1)\symbol{'052}(0\ +\ 10(00)\symbol{'052}1))}
\item[]\textsl{(1(00)\symbol{'052}1\ +\ (0\ +\ 10(00)\symbol{'052}1)(11\ +\ 100(00)\symbol{'052}1)\symbol{'052}(0\ +\ 10(00)\symbol{'052}1))\symbol{'052}}
\item[]\textsl{(10(00)\symbol{'052}\ +\ (0\ +\ 10(00)\symbol{'052}1)(11\ +\ 100(00)\symbol{'052}1)\symbol{'052}(1\ +\ 100(00)\symbol{'052}))}
\item[]\textsl{val\ it\ =\ ()\ :\ unit}
\end{list}

But by using \texttt{faToRegPerms}, we can do much better:
\begin{list}{}
{\setlength{\leftmargin}{\leftmargini}
\setlength{\rightmargin}{0cm}
\setlength{\itemindent}{0cm}
\setlength{\listparindent}{0cm}
\setlength{\itemsep}{0cm}
\setlength{\parsep}{0cm}
\setlength{\labelsep}{0cm}
\setlength{\labelwidth}{0cm}
\catcode`\#=12
\catcode`\$=12
\catcode`\%=12
\catcode`\^=12
\catcode`\_=12
\catcode`\.=12
\catcode`\?=12
\catcode`\!=12
\catcode`\&=12
\ttfamily}
\small
\item[]\textsl{-\ }val\ reg'\ =\ faToRegPerms\ (NONE,\ Reg.weaklySimplify)\ fa;
\item[]\textsl{val\ reg'\ =\ -\ :\ reg}
\item[]\textsl{-\ }Reg.output("",\ reg');
\item[]\textsl{(00\ +\ 11\ +\ (01\ +\ 10)(00\ +\ 11)\symbol{'052}(01\ +\ 10))\symbol{'052}}
\item[]\textsl{val\ it\ =\ ()\ :\ unit}
\end{list}

By using \texttt{faToRegPermsTrace}, we can learn that this answer was
found using the renaming
\begin{gather*}
\mathsf{(A, D), (B, A), (C, B), (D, C)}
\end{gather*}
of $M$'s states.  That is, it was found by making $M$ into
a standard RFA, with new start and accepting states,
and then eliminating the states corresponding to
$\Bsf$, $\Csf$, $\Dsf$ and $\Asf$, in that order.

\subsection{Characterization of Regular Languages}

\index{characterization of regular languages}%
\index{regular languages: characterization}%
Since we have algorithms for converting back and forth between
regular expressions and finite automata, as well as algorithms for
converting FAs to RFAs, RFAs to regular expressions,
FAs to EFAs, EFAs to NFAs, and NFAs to DFAs,
we have the following theorem:
\begin{theorem}
\label{RegularEquiv}
Suppose $L$ is a language.  The following statements are equivalent:
\begin{itemize}
\item $L$ is regular;

\item $L$ is generated by a regular expression;

\item $L$ is accepted by a regular expression finite automaton;

\item $L$ is accepted by a finite automaton;

\item $L$ is accepted by an EFA;

\item $L$ is accepted by an NFA; and

\item $L$ is accepted by a DFA.
\end{itemize}
\end{theorem}

\subsection{More Closure Properties/Algorithms}

\index{language!intersection}%
\index{empty-string finite automaton!intersection}%
\index{intersection!empty-string finite automaton}%
Consider the EFAs $M_1$ and $M_2$:
\begin{center}
\input{chap-3.12-fig14.eepic}
\end{center}
How can we construct an EFA $N$ such that $L(N)=L(M_1)\cap L(M_2)$?
The idea is to make the states of $N$ represent pairs of the form
$(q,r)$, where $q\in Q_{M_1}$ and $r\in Q_{M_2}$.

In order to define our intersection operation on EFAs, we first need
to define two auxiliary functions.  Suppose $M_1$ and $M_2$ are EFAs.
We define a function
\begin{gather*}
\nextSym_{M_1,M_2}\in
(Q_{M_1}\times Q_{M_2})\times\Sym \fun
\powset(Q_{M_1}\times Q_{M_2})
\end{gather*}
by $\nextSym_{M_1,M_2}((q,r),a)={}$
\begin{gather*}
\setof{(q',r')}{q,a\fun q'\in T_{M_1}\eqtxt{and}
r,a\fun r'\in T_{M_2}} .
\end{gather*}
We often abbreviate $\nextSym_{M_1,M_2}$ to $\nextSym$.
If $M_1$ and $M_2$ are our example EFAs, then
$\nextSym((\Asf,\Asf),\zerosf) = \emptyset$ and
$\nextSym((\Asf,\Bsf),\zerosf) = \{(\Asf,\Bsf)\}$.
Suppose $M_1$ and $M_2$ are EFAs.  We define a function
\begin{gather*}
\nextEmp_{M_1,M_2}\in
(Q_{M_1}\times Q_{M_2})\fun\powset(Q_{M_1}\times Q_{M_2})
\end{gather*}
by $\nextEmp_{M_1,M_2}(q,r)={}$
\begin{gather*}
\setof{(q',r)}{q,\%\fun q'\in T_{M_1}} \cup
\setof{(q,r')}{r,\%\fun r'\in T_{M_2}} .
\end{gather*}
We often abbreviate $\nextEmp_{M_1,M_2}$ to $\nextEmp$.  If $M_1$ and
$M_2$ are our example EFAs, then $\nextEmp(\Asf,\Asf) =
\{\mathsf{(B,A),(A,B)}\}$, $\nextEmp(\Asf,\Bsf) = \{(\Bsf,\Bsf)\}$,
$\nextEmp(\Bsf,\Asf) = \{(\Bsf,\Bsf)\}$ and $\nextEmp(\Bsf,\Bsf) =
\emptyset$.

Now, we define a function/algorithm
\index{empty-string finite automaton!inter@$\inter$}%
$\inter\in\EFA\times\EFA\fun\EFA$ such that
$L(\inter(M_1,M_2))=L(M_1)\cap L(M_2)$, for all $M_1,M_2\in\EFA$.
Given EFAs $M_1$ and $M_2$, $\inter(M_1,M_2)$, the \emph{intersection of}
$M_1$ \emph{and} $M_2$, is the EFA $N$ that is constructed as follows.
First, we let $\Sigma=\alphabet\,M_1\cap\alphabet\,M_2$.
Next, we generate the least subset $X$ of $Q_{M_1}\times Q_{M_2}$
such that
\begin{itemize}
\item $(s_{M_1},s_{M_2})\in X$;

\item for all $q\in Q_{M_1}$, $r\in Q_{M_2}$ and $a\in\Sigma$,
if $(q,r)\in X$, then $\nextSym((q,r),a)\sub X$; and

\item for all $q\in Q_{M_1}$ and $r\in Q_{M_2}$,
if $(q,r)\in X$, then $\nextEmp(q,r)\sub X$.
\end{itemize}
Then, the EFA $N$ is defined by:
\begin{itemize}
\item $Q_N=
\setof{\langle q,r\rangle}{(q,r)\in X}$;

\item $s_N=\langle s_{M_1},s_{M_2}\rangle$;

\item $A_N=\setof{\langle q,r\rangle}{(q,r)\in X\eqtxt{and}
q\in A_{M_1}\eqtxt{and}r\in A_{M_2}}$; and

\item $T_N={}$
  \begin{align*}
    &\quad \{\,\langle q,r\rangle,a\fun\langle
    q',r'\rangle\mid(q,r)\in X\eqtxt{and}
    a\in\Sigma\eqtxt{and}\\
    &\hspace{3.45cm}(q',r')\in\nextSym((q,r),a)\,\} \\
    &\cup
    \{\,\langle q,r\rangle,\%\fun\langle q',r'\rangle\mid(q,r)\in
    X\eqtxt{and} \\
    &\hspace{3.6cm}(q',r')\in\nextEmp(q,r)\,\}.
  \end{align*}
\end{itemize}

Suppose $M_1$ and $M_2$ are our example EFAs.
Then $\inter(M_1,M_2)$ is
\begin{center}
\input{chap-3.12-fig15.eepic}
\end{center}

\begin{theorem}
For all $M_1,M_2\in\EFA$:
\begin{itemize}
\item $L(\inter(M_1,M_2))=L(M_1)\cap L(M_2)$; and

\item $\alphabet(\inter(M_1,M_2))\sub \alphabet\,M_1\cap\alphabet\,M_2$.
\end{itemize}
\end{theorem}

\index{nondeterministic finite automaton!intersection}%
\index{nondeterministic finite automaton!inter@$\inter$}%
\index{intersection!nondeterministic finite automaton}%
\begin{proposition}
For all $M_1,M_2\in\NFA$, $\inter(M_1,M_2)\in\NFA$.
\end{proposition}

\index{deterministic finite automaton!intersection}%
\index{deterministic finite automaton!inter@$\inter$}%
\index{intersection!deterministic finite automaton}%
\begin{proposition}
For all $M_1,M_2\in\DFA$:
\begin{enumerate}[\quad(1)]
\item $\inter(M_1,M_2)\in\DFA$.

\item $\alphabet(\inter(M_1,M_2)) = \alphabet\,M_1\cap\alphabet\,M_2$.
\end{enumerate}
\end{proposition}

\index{complementation!deterministic finite automaton}%
\index{deterministic finite automaton!complementation}%
\index{deterministic finite automaton!complement@$\mycomplement$}%
Next, we define a function $\mycomplement\in\DFA\times\Alp\fun\DFA$
such that, for all $M\in\DFA$ and $\Sigma\in\Alp$,
\begin{gather*}
L(\mycomplement(M,\Sigma)) = (\alphabet(L(M))\cup\Sigma)^*-L(M) .
\end{gather*}
In the common case when $L(M)\sub\Sigma^*$, we will have that
$\alphabet(L(M))\sub\Sigma$, and thus that
$(\alphabet(L(M))\cup\Sigma)^*=\Sigma^*$.  Hence, it will be the case
that
\begin{gather*}
L(\mycomplement(M,\Sigma)) = \Sigma^*-L(M) .
\end{gather*}

Given a DFA $M$ and an alphabet $\Sigma$, $\mycomplement(M,\Sigma)$,
the \emph{complement of} $M$ \emph{with reference to} $\Sigma$,
is the DFA $N$ that is produced as follows.  First, we let the DFA
$M'=\determSimplify(M,\Sigma)$.  Thus:
\begin{itemize}
\item $M'$ is equivalent to $M$; and

\item $\alphabet\,M' = \alphabet(L(M))\cup\Sigma$.
\end{itemize}
Then, we define $N$ by:
\begin{itemize}
\item $Q_N = Q_{M'}$;

\item $s_N = s_{M'}$;

\item $A_N = Q_{M'} - A_{M'}$; and

\item $T_N = T_{M'}$.
\end{itemize}
Then, for all $w\in(\alphabet\,M')^*=
(\alphabet\,N)^*=(\alphabet(L(M))\cup\Sigma)^*$,
\begin{align*}
w\in L(N) &\myiff \delta_N(s_N, w)\in A_N \\
&\myiff \delta_N(s_N, w)\in Q_{M'}-A_{M'} \\
&\myiff \delta_{M'}(s_{M'}, w)\not\in A_{M'} \\
&\myiff w\not\in L(M') \\
&\myiff w\not\in L(M) .
\end{align*}
Hence:

\begin{theorem}
For all $M\in\DFA$ and $\Sigma\in\Alp$:
\begin{itemize}
\item $L(\mycomplement(M,\Sigma)) = (\alphabet(L(M))\cup\Sigma)^*-L(M)$; and

\item $\alphabet(\mycomplement(M,\Sigma)) = \alphabet(L(M))\cup\Sigma$.
\end{itemize}
\end{theorem}

For example, suppose the DFA $M$ is
\begin{center}
\input{chap-3.12-fig16.eepic}
\end{center}
Then $\determSimplify(M,\{\twosf\})$ is the DFA
\begin{center}
\input{chap-3.12-fig17.eepic}
\end{center}
Thus $\mycomplement(M,\{\twosf\})$ is
\begin{center}
\input{chap-3.12-fig18.eepic}
\end{center}

Let $X=\setof{w\in\{\mathsf{0,1}\}^*}{\mathsf{000}\eqtxt{is not a
    substring of}w}$.  Then $L(\mycomplement(M,\{\twosf\}))$ is
\begin{alignat*}{2}
&\mathrel{\hspace*{.3cm}} (\alphabet(L(M))\cup\{\twosf\})^*-L(M) \\
&=(\{\mathsf{0,1}\}\cup\{\twosf\})^* - X \\
&=\setof{w\in\{\mathsf{0,1,2}\}^*}{w\not\in X} \\
&=\setof{w\in\{\mathsf{0,1,2}\}^*}{2\in\alphabet\,w\eqtxt{or}
\mathsf{000}\eqtxt{is a substring of}w} .
\end{alignat*}

\index{language!difference}%
\index{deterministic finite automaton!difference}%
\index{deterministic finite automaton!minus@$\minus$}%
\index{difference!deterministic finite automaton}%
We define a function/algorithm $\minus\in\DFA\times\DFA\fun\DFA$ by:
$\minus(M_1,M_2)$, the \emph{difference of} $M_1$ \emph{and} $M_2$, is
\begin{gather*}
\inter(M_1,\mycomplement(M_2,{\alphabet\,M_1})) .
\end{gather*}

\begin{theorem}
\label{DFAMinusTheorem}
For all $M_1,M_2\in\DFA$:
\begin{enumerate}[\quad(1)]
\item $L(\minus(M_1,M_2)) = L(M_1)-L(M_2)$; and

\item $\alphabet(\minus(M_1,M_2)) = \alphabet\,M_1$.
\end{enumerate}
\end{theorem}

\begin{proof}
Suppose $w\in\Str$.  Then:
\begin{align*}
&\hspace{.95cm} w\in L(\minus(M_1,M_2)) \\
&\myiff w\in L(\inter(M_1,\mycomplement(M_2,\alphabet\,M_1))) \\
&\myiff w\in L(M_1)\eqtxt{and}w\in L(\mycomplement(M_2,\alphabet\,M_1)) \\
&\myiff w\in L(M_1)\eqtxt{and}w\in(\alphabet(L(M_2))\cup\alphabet\,M_1)^*\eqtxt{and} \\
&\qquad\;\; w\not\in L(M_2) \\
&\myiff w\in L(M_1)\eqtxt{and}w\not\in L(M_2) \\
&\myiff w\in L(M_1)-L(M_2) .
\end{align*}
\end{proof}

To see why the second argument to $\mycomplement$ is
$\alphabet\,M_1$, in the definition of $\minus(M_1,M_2)$, look at
the ``if'' direction of the second-to-last step of the preceding
proof: since $w\in L(M_1)$, we have that $w\in(\alphabet\,M_1)^*$, so
that $w\in (\alphabet(L(M_2))\cup\alphabet(M_1))^*$.

For example, let $M_1$ and $M_2$ be the EFAs
\begin{center}
\input{chap-3.12-fig14.eepic}
\end{center}
Since $L(M_1)=\{\zerosf\}^*\{\onesf\}^*$ and
$L(M_2)=\{\onesf\}^*\{\zerosf\}^*$, we have that
\begin{displaymath}
L(M_1)-L(M_2)=\{\zerosf\}^*\{\onesf\}^*-\{\onesf\}^*\{\zerosf\}^* =
\{\zerosf\}\{\zerosf\}^*\{\onesf\}\{\onesf\}^* .
\end{displaymath}
Define DFAs $N_1$ and $N_2$ by:
\begin{align*}
N_1 &= \nfaToDFA(\efaToNFA\,M_1) , \eqtxt{and} \\
N_2 &= \nfaToDFA(\efaToNFA\,M_2) .
\end{align*}
Thus we have that
\begin{alignat*}{2}
L(N_1) &= L(\nfaToDFA(\efaToNFA(M_1))) \\
       &= L(\efaToNFA(M_1)) && \by{Theorem~\ref{NFAToDFATheorem}} \\
       &= L(M_1) && \by{Theorem~\ref{EFAToNFATheorem}} \\
L(N_2) &= L(\nfaToDFA(\efaToNFA(M_2))) \\
       &= L(\efaToNFA(M_2)) && \by{Theorem~\ref{NFAToDFATheorem}} \\
       &= L(M_2) && \by{Theorem~\ref{EFAToNFATheorem}} .
\end{alignat*}
Let the DFA $N=\minus(N_1,N_2)$.
Then
\begin{alignat*}{2}
L(N) &= L(\minus(N_1,N_2)) \\
     &= L(N_1)-L(N_2) && \by{Theorem~\ref{DFAMinusTheorem}} \\
     &= L(M_1)-L(M_2) \\
     &= \{\zerosf\}\{\zerosf\}^*\{\onesf\}\{\onesf\}^* .
\end{alignat*}

Next, we consider the reversal of languages, regular expressions,
finite automata and empty-string finite automata.
The \emph{reversal of}
\index{language!reversal}%
a language $L$ ($L^R\in\Lan$)
\index{ reversal@$\cdot^R$}%
\index{language! reversal@$\cdot^R$}%
is $\setof{w}{w^R\in L} = \setof{w^R}{w\in L}$.  I.e., $L^R$ is formed
by reversing all of the elements of $L$.  For example, $\{\mathsf{011,
1011}\}^R = \{\mathsf{110, 1101}\}$.

The module \texttt{StrSet}
\index{StrSet@\texttt{StrSet}}%
defines the function
\begin{verbatim}
val rev : str set -> str set
\end{verbatim}
\index{StrSet@\texttt{StrSet}!rev@\texttt{rev}}%
that implements language reversal. E.g., we can
proceed as follows:
\begin{list}{}
{\setlength{\leftmargin}{\leftmargini}
\setlength{\rightmargin}{0cm}
\setlength{\itemindent}{0cm}
\setlength{\listparindent}{0cm}
\setlength{\itemsep}{0cm}
\setlength{\parsep}{0cm}
\setlength{\labelsep}{0cm}
\setlength{\labelwidth}{0cm}
\catcode`\#=12
\catcode`\$=12
\catcode`\%=12
\catcode`\^=12
\catcode`\_=12
\catcode`\.=12
\catcode`\?=12
\catcode`\!=12
\catcode`\&=12
\ttfamily}
\small
\item[]\textsl{-\ }val\ xs\ =\ StrSet.input\ "";
\item[]\textsl{@\ }123,\ 234,\ 16
\item[]\textsl{@\ }.
\item[]\textsl{val\ xs\ =\ -\ :\ str\ set}
\item[]\textsl{-\ }StrSet.output("",\ StrSet.rev\ xs);
\item[]\textsl{61,\ 321,\ 432}
\item[]\textsl{val\ it\ =\ ()\ :\ unit}
\end{list}


Define $\rev\in\Reg\fun\Reg$ by recursion:
\index{regular expression!reversal}%
\index{reversal!regular expression}%
\index{rev@$\rev$}%
\index{regular expression!rev@$\rev$}%
\begin{align*}
\rev\,\% &= \%; \\
\rev\,\$ &= \$; \\
\rev\,a &= a, \eqtxt{for all}a\in\Sym; \\
\rev(\alpha^*) &= (\rev\,\alpha)^*, \eqtxt{for all}\alpha\in\Reg;\\
\rev(\alpha\,\beta) &= \rev\,\beta\,\rev\,\alpha ,
\eqtxt{for all}\alpha,\beta\in\Reg; \eqtxt{and} \\
\rev(\alpha+\beta) &= \rev\,\alpha + \rev\,\beta ,
\eqtxt{for all}\alpha,\beta\in\Reg .
\end{align*}
We say that $\rev\,\alpha$ is \emph{the reversal of} $\alpha$.
For example $\rev(\mathsf{01+(10)^*}) = \mathsf{10 + (01)^*}$.

\begin{theorem}
\label{RegExpRev}
For all $\alpha\in\Reg$:
\begin{itemize}
\item $L(\rev\,\alpha) = L(\alpha)^R$; and

\item $\alphabet(\rev\,\alpha) = \alphabet\,\alpha$.
\end{itemize}
\end{theorem}

\begin{proof}
By induction on $\alpha$.
\end{proof}

We can also define a reversal operation on FAs and EFAs.  The idea is
to reverse all of the transitions, add a new start state, with
$\%$-transitions to all of the original accepting states, and make the
original start state be the unique accepting state.  Formally, we
define a function $\rev\in\FA\fun\FA$ as
\index{reversal!finite automaton}%
\index{finite automaton!reversal}%
\index{empty-string finite automaton!reversal}%
\index{rev@$\rev$}%
\index{finite automaton!rev@$\rev$}%
\index{empty-string finite automaton!rev@$\rev$}%
follows.
Given an FA $M$, $\rev\,M$, \emph{the reversal of} $M$,
is the FA $N$ such that
\begin{itemize}
\item $Q_N=\{\Asf\}\cup\setof{\langle q\rangle}{q\in Q_M}$;

\item $s_N=\Asf$;

\item $A_N=\{\langle s_M\rangle\}$; and

\item $T_N=\setof{(\langle r\rangle, x^R, \langle q\rangle)}{(q, x, r)\in T_M} \cup
\setof{(\Asf, \%, \langle q\rangle)}{q\in A_M}$.
\end{itemize}
It is easy to see that, for all $M\in\EFA$, $\rev\,M\in\EFA$.

For example, if $M$ is the FA
\begin{center}
\input{chap-3.12-fig23.eepic}
\end{center}
then $\rev\,M$ is
\begin{center}
\input{chap-3.12-fig24.eepic}
\end{center}
We have that $\mathsf{012345}$ and $\mathsf{0123450123}$ are in $L(M)$,
and $\mathsf{543210}$ and $\mathsf{3210543210}$ are in $L(\rev\,M)=L(M)^R$.

\begin{theorem}
\label{FARev}
For all $M\in\FA$:
\begin{itemize}
\item $L(\rev\,M) = L(M)^R$; and

\item $\alphabet(\rev\,M) = \alphabet\,M$.
\end{itemize}
\end{theorem}

Next, we consider the prefix-, suffix- and substring-closures of
\index{language!prefix-closure}%
\index{language!suffix-closure}%
\index{language!substring-closure}%
languages, as well as the associated operations on automata.  Suppose
$L$ is a language.  Then:
\begin{itemize}
\item The \emph{prefix-closure} of $L$ ($L^P$) is $\setof{x}{xy\in L,
\eqtxt{for some}y\in\Str}$.  I.e., $L^P$ is all of the prefixes
of elements of $L$.  E.g., $\{\mathsf{012,3}\}^P =
\{\mathsf{\%,0,01,012,3}\}$.

\item The \emph{suffix-closure} of $L$ ($L^S$) is $\setof{y}{xy\in L,
\eqtxt{for some}x\in\Str}$.  I.e., $L^S$ is all of the suffixes
of elements of $L$.  E.g., $\{\mathsf{012,3}\}^S =
\{\mathsf{\%,2,12,012,3}\}$.

\item The \emph{substring-closure} of $L$ ($L^\SSop$) is $\setof{y}{xyz\in L,
\eqtxt{for some}x,z\in\Str}$.  I.e., $L^\SSop$ is all of the substrings
of elements of $L$.  E.g., $\{\mathsf{012,3}\}^\SSop =
\{\mathsf{\%, 0, 1, 2, 01, 12, 012, 3}\}$.
\end{itemize}

The following proposition shows that we can express suffix- and
substring-closure in terms of prefix-closure and language reversal.

\begin{proposition}
\label{PrefixSuffixSubstringClosure}
For all languages $L$:
\begin{itemize}
\item $L^S = ((L^R)^P)^R$; and

\item $L^\SSop = (L^P)^S$.
\end{itemize}
\end{proposition}

The module \texttt{StrSet}
\index{StrSet@\texttt{StrSet}}%
defines the functions
\begin{verbatim}
val prefix    : str set -> str set
val suffix    : str set -> str set
val substring : str set -> str set
\end{verbatim}
\index{StrSet@\texttt{StrSet}!prefix@\texttt{prefix}}%
\index{StrSet@\texttt{StrSet}!suffix@\texttt{suffix}}%
\index{StrSet@\texttt{StrSet}!substring@\texttt{substring}}%
that implement prefix-, suffix- and substring-closure. 
E.g., we can proceed as follows:
\begin{list}{}
{\setlength{\leftmargin}{\leftmargini}
\setlength{\rightmargin}{0cm}
\setlength{\itemindent}{0cm}
\setlength{\listparindent}{0cm}
\setlength{\itemsep}{0cm}
\setlength{\parsep}{0cm}
\setlength{\labelsep}{0cm}
\setlength{\labelwidth}{0cm}
\catcode`\#=12
\catcode`\$=12
\catcode`\%=12
\catcode`\^=12
\catcode`\_=12
\catcode`\.=12
\catcode`\?=12
\catcode`\!=12
\catcode`\&=12
\ttfamily}
\small
\item[]\textsl{-\ }val\ xs\ =\ StrSet.input\ "";
\item[]\textsl{@\ }123,\ 0234,\ 16
\item[]\textsl{@\ }.
\item[]\textsl{val\ xs\ =\ -\ :\ str\ set}
\item[]\textsl{-\ }StrSet.output("",\ StrSet.prefix\ xs);
\item[]\textsl{%,\ 0,\ 1,\ 02,\ 12,\ 16,\ 023,\ 123,\ 0234}
\item[]\textsl{val\ it\ =\ ()\ :\ unit}
\item[]\textsl{-\ }StrSet.output("",\ StrSet.suffix\ xs);
\item[]\textsl{%,\ 3,\ 4,\ 6,\ 16,\ 23,\ 34,\ 123,\ 234,\ 0234}
\item[]\textsl{val\ it\ =\ ()\ :\ unit}
\item[]\textsl{-\ }StrSet.output("",\ StrSet.substring\ xs);
\item[]\textsl{%,\ 0,\ 1,\ 2,\ 3,\ 4,\ 6,\ 02,\ 12,\ 16,\ 23,\ 34,\ 023,\ 123,\ 234,\ 0234}
\item[]\textsl{val\ it\ =\ ()\ :\ unit}
\end{list}


Now, we define a function $\prefix\in\EFA\fun\EFA$ such that
\index{prefix@$\prefix$}%
\index{prefix-closure!empty-string finite automaton}%
\index{empty-string finite automaton!prefix@$\prefix$}%
\index{empty-string finite automaton!prefix-closure}%
\index{prefix-closure!empty-string finite automaton}%
$L(\prefix\,M)=L(M)^P$, for all $M\in\EFA$.  Given an EFA $M$,
$\prefix\,M$, \emph{the prefix-closure of} $M$,
is the EFA $N$ that is constructed as follows.
First, we simplify $M$, producing an EFA $M'$ that is equivalent to
$M$ and either has no useless states, or consists of a single dead
state and no-transitions.
If $M'$ has no useless states, then we let $N$ be the same as $M'$
except that $A_N = Q_N=Q_{M'}$,
i.e., all states of $N$ are accepting states.
If $M'$ consists of a single dead state and no transitions,
then we let $N=M'$.

For example, suppose $M$ is the EFA
\begin{center}
\input{chap-3.12-fig19.eepic}
\end{center}
so that $L(M)=\{\mathsf{001}\}^*$.  Then
$\prefix\,M$ is the EFA
\begin{center}
\input{chap-3.12-fig20.eepic}
\end{center}
which accepts
$\{\mathsf{001}\}^*\{\%,\zerosf,\zerosf\zerosf\}$.

\begin{theorem}
\label{EFAPrefix}
For all $M\in\EFA$:
\begin{itemize}
\item $L(\prefix\,M) = L(M)^P$; and

\item $\alphabet(\prefix\,M) = \alphabet(L(M))$.
\end{itemize}
\end{theorem}

\begin{proposition}
For all $M\in\NFA$, $\prefix\,M\in\NFA$.
\end{proposition}
\index{nondeterministic finite automaton!prefix@$\prefix$}%
\index{nondeterministic finite automaton!prefix-closure}%
\index{prefix-closure!nondeterministic finite automaton}%
Now we can define suffix-closure and substring-closure
operations on EFAs as follows.  The functions $\suffix,\substring\in
\EFA\fun\EFA$ are defined by:
\index{empty-string finite automaton!suffix@$\suffix$}%
\index{empty-string finite automaton!substring@$\substring$}%
\index{empty-string finite automaton!suffix-closure}%
\index{empty-string finite automaton!substring-closure}%
\index{suffix-closure!empty-string finite automaton}%
\index{substring-closure!empty-string finite automaton}%
\begin{align*}
\suffix\,M &= \rev(\prefix(\rev\,M))) , \eqtxt{and} \\
\substring\,M &= \suffix(\prefix\,M)) .
\end{align*}

\begin{theorem}
For all $M\in\EFA$:
\begin{itemize}
\item $L(\suffix\,M) = L(M)^S$; and

\item $L(\substring\,M) = L(M)^\SSop$.
\end{itemize}
\end{theorem}

Next, we consider the renaming of regular expressions and finite
automata using bijections on symbols.  If $x$ is a string and $f$ is a
bijection from a set of symbols that is a superset of $\alphabet\,x$
(maybe $\alphabet\,x$ itself, i.e., the symbols appearing in $x$) to
some set of symbols, then the \emph{renaming of} $x$ \emph{using} $f$
($x^f\in\Str$) is the result of applying $f$ to
\index{string!alphabet renaming}%
\index{alphabet renaming!string}%
\index{ power@$\cdot^\cdot$}%
\index{string! power@$\cdot^\cdot$}%
each symbol of $x$.  For example, if
$f=\mathsf{\{(0, 1), (1, 2), (2, 3)\}}$, then $\%^f=\%$ and
$(\mathsf{01102})^f=\mathsf{12213}$.

If $L$ is a language, and $f$ is a bijection from a set of symbols
that is a superset of $\alphabet\,L$ (maybe $\alphabet\,L$ itself)
to some set of symbols, then the \emph{renaming of} $L$ \emph{using}
\index{language!alphabet renaming}%
\index{alphabet renaming!language}%
\index{ power@$\cdot^\cdot$}%
\index{language! power@$\cdot^\cdot$}%
$f$ ($L^f\in\Lan$) is formed by applying $f$ to every symbol of every string
of $L$.  For example, if $L=\{\mathsf{012, 12}\}$ and
$f=\mathsf{\{(0, 1), (1, 2), (2, 3)\}}$, then $L^f=\{\mathsf{123,
23}\}$.

\begin{exercise}
Suppose
$L = \setof{w\in\{\mathsf{0,1,2}\}^*}{w\eqtxt{has an even number
    of occurrences of}\zerosf}$. Suppose $f$ is the bijection from
$\{\mathsf{0,1,2}\}$ to $\{\mathsf{0,1,2}\}$ defined by:
$f\,\zerosf = \onesf$, $f\,\onesf = \zerosf$, and
$f\,\twosf = \twosf$. Show that $L^f = 
\setof{w\in\{\mathsf{0,1,2}\}^*}{w\eqtxt{has an even number
    of occurrences of }\onesf}$.
\end{exercise}

The module \texttt{Str}
\index{Str@\texttt{Str}}%
defines the function
\begin{verbatim}
val renameAlphabet : str * sym_rel -> str
\end{verbatim}
that implements alphabet renaming for strings.
And the module \texttt{StrSet}
\index{StrSet@\texttt{StrSet}}%
defines the function
\begin{verbatim}
val renameAlphabet : str set * sym_rel -> str set
\end{verbatim}
\index{StrSet@\texttt{StrSet}!renameAlphabet@\texttt{renameAlphabet}}%
that implements alphabet renaming for languages.
E.g., we can proceed as follows:
\begin{list}{}
{\setlength{\leftmargin}{\leftmargini}
\setlength{\rightmargin}{0cm}
\setlength{\itemindent}{0cm}
\setlength{\listparindent}{0cm}
\setlength{\itemsep}{0cm}
\setlength{\parsep}{0cm}
\setlength{\labelsep}{0cm}
\setlength{\labelwidth}{0cm}
\catcode`\#=12
\catcode`\$=12
\catcode`\%=12
\catcode`\^=12
\catcode`\_=12
\catcode`\.=12
\catcode`\?=12
\catcode`\!=12
\catcode`\&=12
\ttfamily}
\small
\item[]\textsl{-\ }val\ f\ =\ SymRel.input\ "";
\item[]\textsl{@\ }(0,\ a),\ (1,\ b),\ (2,\ c),\ (3,\ d),\ (4,\ e),\ (6,\ f)
\item[]\textsl{@\ }.
\item[]\textsl{val\ f\ =\ -\ :\ sym_rel}
\item[]\textsl{-\ }val\ x\ =\ Str.fromString\ "012364";
\item[]\textsl{val\ x\ =\ \symbol{'133}-,-,-,-,-,-\symbol{'135}\ :\ str}
\item[]\textsl{-\ }Str.toString(Str.renameAlphabet(x,\ f));
\item[]\textsl{val\ it\ =\ "abcdfe"\ :\ string}
\item[]\textsl{-\ }val\ y\ =\ Str.fromString\ "4271";
\item[]\textsl{val\ y\ =\ \symbol{'133}-,-,-,-\symbol{'135}\ :\ str}
\item[]\textsl{-\ }Str.toString(Str.renameAlphabet(y,\ f));
\item[]\textsl{invalid\ alphabet\ renaming\ for\ string}
\item[]
\item[]\textsl{uncaught\ exception\ Error}
\item[]\textsl{-\ }val\ xs\ =\ StrSet.input\ "";
\item[]\textsl{@\ }123,\ 0234,\ 16
\item[]\textsl{@\ }.
\item[]\textsl{val\ xs\ =\ -\ :\ str\ set}
\item[]\textsl{-\ }StrSet.output("",\ StrSet.renameAlphabet(xs,\ f));
\item[]\textsl{bf,\ bcd,\ acde}
\item[]\textsl{val\ it\ =\ ()\ :\ unit}
\item[]\textsl{-\ }val\ ys\ =\ StrSet.fromString\ "123,\ 0271";
\item[]\textsl{val\ ys\ =\ -\ :\ str\ set}
\item[]\textsl{-\ }StrSet.output("",\ StrSet.renameAlphabet(ys,\ f));
\item[]\textsl{invalid\ alphabet\ renaming\ for\ string\ set}
\item[]
\item[]\textsl{uncaught\ exception\ Error}
\end{list}


Let $X=\setof{(\alpha,f)}{\alpha\in\Reg\eqtxt{and}f\eqtxtl{is a
bijection from a set of symbols that}\eqtxtr{is a superset
of}\alphabet\,\alpha\eqtxtl{to some set of symbols}}$.
Then, the function $\renameAlphabet\in X\fun\Reg$ takes in a pair
$(\alpha,f)$ and
\index{regular expression!alphabet renaming}%
\index{alphabet renaming!regular expression}%
\index{renameAlphabet@$\renameAlphabet$}%
\index{regular expression!renameAlphabet@$\renameAlphabet$}%
returns the regular expression produced from $\alpha$ by renaming each
sub-tree of the form $a$, for $a\in\Sym$, to $f(a)$.
For example, $\renameAlphabet(\mathsf{012+12},
\mathsf{\{(0, 1), (1, 2), (2, 3)\}}) =
\mathsf{123+23}$.

\begin{theorem}
For all $\alpha\in\Reg$ and bijections $f$ from sets of symbols that
are supersets of $\alphabet\,\alpha$ to sets of symbols:
\begin{itemize}
\item $L(\renameAlphabet(\alpha,f)) = L(\alpha)^f$; and

\item $\alphabet(\renameAlphabet(\alpha,f)) =
\setof{f\,a}{a\in\alphabet\,\alpha}$.
\end{itemize}
\end{theorem}

For example, if $f=\mathsf{\{(0, 1), (1, 2), (2, 3)\}}$, then
\begin{align*}
L(\renameAlphabet(\mathsf{012+12}, f)) &=
L(\mathsf{012+12})^f=\{\mathsf{012,12}\}^f \\
&= \{\mathsf{123,23}\} .
\end{align*}

Let $X=\setof{(M,f)}{M\in\FA\eqtxt{and}f\eqtxtl{is a
bijection from a set of symbols that}\eqtxtr{is a superset
of}\alphabet\,M\eqtxtl{to some set of symbols}}$.
Then, the function $\renameAlphabet\in X\fun\FA$ takes in a pair $(M,f)$ and
\index{alphabet renaming!finite automaton}%
\index{finite automaton!alphabet renaming}%
\index{empty-string finite automaton!alphabet renaming}%
\index{nondeterministic finite automaton!alphabet renaming}%
\index{deterministic finite automaton!alphabet renaming}%
\index{renameAlphabet@$\renameAlphabet$}%
\index{finite automaton!renameAlphabet@$\renameAlphabet$}%
\index{empty-string finite automaton!renameAlphabet@$\renameAlphabet$}%
\index{nondeterministic finite automaton!renameAlphabet@$\renameAlphabet$}%
\index{deterministic finite automaton!renameAlphabet@$\renameAlphabet$}%
returns the FA produced from $M$ by renaming each symbol of each
label of each transition using $f$.
For example, if $M$ is the FA
\begin{center}
\input{chap-3.12-fig21.eepic}
\end{center}
and $f=\{\mathsf{(0,1), (1,2)}\}$, then
$\renameAlphabet(M,f)$ is the FA
\begin{center}
\input{chap-3.12-fig22.eepic}
\end{center}

\begin{theorem}
For all $M\in\FA$ and bijections $f$ from sets of symbols that
are supersets of $\alphabet\,M$ to sets of symbols:
\begin{itemize}
\item $L(\renameAlphabet(M,f)) = L(M)^f$;

\item $\alphabet(\renameAlphabet(M,f)) =
\setof{f\,a}{a\in\alphabet\,M}$;

\item if $M$ is an EFA, then $\renameAlphabet(M,f)$ is an EFA;

\item if $M$ is an NFA, then $\renameAlphabet(M,f)$ is an NFA; and

\item if $M$ is a DFA, then $\renameAlphabet(M,f)$ is a DFA.
\end{itemize}
\end{theorem}

\index{union!regular languages closed under}%
\index{regular language!closed under union}%
\index{concatenation!regular languages closed under}%
\index{regular language!closed under concatenation}%
\index{closure!regular languages closed under}%
\index{regular language!closed under closure}%
\index{intersection!regular languages closed under}%
\index{regular language!closed under intersection}%
\index{difference!regular languages closed under}%
\index{regular language!closed under difference}%
\index{reversal!regular languages closed under}%
\index{regular language!closed under reversal}%
\index{prefix-closure!regular languages closed under}%
\index{regular language!closed under prefix-closure}%
\index{suffix-closure!regular languages closed under}%
\index{regular language!closed under suffix-closure}%
\index{substring-closure!regular languages closed under}%
\index{regular language!closed under substring-closure}%
\index{alphabet-renaming!regular languages closed under}%
\index{regular language!closed under alphabet-renaming}%
\begin{theorem}
\label{ClosurePropTheorem}
Suppose $L,L_1,L_2\in\RegLan$.
Then:
\begin{enumerate}[\quad(1)]
\item $L_1\cup L_2\in\RegLan$;

\item $L_1L_2\in\RegLan$;

\item $L^*\in\RegLan$;

\item $L_1\cap L_2\in\RegLan$;

\item $L_1 - L_2\in\RegLan$;

\item $L^R\in\RegLan$;

\item $L^P\in\RegLan$;

\item $L^S\in\RegLan$;

\item $L^\SSop\in\RegLan$; and

\item $L^f\in\RegLan$, where $f$ is a bijection from a set of symbols
that is a superset of $\alphabet\,L$ to some set of symbols.
\end{enumerate}
\end{theorem}

\begin{proof}
Parts (1)--(10) hold because of the operations $\union$, $\concat$
and $\closure$ on FAs, the operation $\inter$ on EFAs, the
operation $\minus$ on DFAs, the operation $\rev$ on regular
expressions, the operations $\prefix$, $\suffix$ and $\substring$ on
EFAs, the operation $\renameAlphabet$ on regular expressions, and
Theorem~\ref{RegularEquiv}.
\end{proof}

From Theorem~\ref{ClosurePropTheorem}(5), it follows that the
regular languages are closed under complementation.
\index{complementation!regular languages closed under}%
\index{regular language!closed under complementation}%

The Forlan module \texttt{EFA} defines the function/algorithm
\begin{verbatim}
val inter : efa * efa -> efa
\end{verbatim}
which corresponds to $\inter$.  It is also inherited by the
modules \texttt{DFA} and \texttt{NFA}.

The Forlan module DFA defines the functions
\begin{verbatim}
val complement : dfa * sym set -> dfa
val minus      : dfa * dfa -> dfa
\end{verbatim}
which correspond to $\mycomplement$ and $\minus$.

Suppose the identifiers \texttt{efa1} and \texttt{efa2} of type \texttt{efa}
are bound to our example EFAs $M_1$ and $M_2$:
\begin{center}
\input{chap-3.12-fig14.eepic}
\end{center}
Then, we can construct $\inter(M_1,M_2)$ as follows:
\begin{list}{}
{\setlength{\leftmargin}{\leftmargini}
\setlength{\rightmargin}{0cm}
\setlength{\itemindent}{0cm}
\setlength{\listparindent}{0cm}
\setlength{\itemsep}{0cm}
\setlength{\parsep}{0cm}
\setlength{\labelsep}{0cm}
\setlength{\labelwidth}{0cm}
\catcode`\#=12
\catcode`\$=12
\catcode`\%=12
\catcode`\^=12
\catcode`\_=12
\catcode`\.=12
\catcode`\?=12
\catcode`\!=12
\catcode`\&=12
\ttfamily}
\small
\item[]\textsl{-\ }val\ efa\ =\ EFA.inter(efa1,\ efa2);
\item[]\textsl{val\ efa\ =\ -\ :\ efa}
\item[]\textsl{-\ }EFA.output("",\ efa);
\item[]\textsl{\symbol{'173}states\symbol{'175}\ <A,A>,\ <A,B>,\ <B,A>,\ <B,B>\ \symbol{'173}start\ state\symbol{'175}\ <A,A>}
\item[]\textsl{\symbol{'173}accepting\ states\symbol{'175}\ <B,B>}
\item[]\textsl{\symbol{'173}transitions\symbol{'175}}
\item[]\textsl{<A,A>,\ %\ ->\ <A,B>\ |\ <B,A>;\ <A,B>,\ %\ ->\ <B,B>;\ <A,B>,\ 0\ ->\ <A,B>;}
\item[]\textsl{<B,A>,\ %\ ->\ <B,B>;\ <B,A>,\ 1\ ->\ <B,A>}
\item[]\textsl{val\ it\ =\ ()\ :\ unit}
\end{list}

Thus \texttt{efa} is bound to the EFA
\begin{center}
\input{chap-3.12-fig15.eepic}
\end{center}

Suppose \texttt{dfa} is bound to our example DFA $M$
\begin{center}
\input{chap-3.12-fig16.eepic}
\end{center}
Then we can construct the DFA $\mycomplement(M,\{\twosf\})$
as follows:
\begin{list}{}
{\setlength{\leftmargin}{\leftmargini}
\setlength{\rightmargin}{0cm}
\setlength{\itemindent}{0cm}
\setlength{\listparindent}{0cm}
\setlength{\itemsep}{0cm}
\setlength{\parsep}{0cm}
\setlength{\labelsep}{0cm}
\setlength{\labelwidth}{0cm}
\catcode`\#=12
\catcode`\$=12
\catcode`\%=12
\catcode`\^=12
\catcode`\_=12
\catcode`\.=12
\catcode`\?=12
\catcode`\!=12
\catcode`\&=12
\ttfamily}
\small
\item[]\textsl{-\ }val\ dfa'\ =\ DFA.complement(dfa,\ SymSet.input\ "");
\item[]\textsl{@\ }2
\item[]\textsl{@\ }.
\item[]\textsl{val\ dfa'\ =\ -\ :\ dfa}
\item[]\textsl{-\ }DFA.output("",\ dfa');
\item[]\textsl{\symbol{'173}states\symbol{'175}\ A,\ B,\ C,\ <dead>\ \symbol{'173}start\ state\symbol{'175}\ A\ \symbol{'173}accepting\ states\symbol{'175}\ <dead>}
\item[]\textsl{\symbol{'173}transitions\symbol{'175}}
\item[]\textsl{A,\ 0\ ->\ B;\ A,\ 1\ ->\ A;\ A,\ 2\ ->\ <dead>;\ B,\ 0\ ->\ C;\ B,\ 1\ ->\ A;}
\item[]\textsl{B,\ 2\ ->\ <dead>;\ C,\ 0\ ->\ <dead>;\ C,\ 1\ ->\ A;\ C,\ 2\ ->\ <dead>;}
\item[]\textsl{<dead>,\ 0\ ->\ <dead>;\ <dead>,\ 1\ ->\ <dead>;\ <dead>,\ 2\ ->\ <dead>}
\item[]\textsl{val\ it\ =\ ()\ :\ unit}
\end{list}

Thus \texttt{dfa'} is bound to the DFA
\begin{center}
\input{chap-3.12-fig18.eepic}
\end{center}

Suppose the identifiers \texttt{efa1} and \texttt{efa2} of type \texttt{efa}
are bound to our example EFAs $M_1$ and $M_2$:
\begin{center}
\input{chap-3.12-fig14.eepic}
\end{center}
We can construct an EFA that accepts $L(M_1)-L(M_2)$ as follows:
\begin{list}{}
{\setlength{\leftmargin}{\leftmargini}
\setlength{\rightmargin}{0cm}
\setlength{\itemindent}{0cm}
\setlength{\listparindent}{0cm}
\setlength{\itemsep}{0cm}
\setlength{\parsep}{0cm}
\setlength{\labelsep}{0cm}
\setlength{\labelwidth}{0cm}
\catcode`\#=12
\catcode`\$=12
\catcode`\%=12
\catcode`\^=12
\catcode`\_=12
\catcode`\.=12
\catcode`\?=12
\catcode`\!=12
\catcode`\&=12
\ttfamily}
\small
\item[]\textsl{-\ }val\ dfa1\ =\ nfaToDFA(efaToNFA\ efa1);
\item[]\textsl{val\ dfa1\ =\ -\ :\ dfa}
\item[]\textsl{-\ }val\ dfa2\ =\ nfaToDFA(efaToNFA\ efa2);
\item[]\textsl{val\ dfa2\ =\ -\ :\ dfa}
\item[]\textsl{-\ }val\ dfa\ =\ DFA.minus(dfa1,\ dfa2);
\item[]\textsl{val\ dfa\ =\ -\ :\ dfa}
\item[]\textsl{-\ }val\ efa\ =\ injDFAToEFA\ dfa;
\item[]\textsl{val\ efa\ =\ -\ :\ efa}
\item[]\textsl{-\ }EFA.accepted\ efa\ (Str.input\ "");
\item[]\textsl{@\ }01
\item[]\textsl{@\ }.
\item[]\textsl{val\ it\ =\ true\ :\ bool}
\item[]\textsl{-\ }EFA.accepted\ efa\ (Str.input\ "");
\item[]\textsl{@\ }0
\item[]\textsl{@\ }.
\item[]\textsl{val\ it\ =\ false\ :\ bool}
\end{list}


Next, we see how we can carry out the reversal and alphabet-renaming
of regular expressions in Forlan.  The Forlan module \texttt{Reg}
defines the functions
\begin{verbatim}
val rev            : reg -> reg
val renameAlphabet : reg * sym_rel -> reg
\end{verbatim}
\index{Reg@\texttt{Reg}!rev@\texttt{rev}}%
\index{Reg@\texttt{Reg}!renameAlphabet@\texttt{renameAlphabet}}%
which correspond to $\rev$ and $\renameAlphabet$.
(\texttt{renameAlphabet} issues an error message and raises
an exception if its second argument isn't legal).
Here is an example of how these functions can be used:
\begin{list}{}
{\setlength{\leftmargin}{\leftmargini}
\setlength{\rightmargin}{0cm}
\setlength{\itemindent}{0cm}
\setlength{\listparindent}{0cm}
\setlength{\itemsep}{0cm}
\setlength{\parsep}{0cm}
\setlength{\labelsep}{0cm}
\setlength{\labelwidth}{0cm}
\catcode`\#=12
\catcode`\$=12
\catcode`\%=12
\catcode`\^=12
\catcode`\_=12
\catcode`\.=12
\catcode`\?=12
\catcode`\!=12
\catcode`\&=12
\ttfamily}
\small
\item[]\textsl{-\ }val\ reg\ =\ Reg.fromString\ "(012)\symbol{'052}(21)";
\item[]\textsl{val\ reg\ =\ -\ :\ reg}
\item[]\textsl{-\ }val\ rel\ =\ SymRel.fromString\ "(0,\ 1),\ (1,\ 2),\ (2,\ 3)";
\item[]\textsl{val\ rel\ =\ -\ :\ sym_rel}
\item[]\textsl{-\ }Reg.output("",\ Reg.rev\ reg);
\item[]\textsl{(12)((21)0)\symbol{'052}}
\item[]\textsl{val\ it\ =\ ()\ :\ unit}
\item[]\textsl{-\ }Reg.output("",\ Reg.renameAlphabet(reg,\ rel));
\item[]\textsl{(123)\symbol{'052}32}
\item[]\textsl{val\ it\ =\ ()\ :\ unit}
\end{list}


Next, we see how we can carry out the reversal of FAs and EFAs in
Forlan.  The Forlan module \texttt{FA} defines the function
\begin{verbatim}
val rev : fa -> fa
\end{verbatim}
\index{FA@\texttt{FA}!rev@\texttt{rev}}%
which corresponds to $\rev$.
It is also inherited by the module \texttt{EFA}.
\index{EFA@\texttt{EFA}!rev@\texttt{rev}}%
Here is an example of how this function can be used:
\begin{list}{}
{\setlength{\leftmargin}{\leftmargini}
\setlength{\rightmargin}{0cm}
\setlength{\itemindent}{0cm}
\setlength{\listparindent}{0cm}
\setlength{\itemsep}{0cm}
\setlength{\parsep}{0cm}
\setlength{\labelsep}{0cm}
\setlength{\labelwidth}{0cm}
\catcode`\#=12
\catcode`\$=12
\catcode`\%=12
\catcode`\^=12
\catcode`\_=12
\catcode`\.=12
\catcode`\?=12
\catcode`\!=12
\catcode`\&=12
\ttfamily}
\small
\item[]\textsl{-\ }val\ fa\ =\ FA.input\ "";
\item[]\textsl{@\ }\symbol{'173}states\symbol{'175}
\item[]\textsl{@\ }A,\ B,\ C
\item[]\textsl{@\ }\symbol{'173}start\ state\symbol{'175}
\item[]\textsl{@\ }A
\item[]\textsl{@\ }\symbol{'173}accepting\ states\symbol{'175}
\item[]\textsl{@\ }A,\ C
\item[]\textsl{@\ }\symbol{'173}transitions\symbol{'175}
\item[]\textsl{@\ }A,\ 01\ ->\ B;\ B,\ 23\ ->\ C;\ C,\ 45\ ->\ A
\item[]\textsl{@\ }.
\item[]\textsl{val\ fa\ =\ -\ :\ fa}
\item[]\textsl{-\ }val\ fa'\ =\ FA.rev\ fa;
\item[]\textsl{val\ fa'\ =\ -\ :\ fa}
\item[]\textsl{-\ }FA.output("",\ fa');
\item[]\textsl{\symbol{'173}states\symbol{'175}\ A,\ <A>,\ <B>,\ <C>\ \symbol{'173}start\ state\symbol{'175}\ A\ \symbol{'173}accepting\ states\symbol{'175}\ <A>}
\item[]\textsl{\symbol{'173}transitions\symbol{'175}}
\item[]\textsl{A,\ %\ ->\ <A>\ |\ <C>;\ <A>,\ 54\ ->\ <C>;\ <B>,\ 10\ ->\ <A>;\ <C>,\ 32\ ->\ <B>}
\item[]\textsl{val\ it\ =\ ()\ :\ unit}
\end{list}


Next, we see how we can carry out the prefix-closure of EFAs and NFAs in
Forlan.  The Forlan module \texttt{EFA} defines the function
\begin{verbatim}
val prefix : efa -> efa
\end{verbatim}
\index{EFA@\texttt{EFA}!prefix@\texttt{prefix}}%
which corresponds to $\prefix$.
It is also inherited by the module \texttt{NFA}.
\index{NFA@\texttt{NFA}!prefix@\texttt{prefix}}%
Here is an example of how one of these functions can be used:
\begin{list}{}
{\setlength{\leftmargin}{\leftmargini}
\setlength{\rightmargin}{0cm}
\setlength{\itemindent}{0cm}
\setlength{\listparindent}{0cm}
\setlength{\itemsep}{0cm}
\setlength{\parsep}{0cm}
\setlength{\labelsep}{0cm}
\setlength{\labelwidth}{0cm}
\catcode`\#=12
\catcode`\$=12
\catcode`\%=12
\catcode`\^=12
\catcode`\_=12
\catcode`\.=12
\catcode`\?=12
\catcode`\!=12
\catcode`\&=12
\ttfamily}
\small
\item[]\textsl{-\ }val\ nfa\ =\ NFA.input\ "";
\item[]\textsl{@\ }\symbol{'173}states\symbol{'175}
\item[]\textsl{@\ }A,\ B,\ C,\ D
\item[]\textsl{@\ }\symbol{'173}start\ state\symbol{'175}
\item[]\textsl{@\ }A
\item[]\textsl{@\ }\symbol{'173}accepting\ states\symbol{'175}
\item[]\textsl{@\ }A
\item[]\textsl{@\ }\symbol{'173}transitions\symbol{'175}
\item[]\textsl{@\ }A,\ 0\ ->\ B;\ B,\ 0\ ->\ C;\ C,\ 1\ ->\ A;\ C,\ 0\ ->\ D
\item[]\textsl{@\ }.
\item[]\textsl{val\ nfa\ =\ -\ :\ nfa}
\item[]\textsl{-\ }val\ nfa'\ =\ NFA.prefix\ nfa;
\item[]\textsl{val\ nfa'\ =\ -\ :\ nfa}
\item[]\textsl{-\ }NFA.output("",\ nfa');
\item[]\textsl{\symbol{'173}states\symbol{'175}\ A,\ B,\ C\ \symbol{'173}start\ state\symbol{'175}\ A\ \symbol{'173}accepting\ states\symbol{'175}\ A,\ B,\ C}
\item[]\textsl{\symbol{'173}transitions\symbol{'175}\ A,\ 0\ ->\ B;\ B,\ 0\ ->\ C;\ C,\ 1\ ->\ A}
\item[]\textsl{val\ it\ =\ ()\ :\ unit}
\end{list}


Finally, we see how we can carry out alphabet-renaming of finite
automata using Forlan.
The Forlan module \texttt{FA} defines the function
\begin{verbatim}
val renameAlphabet : FA * sym_rel -> FA
\end{verbatim}
\index{FA@\texttt{FA}!renameAlphabet@\texttt{renameAlphabet}}%
which corresponds $\renameAlphabet$ (it issues an error message and
raises an exception if its second argument isn't legal).
This function is also inherited by the modules \texttt{DFA},
\texttt{NFA} and \texttt{EFA}.
\index{NFA@\texttt{NFA}!renameAlphabet@\texttt{renameAlphabet}}%
\index{DFA@\texttt{DFA}!renameAlphabet@\texttt{renameAlphabet}}%
Here is an example of how one of these functions can be used:
\begin{list}{}
{\setlength{\leftmargin}{\leftmargini}
\setlength{\rightmargin}{0cm}
\setlength{\itemindent}{0cm}
\setlength{\listparindent}{0cm}
\setlength{\itemsep}{0cm}
\setlength{\parsep}{0cm}
\setlength{\labelsep}{0cm}
\setlength{\labelwidth}{0cm}
\catcode`\#=12
\catcode`\$=12
\catcode`\%=12
\catcode`\^=12
\catcode`\_=12
\catcode`\.=12
\catcode`\?=12
\catcode`\!=12
\catcode`\&=12
\ttfamily}
\small
\item[]\textsl{-\ }val\ dfa\ =\ DFA.input\ "";
\item[]\textsl{@\ }\symbol{'173}states\symbol{'175}
\item[]\textsl{@\ }A,\ B
\item[]\textsl{@\ }\symbol{'173}start\ state\symbol{'175}
\item[]\textsl{@\ }A
\item[]\textsl{@\ }\symbol{'173}accepting\ states\symbol{'175}
\item[]\textsl{@\ }A
\item[]\textsl{@\ }\symbol{'173}transitions\symbol{'175}
\item[]\textsl{@\ }A,\ 0\ ->\ B;\ B,\ 0\ ->\ A;
\item[]\textsl{@\ }A,\ 1\ ->\ A;\ B,\ 1\ ->\ B
\item[]\textsl{@\ }.
\item[]\textsl{val\ dfa\ =\ -\ :\ dfa}
\item[]\textsl{-\ }val\ rel\ =\ SymRel.fromString\ "(0,\ a),\ (1,\ b)";
\item[]\textsl{val\ rel\ =\ -\ :\ sym_rel}
\item[]\textsl{-\ }val\ dfa'\ =\ DFA.renameAlphabet(dfa,\ rel);
\item[]\textsl{val\ dfa'\ =\ -\ :\ dfa}
\item[]\textsl{-\ }DFA.output("",\ dfa');
\item[]\textsl{\symbol{'173}states\symbol{'175}\ A,\ B\ \symbol{'173}start\ state\symbol{'175}\ A\ \symbol{'173}accepting\ states\symbol{'175}\ A}
\item[]\textsl{\symbol{'173}transitions\symbol{'175}\ A,\ a\ ->\ B;\ A,\ b\ ->\ A;\ B,\ a\ ->\ A;\ B,\ b\ ->\ B}
\item[]\textsl{val\ it\ =\ ()\ :\ unit}
\end{list}


\subsection{Notes}

The material in this section is mostly standard, but is worked-out in
more detail than is common.  We have given an intersection algorithm
on EFAs, not just on DFAs.  By giving a complementation algorithm on
DFAs that allows us to partially control the alphabet of the resulting
DFA, we are able to give an algorithm for computing the difference of
DFAs that works even when the DFAs have different alphabets.

\index{regular languages!closure properties|)}%

%%% Local Variables: 
%%% mode: latex
%%% TeX-master: "book"
%%% End: 

\section{Equivalence-testing and Minimization of DFAs}
\label{EquivalenceTestingAndMinimizationOfDFAs}

In this section, we give algorithms for testing whether two DFAs are
equivalent, and for minimizing the alphabet size and number of states
of a DFA.  We also see how these functions can be used in Forlan.

\subsection{Testing the Equivalence of DFAs}

Suppose $M$ and $N$ are DFAs.  Our algorithm for checking whether they
are equivalent proceeds as follows.  First, it converts $M$ and $N$
into DFAs with identical alphabets.  Let
$\Sigma=\alphabet\,M\cup\alphabet\,N$, and define the DFAs $M'$ and
$N'$ by:
\begin{align*}
M' &= \determSimplify(M,\Sigma) , \eqtxt{and}\\
N' &= \determSimplify(N,\Sigma) .
\end{align*}
Since $\alphabet(L(M))\sub\alphabet\,M\sub\Sigma$, we have that
$\alphabet\,M'=\alphabet(L(M))\cup\Sigma=\Sigma$.  Similarly,
$\alphabet\,N'=\Sigma$.  Furthermore, $M'\approx M$ and $N'\approx N$,
so that it will suffice to determine whether $M'$ and $N'$ are
equivalent.

For example, if $M$ and $N$ are the DFAs
\begin{center}
\input{chap-3.13-fig1.eepic}
\end{center}
then $\Sigma=\{\zerosf,\onesf\}$, $M'=M$ and $N'=N$.

Next, the algorithm generates the least subset $X$ of $Q_{M'}\times Q_{N'}$
such that
\begin{itemize}
\item $(s_{M'},s_{N'})\in X$; and

\item for all $q\in Q_{M'}$, $r\in Q_{N'}$ and $a\in\Sigma$,
if $(q,r)\in X$, then $(\delta_{M'}(q,a),\delta_{N'}(r,a))\in X$.
\end{itemize}
With our example DFAs $M'$ and $N'$, we have that
\begin{itemize}
\item $(\Asf,\Asf)\in X$;

\item since $(\Asf,\Asf)\in X$, we have that {$(\Bsf,\Bsf)\in X$ and
    $(\Asf,\Csf)\in X$;}

\item since $(\Bsf,\Bsf)\in X$, we have that (again) $(\Asf,\Csf)\in
  X$ and (again) $(\Bsf,\Bsf)\in X$; and

\item since $(\Asf,\Csf)\in X$, we have that (again) $(\Bsf,\Bsf)\in
  X$ and (again) $(\Asf,\Asf)\in X$.
\end{itemize}

Back in the general case, we have the following lemmas.

\begin{lemma}
\label{EquivLem1}
For all $w\in\Sigma^*$, $(\delta_{M'}(s_{M'},w),\delta_{N'}(s_{N'},w))\in X$.
\end{lemma}

\begin{proof}
By left string induction on $w$.
\end{proof}

\begin{lemma}
\label{EquivLem2}
For all $q\in Q_{M'}$ and $r\in Q_{N'}$, if $(q,r)\in X$, then there
is a $w\in\Sigma^*$ such that $q=\delta_{M'}(s_{M'},w)$ and
$r=\delta_{N'}(s_{N'},w)$.
\end{lemma}

\begin{proof}
By induction on $X$.
\end{proof}

Finally, the algorithm checks that, for all $(q,r)\in X$,
\begin{gather*}
q\in A_{M'}\myiff r\in A_{N'} .
\end{gather*}
If this is true, it says that the machines are equivalent; otherwise
it says they are not equivalent.

We can easily prove the correctness of our algorithm:
\begin{itemize}
\item Suppose every pair $(q,r)\in X$ consists of two accepting states
  or of two non-accepting states.  Suppose, toward a contradiction,
  that $L(M')\neq L(N')$.  Then there is a string $w$ that is accepted
  by one of the machines but is not accepted by the other.  Since both
  machines have alphabet $\Sigma$, we have that $w\in\Sigma^*$.  Thus
  Lemma~\ref{EquivLem1} tells us that
  $(\delta_{M'}(s_{M'},w),\delta_{N'}(s_{N'},w))\in X$.  But one side
  of this pair is an accepting state and the other is a non-accepting
  one---contradiction.  Thus $L(M')=L(N')$.

\item Suppose we find a pair $(q,r)\in X$ such that one of $q$ and $r$
  is an accepting state but the other is not.  By
  Lemma~\ref{EquivLem2}, it will follow that there is a $w\in\Sigma^*$
  such that $q=\delta_{M'}(s_{M'},w)$ and $r=\delta_{N'}(s_{N'},w)$.
  Thus $w$ is accepted by one machine but not the other, so that
  $L(M')\neq L(N')$.
\end{itemize}

In the case of our example, we have that
$X=\{(\Asf,\Asf), (\Bsf,\Bsf), (\Asf,\Csf)\}$.
Since $(\Asf,\Asf)$ and $(\Asf,\Csf)$ are pairs of accepting states,
and $(\Bsf,\Bsf)$ is a pair of non-accepting states, it follows
that $L(M')=L(N')$.  Hence $L(M)=L(N)$.

By annotating each element $(q,r)\in X$ with a string $w$ such that
$q=\delta_{M'}(s_{M'},w)$ and $r=\delta_{N'}(s_{N'},w)$, instead of
just reporting that $M'$ and $N'$ are not equivalent, we can
explain why they are not equivalent,
\begin{itemize}
\item giving a string that is accepted by the first machine but not by
  the second; and/or

\item giving a string that is accepted by the second machine but not
  by the first.
\end{itemize}
We can even arrange for these strings to be of minimum length.
The Forlan implementation of our algorithm always produces minimum-length
counterexamples.

The Forlan module \texttt{DFA} defines the functions:
\begin{verbatim}
val relationship : dfa * dfa -> unit
val subset       : dfa * dfa -> bool
val equivalent   : dfa * dfa -> bool
\end{verbatim}
The function \texttt{relationship} figures out the relationship
between the languages accepted by two DFAs (are they equal, is one a
proper subset of the other, is neither a subset of the other), and
supplies minimum-length counterexamples to justify negative answers.
The function \texttt{subset} tests whether its first argument's
language is a subset of its second argument's language.
The function \texttt{equivalent} tests whether two DFAs are
equivalent.

Note that \texttt{subset} (when turned into a function of type
\texttt{reg~*~reg~->~bool}---see below) can be used in conjunction
with the local and global simplification algorithms of Section~3.3.

For example, suppose \texttt{dfa1} and \texttt{dfa2} of type \texttt{dfa} are
bound to our example DFAs $M$ and $N$, respectively:
\begin{center}
\input{chap-3.13-fig1.eepic}
\end{center}
We can verify that these machines are equivalent as follows:
\begin{list}{}
{\setlength{\leftmargin}{\leftmargini}
\setlength{\rightmargin}{0cm}
\setlength{\itemindent}{0cm}
\setlength{\listparindent}{0cm}
\setlength{\itemsep}{0cm}
\setlength{\parsep}{0cm}
\setlength{\labelsep}{0cm}
\setlength{\labelwidth}{0cm}
\catcode`\#=12
\catcode`\$=12
\catcode`\%=12
\catcode`\^=12
\catcode`\_=12
\catcode`\.=12
\catcode`\?=12
\catcode`\!=12
\catcode`\&=12
\ttfamily}
\small
\item[]\textsl{-\ }DFA.relationship(dfa1,\ dfa2);
\item[]\textsl{languages\ are\ equal}
\item[]\textsl{val\ it\ =\ ()\ :\ unit}
\end{list}


On the other hand, suppose that \texttt{dfa3} and \texttt{dfa4} of type
texttt{dfa} are bound to the DFAs:
\begin{center}
\input{chap-3.13-fig2.eepic}
\end{center}
We can find out why these machines are not equivalent as follows:
\begin{list}{}
{\setlength{\leftmargin}{\leftmargini}
\setlength{\rightmargin}{0cm}
\setlength{\itemindent}{0cm}
\setlength{\listparindent}{0cm}
\setlength{\itemsep}{0cm}
\setlength{\parsep}{0cm}
\setlength{\labelsep}{0cm}
\setlength{\labelwidth}{0cm}
\catcode`\#=12
\catcode`\$=12
\catcode`\%=12
\catcode`\^=12
\catcode`\_=12
\catcode`\.=12
\catcode`\?=12
\catcode`\!=12
\catcode`\&=12
\ttfamily}
\small
\item[]\textsl{-\ }DFA.relationship(dfa3,\ dfa4);
\item[]\textsl{neither\ language\ is\ a\ subset\ of\ the\ other\ language:\ "11"\ is\ in}
\item[]\textsl{first\ language\ but\ is\ not\ in\ second\ language;\ "110"\ is\ in\ second}
\item[]\textsl{language\ but\ is\ not\ in\ first\ language}
\item[]\textsl{val\ it\ =\ ()\ :\ unit}
\end{list}


We can find the relationship between the languages generated by regular
expressions \texttt{reg1} and \texttt{reg2} by:
\begin{itemize}
\item  converting \texttt{reg1} and \texttt{reg2} to DFAs
texttt{dfa1} and \texttt{dfa2}, and then

\item  running \texttt{DFA.relationship(dfa1, dfa2)} to find
the relationship between those DFAs.
\end{itemize}

Of course, we can define an ML/Forlan function that
carries out these actions:
\begin{list}{}
{\setlength{\leftmargin}{\leftmargini}
\setlength{\rightmargin}{0cm}
\setlength{\itemindent}{0cm}
\setlength{\listparindent}{0cm}
\setlength{\itemsep}{0cm}
\setlength{\parsep}{0cm}
\setlength{\labelsep}{0cm}
\setlength{\labelwidth}{0cm}
\catcode`\#=12
\catcode`\$=12
\catcode`\%=12
\catcode`\^=12
\catcode`\_=12
\catcode`\.=12
\catcode`\?=12
\catcode`\!=12
\catcode`\&=12
\ttfamily}
\small
\item[]\textsl{-\ }fun\ regToDFA\ reg\ =
\item[]\textsl{=\ }\ \ \ \ \ \ nfaToDFA(efaToNFA(faToEFA(regToFA\ reg)));
\item[]\textsl{val\ regToDFA\ =\ fn\ :\ reg\ ->\ dfa}
\item[]\textsl{-\ }fun\ relationshipReg(reg1,\ reg2)\ =
\item[]\textsl{=\ }\ \ \ \ \ \ DFA.relationship(regToDFA\ reg1,\ regToDFA\ reg2);
\item[]\textsl{val\ relationshipReg\ =\ fn\ :\ reg\ \symbol{'052}\ reg\ ->\ unit}
\end{list}


\subsection{Minimization of DFAs}

Now, we consider an algorithm for minimizing the sizes of the alphabet
and set of states of a DFA $M$.  First, the algorithm minimizes the
size of $M$'s alphabet, and makes the automaton be deterministically
simplified, by letting $M'=\determSimplify(M,\emptyset)$.  Thus
$M'\approx M$ and $\alphabet\,M'=\alphabet(L(M))$.

For example, if $M$ is the DFA
\begin{center}
\input{chap-3.13-fig3.eepic}
\end{center}
then $M'=M$.

Next, the algorithm generates the least subset $X$ of $Q_{M'}\times
Q_{M'}$ such that:
\begin{enumerate}[\quad(1)]
\item $A_{M'}\times(Q_{M'}-A_{M'})\sub X$;

\item $(Q_{M'}-A_{M'})\times A_{M'}\sub X$; and

\item for all $q,q',r,r'\in Q_{M'}$ and $a\in\alphabet\,M'$,
if $(q,r)\in X$, $(q',a, q)\in T_{M'}$ and $(r',a, r)\in T_{M'}$, then
$(q',r')\in X$.
\end{enumerate}
We read ``$(q,r)\in X$'' as ``$q$ and $r$ cannot be merged''.
The idea of (1) and (2) is that an accepting state can never be merged
with a non-accepting state.  And (3) says that if $q$ and $r$ can't
be merged, and we can get from $q'$ to $q$ by processing an $a$, and
from $r'$ to $r$ by processing an $a$, then
$q'$ and $r'$ also can't be merged---since if we merged $q'$ and $r'$,
there would have to be an $a$-transition from the merged state
to the merging of $q$ and $r$.

In the case of our example $M'$, (1) tells us to add the pairs
$(\Esf,\Asf)$, $(\Esf,\Bsf)$, $(\Esf,\Csf)$, $(\Esf,\Dsf)$,
$(\Fsf,\Asf)$, $(\Fsf,\Bsf)$, $(\Fsf,\Csf)$ and $(\Fsf,\Dsf)$ to $X$.
And, (2) tells us to add the pairs $(\Asf,\Esf)$, $(\Bsf,\Esf)$,
$(\Csf,\Esf)$, $(\Dsf,\Esf)$, $(\Asf,\Fsf)$, $(\Bsf,\Fsf)$,
$(\Csf,\Fsf)$ and $(\Dsf,\Fsf)$ to $X$.

Now we use rule (3) to compute the rest of $X$'s elements.  To begin
with, we must handle each pair that has already been added to $X$.
\begin{itemize}
\item Since there are no transitions leading into $\Asf$, no pairs can
  be added using $(\Esf,\Asf)$, $(\Asf,\Esf)$, $(\Fsf,\Asf)$ and
  $(\Asf,\Fsf)$.

\item Since there are no $\zerosf$-transitions leading into $\Esf$,
  and there are no $\onesf$-transitions leading into $\Bsf$, no pairs
  can be added using $(\Esf,\Bsf)$ and $(\Bsf,\Esf)$.

\item Since $(\Esf,\Csf),(\Csf,\Esf)\in X$ and $(\Bsf,\onesf,\Esf)$,
  $(\Dsf,\onesf,\Esf)$, $(\Fsf,\onesf,\Esf)$ and $(\Asf,\onesf,\Csf)$
  are the $\onesf$-transitions leading into $\Esf$ and $\Csf$, we add
  {$(\Bsf,\Asf)$ and $(\Asf,\Bsf)$, and $(\Dsf,\Asf)$ and
    $(\Asf,\Dsf)$} to $X$; {we would also have added $(\Fsf,\Asf)$ and
    $(\Asf,\Fsf)$ to $X$ if they hadn't been previously added.}  Since
  there are no $\zerosf$-transitions into $\Esf$, nothing can be added
  to $X$ using $(\Esf,\Csf)$ and $(\Csf,\Esf)$ and
  $\zerosf$-transitions.

\item Since $(\Esf,\Dsf),(\Dsf,\Esf)\in X$ and $(\Bsf,\onesf,\Esf)$,
  $(\Dsf,\onesf,\Esf)$, $(\Fsf,\onesf,\Esf)$ and $(\Csf,\onesf,\Dsf)$
  are the $\onesf$-transitions leading into $\Esf$ and $\Dsf$, we add
  {$(\Bsf,\Csf)$ and $(\Csf,\Bsf)$, and $(\Dsf,\Csf)$ and
    $(\Csf,\Dsf)$} to $X$; { we would also have added $(\Fsf,\Csf)$
    and $(\Csf,\Fsf)$ to $X$ if they hadn't been previously added.}
  Since there are no $\zerosf$-transitions into $\Esf$, nothing can be
  added to $X$ using $(\Esf,\Dsf)$ and $(\Dsf,\Esf)$ and
  $\zerosf$-transitions.

\item Since $(\Fsf,\Bsf),(\Bsf,\Fsf)\in X$ and $(\Esf,\zerosf,\Fsf)$,
  $(\Fsf,\zerosf,\Fsf)$, $(\Asf,\zerosf,\Bsf)$, and
  $(\Dsf,\zerosf,\Bsf)$ are the $\zerosf$-transitions leading into
  $\Fsf$ and $\Bsf$, we would have to add the following pairs to $X$,
  if they were not already present: $(\Esf,\Asf)$, $(\Asf,\Esf)$,
  $(\Esf,\Dsf)$, $(\Dsf,\Esf)$, $(\Fsf,\Asf)$, $(\Asf,\Fsf)$,
  $(\Fsf,\Dsf)$, $(\Dsf,\Fsf)$.  Since there are no
  $\onesf$-transitions leading into $\Bsf$, no pairs can be added
  using $(\Fsf,\Bsf)$ and $(\Bsf,\Fsf)$ and $\onesf$-transitions.

\item Since $(\Fsf,\Csf),(\Csf,\Fsf)\in X$ and $(\Esf,\onesf,\Fsf)$
  and $(\Asf,\onesf,\Csf)$ are the $\onesf$-transitions leading into
  $\Fsf$ and $\Csf$, we would have to add $(\Esf,\Asf)$ and
  $(\Asf,\Esf)$ to $X$ if these pairs weren't already present.  Since
  there are no $\zerosf$-transitions leading into $\Csf$, no pairs can
  be added using $(\Fsf,\Csf)$ and $(\Csf,\Fsf)$ and
  $\zerosf$-transitions.

\item Since $(\Fsf,\Dsf),(\Dsf,\Fsf)\in X$ and $(\Esf,\zerosf,\Fsf)$,
  $(\Fsf,\zerosf,\Fsf)$, $(\Bsf,\zerosf,\Dsf)$ and
  $(\Csf,\zerosf,\Dsf)$ are the $\zerosf$-transitions leading into
  $\Fsf$ and $\Dsf$, we would add $(\Esf,\Bsf)$, $(\Bsf,\Esf)$,
  $(\Esf,\Csf)$, $(\Csf,\Esf)$, $(\Fsf,\Bsf)$, $(\Bsf,\Fsf)$,
  $(\Fsf,\Csf)$, and $(\Csf,\Fsf)$ to $X$, if these pairs weren't
  already present.  Since $(\Fsf,\Dsf),(\Dsf,\Fsf)\in X$ and
  $(\Esf,\onesf,\Fsf)$ and $(\Csf,\onesf,\Dsf)$ are the
  $\onesf$-transitions leading into $\Fsf$ and $\Dsf$, we would add
  $(\Esf,\Csf)$ and $(\Csf,\Esf)$ to $X$, if these pairs weren't
  already in $X$.
\end{itemize}

We've now handled all of the elements of $X$ that were added using
rules~(1) and (2).  We must now handle the pairs that were
subsequently added: $(\Asf,\Bsf)$, $(\Bsf,\Asf)$, $(\Asf,\Dsf)$,
$(\Dsf,\Asf)$, $(\Bsf,\Csf)$, $(\Csf,\Bsf)$, $(\Csf,\Dsf)$,
$(\Dsf,\Csf)$.
\begin{itemize}
\item Since there are no transitions leading into $\Asf$, no pairs can
  be added using $(\Asf,\Bsf)$, $(\Bsf,\Asf)$, $(\Asf,\Dsf)$ and
  $(\Dsf,\Asf)$.

\item Since there are no $\onesf$-transitions leading into $\Bsf$, and
  there are no $\zerosf$-transitions leading into $\Csf$, no pairs can
  be added using $(\Bsf,\Csf)$ and $(\Csf,\Bsf)$.

\item Since $(\Csf,\Dsf), (\Dsf,\Csf)\in X$ and $(\Asf,\onesf,\Csf)$
  and $(\Csf,\onesf,\Dsf)$ are the $\onesf$-transitions leading into
  $\Csf$ and $\Dsf$, we add the pairs {$(\Asf,\Csf)$ and $(\Csf,\Asf)$
    to $X$}.  Since there are no $\zerosf$-transitions leading into
  $\Csf$, no pairs can be added to $X$ using $(\Csf,\Dsf)$ and
  $(\Dsf,\Csf)$ and $\zerosf$-transitions.
\end{itemize}

Now, we must handle the pairs that were added in the last
phase: $(\Asf,\Csf)$ and $(\Csf,\Asf)$.
\begin{itemize}
\item Since there are no transitions leading into $\Asf$, no pairs can
  be added using $(\Asf,\Csf)$ and $(\Csf,\Asf)$.
\end{itemize}

Since we have handled all the pairs we added to $X$, we are now done.
Here are the 26 elements of $X$: $(\Asf,\Bsf)$, $(\Asf,\Csf)$,
$(\Asf,\Dsf)$, $(\Asf,\Esf)$, $(\Asf,\Fsf)$, $(\Bsf,\Asf)$,
$(\Bsf,\Csf)$, $(\Bsf,\Esf)$, $(\Bsf,\Fsf)$, $(\Csf,\Asf)$,
$(\Csf,\Bsf)$, $(\Csf,\Dsf)$, $(\Csf,\Esf)$, $(\Csf,\Fsf)$,
$(\Dsf,\Asf)$, $(\Dsf,\Csf)$, $(\Dsf,\Esf)$, $(\Dsf,\Fsf)$,
$(\Esf,\Asf)$, $(\Esf,\Bsf)$, $(\Esf,\Csf)$, $(\Esf,\Dsf)$,
$(\Fsf,\Asf)$, $(\Fsf,\Bsf)$, $(\Fsf,\Csf)$, $(\Fsf,\Dsf)$.

Back in the general case, we have the following lemmas.

\begin{lemma}
\label{MinimizationLemma1}
For all $(q,r)\in X$, there is a $w\in(\alphabet\,M')^*$, such that
exactly one of $\delta_{M'}(q,w)$ and $\delta_{M'}(r,w)$ is in $A_{M'}$.
\end{lemma}

\begin{proof}
By induction on $X$.
\end{proof}

\begin{lemma}
\label{MinimizationLemma2}
For all $w\in(\alphabet\,M')^*$, for all $q,r\in Q_{M'}$, if
exactly one of $\delta_{M'}(q,w)$ and $\delta_{M'}(r,w)$ is
in $A_{M'}$, then $(q,r)\in X$.
\end{lemma}

\begin{proof}
By right string induction.
\end{proof}

Next, the algorithm lets the relation $Y=(Q_{M'}\times Q_{M'})-X$.  We
read ``$(q,r)\in Y$'' as ``$q$ and $r$ can be merged''.
Back with our example, we have that $Y$ is
\begin{gather*}
\{\mathsf{(A,A), (B,B), (C,C), (D,D), (E,E), (F,F)}\} \\
\cup \\
\{\mathsf{(B,D), (D,B), (F,E), (E,F)}\} .
\end{gather*}

\begin{lemma}
\label{MinimizationLemma3}
\begin{enumerate}[\quad (1)]
\item For all $q,r\in Q_{M'}$, $(q,r)\in Y$ iff, for all
  $w\in(\alphabet\,M')^*$, $\delta_{M'}(q,w)\in A_{M'}$ iff
  $\delta_{M'}(r,w)\in A_{M'}$.

\item For all $q,r\in Q_{M'}$, if $(q,r)\in Y$, then $q\in A_{M'}$ iff
  $r\in A_{M'}$.

\item For all $q,r\in Q_{M'}$ and $a\in\alphabet\,M'$, if $(q,r)\in
  Y$, then $(\delta_{M'}(q,a),\delta_{M'}(r,a))\in Y$.
\end{enumerate}
\end{lemma}

\begin{proof}
\begin{enumerate}[\quad(1)]
\item Follows using Lemmas~\ref{MinimizationLemma1} and
  \ref{MinimizationLemma2}.

\item Follows by Part~(1), when $w=\%$.

\item Follows by Part~(1).
\end{enumerate}
\end{proof}

The following lemma says that $Y$ is an \emph{equivalence relation on}
$Q_{M'}$.

\begin{lemma}
\label{MinimizationLemma4}
$Y$ is reflexive on $Q_{M'}$, symmetric and transitive.
\end{lemma}

\begin{proof}
Follows from Lemma~\ref{MinimizationLemma3}(1).
\end{proof}

In order to define the DFA $N$ that is the result of our minimization
algorithm, we need a bit more notation.  As in
Section~\ref{DeterministicFiniteAutomata}, we write $\overline{P}$ for
the result of coding a finite set of symbols $P$ as a symbol.  E.g.,
$\overline{\{\Bsf,\Asf\}} = \langle\Asf,\Bsf\rangle$.

If $q\in Q_{M'}$, we write $[q]$ for $\setof{p\in Q_{M'}}{(p,q)\in
  Y}$, which is called the \emph{equivalence class} of $q$.  Using
Lemma~\ref{MinimizationLemma4}, it is easy to show that, $q\in[q]$,
for all $q\in Q_{M'}$, and $[q]=[r]$ iff $(q,r)\in Y$, for all $q,r\in
Q_{M'}$.

If $P$ is a nonempty, finite set of symbols, then we write $\min\,P$
for the least element of $P$, according to our standard ordering on
symbols.

The algorithm lets $Z=\setof{[q]}{q\in Q_{M'}}$, which is finite since
$Q_{M'}$ is finite.
In the case of our example, $Z$ is
\begin{gather*}
\mathsf{\{\{A\}, \{B,D\}, \{C\}, \{E,F\}\}} .
\end{gather*}
Finally, the algorithm defines the DFA $N$ as follows:
\begin{itemize}
\item $Q_N = \setof{\overline{P}}{P\in Z}$;

\item $s_N = \overline{[s_{M'}]}$;

\item $A_N = \setof{\overline{P}}{P\in Z\eqtxt{and}\min\,P\in
 A_{M'}}$; and

\item $T_N = \setof{(\overline{P},a,
{\overline{[\delta_{M'}(\min\,P,a)]}})}%
{P\in Z\eqtxt{and}a\in\alphabet\,M'}$.
\end{itemize}
Then $N$ is a DFA with alphabet $\alphabet\,M'$ and, for all $P\in Z$
and $a\in\alphabet\,M'$, $\delta_N(\overline{P},a)=
\overline{[\delta_{M'}(\min\,P,a)]}$.

In the case of our example, we have that
\begin{itemize}
\item $Q_N = \{\mathsf{\langle A\rangle, \langle B,D\rangle,
\langle C\rangle, \langle E,F\rangle}\}$;

\item $s_N = \langle A\rangle$; and

\item $A_N = \{\langle\Esf,\Fsf\rangle\}$.
\end{itemize}
We compute the elements of $T_N$ as follows.
\begin{itemize}
\item Since $\{\Asf\}\in Z$ and $[\delta_{M'}(\Asf,\zerosf)]=
  [\Bsf]=\{\Bsf,\Dsf\}$, we have that
  $(\langle\Asf\rangle,\zerosf,\langle\Bsf,\Dsf\rangle)\in T_N$.

  Since $\{\Asf\}\in Z$ and $[\delta_{M'}(\Asf,\onesf)]=
  [\Csf]=\{\Csf\}$, we have that
  $(\langle\Asf\rangle,\onesf,\langle\Csf\rangle)\in T_N$.

\item Since $\{\Csf\}\in Z$ and $[\delta_{M'}(\Csf,\zerosf)]=
  [\Dsf]=\{\Bsf,\Dsf\}$, we have that
  $(\langle\Csf\rangle,\zerosf,\langle\Bsf,\Dsf\rangle)\in T_N$.

  Since $\{\Csf\}\in Z$ and $[\delta_{M'}(\Csf,\onesf)]=
  [\Dsf]=\{\Bsf,\Dsf\}$, we have that
  $(\langle\Csf\rangle,\onesf,\langle\Bsf,\Dsf\rangle)\in T_N$.

\item Since $\{\Bsf,\Dsf\}\in Z$ and $[\delta_{M'}(\Bsf,\zerosf)]=
  [\Dsf]=\{\Bsf,\Dsf\}$, we have that
  $(\langle\Bsf,\Dsf\rangle,\zerosf,\langle\Bsf,\Dsf\rangle)\in T_N$.

  Since $\{\Bsf,\Dsf\}\in Z$ and $[\delta_{M'}(\Bsf,\onesf)]=
  [\Esf]=\{\Esf,\Fsf\}$, we have that
  $(\langle\Bsf,\Dsf\rangle,\onesf,\langle\Esf,\Fsf\rangle)\in T_N$.

\item Since $\{\Esf,\Fsf\}\in Z$ and $[\delta_{M'}(\Esf,\zerosf)]=
  [\Fsf]=\{\Esf,\Fsf\}$, we have that
  $(\langle\Esf,\Fsf\rangle,\zerosf,\langle\Esf,\Fsf\rangle)\in T_N$.

  Since $\{\Esf,\Fsf\}\in Z$ and $[\delta_{M'}(\Esf,\onesf)]=
  [\Fsf]=\{\Esf,\Fsf\}$, we have that
  $(\langle\Esf,\Fsf\rangle,\onesf,\langle\Esf,\Fsf\rangle)\in T_N$.
\end{itemize}
Thus our DFA $N$ is:
\begin{center}
\input{chap-3.13-fig4.eepic}
\end{center}

Back in the general case, we have the following lemmas.

\begin{lemma}
\label{MinimizationLemma5}
\begin{enumerate}[\quad(1)]
\item For all $q\in Q_{M'}$, $\overline{[q]}\in A_N$ iff $q\in A_{M'}$.

\item For all $q\in Q_{M'}$ and $a\in\alphabet\,M'$,
  $\delta_N(\overline{[q]},a) = \overline{[\delta_{M'}(q,a)]}$.

\item For all $q\in Q_{M'}$ and $w\in(\alphabet\,M')^*$,
  $\delta_N(\overline{[q]},w) = \overline{[\delta_{M'}(q, w)]}$.

\item For all $w\in(\alphabet\,M')^*$, $\delta_N(s_N,w) =
  \overline{[\delta_{M'}(s_{M'}, w)]}$.
\end{enumerate}
\end{lemma}

\begin{proof}
(1) and (2) follow easily by Lemma~\ref{MinimizationLemma3}(2)--(3).
Part~(3) follows from Part~(2) by left string induction.  For
Part~(4), suppose $w\in(\alphabet\,M')^*$.  By Part~(3), we have
\begin{gather*}
\delta_N(s_N,w) =
\delta_N(\overline{[s_{M'}]}, w) =
\overline{[\delta_{M'}(s_{M'}, w)]} .
\end{gather*}
\end{proof}

\begin{lemma}
\label{MinimizationLemma6}
$L(N)=L(M')$.
\end{lemma}

\begin{proof}
Suppose $w\in L(N)$.  Then $w\in(\alphabet\,N)^*=(\alphabet\,M')^*$ and
$\delta_N(s_N,w)\in A_N$.
By Lemma~\ref{MinimizationLemma5}(4), we have that
\begin{gather*}
  \overline{[\delta_{M'}(s_{M'}, w)]} =
  \delta_N(s_N,w) \in A_N ,
\end{gather*}
so that $\delta_{M'}(s_{M'}, w)\in A_{M'}$, by
Lemma~\ref{MinimizationLemma5}(1).  Thus $w\in L(M')$.

Suppose $w\in L(M')$.  Then $w\in(\alphabet\,M')^*=(\alphabet\,N)^*$ and
$\delta_{M'}(s_{M'}, w)\in A_{M'}$.  By 
Lemma~\ref{MinimizationLemma5}(1) and (4), we have that
\begin{gather*}
  \delta_N(s_N,w) =
  \overline{[\delta_{M'}(s_{M'}, w)]}
  \in A_N .
\end{gather*}
Hence $w\in L(N)$.
\end{proof}

\begin{lemma}
\label{MinimizationLemma7}
$N$ is deterministically simplified.
\end{lemma}

\begin{proof}
To see that all elements of $Q_N$ are reachable, suppose $q\in Q_{M'}$.
Because $M'$ is deterministically simplified, there is a
$w\in(\alphabet\,M')^*$ such that $q=\delta_{M'}(s_{M'},w)$.  Thus
$\delta_N(s_N,w) = \overline{[\delta_{M'}(s_{M'},w)]} = \overline{[q]}$.

Next, we show that, for all $q\in Q_{M'}$, if $q$ is live, then
$\overline{[q]}$ is live.  Suppose $q\in Q_{M'}$ is live,
so there is a $w\in(\alphabet\,M')^*$ such that $\delta_{M'}(q,w)\in A_{M'}$.
Thus $\delta_N(\overline{[q]},w)=\overline{[\delta_{M'}(q,w)]}\in A_N$,
showing that $\overline{[q]}$ is live.

Thus, we have that, for all $q\in Q_{M'}$, if $\overline{[q]}$ is
dead, then $q$ is dead.  But, $M'$ has at most one dead state, and
thus we have that $N$ has at most one dead state.
\end{proof}

\begin{lemma}
\label{MinimizationLemma8}
Suppose $N'$ is a DFA such that $N'\approx M'$,
$\alphabet\,N'=\alphabet\,M'$ and $|Q_{N'}|\leq|Q_N|$.  Then $N'$ is
isomorphic to $N$.
\index{isomorphism!finite automaton}%
\index{finite automaton!isomorphism}%
\end{lemma}

\begin{proof}
We have that $L(N')=L(M')=L(N)$.  And the states of $M'$ and $N$ are
all reachable.  Let the relation $h$ between $Q_{N'}$ and $Q_N$ be
\begin{gather*}
  \setof{(\delta_{N'}(s_{N'},w),\delta_N(s_N,w))}{w\in(\alphabet\,M')^*}
  .
\end{gather*}
Since every state of $N$ is reachable, it follows that $\range\,h=
Q_N$.

To see that $h$ is a function, suppose $x,y\in(\alphabet\,M')^*$ and
$\delta_{N'}(s_{N'},x)=\delta_{N'}(s_{N'},y)$.  We must show that
$\delta_N(s_N,x)=\delta_N(s_N,y)$.  Since
$\delta_N(s_N,x)=\overline{[\delta_{M'}(s_{M'},x)]}$ and
$\delta_N(s_N,y)=\overline{[\delta_{M'}(s_{M'},y)]}$, it will suffice
to show that $(\delta_{M'}(s_{M'},x),\delta_{M'}(s_{M'},y))\in Y$.  By
Lemma~\ref{MinimizationLemma3}(1), it will suffice to show that,
$\delta_{M'}(\delta_{M'}(s_{M'},x), z)\in A_{M'}$ iff
$\delta_{M'}(\delta_{M'}(s_{M'},y), z)\in A_{M'}$, for all
$z\in(\alphabet\,M')^*$.  Suppose $z\in(\alphabet\,M')^*$.  We must
show that $\delta_{M'}(\delta_{M'}(s_{M'},x), z)\in A_{M'}$ iff
$\delta_{M'}(\delta_{M'}(s_{M'},y), z)\in A_{M'}$

We will show the ``only if'' direction, the other direction being
similar.  Suppose $\delta_{M'}(\delta_{M'}(s_{M'},x), z)\in A_{M'}$.
We must show that $\delta_{M'}(\delta_{M'}(s_{M'},y), z)\in A_{M'}$.
Because $\delta_{M'}(s_{M'},xz) = \delta_{M'}(\delta_{M'}(s_{M'},x),
z)\in A_{M'}$, we have that $xz\in L(M')=L(N')$.  Since $xz\in L(N')$
and $\delta_{N'}(s_{N'},x)=\delta_{N'}(s_{N'},y)$, we have that
\begin{align*}
  \delta_{N'}(s_{N'},yz) &= \delta_{N'}(\delta_{N'}(s_{N'},y), z) =
  \delta_{N'}(\delta_{N'}(s_{N'},x), z)\\
  &= \delta_{N'}(s_{N'},xz) \in A_{N'} ,
\end{align*}
so that $yz\in L(N')=L(M')$.  Hence
$\delta_{M'}(\delta_{M'}(s_{M'},y), z) = \delta_{M'}(s_{M'},yz) \in
A_{M'}$.

Because $h$ is a function and $\range\,h=Q_N$, we have that
$|Q_N|\leq |\domain\,f|\leq |Q_{N'}|$. But $|Q_{N'}|\leq |Q_N|$, and thus
$|Q_{N'}| = |Q_N|$. Because $Q_{N'}$ and $Q_N$ are finite, it
follows that $\domain\,h = Q_{N'}$ and $h$ is injective, so that $h$
is a bijection from $Q_{N'}$ to $Q_N$.  Thus, every state of $N'$ is
reachable, and, for all
$w\in(\alphabet\,M')^* = (\alphabet\,N)^* = (\alphabet\,N')^*$,
$h(\delta_{N'}(s_{N'},w))=\delta_N(s_N,w)$.  The remainder of the
proof that $h$ is an isomorphism from $N'$ to $N$ is fairly
straightforward.
\end{proof}

The following exercises formalizes the last sentence of
Lemma~\ref{MinimizationLemma8}.

\begin{exercise}
Suppose $N'$ and $N$ are equivalent DFAs with alphabet $\Sigma$, all
of whose states are reachable (for all $q\in Q_{N'}$, there is a
$w\in\Sigma^*$ such that $\delta_{N'}(s_{N'},w) = q$; and for all
$q\in Q_{N}$, there is a $w\in\Sigma^*$ such that
$\delta_{N}(s_{N},w) = q$). Suppose $h$ is a bijection from $Q_{N'}$
to $Q_N$ such that, for all $w\in\Sigma^*$,
$h(\delta_{N'}(s_{N'},w))=\delta_N(s_N,w)$. Prove that $h$ is
an isomorphism from $N'$ to $N$.
\end{exercise}

We define a function $\minimize\in\DFA\fun\DFA$ by:
$\minimize\,M$ is the result of running the above algorithm on
input $M$.

Putting the above results together, we have the following theorem:
\begin{theorem}
\label{Minimization}
For all $M\in\DFA$:
\begin{itemize}
\item $\minimize\,M\approx M$;

\item $\alphabet(\minimize\,M)=\alphabet(L(M))$;

\item $\minimize\,M$ is deterministically simplified; and

\item for all $N\in\DFA$, if $N\approx M$, $\alphabet\,N=\alphabet(L(M))$ and
$|Q_N|\leq|Q_{\minimize\,M}|$, then $N$ is isomorphic to $\minimize\,M$.
\end{itemize}
\end{theorem}

Thus
\begin{center}
\input{chap-3.13-fig4.eepic}
\end{center}
is, up to isomorphism, the only four-or-fewer state DFA with alphabet
$\{\zerosf,\onesf\}$ that is equivalent to $M$.

The Forlan module DFA includes the function
\begin{verbatim}
val minimize : dfa -> dfa
\end{verbatim}
for minimizing DFAs.

For example, if \texttt{dfa} of type \texttt{dfa} is bound to our example
DFA
\begin{center}
\input{chap-3.13-fig3.eepic}
\end{center}
then we can minimize the alphabet size and number of states of \texttt{dfa}
as follows.
\begin{list}{}
{\setlength{\leftmargin}{\leftmargini}
\setlength{\rightmargin}{0cm}
\setlength{\itemindent}{0cm}
\setlength{\listparindent}{0cm}
\setlength{\itemsep}{0cm}
\setlength{\parsep}{0cm}
\setlength{\labelsep}{0cm}
\setlength{\labelwidth}{0cm}
\catcode`\#=12
\catcode`\$=12
\catcode`\%=12
\catcode`\^=12
\catcode`\_=12
\catcode`\.=12
\catcode`\?=12
\catcode`\!=12
\catcode`\&=12
\ttfamily}
\small
\item[]\textsl{-\ }val\ dfa'\ =\ DFA.minimize\ dfa;
\item[]\textsl{val\ dfa'\ =\ -\ :\ dfa}
\item[]\textsl{-\ }DFA.output("",\ dfa');
\item[]\textsl{\symbol{'173}states\symbol{'175}\ <A>,\ <C>,\ <B,D>,\ <E,F>\ \symbol{'173}start\ state\symbol{'175}\ <A>}
\item[]\textsl{\symbol{'173}accepting\ states\symbol{'175}\ <E,F>}
\item[]\textsl{\symbol{'173}transitions\symbol{'175}}
\item[]\textsl{<A>,\ 0\ ->\ <B,D>;\ <A>,\ 1\ ->\ <C>;\ <C>,\ 0\ ->\ <B,D>;\ <C>,\ 1\ ->\ <B,D>;}
\item[]\textsl{<B,D>,\ 0\ ->\ <B,D>;\ <B,D>,\ 1\ ->\ <E,F>;\ <E,F>,\ 0\ ->\ <E,F>;}
\item[]\textsl{<E,F>,\ 1\ ->\ <E,F>}
\item[]\textsl{val\ it\ =\ ()\ :\ unit}
\end{list}


Finally, let's revisit an example from
Section~\ref{DeterministicFiniteAutomata}. Suppose \texttt{nfa} is the
$4$-state NFA
\begin{center}
\input{chap-3.13-fig5.eepic}
\end{center}
As we saw, our NFA-to-DFA conversion algorithm converts \texttt{nfa}
to a DFA \texttt{dfa} with $16$ states:

\begin{list}{}
{\setlength{\leftmargin}{\leftmargini}
\setlength{\rightmargin}{0cm}
\setlength{\itemindent}{0cm}
\setlength{\listparindent}{0cm}
\setlength{\itemsep}{0cm}
\setlength{\parsep}{0cm}
\setlength{\labelsep}{0cm}
\setlength{\labelwidth}{0cm}
\catcode`\#=12
\catcode`\$=12
\catcode`\%=12
\catcode`\^=12
\catcode`\_=12
\catcode`\.=12
\catcode`\?=12
\catcode`\!=12
\catcode`\&=12
\ttfamily}
\small
\item[]\textsl{-\ }val\ dfa\ =\ nfaToDFA\ nfa;
\item[]\textsl{val\ dfa\ =\ -\ :\ dfa}
\item[]\textsl{-\ }DFA.numStates\ dfa;
\item[]\textsl{val\ it\ =\ 16\ :\ int}
\end{list}


We can now use Forlan to verify that there is no DFA with fewer than
$16$ states that accepts the same language as \texttt{nfa}:

\begin{list}{}
{\setlength{\leftmargin}{\leftmargini}
\setlength{\rightmargin}{0cm}
\setlength{\itemindent}{0cm}
\setlength{\listparindent}{0cm}
\setlength{\itemsep}{0cm}
\setlength{\parsep}{0cm}
\setlength{\labelsep}{0cm}
\setlength{\labelwidth}{0cm}
\catcode`\#=12
\catcode`\$=12
\catcode`\%=12
\catcode`\^=12
\catcode`\_=12
\catcode`\.=12
\catcode`\?=12
\catcode`\!=12
\catcode`\&=12
\ttfamily}
\small
\item[]\textsl{-\ }val\ dfa'\ =\ DFA.minimize\ dfa;
\item[]\textsl{val\ dfa'\ =\ -\ :\ dfa}
\item[]\textsl{-\ }DFA.isomorphic(dfa',\ dfa);
\item[]\textsl{val\ it\ =\ true\ :\ bool}
\item[]\textsl{-\ }DFA.numStates\ dfa';
\item[]\textsl{val\ it\ =\ 16\ :\ int}
\end{list}


Thus we have an example where the smallest DFA accepting a language
requires exponentially more states than the smallest NFA accepting that
language. (This is true even though we haven't proven that an NFA
must have at least $4$ states to accept the same language as \texttt{nfa}.)

\subsection{Notes}

Our algorithm for testing whether two DFAs are equivalent can be found
in the literature, but I don't know of other textbooks that present
it.  As described above, a simple extension of the algorithm provides
counterexamples to justify non-equivalence.  The material on DFA
minimization is completely standard.

%%% Local Variables: 
%%% mode: latex
%%% TeX-master: "book"
%%% End: 

\section{The Pumping Lemma for Regular Languages}
\label{ThePumpingLemmaForRegularLanguages}

\index{pumping lemma!regular languages|(}%
\index{regular language!pumping lemma|(}%
\index{regular language!showing that languages are non-regular|(}%
In this section we consider techniques for showing that
particular languages are not regular.
Consider the language
\begin{gather*}
L=\setof{\zerosf^n\onesf^n}{n\in\nats} =
\mathsf{\{\%, 01, 0011, 000111,\,\ldots\}} .
\end{gather*}
Intuitively, an automaton would have to have infinitely many states to
accept $L$.  A finite automaton won't be able to keep track of how
many $\zerosf$'s it has seen so far, and thus won't be able to insist
that the correct number of $\onesf$'s follow.  We could turn the
preceding ideas into a direct proof that $L$ is not regular.  Instead,
we will first state a general result, called the Pumping Lemma for
regular languages, for proving that languages are non-regular.  Next,
we will show how the Pumping Lemma can be used to prove that $L$ is
non-regular.  Finally, we will prove the Pumping Lemma.

\begin{lemma}[Pumping Lemma for Regular Languages]
For all regular languages $L$, there is a $n\in\nats$ such that,
for all $z\in\Str$, if $z\in L$ and $|z|\geq n$, then
there are $u,v,w\in\Str$ such that $z=uvw$ and
\begin{enumerate}[\quad(1)]
\item $|uv|\leq n$;

\item $v\neq\%$; and

\item $uv^iw\in L$, for all $i\in\nats$.
\end{enumerate}
\end{lemma}

When we use the Pumping Lemma, we can imagine that we are interacting
with it.  We can give the Pumping Lemma a regular language $L$, and
the lemma will give us back a natural number $n$ such that the
property of the lemma holds.  We have no control over the value of
$n$; all we know is that the property of the lemma holds.  We can then
give the lemma a string $z$ that is in $L$ and has at least $n$
symbols. We'll have to choose $z$ as a function of $n$, because we
don't know what $n$ is. The lemma will then break $z$ up into parts
$u$, $v$ and $w$ in such way that (1)--(3) hold.  We have no control
over how $z$ is broken up into these parts; all we know is that
$z=uvw$ and (1)--(3) hold.  (1) says that $uv$ has no more than $n$
symbols.  (2) says that $v$ is nonempty.  And (3) says that, if we
``pump'' (duplicate) $v$ as many times as we like, the resulting
string will still be in $L$.

Before proving the Pumping Lemma, let's see how it can be used to
prove that $L=\setof{\zerosf^n\onesf^n}{n\in\nats}$ is non-regular.

\begin{proposition}
\label{NonRegularProp1}
$L$ is not regular.
\end{proposition}

\begin{proof}
Suppose, toward a contradiction, that $L$ is regular.
Thus there is an $n\in\nats$ with the property of the Pumping Lemma.
Suppose $z=\zerosf^n\onesf^n$.  Since $z\in L$ and
$|z|=2n\geq n$, it follows that there are $u,v,w\in\Str$ such that
$z=uvw$ and properties (1)--(3) of the lemma hold.  Since
$\zerosf^n\onesf^n=z=uvw$, (1) tells us that
there are $i,j,k\in\nats$ such that
\begin{gather*}
u=\zerosf^i,\quad
v=\zerosf^j,\quad
w=\zerosf^k\onesf^n,\quad
i+j+k=n .
\end{gather*}
By (2), we have that
$j\geq 1$, and thus that $i+k = n - j < n$.  By
(3), we have that
\begin{gather*}
\zerosf^{i+k}\onesf^n=\zerosf^i\zerosf^k\onesf^n=uw=u\%w=uv^0w\in L .
\end{gather*}
Thus $i+k=n$---contradiction.  Thus $L$ is not regular.
\end{proof}

In the preceding proof, we obtained a contradiction by pumping zero
times ($uv^0w$), but pumping two or more times ($uv^2w$, $\ldots$) would
also have worked.  For a case when pumping zeros times is
insufficient, consider $A=\setof{\zerosf^n\onesf^m}{n<m}$.  Given
$n\in\nats$ by the Pumping Lemma, we can let
$z=\zerosf^n\onesf^{n+1}$, obliging the lemma to split $z$ into $uvw$,
in such a way that (1)--(3) hold.  Hence $v$ will consist entirely of
$\zerosf$'s.  Pumping $v$ zero times won't take us outside of $A$.  On
the other hand $uv^2w$ will have at least as many $\zerosf$'s as
$\onesf$'s, giving us the needed contradiction.

Now, let's prove the Pumping Lemma.

\begin{proof}
Suppose $L$ is a regular
language.  Thus there is a NFA $M$ such that $L(M)=L$.  Let
$n=|Q_M|$.  Suppose $z\in\Str$, $z\in L$ and $|z|\geq n$.  Let
$m=|z|$.  Thus $1\leq n\leq|z|=m$.
Since $z\in L=L(M)$, there is a valid labeled path for $M$
\begin{gather*}
q_1\lparr{a_1}q_2\lparr{a_2}\cdots\,q_m\lparr{a_m}q_{m+1} ,
\end{gather*}
that is labeled by $z$ and where
$q_1=s_M$, $q_{m+1}\in A_M$ and $a_i\in\Sym$ for all $1\leq
i\leq m$.  Since $|Q_M|=n$,
not all of the states $q_1,\,\ldots,q_{n+1}$ are
distinct.  Thus, there are $1\leq i<j\leq n+1$ such that
$q_i=q_j$.

Hence, our path looks like:
\begin{gather*}
q_1\lparr{a_1}\cdots\,q_{i-1}\lparr{a_{i-1}}q_i
\lparr{a_i}\cdots\,q_{j-1}\lparr{a_{j-1}}q_j
\lparr{a_j}\cdots\,q_m\lparr{a_m}q_{m+1} .
\end{gather*}
Let
\begin{gather*}
u=a_1\cdots a_{i-1},\quad v=a_i\cdots a_{j-1},\quad
w=a_j\cdots a_m .
\end{gather*}
Then $z=uvw$.  Since $|uv|=j-1$ and $j\leq n+1$,
we have that $|uv|\leq n$.  Since $i<j$, we have that $i\leq j-1$,
and thus that $v\neq\%$.

Finally, since
\begin{gather*}
q_i\in\Delta(\{q_1\},u),\quad q_j\in\Delta(\{q_i\},v),\quad
q_{m+1}\in\Delta(\{q_j\},w)
\end{gather*}
and $q_i=q_j$, we have that
\begin{gather*}
q_j\in\Delta(\{q_1\},u),\quad q_j\in\Delta(\{q_j\},v),\quad
q_{m+1}\in\Delta(\{q_j\},w) .
\end{gather*}
Thus, we have that
$q_{m+1}\in\Delta(\{q_1\},uv^iw)$ for all $i\in\nats$.  But
$q_1=s_M$ and $q_{m+1}\in A_M$, and thus $uv^iw\in L(M)=L$ for all $i\in\nats$.
\end{proof}

Suppose $L'=\setof{w\in\mathsf{\{0,1\}^*}}{w\eqtxt{has an equal number of}
\zerosf\eqtxtr{'s and}\onesf\eqtxtn{'s}}$.
We could show that $L'$ is non-regular using the Pumping Lemma.
But we can also prove this result by using some of the closure
properties of Section~\ref{ClosurePropertiesOfRegularLanguages}
plus the fact that
$L=\setof{\zerosf^n\onesf^n}{n\in\nats}$ is non-regular.

Suppose, toward a contradiction, that $L'$ is regular.  It is easy to
see that $\{\zerosf\}$ and $\{\onesf\}$ are regular (e.g., they are
generated by the regular expressions $\zerosf$ and $\onesf$).  Thus,
by Theorem~\ref{ClosurePropTheorem}, we have that
$\{\zerosf\}^*\{\onesf\}^*$ is regular.  Hence, by
Theorem~\ref{ClosurePropTheorem} again, it follows that
$L=L'\cap\{\zerosf\}^*\{\onesf\}^*$ is regular---contradiction.  Thus
$L'$ is non-regular.

As a final example, let $X$ be the least subset of
$\{\mathsf{0,1}\}^*$ such that
\begin{enumerate}[\quad(1)]
\item $\%\in X$; and

\item For all $x,y\in X$, $\zerosf x\onesf y\in X$.
\end{enumerate}
Let's try to prove that $X$ is non-regular, using the Pumping Lemma.
We suppose, toward a contradiction, that $X$ is regular, and give it
to the Pumping Lemma, getting back the $n\in\nats$ with the property
of the lemma, where $X$ has been substituted for $L$.  But then, how
do we go about choosing the $z\in\Str$ such that $z\in X$ and $|z|\geq
n$?  We need to find a string expression $\myexp$ involving the
variable $n$, such that, for all $n\in\nats$, $\myexp\in X$
and $|\myexp|\geq n$.

Because $\%\in X$, we have that $\zerosf\onesf=\zerosf\%\onesf\%\in
X$.  Thus
$\zerosf\onesf\zerosf\onesf=\zerosf\%\onesf(\zerosf\onesf)\in X$.
Generalizing, we can easily prove that, for all $n\in\nats$,
$(\zerosf\onesf)^n\in X$.  Thus we could let $z=(\zerosf\onesf)^n$.
Unfortunately, this won't lead to the needed contradiction, since the
Pumping Lemma could break $z$ up into $u=\%$, $v=\zerosf\onesf$ and
$w=(\zerosf\onesf)^{n-1}$.

Trying again, we have that $\%\in X$, $\zerosf\onesf\in X$ and
$\zerosf(\zerosf\onesf)\onesf\%=\zerosf\zerosf\onesf\onesf\in X$.
Generalizing, it's easy to prove that, for all $n\in\nats$,
$\zerosf^n\onesf^n\in X$.  Thus, we can let $z=\zerosf^n\onesf^n$, so
that $z\in X$ and $|z|\geq n$.  We can then proceed as in the proof
that $\setof{\zerosf^n\onesf^n}{n\in\nats}$ is non-regular, getting to
the point where we learn that $\zerosf^{i+k}\onesf^n\in X$ and
$i+k<n$.  But an easy induction on $X$ suffices to show that, for all
$w\in X$, $w$ has an equal number of $\zerosf$'s and $\onesf$'s.
Hence $i+k=n$, giving us the needed contradiction.
\index{pumping lemma!regular languages|)}%
\index{regular language!pumping lemma|)}%

\subsection{Experimenting with the Pumping Lemma Using Forlan}

The Forlan module \texttt{LP}
(see Section~\ref{FiniteAutomataAndLabeledPaths}) defines a type and
several functions that implement the idea behind the pumping lemma:
\begin{verbatim}
type pumping_division = lp * lp * lp

val checkPumpingDivision       : pumping_division -> unit
val validPumpingDivision       : pumping_division -> bool
val strsOfValidPumpingDivision :
      pumping_division -> str * str * str
val pumpValidPumpingDivision   : pumping_division * int -> lp
val findValidPumpingDivision   : lp -> pumping_division
\end{verbatim}
A \emph{pumping division} is a triple $(\lp_1,\lp_2,\lp_3)$,
where $\lp_1,\lp_2,\lp_3\in\LP$.  We say that a pumping division
$(\lp_1,\lp_2,\lp_3)$ is \emph{valid} iff
\begin{itemize}
\item the end state of $\lp_1$ is equal to the start state of $\lp_2$;

\item the start state of $\lp_2$ is equal to the end state of $\lp_2$;

\item the end state of $\lp_2$ is equal to the start state of $\lp_3$; and

\item the label of $\lp_2$ is nonempty.
\end{itemize}
The function \texttt{checkPumpingDivision} checks whether a pumping
division is valid, silently returning \texttt{()}, if it is, and
issuing an error message explaining why it isn't, if it isn't.  The
function \texttt{validPumpingDivision} tests whether a pumping
division is valid.  The function \texttt{strsOfValidPumpingDivision}
returns the triple consisting of the labels of the three components of
a pumping division, in order.  It issues an error message if the pumping
division isn't valid.
The function \texttt{pumpValidPumpingDivision} expects a pair
$(\pd, n)$, where $\pd$ is a valid pumping division and $n\geq 0$.
It issues an error message if $\pd$ isn't valid, or $n$ is negative.
Otherwise, it returns
\begin{gather*}
\join(\hash{1}\,\pd,\join(\lp',\join(\hash{3}\,\pd))) ,
\end{gather*}
where $\lp'$ is the result of joining $\hash{2}\,\pd$ with itself $n$
times (the empty labeled path whose single state is $\hash{2}\,\pd$'s
start/end state, if $n=0$).  Finally, the function
\texttt{findValidPumpingDivision} takes in a labeled path $\lp$, and
tries to find a pumping division $(\lp_1, \lp_2, \lp_3)$ such that:
\begin{itemize}
\item $(\lp_1, \lp_2, \lp_3)$ is valid;

\item $\mathtt{pumpValidPumpingDivision}((\lp_1, \lp_2, \lp_3), 1) = \lp$;
  and

\item there is no repetition of states in the result of joining $\lp_1$ and
  the result of removing the last step of $\lp_2$.
\end{itemize}
\texttt{findValidPumpingDivision} issues an error message if
no such pumping division exists.

For example, suppose the DFA \texttt{dfa} is bound to the
DFA
\begin{center}
\input{chap-3.14-fig1.eepic}
\end{center}
Then we can proceed as follows:
\begin{list}{}
{\setlength{\leftmargin}{\leftmargini}
\setlength{\rightmargin}{0cm}
\setlength{\itemindent}{0cm}
\setlength{\listparindent}{0cm}
\setlength{\itemsep}{0cm}
\setlength{\parsep}{0cm}
\setlength{\labelsep}{0cm}
\setlength{\labelwidth}{0cm}
\catcode`\#=12
\catcode`\$=12
\catcode`\%=12
\catcode`\^=12
\catcode`\_=12
\catcode`\.=12
\catcode`\?=12
\catcode`\!=12
\catcode`\&=12
\ttfamily}
\small
\item[]\textsl{-\ }val\ lp\ =\ DFA.findAcceptingLP\ dfa\ (Str.input\ "");
\item[]\textsl{@\ }001010
\item[]\textsl{@\ }.
\item[]\textsl{val\ lp\ =\ -\ :\ lp}
\item[]\textsl{-\ }LP.output("",\ lp);
\item[]\textsl{A,\ 0\ =>\ B,\ 0\ =>\ C,\ 1\ =>\ A,\ 0\ =>\ B,\ 1\ =>\ B,\ 0\ =>\ C}
\item[]\textsl{val\ it\ =\ ()\ :\ unit}
\item[]\textsl{-\ }val\ pd\ =\ LP.findValidPumpingDivision\ lp;\ 
\item[]\textsl{val\ pd\ =\ (-,-,-)\ :\ LP.pumping_division}
\item[]\textsl{-\ }val\ (lp1,\ lp2,\ lp3)\ =\ pd;
\item[]\textsl{val\ lp1\ =\ -\ :\ lp}
\item[]\textsl{val\ lp2\ =\ -\ :\ lp}
\item[]\textsl{val\ lp3\ =\ -\ :\ lp}
\item[]\textsl{-\ }LP.output("",\ lp1);
\item[]\textsl{A}
\item[]\textsl{val\ it\ =\ ()\ :\ unit}
\item[]\textsl{-\ }LP.output("",\ lp2);
\item[]\textsl{A,\ 0\ =>\ B,\ 0\ =>\ C,\ 1\ =>\ A}
\item[]\textsl{val\ it\ =\ ()\ :\ unit}
\item[]\textsl{-\ }LP.output("",\ lp3);
\item[]\textsl{A,\ 0\ =>\ B,\ 1\ =>\ B,\ 0\ =>\ C}
\item[]\textsl{val\ it\ =\ ()\ :\ unit}
\item[]\textsl{-\ }val\ (u,\ v,\ w)\ =\ LP.strsOfValidPumpingDivision\ pd;
\item[]\textsl{val\ u\ =\ \symbol{'133}\symbol{'135}\ :\ str}
\item[]\textsl{val\ v\ =\ \symbol{'133}-,-,-\symbol{'135}\ :\ str}
\item[]\textsl{val\ w\ =\ \symbol{'133}-,-,-\symbol{'135}\ :\ str}
\item[]\textsl{-\ }(Str.toString\ u,\ Str.toString\ v,\ Str.toString\ v);
\item[]\textsl{val\ it\ =\ ("%","001","001")\ :\ string\ \symbol{'052}\ string\ \symbol{'052}\ string}
\item[]\textsl{-\ }val\ lp'\ =\ LP.pumpValidPumpingDivision(pd,\ 2);
\item[]\textsl{val\ lp'\ =\ -\ :\ lp}
\item[]\textsl{-\ }LP.output("",\ lp');
\item[]\textsl{A,\ 0\ =>\ B,\ 0\ =>\ C,\ 1\ =>\ A,\ 0\ =>\ B,\ 0\ =>\ C,\ 1\ =>\ A,\ 0\ =>\ B,\ 1\ =>}
\item[]\textsl{B,\ 0\ =>\ C}
\item[]\textsl{val\ it\ =\ ()\ :\ unit}
\item[]\textsl{-\ }Str.output("",\ LP.label\ lp');
\item[]\textsl{001001010}
\item[]\textsl{val\ it\ =\ ()\ :\ unit}
\end{list}

\index{regular language!showing that languages are non-regular|)}%

\subsection{Notes}

The Pumping Lemma is usually proved using a DFA accepting the given
regular language.  But because we have described the meaning of
automata via labeled paths, we can do the proof with an NFA, as it has
nothing to do with determinacy.  Forlan's support for experimenting
with the Pumping Lemma is novel.

%%% Local Variables: 
%%% mode: latex
%%% TeX-master: "book"
%%% End: 

%\section{Applications of Finite Automata and
Regular Expressions}
\label{ApplicationsOfFiniteAutomataAndRegularExpressions}

In this section we consider three applications of the material from
Chapter 3: searching for regular expressions in files; lexical
analysis; and the design of finite state systems.

\subsection{Representing Character Sets and Files}

Our first two applications involve processing files whose characters
come from some character set, e.g., the ASCII character set.  Although
not every character in a typical character set will be an element of
our set $\Sym$ of symbols, we can \emph{represent} all the characters
of a character set by elements of $\Sym$.  E.g., we might represent
the ASCII characters newline and space by the symbols $\newlinesym$
and $\spacesym$, respectively.

In the following two subsections, we will work with a mostly
unspecified alphabet $\Sigma$ representing some character set.  We
assume that the symbols $\mathsf{0}$--$\mathsf{9}$,
$\mathsf{a}$--$\mathsf{z}$, $\mathsf{A}$--$\mathsf{Z}$, $\spacesym$
and $\newlinesym$ are elements of $\Sigma$.  A \emph{line} is a string
consisting of an element of $(\Sigma-\{\newlinesym\})^*$; and, a
\emph{file} consists of the concatenation of some number of lines,
separated by occurrences of $\newlinesym$.  E.g.,
$\mathsf{0a\newlinesym\newlinesym 6}$ is a file with three lines
($\mathsf{0a}$, $\mathsf{\%}$ and $\mathsf{6}$), and
$\mathsf{\newlinesym}$ is a file with two lines, both consisting of
$\%$.

In what follows, we write:
\begin{itemize}
\item $\any$ for the regular expression $a_1+a_2+\cdots+a_n$, where
  $a_1,a_2,\,\ldots,a_n$ are all of the elements of $\Sigma$ except
  $\newlinesym$, listed in strictly ascending order;

\item $\letter$ for the regular expression
  \begin{gather*}
    \mathsf{a + b + \cdots + z + A + B + \cdots + Z};
  \end{gather*}

\item $\digit$ for the regular expression
  \begin{gather*}
    \mathsf{0 + 1 + \cdots + 9}.
  \end{gather*}
\end{itemize}

\subsection{Searching for Regular Expression in Files}

Given a file and a regular expression $\alpha$ whose alphabet is a
subset of $\Sigma-\{\newlinesym\}$, how can we find all lines of the
file with substrings in $L(\alpha)$?  (E.g., $\alpha$ might be
$\mathsf{\asf(\bsf+\csf)^*\asf}$; then we want to find all lines
containing two $\asf$'s, separated by some number of $\bsf$'s and
$\csf$'s.)

It will be sufficient to find all lines in the file that are elements
of $L(\beta)$, where $\beta = \any^*\,\alpha\,\any^*$.  To do this, we
can first translate $\beta$ to a DFA $M$ with alphabet
$\Sigma-\{\newlinesym\}$.  For each line $w$, we simply check whether
$\delta_M(s_M,w)\in A_M$, selecting the line if it is.  If the file is
short, however, it may be more efficient to convert $\beta$ to an FA
(or EFA or NFA) $N$, and use the algorithm from
Section~\ref{CheckingAcceptanceAndFindingAcceptingPaths} to find all
lines that are accepted by $N$.

\subsection{Lexical Analysis}

A lexical analyzer is the part of a compiler that groups the
characters of a program into lexical items or tokens.  The modern
approach to specifying a lexical analyzer for a programming language
uses regular expressions.  E.g., this is the approach taken by the
lexical analyzer generator Lex.

A lexical analyzer specification consists of a list of regular
expressions $\alpha_1,\alpha_2,\,\ldots,\alpha_n$, together with a
corresponding list of code fragments (in some programming language)
$\code_1,\code_2,\,\ldots,\code_n$ that process elements of
$\Sigma^*$.

For example, we might have
\begin{align*}
\alpha_1 &= \spacesym + \newlinesym , \\
\alpha_2 &= \letter\,(\letter + \digit)^* ,\\
\alpha_3 &= \digit\,\digit^*\,(\% + \Esf\,\digit\,\digit^*) , \\
\alpha_4 &= \any .
\end{align*}
The elements of $L(\alpha_1)$, $L(\alpha_2)$ and $L(\alpha_3)$ are
whitespace characters, identifiers and numerals, respectively.  The
code associated with $\alpha_4$ will probably indicate that an error
has occurred.

A lexical analyzer meets such a specification iff it behaves as
follows.  At each stage of processing its file, the lexical analyzer
should consume the \emph{longest} prefix of the remaining input that
is in the language generated by one of the regular expressions.  It
should then supply the prefix to the code associated with the earliest
regular expression whose language contains the prefix.  However, if
there is no such prefix, or if the prefix is $\%$, then the lexical
analyzer should indicate that an error has occurred.

What happens when we process the file $\mathsf{123Easy\spacesym
  1E2\newlinesym}$ using a lexical analyzer meeting our example
specification?
\begin{itemize}
\item The longest prefix of $\mathsf{123Easy\spacesym 1E2\newlinesym}$
  that is in one of our regular expressions is $\mathsf{123}$.  Since
  this prefix is only in $\alpha_3$, it is consumed from the input and
  supplied to $\code_3$.

\item The remaining input is now $\mathsf{Easy\spacesym
    1E2\newlinesym}$.  The longest prefix of the remaining input that
  is in one of our regular expressions is $\mathsf{Easy}$.  Since this
  prefix is only in $\alpha_2$, it is consumed and supplied to
  $\code_2$.

\item The remaining input is then $\mathsf{\spacesym 1E2\newlinesym}$.
  The longest prefix of the remaining input that is in one of our
  regular expressions is $\spacesym$.  Since this prefix is only in
  $\alpha_1$ and $\alpha_4$, we consume it from the input and supply
  it to the code associated with the earlier of these regular
  expressions: $\code_1$.

\item The remaining input is then $\mathsf{1E2\newlinesym}$.  The
  longest prefix of the remaining input that is in one of our regular
  expressions is $\mathsf{1E2}$.  Since this prefix is only in
  $\alpha_3$, we consume it from the input and supply it to $\code_3$.

\item The remaining input is then $\mathsf{\newlinesym}$.  The longest
  prefix of the remaining input that is in one of our regular
  expressions is $\newlinesym$.  Since this prefix is only in
  $\alpha_1$, we consume it from the input and supply it to the code
  associated with this expression: $\code_1$.

\item The remaining input is now empty, and so the lexical analyzer
  terminates.
\end{itemize}

Now, we consider a simple method for generating a lexical analyzer
that meets a given specification.  More sophisticated methods are
described in compilers courses.

First, we convert the regular expressions $\alpha_1,\,\ldots,\alpha_n$
into DFAs $M_1,\,\ldots,M_n$.  Next we determine which of the states
of the DFAs are dead/live.

Given its remaining input $x$, the lexical analyzer consumes the next
token from $x$ and supplies the token to the appropriate code, as
follows.  First, it initializes the following variables to error
values:
\begin{itemize}
\item a string variable $\acc$, which records the longest prefix of
  the prefix of $x$ that has been processed so far that is accepted by
  one of the DFAs;

\item an integer variable $\mach$, which records the smallest $i$ such
  that $\acc\in L(M_i)$;

\item a string variable $\aft$, consisting of the suffix of $x$ that
  one gets by removing $\acc$.
\end{itemize}

Then, the lexical analyzer enters its main loop, in which it processes
$x$, symbol by symbol, in \emph{each} of the DFAs, keeping track of
what symbols have been processed so far, and what symbols remain to be
processed.
\begin{itemize}
\item If, after processing a symbol, at least one of the DFAs is in an
  accepting state, then the lexical analyzer stores the string that
  has been processed so far in the variable $\acc$, stores the index
  of the first machine to accept this string in the integer variable
  $\mach$, and stores the remaining input in the string variable
  $\aft$.  If there is no remaining input, then the lexical analyzer
  supplies $\acc$ to code $\code_\mach$, and returns; otherwise it
  continues.

\item If, after processing a symbol, none of the DFAs are in accepting
  states, but at least one automaton is in a live state (so that,
  without knowing anything about the remaining input, it's possible
  that an automaton will again enter an accepting state), then the
  lexical analyzer leaves $\acc$, $\mach$ and $\aft$ unchanged.  If
  there is no remaining input, the lexical analyzer supplies $\acc$ to
  $\code_\mach$ (it signals an error if $\acc$ is still set to the
  error value), resets the remaining input to $\aft$, and returns;
  otherwise, it continues.

\item If, after processing a symbol, all of the automata are in dead
  states (and so could never enter accepting states again, no matter
  what the remaining input was), the lexical analyzer supplies string
  $\acc$ to code $\code_\mach$ (it signals an error if $\acc$ is still
  set to the error value), resets the remaining input to $\aft$, and
  returns.
\end{itemize}

Let's see what happens when the file $\mathsf{123Easy\newlinesym}$ is
processed by the lexical analyzer generated from our example
specification.
\begin{itemize}
\item After processing $\onesf$, $M_3$ and $M_4$ are in accepting
  states, and so the lexical analyzer sets $\acc$ to $\onesf$, $\mach$
  to $3$, and $\aft$ to $\mathsf{23Easy\newlinesym}$.  It then
  continues.

\item After processing $\twosf$, so that $\mathsf{12}$ has been
  processed so far, only $M_3$ is in an accepting state, and so the
  lexical analyzer sets $\acc$ to $\mathsf{12}$, $\mach$ to $3$, and
  $\aft$ to $\mathsf{3Easy\newlinesym}$.  It then continues.

\item After processing $\threesf$, so that $\mathsf{123}$ has been
  processed so far, only $M_3$ is in an accepting state, and so the
  lexical analyzer sets $\acc$ to $\mathsf{123}$, $\mach$ to $3$, and
  $\aft$ to $\mathsf{Easy\newlinesym}$.  It then continues.

\item After processing $\Esf$, so that $\mathsf{123E}$ has been
  processed so far, none of the DFAs are in accepting states, but
  $M_3$ is in a live state, since $\mathsf{123E}$ is a prefix of a
  string that is accepted by $M_3$.  Thus the lexical analyzer
  continues, but doesn't change $\acc$, $\mach$ or $\aft$.

\item After processing $\mathsf{a}$, so that $\mathsf{123Ea}$ has been
  processed so far, all of the machines are in dead states, since
  $\mathsf{123Ea}$ isn't a prefix of a string that is accepted by one
  of the DFAs.  Thus the lexical analyzer supplies $\acc=\mathsf{123}$
  to $\code_\mach=\code_3$, and sets the remaining input to
  $\aft=\mathsf{Easy\newlinesym}$.

\item In subsequent steps, the lexical analyzer extracts
  $\mathsf{Easy}$ from the remaining input, and supplies this string
  to code $\code_2$, and extracts $\newlinesym$ from the remaining
  input, and supplies this string to code $\code_1$.
\end{itemize}

\subsection{Design of Finite State Systems}

\index{finite automaton!design|(}%
\index{finite state system!design|(}%
Deterministic finite automata give us a means to efficiently---both in
terms of time and space---check membership in a regular language.  In
terms of time, a single left-to-right scan of the string is needed.
And we only need enough space to encode the DFA, and to keep track of
what state we are in at each point, as well as what part of the string
remains to be processed.  But if the string to be checked is
supplied, symbol-by-symbol, from our environment, we don't need to
store the string at all.

Consequently, DFAs may be easily and efficiently implemented in both
hardware and software.  One can design DFAs by hand, and test them
using Forlan.  But DFA minimization plus the operations on automata
and regular expressions of
Section~\ref{ClosurePropertiesOfRegularLanguages}, give us an
alternative---and very powerful---way of designing finite state
systems, which we will illustrate with two examples.

As the first example, suppose we wish to find a DFA $M$ such that
$L(M)=X$, where
\begin{displaymath}
X=\setof{w\in\mathsf{\{0,1\}^*}}{w\eqtxtl{has an even
    number of $\zerosf$'s or an odd number of $\onesf$'s}} .  
\end{displaymath}
First, we can note that $X=Y_1\cup Y_2$, where
\begin{align*}
Y_1 &= \setof{w\in\mathsf{\{0,1\}^*}}{w\eqtxtl{has an even
    number of $\zerosf$'s}}, \eqtxt{and} \\
Y_2 &= \setof{w\in\mathsf{\{0,1\}^*}}{w\eqtxtl{has an odd number of
    $\onesf$'s}}.
\end{align*}
Since we have a union operation on EFAs (Forlan
doesn't provide a union operation on DFAs), if we can find EFAs
accepting $Y_1$ and $Y_2$, we can combine them into a EFA that accepts
$X$.  Then we can convert this EFA to a DFA, and then minimize the
DFA.

Let $N_1$ and $N_2$ be the DFAs
\begin{center}
\input{chap-3.15-fig1.eepic}
\end{center}
It is easy to prove that $L(N_1)=Y_1$ and $L(N_2)=Y_2$.  Let $M$
be the DFA
\begin{gather*}
\renameStatesCanonically(\minimize(N)) ,
\end{gather*}
where $N$ is the DFA
\begin{gather*}
\nfaToDFA(\efaToNFA(\union(N_1,N_2))) .
\end{gather*}
\index{union@$\union$}%
\index{empty-string finite automaton!union@$\union$}%
\index{efaToNFA@$\efaToNFA$}%
\index{empty-string finite automaton!efaToNFA@$\efaToNFA$}%
\index{nondeterministic finite automaton!efaToNFA@$\efaToNFA$}%
\index{nfaToDFA@$\nfaToDFA$}%
\index{nondeterministic finite automaton!nfaToDFA@$\nfaToDFA$}%
\index{deterministic finite automaton!nfaToDFA@$\nfaToDFA$}%
\index{minimize@$\minimize$}%
\index{deterministic finite automaton!minimize@$\minimize$}%
\index{renameStatesCanonically@$\renameStatesCanonically$}%
\index{deterministic finite automaton!renameStatesCanonically@$\renameStatesCanonically$}%
Then
\begin{align*}
L(M) &= L(\renameStatesCanonically(\minimize\,N)) \\
&= L(\minimize\,N) \\
&= L(N) \\
&= L(\nfaToDFA(\efaToNFA(\union(N_1,N_2)))) \\
&= L(\efaToNFA(\union(N_1,N_2))) \\
&= L(\union(N_1,N_2)) \\
&= L(N_1)\cup L(N_2) \\
&= Y_1\cup Y_2 \\
&= X ,
\end{align*}
showing that $M$ is correct.

Suppose $M'$ is a DFA that accepts $X$.  Since
$M'\approx N$, we have that
$\minimize(N)$, and thus $M$, has no more states than
$M'$.  Thus $M$ has as few states as is possible.

But how do we figure out what the components of $M$ are, so that,
e.g., we can draw $M$?  In a simple case like this, we could apply the
definitions $\union$, $\efaToNFA$, $\nfaToDFA$,
$\minimize$ and $\renameStatesCanonically$,
and work out the answer.  But, for more complex examples, there would
be far too much detail involved for this to be a practical approach.

Instead, we can use Forlan to compute the answer.  Suppose
\texttt{dfa1} and \texttt{dfa2} of type \texttt{dfa} are
$N_1$ and $N_2$, respectively.  The we can proceed as follows:
\begin{list}{}
{\setlength{\leftmargin}{\leftmargini}
\setlength{\rightmargin}{0cm}
\setlength{\itemindent}{0cm}
\setlength{\listparindent}{0cm}
\setlength{\itemsep}{0cm}
\setlength{\parsep}{0cm}
\setlength{\labelsep}{0cm}
\setlength{\labelwidth}{0cm}
\catcode`\#=12
\catcode`\$=12
\catcode`\%=12
\catcode`\^=12
\catcode`\_=12
\catcode`\.=12
\catcode`\?=12
\catcode`\!=12
\catcode`\&=12
\ttfamily}
\small
\item[]\textsl{-\ }val\ efa\ =\ EFA.union(injDFAToEFA\ dfa1,\ injDFAToEFA\ dfa2);
\item[]\textsl{val\ efa\ =\ -\ :\ efa}
\item[]\textsl{-\ }val\ dfa'\ =\ nfaToDFA(efaToNFA\ efa);
\item[]\textsl{val\ dfa'\ =\ -\ :\ dfa}
\item[]\textsl{-\ }DFA.numStates\ dfa';
\item[]\textsl{val\ it\ =\ 5\ :\ int}
\item[]\textsl{-\ }val\ dfa\ =\ DFA.renameStatesCanonically(DFA.minimize\ dfa');
\item[]\textsl{val\ dfa\ =\ -\ :\ dfa}
\item[]\textsl{-\ }DFA.numStates\ dfa;
\item[]\textsl{val\ it\ =\ 4\ :\ int}
\item[]\textsl{-\ }DFA.output("",\ dfa);
\item[]\textsl{\symbol{'173}states\symbol{'175}\ A,\ B,\ C,\ D\ \symbol{'173}start\ state\symbol{'175}\ D\ \symbol{'173}accepting\ states\symbol{'175}\ A,\ C,\ D}
\item[]\textsl{\symbol{'173}transitions\symbol{'175}}
\item[]\textsl{A,\ 0\ ->\ C;\ A,\ 1\ ->\ D;\ B,\ 0\ ->\ D;\ B,\ 1\ ->\ C;\ C,\ 0\ ->\ A;\ C,\ 1\ ->\ B;}
\item[]\textsl{D,\ 0\ ->\ B;\ D,\ 1\ ->\ A}
\item[]\textsl{val\ it\ =\ ()\ :\ unit}
\end{list}

\index{injDFAToEFA@\texttt{injDFAToEFA}}%
\index{EFA@\texttt{EFA}!union@\texttt{union}}%
\index{efaToNFA@\texttt{efaToNFA}}%
\index{nfaToDFA@\texttt{nfaToDFA}}%
\index{DFA@\texttt{DFA}!inter@\texttt{inter}}%
\index{DFA@\texttt{DFA}!minimize@\texttt{minimize}}%
\index{DFA@\texttt{DFA}!renameStatesCanonically@\texttt{renameStatesCanonically}}%
Thus $M$ is:
\begin{center}
\input{chap-3.15-fig2.eepic}
\end{center}
Of course, this claim assumes that Forlan is correctly
implemented.

We conclude this subsection by considering a second, more involved
example of DFA design.  Given a string $w\in\{\mathsf{0,1}\}^*$, we
say that:
\begin{itemize}
\item $w$ \emph{stutters} iff $aa$ is a substring of $w$, for
\index{stuttering}%
\index{string!stuttering}%
some $a\in\{\mathsf{0,1}\}$;

\item $w$ is \emph{long} iff $|w|\geq 5$.
\end{itemize}
(We can generalize what follows to work two parameters: an alphabet
$\Sigma$ (instead of $\{\mathsf{0,1}\}$) and a length $n\in\nats$
(instead of $5$---the length above which we consider a string long).)
So, e.g., $\mathsf{1001}$ and $\mathsf{10110}$ both stutter, but
$\mathsf{01010}$ and $\mathsf{101}$ don't.  Saying that strings of
length $5$ or more are ``long'' is arbitrary; what follows can be
repeated with different choices of when strings are long.

Let the language $\AllLongStutter$ be
\begin{displaymath}
\setof{w\in\{\mathsf{0,1}\}^*}{\eqtxtr{for all substrings}v\eqtxt{of}w,
\eqtxt{if}v\eqtxt{is long, then}v\eqtxtl{stutters}} .
\end{displaymath}
In other words, a string of $\zerosf$'s and $\onesf$'s is
in $\AllLongStutter$ iff every long substring of this string
stutters.  Since every substring of $\mathsf{0010110}$ of
length five stutters, every long substring of this string stutters,
and thus the string is in $\AllLongStutter$.  On the other
hand, $\mathsf{0010100}$ is not in $\AllLongStutter$, because
$\mathsf{01010}$ is a long, non-stuttering substring of this
string.

Let's consider the problem of finding a DFA that accepts this
language.  One possibility is to reduce this problem to that of
finding a DFA that accepts the complement of $\AllLongStutter$.  Then
we'll be able to use our set difference operation on DFAs to build a
DFA that accepts $\AllLongStutter$, which we can then mimimize.
(We'll also need a DFA accepting $\{\mathsf{0,1}\}^*$.)  To form the
complement of $\AllLongStutter$, we negate the formula in
$\AllLongStutter$'s expression.  Let $\SomeLongNotStutter$ be the
language
\begin{displaymath}
\mtab{
\{\,w\in\{\mathsf{0,1}\}^* \mid \TS\eqtxtr{there is a substring}v\eqtxt{of}w
\eqtxtl{such that}\NL
v\eqtxtl{is long and doesn't stutter}\,\}.}
\end{displaymath}

\begin{lemma}
\label{Stutter1}
$\AllLongStutter = \{\mathsf{0,1}\}^*-\SomeLongNotStutter$.
\end{lemma}

\begin{proof}
Suppose $w\in\AllLongStutter$, so that
$w\in\{\mathsf{0,1}\}^*$ and, for all substrings $v$ of $w$,
if $v$ is long, then $v$ stutters.  Suppose, toward a contradiction,
that $w\in\SomeLongNotStutter$.  Then there is a substring $v$ of
$w$ such that $v$ is long and doesn't stutter---contradiction.
Thus $w\not\in\SomeLongNotStutter$, completing the proof
that $w\in \{\mathsf{0,1}\}^*-\SomeLongNotStutter$.

Suppose $w\in\{\mathsf{0,1}\}^*-\SomeLongNotStutter$, so that
$w\in\{\mathsf{0,1}\}^*$ and $w\not\in
\SomeLongNotStutter$.  To see that $w\in\AllLongStutter$,
suppose $v$ is a substring of $w$ and $v$ is long.  Suppose, toward a
contradiction, that $v$ doesn't stutter.  Then
$w\in\SomeLongNotStutter$---contradiction.  Hence $v$ stutters.
\end{proof}

Next, it's convenient to work bottom-up for a bit.  Let
\begin{align*}
\Long &= \setof{w\in\{\mathsf{0,1}\}^*}{w\eqtxtl{is long}} , \\
\Stutter &= \setof{w\in\{\mathsf{0,1}\}^*}{w\eqtxtl{stutters}} , \\
\NotStutter &= \setof{w\in\{\mathsf{0,1}\}^*}{w\eqtxtl{doesn't stutter}} ,
  \eqtxt{and} \\
\LongAndNotStutter &=
\setof{w\in\{\mathsf{0,1}\}^*}{w\eqtxtl{is long and doesn't stutter}} .
\end{align*}

The following lemma is easy to prove:

\begin{lemma}
\label{Stutter2}
\begin{enumerate}[\quad(1)]
\item $\NotStutter = \{\mathsf{0,1}\}^* - \Stutter$.

\item $\LongAndNotStutter = \Long\cap\NotStutter$.
\end{enumerate}
\end{lemma}

Clearly, we'll be able to find DFAs accepting $\Long$ and $\Stutter$,
respectively.  Thus, we'll be able to use our set difference operation
on DFAs to come up with a DFA that accepts $\NotStutter$.  Then,
we'll be able to use our intersection operation on DFAs to come
up with a DFA that accepts $\LongAndNotStutter$.

What remains is to find a way of converting $\LongAndNotStutter$
to $\SomeLongNotStutter$.  Clearly, the former language is
a subset of the latter one.  But the two languages are not equal,
since an element of the latter language may have the form
$xvy$, where $x,y\in\{\mathsf{0,1}\}^*$ and $v\in\LongAndNotStutter$.
This suggests the following lemma:

\begin{lemma}
\label{Stutter3}
$\SomeLongNotStutter =
\{\mathsf{0,1}\}^*\,\LongAndNotStutter\,\{\mathsf{0,1}\}^*$.
\end{lemma}

\begin{proof}
Suppose $w\in\SomeLongNotStutter$, so that $w\in\{\mathsf{0,1}\}^*$
and there is a substring $v$ of $w$ such that $v$ is long and doesn't
stutter.  Thus $v\in\LongAndNotStutter$, and $w=xvy$ for some
$x,y\in\{\mathsf{0,1}\}^*$.  Hence
$w=xvy\in\{\mathsf{0,1}\}^*\,\LongAndNotStutter\,\{\mathsf{0,1}\}^*$.

Suppose $w\in
\{\mathsf{0,1}\}^*\,\LongAndNotStutter\,\{\mathsf{0,1}\}^*$,
so that $w=xvy$ for some $x,y\in\{\mathsf{0,1}\}^*$ and
$v\in\LongAndNotStutter$.  Hence $v$ is long and doesn't stutter.
Thus $v$ is a long substring of $w$ that doesn't stutter,
showing that $w\in\SomeLongNotStutter$.
\end{proof}

Because of the preceding lemma, we can construct an EFA accepting
$\SomeLongNotStutter$ from a DFA accepting $\{\mathsf{0,1}\}^*$ and
our DFA accepting $\LongAndNotStutter$, using our concatenation
operation on EFAs.  (We haven't given a concatenation operation on
DFAs.)  We can then convert this EFA to a DFA.

Now, let's take the preceding ideas and turn them into reality.
First, we define functions $\regToEFA\in\Reg\fun\EFA$,
$\efaToDFA\in\EFA\fun\DFA$, $\regToDFA\in\Reg\fun\DFA$
and $\minAndRen\in\DFA\fun\DFA$ by:
\index{regToFA@$\regToFA$}%
\index{regular expression!regToFA@$\regToFA$}%
\index{finite automaton!regToFA@$\regToFA$}%
\index{faToEFA@$\faToEFA$}%
\index{finite automaton!faToEFA@$\faToEFA$}%
\index{empty-string finite automaton!faToEFA@$\faToEFA$}%
\index{efaToNFA@$\efaToNFA$}%
\index{empty-string finite automaton!efaToNFA@$\efaToNFA$}%
\index{nondeterministic finite automaton!efaToNFA@$\efaToNFA$}%
\index{nfaToDFA@$\nfaToDFA$}%
\index{nondeterministic finite automaton!nfaToDFA@$\nfaToDFA$}%
\index{deterministic finite automaton!nfaToDFA@$\nfaToDFA$}%
\index{renameStatesCanonically@$\renameStatesCanonically$}%
\index{deterministic finite automaton!renameStatesCanonically@$\renameStatesCanonically$}%
\index{minimize@$\minimize$}%
\index{deterministic finite automaton!minimize@$\minimize$}%
\index{regToEFA@$\regToEFA$}%
\index{regular expression!regToEFA@$\regToEFA$}%
\index{empty-string finite automaton!regToEFA@$\regToEFA$}%
\index{efaToDFA@$\efaToDFA$}%
\index{empty-string finite automaton!efaToDFA@$\efaToDFA$}%
\index{deterministic finite automaton!efaToDFA@$\efaToDFA$}%
\index{regToDFA@$\regToDFA$}%
\index{regular expression!regToDFA@$\regToDFA$}%
\index{deterministic finite automaton!regToDFA@$\regToDFA$}%
\index{minAndRen@$\minAndRen$}%
\index{deterministic finite automaton!minAndRen@$\minAndRen$}%
\begin{align*}
\regToEFA &= \faToEFA \circ \regToFA , \\
\efaToDFA &= \nfaToDFA \circ \efaToNFA , \\
\regToDFA &= \efaToDFA \circ \regToEFA , \eqtxt{and} \\
\minAndRen &= \renameStatesCanonically \circ \minimize .
\end{align*}

\begin{lemma}
\label{Stutter4}
\begin{enumerate}[\quad(1)]
\item For all $\alpha\in\Reg$, $L(\regToEFA(\alpha))=L(\alpha)$.

\item For all $M\in\EFA$, $L(\efaToDFA(M)) = L(M)$.

\item For all $\alpha\in\Reg$, $L(\regToDFA(\alpha)) = L(\alpha)$.

\item For all $M\in\DFA$, $L(\minAndRen(M)) = L(M)$ and,
for all $N\in\DFA$, if $L(N)=L(M)$, then $\minAndRen(M)$ has no more
states than $N$.
\end{enumerate}
\end{lemma}

\begin{proof}
We show the proof of Part~(4), the proofs of the other parts being
even easier.  Suppose $M\in\DFA$.  By Theorem~\ref{Minimization}(1), we
have that
\begin{align*}
L(\minAndRen(M)) &= L(\renameStatesCanonically(\minimize\,M)) \\
                 &= L(\minimize\,M) \\
                 &= L(M) .
\end{align*}
Suppose $N\in\DFA$ and $L(N)=L(M)$.  By Theorem~\ref{Minimization}(4),
$\minimize(M)$ has no more states than $N$.
Hence $\renameStatesCanonically(\minimize(M))$ has no more states than $N$,
showing that $\minAndRen(M)$ has no more states than $N$.
\end{proof}

Let the regular expression $\allStrReg$ be $\mathsf{(0+1)^*}$.
Clearly $L(\allStrReg)=\{\mathsf{0,1}\}^*$.  Let the DFA
$\allStrDFA$ be
\begin{displaymath}
\minAndRen(\regToDFA\,\allStrReg) .
\end{displaymath}

\begin{lemma}
\label{Stutter5}
$L(\allStrDFA)=\{\mathsf{0,1}\}^*$.
\end{lemma}

\begin{proof}
By Lemma~\ref{Stutter4}, we have that
\begin{align*}
L(\allStrDFA) &= L(\minAndRen(\regToDFA\,\allStrReg)) \\
              &= L(\regToDFA\,\allStrReg) \\
              &= L(\allStrReg) \\
              &= \{\mathsf{0,1}\}^* .
\end{align*}
\end{proof}

(Not surprisingly, $\allStrDFA$ will have a single state.)
Let the EFA $\allStrEFA$ be the DFA $\allStrDFA$.  Thus
$L(\allStrEFA)=\{\mathsf{0,1}\}^*$.

Let the regular expression $\longReg$ be
\begin{displaymath}
(\mathsf{0+1})^5(\mathsf{0+1})^* .
\end{displaymath}

\begin{lemma}
\label{Stutter6}

$L(\longReg) = \Long$.
\end{lemma}

\begin{proof}
Since $L(\longReg)=\{\mathsf{0,1}\}^5\{\mathsf{0,1}\}^*$,
it will suffice to show that 
$\{\mathsf{0,1}\}^5\{\mathsf{0,1}\}^* = \Long$.

Suppose $w\in\{\mathsf{0,1}\}^5\{\mathsf{0,1}\}^*$, so that
$w=xy$, for some $x\in\{\mathsf{0,1}\}^5$ and $y\in
\{\mathsf{0,1}\}^*$.  Thus $w=xy\in\{\mathsf{0,1}\}^*$
and $|w|\geq|x|=5$, showing that $w\in\Long$.

Suppose $w\in\Long$, so that $w\in\{\mathsf{0,1}\}^*$ and
$|w|\geq 5$.  Then $w=abcdex$, for some $a,b,c,d,e\in\{\mathsf{0,1}\}$
and $x\in\{\mathsf{0,1}\}^*$.  Hence $w=(abcde)x\in
\{\mathsf{0,1}\}^5\{\mathsf{0,1}\}^*$.
\end{proof}

Let the DFA $\longDFA$ be
\begin{displaymath}
\minAndRen(\regToDFA(\longReg)) .
\end{displaymath}
An easy calculation shows that $L(\longDFA) = \Long$.

Let $\stutterReg$ be the regular expression
\begin{displaymath}
\mathsf{(0+1)^*(00+11)(0 + 1)^*} .
\end{displaymath}

\begin{lemma}
\label{Stutter7}
$L(\stutterReg) = \Stutter$.
\end{lemma}

\begin{proof}
Since $L(\stutterReg)=
\{\mathsf{0,1}\}^*\{\mathsf{00,11}\}\{\mathsf{0,1}\}^*$,
it will suffice to show that
$\{\mathsf{0,1}\}^*\{\mathsf{00,11}\}\{\mathsf{0,1}\}^* = \Stutter$,
and this is easy.
\end{proof}

Let $\stutterDFA$ be the DFA
\begin{displaymath}
\minAndRen(\regToDFA(\stutterReg)) .
\end{displaymath}
An easy calculation shows that $L(\stutterDFA) = \Stutter$.
Let $\notStutterDFA$ be the DFA
\index{minus@$\minus$}%
\index{deterministic finite automaton!minus@$\minus$}%
\begin{displaymath}
\minAndRen(\minus(\allStrDFA, \stutterDFA)) .
\end{displaymath}

\begin{lemma}
\label{Stutter8}
$L(\notStutterDFA) = \NotStutter$.
\end{lemma}

\begin{proof}
Let $M$ be
\begin{displaymath}
\minAndRen(\minus(\allStrDFA, \stutterDFA)) .
\end{displaymath}
By Lemma~\ref{Stutter2}(1), we have that
\begin{align*}
L(\notStutterDFA) &= L(M) \\
&= L(\minus(\allStrDFA, \stutterDFA)) \\
&= L(\allStrDFA) - L(\stutterDFA) \\
&= \{\mathsf{0,1}\}^* - \Stutter \\
&= \NotStutter .
\end{align*}
\end{proof}

Let $\longAndNotStutterDFA$ be the DFA
\begin{displaymath}
\minAndRen(\inter(\longDFA, \notStutterDFA)) .
\end{displaymath}
\index{inter@$\inter$}%
\index{empty-string finite automaton!inter@$\inter$}%

\begin{lemma}
\label{Stutter9}
$L(\longAndNotStutterDFA) = \LongAndNotStutter$.
\end{lemma}

\begin{proof}
Let $M$ be
\begin{displaymath}
\minAndRen(\inter(\longDFA, \notStutterDFA)) .
\end{displaymath}
By Lemma~\ref{Stutter2}(2), we have that
\begin{align*}
L(\longAndNotStutterDFA) &= L(M) \\
&= L(\inter(\longDFA, \notStutterDFA)) \\
&= L(\longDFA)\cap L(\notStutterDFA) \\
&= \Long\cap\NotStutter \\
&= \LongAndNotStutter .
\end{align*}
\end{proof}

Because $\longAndNotStutterDFA$ is an EFA, we can simply let
the EFA $\longAndNotStutterEFA$ be
$\longAndNotStutterDFA$.  Thus we have that
$L(\longAndNotStutterEFA) = \LongAndNotStutter$.

Let $\someLongNotStutterEFA$ be the EFA
\begin{displaymath}
\mtab{
\renameStatesCanonically(\concat(\TS\allStrEFA, \NL
                                 \concat(\TS\longAndNotStutterEFA, \NL
                                             \allStrEFA))).}
\end{displaymath}
\index{renameStatesCanonically@$\renameStatesCanonically$}%
\index{empty-string finite automaton!renameStatesCanonically@$\renameStatesCanonically$}%
\index{concat@$\concat$}%
\index{empty-string finite automaton!concat@$\concat$}%

\begin{lemma}
\label{Stutter10}
$L(\someLongNotStutterEFA) = \SomeLongNotStutter$.
\end{lemma}

\begin{proof}
We have that
\begin{align*}
L(\someLongNotStutterEFA) &= L(\renameStatesCanonically\,M) \\
&= L(M) ,
\end{align*}
where $M$ is
\begin{displaymath}
\concat(\allStrEFA,\concat(\longAndNotStutterEFA,\allStrEFA)) .
\end{displaymath}
And, by Lemma~\ref{Stutter3}, we have that
\begin{align*}
L(M) &= L(\allStrEFA)\,L(\longAndNotStutterEFA)\,L(\allStrEFA) \\
     &= \{\mathsf{0,1}\}^*\,\LongAndNotStutter\,\{\mathsf{0,1}\}^* \\
     &= \SomeLongNotStutter .
\end{align*}
\end{proof}

Let $\someLongNotStutterDFA$ be the DFA
\begin{displaymath}
\minAndRen(\efaToDFA\,\someLongNotStutterEFA) .
\end{displaymath}

\begin{lemma}
\label{Stutter11}
$L(\someLongNotStutterDFA) = \SomeLongNotStutter$.
\end{lemma}

\begin{proof}
Follows by an easy calculation.
\end{proof}

Finally, let $\allLongStutterDFA$ be the DFA
\begin{displaymath}
\minAndRen(\minus(\allStrDFA, \someLongNotStutterDFA)) .
\end{displaymath}

\begin{lemma}
\label{Stutter12}
$L(\allLongStutterDFA) = \AllLongStutter$ and,
for all $N\in\DFA$, if $L(N)=\AllLongStutter$,
then $\allLongStutterDFA$ has no more states than $N$.
\end{lemma}

\begin{proof}
We have that
\begin{displaymath}
L(\allLongStutterDFA) = L(\minAndRen(M)) = L(M) ,
\end{displaymath}
where $M$ is
\begin{displaymath}
\minus(\allStrDFA, \someLongNotStutterDFA) .
\end{displaymath}
Then, by Lemma~\ref{Stutter1}, we have that
\begin{align*}
L(M) &= L(\allStrDFA) - L(\someLongNotStutterDFA) \\
&= \{\mathsf{0,1}\}^* - \SomeLongNotStutter \\
&= \AllLongStutter .
\end{align*}
Suppose $N\in\DFA$ and $L(N)=\AllLongStutter$.  Thus $L(N)=L(M)$, so
that $\allLongStutterDFA$ has no more states than $N$, by
Lemma~\ref{Stutter4}(4).
\end{proof}

The preceding lemma tells us that the DFA $\allLongStutterDFA$
is correct and has as few states as is possible.
To find out what it looks like, though, we'll have to use
Forlan.  First we put the text
\verbatiminput{stutter.sml}

\index{faToEFA@\texttt{faToEFA}}%
\index{regToFA@\texttt{regToFA}}%
\index{nfaToDFA@\texttt{nfaToDFA}}%
\index{efaToNFA@\texttt{efaToNFA}}%
\index{DFA@\texttt{DFA}!renameStatesCanonically@\texttt{renameStatesCanonically}}%
\index{DFA@\texttt{DFA}!minimize@\texttt{minimize}}%
\index{Reg@\texttt{Reg}!fromString@\texttt{fromString}}%
\index{injDFAToEFA@\texttt{injDFAToEFA}}%
\index{Reg@\texttt{Reg}!concat@\texttt{concat}}%
\index{Reg@\texttt{Reg}!power@\texttt{power}}%
\index{Reg@\texttt{Reg}!fromString@\texttt{fromString}}%
\index{DFA@\texttt{DFA}!minus@\texttt{minus}}%
\index{DFA@\texttt{DFA}!inter@\texttt{inter}}%
\index{EFA@\texttt{EFA}!concat@\texttt{concat}}%
\index{EFA@\texttt{EFA}!renameStatesCanonically@\texttt{renameStatesCanonically}}%
in the file \texttt{stutter.sml}.  Then, we proceed as follows
\begin{list}{}
{\setlength{\leftmargin}{\leftmargini}
\setlength{\rightmargin}{0cm}
\setlength{\itemindent}{0cm}
\setlength{\listparindent}{0cm}
\setlength{\itemsep}{0cm}
\setlength{\parsep}{0cm}
\setlength{\labelsep}{0cm}
\setlength{\labelwidth}{0cm}
\catcode`\#=12
\catcode`\$=12
\catcode`\%=12
\catcode`\^=12
\catcode`\_=12
\catcode`\.=12
\catcode`\?=12
\catcode`\!=12
\catcode`\&=12
\ttfamily}
\small
\item[]\textsl{-\ }use\ "stutter.sml";
\item[]\textsl{\symbol{'133}opening\ stutter.sml\symbol{'135}}
\item[]\textsl{val\ regToEFA\ =\ fn\ :\ reg\ ->\ efa}
\item[]\textsl{val\ efaToDFA\ =\ fn\ :\ efa\ ->\ dfa}
\item[]\textsl{val\ regToDFA\ =\ fn\ :\ reg\ ->\ dfa}
\item[]\textsl{val\ minAndRen\ =\ fn\ :\ dfa\ ->\ dfa}
\item[]\textsl{val\ allStrReg\ =\ -\ :\ reg}
\item[]\textsl{val\ allStrDFA\ =\ -\ :\ dfa}
\item[]\textsl{val\ allStrEFA\ =\ -\ :\ efa}
\item[]\textsl{val\ longReg\ =\ -\ :\ reg}
\item[]\textsl{val\ longDFA\ =\ -\ :\ dfa}
\item[]\textsl{val\ stutterReg\ =\ -\ :\ reg}
\item[]\textsl{val\ stutterDFA\ =\ -\ :\ dfa}
\item[]\textsl{val\ notStutterDFA\ =\ -\ :\ dfa}
\item[]\textsl{val\ longAndNotStutterDFA\ =\ -\ :\ dfa}
\item[]\textsl{val\ longAndNotStutterEFA\ =\ -\ :\ efa}
\item[]\textsl{val\ someLongNotStutterEFA'\ =\ -\ :\ efa}
\item[]\textsl{val\ someLongNotStutterEFA\ =\ -\ :\ efa}
\item[]\textsl{val\ someLongNotStutterDFA\ =\ -\ :\ dfa}
\item[]\textsl{val\ allLongStutterDFA\ =\ -\ :\ dfa}
\item[]\textsl{val\ it\ =\ ()\ :\ unit}
\item[]\textsl{-\ }DFA.output("",\ allLongStutterDFA);
\item[]\textsl{\symbol{'173}states\symbol{'175}\ A,\ B,\ C,\ D,\ E,\ F,\ G,\ H,\ I,\ J\ \symbol{'173}start\ state\symbol{'175}\ A}
\item[]\textsl{\symbol{'173}accepting\ states\symbol{'175}\ A,\ B,\ C,\ D,\ E,\ F,\ G,\ H,\ I}
\item[]\textsl{\symbol{'173}transitions\symbol{'175}}
\item[]\textsl{A,\ 0\ ->\ B;\ A,\ 1\ ->\ C;\ B,\ 0\ ->\ B;\ B,\ 1\ ->\ E;\ C,\ 0\ ->\ D;\ C,\ 1\ ->\ C;}
\item[]\textsl{D,\ 0\ ->\ B;\ D,\ 1\ ->\ G;\ E,\ 0\ ->\ F;\ E,\ 1\ ->\ C;\ F,\ 0\ ->\ B;\ F,\ 1\ ->\ I;}
\item[]\textsl{G,\ 0\ ->\ H;\ G,\ 1\ ->\ C;\ H,\ 0\ ->\ B;\ H,\ 1\ ->\ J;\ I,\ 0\ ->\ J;\ I,\ 1\ ->\ C;}
\item[]\textsl{J,\ 0\ ->\ J;\ J,\ 1\ ->\ J}
\item[]\textsl{val\ it\ =\ ()\ :\ unit}
\end{list}

Thus, $\allLongStutterDFA$ is the DFA of Figure~\ref{StuttDFASynExamp}.
\begin{figure}
\begin{center}
\input{chap-3.15-fig3.eepic}
\end{center}
\caption{DFA Accepting $\AllLongStutter$}
\label{StuttDFASynExamp}
\end{figure}

\begin{exercise}
\label{StutGen}
Generalize the preceding example to work with two parameters: an alphabet
$\Sigma$ (instead of $\{\mathsf{0,1}\}$) and a length $n\in\nats$ (instead
of $5$---the length above which we consider a string long).
\end{exercise}

\begin{exercise}
Define $\diff\in\{\mathsf{0,1}\}^*\fun\ints$ by:
\index{diff@$\diff$}%
\index{string!diff@$\diff$}%
\index{difference function}%
\index{string!difference function}%
for all $w\in\{\mathsf{0,1}\}^*$,
\begin{displaymath}
\diff\,w =
\eqtxtr{the number of $\mathsf{1}$'s in}w -
\eqtxtr{the number of $\mathsf{0}$'s in}w .
\end{displaymath}
Let
$\AllPrefixGood = \setof{w\in\{\mathsf{0,1}\}^*}{\eqtxtr{for all
    prefixes} v\eqtxt{of}w, |\diff\,v|\leq 2}$.  Suppose we have a DFA
$\allPrefixGoodDFA$, and we have already proved (see
Exercise~\ref{AllGoodDFACorrLem})
$L(\allPrefixGoodDFA) = \AllPrefixGood$. Let
$\AllSubstringGood = \setof{w\in\{\mathsf{0,1}\}^*}{\eqtxtr{for all
    substrings} v\eqtxt{of}w, |\diff\,v|\leq 2}$.  Show how we can
transform $\allPrefixGoodDFA$ into a minimized DFA
$\allSubstringGoodDFA$ such that
$L(\allSubstringGoodDFA) = \AllSubstringGood$. Prove that your
transformation is correct. Finally, realize your transformation in
Forlan, and draw the resulting DFA.
\end{exercise}

\index{finite state system!synthesis|)}%
\index{finite automaton!synthesis|)}%

\subsection{Notes}

Our treatment of searching for regular expressions in text file is
standard, as is that of lexical analysis.  But our approach to
designing finite state systems depends upon having access to a
toolset, like Forlan, that is embedded in a programming language and
implements our algorithms for manipulating finite automata and regular
expressions.

%%% Local Variables: 
%%% mode: latex
%%% TeX-master: "book"
%%% End: 


%%% Local Variables: 
%%% mode: latex
%%% TeX-master: "book"
%%% End: 
