\section{The Pumping Lemma for Context-free Languages}
\label{ThePumpingLemmaForContextFreeLanguages}

Consider the language $L =
\setof{\zerosf^n\onesf^n\twosf^n}{n\in\nats}$.  Is $L$ context-free,
i.e., is there a grammar that generates $L$?  Although it's easy to
find a grammar that keeps the $\zerosf$'s and $\onesf$'s matched, or
one that keeps the $\onesf$'s and $\twosf$'s matched, or one that
keeps the $\zerosf$'s and $\twosf$'s matched, it seems that there is
no way to keep all three symbols matched simultaneously.

In this section, we will study the pumping lemma for context-free
languages, which can be used to show that many languages are not
context-free.  We will use the pumping lemma to prove that $L$ is not
context-free, and then we will prove the lemma.  Building on this
result, we'll be able to show that the context-free languages are not
closed under intersection, complementation or set-difference.

\subsection{Statement, Application and Proof of Pumping Lemma}

\begin{lemma}[Pumping Lemma for Context Free Languages]
For all context-free languages $L$, there is a
$n\in\nats-\{0\}$ such that, for all $z\in\Str$, if $z\in L$
and $|z|\geq n$, then there are $u,v,w,x,y\in\Str$ such
that $z=uvwxy$ and
\begin{enumerate}[\quad(1)]
\item $|vwx|\leq n$;

\item $vx\neq\%$; and

\item $uv^iwx^iy\in L$, for all $i\in\nats$.
\end{enumerate}
\end{lemma}

Before proving the pumping lemma, let's see how it can be used to show
that $L=\setof{\zerosf^n\onesf^n\twosf^n}{n\in\nats}$ is not
context-free.  Suppose, toward a contradiction that $L$ is
context-free.  Thus there is an $n\in\nats-\{0\}$ with the property of the
lemma.  Let $z={\zerosf^n\onesf^n\twosf^n}$.
Since $z\in L$ and $|z|=3n\geq n$, we have that there are
$u,v,w,x,y\in\Str$ such that $z=uvwxy$ and
\begin{enumerate}[\quad(1)]
\item $|vwx|\leq n$;

\item $vx\neq\%$; and

\item $uv^iwx^iy\in L$, for all $i\in\nats$.
\end{enumerate}
Since $\zerosf^n\onesf^n\twosf^n=z=uvwxy$, (1) tells us that
$vwx$ doesn't contain both a $\zerosf$ and a $\twosf$.
Thus, $vwx$ has no $\zerosf$'s or $vwx$ has no $\twosf$'s, so that
there are two cases to consider.

Suppose $vwx$ has no $\zerosf$'s.  Thus $vx$ has no $\zerosf$'s.
By (2), we have that $vx$ contains a $\onesf$ or a $\twosf$.
Thus $uwy$:
\begin{itemize}
\item has $n$ $\zerosf$'s;

\item has less than $n$ $\onesf$'s or has less than $n$ $\twosf$'s.
\end{itemize}
But (3) tells us that $uwy=uv^0wx^0y\in L$, so that $uwy$ has an equal
number of $\zerosf$'s, $\onesf$'s and $\twosf$'s---contradiction.
The case where $vwx$ has no $\twosf$'s is similar.
Since we obtained a contradiction in both cases, we have an overall
contradiction.  Thus $L$ is not context-free.

When we prove the pumping lemma for context-free languages, we will
make use of a fact about grammars in Chomsky Normal Form.
Suppose $G$ is a grammar in CNF and that $w\in(\alphabet\,G)^*$
is the yield of a valid parse tree $\pt$ for $G$ whose root label
is a variable.
For instance, if $G$ is the grammar with variable $\Asf$ and
productions $\Asf\fun\Asf\Asf$ and $\Asf\fun\zerosf$, then
$w$ could be $\mathsf{0000}$ and $\pt$ could be the following
tree of height $3$:
\begin{center}
\input{chap-4.10-fig1.eepic}
\end{center}

Generalizing from this example, we can see that if $\pt$ has height
$3$, $|w|$ will never be greater than $4=2^2=2^{3-1}$.  Generalizing
still more, we can prove that, for all parse trees $\pt$, for all
strings $w$, if $w$ is the yield of $\pt$, then $|w|\leq { 2^{k-1}}$.
This can be proved by induction on $\pt$.

\begin{proof}
Suppose $L$ is a context-free language.  By the results of the
preceding section, there is a grammar $G$ in Chomsky Normal Form such
that $L(G)=L-\{\%\}$.  Let $k=|Q_G|$ and $n= 2^k$.  Thus
$n\in\nats-\{0\}$. Suppose $z\in Str$, $z\in L$ and $|z|\geq n$.
Since $n\geq 1$, we have that $z\neq\%$.  Thus $z\in L-\{\%\}=L(G)$,
so that there is a parse tree $\pt$ such that $\pt$ is valid for $G$,
$\rootLabel\,\pt=s_G$ and $\yield\,\pt=z$.  By our fact about CNF
grammars, we have that the height of $\pt$ is at least $k+1$.  (If
$\pt$'s height were only $k$, then $|z|\leq 2^{k-1}<n$, which is
impossible.)

The rest of the proof can be visualized using the
diagram
\begin{center}
\input{chap-4.10-fig2.eepic}
\end{center}

Let $\pat$ be a valid path for $\pt$ whose length is equal to the
height of $\pt$.  Thus the length of $\pat$ is at least $k+1$, so that
the path visits at least $k+1$ variables, with the consequence that at
least one variable must be visited twice.  Working from the last
variable visited upwards, we look for the first repetition of
variables.  Suppose $q$ is this repeated variable, and let $\pat'$
and $\pat''$ be the initial parts of $\pat$ that take us
to the upper and lower occurrences of $q$, respectively.

Let $\pt'$ and $\pt''$ be the subtrees of $\pt$ at positions
$\pat'$ and $\pat''$, i.e., the positions of the upper and lower
occurrences of $q$, respectively.
Consider the tree formed from $\pt$ by replacing the subtree
at position $\pat'$ by $q$.  This tree has yield
$uqy$, for unique strings $u$ and $y$.

Consider the tree formed from $\pt'$ by replacing the subtree
$\pt''$ by $q$.  More precisely, form the path $\pat'''$ 
by removing $\pat'$ from the beginning of $\pat''$.
Then replace the subtree of $\pt'$ at position $\pat'''$ by
$q$.  This tree has yield $vqx$, for unique strings $v$ and $x$.

Furthermore, since $|\pat|$ is the height of $\pt$, the
length of the path formed by removing $\pat'$ from $\pat$ will be the
height of $\pt'$.  But we know that this length is at most $k+1$,
because, when working upwards through the variables visited by $\pat$,
we stopped as soon as we found a repetition of variables.  Thus the
height of $\pt'$ is at most $k+1$.

Let $w$ be the yield of $\pt''$.  Thus $vwx$ is the yield of $\pt'$,
so that $z=uvwxy$ is the yield of $\pt$.  Because the height of $\pt'$
is at most $k+1$, our fact about valid parse trees of grammars in CNF,
tells us that $|vwx|\leq 2^{(k+1)-1} = 2^k=n$, showing that Part~(1)
holds.

Because $G$ is in CNF, $\pt'$, which has $q$ as its root label, has
two children.  The child whose root node isn't visited by $\pat'''$
will have a non-empty yield, and this yield will be a prefix of $v$,
if this child is the left child, and will be a suffix of $x$, if this
child is the right child.  Thus $vx\neq\%$, showing that Part~(2)
holds.

It remains to show Part~(3), i.e., that $uv^iwx^iy\in L(G)\sub L$, for
all $i\in\nats$.  We define a valid parse tree $\pt_i$ for $G$, with root
label $q$ and yield $v^iwx^i$, by recursion on $i\in\nats$.  We let $\pt_0$ be
$\pt''$.  Then, if $i\in\nats$, we form $\pt_{i+1}$ from $\pt'$ by
replacing the subtree at position $\pat'''$ by $\pt_i$.

Suppose $i\in\nats$.  Then the parse tree formed from $\pt$ by
replacing the subtree at position $\pat'$ by $\pt_i$ is valid for $G$,
has root label $s_G$, and has yield $uv^iwx^iy$, showing that
$uv^iwx^iy\in L(G)$.
\end{proof}

\subsection{Experimenting with the Pumping Lemma Using Forlan}

The Forlan module \texttt{PT} defines a type and several functions
that implement the idea behind the pumping lemma:
\begin{verbatim}
type pumping_division = (pt * int list) * (pt * int list) * pt

val checkPumpingDivision       : pumping_division -> unit
val validPumpingDivision       : pumping_division -> bool
val strsOfValidPumpingDivision :
      pumping_division -> str * str * str * str * str
val pumpValidPumpingDivision   : pumping_division * int -> pt
val findValidPumpingDivision   : pt -> pumping_division
\end{verbatim}

A \emph{pumping division} is a triple $((\pt_1, \pat_1), (\pt_2,
\pat_2), \pt3)$, where $\pt_1, \pt_2, \pt_3\in\PT$ and $\pat_1,
\pat_2\in\List\,\ints$.  We say that a pumping division $((\pt_1,
\pat_1), (\pt_2, \pat_2), \pt_3)$ is \emph{valid} iff
\begin{itemize}
\item $\pat_1$ is a valid path for $\pt_1$, pointing to a leaf whose
  label isn't $\%$;

\item $\pat_2$ is a valid path for $\pt_2$, pointing to a leaf whose
  label isn't $\%$;

\item the label of the leaf of $\pt_1$ pointed to by $\pat_1$ is equal
  to the root label of $\pt_2$;

\item the label of the leaf of $\pt_2$ pointed to by $\pat_2$ is equal
  to the root label of $\pt_2$;

\item the root label of $\pt_3$ is equal to the root label of $\pt_2$;

\item the yield of $\pt_2$ has at least two symbols;

\item the yield of $\pt_1$ has only one occurrence of the root label
  of $\pt_2$;

\item the yield of $\pt_2$ has only one occurrence of the root label
  of $\pt_2$; and

\item the yield of $\pt_3$ does not contain the root label of $\pt_2$.
\end{itemize}
The function \texttt{checkPumpingDivision} checks whether a pumping division
is valid, silently returning \texttt{()} if it is, and explaining why
it isn't, otherwise.  The function \texttt{validPumpingDivision} tests
whether a pumping division is valid.

When the function \texttt{strsOfValidPumpingDivision} is applied to a
valid pumping division $((\pt_1, \pat_1), (\pt_2, \pat_2), \pt_3)$, it
returns $(u, v, w, x, y)$, where:
\begin{itemize}
\item $u$ is the prefix of $\yield\,\pt_1$ that precedes the unique
  occurrence of the root label of $\pt_2$;

\item $v$ is the prefix of $\yield\,\pt_2$ that precedes the unique
  occurrence of the root label of $\pt_2$;

\item $w = \yield\,\pt_3$;

\item $x$ is the suffix of $\yield\,\pt_2$ that follows the unique
  occurrence of the root label of $\pt_2$; and

\item $y$ is the suffix of $\yield\,\pt_1$ that follows the unique
  occurrence of the root label of $\pt_2$.
\end{itemize}
The function issues an error message if the supplied pumping division
isn't valid.

When the function \texttt{pumpValidPumpingDivision} is applied to the
pair of a
valid pumping division $((\pt_1, \pat_1), (\pt_2, \pat_2), \pt_3)$
and a natural number $n$,
it returns $\update(\pt_1, \pat_1, \pow\,n)$, where the function
$\pow\in\nats\fun\PT$ is defined by:
\begin{align*}
  \pow\,0 &= \pt_3 , \\
  \pow(n + 1) &= \update(\pt_2, \pat_2, \pow\,n) ,
  \eqtxt{for all} n\in\nats .
\end{align*}
The function issues an error message if its first argument isn't valid,
or its second argument is negative.

When the function \texttt{findValidPumpingDivision} is called with a
parse tree $\pt$, it tries to find a valid pumping division $\pd$ such
that
\begin{displaymath}
\mathtt{pumpValidPumpingDivision}(\pd, 1) = \pt .  
\end{displaymath}
It works as follows. First, the leftmost, maximum length path $\pat$
through pt is found. If this path points to $\%$, then an error
message is issued. Otherwise, \texttt{findValidPumpingDivision}
generates the following list of variables paired with prefixes of
$\pat$:
\begin{itemize}
\item the root label of the subtree pointed to by the path consisting
  of all but the last element of $\pat$, paired with that path;

\item the root label of the subtree pointed to by the path consisting
  of all but the last two elements of $\pat$, paired with that path;

\item \ldots;

\item the root label of the subtree pointed to by the path consisting
  of the first element of $\pat$, paired with that path; and

\item the root label of the subtree pointed to by $[\,]$, paired with $[\,]$. 
\end{itemize}
(Of course, the left-hand side of the last of these pairs is the root
label of $\pt$.)
As it works through these pairs, it looks for the first
repetition of variables. If there is no such repetition, it issues an
error message. Otherwise, suppose that:
\begin{itemize}
\item $q$ was the first repeated variable;

\item $\pat_1$ was the path paired with $q$ at the point of the first
  repetition; and

\item $\pat'$ was the path paired with $q$ when it was first seen. 
\end{itemize}
Now, it lets:
\begin{itemize}
\item $\pat_2$ be the result of dropping $\pat_1$ from the beginning of
  $\pat'$;

\item $\pt_1$ be $\update(\pt, \pat_1, q)$;

\item $pt'$ be the subtree of $\pt$ pointed to by $\pat_1$;

\item $pt_2$ be $\update(\pt', \pat_2, q)$;

\item $\pt_3$ be the subtree of $\pt'$ pointed to by $\pat_2$; and

\item $\pd = ((\pt_1, \pat_1), (\pt_2, \pat_2), \pt_3)$.
\end{itemize}
If $\pd$ is a valid pumping division (only the last four conditions of
the definition of validity remain to be checked), it is returned by
\texttt{findValidPumpingDivision}. Otherwise, an error message is
issued.

For example, suppose that \texttt{gram} is bound to the
grammar
\begin{align*}
\Asf &\fun \% \mid \zerosf\Bsf\Asf \mid \onesf\Csf\Asf , \\
\Bsf &\fun \onesf \mid \zerosf\Bsf\Bsf , \\
\Csf &\fun \zerosf \mid \onesf\Csf\Csf .
\end{align*}
Then we can proceed as follows:
\begin{list}{}
{\setlength{\leftmargin}{\leftmargini}
\setlength{\rightmargin}{0cm}
\setlength{\itemindent}{0cm}
\setlength{\listparindent}{0cm}
\setlength{\itemsep}{0cm}
\setlength{\parsep}{0cm}
\setlength{\labelsep}{0cm}
\setlength{\labelwidth}{0cm}
\catcode`\#=12
\catcode`\$=12
\catcode`\%=12
\catcode`\^=12
\catcode`\_=12
\catcode`\.=12
\catcode`\?=12
\catcode`\!=12
\catcode`\&=12
\ttfamily}
\small
\item[]\textsl{-\ }val\ pt\ =\ Gram.parseAlphabet\ gram\ (Str.input\ "");
\item[]\textsl{@\ }1110010010
\item[]\textsl{@\ }.
\item[]\textsl{val\ pt\ =\ -\ :\ pt}
\item[]\textsl{-\ }PT.output("",\ pt);
\item[]\textsl{A}
\item[]\textsl{(1,\ C(1,\ C(1,\ C(0),\ C(0)),\ C(1,\ C(0),\ C(0))),}
\item[]\textsl{\ A(1,\ C(0),\ A(%)))}
\item[]\textsl{val\ it\ =\ ()\ :\ unit}
\item[]\textsl{-\ }val\ pd\ =\ PT.findValidPumpingDivision\ pt;\ 
\item[]\textsl{val\ pd\ =\ ((-,\symbol{'133}2,2\symbol{'135}),(-,\symbol{'133}2\symbol{'135}),-)\ :\ PT.pumping_division}
\item[]\textsl{-\ }val\ ((pt1,\ pat1),\ (pt2,\ pat2),\ pt3)\ =\ pd;
\item[]\textsl{val\ pt1\ =\ -\ :\ pt}
\item[]\textsl{val\ pat1\ =\ \symbol{'133}2,2\symbol{'135}\ :\ int\ list}
\item[]\textsl{val\ pt2\ =\ -\ :\ pt}
\item[]\textsl{val\ pat2\ =\ \symbol{'133}2\symbol{'135}\ :\ int\ list}
\item[]\textsl{val\ pt3\ =\ -\ :\ pt}
\item[]\textsl{-\ }PT.output("",\ pt1);
\item[]\textsl{A(1,\ C(1,\ C,\ C(1,\ C(0),\ C(0))),\ A(1,\ C(0),\ A(%)))}
\item[]\textsl{val\ it\ =\ ()\ :\ unit}
\item[]\textsl{-\ }PT.output("",\ pt2);
\item[]\textsl{C(1,\ C,\ C(0))}
\item[]\textsl{val\ it\ =\ ()\ :\ unit}
\item[]\textsl{-\ }PT.output("",\ pt3);
\item[]\textsl{C(0)}
\item[]\textsl{val\ it\ =\ ()\ :\ unit}
\item[]\textsl{-\ }val\ (u,\ v,\ w,\ x,\ y)\ =\ PT.strsOfValidPumpingDivision\ pd;
\item[]\textsl{val\ u\ =\ \symbol{'133}-,-\symbol{'135}\ :\ str}
\item[]\textsl{val\ v\ =\ \symbol{'133}-\symbol{'135}\ :\ str}
\item[]\textsl{val\ w\ =\ \symbol{'133}-\symbol{'135}\ :\ str}
\item[]\textsl{val\ x\ =\ \symbol{'133}-\symbol{'135}\ :\ str}
\item[]\textsl{val\ y\ =\ \symbol{'133}-,-,-,-,-\symbol{'135}\ :\ str}
\item[]\textsl{-\ }(Str.toString\ u,\ Str.toString\ v,\ Str.toString\ w,
\item[]\textsl{=\ }\ Str.toString\ x,\ Str.toString\ y);
\item[]\textsl{val\ it\ =\ ("11","1","0","0","10010")}
\item[]\textsl{\ \ :\ string\ \symbol{'052}\ string\ \symbol{'052}\ string\ \symbol{'052}\ string\ \symbol{'052}\ string}
\item[]\textsl{-\ }val\ pt'\ =\ PT.pumpValidPumpingDivision(pd,\ 2);
\item[]\textsl{val\ pt'\ =\ -\ :\ pt}
\item[]\textsl{-\ }PT.output("",\ pt');
\item[]\textsl{A}
\item[]\textsl{(1,\ C(1,\ C(1,\ C(1,\ C(0),\ C(0)),\ C(0)),\ C(1,\ C(0),\ C(0))),}
\item[]\textsl{\ A(1,\ C(0),\ A(%)))}
\item[]\textsl{val\ it\ =\ ()\ :\ unit}
\item[]\textsl{-\ }Str.output("",\ PT.yield\ pt');
\item[]\textsl{111100010010}
\item[]\textsl{val\ it\ =\ ()\ :\ unit}
\end{list}


\subsection{Consequences of Pumping Lemma}

We are now in a position to show that the context-free languages are
\emph{not} closed under either intersection or set difference.
Suppose
\begin{align*}
  L &= \setof{\zerosf^n\onesf^n\twosf^n}{n\in\nats} , \\
  A &= \setof{\zerosf^n\onesf^n\twosf^m}{n,m\in\nats} , \eqtxtl{and} \\
  B &= \setof{\zerosf^n\onesf^m\twosf^m}{n,m\in\nats} .
\end{align*}
As we proved above, $L$ is not context-free.  In contrast, it's easy
to find grammars generating $A$ and $B$, showing that $A$ and $B$ are
context-free.  But $A\cap B=L$, and thus we have a counterexample to
the context-free languages being closed under intersection.

Now, we have that
$\{\zerosf,\onesf,\twosf\}^*-A$ context-free, since it is the
union of the context-free languages
\begin{gather*}
\{\zerosf,\onesf,\twosf\}^*-\{\zerosf\}^*\{\onesf\}^*\{\twosf\}^*
\end{gather*}
and
\begin{gather*}
\setof{\zerosf^{n_1}\onesf^{n_2}\twosf^m}{n_1,n_2,m\in\nats
\eqtxt{and}n_1\neq n_2} ,
\end{gather*}
(the first of these languages is regular), and the context-free languages
are closed under union.
Similarly, we have that $\{\zerosf,\onesf,\twosf\}^*-B$ is
context-free.

Let
\begin{gather*}
C=(\{\zerosf,\onesf,\twosf\}^*-A) \cup (\{\zerosf,\onesf,\twosf\}^*-B) .
\end{gather*}
Thus $C$ is a context-free subset of $\{\zerosf,\onesf,\twosf\}^*$.
Since $A,B\sub\{\zerosf,\onesf,\twosf\}^*$, it is easy to show that
\begin{align*}
A\cap B &= \{\zerosf,\onesf,\twosf\}^*-
((\{\zerosf,\onesf,\twosf\}^*-A)\cup(\{\zerosf,\onesf,\twosf\}^*-B)) \\
&= \{\zerosf,\onesf,\twosf\}^*-C .
\end{align*}
Thus
\begin{gather*}
\{\zerosf,\onesf,\twosf\}^*-C = A\cap B = L
\end{gather*}
is not context-free, giving us a counterexample to the
context-free languages being closed under set difference.
Of course, this is also a counterexample to the context-free
languages being closed under complementation.

\subsection{Notes}

Apart from the subsection on Forlan's support for experimenting with
the pumping lemma, the material in this section is completely
standard.

%%% Local Variables: 
%%% mode: latex
%%% TeX-master: "book"
%%% End: 
