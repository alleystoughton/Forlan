\section{Diagonalization and Undecidable Problems}
\label{DiagonalizationAndUndecidableProblems}

In this section, we will use a technique called diagonalization to
find a natural language that isn't recursively enumerable.  This will
lead us to a language that is recursively enumerable but is not
recursive.  It will also enable us to prove the undecidability of the
halting problem.

\subsection{Diagonalization}

\index{diagonalization}
To find a non-r.e.\ language, we can use diagonalization, a technique
we used in Section~\ref{BasisSetTheory} to show the uncountability of
$\powset\,\nats$.

Let $\Sigma$ be the alphabet used to describe programs: the digits and
letters, plus the elements of $\{\commasym, \percsym, \mytildesym,
\openparsym, \closparsym, \lesssym, \greatsym\}$.  Every element of
$\Sigma^*$ either describes a unique closed program, or describes no
closed programs.  Given $w\in\Sigma^*$, we write $L(w)$ for:
\begin{itemize}
\item $\emptyset$, if $w$ doesn't describe a closed program; and

\item $L(\pr)$, where $\pr$ is the unique closed program described by
  $w$, if $w$ does describe a closed program.
\end{itemize}
Thus $L(w)$ will always be a set of strings, even though it won't
always be a language.

Consider the infinite table of $0$'s and $1$'s in which both the rows
and the columns are indexed by the elements of $\Sigma^*$, listed in
ascending order according to our standard total ordering, and where a
cell $(w_n, w_m)$ contains $1$ iff $w_n\in L(w_m)$, and contains $0$
iff $w_n\not\in L(w_m)$.
Each recursively enumerable language is $L(w_m)$ for some (non-unique)
$m$, but not all the $L(w_m)$ are languages.
Figure~\ref{DiagTable} shows
\begin{figure}
\begin{center}
\input{chap-5.3-fig1.eepic}
\end{center}
\caption{Example Diagonalization Table}
\label{DiagTable}
\end{figure}
how part of this table might look, where $w_i$, $w_j$ and $w_k$ are
sample elements of $\Sigma^*$.  Because of the table's data, we have
that $w_i\in L(w_j)$ and $w_j\not\in L(w_i)$.

To define a non-r.e.\ $\Sigma$-language, we work our way down the diagonal of
the table, putting $w_n$ into our language just when cell $(w_n,w_n)$
of the table is $0$, i.e., when $w_n\not\in L(w_n)$.
With our example table:
\begin{itemize}
\item $L(w_i)$ is not our language, since $w_i\in L(w_i)$, but $w_i$
  is not in our language;

\item $L(w_j)$ is not our language, since $w_j\not\in L(w_j)$, but
  $w_j$ is in our language; and

\item $L(w_k)$ is not our language, since $w_k\in L(w_k)$, but $w_k$
  is not in our language.
\end{itemize}
In general, there is no $n\in\nats$ such that $L(w_n)$ is our
language.  Consequently our language is not recursively enumerable.

We formalize the above ideas as follows.  Define languages
\index{language!not recursively enumerable}%
\index{language!diagonal}%
\index{language!Ld@$L_d$}%
\index{language!La@$L_a$}%
$L_d$ (``d'' for ``diagonal'') and $L_a$ (``a'' for ``accepted'') by:
\begin{align*}
L_d &= \setof{w\in\Sigma^*}{w\not\in L(w)} , \eqtxtl{and} \\
L_a &= \setof{w\in\Sigma^*}{w\in L(w)} .
\end{align*}

Thus $L_d = \Sigma^* - L_a$.  We have that, for all $w\in\Sigma^*$,
$w\in L_a$ iff $w\in L(\pr)$, for some closed program (which will be
unique) described by $w$. (When $w$ doesn't describe a closed program,
$L(w)=\emptyset$.)

\begin{theorem}
$L_d$ is not recursively enumerable.
\end{theorem}

\begin{proof}
Suppose, toward a contradiction, that $L_d$ is recursively
enumerable.  Thus, there is a closed program $\pr$ such that
$L_d=L(\pr)$.  Let $w\in\Sigma^*$ be the string describing $\pr$.
Thus $L(w)=L(\pr)=L_d$.

There are two cases to consider.
\begin{itemize}
\item Suppose $w\in L_d$.  Then $w\not\in L(w)=L_d$---contradiction.

\item Suppose $w\not\in L_d$.  Since $w\in\Sigma^*$, we have that
  $w\in L(w)=L_d$---contradiction.
\end{itemize}
Since we obtained a contradiction in both cases, we have an
overall contradiction.  Thus $L_d$ is not recursively enumerable.
\end{proof}

\begin{theorem}
$L_a$ is recursively enumerable.
\end{theorem}

\begin{proof}
Let $\mathit{acc}$ be the closed program that, when given $\progstr(w)$, for
some $w\in\Str$, acts as follows.  First, it attempts to parse
$\progstr(w)$ as a program $\pr$, represented as the value
$\overline{\pr}$.  If this attempt fails, $\mathit{acc}$ returns
$\progconst(\progfalse)$.  If $\pr$ is not closed, then $\mathit{acc}$
returns $\progconst(\progfalse)$.  Otherwise, it uses our interpreter
function to evaluate $\progapp(\pr,\progstr(w))$, using
$\overline{\progapp(\pr,\progstr(w))}$.  If this interpretation
returns $\overline{\progconst(\progtrue)}$, then $\mathit{acc}$ returns
$\progconst(\progtrue)$.  If it returns anything other than
$\overline{\progconst(\progtrue)}$, then $\mathit{acc}$ returns
$\progconst(\progfalse)$.  (Thus, if the interpretation never returns,
then $\mathit{acc}$ never terminates.)

We can check that, for all $w\in\Str$, $w\in L_a$ iff
$\eval(\progapp(\mathit{acc},\progstr(w))) =
\norm(\progconst(\progtrue))$.  Thus $L_a$ is recursively enumerable.
\end{proof}

\begin{corollary}
\label{RECor1}

There is an alphabet $\Sigma$ and a recursively enumerable language
$L\sub\Sigma^*$ such that $\Sigma^*-L$ is not recursively enumerable.
\end{corollary}
\index{recursively enumerable language!not closed under complementation}%
\index{complementation!recursively enumerable languages not closed under}%

\begin{proof}
$L_a\sub\Sigma^*$ is recursively enumerable, but
$\Sigma^*-L_a=L_d$ is not recursively enumerable.
\end{proof}

\begin{corollary}
\label{RECor2}

\index{recursively enumerable language!not closed under difference}%
\index{difference!recursively enumerable languages not closed under}%
There are recursively enumerable languages $L_1$ and $L_2$
such that $L_1-L_2$ is not recursively enumerable.
\end{corollary}

\begin{proof}
Follows from Corollary~\ref{RECor1}, since $\Sigma^*$ is
recursively enumerable.
\end{proof}

\begin{corollary}
\label{RECor3}

$L_a$ is not recursive.
\end{corollary}

\begin{proof}
Suppose, toward a contradiction, that $L_a$ is recursive.
Since the recursive languages are closed under complementation,
and $L_a\sub\Sigma^*$, we have that $L_d=\Sigma^*-L_a$ is
recursive---contradiction.  Thus $L_a$ is not recursive.
\end{proof}

Since $L_a\in\RELan$ and $L_a\not\in\RecLan$, we have:

\index{recursive language!proper subset of recursively enumerable
  languages}%
\index{recursively enumerable language!proper superset of recursively
  enumerable languages}%
\begin{theorem}
The recursive languages are a proper subset of the recursively
enumerable languages: $\RecLan\subsetneq \RELan$.
\end{theorem}

Combining this result with results
from Sections~4.8 and 5.1, we have that
\begin{gather*}
\RegLan\subsetneq
\CFLan\subsetneq
\RecLan\subsetneq
\RELan\subsetneq
\Lan .
\end{gather*}

\subsection{Undecidability of the Halting Problem}

\index{halting problem|(}%

\index{program!halts}%
We say that a closed program $\pr$ \emph{halts} iff
$\eval\,\pr\neq\nonterm$.

\begin{theorem}
\label{Halting}

There is no value $\halts$ such that, for all closed programs $\pr$,
\begin{itemize}
\item If $\pr$ halts, then $\eval(\progapp(\halts,\overline{\pr})) =
  \norm(\progconst(\progtrue))$; and

\item If $\pr$ does not halt, then
  $\eval(\progapp(\halts,\overline{\pr})) = \norm(\progconst(\progfalse))$.
\end{itemize}
\end{theorem}

\begin{proof}
Suppose, toward a contradiction, that such a $\halts$ does exist.  We
use $\halts$ to construct a closed program $\mathit{acc}$ that behaves as
follows when run on $\progstr(w)$, for some $w\in\Str$.  First, it
attempts to parse $\progstr(w)$ as a program $\pr$, represented as
the value $\overline{\pr}$.  If this attempt fails, it returns
$\progconst(\progfalse)$.  If $\pr$ is not closed, then it returns
$\progconst(\progfalse)$.  Otherwise, it calls $\halts$ with argument
$\overline{\progapp(\pr,\progstr(w))}$.

\begin{itemize}
\item If $\halts$ returns $\progconst(\progtrue)$ (so we know that
  $\progapp(\pr,\progstr(w))$ halts), then $\mathit{acc}$ applies the
  interpreter function to $\overline{\progapp(\pr,\progstr(w))}$,
  using it to evaluate $\progapp(\pr,\progstr(w))$.  If the
  interpreter returns $\overline{\progconst(\progtrue)}$, then
  $\mathit{acc}$ returns $\progconst(\progtrue)$.  Otherwise, the
  interpreter returns some other value, and $\mathit{acc}$ returns
  $\progconst(\progfalse)$.

\item Otherwise, $\halts$ returns $\progconst(\progfalse)$ (so we know that
  $\progapp(\pr,\progstr(w))$ does not halt), in which case
  $\mathit{acc}$ returns $\progconst(\progfalse)$.
\end{itemize}

Now, we prove that $\mathit{acc}$ is a string predicate program
testing whether a string is in $L_a$.

\begin{itemize}
\item Suppose $w\in L_a$.  Thus $w\in L(\pr)$, where $\pr$ is the
  unique closed program described by $w$.  Hence
  $\eval(\progapp(\pr,\progstr(w))) = \norm(\progconst(\progtrue))$.  It
  is easy to show that $\eval(\progapp(\acc,\progstr(w))) =
  \norm(\progconst(\progtrue))$.

\item Suppose $w\not\in L_a$.  If $w\not\in\Sigma^*$, or
  $w\in\Sigma^*$ but $w$ does not describe a program, or $w$ describes
  a program that isn't closed, then $\eval(\progapp(\acc,\progstr(w)))
  = \norm(\progconst(\progfalse))$.  So, suppose $w$ describes the closed
  program $\pr$.  Then $w\not\in L(\pr)$, i.e.,
  $\eval(\progapp(\pr,\progstr(w))) \neq \norm(\progconst(\progtrue))$.
  It is easy to show that $\eval(\progapp(\acc,\progstr(w))) =
  \norm(\progconst(\progfalse))$.
\end{itemize}

Thus $L_a$ is recursive---contradiction.  Thus there is no such
$\mathit{halt}$.
\end{proof}

\index{program: halts on}%
We say that a value $\pr$ \emph{halts on} a value $\pr'$ iff
$\eval(\progapp(\pr,\pr'))\neq\nonterm$.

\begin{corollary}[Undecidability of the Halting Problem]
There is no value $\haltsOn$ such that, for all values
$\pr$ and $\pr'$:
\begin{itemize}
\item if $\pr$ halts on $\pr'$, then
  \begin{displaymath}
    \eval(\progapp(\haltsOn,\progpair(\overline{\pr},\overline{\pr'}))) =
    \norm(\progconst(\progtrue)) ; \eqtxt{and}
  \end{displaymath}

\item If $\pr$ does not halt on $\pr'$, then
  \begin{displaymath}
    \eval(\progapp(\haltsOn,\progpair(\overline{\pr},\overline{\pr'}))) =
    \norm(\progconst(\progfalse)) .
  \end{displaymath}
\end{itemize}
\end{corollary}

\begin{proof}
Suppose, toward a contradiction, that such a $\haltsOn$ exists.  Let
$\halts$ be the value that takes in a value $\overline{\pr}$
representing a closed program $\pr$, and then returns the result of
calling $\haltsOn$ with
$\progpair(\overline{\proglam(\mathsf{x},\pr)},\overline{\progconst(\prognil)})$.
Then this value satisfies the property of
Theorem~\ref{Halting}---contradiction.  Thus such a $\haltsOn$ does
not exist.
\end{proof}

\index{halting problem|)}%

\subsection{Other Undecidable Problems}

Here are two other undecidable problems:
\begin{itemize}
\item Determining whether two grammars generate the same language.
(In contrast, we gave an algorithm for checking whether two FAs are
equivalent, and this algorithm can be implemented as a program.)

\item Determining whether a grammar is ambiguous.
\end{itemize}

\subsection{Notes}

Because a closed program can be evaluated by itself, i.e., without
being supplied an input, our treatment of the undecidability of
the halting problem is nonstandard.  First, we prove that there is
no program that takes in a program as data and tests whether that
program halts.  Our version of the undecidability of the halting
then follows as a corollary.

%%% Local Variables: 
%%% mode: latex
%%% TeX-master: "book"
%%% End: 
