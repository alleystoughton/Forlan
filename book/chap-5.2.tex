\section{Closure Properties of Recursive and R.E.\ Languages}
\label{ClosurePropertiesOfRecursiveAndRELanguages}

In this section, we will see that the recursive and recursively
enumerable languages are closed under union, concatenation, closure
and intersection.  The recursive languages are also closed under set
difference and complementation.  In the next section, we will see that
the recursively enumerable languages are not closed under
complementation or set difference.  On the other hand, we will see in
this section that, if a language and its complement are both r.e.,
then the language is recursive.

\subsection{Closure Properties of Recursive Languages}

\begin{theorem}
\label{RecClose}

If $L$, $L_1$ and $L_2$ are recursive languages, then so are
$L_1\cup L_2$, $L_1L_2$, $L^*$, $L_1\cap L_2$ and $L_1-L_2$.
\end{theorem}

\begin{proof}
Let's consider the concatenation case as an example.  Since $L_1$
and $L_2$ are recursive languages, there are string predicate programs
$\pr_1$ and $pr_2$ that test whether strings are in $L_1$ and $L_2$,
respectively.
We write a program $\pr$ with form
\begin{gather*}
\mtab{\proglam(\TS\mathsf{w},\NL
\progletSimp(\TS\mathsf{f1},\NL
\pr_1,\NL
\progletSimp(\TS\mathsf{f2},\NL
\pr_2,\NL
\cdots)))},
\end{gather*}
which tests whether its input is an element of $L_1L_2$.  

The elided part of $\pr$ generates all of the
pairs of strings $(x_1,x_2)$ such that $x_1x_2$ is equal to the
value of $\mathsf{w}$.  Then it works
though these pairs, one by one.  Given such a pair $(x_1,x_2)$, $\pr$
calls $\mathsf{f1}$ with $x_1$ to check whether $x_1\in L_1$.  If the
answer is $\progconst(\progfalse)$, then it goes on to the next pair.
Otherwise, it calls $\mathsf{f2}$ with $x_2$ to check whether $x_2\in
L_2$.  If the answer is $\progconst(\progfalse)$, then it goes on to the
next pair.  Otherwise, it returns $\progconst(\progtrue)$.  If $\pr$ runs
out of pairs to check, then it returns $\progconst(\progfalse)$.

We can check that $\pr$ is a string predicate program testing whether
$w\in L_1L_2$.  Thus $L_1L_2$ is recursive.
\end{proof}

\begin{corollary}
\label{RecComp}

If $\Sigma$ is an alphabet and $L\sub\Sigma^*$ is recursive,
then so is $\Sigma^*-L$.
\end{corollary}

\begin{proof}
Follows from Theorem~\ref{RecClose}, since $\Sigma^*$ is recursive.
\end{proof}
\subsection{Closure Properties of Recursively Enumerable Languages}

\begin{theorem}
If $L$, $L_1$ and $L_2$ are recursively enumerable languages, then so
are $L_1\cup L_2$, $L_1L_2$, $L^*$ and $L_1\cap L_2$.
\end{theorem}

\begin{proof}
We consider the concatenation case as an example.  Since $L_1$ and
$L_2$ are recursively enumerable, there are closed programs $\pr_1$ and
$\pr_2$ such that, for all $w\in\Str$, $w\in L_1$ iff
$\eval(\progapp(\pr_1,\progstr(w)))=\norm(\progconst(\progtrue))$, and
for all $w\in\Str$, $w\in L_2$ iff
$\eval(\progapp(\pr_2,\progstr(w)))=\norm(\progconst(\progtrue))$.
(Remember that $\pr_1$ and $\pr_2$ may fail to terminate on some
inputs.)

To show that $L_1L_2$ is recursively enumerable, we will construct a
closed program $\pr$ such that, for all $w\in\Str$, $w\in L_1L_2$ iff
$\eval(\progapp(\pr,\progstr(w)))=\norm(\progconst(\progtrue))$.

When $\pr$ is called with $\progstr(w)$, for some $w\in\Str$, it
behaves as follows.  First, it generates all the pairs of strings
$(x_1,x_2)$ such that $w=x_1x_2$.  Let these pairs be
$(x_{1,1},x_{2,1}),\ldots,(x_{1,n},x_{2,n})$.  Now, $\pr$ uses our
\emph{incremental} interpretation function to run a fixed number of
steps of $\progapp(\pr_1,\progstr(x_{1,i}))$ and
$\progapp(\pr_2,\progstr(x_{2,i}))$ (working with
$\overline{\progapp(\pr_1,\progstr(x_{1,i}))}$ and
$\overline{\progapp(\pr_2,\progstr(x_{2,i}))}$), for all $i\in[1:n]$,
and then repeat this over and over again.

\begin{itemize}
\item If, at some stage, the incremental interpretation of
  $\progapp(\pr_1,\progstr(x_{1,i}))$ returns
  $\overline{\progconst(\progtrue)}$, then $x_{1,i}$ is marked as being in
  $L_1$.

\item If, at some stage, the incremental interpretation of
  $\progapp(\pr_2,\progstr(x_{2,i}))$ returns
  $\overline{\progconst(\progtrue)}$, then the $x_{2,i}$ is marked as
  being in $L_2$.

\item If, at some stage, the incremental interpretation of
  $\progapp(\pr_1,\progstr(x_{1,i}))$ returns something other than
  $\overline{\progconst(\progtrue)}$, then the $i$'th pair is marked as
  discarded.

\item If, at some stage, the incremental interpretation of
  $\progapp(\pr_2,\progstr(x_{2,i}))$ returns something other than
  $\overline{\progconst(\progtrue)}$, then the $i$'th pair is marked as
  discarded.

\item If, at some stage, $x_{1,i}$ is marked as in $L_1$ and $x_{2,i}$
  is marked as in $L_2$, then $Q$ returns $\progconst(\progtrue)$.

\item If, at some stage, there are no remaining pairs, then $\pr$
  returns $\progconst(\progfalse)$.
\end{itemize}
\end{proof}

\begin{theorem}
If $\Sigma$ is an alphabet, $L\sub\Sigma^*$ is a recursively
enumerable language, and $\Sigma^*-L$ is recursively enumerable, then
$L$ is recursive.
\end{theorem}

\begin{proof}
Since $L$ and $\Sigma^*-L$ are recursively enumerable languages,
there are closed programs $\pr_1$ and $\pr_2$ such that, for all $w\in\Str$,
$w\in L$ iff
$\eval(\progapp(\pr_1,\progstr(w)))=\norm(\progconst(\progtrue))$, and
for all $w\in\Str$, $w\in\Sigma^*-L$ iff
$\eval(\progapp(\pr_2,\progstr(w)))=\norm(\progconst(\progtrue))$.

We construct a string predicate program $\pr$ that tests whether its
input is in $L$.  Given $\progstr(w)$, for $w\in\Str$, $\pr$ proceeds
as follows.  If $w\not\in\Sigma^*$, then $\pr$ returns
$\progconst(\progfalse)$.  Otherwise, $\pr$ alternates between
incrementally interpreting $\progapp(\pr_1,\progstr(w))$ (working with
$\overline{\progapp(\pr_1,\progstr(w))}$) and
$\progapp(\pr_2,\progstr(w))$ (working with
$\overline{\progapp(\pr_2,\progstr(w))}$).

\begin{itemize}
\item If, at some stage, the first incremental interpretation returns
  $\overline{\progconst(\progtrue)}$, then $\pr$ returns
  $\progconst(\progtrue)$.

\item If, at some stage, the second incremental interpretation returns
  $\overline{\progconst(\progtrue)}$, then $\pr$ returns
  $\progconst(\progfalse)$.

\item If, at some stage, the first incremental interpretation returns
  anything other than $\overline{\progconst(\progtrue)}$, then $\pr$
  returns $\progconst(\progfalse)$.

\item If, at some stage, the second incremental interpretation returns
  anything other than $\overline{\progconst(\progtrue)}$, then $\pr$
  returns $\progconst(\progtrue)$.
\end{itemize}
\end{proof}

\subsection{Notes}

Our approach to this section is standard, except for our using
programs instead of Turing machines.

%%% Local Variables: 
%%% mode: latex
%%% TeX-master: "book"
%%% End: 
