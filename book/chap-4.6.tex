\section{Ambiguity of Grammars}
\label{AmbiguityOfGrammars}

In this section, we say what it means for a grammar to be ambiguous.
We also give a straightforward method for disambiguating
grammars for languages with operators of various precedences and
associativities, and consider an efficient parsing algorithm for
such disambiguated grammars.

\subsection{Definition}

Suppose $G$ is our grammar of arithmetic expressions:
\begin{gather*}
\mathsf{E\fun E\plussym E\mid E\timessym E\mid \openparsym E\closparsym \mid
\idsym} .
\end{gather*}
Unfortunately,
there are multiple ways of parsing
$\idsym\timessym\idsym\plussym\idsym$ according to this grammar:
\begin{center}
\input{chap-4.6-fig1.eepic}
\end{center}
In $\pt_1$, multiplication has higher precedence than addition; in
$\pt_2$, the situation is reversed.  Because there are multiple ways
of parsing this string, we say that our grammar is ``ambiguous''.

A grammar $G$ is \emph{ambiguous} iff there is a
$w\in(\alphabet\,G)^*$ such that $w$ is the yield of multiple valid
parse trees for $G$ whose root labels are $s_G$; otherwise, $G$ is
\emph{unambiguous}.

The grammar
\begin{gather*}
\Asf \fun \% \mid \mathsf{0A1A} \mid \mathsf{1A0A}
\end{gather*}
is a grammar generating all elements
of $\{\mathsf{0,1}\}^*$ with a $\diff$ of $0$, for the $\diff$
function such that $\diff\,\zerosf = -1$ and $\diff\,\onesf = 1$.
It is ambiguous as, e.g., $\mathsf{0101}$ can be parsed as
$\mathsf{0\%1(01)}$ or $\mathsf{0(10)1\%}$.
But in Section~4.5, we saw another grammar for this
language:
\begin{align*}
\Asf &\fun \% \mid \zerosf\Bsf\Asf \mid \onesf\Csf\Asf , \\
\Bsf &\fun \onesf \mid \zerosf\Bsf\Bsf , \\
\Csf &\fun \zerosf \mid \onesf\Csf\Csf ,
\end{align*}
which turns out to be unambiguous.
The reason is that $\Pi_\Bsf$ is all elements of $\{\mathsf{0,1}\}^*$
with a $\diff$ of $1$, but with no proper prefixes with positive
$\diff$'s, and $\Pi_\Csf$ has the corresponding property for
$0$/negative.

\subsection{Disambiguating Grammars of Operators}

Not every ambiguous grammar can be turned into an equivalent
unambiguous one.  However, we can use a simple technique to
disambiguate our grammar of arithmetic expressions, and this technique
works for many commonly occurring grammars involving operators of
various precedences and associativities.

Since there are two binary operators in our language of arithmetic
expressions, we have to decide:
\begin{itemize}
\item whether multiplication has higher or lower precedence than
  addition; and

\item whether multiplication and addition are left or right
  associative.
\end{itemize}
As usual, we'll make multiplication have higher precedence than
addition, and let addition and multiplication be left associative.

As a first step towards disambiguating our grammar, we can form
a new grammar with the three variables: $\Esf$ (expressions),
$\Tsf$ (terms) and $\Fsf$ (factors), start variable $\Esf$
and productions:
\begin{align*}
  \Esf &\fun \Tsf\mid \Esf\plussym\Esf , \\
  \Tsf &\fun \Fsf\mid \Tsf\timessym\Tsf , \\
  \Fsf &\fun \idsym \mid \openparsym\Esf\closparsym .
\end{align*}
The idea is that the lowest precedence operator ``lives'' at the
highest level of the grammar, that the highest precedence operator
lives at the middle level of the grammar, and that the basic
expressions, including the parenthesized expressions, live at
the lowest level of the grammar.

Now, there is only one way to parse the string
$\idsym\timessym\idsym\plussym\idsym$, since, if we begin
by using the production $\Esf\fun\Tsf$, our yield will only
include a $\plussym$ if this symbol occurs within parentheses.
If we had more levels of precedence in our language, we would simply
add more levels to our grammar.

On the other hand, there are still two ways of parsing the string
$\idsym\plussym\idsym\plussym\idsym$: with left associativity or right
associativity.  To finish disambiguating our grammar, we must break
the symmetry of the right-sides of the productions
\begin{align*}
  \Esf &\fun \Esf\plussym\Esf , \\
  \Tsf &\fun \Tsf\timessym\Tsf ,
\end{align*}
turning one of the $\Esf$'s into $\Tsf$, and one of the $\Tsf$'s into
$\Fsf$.  To make our operators be left associative, we must use
\emph{left recursion}, changing the second $\Esf$ to $\Tsf$, and the
second $\Tsf$ to $\Fsf$; right associativity would result from making
the opposite choices, i.e., using \emph{right recursion}.

Thus, our unambiguous grammar of arithmetic expressions is
\begin{align*}
\Esf &\fun \Tsf \mid \Esf\plussym\Tsf , \\
\Tsf &\fun \Fsf \mid \Tsf\timessym\Fsf , \\
\Fsf &\fun \idsym \mid \openparsym\Esf\closparsym .
\end{align*}
It can be proved that this grammar is indeed unambiguous, and that it
is equivalent to the original grammar.

Now, the only parse of $\idsym\timessym\idsym\plussym\idsym$ is
\begin{center}
\input{chap-4.6-fig2.eepic}
\end{center}
And, the only parse of $\idsym\plussym\idsym\plussym\idsym$
is
\begin{center}
\input{chap-4.6-fig3.eepic}
\end{center}

\subsection{Top-down Parsing}

Top-down parsing is a simple and efficient parsing method for
unambiguous grammars of operators like
\begin{align*}
\Esf &\fun \Tsf \mid \Esf\plussym\Tsf , \\
\Tsf &\fun \Fsf \mid \Tsf\timessym\Fsf , \\
\Fsf &\fun \idsym \mid \openparsym\Esf\closparsym .
\end{align*}

Let $\cal E$, $\cal T$ and $\cal F$ be all of the parse trees that
are valid for our grammar, have yields containing no variables,
and whose root labels are $\mathsf{E}$, $\mathsf{T}$ and $\mathsf{F}$, 
respectively.
Because this grammar has three mutually recursive variables, we
will need three mutually recursive parsing functions,
\begin{align*}
\parE &\in \Str\fun\Option({\cal E}\times\Str) , \\
\parT &\in \Str\fun\Option({\cal T}\times\Str) , \\
\parF &\in \Str\fun\Option({\cal F}\times\Str) ,
\end{align*}
which attempt to parse an element $\pt$ of $\cal E$, $\cal T$ or $\cal
F$ out of a string $w$, returning $\none$ to indicate failure, and
$\some(\pt,y)$, where $y$ is the remainder of $w$, otherwise.

Although most programming languages support mutual recursion, in this
book, we haven't formally justified well-founded mutual recursion.  
Instead, we can work with a single recursive function with domain
$\{0,1,2\}\times\Str$, where the $0$, $1$ or $2$ indicates whether
it's $\parE$, $\parT$ or $\parF$, respectively, that is being called.

The well-founded ordering we are using allows:
\begin{itemize}
\item $\parE$ to call $\parT$ with strings that are no longer
  than its argument;

\item $\parT$ to call $\parF$ with strings that are no longer
  than its argument; and

\item $\parF$ to call $\parE$ with strings that are strictly shorter
  than its argument.
\end{itemize}

When called with a string $w$, $\parE$ is supposed to
determine whether there is a prefix $x$ of $w$ that is the
yield of an element of $\cal E$.  If there is such an $x$,
then it finds the \emph{longest} prefix $x$ of $w$ with this
property, and returns $\some(\pt,y)$, where $\pt$ is the
element of $\cal E$ whose yield is $x$, and $y$ is such
that $w=xy$.  Otherwise, it returns $\none$.
$\parT$ and $\parF$ have similar specifications.

Given a string $w$, $\parE$ operates as follows.  Because all elements
of $\cal E$ have yields beginning with the yield of an element of
$\cal T$, it starts by evaluating $\parT\,w$.  If this results in
$\none$, it returns $\none$.  Otherwise, it results in $\some(\pt,x)$,
for some $\pt\in{\cal T}$ and $x\in\Str$, in which case $\parE$
returns $\parELoop(E(\pt), x)$, where $\parELoop\in{\cal
  E}\times\Str\fun\Option({\cal E}\times\Str)$ is defined recursively,
as follows.

Given $(\pt, x)\in{\cal E}\times\Str$, $\parELoop$ proceeds as
follows.
\begin{itemize}
\item If $x={\plussym}y$ for some $y$, then $\parELoop$ evaluates
  $\parT\,y$.
  \begin{itemize}
  \item If this results in $\none$, then $\parELoop$ returns $\none$.
  
  \item Otherwise, it results in $\some(\pt',z)$ for some
    $\pt'\in{\cal T}$ and $z\in\Str$, and $\parELoop$ returns
    $\parELoop(E(\pt,\plussym,\pt'),z)$.
  \end{itemize}

\item Otherwise, $\parELoop$ returns $\some(\pt,x)$.
\end{itemize}
The function $\parT$ operates analogously.

Given a string $w$, $\parF$ proceeds as follows.
\begin{itemize}
\item If $w=\idsym x$ for some $x$, then it returns
  $\some(F(\idsym),x)$.

\item Otherwise, if $w=\openparsym x$, then $\parF$ evaluates
  $\parE\,x$.
  \begin{itemize}
  \item If this results in $\none$, it returns $\none$.
      
  \item Otherwise, this results in $\some(\pt,y)$ for some
    $\pt\in{\cal E}$ and $y\in\Str$.
    \begin{itemize}
    \item If $y=\closparsym z$ for some $z$, then $\parF$ returns
      \begin{displaymath}
       \some(F(\openparsym,\pt,\closparsym), z) . 
      \end{displaymath}
    
    \item Otherwise, $\parF$ returns $\none$.
    \end{itemize}
  \end{itemize}

  \item Otherwise $\parF$ returns $\none$.
\end{itemize}

Given a string $w$ to parse, the algorithm evaluates
$\parE\,w$.  If the result of this evaluation is:
\begin{itemize}
\item $\none$, then the algorithm reports failure;

\item $\some(\pt,\%)$, then the algorithm returns $\pt$;

\item $\some(\pt,y)$, where $y\neq\%$, then the algorithm reports failure,
  because not all of the input could be parsed.
\end{itemize}

\subsection{Notes}

The standard approach to doing top-down parsing in the presence of
left recursive productions is to first translate the left recursion to
right recursion, and then restructure the parse trees produced by the
parser.  In contrast, we showed a direct approach to handling left
recursion that works for grammars of operators.

%%% Local Variables: 
%%% mode: latex
%%% TeX-master: "book"
%%% End: 
