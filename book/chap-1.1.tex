\section{Basic Set Theory}
\label{BasisSetTheory}

\index{set|(}%
In this section, we will cover the material on logic, sets, relations,
functions and data structures that will be needed in what follows.
Much of this material should be at least partly familiar.

\subsection{Review of Classical Logic}

\index{logic|(}%
\index{classical logic|(}%
We begin by very briefly reviewing the rules of classical logic. We
can build \emph{formulas} using basic predicates (like ``x = y''),
\index{formulas}%
\index{logic!formulas}%
plus the following quantifiers and connectives:
\begin{itemize}
\item $\textbf{For all}\; x, P(x)$ (\emph{universal
\index{universal quantification}%
\index{logic!universal quantification}%
    quantification}). The most typical way of proving such a formula
  is to add the variable $x$ to the \emph{variable context} (each
\index{variable context}%
\index{logic!variable context}%
  element of which stands for some element of the universe of
  values)\footnote{We rename $x$ first, if $x$ is already in the
    variable context.}, and then go about proving $P(x)$ (possibly
  making use of any of the current \emph{assumptions}).
\index{assumptions}%
\index{logic!assumptions}%

\item $\textbf{There exists an}\; x \;\textbf{such that}\; P(x)$
  (\emph{existential quantification}).  The most typical way of
\index{existential quantification}%
\index{logic!existential quantification}%
  proving such a formula is to prove that $P(t)$ holds, where $t$ is
  some term (expression), all of whose variables appear in the
  variable context.

\item $P \myand Q$ (\emph{conjunction}). The most typical way of
\index{conjunction}%
\index{logic!conjunction}%
  proving such a formula is to prove \emph{both} $P$ and $Q$, individually.

\item $P \myor Q$ (\emph{disjunction}). The most typical way
\index{disjunction}%
\index{logic!disjunction}%
  of proving such a formula is to prove \emph{one} of $P$ or $Q$.

\item $P \myimplies Q$ (also written \textbf{if} $P$,
  \textbf{then} $Q$) (\emph{implication}).  The most typical way to
\index{implication}%
\index{logic!implication}%
  prove such a formula is to add $P$ to our current assumptions, and
  then use these assumptions to establish the truth of $Q$.

\item $\textbf{Not}\; P$ ($P$ does not hold; negation). The most typical
\index{negation}%
\index{logic!negation}%
  way to prove such a formula is to show that, if we add $P$ to our
  current assumptions, we can prove a contradiction---something obviously
  false.
\end{itemize}

In formulas involving the connectives \textbf{not}, \textbf{and} and
\textbf{or}, we give \textbf{not} the highest precedence, \textbf{and}
the next highest precedence, and \textbf{or} the lowest
precedence. Because $(P \myand Q) \myand R$ is
equivalent to $P \myand (Q \myand R)$, we don't need
to use parentheses when combining multiple occurrences of
\textbf{and}.  The same is true for \textbf{or}.
When we write $P \myiffbf Q$ (``if and only if''), this means that
$(P \myimplies Q) \myand (Q \myimplies P)$.  $P \myimplies Q$
is equivalent to $\textbf{not}\;P \myor Q$.

Universal and existential quantification have lower precedence
than the connectives, and extend as far as possible. E.g.,
$\textbf{for all}\; x, \;\textbf{for all}\; y, P(x)\myand Q(y)$ means
$\textbf{for all}\; x, \;\textbf{for all}\; y, (P(x)\myand Q(y))$.
We can rename quantified variables in formulas, maintaining equivalence,
e.g., turning
$\textbf{for all}\; x, \;\textbf{for all}\; y, P(x)\myand Q(y)$ into
$\textbf{for all}\; y, \;\textbf{for all}\; z, P(y)\myand Q(z)$.
We often abbreviate nested universal or existential quantifications,
writing, e.g., 
$\textbf{for all}\; x, y, P(x)\myand Q(y)$ instead of
$\textbf{for all}\; x, \;\textbf{for all}\; y, P(x)\myand Q(y)$.

We have that:
\begin{itemize}
\item $\textbf{Not for all}\; x, P(x)$, is equivalent to
  $\textbf{there exists an}\; x \;\textbf{such that not}\; P(x)$;

\item $\textbf{Not there exists an}\; x \;\textbf{such that}\; P(x)$ is
  equivalent to $\textbf{for all}\; x, \;\textbf{not}\; P(x)$;

\item $\textbf{Not} (P \myand Q)$ is equivalent to
  $\textbf{not}\; P \myor \,\textbf{not}\; Q$;

\item $\textbf{Not} (P \myor Q)$ is equivalent to $\textbf{not}\;
  P \myand \,\textbf{not}\; Q$;

\item $\textbf{Not} (P \myimplies Q)$ is equivalent to $P
  \myand \textbf{not}\; Q$;

\item $\textbf{Not not}\; P$ is equivalent to $P$.
\end{itemize}

Furthermore:
\begin{itemize}
\item If the current assumptions are contradictory, we can prove
\index{logical contradiction}%
\index{logic!contradiction}%
  any conclusion, $Q$.

\item If we have the assumptions $P \,\myimplies\, Q$ and $P$, we
  may conclude $Q$.

\item If we have the assumption $\textbf{for all}\; x, P(x)$, and $t$ is
  a term whose variables come from the variable context, then
  we may conclude $P(t)$.

\item If one of our assumptions is a disjunction $P \myor Q$,
  and we are trying to prove $R$, then it suffices to:
  \begin{itemize}
  \item show that $R$ follows from $P$ plus whatever other assumptions
    we have; and
  \item show that $R$ follows from $Q$ plus whatever other assumptions
    we have.
  \end{itemize}
  This rule is called \emph{disjunction elimination}. It generalizes
\index{disjunction elimination}%
\index{logic!disjunction elimination}%
  to when there are more than two disjuncts.

\item If one of our assumptions is $\textbf{there exists an}\; x
  \;\textbf{such that}\; P(x)$, and we are trying to prove $R$, it
  suffices to introduce the assumption that $P(x)$ holds, where the
  variable $x$ is added to our variable context (and so stands for
  some element of the value universe)\footnote{We rename $x$ first,
  if $x$ is already in the variable context.}, and then establish that $R$
  holds. This rule is called \emph{existential elimination}.
\index{existential elimination}%
\index{logic!existential elimination}%

\item When trying to prove a conclusion $P$ from some set of assumptions,
  we can suspend this proof, and go about proving a formula $Q$ from
  the same set of assumptions. If this sub-proof succeeds, we can
\index{sub-proof}%
\index{logic!sub-proof}%
  add $Q$ to our original set of assumptions, and then go on with
  our proof of $P$.
\end{itemize}

We have that $P \myimplies Q$ is equivalent to
$\textbf{not}\; P \myor Q$.  It is also equivalent to its
\emph{contrapositive},
\index{contrapositive}%
\index{logic!contrapositive}%
$\textbf{not}\; Q \myimplies \textbf{not}\; P$.
We have that $P \myiffbf Q$ equivalent to
$\textbf{not}\;P\myiffbf \textbf{not}\;Q$.

By the Law of the Excluded Middle, we have that $P\myor \textbf{not}\;P$,
\index{Law of the Excluded Middle}%
\index{logic!Law of the Excluded Middle}%
i.e., that either $P$ or its negation holds.

When we do a \emph{case analysis}, we first prove (this may be
\index{case analysis}%
\index{logic!case analysis}%
obvious, in which case we may leave the proof implicit) that a
disjunction $P_1 \myor \cdots \myor P_n$ holds, adding it to our
assumptions. We then use disjunction elimination, showing our
conclusion follows from each of $P_1,\ldots,P_n$.  Case analysis is
often done in conjunction with the Law of the Excluded Middle.

Finally, when proving $P$ using \emph{proof by contradiction}, we
\index{proof by contradiction}%
\index{logic!proof by contradiction}%
temporarily add $\textbf{not}\;P$ to our assumptions and then try to
derive a contradiction (i.e., prove something that's obviously
false). If we succeed, then $P$ follows from our original assumptions.

We write
\begin{displaymath}
  \textbf{for all sets}\; X, P(X)
\end{displaymath}
as an abbreviation for
\begin{displaymath}
  \textbf{for all}\; X, X\; \textbf{is a set} \mathbin{\textbf{implies}} P(X) .
\end{displaymath}
And we write 
\begin{displaymath}
  \textbf{there exists a set}\; X \;\textbf{such that}\; P(X)
\end{displaymath}
as an abbreviation for
\begin{displaymath}
  \textbf{there exists an} \; X \;\textbf{such that}\;
  X \;\textbf{is a set} \mathbin{\textbf{and}} P(X) .
\end{displaymath}
Thus we have that:
\begin{displaymath}
  \textbf{not for all sets}\; X, P(X)
\end{displaymath}
is equivalent to
\begin{displaymath}
  \textbf{there exists a set}\; X \; \textbf{such that not}\; P(X);
\end{displaymath}
and
\begin{displaymath}
  \textbf{not there exists a set}\; X \;\textbf{such that}\; P(X)
\end{displaymath}
is equivalent to
\begin{displaymath}
  \textbf{for all sets}\, X, \textbf{not}\; P(X) .
\end{displaymath}
When writing proofs involving these abbreviated formulas, we compress
some proof steps into single steps in obvious ways.  For example, when
proving $\textbf{for all sets}\; X, P(X)$, we simply introduce the
assumption that $X$ is a set, and then go about proving $P(X)$,
instead of first adding $X$ to our variable context, and then adding
the fact that $X$ is a set to our current assumptions.  We make use of
similar formula abbreviations without comment.
\index{logic|)}%
\index{classical logic|)}%

\subsection{Describing Sets by Listing Their Elements}

We write $\emptyset$
\index{empty set}%
\index{set!empty}%
\index{ empty@$\emptyset$}%
\index{set! empty@$\emptyset$}%
for the \emph{empty set}---the set with no elements.  Finite sets can
be described by listing their elements inside set braces: $\{x_1,
\ldots, x_n\}$.
\index{ set@$\{\ldots\}$}%
\index{set! set@$\{\ldots\}$}%
E.g., $\{3\}$ is the \emph{singleton set}
\index{singleton set}%
\index{set!singleton}%
whose only element is $3$, and $\{1, 3, 5, 7\}$ is the set consisting
of the first four odd numbers.

\subsection{Sets of Numbers}

We write:
\begin{itemize}
\item $\nats$
\index{n@$\nats$}%
for the set $\{0, 1, \ldots\}$ of all natural numbers;
\index{natural number}%

\item $\ints$
\index{z@$\ints$}%
for the set $\{\ldots, -1, 0, 1, \ldots\}$ of all integers;
\index{integer}%

\item $\reals$
\index{r@$\reals$}%
for the set of all real numbers.
\index{real number}%
\end{itemize}
Note that, for us, $0$ \emph{is} a natural number.  This has many
pleasant consequences, e.g., that the size of a finite set or
the length of a list will always be a natural number.

\subsection{Relationships between Sets}

As usual, we write $x\in Y$
\index{element of}%
\index{set!element of}%
\index{ element of@$\in$}%
\index{set! element of@$\in$}%
to mean that $x$ is one of the elements (members) of the set $Y$.
Sets $A$ and $B$ are equal ($A=B$)
\index{equal!set}%
\index{set!equal}%
\index{ equal@$=$}%
\index{set! equal@$=$}%
iff (if and only if) they have the same elements, i.e., for all $x$,
\index{iff}%
$x\in A$ iff $x\in B$.
We write
\begin{displaymath}
  \textrm{for all}\; x \in X, P(x)
\end{displaymath}
as an abbreviation for
\begin{displaymath}
  \textrm{for all}\; x, x\in X \mathbin{\textrm{implies}} P(x) .
\end{displaymath}
And we write 
\begin{displaymath}
  \textrm{there exists an}\; x \in X \;\textrm{such that}\; P(x)
\end{displaymath}
as an abbreviation for
\begin{displaymath}
  \textrm{there exists an}\; x \;\textrm{such that}\; x\in X
  \mathbin{\textrm{and}} P(x) .
\end{displaymath}
Thus, we have that:
\begin{displaymath}
\textrm{Not for all}\; x\in X, P(x)  
\end{displaymath}
is equivalent to
\begin{displaymath}
\textrm{there exists an}\; x\in X \;\textrm{such that not}\; P(x);
\end{displaymath}
and
\begin{displaymath}
\textrm{Not there exists an}\; x\in X \;\textrm{such that}\; P(x)  
\end{displaymath}
is equivalent to
\begin{displaymath}
\textrm{for all}\; x\in X, \;\textrm{not}\; P(x) .  
\end{displaymath}

Suppose $A$ and $B$ are sets.  We say that:
\begin{itemize}
\item $A$ is a \emph{subset}
\index{subset}%
\index{set!subset}%
of $B$ ($A\sub B$)
\index{ subset@$\sub$}%
\index{set! subset@$\sub$}%
iff, for all $x\in A$, $x\in B$;

\item $A$ is a
\emph{proper subset}
\index{proper!subset}%
\index{subset!proper}%
\index{set!subset!proper}%
of $B$ ($A\subsetneq B$)
\index{ proper subset@$\subsetneq$}%
\index{set! proper subset@$\subsetneq$}%
iff $A\sub B$ but $A\neq B$.
\end{itemize}
In other words: $A$ is a subset of $B$ iff every everything in $A$ is
also in $B$, and $A$ is a proper subset of $B$ iff everything in $A$
is in $B$, but there is at least one element of $B$ that is not in
$A$. For example, $\emptyset\subsetneq\nats$, $\nats\sub\nats$ and
$\nats\subsetneq\ints$.

The definition of $\sub$ gives us the most common way of showing that
$A\sub B$: we suppose that $x\in A$, and show (with no additional
assumptions about $x$) that $x\in B$.  If we want to show that $A=B$,
it will suffice to show that $A\sub B$ and $B\sub A$, i.e., that
everything in $A$ is in $B$, and everything in $B$ is in $A$.  Of
course, we can also use the usual properties of equality to show set
equalities.  E.g., if $A=B$ and $B=C$, we have that $A=C$.

Note that, for all sets $A$, $B$ and $C$:
\begin{itemize}
\item if $A\sub B\sub C$, then $A\sub C$;

\item if $A\sub B\subsetneq C$, then $A\subsetneq C$;

\item if $A\subsetneq B\sub C$, then $A\subsetneq C$;

\item if $A\subsetneq B\subsetneq C$, then $A\subsetneq C$.
\end{itemize}

Given sets $A$ and $B$, we say that:
\begin{itemize}
\item $A$ is a \emph{superset}
\index{superset}%
\index{set!superset}%
of $B$ ($A\supseteq B$)
\index{ superset@$\supseteq$}%
\index{set! superset@$\supseteq$}%
iff, for all $x\in B$, $x\in A$;

\item $A$ is a
\emph{proper superset}
\index{proper!superset}%
\index{superset!proper}%
\index{set!superset!proper}%
of $B$ ($A\supsetneq B$)
\index{ proper superset@$\supsetneq$}%
\index{set! proper superset@$\supsetneq$}%
iff $A\supseteq B$ but $A\neq B$.
\end{itemize}
Of course, for all sets $A$ and $B$, we have that:
$A=B$ iff $A\supseteq B\supseteq A$; and $A\sub B$ iff $B\supseteq A$.
Furthermore, for all sets $A$, $B$ and $C$:
\begin{itemize}
\item if $A\supseteq B\supseteq C$, then $A\supseteq C$;

\item if $A\supseteq B\supsetneq C$, then $A\supsetneq C$;

\item if $A\supsetneq B\supseteq C$, then $A\supsetneq C$;

\item if $A\supsetneq B\supsetneq C$, then $A\supsetneq C$.
\end{itemize}

We write
\begin{displaymath}
  \textrm{for all}\; X \sub Y, P(X)
\end{displaymath}
as an abbreviation for
\begin{displaymath}
  \textrm{for all sets}\; X, X\sub Y \mathbin{\textrm{implies}} P(X) .
\end{displaymath}
And we write 
\begin{displaymath}
  \textrm{there exists an}\; X \sub Y \;\textrm{such that}\; P(X)
\end{displaymath}
as an abbreviation for
\begin{displaymath}
  \textrm{there exists a set}\; X \;\textrm{such that}\; X\sub Y
  \mathbin{\textrm{and}} P(X) .
\end{displaymath}
And we make use of similar abbreviations without comment.

\subsection{Set Formation}

We will make extensive use of the $\setof{\cdots}{\cdots}$ notation
for forming sets.
\index{forming sets|(}%
\index{set!formation|(}%
\index{ set of all@$\setof{\cdots}{\cdots}$}%
\index{set! set of all@$\setof{\cdots}{\cdots}$}%
Let's consider two representative examples of its use.

For the first example, let
\begin{gather*}
A=\setof{n}{n\in\nats\eqtxt{and}n^2\geq 20}
 =\setof{n\in\nats}{n^2\geq 20}.
\end{gather*}
(where the third of these expressions abbreviates the second one).
Here, $n$ is a bound variable
\index{bound variable}%
and is universally quantified%
\index{universally quantified}%
\index{quantification!universal}%
---changing it uniformly to $m$, for instance, wouldn't change the
meaning of $A$.  By the definition of $A$, we have that, for all $n$,
\begin{gather*}
n\in A\myiff n\in\nats\eqtxt{and}n^2\geq 20.
\end{gather*}
Thus, e.g.,
\begin{gather*}
5\in A\myiff 5\in\nats\eqtxt{and}5^2\geq 20 .
\end{gather*}
Since $5\in\nats$ and $5^2=25\geq 20$, it follows that $5\in A$.  On
the other hand, $5.5\not\in A$, since $5.5\not\in\nats$, and $4\not\in A$,
since $4^2\not\geq 20$.

For the second example, let
\begin{gather*}
B=\setof{n^3+m^2}{n,m\in\nats\eqtxt{and}n,m\geq 1} .
\end{gather*}
Note that $n^3+m^2$ is a term (expression), rather than a variable.
The variables $n$ and $m$ are existentially quantified,
\index{existentially quantified}%
\index{quantification!existential}%
rather than universally quantified, so that, for all $l$,
\begin{align*}
l\in B&\myiff l = n^3 + m^2, \eqtxt{for some}
         n, m\eqtxt{such that}n,m\in\nats\eqtxt{and}n,m\geq 1 \\
      &\myiff l = n^3 + m^2, \eqtxt{for some}
         n, m\in\nats\eqtxt{such that}n,m\geq 1.
\end{align*}
Thus, to show that $9\in B$, we would have to show that
\begin{gather*}
9 = n^3 + m^2\eqtxt{and}n,m\in\nats\eqtxt{and}n,m\geq 1 ,
\end{gather*}
for some values of $n,m$.  And, this holds, since $9 = 2^3 +
1^2\eqtxt{and}2,1\in\nats\eqtxt{and}2,1\geq 1$.

We use set formation in the following definition.
Given $n,m\in\ints$, we write $[n:m]$ for $\setof{l\in\ints}{l\geq
  n\eqtxt{and}l\leq m}$.
\index{integers!interval}%
\index{integers! interval@$[\cdot:\cdot]$}%
\index{natural numbers!interval}%
\index{natural numbers! interval@$[\cdot:\cdot]$}%
\index{ interval@$[\cdot:\cdot]$}%
Thus $[n:m]$ is all of the integers that are at least $n$ and no
more than $m$.  For example, $[-2:1]$ is $\{-2,-1,0,1\}$ and
$[3:2]$ is $\emptyset$.

Some uses of the $\setof{\cdots}{\cdots}$ notation are too ``big'' to
be sets, and instead are \emph{proper classes}. A \emph{class} is a
collection of universe elements, and a class is \emph{proper} iff it
is not a set.  E.g., $\setof{A}{A\eqtxt{is a set and} A\not\in A}$ is
a proper class: assuming that it is a set is inconsistent---makes it
possible to prove anything. This is the so-called Russell's Paradox.
\index{Russell's Paradox}%
To know that a set formation is valid, it suffices to find a set that
includes all the elements in the class being defined.  E.g., the
definitions of $A$ and $B$ are valid, because the classes being
defined are subsets of $\nats$.
\index{forming sets|)}%
\index{set!formation|)}%

\subsection{Operations on Sets}

Next, we consider some standard operations
on sets.  Recall the following operations on sets $A$ and $B$:
\begin{alignat*}{2}
A\cup B&=\setof{x}{x\in A\eqtxt{or}x\in B}
&&\by{union}\\
\index{ union@$\cup$}%
\index{set! union@$\cup$}%
\index{union!set}%
\index{set!union|see{union, set}}%
A\cap B&=\setof{x}{x\in A\eqtxt{and}x\in B}
&&\by{intersection} \\
\index{ intersection@$\cap$}%
\index{set! intersection@$\cap$}%
\index{intersection!set}%
\index{set!intersection|see{intersection, set}}%
A-B&=\setof{x\in A}{x\not\in B}
&&\by{difference} \\
\index{ difference@$-$}%
\index{set! difference@$-$}%
\index{difference!set}%
\index{set!difference}%
A\times B&=\setof{(x,y)}{x\in A\eqtxt{and}y\in B}
&&\by{product} \\
\index{ product@$\times$}%
\index{set! product@$\times$}%
\index{product}%
\index{set!product}%
\powset\,A&=\setof{X}{X\sub A}
&&\by{power set}.
\index{powerset@$\powset$}%
\index{set!powerset@$\powset$}%
\index{powerset}%
\index{set!powerset}%
\end{alignat*}
The axioms of set theory assert that all of these set formations
are valid (intersection and difference are obviously
valid).

Of course, union and intersection are both commutative
\index{commutative!union}%
\index{commutative!intersection}%
\index{union!set!commutative}%
\index{intersection!set!commutative}%
and associative
\index{associative!union}%
\index{associative!intersection}%
\index{union!set!associative}%
\index{intersection!set!associative}%
($A\cup B=B\cup A$, $(A\cup B)\cup C=A\cup
(B\cup C)$, $A\cap B=B\cap A$ and $(A\cap B)\cap C=A\cap (B\cap C)$,
for all sets $A, B, C$).
Furthermore, we have that union is idempotent
\index{idempotent!union}%
\index{union!set!idempotent}%
($A\cup A=A$, for all
sets $A$), and that $\emptyset$ is the identity
\index{identity!union}%
\index{union!set!identity}%
for union ($\emptyset\cup A=A=A\cup\emptyset$, for all sets $A$).
Also, intersection is idempotent
\index{idempotent!intersection}%
\index{intersection!set!idempotent}%
($A\cap A=A$, for all sets $A$), and
$\emptyset$ is the zero
\index{zero!intersection}%
\index{intersection!set!zero}%
for intersection ($\emptyset\cap
A=\emptyset=A\cap\emptyset$, for all sets $A$).

It is easy to see that, for all sets $X$ and $Y$,
$X\sub X\cup Y$ and $Y\sub X\cup Y$. We say that sets $X$ and
$Y$ are \emph{disjoint} iff $X\cap Y=\emptyset$, i.e., iff
\index{disjoint sets}%
\index{set!disjoint}%
$X$ and $Y$ have nothing in common.

$A-B$ is formed by removing the elements of $B$ from $A$, if
necessary.  For example, $\{0,1,2\}-\{1,4\}=\{0,2\}$.
$A\times B$ consists of all ordered pairs $(x,y)$,
\index{ordered pair}%
\index{ ordered pair@$(\cdot,\cdot)$}%
where $x$ comes from $A$ and $y$ comes from $B$.  For example,
$\{0,1\}\times\{1,2\}=\{(0,1),(0,2),(1,1),(1,2)\}$.  Remember that an
ordered pair $(x,y)$ is different from $\{x,y\}$, the set containing
just $x$ and $y$.  In particular, we have that, for all
$x$, $x'$, $y$, $y'$, $(x,y)=(x',y')$ iff
$x=x'$ and $y=y'$, whereas $\{1,2\}=\{2,1\}$.
If $A$ and $B$ have $n$ and $m$ elements, respectively, for
$n,m\in\nats$, then $A\times B$ will have $nm$ elements.  Finally,
$\powset\,A$ consists of all of the subsets of $A$.  For example,
$\powset\,\{0,1\}=\{\emptyset,\{0\},\{1\},\{0,1\}\}$.  If $A$ has $n$
elements, for $n\in\nats$, then $\powset\,A$ will have $2^n$ elements.

We let $\times$ associate to the right, so that, e.g.,
\index{ product@$\times$}%
\index{set! product@$\times$}%
$A\times B\times C = A\times(B\times C)$.  And, we abbreviate
$(x_1,(x_2,\cdots(x_{n-1},x_n)\cdots))$ to
$(x_1,x_2,\ldots,x_{n-1},x_n)$, thinking of it as an \emph{ordered}
$n$-\emph{tuple}.
\index{ordered tuple@ordered $n$-tuple}%
For example
$(x,(y,z))$ is abbreviated to $(x,y,z)$, and we think of it
as an \emph{ordered triple}.
\index{ordered triple}%
\index{ ordered triple@$(\cdot,\cdot,\cdot)$}%

As an example of a proof involving sets, let's prove
the following simple proposition, which says that intersections
may be distributed over unions:
\index{distributivity}%

\begin{proposition}
\label{InterOverUnionProp}
Suppose $A$, $B$ and $C$ are sets.

\begin{enumerate}[\quad(1)]
\item $A\cap(B\cup C)=(A\cap B)\cup(A\cap C)$.

\item $(A\cup B)\cap C=(A\cap C)\cup(B\cap C)$.
\end{enumerate}
\end{proposition}

\begin{proof}
We show (1), the proof of (2) being similar.
It will suffice to show that $A\cap(B\cup C)\sub(A\cap B)\cup(A\cap C)\sub
A\cap(B\cup C)$.

\begin{description}
\item[\quad($A\cap(B\cup C)\sub(A\cap B)\cup(A\cap C)$)] Suppose
$x\in A\cap(B\cup C)$.  We must show that
$x\in (A\cap B)\cup(A\cap C)$.
By our assumption, we have that  $x\in A$ and $x\in B\cup C$.
Since $x\in B\cup C$, there are two cases to consider.
\begin{itemize}
\item Suppose $x\in B$.  Then $x\in A\cap B\sub (A\cap B)\cup(A\cap C)$,
so that $x\in(A\cap B)\cup(A\cap C)$.

\item Suppose $x\in C$.  Then $x\in A\cap C\sub (A\cap B)\cup(A\cap C)$,
so that $x\in(A\cap B)\cup(A\cap C)$.
\end{itemize}

\item[\quad($(A\cap B)\cup(A\cap C)\sub A\cap(B\cup C)$)] Suppose
$x\in (A\cap B)\cup(A\cap C)$.  We must show that
$x\in A\cap(B\cup C)$.
There are two cases to consider.
\begin{itemize}
\item Suppose $x\in A\cap B$.  Then $x\in A$ and $x\in B\sub B\cup C$,
so that $x\in A\cap(B\cup C)$.

\item Suppose $x\in A\cap C$.  Then $x\in A$ and $x\in C\sub B\cup C$,
so that $x\in A\cap(B\cup C)$.
\end{itemize}
\end{description}
\end{proof}

\begin{exercise}
Suppose $A$, $B$ and $C$ are sets.  Prove that union distributes over
intersection, i.e., for all sets $A$, $B$ and $C$:
\begin{enumerate}[\quad(1)]
\item $A\cup(B\cap C)=(A\cup B)\cap(A\cup C)$.

\item $(A\cap B)\cup C=(A\cup C)\cap(B\cup C)$.
\end{enumerate}
\end{exercise}

Next, we consider generalized versions of union and intersection that
work on sets of sets.  If $X$ is a set of sets, then the
\emph{generalized union}
\index{generalized union}%
\index{union!set!generalized}%
of $X$ $(\bigcup X)$
\index{ generalized union@$\bigcup$}%
\index{set! generalized union@$\bigcup$}%
is
\begin{gather*}
\setof{a}{a\in A,\eqtxtr{for some}A\in X} .
\end{gather*}
Thus, to show that $a\in\bigcup X$, we must show that $a$ is in at
least one element $A$ of $X$.  For example
\begin{align*}
\bigcup\{\{0,1\},\{1,2\},\{2,3\}\} &=
\{0,1,2,3\}=\{0,1\}\cup\{1,2\}\cup\{2,3\}, \\
\bigcup\emptyset &= \emptyset.
\end{align*}
(Again, we rely on set theory's axioms to know this set formation
is valid.)

If $X$ is a \emph{nonempty} set of sets, then the \emph{generalized
intersection}
\index{generalized intersection}%
\index{intersection!set!generalized}%
of $X$ $(\bigcap X)$
\index{ generalized intersection@$\bigcap$}%
\index{set! generalized intersection@$\bigcap$}%
is
\begin{gather*}
\setof{a}{a\in A,\eqtxtr{for all}A\in X} .
\end{gather*}
Thus, to show that $a\in\bigcap X$, we must show that
$a$ is in every element $A$ of $X$.
For example
\begin{gather*}
\bigcap\{\{0,1\},\{1,2\},\{2,3\}\} =
\emptyset=\{0,1\}\cap\{1,2\}\cap\{2,3\} .
\end{gather*}

If we allowed $\bigcap\emptyset$, then it would contain all elements
$a$ of our universe that are in all of the nonexistent elements of
$\emptyset$, i.e., it would contain all elements of our universe. But
this collection is a proper class, not a set. Let's consider the above
reasoning in more detail.  Suppose $a$ is a universe element. To prove
that $a\in A$, for all $A\in\emptyset$, suppose $A\in\emptyset$. But this is
a logical contradiction, from which we may prove anything, in particular our
\index{logical contradiction}%
\index{logic!contradiction}%
desired conclusion, $a\in A$.

\subsection{Relations and Functions}

Next, we consider relations and functions.  A \emph{relation}
\index{relation}%
$R$ is a set of ordered pairs.  The \emph{domain}
\index{domain}%
\index{relation!domain}%
of a relation $R$ ($\domain\,R$)
\index{domain@$\domain$}%
\index{relation!domain@$\domain$}%
is $\setof{x}{(x,y)\in R, \eqtxtr{for
some}y}$, and the \emph{range}
\index{range}%
\index{relation!range}%
of $R$ ($\range\,R$)
\index{range@$\range$}%
\index{relation!range@$\range$}%
is $\setof{y}{(x,y)\in R,
\eqtxtr{for some}x}$.  We say that $R$ is a \emph{relation from} a
\index{relation from set to set}%
set $X$ \emph{to} a set $Y$ iff $\domain\,R\sub X$ and $\range\,R\sub Y$,
and that $R$ is a \emph{relation on} a set $A$ iff
$\domain\,R\cup\range\,R\sub A$.  We often write $x\mathrel{R}y$ for
$(x, y)\in R$.

Consider the relation
\begin{gather*}
R=\{(0,1),(1,2),(0,2)\} .
\end{gather*}
Then, $\domain\,R=\{0,1\}$, $\range\,R=\{1,2\}$, $R$ is a relation from
$\{0,1\}$ to $\{1,2\}$, and $R$ is a relation on $\{0,1,2\}$.
Of course, $R$ is also, e.g., a relation between $\nats$ and $\reals$,
as well as relation on $\reals$.

We often form relations using set formation.
\index{forming sets}%
\index{set!formation}%
\index{ set of all@$\setof{\cdots}{\cdots}$}%
\index{set! set of all@$\setof{\cdots}{\cdots}$}%
For example, suppose
$\phi(x,y)$ is a formula involving variables $x$ and $y$.  Then we can
let $R=\setof{(x,y)}{\phi(x,y)}$, and it is easy to show that, for all
$x, y$, $(x,y)\in R$ iff $\phi(x,y)$.

Given a set $A$, the \emph{identity relation}
\index{identity relation}%
\index{relation!identity}%
on $A$ ($\id_A$)
\index{id@$\id$}%
\index{relation!id@$\id$}%
is $\setof{(x,x)}{x\in A}$.  For example, $\id_{\{1,3,5\}}$ is
$\{(1,1),(3,3),(5,5)\}$.
Given relations $R$ and $S$, the
\emph{composition of}
\index{relation!composition}%
\index{composition!relation|see{relation, composition}}%
\index{associative!relation composition}%
$S$ and $R$ ($S\circ R$)
\index{ composition@$\circ$}%
\index{relation! composition@$\circ$}%
is $\setof{(x,z)}{(x,y)\in R\eqtxt{and} (y,z)\in S,\eqtxt{for some} y}$.
Intuitively, it's the relation formed by starting with a pair
$(x,y)\in R$, following it with a pair $(y,z)\in S$ (one that begins
where the pair from $R$ left off), and suppressing the intermediate
value $y$, leaving us with $(x,z)$.  For example, if $R=\{(1, 1), (1,
2), (2, 3)\}$ and $S=\{(2, 3), (2, 4), (3, 4)\}$, then from
$(1,2)\in R$ and $(2,3)\in S$, we can conclude $(1,3)\in S\circ R$.
There are two more pairs in $S\circ R$, giving us
$S\circ R=\{(1, 3), (1, 4), (2, 4)\}$.  For all sets $A$, $B$ and $C$,
and relations $R$ and $S$, if $R$ is a relation from $A$ to $B$,
and $S$ is a relation from $B$ to $C$, then $S\circ R$ is a relation
from $A$ to $C$.

It is easy to show that $\circ$ is associative
\index{associative!relation composition}%
\index{relation!composition!associative}%
\index{identity!relation composition}%
\index{relation!composition!identity}%
and has the identity relations as its identities:
\begin{enumerate}[\quad(1)]
\item For all sets $A$ and $B$, and relations $R$ from $A$ to $B$,
$\id_B\circ R = R = R\circ\id_A$.

\item For all sets $A$, $B$, $C$ and $D$, and relations
$R$ from $A$ to $B$, $S$ from $B$ to $C$, and $T$ from $C$ to $D$,
$(T\circ S)\circ R = T\circ (S\circ R)$.
\end{enumerate}
Because of (2), we can write $T\circ S\circ R$, without worrying about
how it is parenthesized.

The \emph{inverse}
\index{relation!inverse}%
of a relation $R$ ($R^{-1}$) is the relation
$\setof{(y, x)}{(x, y)\in R}$, i.e., it is the relation obtained by
reversing each of the pairs in $R$.  For example, if
$R=\{(0, 1), (1, 2), (1, 3)\}$, then the inverse of $R$ is
$\{(1, 0), (2, 1), (3, 1)\}$.  So for all sets $A$ and $B$, and
relations $R$, $R$ is a relation from $A$ to $B$ iff $R^{-1}$ is a
relation from $B$ to $A$. We also have that, for all sets $A$ and $B$
and relations $R$ from $A$ to $B$, $(R^{-1})^{-1} = R$. Furthermore,
for all sets $A$, $B$ and $C$, relations $R$ from $A$ to $B$, and
relations $S$ from $B$ to $C$, $(S\circ R)^{-1} = R^{-1}\circ S^{-1}$.

A relation $R$ is:
\begin{itemize}
\item \emph{reflexive on}
\index{reflexive on set}%
\index{relation!reflexive on set}%
a set $A$ iff, for all $x\in A$, $(x,x)\in R$;

\item \emph{transitive}
\index{transitive}%
\index{relation!transitive}%
iff, for all $x,y,z$, if $(x,y)\in R$ and
$(y,z)\in R$, then $(x, z)\in R$;

\item \emph{symmetric}
\index{symmetric}%
\index{relation!symmetric}%
iff, for all $x,y$, if $(x,y)\in R$, then $(y, x)\in R$;

\item \emph{antisymmetric}
\index{antisymmetric}%
\index{relation!antisymmetric}%
iff, for all $x,y$, if $(x,y)\in R$ and $(y,x)\in R$, then $x=y$;

\item \emph{total on}
\index{total}%
\index{relation!total}%
a set $A$ iff, for all $x,y\in A$, $(x,y)\in R$ or $(y,x)\in R$;

\item a \emph{function}
\index{function}%
\index{relation!function|see{function}}%
iff, for all $x,y,z$, if $(x,y)\in R$ and $(x, z)\in R$, then $y=z$.
\end{itemize}
Note that being antisymmetric is \emph{not} the same as not being symmetric.

Suppose, e.g., that $R=\{(0,1),(1,2),(0,2)\}$.
Then:
\begin{itemize}
\item $R$ is not reflexive on $\{0,1,2\}$, since $(0,0)\not\in R$.

\item $R$ is transitive, since whenever $(x,y)$ and $(y,z)$
are in $R$, it follows that $(x,z)\in R$.  Since
$(0,1)$ and $(1,2)$ are in $R$, we must have that $(0,2)$ is in $R$,
which is indeed true.

\item $R$ is not symmetric, since $(0,1)\in R$, but $(1,0)\not\in R$.

\item $R$ is antisymmetric, since there are no $x,y$ such
  that $(x,y)$ and $(y,x)$ are both in $R$.  (If we added
  $(1,0)$ to $R$, then $R$ would not be antisymmetric, since
  then $R$ would contain $(0,1)$ and $(1,0)$, but $0\neq 1$.)

\item $R$ is not total on $\{0,1,2\}$, since $(0,0)\not\in R$.

\item $R$ is not a function, since $(0,1)\in R$ and $(0,2)\in R$.
Intuitively, given an input of $0$, it's not clear whether
$R$'s output is $1$ or $2$.
\end{itemize}

We say that $R$ is a \emph{total ordering on} a set $A$ iff
\index{total ordering}%
$R$ is a transitive, antisymmetric and total relation on $A$.
It is easy to see that such an $R$ will also be reflexive on $A$.
Furthermore, if $R$ is a total ordering on $A$, then $R^{-1}$ is also a
total ordering on $A$.

We often use a symbol like $\leq$ to stand for a total ordering on a
set $A$, which lets us use its mirror image, $\geq$, for its inverse,
as well as to write $<$ for the relation on $A$ defined by: for all
$x,y\in A$, $x<y$ iff $x\leq y$ but $x\neq y$. $<$ is a \emph{strict
  total ordering on} $A$, i.e., a transitive relation on $A$ such
that, for all $x,y\in A$, exactly one of $x<y$, $x=y$ and $y<x$
holds. We write $>$ for the inverse of $<$.  We can also start with
a strict total ordering $<$ on $A$, and then define a total ordering
$\leq$ on $A$ by: $x\leq y$ iff $x<y$ or $x=y$.
The relations $\leq$ and $<$ on the natural numbers are
examples of such relations.

The relation
\begin{gather*}
f=\{(0,1),(1,2),(2,0)\}
\end{gather*}
is a function.  We think of it as sending the input $0$ to the
output $1$, the input $1$ to the output $2$, and the
input $2$ to the output $0$.

If $f$ is a function and $x\in\domain\,f$, we write $f\,x$ for the
\emph{application}
\index{application|see{function, application}}%
\index{function!application}%
\index{ function application@$\cdot\,\cdot$}%
\index{function! application@$\cdot\,\cdot$}%
\emph{of} $f$ \emph{to} $x$, i.e., the unique $y$ such that $(x,y)\in f$.
We say that $f$ is a \emph{function from}
\index{function!from set to set}%
a set $X$ \emph{to} a set $Y$ iff $f$ is a function, $\domain\,f=X$
and $\range\,f\sub Y$. Such an $f$ is also a function from
$X$ to $\range\,f$. We write $X\fun Y$
\index{ set of all functions@$\fun$}%
\index{set! set of all functions@$\fun$}%
for the set of all functions from $X$ to $Y$.
If $A$ has $n$ elements and $B$ has $m$ elements, for $n,m\in\nats$,
then $A\fun B$ will have $m^n$ elements.

For the $f$ defined above, we have that $f\,0=1$, $f\,1=2$,
$f\,2=0$, $f$ is a function from $\{0,1,2\}$ to $\{0,1,2\}$,
and $f\in\{0,1,2\}\fun\{0,1,2\}$.  Of course, $f$ is
also in $\{0,1,2\}\fun\nats$, but it is not in
$\nats\fun\{0,1,2\}$.

\begin{exercise}
Suppose $X$ is a set, and $x\in X$.  What are the elements of
$\emptyset\fun X$, $X\fun\emptyset$, $\{x\}\fun X$ and $X\fun\{x\}$?
Prove that your answers are correct.
\end{exercise}

We let $\fun$ associate to the right and have lower precedence than
$\times$, so that, e.g., $A\times B\fun C\times D\fun E\times F$ means
$(A\times B)\fun((C\times D)\fun(E\times F))$.  An element of this set
takes in a pair $(a,b)$ in $A\times B$, and returns a function
that takes in a pair $(c,d)$ in $C\times D$, and returns an
element of $E\times F$.

Suppose $f,g\in A\fun B$.  It is easy to show that $f=g$ iff, for all
$x\in A$, $f\,x=g\,x$.  This is the most common way of showing
the equality of functions.
\index{function!equality}%

Given a set $A$, it is easy to see that $\id_A$, the identity
relation on $A$, is a function from $A$ to $A$, and we call it
the \emph{identity function}
\index{identity relation}%
\index{relation!identity}%
\index{identity function}%
\index{function!identity}%
\index{id@$\id$}%
\index{function!id@$\id$}%
on $A$.  It is the function that returns its input.
Given sets $A$, $B$ and $C$, if $f$ is a function from $A$ to $B$, and
$g$ is a function from $B$ to $C$, then the composition $g\circ f$
\index{relation!composition}%
\index{function!composition}%
\index{composition!function|see{function, composition}}%
\index{ composition@$\circ$}%
\index{relation! composition@$\circ$}%
\index{function! composition@$\circ$}%
of (the relations) $g$ and $f$ is the function $h$ from $A$ to $C$ such
that $h\,x=g(f\,x)$, for all $x\in A$.  In other words, $g\circ f$ is
the function that runs $f$ and then $g$, in sequence.  Because
of how composition of relations works, we
have that $\circ$ is associative
\index{associative!function composition}%
\index{function!composition!associative}%
\index{identity!function composition}%
\index{function!composition!identity}%
and has the identity functions as its identities:
\begin{enumerate}[\quad(1)]
\item For all sets $A$ and $B$, and functions $f$ from $A$ to $B$,
$\id_B\circ f = f = f\circ\id_A$.

\item For all sets $A$, $B$, $C$ and $D$, and functions
$f$ from $A$ to $B$, $g$ from $B$ to $C$, and $h$ from $C$ to $D$,
$(h\circ g)\circ f = h\circ (g\circ f)$.
\end{enumerate}
Because of (2), we can write $h\circ g\circ f$, without worrying about
how it is parenthesized.  It is the function that runs $f$, then
$g$, then $h$, in sequence.

Given $f\in X\fun Y$ and a subset $A$ of $X$, we write
$f(A)$ for the \emph{image of} $A$ \emph{under} $f$,
\index{image under!function|see{function, image under}}%
\index{function!image under}%
\index{function! image under@$\cdot(\cdot)$}%
\index{ image under@$\cdot(\cdot)$}%
$\setof{f\,x}{x\in A}$.  And given $f\in X\fun Y$ and a subset
$B$ of $Y$, we write $f^{-1}(B)$ for the \emph{inverse image of}
\index{inverse image under!function|see{function, inverse image under}}%
\index{function!inverse image under}%
\index{function! inverse image under@$\cdot^{-1}(\cdot)$}%
\index{ inverse image under@$\cdot^{-1}(\cdot)$}%
$B$ \emph{under} $f$, $\setof{x\in X}{f\,x\in B}$.
For example, if $f\in\nats\fun\nats$ is the function that
doubles its argument, then $f(\{3, 5, 7\})=\{6, 10, 14\}$ and
$f^{-1}(\{1, 2, 3, 4\}) = \{1,2\}$.

Given a function $f$ and a set $X\sub\domain\,f$, we write
$f\restr X$ for the \emph{restriction of} $f$ \emph{to} $X$,
\index{restriction!function|see{function, restriction}}%
\index{function!restriction}%
\index{function! restriction@$\cdot\restr\cdot$}%
\index{ restriction@$\cdot\restr \cdot$}%
$\setof{(x,y)\in f}{x\in X}$. Hence $\domain(f\restr X)=X$.  For example,
if $f$ is the function over $\ints$ that increases its
argument by $2$, then $f\restr \nats$ is the function over $\nats$ that
increases its argument by $2$.
Given a function $f$ and elements $x,y$ of our universe,
we write $f[x\mapsto y]$ for the \emph{updating of} $f$
\index{updating!function|see{function, updating}}%
\index{function!updating}%
\index{function! updating@$\cdot[\cdot\mapsto\cdot]$}%
\index{ updating@$\cdot[\cdot\mapsto\cdot]$}%
\emph{to send} $x$ \emph{to} $y$,
$(f\restr(\domain\,f-\{x\}))\cup\{(x,y)\}$.  This function is
the same as $f$, except that it sends $x$ to $y$.
For example, if $f=\{(0, 1), (1, 2)\}$, then
$f[1\mapsto 0]=\{(0,1), (1, 0)\}$, and $f[2\mapsto 3]=
\{(0, 1), (1, 2), (2,3)\}$.

We often define a function by saying how an element of its domain
is transformed into an element of its range.  E.g.,
we might say that $f\in\nats\fun\ints$ is the unique function such that,
for all $n\in\nats$,
\begin{displaymath}
f\,n =
\casesdef{-(n/2),}{\eqtxtr{if}n\eqtxtl{is even},}%
         {(n + 1)/2,}{\eqtxtr{if}n\eqtxtl{is odd}.}
\end{displaymath}
This is shorthand for saying that $f$ is the set of all
$(n, m)$ such that
\begin{itemize}
\item $n\in\nats$,

\item if $n$ is even, then $m=-(n/2)$, and

\item if $n$ is odd, then $m=(n + 1)/2$.
\end{itemize}
Then $f\,0 = 0$, $f\,1 = 1$, $f\,2 = -1$, $f\, 3 = 2$,
$f\,4 = -2$, etc.

\begin{exercise}
There are three things wrong with the following ``definition''---what
are they?
Let $f\in\nats\fun\nats$ be the unique function such that,
for all $n\in\nats$,
\begin{displaymath}
f\,n =
\casesdef{n-2,}{\eqtxtr{if}n\geq 1\eqtxt{and}n\leq 10,}%
         {n+2,}{\eqtxtr{if}n\geq 10.}
\end{displaymath}
\end{exercise}

If $X_1,\ldots,X_n$ are sets, for $n\geq 1$, we write \index{product}%
\index{ hash@$\hash{\cdot}$}%
\index{projection}%
\index{tuple!projection}%
$\hash{i}_{X_1,\ldots,X_n}$ (or just $\hash{i}$, if $n$ and the $X_i$
are clear from the context) for the $i$-th \emph{projection} function
from $X_1\times\cdots\times X_n$ to $X_i$, which selects the $i$-th component of
a tuple, i.e., transforms an input $(x_1,\ldots,x_n)$ to $x_i$.

One of the forms of set theory's Axiom of Choice
\index{Axiom of Choice}%
says that, for all sets $X$ of nonempty sets, there is a function $f$
with domain $X$ such that, for all $Y\in X$, $f\,Y\in Y$. In other
words, $f$ is a \emph{choice function}
\index{choice function}%
that, given an element $Y$ of $X$, is able to pick an element of $Y$.

\subsection{Set Cardinality}

\index{set!cardinality|(}%
\index{set!size|(}%
\index{size!set|(}%
\index{cardinality|(}%
Next, we see how we can use functions to compare the sizes
(or cardinalities) of sets.
A \emph{bijection}
\index{bijection from set to set}%
\index{function!bijection from set to set}%
$f$ \emph{from} a set $X$ \emph{to} a set $Y$ is
a function from $X$ to $Y$ such that, for all $y\in Y$, there is a
unique $x\in X$ such that $(x,y)\in f$.
For example,
\begin{gather*}
f=\{(0,5.1),(1,2.6),(2,0.5)\}
\end{gather*}
is a bijection from $\{0,1,2\}$ to $\{0.5,2.6,5.1\}$.
We can visualize $f$ as a one-to-one correspondence
\index{one-to-one correspondence}%
between these sets:
\begin{center}
\input{chap-1.1-fig1.eepic}
\end{center}

A function $f$ is an \emph{injection} (or is \emph{injective})
\index{injection}%
\index{injective}%
\index{function!injection}%
\index{function!injective}%
iff, for all $x$, $y$ and $z$, if $(x,z)\in f$ and $(y,z)\in f$, then
$x=y$, i.e., for all $x,y\in\domain\,f$, if $f\,x=f\,y$, then $x=y$.
In other words, a function is injective iff it never sends two
different elements of its domain to the same element of its range.
For example, the function
\begin{gather*}
\{(0, 1), (1, 2), (2, 3), (3, 0)\}
\end{gather*}
is injective, but the function
\begin{gather*}
\{(0, 1), (1, 2), (2, 1)\}
\end{gather*}
is not injective (both $0$ and $2$ are sent to $1$).  We say that $f$
is an \emph{injection from} a set $X$ \emph{to} a set $Y$, iff $f$ is
\index{injection from set to set}%
\index{function!injection from set to set}%
a function from $X$ to $Y$ and $f$ is injective.

It is easy to see that:
\begin{itemize}
\item For all sets $A$, $\id_A$ is injective.

\item For all sets $A$, $B$ and $C$, functions $f$ from $A$ to $B$,
and functions $g$ from $B$ to $C$, if $f$ and $g$ are injective,
then so is $g\circ f$.
\end{itemize}

\begin{exercise}
\label{BijectionEx1}
Suppose $A$ and $B$ are sets.  Show that for all $f$,
$f$ is a bijection from $A$ to $B$ iff
\begin{itemize}
\item $f$ is a function from $A$ to $B$;

\item $\range\,f = B$; and

\item $f$ is injective.
\end{itemize}
\end{exercise}

Consequently, if $f$ is an injection from $A$ to $B$, then $f$ is
a bijection from $A$ to $\range\,f\sub B$.

\begin{exercise}
\label{BijectionEx2}
Show that:
\begin{enumerate}[\quad(1)]
\item For all sets $A$, $\id_A$ is a bijection from $A$ to $A$.

\item For all sets $A$, $B$ and $C$, bijections $f$ from $A$ to $B$,
and bijections $g$ from $B$ to $C$, $g\circ f$ is a bijection from $A$ to $C$.

\item For all sets $A$ and $B$, and bijections $f$ from $A$ to $B$,
$f^{-1}$ is a bijection from $B$ to $A$.
\end{enumerate}
\end{exercise}

We say that a set $X$ is \emph{equinumerous} to a set $Y$
\index{equinumerous}%
\index{set!equinumerous}%
($X\cong Y$)
\index{ equinumerous@$\cong$}%
\index{set! equinumerous@$\cong$}%
iff there is a bijection from $X$ to $Y$.  By Exercise~\ref{BijectionEx2},
we have that, for all sets $A,B,C$:
\begin{enumerate}[\quad (1)]
\item $A\cong A$;

\item If $A\cong B\cong C$, then $A\cong C$; and

\item If $A\cong B$, then $B\cong A$.
\end{enumerate}

We say that a set $X$ is:
\begin{itemize}
\item \emph{finite} iff $X\cong[1:n]$, for some $n\in\nats$
\index{finite}%
\index{set!finite}%
(recall that $[1:n]$ is all of the natural numbers that are at least 1
and no more than $n$, so that $[1:0]=\emptyset$);

\item \emph{infinite} iff it is not finite;
\index{infinite}%
\index{set!infinite}%

\item \emph{countably infinite} iff $X\cong\,\nats$;
\index{countably infinite}%
\index{infinite!countably|see{countably infinite}}%
\index{set!infinite!countably|see{countably infinite}}%

\item \emph{countable} iff $X$ is either finite or countably infinite; and
\index{countable}%
\index{set!countable}%

\item \emph{uncountable} iff $X$ is not countable.
\index{uncountable}%
\index{set!uncountable|see{uncountable}}%
\end{itemize}

Every set $X$ has a \emph{size} or \emph{cardinality} ($|X|$) \index{
  size of@$\sizedot$}%
\index{ size of@$\sizedot$}%
\index{set! size of@$\sizedot$}%
and we have that, for all sets $X$ and $Y$, $|X|=|Y|$ iff $X\cong
Y$. The sizes of finite sets are natural numbers.

We have that:
\begin{itemize}
\item The sets $\emptyset$ and $\{0.5,2.6,5.1\}$ are finite, and are thus
also countable;

\item The sets $\nats$, $\ints$, $\reals$ and $\powset\,\nats$ are infinite;

\item The set $\nats$ is countably infinite, and is thus countable; and

\item The set $\ints$ is countably infinite, and is thus countable,
because of the existence of the following bijection:
\begin{center}
\input{chap-1.1-fig2.eepic}
\end{center}

\item The sets $\reals$ and $\powset\,\nats$ are uncountable.
\end{itemize}

Because $\nats\cong\ints$, we see that for infinite sets $X$ and $Y$,
it is possible that $X\cong Y$ even though $X$ is a proper subset of $Y$.
With finite sets this is of course impossible.

We have that if $\cal W$ is a finite set of finite sets, that
$\bigcup{\cal W}$ is finite.

To prove that $\reals$ and $\powset\,\nats$ are uncountable, we can
use an important technique called ``diagonalization'',
\index{diagonalization}%
which we will see again in
Chapter~\ref{RecursiveAndRecursivelyEnumerableLanguages}.  Let's consider
the proof that $\powset\,\nats$ is uncountable.

We proceed using proof by contradiction.
\index{logic!proof by contradiction}%
\index{proof by contradiction}%
Suppose $\powset\,\nats$ is countable.  If we can obtain a
contradiction, it will follow that $\powset\,\nats$ is uncountable.
Since $\powset\,\nats$ is not finite, it follows that there is a
bijection $f$ from $\nats$ to $\powset\,\nats$.  Our plan is to define
a subset $X$ of $\nats$ such that $X\not\in\range\,f$, thus obtaining
a contradiction, since this will show that $f$ is not a bijection from
$\nats$ to $\powset\,\nats$.

Consider the infinite table in which both the rows and the columns are
indexed by the elements of $\nats$, listed in ascending order, and
where a cell $(m, n)$ ($m$ is the row, $n$ is the column) contains
$1$ iff $m\in f\,n$, and contains $0$ iff $m\not\in f\,n$.  Thus the
$n$th column of this table represents the set $f\,n$ of natural
numbers.

Figure~\ref{DiagCard} shows how part of this table might look, where
$i$, $j$ and $k$ are sample elements of $\nats$.
\begin{figure}
\begin{center}
\input{chap-1.1-fig3.eepic}
\end{center}
\caption{Example Diagonalization Table for Cardinality Proof}
\label{DiagCard}
\end{figure}
Because of the table's data, we have, e.g., that $i\in f\,i$
and $j\not\in f\,i$.

To define our $X\sub\nats$, we work our way down the diagonal of the
table, putting $n$ into our set just when cell $(n,n)$ of the
table is $0$, i.e., when $n\not\in f\,n$.  This will ensure that, for
all $n\in\nats$, $f\,n\neq X$.
With our example table:
\begin{itemize}
\item since $i\in f\,i$, but $i\not\in X$, we have that $f\,i\neq X$;

\item since $j\not\in f\,j$, but $j\in X$, we have that $f\,j\neq X$; and

\item since $k\in f\,k$, but $k\not\in X$, we have that $f\,k\neq X$.
\end{itemize}

Next, we turn the above ideas into a shorter, but more opaque, proof
that:

\index{uncountable}%
\begin{proposition}
\label{PowsetNatsUncountableProp}
$\powset\,\nats$ is uncountable.
\end{proposition}

\begin{proof}
Suppose, toward a contradiction, that $\powset\,\nats$ is countable.
Because $\powset\,\nats$ is not finite, there is a bijection $f$
from $\nats$ to $\powset\,\nats$.  Define
$X\in\setof{n\in\nats}{n\not\in f\,n}$, so that $X\in\powset\,\nats$.
By the definition of $f$, it follows that $X=f\,n$, for some
$n\in\nats$.\footnote{Here we are doing an existential elimination,
introducing $n$ into our variable context, and introducing $X=f\,n$ into
our current assumptions. If we already had $n$ in our variable context, we
would have had to first rename it.}
\index{existential elimination}%
\index{logic!existential elimination}%
\index{variable context}%
\index{logic!variable context}%
There are two cases to consider.
\begin{itemize}
\item Suppose $n\in X$.  By the definition of $X$, it follows that
  $n\not\in f\,n=X$---contradiction.

\item Suppose $n\not\in X$.  Because $X=f\,n$, we have that $n\not\in
  f\,n$.  Thus, since $n\in\nats$ and $n\not\in f\,n$, it follows
  that $n\in X$---contradiction.
\end{itemize}
Since we obtained a contradiction in both cases, we have
an overall contradiction. Thus $\powset\,\nats$ is uncountable.
\end{proof}

We have seen how bijections may be used to determine whether sets have
the same size.  But how can we compare the relative sizes of sets,
i.e., say whether one set is smaller or larger than another?  The
answer is to make use of injective functions.
\index{dominated}%
\index{set!dominated}%
We say that a set $X$ is \emph{dominated} by a set $Y$ ($X\preceq Y$)
\index{ dominated@$\preceq$}%
\index{set! dominated@$\preceq$}%
iff there is an injection from $X$ to $Y$, i.e., an
injective function whose domain is $X$ and whose range
is a subset of $Y$. Because identity functions are injective,
we have that $X\sub Y$ implies $X\preceq Y$; e.g.,
$\nats\preceq\reals$.  This definition makes sense, because $X$ is
dominated by $Y$ iff $X$ is equinumerous to a subset of $Y$.
We say that a set $X$ is \emph{strictly dominated} by a set $Y$
\index{strictly dominated}%
\index{set!strictly dominated}%
($X\prec Y$)
\index{ strictly dominated@$\prec$}%
\index{set! strictly dominated@$\prec$}%
iff $X\preceq Y$ but $X\not\cong Y$.

Using our observations about injections, we have that,
for all sets $A$, $B$ and $C$:
\begin{enumerate}[\quad (1)]
\item $A\preceq A$;

\item If $A\preceq B\preceq C$, then $A\preceq C$.
\end{enumerate}
We can also characterize $\preceq$ using ``surjectivity''. We
say that $f$ is a \emph{surjection} from $X$ to $Y$ iff $f$ is a function
\index{surjection from set to set}%
\index{function!surjection from set to set}%
from $X$ to $Y$ and $\range\,f = Y$. A consequence of the
Axiom of Choice is that, for all sets $X$ and $Y$, $X\preceq Y$ iff
\index{Axiom of Choice}%
$X = \emptyset$ or there is a surjection from $Y$ to $X$.

Clearly, for all sets $A$ and $B$,
if $A\cong B$, then $A\preceq B\preceq A$.  And, a famous result
of set theory, the Schr\"oder-Bernstein Theorem,
\index{Schr\"oder-Bernstein Theorem}%
says that the converse holds, i.e.,
for all sets $A$ and $B$, if $A\preceq B\preceq A$, then
$A\cong B$. This give us a powerful method for proving that two
sets have the same size.

\begin{exercise}
Use the Schr\"oder-Bernstein Theorem to show that
$\nats\cong\nats\times\nats$.  Hint: use the following consequence of
the Fundamental Theorem of Arithmetic: if two finite, ascending (each
\index{Fundamental Theorem of Arithmetic}%
element is $\leq$ the next) sequences of prime numbers (natural
numbers that are at least $2$ and have no divisors other than $1$ and
themselves) have the same product (the product of the empty sequence is
$1$), then they are equal.
\end{exercise}

One of the forms of the Axiom of Choice
\index{Axiom of Choice}%
says that, for all sets $A$ and $B$, $A\preceq B$ or
$B\preceq A$, i.e., $A$ is dominated by $B$, or $B$ is
dominated by $A$. The sizes of sets are ordered in such a way that,
for all sets $A$ and $B$: $|A|\leq|B|$ iff $A\preceq B$; and $|A|<|B|$
iff $A\prec B$. $\leq$ and $<$ on set sizes have all the properties of a
total ordering and the corresponding strict total ordering, except
that the collection of all set sizes is a proper class, and so
$\leq$ and $<$ are also proper classes.

Given the above machinery, one can strengthen
Proposition~\ref{PowsetNatsUncountableProp} into: for
all sets $X$, $X\prec \powset\,X$, and so $|X|<|\powset\,X|$.
\index{set!cardinality|)}%
\index{set!size|)}%
\index{size!set|)}%
\index{cardinality|)}%

\subsection{Data Structures}

\index{data structure|(}%
\index{boolean|(}%
We conclude this section by introducing some data structures that
are built out of sets.
We write $\Bool$ for the set of booleans, $\{\true,\false\}$.
\index{true@$\true$}%
\index{false@$\false$}%
\index{Bool@$\Bool$}%
\index{boolean!true@$\true$}%
\index{boolean!false@$\false$}%
\index{boolean!Bool@$\Bool$}%
(We can actually let $\true=1$ and $\false=0$, although we'll
never make use of these equalities.)  We define the negation
function $\mynot\in\Bool\fun\Bool$ by:
\begin{gather*}
\mynot\,\true = \false , \qquad
\mynot\,\false = \true .
\end{gather*}
\index{not@$\mynot$}%
\index{boolean!not@$\mynot$}%
\index{boolean!negation}%
And the conjunction and disjunction operations on the booleans are
defined by:
\begin{align*}
\true\myand\true &= \true , \\
\true\myand\false &= \false\myand\true = \false\myand\false = \false ,
\end{align*}
\index{and@$\myand$}%
\index{boolean!and@$\myand$}%
\index{boolean!conjunction}%
and
\index{or@$\myor$}%
\index{boolean!or@$\myor$}%
\index{boolean!disjunction}%
\index{boolean|)}%
\begin{align*}
\true\myor\true &= \true\myor\false = \false\myor\true = \true , \\
\false\myor\false &= \false .
\end{align*}

Given a set $X$, we write $\Option\,X$ for
\index{option}%
\index{none@$\none$}%
\index{some@$\some\,\cdot$}%
\index{Option@$\Option$}%
\index{option!none@$\none$}%
\index{option!some@$\some\,\cdot$}%
\index{option!Option@$\Option$}%
$\{\none\}\cup\setof{\some\,x}{x\in X}$, where we define $\none=(0,0)$
and $\some\,x=(1,x)$, which guarantees that $\none=\some\,x$ can't
hold, and that $\some\,x=\some\,y$ only holds when $x=y$.
(We won't make use of the particular way we've defined $\none$ and
$\some\,x$.)
For example, $\Option\,\Bool=\{\none,\some\,\true,\some\,\false\}$.

The idea is that an element of $\Option\,X$ is an optional element
of $X$.  E.g., when a function needs to either return an element of $X$
or indicate that an error has occurred, it could return an
element of $\Option\,X$, using $\none$ to indicate an error,
and returning $\some\,x$ to indicate success with value $x$.
E.g., we could define a function
$f\in\nats\times\nats\fun\Option\,\Bool$ by:
\begin{gather*}
f(n,m) =
\left\{ \begin{array}{ll}
\none, & \eqtxt{if}m=0, \\
\some\,\true & \eqtxt{if}m\neq 0 \eqtxt{and}
n=ml \eqtxt{for some}l\in\nats, \\
\some\,\false & \eqtxt{if}m\neq 0 \eqtxt{and}
n\neq ml \eqtxt{for all}l\in\nats .
\end{array} \right.  
\end{gather*}

\index{list|(}%
Finally, we consider lists.  A \emph{list} is a function with domain
$[1:n]$, for some $n\in\nats$.  (Recall that $[1:n]$ is all of the
natural numbers that are at least 1 and no more than $n$.)
For example, $\emptyset$ is a list, as
it is a function with domain $\emptyset=[1:0]$.  And $\{(1, 3), (2,
5), (3, 7)\}$ is a list, as it is a function with domain $[1:3]$.
Note that, if $x$ is a list, then $|x|$, the size of the set $x$,
\index{ size of@$\sizedot$}%
\index{set! size of@$\sizedot$}%
doubles as the \emph{length} of $x$.

We abbreviate a list $\{(1,x_1),(2,x_2),\ldots,(n,x_n)\}$ to
$[x_1,x_2,\ldots,x_n]$.
\index{ list@$[\cdot,\cdots,\cdot]$}%
\index{lists! list@$[\cdot,\cdots,\cdot]$}%
Thus $\emptyset$ and $\{(1, 3), (2, 5), (3,
7)\}$ are abbreviated to $[\,]$ and $[3,5,7]$, respectively.
If $[x_1,x_2,\ldots,x_n]=[y_1,y_2,\ldots,y_m]$, it is easy to
see that $n=m$ and $x_i=y_i$, for all $i\in[1:n]$.

Given lists $f$ and $g$, the \emph{concatenation} of $f$ and $g$
($f\myconcat g$)
\index{list!concatenation}%
\index{concatenation!list}%
\index{ at@$\myconcat$}%
\index{list! at@$\myconcat$}%
is the list
\begin{displaymath}
f\cup\setof{(m+|f|,y)}{(m,y)\in g} .  
\end{displaymath}
For example,
\begin{align*}
[2, 3]\myconcat[4,5,6] &=
\{(1, 2), (2, 3)\}\myconcat\{(1, 4), (2, 5), (3, 6)\} \\
&= \{(1, 2), (2, 3)\} \cup
\setof{(m+2,y)}{(m,y)\in\{(1, 4), (2, 5), (3, 6)\}} \\
&= \{(1, 2), (2, 3)\} \cup
\{(1 + 2, 4), (2 + 2, 5), (3 + 2, 6)\} \\
&= \{(1, 2), (2, 3), (3, 4), (4, 5), (5, 6)\} \\
&= [2,3,4,5,6] .
\end{align*}

Given lists $f$ and $g$, it is easy to see that $|f\myconcat g| =
|f| + |g|$.  And, given $n\in[1:|f\myconcat g|]$,
we have that
\begin{displaymath}
(f\myconcat g)\,n =
\casesdef{f\,n,}{\eqtxtr{if}n\in[1:|f|],}%
         {g(n-|f|),}{\eqtxtr{if}n>|f| .}
\end{displaymath}
Using this fact, it is easy to prove that:
\begin{itemize}
\item $[\,]$ is the identity for concatenation: for all lists $f$,
\index{identity!list concatenation}%
\index{concatenation!list!identity}%
\index{list!concatenation!identity}%
  \begin{displaymath}
    [\,]\myconcat f = f = f\myconcat [\,] .
  \end{displaymath}

\item Concatenation is associative: for all lists $f$, $g$, $h$,
\index{associative!list concatenation}%
\index{concatenation!list!associative}%
\index{list!concatenation!associative}%
  \begin{displaymath}
    (f\myconcat g)\myconcat h = f\myconcat(g\myconcat h) .
  \end{displaymath}
\end{itemize}
Because concatenation is associative, we can write
$f\myconcat g\myconcat h$ without worrying where the parentheses go.

Given a set $X$, we write $\List\,X$ for the set of all
$X$-\emph{lists}, i.e., lists whose
\index{List@$\List\,\cdot$}%
\index{list!List@$\List\,\cdot$}%
\index{list!list@$\cdot$-list}%
\index{list@$\cdot$-list}%
ranges are subsets of $X$, i.e., all of whose elements come
from $X$.  E.g., $[\,]$ and $[3,5,7]$ are elements of $\List\,\nats$,
the set of all lists of natural numbers.  It is easy to see that
$[\,]\in\List\,X$, for all sets $X$, and that, for all sets $X$ and
$f,g\in\List\,X$, $f\myconcat g\in\List\,X$.
\index{list|)}%
\index{data structure|)}%

\subsection{Notes}

In set theory (see, e.g., \cite{Enderton77}), the natural numbers,
integers, real numbers and ordered pairs $(x,y)$ are encoded as sets.
But for our purposes, it is clearer to suppress this detail.

Furthermore, in set theory, $\nats$ is not a subset of $\ints$, and
$\ints$ is not a subset of $\reals$.  On the other hand, there are
proper subsets of $\reals$ corresponding to the natural numbers and
the integers, and these are what we take $\nats$ and $\ints$ to be, so
that $\nats\subsetneq\ints\subsetneq\reals$. Thus, for us, the sizes of
finite sets are also elements of the subset of $\reals$ that we
treat as the natural numbers.
\index{set|)}%

%%% Local Variables: 
%%% mode: latex
%%% TeX-master: "book"
%%% End: 
