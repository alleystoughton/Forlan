\section{Isomorphism of Finite Automata}
\label{IsomorphismOfFiniteAutomata}

\index{isomorphism!finite automaton|(}%
\index{finite automaton!isomorphism|(}%

\subsection{Definition and Algorithm}

Let $M$ and $N$ be the finite automata
\begin{center}
\input{chap-3.5-fig1.eepic}
\end{center}
How are $M$ and $N$ related?
Although they are not equal, they do have the same
``structure'', in that $M$ can be turned into $N$ by replacing
$\Asf$, $\Bsf$ and $\Csf$ by $\Asf$, $\Csf$ and $\Bsf$, respectively.
When FAs have the same structure, we will say they are ``isomorphic''.

In order to say more formally what it means for two FAs to be
isomorphic, we define the notion of an isomorphism from one FA to
another.  An {\em isomorphism}
\index{finite automaton!isomorphism from FA to FA}%
\index{isomorphism!finite automaton!isomorphism from FA to FA}%
$h$ from an FA $M$ to an FA $N$ is a bijection from $Q_M$ to $Q_N$
such that:
\begin{itemize}
\item $h\,s_M = s_N$;

\item $\setof{h\,q}{q\in A_M} = A_N$; and

\item $\setof{(h\,q), x\fun (h\,r)}{q,x\fun r\in T_M} =
T_N$.
\end{itemize}
We define a relation $\iso$ on $\FA$ by: $M\iso N$
\index{iso@$\iso$}%
\index{finite automaton!iso@$\iso$}%
\index{isomorphism!finite automaton!iso@$\iso$}%
iff there
is an isomorphism from $M$ to $N$.  We say that $M$ and $N$ are
{\em isomorphic\/}
\index{isomorphic!finite automaton}%
\index{finite automaton!isomorphic}%
\index{isomorphism!finite automaton!isomorphic}%
iff $M\iso N$.

Consider our example FAs $M$ and $N$, and
let $h$ be the function
\begin{gather*}
\mathsf{\{(A, A), (B, C), (C, B)\}}.
\end{gather*}
Then it is easy to check that $h$ is an isomorphism from $M$ to $N$.
Hence $M\iso N$.

Clearly, if $M$ and $N$ are isomorphic, then they have the same
alphabet.

\begin{proposition}
\label{IsoEquivProp}
The relation $\iso$ is reflexive on $\FA$, symmetric and transitive.
\index{reflexive on set!iso@$\iso$}%
\index{symmetry!iso@$\iso$}%
\index{transitive!iso@$\iso$}%
\index{iso@$\iso$!reflexive}%
\index{iso@$\iso$!symmetric}%
\index{iso@$\iso$!transitive}%
\end{proposition}

\begin{proof}
If $M$ is an FA, then $\id_M$ is an isomorphism from $M$ to $M$.

If $M,N$ are FAs, and $h$ is a isomorphism from $M$ to $N$, then
$h^{-1}$ is an isomorphism from $N$ to $M$.

If $M_1,M_2,M_3$ are FAs, $f$ is an isomorphism from $M_1$ to $M_2$,
and $g$ is an isomorphism from $M_2$ to $M_3$, then $g\circ f$ an
isomorphism from $M_1$ to $M_3$.
\end{proof}

Next, we see that, if $M$ and $N$ are isomorphic, then
every string accepted by $M$ is also accepted by $N$.

\begin{proposition}
\label{IsoSubProp}
Suppose $M$ and $N$ are isomorphic FAs.  Then $L(M)\sub L(N)$.
\end{proposition}

\begin{proof}
Let $h$ be an isomorphism from $M$ to $N$.  Suppose $w\in L(M)$.
Then, there is a labeled path
\begin{gather*}
\lp = q_1\lparr{x_1}q_2\lparr{x_2}\cdots\,q_n\lparr{x_n}q_{n+1} ,
\end{gather*}
such that $w=x_1x_2\cdots x_n$, $\lp$ is valid for $M$,
$q_1 = s_M$ and $q_{n+1}\in A_M$.  Let
\begin{gather*}
\lp' = h\,q_1\lparr{x_1}h\,q_2\lparr{x_2}\cdots\,h\,q_n
\lparr{x_n}h\,q_{n+1}.
\end{gather*}
Then the label of $\lp'$ is $w$, $\lp'$ is valid for $N$,
$h\,q_1=h\,s_M=s_N$ and $h\,q_{n+1}\in A_N$, showing that $w\in L(N)$.
\end{proof}

A consequence of the two preceding propositions is that
isomorphic FAs are equivalent.  Of course, the converse is
not true, in general, since there are many FAs that accept the
same language and yet don't have the same structure.

\begin{proposition}
Suppose $M$ and $N$ are isomorphic FAs. Then $M\approx N$.
\end{proposition}

\begin{proof}
Since $M\iso N$, we have that $N\iso M$, by Proposition~\ref{IsoEquivProp}.
Thus, by Proposition~\ref{IsoSubProp}, we have that $L(M)\sub L(N)\sub L(M)$.
Hence $L(M)=L(N)$, i.e., $M\approx N$.
\end{proof}

The function $\renameStates$ takes in a pair $(M,f)$, where $M\in\FA$
\index{finite automaton!renaming states}%
\index{renameStates@$\renameStates$}%
\index{finite automaton!renameStates@$\renameStates$}%
and $f$ is a bijection from $Q_M$ to some set of symbols, and returns
the $\FA$ produced from $M$ by renaming $M$'s states using the
bijection $f$.

\begin{proposition}
Suppose $M$ is an FA and $f$ is a bijection from $Q_M$ to some set of
symbols.  Then $\renameStates(M,f)\iso M$.
\end{proposition}

The following function is a special case of $\renameStates$.
The function $\renameStatesCanonically\in\FA\fun\FA$ renames the
\index{renameStatesCanonically@$\renameStatesCanonically$}%
\index{finite automaton!renameStatesCanonically@$\renameStatesCanonically$}%
states of an FA $M$ to:
\begin{itemize}
\item $\mathsf{A}$, $\mathsf{B}$, etc., when the automaton has no more
than 26 states (the smallest state of $M$ will be renamed to
$\mathsf{A}$, the next smallest one to $\mathsf{B}$, etc.); or

\item $\mathsf{\langle 1\rangle}$, $\mathsf{\langle 2\rangle}$, etc.,
otherwise.
\end{itemize}
Of course, the resulting automaton will always be isomorphic to the original
one.

Next, we consider an algorithm that finds an isomorphism from an FA
\index{finite automaton!isomorphism!checking whether FAs are isomorphic}%
\index{isomorphism!finite automaton!checking whether FAs are isomorphic}%
$M$ to an FA $N$, if one exists, and that indicates that no such
isomorphism exists, otherwise.

Our algorithm is based on the following lemma.
\begin{lemma}
\label{IsoLem}
Suppose that $h$ is a bijection from $Q_M$ to $Q_N$.  Then
\begin{gather*}
\setof{h\,q \tranarr{x} h\,r}{q \tranarr{x} r\in T_M} = T_N
\end{gather*}
iff, for all $(q, r)\in h$ and
$x\in\Str$, there is a subset of $h$ that is a bijection from
\begin{gather*}
\setof{p\in Q_M}{q \tranarr{x} p\in T_M}
\end{gather*}
to
\begin{gather*}
\setof{p\in Q_N}{r \tranarr{x} p\in T_N} .
\end{gather*}
\end{lemma}

If any of the following conditions are true, then the algorithm reports
that there is no isomorphism from $M$ to $N$:
\begin{itemize}
\item $|Q_M|\neq|Q_N|$;

\item $|A_M|\neq|A_N|$;

\item $|T_M|\neq|T_N|$;

\item $s_M\in A_M$, but $s_N\not\in A_N$; and

\item $s_N\in A_N$, but $s_M\not\in A_M$.
\end{itemize}
Otherwise, it calls its main function, $\findIso$, which is
defined by well-founded recursion.

The function $\findIso$ is called with an argument $(f,
[C_1,\ldots,C_n])$, where:
\begin{itemize}
\item $f$ is a bijection from a subset of $Q_M$ to a subset of $Q_N$, and

\item the $C_i$ are \emph{constraints} of the form $(X,Y)$, where
  $X\sub Q_M$, $Y\sub Q_N$ and $|X|=|Y|$.
\end{itemize}
It returns an element of $\Option\,X$, where $X$ is the set of
bijections from $Q_M$ to $Q_N$.  $\none$ is returned to indicate
failure, and $\some\,h$ is returned when it has produced the bijection $h$.
We say that a bijection \emph{satisfies} a constraint $(X,Y)$ iff it
has a subset that is a bijection from $X$ to $Y$.

We say that the \emph{weight} of a constraint $(X,Y)$ is $3^{|X|}$.
Thus, we have the following facts:
\begin{itemize}
\item If $(X,Y)$ is a constraint, then its weight is at least $3^0=1$.

\item If $(\{p\}\cup X,\{q\}\cup Y)$ is a constraint, $p\not\in X$,
  $q\not\in Y$ and $|X|\geq 1$, then the weight of $(\{p\}\cup
  X,\{q\}\cup Y)$ is $3^{1+|X|}=3\cdot 3^{|X|}$, the weight of
  $(\{p\},\{q\})$ is $3^1=3$, and the weight of $(X,Y)$ is $3^{|X|}$.
  Because $|X|\geq 1$, it follows that the sum of the weights of
  $(\{p\},\{q\})$ and $(X,Y)$ ($3 + 3^{|X|}$) is strictly less than
  the weight of $(\{p\}\cup X,\{q\}\cup Y)$.
\end{itemize}

Each argument to a recursive call of $\findIso$ will be strictly
smaller than the argument to the original call in the well-founded
\emph{termination relation} in which argument $(f, [C_1,\ldots,C_n])$
is less than argument $(f', [C'_1,\ldots,C'_m])$ iff either:
\begin{itemize}
\item $|f|>|f'|$ (remember that $|f|\leq|Q_M|=|Q_N|$); or

\item $|f|=|f'|$ but the sum of the weights of the constraints
  $C_1,\,\ldots,C_n$ is strictly less than the sum of the weights of
  the constraints $C'_1, \,\ldots,C'_m$.
\end{itemize}

When $\findIso$ is called with argument $(f, [C_1,\ldots,C_n])$, the
following property, which we call (\dag), will hold: for all bijections
$h$ from a subset of $Q_M$ to a subset of $Q_N$, if $h\supseteq f$ and $h$
satisfies all of the $C_i$'s, then:
\begin{itemize}
\item $h$ is a bijection from $Q_M$ to $Q_N$;

\item $h\,s_M = s_N$;

\item $\setof{h\,q}{q\in A_M} = A_N$; and

\item for all $(q, r)\in f$ and $x\in\Str$, there is a subset of $h$
  that is a bijection from $\setof{p\in Q_M}{q,x\fun p\in T_M}$ to
  $\setof{p\in Q_N}{r,x\fun p\in T_N}$.
\end{itemize}
Thus, if $\findIso$ is called with a bijection $f$ and an empty list
of constraints, it will follow, by Lemma~\ref{IsoLem},
that $f$ is an isomorphism from $M$ to $N$.

Initially, the algorithm calls $\findIso$ with the \emph{initial argument}:
\begin{displaymath}
(\emptyset, [(\{s_M\}, \{s_N\}), (U, V), (X, Y)]) ,
\end{displaymath}
where $U = A_M - \{s_M\}$, $V = A_N - \{s_N\}$, $X = (Q_M - A_M) -
\{s_M\}$ and $Y= (Q_N - A_N) - \{s_N\}$. Clearly the initial
argument satisfies (\dag).

If $\findIso$ is called with argument $(f, [\,])$, then it returns
$\some\,f$.

Otherwise, if $\findIso$ is called with argument $(f, [(\emptyset,
\emptyset), C_2, \ldots, C_n])$, then it calls itself recursively with
argument $(f, [C_2, \ldots, C_n])$.  (The size of the bijection has
been preserved, but the sum of the weights of the constraints has gone
down by one.)

Otherwise, if $\findIso$ is called with argument $(f, [(\{q\},\{r\}),
C_2,\ldots,C_n])$, then it proceeds as follows:
\begin{itemize}
\item If $(q,r)\in f$, then it calls itself recursively with argument
  $(f,[C_2,\ldots,C_n])$ and returns what the recursive call returns.
  (The size of the bijection has been preserved, but the sum of the
  weights of the constraints has gone down by three.)

\item Otherwise, if $q\in\domain\,f$ or $r\in\range\,f$, then
  $\findIso$ returns $\none$.

\item Otherwise, it works its way through the strings appearing in the
  transitions of $M$ and $N$, forming a list of new constraints,
  $C'_1,\,\ldots,C'_m$.  Given such a string, $x$, it lets
  $A_x=\setof{p\in Q_M}{q,x\fun p\in T_M}$ and $B_x=\setof{p\in
    Q_N}{r,x\fun p\in T_N}$.  If $|A_x|\neq|B_x|$, then it returns
  $\none$.  Otherwise, it adds the constraint $(A_x,B_x)$ to our
  list of new constraints.  When all such strings have been exhausted,
  it calls itself recursively with argument $(f \cup \{(q,r)\},
  [C'_1,\ldots,C'_m,C_2,\ldots,C_n])$ and returns what this recursive
  call returns.  (The size of the bijection has been increased by
  one.)
\end{itemize}

Otherwise, $\findIso$ has been called with argument $(f, [(A, A'),
C_2, \ldots, C_n])$, where $|A|>1$, and it proceeds as follows.  It
picks the smallest symbol $q\in A$, and lets $B=A-\{q\}$.  Then,
it works its way through the elements of $A'$.  Given $r\in A'$, it
lets $B'=A'-\{r\}$.  Then, it tries calling itself recursively with
argument $(f, [(\{q\},\{r\}), (B,B'), C_2,\ldots, C_n])$.  (The
size of the bijection has been preserved, but the sum of the sizes of
the weights of the constraints has gone down by $2\cdot
3^{|B_1|}-3\geq 3$.)  If this call returns a result of form $\some\,h$,
then it returns this to its caller.  But if the call returns $\none$, it
tries the next element of $A'$.  If it exhausts the elements of $A'$,
then it returns $\none$.

\begin{lemma}
\label{FindIsoLem}
If $\findIso$ is called with an argument $(f,[C_1,\,\ldots,C_n])$ satisfying
property (\dag), then it returns $\none$, if there is no isomorphism
from $M$ to $N$ that is a superset of $f$ and satisfies the constaints
$C_1,\ldots,C_n$, and returns $\some\,h$ where $h$ is such an isomorphism,
otherwise.
\end{lemma}

\begin{proof}
By well-founded induction on our termination relation.  I.e., when
proving the result for $(f,[C_1,\ldots,C_n])$, we may assume that the
result holds for all arguments $(f',[C'_1,\ldots,C'_m])$ that are
strictly smaller in our termination ordering.
\end{proof}

\begin{theorem}
If $\findIso$ is called with its initial argument, then it returns
$\none$, if there is no isomorphism from $M$ to $N$, and returns
$\some\,h$ where $h$ is an isomorphism from $M$ to $N$, otherwise.
\end{theorem}

\begin{proof}
Follows easily from Lemma~\ref{FindIsoLem}.
\end{proof}

\subsection{Isomorphism Finding/Checking in Forlan}

The Forlan module \texttt{FA}
\index{FA@\texttt{FA}}%
also defines the functions
\begin{verbatim}
val isomorphism             : fa * fa * sym_rel -> bool
val findIsomorphism         : fa * fa -> sym_rel
val isomorphic              : fa * fa -> bool
val renameStates            : fa * sym_rel -> fa
val renameStatesCanonically : fa -> fa
\end{verbatim}
\index{FA@\texttt{FA}!isomorphism@\texttt{isomorphism}}%
\index{FA@\texttt{FA}!findIsomorphism@\texttt{findIsomorphism}}%
\index{FA@\texttt{FA}!isomorphic@\texttt{isomorphic}}%
\index{FA@\texttt{FA}!renameStates@\texttt{renameStates}}%
\index{FA@\texttt{FA}!renameStatesCanonically@\texttt{renameStatesCanonically}}%
The function \texttt{isomorphism} checks whether a relation on symbols
is an isomorphism from one FA to another.  The function
\texttt{findIsomorphism} tries to find an isomorphism from one FA to
another; it issues an error message if there isn't one.  The
function \texttt{isomorphic} checks whether two FAs are isomorphic.
The function \texttt{renameStates} issues an error message
if the supplied relation isn't a bijection from the set
of states of the supplied FA to some set; otherwise, it returns the
result of $\renameStates$.  And the function
\texttt{renameStatesCanonically} acts like $\renameStatesCanonically$.

Suppose \texttt{fa1} and \texttt{fa2} have been bound to our example
our example finite automata $M$ and $N$:
\begin{center}
\input{chap-3.5-fig1.eepic}
\end{center}
Then, here are some example uses of the above functions:
\begin{list}{}
{\setlength{\leftmargin}{\leftmargini}
\setlength{\rightmargin}{0cm}
\setlength{\itemindent}{0cm}
\setlength{\listparindent}{0cm}
\setlength{\itemsep}{0cm}
\setlength{\parsep}{0cm}
\setlength{\labelsep}{0cm}
\setlength{\labelwidth}{0cm}
\catcode`\#=12
\catcode`\$=12
\catcode`\%=12
\catcode`\^=12
\catcode`\_=12
\catcode`\.=12
\catcode`\?=12
\catcode`\!=12
\catcode`\&=12
\ttfamily}
\small
\item[]\textsl{-\ }val\ rel\ =\ FA.findIsomorphism(fa1,\ fa2);
\item[]\textsl{val\ rel\ =\ -\ :\ sym_rel}
\item[]\textsl{-\ }SymRel.output("",\ rel);
\item[]\textsl{(A,\ A),\ (B,\ C),\ (C,\ B)}
\item[]\textsl{val\ it\ =\ ()\ :\ unit}
\item[]\textsl{-\ }FA.isomorphism(fa1,\ fa2,\ rel);
\item[]\textsl{val\ it\ =\ true\ :\ bool}
\item[]\textsl{-\ }FA.isomorphic(fa1,\ fa2);
\item[]\textsl{val\ it\ =\ true\ :\ bool}
\item[]\textsl{-\ }val\ rel'\ =\ FA.findIsomorphism(fa1,\ fa1);
\item[]\textsl{val\ rel'\ =\ -\ :\ sym_rel}
\item[]\textsl{-\ }SymRel.output("",\ rel');
\item[]\textsl{(A,\ A),\ (B,\ B),\ (C,\ C)}
\item[]\textsl{val\ it\ =\ ()\ :\ unit}
\item[]\textsl{-\ }FA.isomorphism(fa1,\ fa1,\ rel');
\item[]\textsl{val\ it\ =\ true\ :\ bool}
\item[]\textsl{-\ }FA.isomorphism(fa1,\ fa2,\ rel');
\item[]\textsl{val\ it\ =\ false\ :\ bool}
\item[]\textsl{-\ }val\ rel''\ =\ SymRel.input\ "";
\item[]\textsl{@\ }(A,\ 2),\ (B,\ 1),\ (C,\ 0)
\item[]\textsl{@\ }.
\item[]\textsl{val\ rel''\ =\ -\ :\ sym_rel}
\item[]\textsl{-\ }val\ fa3\ =\ FA.renameStates(fa1,\ rel'');
\item[]\textsl{val\ fa3\ =\ -\ :\ fa}
\item[]\textsl{-\ }FA.output("",\ fa3);
\item[]\textsl{\symbol{'173}states\symbol{'175}\ 0,\ 1,\ 2\ \symbol{'173}start\ state\symbol{'175}\ 2\ \symbol{'173}accepting\ states\symbol{'175}\ 0,\ 1,\ 2}
\item[]\textsl{\symbol{'173}transitions\symbol{'175}\ 0,\ 1\ ->\ 1;\ 2,\ 0\ ->\ 1\ |\ 2;\ 2,\ 1\ ->\ 0}
\item[]\textsl{val\ it\ =\ ()\ :\ unit}
\item[]\textsl{-\ }val\ fa4\ =\ FA.renameStatesCanonically\ fa3;
\item[]\textsl{val\ fa4\ =\ -\ :\ fa}
\item[]\textsl{-\ }FA.output("",\ fa4);
\item[]\textsl{\symbol{'173}states\symbol{'175}\ A,\ B,\ C\ \symbol{'173}start\ state\symbol{'175}\ C\ \symbol{'173}accepting\ states\symbol{'175}\ A,\ B,\ C}
\item[]\textsl{\symbol{'173}transitions\symbol{'175}\ A,\ 1\ ->\ B;\ C,\ 0\ ->\ B\ |\ C;\ C,\ 1\ ->\ A}
\item[]\textsl{val\ it\ =\ ()\ :\ unit}
\item[]\textsl{-\ }FA.equal(fa4,\ fa1);
\item[]\textsl{val\ it\ =\ false\ :\ bool}
\item[]\textsl{-\ }FA.isomorphic(fa4,\ fa1);
\item[]\textsl{val\ it\ =\ true\ :\ bool}
\end{list}


\subsection{Notes}

Books on formal language theory rarely formalize the isomorphism
of finite automata, although most note or prove that the minimization of
deterministic finite automata yields a result that is unique up
to the renaming of states.  Our algorithm for trying to find an
isomorphism between finite automata will be unsurprising to those
familiar with graph algorithms.

\index{isomorphism!finite automaton|)}%
\index{finite automaton!isomorphism|)}%

%%% Local Variables: 
%%% mode: latex
%%% TeX-master: "book"
%%% End: 
