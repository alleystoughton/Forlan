\section{Regular Expressions and Languages}
\label{RegularExpressionsAndLanguages}

In this section, we define several operations on languages, say what
regular expressions are, what they mean, and what regular languages
are, and begin to show how regular expressions can be processed by Forlan.

\subsection{Operations on Languages}

\index{union!language}%
\index{intersection!language}%
\index{set difference!language}%
The union, intersection and set-difference operations on sets are also
operations on languages, i.e., if $L_1,L_2\in\Lan$, then $L_1\cup
L_2$, $L_1\cap L_2$ and $L_1-L_2$ are all languages.
(Since $L_1,L_2\in\Lan$, we have that $L_1\sub\Sigma_1^*$
and $L_2\sub\Sigma_2^*$, for alphabets $\Sigma_1$ and $\Sigma_2$.
Let $\Sigma=\Sigma_1\cup\Sigma_2$, so that $\Sigma$ is an alphabet,
$L_1\sub\Sigma^*$ and $L_2\sub\Sigma^*$.
Thus $L_1\cup L_2$, $L_1\cap L_2$ and $L_1-L_2$ are all subsets
of $\Sigma^*$, and so are all languages.)

\index{language!concatenation}%
\index{concatenation!language}%
The first new operation on languages is language concatenation.
The \emph{concatenation} of languages $L_1$ and $L_2$ ($L_1\myconcat
L_2$)
\index{ at@$\myconcat$}%
\index{language! at@$\myconcat$}%
is the language
\begin{gather*}
\setof{x_1\myconcat x_2}{x_1\in L_1\eqtxt{and}x_2\in L_2} .
\end{gather*}
I.e., $L_1\myconcat L_2$ consists of all strings that can be formed by
concatenating an element of $L_1$ with an element of $L_2$.
For example,
\begin{align*}
\mathsf{\{ab,abc\}\myconcat\{cd,d\}} &=
\mathsf{\{(ab)(cd), (ab)(d), (abc)(cd), (abc)(d)\}} \\
&= \mathsf{\{abcd, abd, abccd\}}.
\end{align*}
Note that, if $L_1,L_2\sub\Sigma^*$, for an alphabet $\Sigma$,
then $L_1\myconcat L_2\sub\Sigma^*$.

Concatenation of languages is associative: for all $L_1,L_2,L_3\in\Lan$,
\index{associative!language concatenation}%
\index{language!concatenation!associative}%
\index{concatenation!language!associative}%
\begin{gather*}
(L_1\myconcat L_2)\myconcat L_3 = L_1\myconcat(L_2\myconcat L_3) .
\end{gather*}
And, $\{\%\}$ is the identity for concatenation:
\index{identity!language concatenation}%
\index{language!concatenation!identity}%
\index{concatenation!language!identity}%
for all $L\in\Lan$,
\begin{gather*}
{\{\%\}}\myconcat L=L\myconcat{\{\%\}}=L.
\end{gather*}
Furthermore, $\emptyset$ is the zero for concatenation:
\index{zero!language concatenation}%
\index{language!concatenation!zero}%
\index{concatenation!language!zero}%
for all $L\in\Lan$,
\begin{gather*}
\emptyset\myconcat L=L\myconcat\emptyset={\emptyset} .
\end{gather*}
We often abbreviate $L_1\myconcat L_2$ to $L_1L_2$.

Now that we know what language concatenation is, we can say what it
means to raise a language to a power.  We define the \emph{language}
\index{concatenation!language!power}%
\index{language!concatenation!power}%
\index{concatenation!language!exponentiation}%
\index{language!concatenation!exponentiation}%
$L^n$ \emph{formed by raising} a language $L$ \emph{to the
  power} $n\in\nats$
\index{ power@$\cdot^\cdot$}%
\index{language! power@$\cdot^\cdot$}%
by recursion on $n$:
\begin{align*}
L^0      &= {\{\%\}} , \eqtxt{for all}L\in\Lan ; \eqtxt{and} \\
L^{n + 1} &= LL^n ,\eqtxt{for all}L\in\Lan\eqtxt{and}n\in\nats .
\end{align*}
We assign this exponentiation operation higher precedence than
concatenation, so that $LL^n$ means $L(L^n)$ in the above definition.
Note that, if $L\sub\Sigma^*$, for an alphabet $\Sigma$, then
$L^n\sub\Sigma^*$, for all $n\in\nats$.

For example, we have that
\begin{align*}
\mathsf{\{a,b\}}^2 &= 
\mathsf{\{a,b\}}\mathsf{\{a,b\}}^1 =
\mathsf{\{a,b\}}\mathsf{\{a,b\}}\mathsf{\{a,b\}}^0 \\
&=\mathsf{\{a,b\}}\mathsf{\{a,b\}}\{\%\} =
\mathsf{\{a,b\}}\mathsf{\{a,b\}} \\
&=\mathsf{\{aa, ab, ba, bb\}}.
\end{align*}

\begin{proposition}
\label{LangExponProp1}
For all $L\in\Lan$ and $n,m\in\nats$, $L^{n+m}=L^nL^m$.
\end{proposition}

\begin{proof}
An easy mathematical induction on $n$.  The language $L$ and the natural
number $m$ can be fixed at the beginning of the proof.
\end{proof}

Thus, if $L\in\Lan$ and $n\in\nats$, then
\begin{alignat*}{2}
L^{n+1} &= LL^n && \by{definition}, \\
\intertext{and}
L^{n+1} &= L^nL^1 = L^nL && \by{Proposition~\ref{LangExponProp1}} .
\end{alignat*}

Another useful fact about language exponentiation is:

\begin{proposition}
\label{LangExponProp2}
For all $w\in\Str$ and $n\in\nats$, $\{w\}^n = \{w^n\}$.
\end{proposition}

\begin{proof}
By mathematical induction on $n$.
\end{proof}

For example, we have that $\{\mathsf{01}\}^4=\{(\mathsf{01})^4\}=
\{\mathsf{01010101}\}$.

Now we consider a language operation that is named after
Stephen Cole Kleene, one of the founders of formal language theory.
\index{closure!language}%
\index{Kleene closure|see{closure}}%
The \emph{Kleene closure} (or just \emph{closure}) of a language
$L$ ($L^*$) is the language
\begin{gather*}
\bigcup\setof{L^n}{n\in\nats}.
\end{gather*}
Thus, for all $w$,
\begin{align*}
w\in L^*&\myiff w\in A,\eqtxt{for some}A\in\setof{L^n}{n\in\nats} \\
&\myiff w\in L^n\eqtxt{for some}n\in\nats .
\end{align*}
Or, in other words:
\begin{itemize}
\item $L^*=L^0\cup L^1\cup L^2\cup{\cdots}$; and

\item $L^*$ consists of all strings that can be formed by concatenating
together some number (maybe none) of elements of $L$ (the same element of
$L$ can be used as many times as is desired).
\end{itemize}
For example,
\begin{align*}
\mathsf{\{a,ba\}^*} &=
\mathsf{\{a,ba\}^0\cup\{a,ba\}^1\cup\{a,ba\}^2\cup\cdots} \\
&= \mathsf{\{\%\}\cup
   \{a,ba\}\cup
   \{aa,aba,baa,baba\}\cup\cdots}
\end{align*}

If $L$ is a language, then $L\sub\Sigma^*$ for some alphabet $\Sigma$,
and thus $L^*$ is also a subset of $\Sigma^*$---showing that $L^*$
is a language, not just a set of strings.

Suppose $w\in\Str$.
By Proposition~\ref{LangExponProp2}, we have
that, for all $x$,
\begin{align*}
x\in\{w\}^* &\myiff x\in\{w\}^n,\eqtxt{for some}n\in\nats , \\
&\myiff x\in\{w^n\},\eqtxt{for some}n\in\nats , \\
&\myiff x=w^n, \eqtxt{for some} n\in\nats .
\end{align*}

If we write $\mathsf{\{0,1\}}^*$, then this could mean:
\begin{itemize}
\item
all strings over the alphabet $\mathsf{\{0,1\}}$
(Section~\ref{SymbolsStringsAlphabetsAndFormalLanguages}); or

\item
the closure of the language $\mathsf{\{0,1\}}$.
\end{itemize}
Fortunately, these languages are equal (both are all strings of
$\zerosf$'s and $\onesf$'s), and this kind of ambiguity is harmless.

\index{language!operation precedence}%
We assign our operations on languages relative precedences as follows:
\begin{description}
\item[\quad Highest:] closure ($(\cdot)^*$) and raising to a power
($(\cdot)^n$);

\item[\quad Intermediate:] concatenation ($\myconcat$, or just
  juxtapositioning); and

\item[\quad Lowest:] union ($\cup$), intersection ($\cap$) and difference ($-$).
\end{description}
For example, if $n\in\nats$ and $A,B,C\in\Lan$, then $A^*BC^n\cup B$
abbreviates $((A^*)B(C^n))\cup B$.  The language $((A\cup B)C)^*$
can't be abbreviated, since removing either pair of parentheses will
change its meaning.  If we removed the outer pair, then we would have
$(A\cup B)(C^*)$, and removing the inner pair would yield $(A\cup
(BC))^*$.

\index{alphabet@$\alphabet$}%
\index{language!alphabet@$\alphabet$}%
\index{alphabet!language}%
\index{language!alphabet}%
Suppose $L$, $L_1$ and $L_2$ are languages, and $n\in\nats$.  It
is easy to see that $\alphabet(L_1\cup L_2)=\alphabet(L_1)\cup\alphabet(L_2)$.
And, if $L_1$ and $L_2$ are both nonempty, then
$\alphabet(L_1L_2)=\alphabet(L_1)\cup\alphabet(L_2)$, and otherwise,
$\alphabet(L_1L_2)=\emptyset$.
Furthermore, if $n\geq 1$, then $\alphabet(L^n)=\alphabet(L)$; otherwise,
$\alphabet(L^n)=\emptyset$.
Finally, we have that $\alphabet(L^*)=\alphabet(L)$.

In Section~\ref{IntroductionToForlan}, we introduced the
Forlan module \texttt{StrSet},
which defines various functions for processing finite sets of strings,
i.e., finite languages.  This module also defines the
functions
\index{StrSet@\texttt{StrSet}!concat@\texttt{concat}}%
\index{StrSet@\texttt{StrSet}!power@\texttt{power}}%
\begin{verbatim}
val concat : str set * str set -> str set
val power  : str set * int -> str set
\end{verbatim}
which implement our concatenation and exponentiation operations
on finite languages.  Here are some examples of how these functions
can be used:
\begin{list}{}
{\setlength{\leftmargin}{\leftmargini}
\setlength{\rightmargin}{0cm}
\setlength{\itemindent}{0cm}
\setlength{\listparindent}{0cm}
\setlength{\itemsep}{0cm}
\setlength{\parsep}{0cm}
\setlength{\labelsep}{0cm}
\setlength{\labelwidth}{0cm}
\catcode`\#=12
\catcode`\$=12
\catcode`\%=12
\catcode`\^=12
\catcode`\_=12
\catcode`\.=12
\catcode`\?=12
\catcode`\!=12
\catcode`\&=12
\ttfamily}
\small
\item[]\textsl{-\ }val\ xs\ =\ StrSet.fromString\ "ab,\ cd";
\item[]\textsl{val\ xs\ =\ -\ :\ str\ set}
\item[]\textsl{-\ }val\ ys\ =\ StrSet.fromString\ "uv,\ wx";
\item[]\textsl{val\ ys\ =\ -\ :\ str\ set}
\item[]\textsl{-\ }StrSet.output("",\ StrSet.concat(xs,\ ys));
\item[]\textsl{abuv,\ abwx,\ cduv,\ cdwx}
\item[]\textsl{val\ it\ =\ ()\ :\ unit}
\item[]\textsl{-\ }StrSet.output("",\ StrSet.power(xs,\ 0));
\item[]\textsl{%}
\item[]\textsl{val\ it\ =\ ()\ :\ unit}
\item[]\textsl{-\ }StrSet.output("",\ StrSet.power(xs,\ 1));
\item[]\textsl{ab,\ cd}
\item[]\textsl{val\ it\ =\ ()\ :\ unit}
\item[]\textsl{-\ }StrSet.output("",\ StrSet.power(xs,\ 2));
\item[]\textsl{abab,\ abcd,\ cdab,\ cdcd}
\item[]\textsl{val\ it\ =\ ()\ :\ unit}
\item[]\textsl{-\ }StrSet.output("",\ StrSet.power(xs,\ 3));
\item[]\textsl{ababab,\ ababcd,\ abcdab,\ abcdcd,\ cdabab,\ cdabcd,\ cdcdab,\ cdcdcd}
\item[]\textsl{val\ it\ =\ ()\ :\ unit}
\end{list}


\subsection{Regular Expressions}

\index{regular expression|(}%
\index{regular expression!label}%
\index{RegLab@$\RegLab$}%
Next, we define the set of all regular expressions.  Let the set
$\RegLab$ of \emph{regular expression labels} be
\begin{gather*}
\Sym\cup\{\%,\$,{*},@,{+}\} .
\end{gather*}
\index{Reg@$\Reg$}%
Let the set $\Reg$ of
\emph{regular expressions} be the least subset of
$\Tree_\RegLab$ such that:
\index{Tree sub X@$\Tree_X$}%
\index{tree}%
\index{tree!Tree sub X@$\Tree_X$}%
\begin{description}
\item[\quad(empty string)] $\%\in\Reg$;

\item[\quad(empty set)] $\$\in\Reg$;

\item[\quad(symbol)] for all $a\in\Sym$, $a\in\Reg$;

\item[\quad(closure)] for all $\alpha\in\Reg$, ${*}(\alpha)\in\Reg$;
\index{closure!regular expression}%
\index{regular expression!closure}%

\item[\quad(concatenation)] for all $\alpha,\beta\in\Reg$,
$@(\alpha,\beta)\in\Reg$; and
\index{concatenation!regular expression}%
\index{regular expression!concatenation}%

\item[\quad(union)] for all $\alpha,\beta\in\Reg$, ${+}(\alpha,\beta)\in\Reg$.
\index{union!regular expression}%
\index{regular expression!union}%
\end{description}
This is yet another example of an inductive definition.  The elements
of $\Reg$ are precisely those $\RegLab$-trees
(trees (See Section~\ref{TreesAndInductiveDefinitions})
whose labels come from $\RegLab$) that can be built using these six
rules.

Whenever possible, we will use the mathematical variables
$\alpha$, $\beta$ and $\gamma$
\index{alpha, beta, gamma@$\alpha$, $\beta$, $\gamma$}%
\index{regular expression!alpha, beta, gamma@$\alpha$, $\beta$, $\gamma$}%
to name regular expressions. Since regular expressions are
$\RegLab$-trees, we may talk of their sizes and heights.

For example,
\begin{gather*}
\mathsf{{+}(@({*}(0),@(1,{*}(0))),\%)} ,
\end{gather*}
i.e.,
\begin{center}
\input{chap-3.1-fig1.eepic}
\end{center}
is a regular expression.  On the other hand, the $\RegLab$-tree
$*(*,*)$ is \emph{not} a regular expression, since it can't be
built using our six rules.

We order the elements of $\RegLab$ as follows:
\begin{gather*}
\% < \$ < \eqtxtn{symbols in order} < {*} < @ < {+} .
\end{gather*}
\index{regular expression!order}%
It is important that ${+}$ be the greatest element of $\RegLab$;
if this were not so, then the definition of weakly simplified regular
expressions (see Section~\ref{SimplificationOfRegularExpressions})
would have to be altered.

We order regular expressions first by their root labels, and then,
recursively, by their children, working from left to right.
For example, we have that
\begin{gather*}
\% < {*}(\%)
   < {*}(@(\$,{*}(\$)))
   < {*}(@(\asf,\%))
   < @(\%,\$) .
\end{gather*}

Because $\Reg$ is defined inductively, it gives rise to an induction
principle.

\begin{theorem}[Principle of Induction on Regular Expressions]
Suppose $P(\alpha)$ is a property of a regular expression $\alpha$.
If
\begin{itemize}
\item $P(\%)$,

\item $P(\$)$,

\item for all $a\in\Sym$, $P(a)$,

\item for all $\alpha\in\Reg$, if $P(\alpha)$, then
$P({*}(\alpha))$,

\item for all $\alpha,\beta\in\Reg$, if $P(\alpha)$ and
$P(\beta)$, then $P(@(\alpha,\beta))$, and

\item for all $\alpha,\beta\in\Reg$, if $P(\alpha)$ and
$P(\beta)$, then $P({+}(\alpha,\beta))$,
\end{itemize}
then
\begin{gather*}
\eqtxtr{for all}\alpha\in\Reg,\,P(\alpha) .
\end{gather*}
\end{theorem}

To increase readability, we use infix and postfix notation, abbreviating:
\begin{itemize}
\item ${*}(\alpha)$ to $\alpha^*$ or $\alpha{*}$;

\item $@(\alpha,\beta)$ to $\alpha\myconcat \beta$; and

\item $+(\alpha,\beta)$ to $\alpha+\beta$.
\end{itemize}
\index{regular expression!abbreviated notation}%
\index{regular expression!operator precedence}%
\index{regular expression!operator associativity}%
We assign the operators $(\cdot)^*$, $\myconcat$ and $+$ the following
precedences and associativities:
\begin{description}
\item[\quad Highest:] $(\cdot)^*$;

\item[\quad Intermediate:] $\myconcat$ (right associative); and

\item[\quad Lowest:] $+$ (right associative).
\end{description}
We parenthesize regular expressions when we need to override the
default precedences and associativities, and for reasons of clarity.
Furthermore, we often abbreviate $\alpha\myconcat\beta$ to $\alpha\beta$.

For example, we can abbreviate the regular expression
\begin{displaymath}
\mathsf{{+}(@({*}(0),@(1,{*}(0))),\%)}
\end{displaymath}
to $\mathsf{0^*\myconcat 1\myconcat 0^*+\%}$ or $\mathsf{0^*10^*+\%}$.  On
the other hand, the regular expression $\mathsf{((0+1)2)^*}$ can't be
further abbreviated, since removing either pair of parentheses would
result in a different regular expression.  Removing the outer pair
would result in $\mathsf{(0+1)(2^*)}=\mathsf{(0+1)2^*}$, and removing
the inner pair would yield $\mathsf{(0+(12))^*}=\mathsf{(0+12)^*}$.

Now we can say what regular expressions mean, using some of our
language operations.  The \emph{language generated by} a regular
expression $\alpha$ ($L(\alpha)$) is defined by structural recursion:
\index{regular expression!meaning}%
\index{regular expression!language generated by}%
\index{L(@$L(\cdot)$}%
\index{regular expression!L(@$L(\cdot)$}%}%
\begin{align*}
L(\%) &= \{\%\} ; \\
L(\$) &= \emptyset ; \\
L(a) &= \{[a]\} = \{a\}, \eqtxt{for all}a\in\Sym ; \\
L({*}(\alpha)) &= L(\alpha)^* , \eqtxt{for all}\alpha\in\Reg ; \\
L(@(\alpha,\beta)) &= L(\alpha)\myconcat L(\beta) ,
\eqtxt{for all}\alpha,\beta\in\Reg ; \eqtxt{and} \\
L({+}(\alpha,\beta)) &= L(\alpha)\cup L(\beta) ,
\eqtxt{for all}\alpha,\beta\in\Reg .
\end{align*}
This is a good definition since, if $L$ is a language, then so is
$L^*$, and, if $L_1$ and $L_2$ are languages, then so are $L_1L_2$ and
$L_1\cup L_2$.  We say that $w$ \emph{is generated by} $\alpha$ iff
$w\in L(\alpha)$.

For example,
\begin{align*}
L(\mathsf{0^*10^*+\%}) &=
L(\mathsf{{+}(@({*}(0),@(1,{*}(0))),\%)}) \\
&= L(\mathsf{@({*}(0),@(1,{*}(0)))})\cup L(\%) \\
&= L(\mathsf{{*}(0)})L(\mathsf{@(1,{*}(0))})\cup \{\%\} \\
&= L(\mathsf{0})^* L(\mathsf{1}) L(\mathsf{{*}(0)}) \cup \{\%\} \\
&= \mathsf{\{0\}}^* \mathsf{\{1\}} L(\mathsf{0})^*
  \cup \{\%\} \\
&= \mathsf{\{0\}}^* \mathsf{\{1\}} \mathsf{\{0\}}^* 
  \cup \{\%\} \\
&= \setof{{\mathsf 0}^n{\mathsf 1}{\mathsf 0}^m}{n,m\in\nats} \cup
\{\%\} .
\end{align*}
E.g., $\mathsf{0001000}$, $\mathsf{10}$,
$\mathsf{001}$ and $\%$ are generated by $\mathsf{0^*10^*+\%}$.

We define functions $\symToReg\in\Sym\fun\Reg$ and
$\strToReg\in\Str\fun\Reg$, as follows.  Given a symbol $a\in\Sym$,
$\symToReg\,a = a$. And, given a string $x$, $\strToReg\,x$ is the
\emph{canonical regular expression for} $x$: $\%$, if $x = \%$, and
${@}(a_1, {@}(a_2, \ldots a_n \ldots)) = a_1a_2\ldots a_n$, if $x =
a_1a_2\ldots a_n$, for symbols $a_1,a_2,\ldots,a_n$ and $n\geq 1$.  It
is easy to see that, for all $a\in\Sym$, $L(\symToReg\,a)=\{a\}$, and,
for all $x\in\Str$, $L(\strToReg\,x)=\{x\}$.

We define the \emph{regular expression} $\alpha^n$ \emph{formed by
raising} a regular expression $\alpha$ \emph{to the power} $n\in\nats$ by
recursion on $n$:
\index{regular expression!power}%
\index{regular expression!exponentiation}%
\begin{align*}
\alpha^0       &= \%, \eqtxt{for all}\alpha\in\Reg; \\
\alpha^1       &= \alpha, \eqtxt{for all}\alpha\in\Reg; \eqtxt{and} \\
\alpha^{n + 1} &= \alpha\alpha^n,\eqtxt{for all}\alpha\in\Reg\eqtxt{and}
n\in\nats-\{0\} .
\end{align*}
We assign this operation the same precedence as closure, so that
$\alpha\alpha^n$ means $\alpha(\alpha^n)$ in the above definition.
Note that, in contrast to the definitions of $x^n$ and $L^n$,
we have split the case $n+1$ into two subcases, depending upon
whether $n=0$ or $n\geq 1$. Thus
$\alpha^1$ is $\alpha$, not $\alpha\%$.
For example, $\mathsf{(0+1)^3}=\mathsf{(0+1)(0+1)(0+1)}$.

\begin{proposition}
\label{RegExponProp}
For all $\alpha\in\Reg$ and $n\in\nats$, $L(\alpha^n)=L(\alpha)^n$.
\end{proposition}

\begin{proof}
By mathematical induction on $n$, case-splitting in the inductive
step. $\alpha$ may be fixed at the beginning of the proof.
\end{proof}

An example consequence of the proposition is that
$L(\mathsf{(0+1)^3})=L(\mathsf{0+1})^3=\{\mathsf{0,1}\}^3$, the
set of all strings of $\zerosf$'s and $\onesf$'s of length $3$.

We define $\alphabet\in\Reg\fun\Alp$ by structural recursion:
\index{regular expression!alphabet}%
\index{alphabet@$\alphabet$}%
\begin{align*}
\alphabet\,\% &= \emptyset; \\
\alphabet\,\$ &= \emptyset; \\
\alphabet\,a &= \{a\}, \eqtxt{for all}a\in\Sym; \\
\alphabet({*}(\alpha)) &= \alphabet\,\alpha, \eqtxt{for all}\alpha\in\Reg;\\
\alphabet(@(\alpha,\beta)) &= \alphabet\,\alpha\cup\alphabet\,\beta ,
\eqtxt{for all}\alpha,\beta\in\Reg; \eqtxt{and} \\
\alphabet({+}(\alpha,\beta)) &= \alphabet\,\alpha\cup\alphabet\,\beta,
\eqtxt{for all}\alpha,\beta\in\Reg .
\end{align*}
This is a good definition, since the union of two alphabets is
an alphabet. For example,
$\alphabet(\mathsf{0^*10^*+\%}) = \mathsf{\{0,1\}}$.
We say that $\alphabet\,\alpha$ is \emph{the alphabet of} a regular
expression $\alpha$.

Because the alphabet of regular expressions is defined by structural
recursion, we can prove the following proposition by induction
on regular expressions:
\begin{proposition}
\label{RegAlphabetSubtreeSubstitute}
\begin{enumerate}[\quad(1)]
\item Suppose $\alpha,\beta,\beta'\in\Reg$,
  $\alphabet\,\beta' = \alphabet\,\beta$, $\pat\in\Path$ is valid for
  $\alpha$, and $\beta$ is the subtree of $\alpha$ at position $\pat$.
  Let $\alpha'$ be the result of replacing the subtree at position
  $\pat$ in $\alpha$ by $\beta'$. Then
  $\alphabet\,\alpha' = \alphabet\,\alpha$.

\item Suppose $\alpha,\beta,\beta'\in\Reg$,
  $\alphabet\,\beta' \sub \alphabet\,\beta$, $\pat\in\Path$ is valid for
  $\alpha$, and $\beta$ is the subtree of $\alpha$ at position $\pat$.
  Let $\alpha'$ be the result of replacing the subtree at position
  $\pat$ in $\alpha$ by $\beta'$. Then
  $\alphabet\,\alpha' \sub \alphabet\,\alpha$.
\end{enumerate}
\end{proposition}

\begin{exercise}
\label{RegExactExpOverAlphabetFinite}
Suppose $\Sigma$ is an alphabet.
For $n\in\nats$, define
$X_n = \setof{\alpha\in\Reg}{\alphabet\,\alpha\sub\Sigma
  \eqtxt{and} \size\,\alpha = n}$. Prove that, for all
$n\in\nats$, $X_n$ is finite. Hint: use strong induction
on $n$.
\end{exercise}

\begin{proposition}
\label{AlphabetRegMeaning}
For all $\alpha\in\Reg$, $\alphabet(L(\alpha))\sub\alphabet\,\alpha$.
\end{proposition}

In other words, the proposition says that every symbol of every string in
$L(\alpha)$  comes from $\alphabet\,\alpha$.

\begin{proof}
An easy induction on regular expressions.
\end{proof}

For example, since $L(\onesf\$) = \{\onesf\}\emptyset = \emptyset$,
we have that
\begin{align*}
\alphabet(L(\mathsf{0^*+1\$})) &=
\alphabet(\{\zerosf\}^*) \\
&= \{\zerosf\} \\
&\sub \{\mathsf{0,1}\} \\
&= \alphabet(\mathsf{0^*+1\$}) .
\end{align*}

Next, we define some useful auxiliary functions on regular expressions.
\index{regular expression!generalized concatenation}
\index{regular expression!genConcat@$\genConcat$}
The \emph{generalized concatenation} function
$\genConcat\in\List\,\Reg\fun\Reg$ is defined by right recursion:
\begin{align*}
\genConcat\,[\,] &= \% , \\
\genConcat\,[\alpha] &= \alpha , \eqtxt{and} \\
\genConcat([\alpha] \myconcat \bar{\alpha}) &=
{@}(\alpha, \genConcat\,\bar{\alpha}) , \eqtxt{if} \bar{\alpha}\neq [\,] .
\end{align*}
\index{regular expression!generalized union}
\index{regular expression!genUnion@$\genUnion$}
And the \emph{generalized union} function
$\genUnion\in\List\,\Reg\fun\Reg$ is defined by right recursion:
\begin{align*}
\genUnion\,[\,] &= \% , \\
\genUnion\,[\alpha] &= \alpha , \eqtxt{and} \\
\genUnion([\alpha] \myconcat \bar{\alpha}) &=
{+}(\alpha, \genUnion\,\bar{\alpha}) , \eqtxt{if} \bar{\alpha}\neq [\,] .
\end{align*}
E.g., $\genConcat[\mathsf{1, 0, 12, 3+4}] = \mathsf{10(12)(3+4)}$ and
$\genUnion[\mathsf{1, 0, 12, 3+4}] = \mathsf{1 + 0 + (12) + 3 + 4}$.

\index{regular expression!rightConcat@$\rightConcat$}
$\rightConcat\in\Reg\times\Reg\fun\Reg$ is defined by
structural recursion on its first argument:
\begin{align*}
\rightConcat({@}(\alpha_1, \alpha_2), \beta) &=
{@}(\alpha_1, \rightConcat(\alpha_2, \beta)) , \eqtxt{and}\\
\rightConcat(\alpha, \beta) &= {@}(\alpha, \beta),
\eqtxt{if} \alpha \eqtxtl{is not a concatenation}.
\end{align*}
\index{regular expression!rightUnion@$\rightUnion$}
And $\rightUnion\in\Reg\times\Reg\fun\Reg$ is defined by
structural recursion on its first argument:
\begin{align*}
\rightUnion({+}(\alpha_1, \alpha_2), \beta) &=
{+}(\alpha_1, \rightUnion(\alpha_2, \beta)) , \eqtxt{and} \\
\rightUnion(\alpha, \beta) &= {+}(\alpha, \beta),
\eqtxt{if} \alpha \eqtxtl{is not a union}.
\end{align*}
E.g., $\rightConcat(\mathsf{012}, \mathsf{345}) =
\mathsf{012345}$ and $\rightUnion(\mathsf{0 + 1 + 2}, \mathsf{1 + 2 + 3}) =
\mathsf{0 + 1 + 2 + 1 + 2 + 3}$.

\index{regular expression!concatsToList@$\concatsToList$}
$\concatsToList\in\Reg\fun\List\,\Reg$ is defined by structural
recursion:
\begin{align*}
\concatsToList({@}(\alpha, \beta)) &=
[\alpha] \myconcat \concatsToList\,\beta, \eqtxt{and} \\
\concatsToList\,\alpha &= [\alpha],
\eqtxt{if} \alpha \eqtxtl{is not a concatenation}.
\end{align*}
\index{regular expression!unionsToList@$\unionsToList$}
And $\unionsToList\in\Reg\fun\List\,\Reg$ is defined by structural
recursion:
\begin{align*}
\unionsToList({+}(\alpha, \beta)) &=
[\alpha] \myconcat \unionsToList\,\beta, \eqtxt{and} \\
\unionsToList\,\alpha &= [\alpha],
\eqtxt{if} \alpha \eqtxtl{is not a union}.
\end{align*}
E.g., $\concatsToList(\mathsf{(12)34}) = [\mathsf{12, 3, 4}]$
and $\unionsToList(\mathsf{(0+1)+2+3}) = [\mathsf{0 + 1, 2, 3}]$.

\begin{lemma}
\label{ConcatsToListRightConcat}
For all $\alpha,\beta\in\Reg$,
$\concatsToList(\rightConcat(\alpha,\beta)) = \concatsToList\,\alpha
\myconcat \concatsToList\,\beta$.
\end{lemma}

\begin{lemma}
\label{UnionsToListRightUnion}
For all $\alpha,\beta\in\Reg$,
$\unionsToList(\rightUnion(\alpha,\beta)) = \unionsToList\,\alpha
  \myconcat \unionsToList\,\beta$.
\end{lemma}

\index{regular expression!sortUnions@$\sortUnions$}
Finally, $\sortUnions\in\Reg\fun\Reg$ is defined by:
\begin{displaymath}
\sortUnions\,\alpha = \genUnion\,\bar{\beta} ,
\end{displaymath}
where $\bar{\beta}$ is the result of sorting the elements of
$\unionsToList\,\alpha$ into strictly ascending order (without
duplicates), according to our total ordering on regular expressions.
E.g., $\sortUnions(\mathsf{1 + 0 + 23 + 1}) = \mathsf{0 + 1 + 23}$.

\index{regular expression!allSym@$\allSym$}
\index{regular expression!allStr@$\allStr$}
We define functions $\allSym\in\Alp\fun\Reg$ and $\allStr\in
\Alp\fun\Reg$ as follows.  Given an alphabet $\Sigma$, $\allSym\,\Sigma$
is the \emph{all symbols regular expression} for $\Sigma$:
$a_1+\cdots+a_n$, where $a_1,\ldots,a_n$
  are the elements of $\Sigma$, listed in order and without repetition
  (when $n=0$, we use $\$$, and when $n=1$, we use $a_1$).
And, given an alphabet $\Sigma$, $\allStr\,\Sigma$ is
the \emph{all strings regular expression} for $\Sigma$:
$(\allSym\,\Sigma)^*$.
For example, 
\begin{align*}
\allSym\,\{\mathsf{0,1,2}\} &= \mathsf{0 + 1 + 2} , \eqtxt{and} \\
\allStr\,\{\mathsf{0,1,2}\} &= (\mathsf{0 + 1 + 2})^* .
\end{align*}
Thus, for all $\Sigma\in\Alp$,
\begin{align*}
L(\allSym\,\Sigma) &= \setof{[a]}{a\in\Sigma} = \setof{a}{a\in\Sigma},
\eqtxt{and} \\
L(\allStr\,\Sigma) &= \Sigma^* .
\end{align*}

\index{language!regular|see{regular language}}%
\index{regular language}%
Now we are able to say what it means for a language to be regular:
a language $L$ is \emph{regular} iff $L=L(\alpha)$ for some
$\alpha\in\Reg$.  We define
\index{regular language!RegLan@$\RegLan$}%
\index{language: RegLan@$\RegLan$}%
\begin{align*}
\RegLan &= \setof{L(\alpha)}{\alpha\in\Reg} \\
&= \setof{L\in\Lan}{L\eqtxtl{is regular}} .
\end{align*}

Since every regular expression can be uniquely described, e.g., in fully
parenthesized form, by a finite sequence of ASCII characters, we can
enumerate the regular expressions, and consequently we have that
$\Reg$ is countably infinite.  Since $\{\mathsf{0}^0\}$,
$\{\mathsf{0}^1\}$, $\{\mathsf{0}^2\}$, \ldots, are all regular
languages, we have that $\RegLan$ is infinite.  Furthermore, we can
establish an injection $h$ from $\RegLan$ to $\Reg$: $h\,L$ is the
first (in our enumeration of regular expressions) $\alpha$ such that
$L(\alpha)=L$.  Because $\Reg$ is countably infinite, it follows that
there is an injection from $\Reg$ to $\nats$.  Composing these
injections, gives us an injection from $\RegLan$ to $\nats$.  And when
observing that $\RegLan$ is infinite, we implicitly gave an injection
from $\nats$ to $\RegLan$. Thus, by the Schr\"oder-Bernstein Theorem,
we have that $\RegLan$ and $\nats$ have the same size, so that
$\RegLan$ is countably infinite.

Since $\RegLan$ is countably infinite but $\Lan$ is uncountable, it
follows that $\RegLan\subsetneq\Lan$, i.e., there are non-regular
languages.  In Section~\ref{ThePumpingLemmaForRegularLanguages}, we
will see a concrete example of a non-regular language.

\subsection{Processing Regular Expressions in Forlan}

Now, we turn to the Forlan implementation of regular expressions.  The
Forlan module \texttt{Reg}
\index{Reg@\texttt{Reg}}%
defines the abstract type \texttt{reg}
\index{reg@\texttt{reg}}%
\index{Reg@\texttt{Reg}!reg@\texttt{reg}}%
(in the top-level environment) of regular expressions, as well as various
functions and constants for processing regular expressions, including:
\index{Reg@\texttt{Reg}!input@\texttt{input}}%
\index{Reg@\texttt{Reg}!output@\texttt{output}}%
\index{Reg@\texttt{Reg}!size@\texttt{size}}%
\index{Reg@\texttt{Reg}!numLeaves@\texttt{numLeaves}}%
\index{Reg@\texttt{Reg}!height@\texttt{height}}%
\index{Reg@\texttt{Reg}!emptyStr@\texttt{emptyStr}}%
\index{Reg@\texttt{Reg}!emptySet@\texttt{emptySet}}%
\index{Reg@\texttt{Reg}!fromSym@\texttt{fromSym}}%
\index{Reg@\texttt{Reg}!closure@\texttt{closure}}%
\index{Reg@\texttt{Reg}!concat@\texttt{concat}}%
\index{Reg@\texttt{Reg}!union@\texttt{union}}%
\index{Reg@\texttt{Reg}!compare@\texttt{compare}}%
\index{Reg@\texttt{Reg}!equal@\texttt{equal}}%
\index{Reg@\texttt{Reg}!fromStr@\texttt{fromStr}}%
\index{Reg@\texttt{Reg}!power@\texttt{power}}%
\index{Reg@\texttt{Reg}!alphabet@\texttt{alphabet}}%
\index{Reg@\texttt{Reg}!genConcat@\texttt{genConcat}}%
\index{Reg@\texttt{Reg}!genUnion@\texttt{genUnion}}%
\index{Reg@\texttt{Reg}!rightConcat@\texttt{rightConcat}}%
\index{Reg@\texttt{Reg}!rightUnion@\texttt{rightUnion}}%
\index{Reg@\texttt{Reg}!concatsToList@\texttt{concatsToList}}%
\index{Reg@\texttt{Reg}!unionsToList@\texttt{unionsToList}}%
\index{Reg@\texttt{Reg}!sortUnions@\texttt{sortUnions}}%
\index{Reg@\texttt{Reg}!allSym@\texttt{allSym}}%
\index{Reg@\texttt{Reg}!allStr@\texttt{allStr}}%
\index{Reg@\texttt{Reg}!fromStrSet@\texttt{fromStrSet}}%
\index{Reg@\texttt{Reg}!select@\texttt{select}}%
\index{Reg@\texttt{Reg}!update@\texttt{update}}%
\begin{verbatim}
val input         : string -> reg
val output        : string * reg -> unit
val size          : reg -> int
val numLeaves     : reg -> int
val height        : reg -> int
val emptyStr      : reg
val emptySet      : reg
val fromSym       : sym -> reg
val closure       : reg -> reg
val concat        : reg * reg -> reg
val union         : reg * reg -> reg
val compare       : reg * reg -> order
val equal         : reg * reg -> bool
val fromStr       : str -> reg
val power         : reg * int -> reg
val alphabet      : reg -> sym set
val genConcat     : reg list -> reg
val genUnion      : reg list -> reg
val rightConcat   : reg * reg -> reg
val rightUnion    : reg * reg -> reg
val concatsToList : reg -> reg list
val unionsToList  : reg -> reg list
val sortUnions    : reg -> reg
val allSym        : sym set -> reg
val allStr        : sym set -> reg
val fromStrSet    : str set -> reg
val select        : reg * int list -> reg
val update        : reg * int list * reg -> reg
\end{verbatim}

The Forlan syntax for regular expressions is the infix/postfix one
\index{Forlan!regular expression syntax}%
\index{regular expression!Forlan syntax}%
introduced in the preceding subsection, where $\alpha\myconcat\beta$
is always written as $\alpha\beta$, and we use parentheses
to override default precedences/associativities, or simply
for clarity.
For example, \texttt{0*10* + \%} and \texttt{(0*(1(0*))) + \%}
are the same regular expression.  And, \texttt{((0*)1)0* + \%}
is a different regular expression, but one with the same meaning.
Furthermore, \texttt{0*1(0* + \%)} is not only different from
the two preceding regular expressions, but it has a different
meaning (it fails to generate $\%$).
When regular expressions are outputted, as few parentheses as
possible are used.

The functions \texttt{size}, \texttt{numLeaves}
and \texttt{height} return the size, number of leaves and
height, respectively, of a regular expression.
The values \texttt{emptyStr} and \texttt{emptySet} are
$\%$ and $\$$, respectively.  The function \texttt{fromSym}
takes in a symbol $a$ and returns the regular expression $a$.
It is available in the
top-level environment as \texttt{symToReg}.
\index{symToReg@\texttt{symToReg}}%
The function \texttt{closure} takes in a regular expression $\alpha$ and
returns ${*}(\alpha)$.  The function \texttt{concat} takes a pair $(\alpha,
\beta)$ of regular expressions and returns ${@}(\alpha,\beta)$.
The function \texttt{union} takes a pair $(\alpha,
\beta)$ of regular expressions and returns ${+}(\alpha,\beta)$.
The function \texttt{compare} implements our total ordering on
regular expressions, and \texttt{equal} tests whether two
regular expressions are equal.
The function \texttt{fromStr} implements the function $\strToReg$,
and is also available in the top-level environment as \texttt{strToReg}.
\index{strToReg@\texttt{strToReg}}%
The function \texttt{power} raises a regular
expression to a power, and the
function \texttt{alphabet} returns the alphabet of a regular
expression.
The functions \texttt{genConcat}, \texttt{genUnion},
\texttt{rightConcat}, \texttt{rightUnion}, \texttt{concatsToList},
\texttt{unionsToList}, \texttt{sortUnions}, \texttt{allSym} and
\texttt{allStr} implement the functions with the same names.
The function \texttt{fromStrSet} returns $\$$, if called with
the empty set. Otherwise, it returns $\mathtt{fromStr}\,x_1 + \cdots +
\mathtt{fromStr}\,x_n$, where $x_1,\ldots,x_n$ are the elements
of its argument, listed in strictly ascending order.
The function \texttt{select} returns the subtree of a regular expression
at a given position (path), whereas the function \texttt{update}
makes a regular expression that is the same as its first argument except
that the subtree at the given position is equal to its third
argument. These functions issue error messages when given invalid
paths for their first arguments.

Here are some example uses of the functions of \texttt{Reg}:
\begin{list}{}
{\setlength{\leftmargin}{\leftmargini}
\setlength{\rightmargin}{0cm}
\setlength{\itemindent}{0cm}
\setlength{\listparindent}{0cm}
\setlength{\itemsep}{0cm}
\setlength{\parsep}{0cm}
\setlength{\labelsep}{0cm}
\setlength{\labelwidth}{0cm}
\catcode`\#=12
\catcode`\$=12
\catcode`\%=12
\catcode`\^=12
\catcode`\_=12
\catcode`\.=12
\catcode`\?=12
\catcode`\!=12
\catcode`\&=12
\ttfamily}
\small
\item[]\textsl{-\ }val\ reg\ =\ Reg.input\ "";
\item[]\textsl{@\ }0\symbol{'052}10\symbol{'052}\ +\ %
\item[]\textsl{@\ }.
\item[]\textsl{val\ reg\ =\ -\ :\ reg}
\item[]\textsl{-\ }Reg.size\ reg;
\item[]\textsl{val\ it\ =\ 9\ :\ int}
\item[]\textsl{-\ }Reg.numLeaves\ reg;
\item[]\textsl{val\ it\ =\ 4\ :\ int}
\item[]\textsl{-\ }val\ reg'\ =\ Reg.fromStr(Str.power(Str.input\ "",\ 3));
\item[]\textsl{@\ }01
\item[]\textsl{@\ }.
\item[]\textsl{val\ reg'\ =\ -\ :\ reg}
\item[]\textsl{-\ }Reg.output("",\ reg');
\item[]\textsl{010101}
\item[]\textsl{val\ it\ =\ ()\ :\ unit}
\item[]\textsl{-\ }Reg.size\ reg';
\item[]\textsl{val\ it\ =\ 11\ :\ int}
\item[]\textsl{-\ }Reg.numLeaves\ reg';
\item[]\textsl{val\ it\ =\ 6\ :\ int}
\item[]\textsl{-\ }Reg.compare(reg,\ reg');
\item[]\textsl{val\ it\ =\ GREATER\ :\ order}
\item[]\textsl{-\ }val\ reg''\ =\ Reg.concat(Reg.closure\ reg,\ reg');
\item[]\textsl{val\ reg''\ =\ -\ :\ reg}
\item[]\textsl{-\ }Reg.output("",\ reg'');
\item[]\textsl{(0\symbol{'052}10\symbol{'052}\ +\ %)\symbol{'052}010101}
\item[]\textsl{val\ it\ =\ ()\ :\ unit}
\item[]\textsl{-\ }SymSet.output("",\ Reg.alphabet\ reg'');
\item[]\textsl{0,\ 1}
\item[]\textsl{val\ it\ =\ ()\ :\ unit}
\item[]\textsl{-\ }val\ reg'''\ =\ Reg.power(reg,\ 3);
\item[]\textsl{val\ reg'''\ =\ -\ :\ reg}
\item[]\textsl{-\ }Reg.output("",\ reg''');
\item[]\textsl{(0\symbol{'052}10\symbol{'052}\ +\ %)(0\symbol{'052}10\symbol{'052}\ +\ %)(0\symbol{'052}10\symbol{'052}\ +\ %)}
\item[]\textsl{val\ it\ =\ ()\ :\ unit}
\item[]\textsl{-\ }Reg.size\ reg''';
\item[]\textsl{val\ it\ =\ 29\ :\ int}
\item[]\textsl{-\ }Reg.numLeaves\ reg''';
\item[]\textsl{val\ it\ =\ 12\ :\ int}
\item[]\textsl{-\ }Reg.output("",\ Reg.fromString\ "(0\symbol{'052}(1(0\symbol{'052})))\ +\ %");
\item[]\textsl{0\symbol{'052}10\symbol{'052}\ +\ %}
\item[]\textsl{val\ it\ =\ ()\ :\ unit}
\item[]\textsl{-\ }Reg.output("",\ Reg.fromString\ "(0\symbol{'052}1)0\symbol{'052}\ +\ %");
\item[]\textsl{(0\symbol{'052}1)0\symbol{'052}\ +\ %}
\item[]\textsl{val\ it\ =\ ()\ :\ unit}
\item[]\textsl{-\ }Reg.output("",\ Reg.fromString\ "0\symbol{'052}1(0\symbol{'052}\ +\ %)");
\item[]\textsl{0\symbol{'052}1(0\symbol{'052}\ +\ %)}
\item[]\textsl{val\ it\ =\ ()\ :\ unit}
\item[]\textsl{-\ }Reg.equal
\item[]\textsl{=\ }(Reg.fromString\ "0\symbol{'052}10\symbol{'052}\ +\ %",
\item[]\textsl{=\ }\ Reg.fromString\ "0\symbol{'052}1(0\symbol{'052}\ +\ %)");
\item[]\textsl{val\ it\ =\ false\ :\ bool}
\end{list}

We can use the functions \texttt{genConcat}, \texttt{genUnion},
\texttt{rightConcat}, \texttt{rightUnion}, \texttt{concatsToList},
\texttt{unionsToList} and \texttt{sortUnions} as follows:
\begin{list}{}
{\setlength{\leftmargin}{\leftmargini}
\setlength{\rightmargin}{0cm}
\setlength{\itemindent}{0cm}
\setlength{\listparindent}{0cm}
\setlength{\itemsep}{0cm}
\setlength{\parsep}{0cm}
\setlength{\labelsep}{0cm}
\setlength{\labelwidth}{0cm}
\catcode`\#=12
\catcode`\$=12
\catcode`\%=12
\catcode`\^=12
\catcode`\_=12
\catcode`\.=12
\catcode`\?=12
\catcode`\!=12
\catcode`\&=12
\ttfamily}
\small
\item[]\textsl{-\ }Reg.output("",\ Reg.genConcat\ nil);
\item[]\textsl{%}
\item[]\textsl{val\ it\ =\ ()\ :\ unit}
\item[]\textsl{-\ }Reg.output("",\ Reg.genUnion\ nil);
\item[]\textsl{$}
\item[]\textsl{val\ it\ =\ ()\ :\ unit}
\item[]\textsl{-\ }val\ regs\ =
\item[]\textsl{=\ }\ \ \ \ \ \ \symbol{'133}Reg.fromString\ "01",\ Reg.fromString\ "01\ +\ 12",
\item[]\textsl{=\ }\ \ \ \ \ \ \ Reg.fromString\ "(1\ +\ 2)\symbol{'052}",\ Reg.fromString\ "3\ +\ 4"\symbol{'135};
\item[]\textsl{val\ regs\ =\ \symbol{'133}-,-,-,-\symbol{'135}\ :\ reg\ list}
\item[]\textsl{-\ }Reg.output("",\ Reg.genConcat\ regs);
\item[]\textsl{(01)(01\ +\ 12)(1\ +\ 2)\symbol{'052}(3\ +\ 4)}
\item[]\textsl{val\ it\ =\ ()\ :\ unit}
\item[]\textsl{-\ }Reg.output("",\ Reg.genUnion\ regs);
\item[]\textsl{01\ +\ (01\ +\ 12)\ +\ (1\ +\ 2)\symbol{'052}\ +\ 3\ +\ 4}
\item[]\textsl{val\ it\ =\ ()\ :\ unit}
\item[]\textsl{-\ }Reg.output
\item[]\textsl{=\ }("",
\item[]\textsl{=\ }\ Reg.rightConcat
\item[]\textsl{=\ }\ (Reg.fromString\ "0123",\ Reg.fromString\ "4567"));
\item[]\textsl{01234567}
\item[]\textsl{val\ it\ =\ ()\ :\ unit}
\item[]\textsl{-\ }Reg.output
\item[]\textsl{=\ }("",
\item[]\textsl{=\ }\ Reg.rightUnion
\item[]\textsl{=\ }\ (Reg.fromString\ "0\ +\ 1\ +\ 2",\ Reg.fromString\ "1\ +\ 2\ +\ 3"));
\item[]\textsl{0\ +\ 1\ +\ 2\ +\ 1\ +\ 2\ +\ 3}
\item[]\textsl{val\ it\ =\ ()\ :\ unit}
\item[]\textsl{-\ }map
\item[]\textsl{=\ }Reg.toString
\item[]\textsl{=\ }(Reg.concatsToList(Reg.fromString\ "0(12)34"));
\item[]\textsl{val\ it\ =\ \symbol{'133}"0","12","3","4"\symbol{'135}\ :\ string\ list}
\item[]\textsl{-\ }map
\item[]\textsl{=\ }Reg.toString
\item[]\textsl{=\ }(Reg.unionsToList(Reg.fromString\ "0\ +\ (1\ +\ 2)\ +\ 3\ +\ 0"));
\item[]\textsl{val\ it\ =\ \symbol{'133}"0","1\ +\ 2","3","0"\symbol{'135}\ :\ string\ list}
\item[]\textsl{-\ }Reg.output
\item[]\textsl{=\ }("",\ Reg.sortUnions(Reg.fromString\ "12\ +\ 0\ +\ 3\ +\ 0"));
\item[]\textsl{0\ +\ 3\ +\ 12}
\item[]\textsl{val\ it\ =\ ()\ :\ unit}
\end{list}

We can use the functions \texttt{allSym}, \texttt{allStr} and
\texttt{fromStrSet} like this:
\begin{list}{}
{\setlength{\leftmargin}{\leftmargini}
\setlength{\rightmargin}{0cm}
\setlength{\itemindent}{0cm}
\setlength{\listparindent}{0cm}
\setlength{\itemsep}{0cm}
\setlength{\parsep}{0cm}
\setlength{\labelsep}{0cm}
\setlength{\labelwidth}{0cm}
\catcode`\#=12
\catcode`\$=12
\catcode`\%=12
\catcode`\^=12
\catcode`\_=12
\catcode`\.=12
\catcode`\?=12
\catcode`\!=12
\catcode`\&=12
\ttfamily}
\small
\item[]\textsl{-\ }Reg.output("",\ Reg.allSym(SymSet.fromString\ ""));
\item[]\textsl{$}
\item[]\textsl{val\ it\ =\ ()\ :\ unit}
\item[]\textsl{-\ }Reg.output("",\ Reg.allSym(SymSet.fromString\ "0"));
\item[]\textsl{0}
\item[]\textsl{val\ it\ =\ ()\ :\ unit}
\item[]\textsl{-\ }Reg.output("",\ Reg.allSym(SymSet.fromString\ "0,\ 1,\ 2"));
\item[]\textsl{0\ +\ 1\ +\ 2}
\item[]\textsl{val\ it\ =\ ()\ :\ unit}
\item[]\textsl{-\ }Reg.output("",\ Reg.allStr(SymSet.fromString\ ""));
\item[]\textsl{$\symbol{'052}}
\item[]\textsl{val\ it\ =\ ()\ :\ unit}
\item[]\textsl{-\ }Reg.output("",\ Reg.allStr(SymSet.fromString\ "0"));
\item[]\textsl{0\symbol{'052}}
\item[]\textsl{val\ it\ =\ ()\ :\ unit}
\item[]\textsl{-\ }Reg.output("",\ Reg.allStr(SymSet.fromString\ "2,\ 1,\ 0"));
\item[]\textsl{(0\ +\ 1\ +\ 2)\symbol{'052}}
\item[]\textsl{val\ it\ =\ ()\ :\ unit}
\item[]\textsl{-\ }Reg.output
\item[]\textsl{=\ }("",
\item[]\textsl{=\ }\ Reg.fromStrSet(StrSet.fromString\ "one,\ two,\ three,\ four"));
\item[]\textsl{one\ +\ two\ +\ four\ +\ three}
\item[]\textsl{val\ it\ =\ ()\ :\ unit}
\end{list}

Finally, we can use \texttt{select} and \texttt{update} like this:
\begin{list}{}
{\setlength{\leftmargin}{\leftmargini}
\setlength{\rightmargin}{0cm}
\setlength{\itemindent}{0cm}
\setlength{\listparindent}{0cm}
\setlength{\itemsep}{0cm}
\setlength{\parsep}{0cm}
\setlength{\labelsep}{0cm}
\setlength{\labelwidth}{0cm}
\catcode`\#=12
\catcode`\$=12
\catcode`\%=12
\catcode`\^=12
\catcode`\_=12
\catcode`\.=12
\catcode`\?=12
\catcode`\!=12
\catcode`\&=12
\ttfamily}
\small
\item[]\textsl{-\ }val\ reg\ =\ Reg.fromString\ "((0\ +\ 1)1\symbol{'052})\symbol{'052}";
\item[]\textsl{val\ reg\ =\ -\ :\ reg}
\item[]\textsl{-\ }Reg.output("",\ Reg.select(reg,\ \symbol{'133}1,\ 2\symbol{'135}));
\item[]\textsl{1\symbol{'052}}
\item[]\textsl{val\ it\ =\ ()\ :\ unit}
\item[]\textsl{-\ }val\ reg'\ =\ Reg.update(reg,\ \symbol{'133}1,\ 1\symbol{'135},\ Reg.fromString\ "2");
\item[]\textsl{val\ reg'\ =\ -\ :\ reg}
\item[]\textsl{-\ }Reg.output("",\ reg');
\item[]\textsl{(21\symbol{'052})\symbol{'052}}
\item[]\textsl{val\ it\ =\ ()\ :\ unit}
\end{list}


\subsection{JForlan}

The Java program JForlan
\index{JForlan}%
can be used to view and edit regular expression trees.  It can be
invoked directly, or run via Forlan.  See the Forlan website for more
information.

\subsection{Notes}

A novel feature of this book is that regular expressions are trees, so
that our linear syntax for regular expressions is derived rather than
primary.  Thus regular expression equality is just tree equality,
and it's easy to explain when parentheses are necessary in
a linear description of a regular expression.  Furthermore,
tree-oriented concepts, notation and operations automatically
apply to regular expressions, letting us, e.g., give definitions
by structural recursion.
\index{regular expression|)}%

%%% Local Variables: 
%%% mode: latex
%%% TeX-master: "book"
%%% End: 
