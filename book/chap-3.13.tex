\section{Equivalence-testing and Minimization of DFAs}
\label{EquivalenceTestingAndMinimizationOfDFAs}

In this section, we give algorithms for testing whether two DFAs are
equivalent, and for minimizing the alphabet size and number of states
of a DFA.  We also see how these functions can be used in Forlan.

\subsection{Testing the Equivalence of DFAs}

Suppose $M$ and $N$ are DFAs.  Our algorithm for checking whether they
are equivalent proceeds as follows.  First, it converts $M$ and $N$
into DFAs with identical alphabets.  Let
$\Sigma=\alphabet\,M\cup\alphabet\,N$, and define the DFAs $M'$ and
$N'$ by:
\begin{align*}
M' &= \determSimplify(M,\Sigma) , \eqtxt{and}\\
N' &= \determSimplify(N,\Sigma) .
\end{align*}
Since $\alphabet(L(M))\sub\alphabet\,M\sub\Sigma$, we have that
$\alphabet\,M'=\alphabet(L(M))\cup\Sigma=\Sigma$.  Similarly,
$\alphabet\,N'=\Sigma$.  Furthermore, $M'\approx M$ and $N'\approx N$,
so that it will suffice to determine whether $M'$ and $N'$ are
equivalent.

For example, if $M$ and $N$ are the DFAs
\begin{center}
\input{chap-3.13-fig1.eepic}
\end{center}
then $\Sigma=\{\zerosf,\onesf\}$, $M'=M$ and $N'=N$.

Next, the algorithm generates the least subset $X$ of $Q_{M'}\times Q_{N'}$
such that
\begin{itemize}
\item $(s_{M'},s_{N'})\in X$; and

\item for all $q\in Q_{M'}$, $r\in Q_{N'}$ and $a\in\Sigma$,
if $(q,r)\in X$, then $(\delta_{M'}(q,a),\delta_{N'}(r,a))\in X$.
\end{itemize}
With our example DFAs $M'$ and $N'$, we have that
\begin{itemize}
\item $(\Asf,\Asf)\in X$;

\item since $(\Asf,\Asf)\in X$, we have that {$(\Bsf,\Bsf)\in X$ and
    $(\Asf,\Csf)\in X$;}

\item since $(\Bsf,\Bsf)\in X$, we have that (again) $(\Asf,\Csf)\in
  X$ and (again) $(\Bsf,\Bsf)\in X$; and

\item since $(\Asf,\Csf)\in X$, we have that (again) $(\Bsf,\Bsf)\in
  X$ and (again) $(\Asf,\Asf)\in X$.
\end{itemize}

Back in the general case, we have the following lemmas.

\begin{lemma}
\label{EquivLem1}
For all $w\in\Sigma^*$, $(\delta_{M'}(s_{M'},w),\delta_{N'}(s_{N'},w))\in X$.
\end{lemma}

\begin{proof}
By left string induction on $w$.
\end{proof}

\begin{lemma}
\label{EquivLem2}
For all $q\in Q_{M'}$ and $r\in Q_{N'}$, if $(q,r)\in X$, then there
is a $w\in\Sigma^*$ such that $q=\delta_{M'}(s_{M'},w)$ and
$r=\delta_{N'}(s_{N'},w)$.
\end{lemma}

\begin{proof}
By induction on $X$.
\end{proof}

Finally, the algorithm checks that, for all $(q,r)\in X$,
\begin{gather*}
q\in A_{M'}\myiff r\in A_{N'} .
\end{gather*}
If this is true, it says that the machines are equivalent; otherwise
it says they are not equivalent.

We can easily prove the correctness of our algorithm:
\begin{itemize}
\item Suppose every pair $(q,r)\in X$ consists of two accepting states
  or of two non-accepting states.  Suppose, toward a contradiction,
  that $L(M')\neq L(N')$.  Then there is a string $w$ that is accepted
  by one of the machines but is not accepted by the other.  Since both
  machines have alphabet $\Sigma$, we have that $w\in\Sigma^*$.  Thus
  Lemma~\ref{EquivLem1} tells us that
  $(\delta_{M'}(s_{M'},w),\delta_{N'}(s_{N'},w))\in X$.  But one side
  of this pair is an accepting state and the other is a non-accepting
  one---contradiction.  Thus $L(M')=L(N')$.

\item Suppose we find a pair $(q,r)\in X$ such that one of $q$ and $r$
  is an accepting state but the other is not.  By
  Lemma~\ref{EquivLem2}, it will follow that there is a $w\in\Sigma^*$
  such that $q=\delta_{M'}(s_{M'},w)$ and $r=\delta_{N'}(s_{N'},w)$.
  Thus $w$ is accepted by one machine but not the other, so that
  $L(M')\neq L(N')$.
\end{itemize}

In the case of our example, we have that
$X=\{(\Asf,\Asf), (\Bsf,\Bsf), (\Asf,\Csf)\}$.
Since $(\Asf,\Asf)$ and $(\Asf,\Csf)$ are pairs of accepting states,
and $(\Bsf,\Bsf)$ is a pair of non-accepting states, it follows
that $L(M')=L(N')$.  Hence $L(M)=L(N)$.

By annotating each element $(q,r)\in X$ with a string $w$ such that
$q=\delta_{M'}(s_{M'},w)$ and $r=\delta_{N'}(s_{N'},w)$, instead of
just reporting that $M'$ and $N'$ are not equivalent, we can
explain why they are not equivalent,
\begin{itemize}
\item giving a string that is accepted by the first machine but not by
  the second; and/or

\item giving a string that is accepted by the second machine but not
  by the first.
\end{itemize}
We can even arrange for these strings to be of minimum length.
The Forlan implementation of our algorithm always produces minimum-length
counterexamples.

The Forlan module \texttt{DFA} defines the functions:
\begin{verbatim}
val relationship : dfa * dfa -> unit
val subset       : dfa * dfa -> bool
val equivalent   : dfa * dfa -> bool
\end{verbatim}
The function \texttt{relationship} figures out the relationship
between the languages accepted by two DFAs (are they equal, is one a
proper subset of the other, is neither a subset of the other), and
supplies minimum-length counterexamples to justify negative answers.
The function \texttt{subset} tests whether its first argument's
language is a subset of its second argument's language.
The function \texttt{equivalent} tests whether two DFAs are
equivalent.

Note that \texttt{subset} (when turned into a function of type
\texttt{reg~*~reg~->~bool}---see below) can be used in conjunction
with the local and global simplification algorithms of Section~3.3.

For example, suppose \texttt{dfa1} and \texttt{dfa2} of type \texttt{dfa} are
bound to our example DFAs $M$ and $N$, respectively:
\begin{center}
\input{chap-3.13-fig1.eepic}
\end{center}
We can verify that these machines are equivalent as follows:
\begin{list}{}
{\setlength{\leftmargin}{\leftmargini}
\setlength{\rightmargin}{0cm}
\setlength{\itemindent}{0cm}
\setlength{\listparindent}{0cm}
\setlength{\itemsep}{0cm}
\setlength{\parsep}{0cm}
\setlength{\labelsep}{0cm}
\setlength{\labelwidth}{0cm}
\catcode`\#=12
\catcode`\$=12
\catcode`\%=12
\catcode`\^=12
\catcode`\_=12
\catcode`\.=12
\catcode`\?=12
\catcode`\!=12
\catcode`\&=12
\ttfamily}
\small
\item[]\textsl{-\ }DFA.relationship(dfa1,\ dfa2);
\item[]\textsl{languages\ are\ equal}
\item[]\textsl{val\ it\ =\ ()\ :\ unit}
\end{list}


On the other hand, suppose that \texttt{dfa3} and \texttt{dfa4} of type
texttt{dfa} are bound to the DFAs:
\begin{center}
\input{chap-3.13-fig2.eepic}
\end{center}
We can find out why these machines are not equivalent as follows:
\begin{list}{}
{\setlength{\leftmargin}{\leftmargini}
\setlength{\rightmargin}{0cm}
\setlength{\itemindent}{0cm}
\setlength{\listparindent}{0cm}
\setlength{\itemsep}{0cm}
\setlength{\parsep}{0cm}
\setlength{\labelsep}{0cm}
\setlength{\labelwidth}{0cm}
\catcode`\#=12
\catcode`\$=12
\catcode`\%=12
\catcode`\^=12
\catcode`\_=12
\catcode`\.=12
\catcode`\?=12
\catcode`\!=12
\catcode`\&=12
\ttfamily}
\small
\item[]\textsl{-\ }DFA.relationship(dfa3,\ dfa4);
\item[]\textsl{neither\ language\ is\ a\ subset\ of\ the\ other\ language:\ "11"\ is\ in}
\item[]\textsl{first\ language\ but\ is\ not\ in\ second\ language;\ "110"\ is\ in\ second}
\item[]\textsl{language\ but\ is\ not\ in\ first\ language}
\item[]\textsl{val\ it\ =\ ()\ :\ unit}
\end{list}


We can find the relationship between the languages generated by regular
expressions \texttt{reg1} and \texttt{reg2} by:
\begin{itemize}
\item  converting \texttt{reg1} and \texttt{reg2} to DFAs
texttt{dfa1} and \texttt{dfa2}, and then

\item  running \texttt{DFA.relationship(dfa1, dfa2)} to find
the relationship between those DFAs.
\end{itemize}

Of course, we can define an ML/Forlan function that
carries out these actions:
\begin{list}{}
{\setlength{\leftmargin}{\leftmargini}
\setlength{\rightmargin}{0cm}
\setlength{\itemindent}{0cm}
\setlength{\listparindent}{0cm}
\setlength{\itemsep}{0cm}
\setlength{\parsep}{0cm}
\setlength{\labelsep}{0cm}
\setlength{\labelwidth}{0cm}
\catcode`\#=12
\catcode`\$=12
\catcode`\%=12
\catcode`\^=12
\catcode`\_=12
\catcode`\.=12
\catcode`\?=12
\catcode`\!=12
\catcode`\&=12
\ttfamily}
\small
\item[]\textsl{-\ }fun\ regToDFA\ reg\ =
\item[]\textsl{=\ }\ \ \ \ \ \ nfaToDFA(efaToNFA(faToEFA(regToFA\ reg)));
\item[]\textsl{val\ regToDFA\ =\ fn\ :\ reg\ ->\ dfa}
\item[]\textsl{-\ }fun\ relationshipReg(reg1,\ reg2)\ =
\item[]\textsl{=\ }\ \ \ \ \ \ DFA.relationship(regToDFA\ reg1,\ regToDFA\ reg2);
\item[]\textsl{val\ relationshipReg\ =\ fn\ :\ reg\ \symbol{'052}\ reg\ ->\ unit}
\end{list}


\subsection{Minimization of DFAs}

Now, we consider an algorithm for minimizing the sizes of the alphabet
and set of states of a DFA $M$.  First, the algorithm minimizes the
size of $M$'s alphabet, and makes the automaton be deterministically
simplified, by letting $M'=\determSimplify(M,\emptyset)$.  Thus
$M'\approx M$, $\alphabet\,M'=\alphabet(L(M))$ $|Q_{M'}|\leq|Q_M|$.

For example, if $M$ is the DFA
\begin{center}
\input{chap-3.13-fig3.eepic}
\end{center}
then $M'=M$.

Next, the algorithm generates the least subset $X$ of $Q_{M'}\times
Q_{M'}$ such that:
\begin{enumerate}[\quad(1)]
\item $A_{M'}\times(Q_{M'}-A_{M'})\sub X$;

\item $(Q_{M'}-A_{M'})\times A_{M'}\sub X$; and

\item for all $q,q',r,r'\in Q_{M'}$ and $a\in\alphabet\,M'$,
if $(q,r)\in X$, $(q',a, q)\in T_{M'}$ and $(r',a, r)\in T_{M'}$, then
$(q',r')\in X$.
\end{enumerate}
We read ``$(q,r)\in X$'' as ``$q$ and $r$ cannot be merged''.
The idea of (1) and (2) is that an accepting state can never be merged
with a non-accepting state.  And (3) says that if $q$ and $r$ can't
be merged, and we can get from $q'$ to $q$ by processing an $a$, and
from $r'$ to $r$ by processing an $a$, then
$q'$ and $r'$ also can't be merged---since if we merged $q'$ and $r'$,
there would have to be an $a$-transition from the merged state
to the merging of $q$ and $r$.

In the case of our example $M'$, (1) tells us to add the pairs
$(\Esf,\Asf)$, $(\Esf,\Bsf)$, $(\Esf,\Csf)$, $(\Esf,\Dsf)$,
$(\Fsf,\Asf)$, $(\Fsf,\Bsf)$, $(\Fsf,\Csf)$ and $(\Fsf,\Dsf)$ to $X$.
And, (2) tells us to add the pairs $(\Asf,\Esf)$, $(\Bsf,\Esf)$,
$(\Csf,\Esf)$, $(\Dsf,\Esf)$, $(\Asf,\Fsf)$, $(\Bsf,\Fsf)$,
$(\Csf,\Fsf)$ and $(\Dsf,\Fsf)$ to $X$.

Now we use rule (3) to compute the rest of $X$'s elements.  To begin
with, we must handle each pair that has already been added to $X$.
\begin{itemize}
\item Since there are no transitions leading into $\Asf$, no pairs can
  be added using $(\Esf,\Asf)$, $(\Asf,\Esf)$, $(\Fsf,\Asf)$ and
  $(\Asf,\Fsf)$.

\item Since there are no $\zerosf$-transitions leading into $\Esf$,
  and there are no $\onesf$-transitions leading into $\Bsf$, no pairs
  can be added using $(\Esf,\Bsf)$ and $(\Bsf,\Esf)$.

\item Since $(\Esf,\Csf),(\Csf,\Esf)\in X$ and $(\Bsf,\onesf,\Esf)$,
  $(\Dsf,\onesf,\Esf)$, $(\Fsf,\onesf,\Esf)$ and $(\Asf,\onesf,\Csf)$
  are the $\onesf$-transitions leading into $\Esf$ and $\Csf$, we add
  {$(\Bsf,\Asf)$ and $(\Asf,\Bsf)$, and $(\Dsf,\Asf)$ and
    $(\Asf,\Dsf)$} to $X$; {we would also have added $(\Fsf,\Asf)$ and
    $(\Asf,\Fsf)$ to $X$ if they hadn't been previously added.}  Since
  there are no $\zerosf$-transitions into $\Esf$, nothing can be added
  to $X$ using $(\Esf,\Csf)$ and $(\Csf,\Esf)$ and
  $\zerosf$-transitions.

\item Since $(\Esf,\Dsf),(\Dsf,\Esf)\in X$ and $(\Bsf,\onesf,\Esf)$,
  $(\Dsf,\onesf,\Esf)$, $(\Fsf,\onesf,\Esf)$ and $(\Csf,\onesf,\Dsf)$
  are the $\onesf$-transitions leading into $\Esf$ and $\Dsf$, we add
  {$(\Bsf,\Csf)$ and $(\Csf,\Bsf)$, and $(\Dsf,\Csf)$ and
    $(\Csf,\Dsf)$} to $X$; { we would also have added $(\Fsf,\Csf)$
    and $(\Csf,\Fsf)$ to $X$ if they hadn't been previously added.}
  Since there are no $\zerosf$-transitions into $\Esf$, nothing can be
  added to $X$ using $(\Esf,\Dsf)$ and $(\Dsf,\Esf)$ and
  $\zerosf$-transitions.

\item Since $(\Fsf,\Bsf),(\Bsf,\Fsf)\in X$ and $(\Esf,\zerosf,\Fsf)$,
  $(\Fsf,\zerosf,\Fsf)$, $(\Asf,\zerosf,\Bsf)$, and
  $(\Dsf,\zerosf,\Bsf)$ are the $\zerosf$-transitions leading into
  $\Fsf$ and $\Bsf$, we would have to add the following pairs to $X$,
  if they were not already present: $(\Esf,\Asf)$, $(\Asf,\Esf)$,
  $(\Esf,\Dsf)$, $(\Dsf,\Esf)$, $(\Fsf,\Asf)$, $(\Asf,\Fsf)$,
  $(\Fsf,\Dsf)$, $(\Dsf,\Fsf)$.  Since there are no
  $\onesf$-transitions leading into $\Bsf$, no pairs can be added
  using $(\Fsf,\Bsf)$ and $(\Bsf,\Fsf)$ and $\onesf$-transitions.

\item Since $(\Fsf,\Csf),(\Csf,\Fsf)\in X$ and $(\Esf,\onesf,\Fsf)$
  and $(\Asf,\onesf,\Csf)$ are the $\onesf$-transitions leading into
  $\Fsf$ and $\Csf$, we would have to add $(\Esf,\Asf)$ and
  $(\Asf,\Esf)$ to $X$ if these pairs weren't already present.  Since
  there are no $\zerosf$-transitions leading into $\Csf$, no pairs can
  be added using $(\Fsf,\Csf)$ and $(\Csf,\Fsf)$ and
  $\zerosf$-transitions.

\item Since $(\Fsf,\Dsf),(\Dsf,\Fsf)\in X$ and $(\Esf,\zerosf,\Fsf)$,
  $(\Fsf,\zerosf,\Fsf)$, $(\Bsf,\zerosf,\Dsf)$ and
  $(\Csf,\zerosf,\Dsf)$ are the $\zerosf$-transitions leading into
  $\Fsf$ and $\Dsf$, we would add $(\Esf,\Bsf)$, $(\Bsf,\Esf)$,
  $(\Esf,\Csf)$, $(\Csf,\Esf)$, $(\Fsf,\Bsf)$, $(\Bsf,\Fsf)$,
  $(\Fsf,\Csf)$, and $(\Csf,\Fsf)$ to $X$, if these pairs weren't
  already present.  Since $(\Fsf,\Dsf),(\Dsf,\Fsf)\in X$ and
  $(\Esf,\onesf,\Fsf)$ and $(\Csf,\onesf,\Dsf)$ are the
  $\onesf$-transitions leading into $\Fsf$ and $\Dsf$, we would add
  $(\Esf,\Csf)$ and $(\Csf,\Esf)$ to $X$, if these pairs weren't
  already in $X$.
\end{itemize}

We've now handled all of the elements of $X$ that were added using
rules~(1) and (2).  We must now handle the pairs that were
subsequently added: $(\Asf,\Bsf)$, $(\Bsf,\Asf)$, $(\Asf,\Dsf)$,
$(\Dsf,\Asf)$, $(\Bsf,\Csf)$, $(\Csf,\Bsf)$, $(\Csf,\Dsf)$,
$(\Dsf,\Csf)$.
\begin{itemize}
\item Since there are no transitions leading into $\Asf$, no pairs can
  be added using $(\Asf,\Bsf)$, $(\Bsf,\Asf)$, $(\Asf,\Dsf)$ and
  $(\Dsf,\Asf)$.

\item Since there are no $\onesf$-transitions leading into $\Bsf$, and
  there are no $\zerosf$-transitions leading into $\Csf$, no pairs can
  be added using $(\Bsf,\Csf)$ and $(\Csf,\Bsf)$.

\item Since $(\Csf,\Dsf), (\Dsf,\Csf)\in X$ and $(\Asf,\onesf,\Csf)$
  and $(\Csf,\onesf,\Dsf)$ are the $\onesf$-transitions leading into
  $\Csf$ and $\Dsf$, we add the pairs {$(\Asf,\Csf)$ and $(\Csf,\Asf)$
    to $X$}.  Since there are no $\zerosf$-transitions leading into
  $\Csf$, no pairs can be added to $X$ using $(\Csf,\Dsf)$ and
  $(\Dsf,\Csf)$ and $\zerosf$-transitions.
\end{itemize}

Now, we must handle the pairs that were added in the last
phase: $(\Asf,\Csf)$ and $(\Csf,\Asf)$.
\begin{itemize}
\item Since there are no transitions leading into $\Asf$, no pairs can
  be added using $(\Asf,\Csf)$ and $(\Csf,\Asf)$.
\end{itemize}

Since we have handled all the pairs we added to $X$, we are now done.
Here are the 26 elements of $X$: $(\Asf,\Bsf)$, $(\Asf,\Csf)$,
$(\Asf,\Dsf)$, $(\Asf,\Esf)$, $(\Asf,\Fsf)$, $(\Bsf,\Asf)$,
$(\Bsf,\Csf)$, $(\Bsf,\Esf)$, $(\Bsf,\Fsf)$, $(\Csf,\Asf)$,
$(\Csf,\Bsf)$, $(\Csf,\Dsf)$, $(\Csf,\Esf)$, $(\Csf,\Fsf)$,
$(\Dsf,\Asf)$, $(\Dsf,\Csf)$, $(\Dsf,\Esf)$, $(\Dsf,\Fsf)$,
$(\Esf,\Asf)$, $(\Esf,\Bsf)$, $(\Esf,\Csf)$, $(\Esf,\Dsf)$,
$(\Fsf,\Asf)$, $(\Fsf,\Bsf)$, $(\Fsf,\Csf)$, $(\Fsf,\Dsf)$.

Back in the general case, we have the following lemmas.

\begin{lemma}
\label{MinimizationLemma1}
For all $(q,r)\in X$, there is a $w\in(\alphabet\,M')^*$, such that
exactly one of $\delta_{M'}(q,w)$ and $\delta_{M'}(r,w)$ is in $A_{M'}$.
\end{lemma}

\begin{proof}
By induction on $X$.
\end{proof}

\begin{lemma}
\label{MinimizationLemma2}
For all $w\in(\alphabet\,M')^*$, for all $q,r\in Q_{M'}$, if
exactly one of $\delta_{M'}(q,w)$ and $\delta_{M'}(r,w)$ is
in $A_{M'}$, then $(q,r)\in X$.
\end{lemma}

\begin{proof}
By right string induction.
\end{proof}

Next, the algorithm lets the relation $Y=(Q_{M'}\times Q_{M'})-X$.  We
read ``$(q,r)\in Y$'' as ``$q$ and $r$ can be merged''.
Back with our example, we have that $Y$ is
\begin{gather*}
\{\mathsf{(A,A), (B,B), (C,C), (D,D), (E,E), (F,F)}\} \\
\cup \\
\{\mathsf{(B,D), (D,B), (F,E), (E,F)}\} .
\end{gather*}

\begin{lemma}
\label{MinimizationLemma3}
\begin{enumerate}[\quad (1)]
\item For all $q,r\in Q_{M'}$, $(q,r)\in Y$ iff, for all
  $w\in(\alphabet\,M')^*$, $\delta_{M'}(q,w)\in A_{M'}$ iff
  $\delta_{M'}(r,w)\in A_{M'}$.

\item For all $q,r\in Q_{M'}$, if $(q,r)\in Y$, then $q\in A_{M'}$ iff
  $r\in A_{M'}$.

\item For all $q,r\in Q_{M'}$ and $a\in\alphabet\,M'$, if $(q,r)\in
  Y$, then $(\delta_{M'}(q,a),\delta_{M'}(r,a))\in Y$.
\end{enumerate}
\end{lemma}

\begin{proof}
\begin{enumerate}[\quad(1)]
\item Follows using Lemmas~\ref{MinimizationLemma1} and
  \ref{MinimizationLemma2}.

\item Follows by Part~(1), when $w=\%$.

\item Follows by Part~(1).
\end{enumerate}
\end{proof}

The following lemma says that $Y$ is an \emph{equivalence relation on}
$Q_{M'}$.

\begin{lemma}
\label{MinimizationLemma4}
$Y$ is reflexive on $Q_{M'}$, symmetric and transitive.
\end{lemma}

\begin{proof}
Follows from Lemma~\ref{MinimizationLemma3}(1).
\end{proof}

In order to define the DFA $N$ that is the result of our minimization
algorithm, we need a bit more notation.  As in
Section~\ref{DeterministicFiniteAutomata}, we write $\overline{P}$ for
the result of coding a finite set of symbols $P$ as a symbol.  E.g.,
$\overline{\{\Bsf,\Asf\}} = \langle\Asf,\Bsf\rangle$.

If $q\in Q_{M'}$, we write $[q]$ for $\setof{p\in Q_{M'}}{(p,q)\in
  Y}$, which is called the \emph{equivalence class} of $q$.  Using
Lemma~\ref{MinimizationLemma4}, it is easy to show that, $q\in[q]$,
for all $q\in Q_{M'}$, and $[q]=[r]$ iff $(q,r)\in Y$, for all $q,r\in
Q_{M'}$.

If $P$ is a nonempty, finite set of symbols, then we write $\min\,P$
for the least element of $P$, according to our standard ordering on
symbols.

The algorithm lets $Z=\setof{[q]}{q\in Q_{M'}}$, which is finite since
$Q_{M'}$ is finite.
In the case of our example, $Z$ is
\begin{gather*}
\mathsf{\{\{A\}, \{B,D\}, \{C\}, \{E,F\}\}} .
\end{gather*}
Finally, the algorithm defines the DFA $N$ as follows:
\begin{itemize}
\item $Q_N = \setof{\overline{P}}{P\in Z}$;

\item $s_N = \overline{[s_{M'}]}$;

\item $A_N = \setof{\overline{P}}{P\in Z\eqtxt{and}\min\,P\in
 A_{M'}}$; and

\item $T_N = \setof{(\overline{P},a,
{\overline{[\delta_{M'}(\min\,P,a)]}})}%
{P\in Z\eqtxt{and}a\in\alphabet\,M'}$.
\end{itemize}
Then $N$ is a DFA with alphabet
$\alphabet\,M'=\alphabet(L(M))$, $|Q_N|\leq|Q_{M'}|\leq|Q_M|$, and,
for all $P\in Z$ and $a\in\alphabet\,M'$,
$\delta_N(\overline{P},a)= \overline{[\delta_{M'}(\min\,P,a)]}$.

In the case of our example, we have that
\begin{itemize}
\item $Q_N = \{\mathsf{\langle A\rangle, \langle B,D\rangle,
\langle C\rangle, \langle E,F\rangle}\}$;

\item $s_N = \langle A\rangle$; and

\item $A_N = \{\langle\Esf,\Fsf\rangle\}$.
\end{itemize}
We compute the elements of $T_N$ as follows.
\begin{itemize}
\item Since $\{\Asf\}\in Z$ and $[\delta_{M'}(\Asf,\zerosf)]=
  [\Bsf]=\{\Bsf,\Dsf\}$, we have that
  $(\langle\Asf\rangle,\zerosf,\langle\Bsf,\Dsf\rangle)\in T_N$.

  Since $\{\Asf\}\in Z$ and $[\delta_{M'}(\Asf,\onesf)]=
  [\Csf]=\{\Csf\}$, we have that
  $(\langle\Asf\rangle,\onesf,\langle\Csf\rangle)\in T_N$.

\item Since $\{\Csf\}\in Z$ and $[\delta_{M'}(\Csf,\zerosf)]=
  [\Dsf]=\{\Bsf,\Dsf\}$, we have that
  $(\langle\Csf\rangle,\zerosf,\langle\Bsf,\Dsf\rangle)\in T_N$.

  Since $\{\Csf\}\in Z$ and $[\delta_{M'}(\Csf,\onesf)]=
  [\Dsf]=\{\Bsf,\Dsf\}$, we have that
  $(\langle\Csf\rangle,\onesf,\langle\Bsf,\Dsf\rangle)\in T_N$.

\item Since $\{\Bsf,\Dsf\}\in Z$ and $[\delta_{M'}(\Bsf,\zerosf)]=
  [\Dsf]=\{\Bsf,\Dsf\}$, we have that
  $(\langle\Bsf,\Dsf\rangle,\zerosf,\langle\Bsf,\Dsf\rangle)\in T_N$.

  Since $\{\Bsf,\Dsf\}\in Z$ and $[\delta_{M'}(\Bsf,\onesf)]=
  [\Esf]=\{\Esf,\Fsf\}$, we have that
  $(\langle\Bsf,\Dsf\rangle,\onesf,\langle\Esf,\Fsf\rangle)\in T_N$.

\item Since $\{\Esf,\Fsf\}\in Z$ and $[\delta_{M'}(\Esf,\zerosf)]=
  [\Fsf]=\{\Esf,\Fsf\}$, we have that
  $(\langle\Esf,\Fsf\rangle,\zerosf,\langle\Esf,\Fsf\rangle)\in T_N$.

  Since $\{\Esf,\Fsf\}\in Z$ and $[\delta_{M'}(\Esf,\onesf)]=
  [\Fsf]=\{\Esf,\Fsf\}$, we have that
  $(\langle\Esf,\Fsf\rangle,\onesf,\langle\Esf,\Fsf\rangle)\in T_N$.
\end{itemize}
Thus our DFA $N$ is:
\begin{center}
\input{chap-3.13-fig4.eepic}
\end{center}

Back in the general case, we have the following lemmas.

\begin{lemma}
\label{MinimizationLemma5}
\begin{enumerate}[\quad(1)]
\item For all $q\in Q_{M'}$, $\overline{[q]}\in A_N$ iff $q\in A_{M'}$.

\item For all $q\in Q_{M'}$ and $a\in\alphabet\,M'$,
  $\delta_N(\overline{[q]},a) = \overline{[\delta_{M'}(q,a)]}$.

\item For all $q\in Q_{M'}$ and $w\in(\alphabet\,M')^*$,
  $\delta_N(\overline{[q]},w) = \overline{[\delta_{M'}(q, w)]}$.

\item For all $w\in(\alphabet\,M')^*$, $\delta_N(s_N,w) =
  \overline{[\delta_{M'}(s_{M'}, w)]}$.
\end{enumerate}
\end{lemma}

\begin{proof}
(1) and (2) follow easily by Lemma~\ref{MinimizationLemma3}(2)--(3).
Part~(3) follows from Part~(2) by left string induction.  For
Part~(4), suppose $w\in(\alphabet\,M')^*$.  By Part~(3), we have
\begin{gather*}
\delta_N(s_N,w) =
\delta_N(\overline{[s_{M'}]}, w) =
\overline{[\delta_{M'}(s_{M'}, w)]} .
\end{gather*}
\end{proof}

\begin{lemma}
\label{MinimizationLemma6}
$L(N)=L(M')$.
\end{lemma}

\begin{proof}
Suppose $w\in L(N)$.  Then $w\in(\alphabet\,N)^*=(\alphabet\,M')^*$ and
$\delta_N(s_N,w)\in A_N$.
By Lemma~\ref{MinimizationLemma5}(4), we have that
\begin{gather*}
  \overline{[\delta_{M'}(s_{M'}, w)]} =
  \delta_N(s_N,w) \in A_N ,
\end{gather*}
so that $\delta_{M'}(s_{M'}, w)\in A_{M'}$, by
Lemma~\ref{MinimizationLemma5}(1).  Thus $w\in L(M')$.

Suppose $w\in L(M')$.  Then $w\in(\alphabet\,M')^*=(\alphabet\,N)^*$ and
$\delta_{M'}(s_{M'}, w)\in A_{M'}$.  By 
Lemma~\ref{MinimizationLemma5}(1) and (4), we have that
\begin{gather*}
  \delta_N(s_N,w) =
  \overline{[\delta_{M'}(s_{M'}, w)]}
  \in A_N .
\end{gather*}
Hence $w\in L(N)$.
\end{proof}

\begin{lemma}
\label{MinimizationLemma7}
$N$ is deterministically simplified.
\end{lemma}

\begin{proof}
To see that all elements of $Q_N$ are reachable, suppose $q\in Q_{M'}$.
Because $M'$ is deterministically simplified, there is a
$w\in(\alphabet\,M')^*$ such that $q=\delta_{M'}(s_{M'},w)$.  Thus
$\delta_N(s_N,w) = \overline{[\delta_{M'}(s_{M'},w)]} = \overline{[q]}$.

Next, we show that, for all $q\in Q_{M'}$, if $q$ is live, then
$\overline{[q]}$ is live.  Suppose $q\in Q_{M'}$ is live,
so there is a $w\in(\alphabet\,M')^*$ such that $\delta_{M'}(q,w)\in A_{M'}$.
Thus $\delta_N(\overline{[q]},w)=\overline{[\delta_{M'}(q,w)]}\in A_N$,
showing that $\overline{[q]}$ is live.

Thus, we have that, for all $q\in Q_{M'}$, if $\overline{[q]}$ is
dead, then $q$ is dead.  But, $M'$ has at most one dead state, and
thus we have that $N$ has at most one dead state.
\end{proof}

\begin{lemma}
\label{MinimizationLemma8}
Suppose $N'$ is a DFA such that $N'\approx M'$,
$\alphabet\,N'=\alphabet\,M'$ and $|Q_{N'}|\leq|Q_N|$.  Then $N'$ is
isomorphic to $N$.
\index{isomorphism!finite automaton}%
\index{finite automaton!isomorphism}%
\end{lemma}

\begin{proof}
We have that $L(N')=L(M')=L(N)$.  And the states of $M'$ and $N$ are
all reachable.  Let the relation $h$ between $Q_{N'}$ and $Q_N$ be
\begin{gather*}
  \setof{(\delta_{N'}(s_{N'},w),\delta_N(s_N,w))}{w\in(\alphabet\,M')^*}
  .
\end{gather*}
Since every state of $N$ is reachable, it follows that $\range\,h=
Q_N$.

To see that $h$ is a function, suppose $x,y\in(\alphabet\,M')^*$ and
$\delta_{N'}(s_{N'},x)=\delta_{N'}(s_{N'},y)$.  We must show that
$\delta_N(s_N,x)=\delta_N(s_N,y)$.  Since
$\delta_N(s_N,x)=\overline{[\delta_{M'}(s_{M'},x)]}$ and
$\delta_N(s_N,y)=\overline{[\delta_{M'}(s_{M'},y)]}$, it will suffice
to show that $(\delta_{M'}(s_{M'},x),\delta_{M'}(s_{M'},y))\in Y$.  By
Lemma~\ref{MinimizationLemma3}(1), it will suffice to show that,
$\delta_{M'}(\delta_{M'}(s_{M'},x), z)\in A_{M'}$ iff
$\delta_{M'}(\delta_{M'}(s_{M'},y), z)\in A_{M'}$, for all
$z\in(\alphabet\,M')^*$.  Suppose $z\in(\alphabet\,M')^*$.  We must
show that $\delta_{M'}(\delta_{M'}(s_{M'},x), z)\in A_{M'}$ iff
$\delta_{M'}(\delta_{M'}(s_{M'},y), z)\in A_{M'}$

We will show the ``only if'' direction, the other direction being
similar.  Suppose $\delta_{M'}(\delta_{M'}(s_{M'},x), z)\in A_{M'}$.
We must show that $\delta_{M'}(\delta_{M'}(s_{M'},y), z)\in A_{M'}$.
Because $\delta_{M'}(s_{M'},xz) = \delta_{M'}(\delta_{M'}(s_{M'},x),
z)\in A_{M'}$, we have that $xz\in L(M')=L(N')$.  Since $xz\in L(N')$
and $\delta_{N'}(s_{N'},x)=\delta_{N'}(s_{N'},y)$, we have that
\begin{align*}
  \delta_{N'}(s_{N'},yz) &= \delta_{N'}(\delta_{N'}(s_{N'},y), z) =
  \delta_{N'}(\delta_{N'}(s_{N'},x), z)\\
  &= \delta_{N'}(s_{N'},xz) \in A_{N'} ,
\end{align*}
so that $yz\in L(N')=L(M')$.  Hence
$\delta_{M'}(\delta_{M'}(s_{M'},y), z) = \delta_{M'}(s_{M'},yz) \in
A_{M'}$.

Because $h$ is a function and $\range\,h=Q_N$, we have that
$|Q_N|\leq |\domain\,f|\leq |Q_{N'}|$. But $|Q_{N'}|\leq |Q_N|$, and thus
$|Q_{N'}| = |Q_N|$. Because $Q_{N'}$ and $Q_N$ are finite, it
follows that $\domain\,h = Q_{N'}$ and $h$ is injective, so that $h$
is a bijection from $Q_{N'}$ to $Q_N$.  Thus, every state of $N'$ is
reachable, and, for all
$w\in(\alphabet\,M')^* = (\alphabet\,N)^* = (\alphabet\,N')^*$,
$h(\delta_{N'}(s_{N'},w))=\delta_N(s_N,w)$.  The remainder of the
proof that $h$ is an isomorphism from $N'$ to $N$ is fairly
straightforward.
\end{proof}

The following exercise formalizes the last sentence of
Lemma~\ref{MinimizationLemma8}.

\begin{exercise}
Suppose $N'$ and $N$ are equivalent DFAs with alphabet $\Sigma$, all
of whose states are reachable (for all $q\in Q_{N'}$, there is a
$w\in\Sigma^*$ such that $\delta_{N'}(s_{N'},w) = q$; and for all
$q\in Q_{N}$, there is a $w\in\Sigma^*$ such that
$\delta_{N}(s_{N},w) = q$). Suppose $h$ is a bijection from $Q_{N'}$
to $Q_N$ such that, for all $w\in\Sigma^*$,
$h(\delta_{N'}(s_{N'},w))=\delta_N(s_N,w)$. Prove that $h$ is
an isomorphism from $N'$ to $N$.
\end{exercise}

We define a function $\minimize\in\DFA\fun\DFA$ by:
$\minimize\,M$ is the result of running the above algorithm on
input $M$.

Putting the above results together, we have the following theorem:
\begin{theorem}
\label{Minimization}
For all $M\in\DFA$:
\begin{itemize}
\item $\minimize\,M\approx M$;

\item $\alphabet(\minimize\,M)=\alphabet(L(M))$;

\item $|Q_{\minimize\,M}| \leq |Q_M|$;

\item $\minimize\,M$ is deterministically simplified; and

\item for all $N\in\DFA$, if $N\approx M$, $\alphabet\,N=\alphabet(L(M))$ and
$|Q_N|\leq|Q_{\minimize\,M}|$, then $N$ is isomorphic to $\minimize\,M$.
\end{itemize}
\end{theorem}

Thus
\begin{center}
\input{chap-3.13-fig4.eepic}
\end{center}
is, up to isomorphism, the only four-or-fewer state DFA with alphabet
$\{\zerosf,\onesf\}$ that is equivalent to $M$.

\begin{exercise}
Use Theorem~\ref{Minimization} to show that, for all DFAs $M$ and $N$,
if $M\approx N$, then $\minimize\,M$ is isomorphic to $\minimize\,N$.
\end{exercise}

The Forlan module DFA includes the function
\begin{verbatim}
val minimize : dfa -> dfa
\end{verbatim}
for minimizing DFAs.

For example, if \texttt{dfa} of type \texttt{dfa} is bound to our example
DFA
\begin{center}
\input{chap-3.13-fig3.eepic}
\end{center}
then we can minimize the alphabet size and number of states of \texttt{dfa}
as follows.
\begin{list}{}
{\setlength{\leftmargin}{\leftmargini}
\setlength{\rightmargin}{0cm}
\setlength{\itemindent}{0cm}
\setlength{\listparindent}{0cm}
\setlength{\itemsep}{0cm}
\setlength{\parsep}{0cm}
\setlength{\labelsep}{0cm}
\setlength{\labelwidth}{0cm}
\catcode`\#=12
\catcode`\$=12
\catcode`\%=12
\catcode`\^=12
\catcode`\_=12
\catcode`\.=12
\catcode`\?=12
\catcode`\!=12
\catcode`\&=12
\ttfamily}
\small
\item[]\textsl{-\ }val\ dfa'\ =\ DFA.minimize\ dfa;
\item[]\textsl{val\ dfa'\ =\ -\ :\ dfa}
\item[]\textsl{-\ }DFA.output("",\ dfa');
\item[]\textsl{\symbol{'173}states\symbol{'175}\ <A>,\ <C>,\ <B,D>,\ <E,F>\ \symbol{'173}start\ state\symbol{'175}\ <A>}
\item[]\textsl{\symbol{'173}accepting\ states\symbol{'175}\ <E,F>}
\item[]\textsl{\symbol{'173}transitions\symbol{'175}}
\item[]\textsl{<A>,\ 0\ ->\ <B,D>;\ <A>,\ 1\ ->\ <C>;\ <C>,\ 0\ ->\ <B,D>;\ <C>,\ 1\ ->\ <B,D>;}
\item[]\textsl{<B,D>,\ 0\ ->\ <B,D>;\ <B,D>,\ 1\ ->\ <E,F>;\ <E,F>,\ 0\ ->\ <E,F>;}
\item[]\textsl{<E,F>,\ 1\ ->\ <E,F>}
\item[]\textsl{val\ it\ =\ ()\ :\ unit}
\end{list}


Finally, let's revisit an example from
Section~\ref{DeterministicFiniteAutomata}. Suppose \texttt{nfa} is the
$4$-state NFA
\begin{center}
\input{chap-3.13-fig5.eepic}
\end{center}
As we saw, our NFA-to-DFA conversion algorithm converts \texttt{nfa}
to a DFA \texttt{dfa} with $16$ states:

\begin{list}{}
{\setlength{\leftmargin}{\leftmargini}
\setlength{\rightmargin}{0cm}
\setlength{\itemindent}{0cm}
\setlength{\listparindent}{0cm}
\setlength{\itemsep}{0cm}
\setlength{\parsep}{0cm}
\setlength{\labelsep}{0cm}
\setlength{\labelwidth}{0cm}
\catcode`\#=12
\catcode`\$=12
\catcode`\%=12
\catcode`\^=12
\catcode`\_=12
\catcode`\.=12
\catcode`\?=12
\catcode`\!=12
\catcode`\&=12
\ttfamily}
\small
\item[]\textsl{-\ }val\ dfa\ =\ nfaToDFA\ nfa;
\item[]\textsl{val\ dfa\ =\ -\ :\ dfa}
\item[]\textsl{-\ }DFA.numStates\ dfa;
\item[]\textsl{val\ it\ =\ 16\ :\ int}
\end{list}


We can now use Forlan to verify that there is no DFA with fewer than
$16$ states that accepts the same language as \texttt{nfa}:

\begin{list}{}
{\setlength{\leftmargin}{\leftmargini}
\setlength{\rightmargin}{0cm}
\setlength{\itemindent}{0cm}
\setlength{\listparindent}{0cm}
\setlength{\itemsep}{0cm}
\setlength{\parsep}{0cm}
\setlength{\labelsep}{0cm}
\setlength{\labelwidth}{0cm}
\catcode`\#=12
\catcode`\$=12
\catcode`\%=12
\catcode`\^=12
\catcode`\_=12
\catcode`\.=12
\catcode`\?=12
\catcode`\!=12
\catcode`\&=12
\ttfamily}
\small
\item[]\textsl{-\ }val\ dfa'\ =\ DFA.minimize\ dfa;
\item[]\textsl{val\ dfa'\ =\ -\ :\ dfa}
\item[]\textsl{-\ }DFA.isomorphic(dfa',\ dfa);
\item[]\textsl{val\ it\ =\ true\ :\ bool}
\item[]\textsl{-\ }DFA.numStates\ dfa';
\item[]\textsl{val\ it\ =\ 16\ :\ int}
\end{list}


Thus we have an example where the smallest DFA accepting a language
requires exponentially more states than the smallest NFA accepting that
language. (This is true even though we haven't proven that an NFA
must have at least $4$ states to accept the same language as \texttt{nfa}.)

\subsection{Notes}

Our algorithm for testing whether two DFAs are equivalent can be found
in the literature, but I don't know of other textbooks that present
it.  As described above, a simple extension of the algorithm provides
counterexamples to justify non-equivalence.  The material on DFA
minimization is completely standard.

%%% Local Variables: 
%%% mode: latex
%%% TeX-master: "book"
%%% End: 
