\section{Checking Acceptance and Finding Accepting Paths}
\label{CheckingAcceptanceAndFindingAcceptingPaths}

In this section we study algorithms for checking whether a string is
\index{finite automaton!checking for string acceptance}%
\index{finite automaton!searching for labeled paths}%
accepted by a finite automaton, and for finding a labeled path that
explains why a string is accepted by a finite automaton.

\subsection{Processing a String from a Set of States}

Suppose $M$ is a finite automaton.  We define a function
$\Delta_M\in\powset\,Q_M\times\Str\fun\powset\,Q_M$ by:
\index{Delta@$\Delta_\cdot$}%
\index{finite automaton!Delta@$\Delta_\cdot$}%
$\Delta_M(P,w)$ is the set of all $r\in Q_M$ such that there is
an $\lp\in\LP$ such that
\begin{itemize}
\item $w$ is the label of $\lp$;

\item $\lp$ is valid for $M$;

\item the start state of $\lp$ is in $P$; and

\item $r$ is the end state of $\lp$.
\end{itemize}
In other words, $\Delta_M(P,w)$ consists of all of the states that can
be reached from elements of $P$ by labeled paths that are labeled by
$w$ and valid for $M$.  When the FA $M$ is clear from the context, we
sometimes abbreviate $\Delta_M$ to $\Delta$.

Suppose $M$ is the finite automaton
\begin{center}
\input{chap-3.6-fig1.eepic}
\end{center}
Then, $\Delta_M(\{\Asf\},\mathsf{12111111}) =
\{\Bsf,\Csf\}$, since
\begin{gather*}
\mathsf{A\lparr{1}A\lparr{2}B\lparr{11}B\lparr{11}B\lparr{11}B}
\quad\eqtxt{and}\quad
\mathsf{A\lparr{1}A\lparr{2}C\lparr{111}C\lparr{111}C}
\end{gather*}
are all of the labeled paths that are labeled by $\mathsf{12111111}$,
valid in $M$ and whose start states are $\Asf$.
Furthermore, $\Delta_M(\{\Asf,\Bsf,\Csf\},\mathsf{11}) =
\{\Asf,\Bsf\}$, since
\begin{gather*}
\mathsf{A\lparr{1}A\lparr{1}A}
\quad\eqtxt{and}\quad
\mathsf{B\lparr{11}B}
\end{gather*}
are all of the labeled paths that are labeled by $\mathsf{11}$ and
valid in $M$.

Suppose $M$ is a finite automaton, $P\sub Q_M$ and $w\in Str$.
We can calculate $\Delta_M(P,w)$ as follows.
\index{Delta@$\Delta_\cdot$}%
\index{finite automaton!Delta@$\Delta_\cdot$}%
\index{finite automaton!calculating Delta@calculating $\Delta_\cdot(\cdot,\cdot)$}%

Let $S$ be the set of all suffixes of $w$.  Given $y\in S$, we write
$\pre\,y$ for the unique $x$ such that $w=xy$.

First, we generate the least subset $X$ of $Q_M\times S$ such that:
\begin{enumerate}[(1)]
\item for all $p\in P$, $(p,w)\in X$; and

\item for all $q,r\in Q_M$ and $x,y\in\Str$, if
$(q,xy)\in X$ and $q,x\fun r\in T_M$, then
$(r,y)\in X$.
\end{enumerate}
We start by using rule (1), adding $(p,w)$ to $X$, whenever $p\in P$.
Then $X$ (and any superset of $X$) will satisfy property (1).
Then, rule (2) is used repeatedly to add more pairs to $X$.  Since
$Q_M\times S$ is a finite set, eventually $X$ will satisfy property (2).

If $M$ is our example finite automaton, then here are the elements of
$X$, when $P=\{\Asf\}$ and $w=\mathsf{2111}$:
\begin{itemize}
\item $(\Asf,\mathsf{2111})$;

\item $(\Bsf,\mathsf{111})$, because of $(\Asf,\mathsf{2111})$ and the
  transition $\Asf,\twosf\fun\Bsf$;

\item $(\Csf, \mathsf{111})$, because of $(\Asf,\mathsf{2111})$ and
  the transition $\Asf,\twosf\fun\Csf$ (now, we're done with
  $(\Asf,\mathsf{2111})$);

\item $(\Bsf,\mathsf{1})$, because of $(\Bsf,\mathsf{111})$ and the
  transition $\Bsf,\mathsf{11}\fun\Bsf$ (now, we're done with
  $(\Bsf,\mathsf{111})$);

\item $(\Csf, \%)$, because of $(\Csf, \mathsf{111})$ and the
  transition $\Csf,\mathsf{111}\fun\Csf$ (now, we're done with $(\Csf,
  \mathsf{111})$); and

\item nothing can be added using $(\Bsf,\mathsf{1})$ and $(\Csf, \%)$,
and so we've found all the elements of $X$.
\end{itemize}

The following lemma explains when pairs show up in $X$.

\begin{lemma}
\label{AcceptLem1}
For all $q\in Q_M$ and $y\in S$,
\begin{gather*}
(q,y)\in X\quad\eqtxt{iff}\quad q\in\Delta_M(P,\pre\,y) .
\end{gather*}
\end{lemma}

\begin{proof}
The ``only if'' (left-to-right) direction is by induction on $X$:
we show that, for all $(q,y)\in X$, $q\in\Delta_M(P,\pre\,y)$.
\begin{itemize}
\item Suppose $p\in P$.  Then $p\in\Delta_M(P,\%)$.  But
$\pre\,w=\%$, so that $p\in\Delta_M(P,\pre\,w)$.

\item Suppose $q,r\in Q_M$, $x,y\in\Str$, $(q,xy)\in X$ and
$(q,x,r)\in T_M$.  Assume the inductive hypothesis:
$q\in\Delta_M(P,\pre(xy))$.
Thus there is an $\lp\in\LP$
such that $\pre(xy)$ is the label of $\lp$, $\lp$ is valid for $M$,
the start state of $\lp$ is in $P$, and $q$ is the end state of $\lp$.
Let $lp'\in\LP$ be the result of adding the step $q,x\Rightarrow r$ at
the end of $\lp$.  Thus $\pre\,y$ is the label of $\lp'$, $\lp'$ is
valid for $M$, the start state of $\lp'$ is in $P$, and $r$ is the end
state of $\lp'$, showing that $r\in\Delta_M(P,\pre\,y)$.
\end{itemize}

For the `if'' (right-to-left) direction, we have that there is a
labeled path
\begin{gather*}
q_1\lparr{x_1}q_2\lparr{x_2}\cdots\,q_{n-1}\lparr{x_{n-1}}q_n ,
\end{gather*}
that is valid for $M$ and where $\pre\,y=x_1x_2\cdots x_{n-1}$,
$q_1\in P$ and $q_n=q$.  Since $q_1\in P$ and
$w=(\pre\,y)y=x_1x_2\cdots x_{n-1}y$, we have that
$(q_1, x_1x_2\cdots x_{n-1}y)=(q_1,w)\in X$, by (1).  But
$(q_1,x_1,q_2)\in T_M$, and thus $(q_2, x_2\cdots x_{n-1}y)\in X$,
by (2). Continuing on in this way (we could do this by mathematical
induction), we finally get that $(q,y)=(q_n,y)\in X$.
\end{proof}

\begin{lemma}
\label{AcceptLem2}
For all $q\in Q_M$, $(q,\%)\in X$ iff $q\in\Delta_M(P,w)$.
\end{lemma}

\begin{proof}
Suppose $(q,\%)\in X$.  Lemma~\ref{AcceptLem1} tells us that
$q\in\Delta_M(P,\pre\,\%)$.  But $\pre\,\%=w$, and thus
$q\in\Delta_M(P,w)$.

Suppose $q\in\Delta_M(P,w)$.  Since $w=\pre\,\%$, we have that
$q\in\Delta_M(P,\pre\,\%)$.  Lemma~\ref{AcceptLem1} tells us that
$(q,\%)\in X$.
\end{proof}

By Lemma~\ref{AcceptLem2}, we have that
\begin{gather*}
\Delta_M(P,w)=\setof{q\in Q_M}{(q,\%)\in X} .
\end{gather*}
Thus, we return the set of all states $q$ that are paired with $\%$ in
$X$.

\subsection{Checking String Acceptance and Finding Accepting Paths}

\begin{proposition}
\label{AcceptLem3}
Suppose $M$ is a finite automaton.  Then
\begin{gather*}
L(M) = \setof{w\in\Str}{\Delta_M(\{s_M\}, w)
\index{L(.)@$L(\cdot)$}%
\index{finite automaton!L(.)@$L(\cdot)$}%
\index{finite automaton!characterizing L(.)@characterizing $L(\cdot)$}%
\mathbin{\cap}A_M\mathrel{\neq}\emptyset} .
\end{gather*}
\end{proposition}

\begin{proof}
Suppose $w\in L(M)$.  Then $w$ is the label of a labeled path $\lp$
such that $\lp$ is valid in $M$, the start state of $\lp$ is $s_M$ and
the end state of $\lp$ is in $A_M$.  Let $q$ be the end state of
$\lp$.  Thus $q\in\Delta_M(\{s_M\}, w)$ and $q\in A_M$, showing that
$\Delta_M(\{s_M\}, w)\cap A_M\neq\emptyset$.

Suppose $\Delta_M(\{s_M\}, w)\cap A_M\neq\emptyset$, so that
there is a $q$ such that $q\in\Delta_M(\{s_M\}, w)$ and
$q\in A_M$.  Thus $w$ is the label of a labeled path $\lp$ such that
$\lp$ is valid in $M$, the start state of $\lp$ is $s_M$,
and the end state of $\lp$ is $q\in A_M$.  Thus $w\in L(M)$.
\end{proof}

According to Proposition~\ref{AcceptLem3}, to check if a string $w$ is
accepted by a finite automaton $M$, we simply 
use our algorithm to generate $\Delta_M(\{s_M\}, w)$,
and then check if this set contains at least one accepting state.

Given a finite automaton $M$, subsets $P,R$ of $Q_M$ and a string $w$,
\index{finite automaton!searching for labeled paths}%
how do we search for a labeled path that is labeled by $w$, valid
in $M$, starts from an element of $P$, and ends with an element of $R$?
What we need to do is associate with each pair
\begin{gather*}
(q,y)
\end{gather*}
of the set $X$ that we generate when computing $\Delta_M(P,w)$ a
labeled path $\lp$ such that $\lp$ is labeled by $\pre(y)$, $\lp$ is
valid in $M$, the start state of $\lp$ is an element of $P$, and the
end state of $\lp$ is $q$.  If we process the elements of $X$ in a
breadth-first (rather than depth-first) manner, this will ensure that
these labeled paths are as short as possible.  As we generate the
elements of $X$, we look for a pair of the form $(q,\%)$, where $q\in
R$.  Our answer will then be the labeled path associated with this
pair.

The Forlan module \texttt{FA}
\index{FA@\texttt{FA}}%
also contains the following functions
for processing strings and checking string acceptance:
\begin{verbatim}
val processStr          : fa -> sym set * str -> sym set
val accepted            : fa -> str -> bool
\end{verbatim}
\index{FA@\texttt{FA}!processStr@\texttt{processStr}}%
\index{FA@\texttt{FA}!accepted@\texttt{accepted}}%
The function \texttt{processStr} takes in a finite automaton $M$,
and returns a function that takes in a pair $(P, w)$ and returns
$\Delta_M(P, w)$.
And, the function \texttt{accepted} takes in a finite automaton $M$,
and returns a function that checks whether a string $x$ is
accepted by $M$.

The Forlan module \texttt{FA}
\index{FA@\texttt{FA}}%
also contains the following functions for finding labeled paths:
\begin{verbatim}
val findLP          : fa -> sym set * str * sym set -> lp
val findAcceptingLP : fa -> str -> lp
\end{verbatim}
\index{FA@\texttt{FA}!findLP@\texttt{findLP}}%
\index{FA@\texttt{FA}!findAcceptingLP@\texttt{findAcceptingLP}}%
The function \texttt{findLP} takes in a finite automaton $M$, and
returns a function that takes in a triple $(P,w,R)$ and tries
to find a labeled path $\lp$ that is labeled by $w$, valid for $M$,
starts out with an element of $P$, and ends up at an element of $R$.
It issues an error message when there is no such labeled path.
The function \texttt{findAcceptingLP} takes in a finite automaton $M$,
and returns a function that looks for a labeled path $\lp$ that
explains why a string $w$ is accepted by $M$.  It issues an error
message when there is no such labeled path.  The labeled paths
returned by these functions are always of minimal length.

Suppose \texttt{fa} is the finite automaton
\begin{center}
\input{chap-3.6-fig1.eepic}
\end{center}
We begin by applying our five functions to \texttt{fa}, and giving names
to the resulting functions:
\begin{list}{}
{\setlength{\leftmargin}{\leftmargini}
\setlength{\rightmargin}{0cm}
\setlength{\itemindent}{0cm}
\setlength{\listparindent}{0cm}
\setlength{\itemsep}{0cm}
\setlength{\parsep}{0cm}
\setlength{\labelsep}{0cm}
\setlength{\labelwidth}{0cm}
\catcode`\#=12
\catcode`\$=12
\catcode`\%=12
\catcode`\^=12
\catcode`\_=12
\catcode`\.=12
\catcode`\?=12
\catcode`\!=12
\catcode`\&=12
\ttfamily}
\small
\item[]\textsl{-\ }val\ processStr\ =\ FA.processStr\ fa;
\item[]\textsl{val\ processStr\ =\ fn\ :\ sym\ set\ \symbol{'052}\ str\ ->\ sym\ set}
\item[]\textsl{-\ }val\ accepted\ =\ FA.accepted\ fa;
\item[]\textsl{val\ accepted\ =\ fn\ :\ str\ ->\ bool}
\item[]\textsl{-\ }val\ findLP\ =\ FA.findLP\ fa;
\item[]\textsl{val\ findLP\ =\ fn\ :\ sym\ set\ \symbol{'052}\ str\ \symbol{'052}\ sym\ set\ ->\ lp}
\item[]\textsl{-\ }val\ findAcceptingLP\ =\ FA.findAcceptingLP\ fa;
\item[]\textsl{val\ findAcceptingLP\ =\ fn\ :\ str\ ->\ lp}
\end{list}

Next, we'll define a set of states and a string to use later:
\begin{list}{}
{\setlength{\leftmargin}{\leftmargini}
\setlength{\rightmargin}{0cm}
\setlength{\itemindent}{0cm}
\setlength{\listparindent}{0cm}
\setlength{\itemsep}{0cm}
\setlength{\parsep}{0cm}
\setlength{\labelsep}{0cm}
\setlength{\labelwidth}{0cm}
\catcode`\#=12
\catcode`\$=12
\catcode`\%=12
\catcode`\^=12
\catcode`\_=12
\catcode`\.=12
\catcode`\?=12
\catcode`\!=12
\catcode`\&=12
\ttfamily}
\small
\item[]\textsl{-\ }val\ bs\ =\ SymSet.input\ "";
\item[]\textsl{@\ }A,\ B,\ C
\item[]\textsl{@\ }.
\item[]\textsl{val\ bs\ =\ -\ :\ sym\ set}
\item[]\textsl{-\ }val\ x\ =\ Str.input\ "";
\item[]\textsl{@\ }11
\item[]\textsl{@\ }.
\item[]\textsl{val\ x\ =\ \symbol{'133}-,-\symbol{'135}\ :\ str}
\end{list}

Here are some example uses of our functions:
\begin{list}{}
{\setlength{\leftmargin}{\leftmargini}
\setlength{\rightmargin}{0cm}
\setlength{\itemindent}{0cm}
\setlength{\listparindent}{0cm}
\setlength{\itemsep}{0cm}
\setlength{\parsep}{0cm}
\setlength{\labelsep}{0cm}
\setlength{\labelwidth}{0cm}
\catcode`\#=12
\catcode`\$=12
\catcode`\%=12
\catcode`\^=12
\catcode`\_=12
\catcode`\.=12
\catcode`\?=12
\catcode`\!=12
\catcode`\&=12
\ttfamily}
\small
\item[]\textsl{-\ }SymSet.output("",\ processStr(bs,\ x));
\item[]\textsl{A,\ B}
\item[]\textsl{val\ it\ =\ ()\ :\ unit}
\item[]\textsl{-\ }accepted(Str.input\ "");
\item[]\textsl{@\ }12111111
\item[]\textsl{@\ }.
\item[]\textsl{val\ it\ =\ true\ :\ bool}
\item[]\textsl{-\ }accepted(Str.input\ "");
\item[]\textsl{@\ }1211
\item[]\textsl{@\ }.
\item[]\textsl{val\ it\ =\ false\ :\ bool}
\item[]\textsl{-\ }LP.output("",\ findLP(bs,\ x,\ bs));
\item[]\textsl{B,\ 11\ =>\ B}
\item[]\textsl{val\ it\ =\ ()\ :\ unit}
\item[]\textsl{-\ }LP.output("",\ findAcceptingLP(Str.input\ ""));
\item[]\textsl{@\ }12111111
\item[]\textsl{@\ }.
\item[]\textsl{A,\ 1\ =>\ A,\ 2\ =>\ C,\ 111\ =>\ C,\ 111\ =>\ C}
\item[]\textsl{val\ it\ =\ ()\ :\ unit}
\item[]\textsl{-\ }LP.output("",\ findAcceptingLP(Str.input\ ""));
\item[]\textsl{@\ }222
\item[]\textsl{@\ }.
\item[]\textsl{no\ such\ labeled\ path\ exists}
\item[]
\item[]\textsl{uncaught\ exception\ Error}
\end{list}


\subsection{Notes}

The material in this section is original.  Our definition of the
meaning of FAs via labeled paths allows us not simply to test \emph{whether}
an FA accepts a string $w$, but to ask for evidence---in the form of a
labeled path---for \emph{why} FA accepts $w$.

%%% Local Variables: 
%%% mode: latex
%%% TeX-master: "book"
%%% End: 
