\section{Proving the Correctness of Grammars}
\label{ProvingTheCorrectnessOfGrammars}

In this section, we consider techniques for proving the correctness of
grammars, i.e., for proving that grammars generate the languages we want
them to.

\subsection{Preliminaries}

Suppose $G$ is a grammar and $a\in Q_G\cup\alphabet\,G$.  Then
\begin{displaymath}
\Pi_{G,a}=\setof{w\in(\alphabet\,G)^*}{w \eqtxt{is parsable from} a
  \eqtxt{using} G} .  
\end{displaymath}
I.e., $\Pi_{G,a}$ is all strings over $\alphabet\,G$ that are the
yields of valid parse trees for $G$ whose root labels are $a$ If it's
clear which grammar we are talking about, we often abbreviate
$\Pi_{G,a}$ to $\Pi_a$.  Clearly, $\Pi_{G,s_G} = L(G)$.

For example, if $G$ is the grammar
\begin{gather*}
  \Asf \fun \% \mid \zerosf\Asf\onesf
\end{gather*}
then $\Pi_\zerosf=\{\zerosf\}$, $\Pi_\onesf=\{\onesf\}$ and $\Pi_\Asf=
\setof{\zerosf^n\onesf^n}{n\in\nats}= L(G)$.

\begin{proposition}
\label{PiProp}
Suppose $G$ is a grammar.
\begin{enumerate}[\quad(1)]
\item For all $a\in\alphabet\,G$, ${a}\in\Pi_{G,a}$.

\item For all $q\in Q_G$, if $q\fun\%\in P_G$, then
  ${\%}\in\Pi_{G,q}$.

\item For all $q\in Q_G$, $n\in\nats-\{0\}$, $a_1,\ldots,a_n\in\Sym$
  and $w_1,\ldots,w_n\in\Str$, if $q\fun a_1\cdots a_n\in P_G$ and
  $w_1\in\Pi_{G,a_1}$, \ldots, $w_n\in\Pi_{G,a_n}$, then
  ${w_1\cdots w_n}\in\Pi_{G,q}$.
\end{enumerate}
\end{proposition}

Our main example will be the grammar $G$:
\begin{align*}
\Asf &\fun \% \mid \zerosf\Bsf\Asf \mid \onesf\Csf\Asf , \\
\Bsf &\fun \onesf \mid \zerosf\Bsf\Bsf , \\
\Csf &\fun \zerosf \mid \onesf\Csf\Csf .
\end{align*}
Define $\diff\in\{\mathsf{0,1}\}^*\fun\ints$ by:
for all $w\in\{\mathsf{0,1}\}^*$,
\begin{gather*}
\diff\,w =
\eqtxtr{the number of $\mathsf{1}$'s in}w -
\eqtxtr{the number of $\mathsf{0}$'s in}w .
\end{gather*}
Then: $\diff\,\% = 0$, $\diff\,\mathsf{1} = 1$, $\diff\,\mathsf{0} =
-1$, and, for all $x,y\in\{\mathsf{0,1}\}^*$, $\diff(xy) = \diff\,x +
\diff\,y$.
Let
\begin{align*}
  X &= \setof{w\in\{\mathsf{0,1}\}^*}{\diff\,w = 0} , \\
  Y &= \setof{w\in\{\mathsf{0,1}\}^*}{\diff\,w = 1\eqtxt{and,}\\
    &\quad\hspace{.25cm}
    \eqtxt{for all proper prefixes}v\eqtxt{of}w,\diff\,v\leq 0} , \eqtxt{and}\\
  Z &= \setof{w\in\{\mathsf{0,1}\}^*}{\diff\,w = -1\eqtxt{and,}\\
    &\quad\hspace{.25cm} \eqtxt{for all proper
      prefixes}v\eqtxt{of}w,\diff\,v\geq 0} .
\end{align*}
We will prove that $L(G) = \Pi_{G,\Asf} = X$, $\Pi_{G,\Bsf} = Y$ and
$\Pi_{G,\Csf} = Z$.

\begin{lemma}
\label{GramCorrLem1}
Suppose $x\in\{\mathsf{0,1}\}^*$.
\begin{enumerate}[\quad(1)]
\item If $\diff\,x\geq 1$, then $x=yz$ for some $y,z\in\{\mathsf{0,1}\}^*$
  such that $y\in Y$ and $\diff\,z = \diff\,x - 1$.

\item If $\diff\,x\leq -1$, then $x=yz$ for some $y,z\in\{\mathsf{0,1}\}^*$
  such that $y\in Z$ and $\diff\,z = \diff\,x + 1$.
\end{enumerate}
\end{lemma}

\begin{proof}
We show the proof of (1), the proof of (2) being similar.

Let $y\in\{\mathsf{0,1}\}^*$ be the shortest prefix of $x$ such
that $\diff\,y\geq 1$, and let $z\in\{\mathsf{0,1}\}^*$ be such
that $x=yz$.

Because $\diff\,y\geq 1$, we have that $y\neq\%$.  Thus $y=y'a$
for some $y'\in\{\mathsf{0,1}\}^*$ and $a\in\{\mathsf{0,1}\}$.
By the definition of $y$, we have that $\diff\,y'\leq 0$.
Suppose, toward a contradiction, that $a=\zerosf$.  Since
$\diff\,y' + -1=\diff\,y\geq 1$, we have that $\diff\,y'\geq 2$,
contradicting the definition of $y$.  Thus $a=\onesf$, so
that $y=y'\onesf$.

Because $\diff\,y' + 1 = \diff\,y\geq 1$, we have that $\diff\,y'\geq
0$.  Thus $\diff\,y' = 0$, so that $\diff y = \diff\,y' + 1 = 1$.  By
the definition of $y$, every prefix of $y'$ has a diff that is
$\leq$ $0$.  Thus $y\in Y$.

Finally, because $\diff\,x = \diff\,y + \diff\,z = 1 + \diff\,z$ we have
that $\diff\,z = \diff\,x - 1$.
\end{proof}

\subsection{Proving that Enough is Generated}

First we study techniques for showing that everything we want a
grammar to generate is really generated.

Since $X,Y,Z\sub\{\mathsf{0,1}\}^*$, to prove that
$X\sub\Pi_{G,\Asf}$, $Y\sub\Pi_{G,\Bsf}$ and $Z\sub\Pi_{G,\Csf}$, it
will suffice to use strong string induction to show that, for all
$w\in\mathsf{\{0,1\}^*}$:

\begin{enumerate}[\quad(A)]
\item if $w\in X$, then $w\in\Pi_{G,\Asf}$;

\item if $w\in Y$, then $w\in\Pi_{G,\Bsf}$; and

\item if $w\in Z$, then $w\in\Pi_{G,\Csf}$.
\end{enumerate}

We proceed by strong string induction.  Suppose
$w\in\{\mathsf{0,1}\}^*$, and assume the inductive hypothesis:
for all $x\in\{\mathsf{0,1}\}^*$, if $x$ is a proper substring of
$w$, then:
\begin{enumerate}[\quad(A)]
\item if $x\in X$, then $x\in\Pi_\Asf$;

\item if $x\in Y$, then $x\in\Pi_\Bsf$; and

\item if $x\in Z$, then $x\in\Pi_\Csf$.
\end{enumerate}
We must prove that:
\begin{enumerate}[\quad(A)]
\item if $w\in X$, then $w\in\Pi_\Asf$;

\item if $w\in Y$, then $w\in\Pi_\Bsf$; and

\item if $w\in Z$, then $w\in\Pi_\Csf$.
\end{enumerate}

\begin{enumerate}[\quad(A)]
\item Suppose $w\in X$.  We must show that $w\in\Pi_\Asf$.  There are
  three cases to consider.
  \begin{itemize}
  \item Suppose $w=\%$.  Because $\Asf\fun\%\in P$,
    Proposition~\ref{PiProp}(2) tells us that $w=\%\in\Pi_\Asf$.
  
  \item Suppose $w=\zerosf x$, for some $x\in\{\mathsf{0,1}\}^*$.
    Because $-1 + \diff\,x = \diff\,w = 0$, we have that $\diff\,x =
    1$.  Thus, by Lemma~\ref{GramCorrLem1}(1), we have that $x=yz$,
    for some $y,z\in\{\mathsf{0,1}\}^*$ such that $y\in Y$ and
    $\diff\,z = \diff\,x - 1 = 1 - 1 = 0$.  Thus $w=\zerosf yz$, $y\in
    Y$ and $z\in X$.  By Proposition~\ref{PiProp}(1), we have
    $\zerosf\in\Pi_\zerosf$.  Because $y\in Y$ and $z\in X$ are proper
    substrings of $w$, parts~(B) and (A) of the inductive hypothesis
    tell us that $y\in\Pi_\Bsf$ and $z\in\Pi_\Asf$.  Thus, because
    $\Asf\fun\zerosf\Bsf\Asf\in P$, Proposition~\ref{PiProp}(3) tells
    us that $w=\zerosf yz\in\Pi_\Asf$.
  
  \item Suppose $w=\onesf x$, for some $x\in\{\mathsf{0,1}\}^*$.  The
    proof is analogous to the preceding case.
  \end{itemize}

\item Suppose $w\in Y$.  We must show that $w\in\Pi_\Bsf$.  Because
  $\diff\,w=1$, there are two cases to consider.
  \begin{itemize}
  \item Suppose $w=\onesf x$, for some $x\in\{\mathsf{0,1}\}^*$.
    Because all proper prefixes of $w$ have diffs $\leq$ $0$, we
    have that $x=\%$, so that $w=\onesf$.  Since $\Bsf\fun\onesf\in P$,
    we have that $w=\onesf\in\Pi_\Bsf$.
  
  \item Suppose $w=\zerosf x$, for some $x\in\{\mathsf{0,1}\}^*$.
    Thus $\diff\,x = 2$.  Because $\diff\,x\geq 1$, by
    Lemma~\ref{GramCorrLem1}(1), we have that $x=yz$, for some
    $y,z\in\{\mathsf{0,1}\}^*$ such that $y\in Y$ and $\diff\,z =
    \diff\,x - 1 = 2 - 1 = 1$.  Hence $w=\zerosf yz$.  To finish the
    proof that $z\in Y$, suppose $v$ is a proper prefix of $z$.
    Thus $\zerosf yv$ is a proper prefix of $w$.  Since $w\in Y$, it
    follows that $\diff\,v = \diff(\zerosf yv)\leq 0$, as required.
    Since $y,z\in Y$, part~(B) of the inductive hypothesis tell us
    that $y,z\in\Pi_\Bsf$.  Thus, because $\Bsf\fun\zerosf\Bsf\Bsf\in
    P$ we have that $w=\zerosf yz\in\Pi_\Bsf$.
  \end{itemize}

\item Suppose $w\in Z$.  We must show that $w\in\Pi_\Csf$.  The
  proof is analogous to the proof of part~(B).
\end{enumerate}

Suppose $H$ is the grammar
\begin{gather*}
\Asf \fun \Bsf \mid \zerosf\Asf\threesf , \qquad
\Bsf \fun \% \mid \onesf\Bsf\twosf ,
\end{gather*}
and let
\begin{gather*}
X = \setof{\zerosf^n\onesf^m\twosf^m\threesf^n}{n,m\in\nats} , \qquad
Y = \setof{\onesf^m\twosf^m}{m\in\nats} .
\end{gather*}
We can prove that $X\sub \Pi_{H,\Asf} = L(H)$ and $Y\sub \Pi_{H,\Bsf}$
using the above technique, but the production $\Asf\fun\Bsf$, which is
called a \emph{unit production} because its right side is a single
variable, makes part~(A) tricky.  In
Section~\ref{ProvingTheCorrectnessOfFiniteAutomata}, when considering
techniques for showing the correctness of finite automata, we ran into
a similar problem having to do with $\%$-transitions.

If $w=\zerosf^0\onesf^m\twosf^m\threesf^0 = \onesf^m\twosf^m\in Y$, we
would like to use part~(B) of the inductive hypothesis to conclude
$w\in\Pi_\Bsf$, and then use the fact that $\Asf\fun\Bsf\in P$ to
conclude that $w\in\Pi_\Asf$.  But $w$ is not a proper substring of
itself, and so the inductive hypothesis in not applicable.  Instead,
we must split into cases $m=0$ and $m\geq 1$, using $\Asf\fun\Bsf$ and
$\Bsf\fun\%$, in the first case, and $\Asf\fun\Bsf$ and
$\Bsf\fun\onesf\Bsf\twosf$, as well as the inductive hypothesis on
$\onesf^{m-1}\twosf^{m-1}\in Y$, in the second case.

Because there are no productions from $\Bsf$ back to $\Asf$, we could
also first use strong string induction to prove that, for all
$w\in\mathsf{\{0,1\}^*}$,
\begin{enumerate}[\quad(A)]
\setcounter{enumi}{1}
\item if $w\in Y$, then $w\in\Pi_\Bsf$,
\end{enumerate}
and then use the result of this induction along with
strong string induction to prove that
for all $w\in\mathsf{\{0,1\}^*}$,
\begin{enumerate}[\quad(A)]
\item if $w\in X$, then $w\in\Pi_\Asf$.
\end{enumerate}
This works whenever two parts of a grammar are not mutually recursive.

With this grammar, we could also first use mathematical induction to
prove that, for all $m\in\nats$, $\onesf^m\twosf^m\in\Pi_\Bsf$, and
then use the result of this induction to prove, by mathematical
induction on $n$, that for all $n,m\in\nats$,
$\zerosf^n\onesf^m\twosf^m\threesf^n\in\Pi_\Asf$.

Note that $\%$-productions, i.e., productions of the form $q\fun\%$,
can cause similar problems to those caused by unit productions.  E.g.,
if we have the productions
\begin{displaymath}
  \Asf\fun \Bsf\Csf \quad\eqtxt{and}\quad \Bsf\fun\% ,
\end{displaymath}
then $\Asf\fun\Bsf\Csf$ behaves like a unit production.

\subsection{Proving that Everything Generated is Wanted}

When proving that everything generated by a grammar is wanted, we
could sometimes use strong induction, simply reversing the
implications used when proving that enough is generated.  But this
approach fails for grammars with unit productions, where we would have
to resort to an induction on parse trees.  

In Section~\ref{ProvingTheCorrectnessOfFiniteAutomata}, when
considering techniques for showing the correctness of finite automata,
we ran into a similar problem having to do with $\%$-transitions, and
this led to our introducing the Principle of Induction on $\Lambda$.
Here, we introduce an induction principle called Induction on $\Pi$.

\begin{theorem}[Principle of Induction on $\Pi$]
Suppose $G$ is a grammar, $P_q(w)$ is a property of a
string $w\in\Pi_{G,q}$, for all $q\in Q_G$, and
$P_a(w)$, for $a\in\alphabet\,G$, says ``$w=a$''.
If
\begin{enumerate}[\quad(1)]
\item for all $q\in Q_G$, if $q\fun\%\in P_G$, then $P_q(\%)$, and

\item for all $q\in Q_G$, $n\in\nats-\{0\}$, $a_1,\ldots,a_n\in
  Q_G\cup\alphabet\,G$, and $w_1\in\Pi_{G,a_1}$, \ldots,
  $w_n\in\Pi_{G,a_n}$, if $q\fun a_1\cdots a_n\in P_G$ and (\dag)
  $P_{a_1}(w_1)$, \ldots, $P_{a_n}(w_n)$, then $P_q(w_1\cdots w_n)$,
\end{enumerate}
then
\begin{gather*}
  \eqtxtr{for all}q\in Q_G, \eqtxt{for all}w\in\Pi_{G,q},\,P_q(w).
\end{gather*}
\end{theorem}
We refer to (\dag) as the inductive hypothesis.

\begin{proof}
It suffices to show that, for all $\pt\in\PT$, for all 
$q\in Q_G$ and $w\in(\alphabet\,G)^*$, if
$\pt$ is valid for $G$, $\rootLabel\,\pt = q$ and
$\yield\,\pt = w$, then $P_q(w)$.
We prove this using the principle of induction on parse trees.
\end{proof}

When proving part~(2), we can make use of the fact that, for
$a_i\in\alphabet\,G$, $\Pi_{a_i} = \{a_i\}$, so that $w_i\in\Pi_{a_i}$
will be $a_i$.  Hence it will be unnecessary to assume that
$P_{a_i}(a_i)$, since this says ``$a_i=a_i$'', and so is always true.

Consider, again, our main example grammar $G$:
\begin{align*}
\Asf &\fun \% \mid \zerosf\Bsf\Asf \mid \onesf\Csf\Asf , \\
\Bsf &\fun \onesf \mid \zerosf\Bsf\Bsf , \\
\Csf &\fun \zerosf \mid \onesf\Csf\Csf .
\end{align*}
Let
\begin{align*}
X &= \setof{w\in\{\mathsf{0,1}\}^*}{\diff\,w = 0} , \\
Y &= \setof{w\in\{\mathsf{0,1}\}^*}{\diff\,w = 1\eqtxt{and,}\\
&\quad\hspace{.25cm}
\eqtxt{for all proper prefixes}v\eqtxt{of}w,\diff\,v\leq 0} , \\
Z &= \setof{w\in\{\mathsf{0,1}\}^*}{\diff\,w = -1\eqtxt{and,}\\
&\quad\hspace{.25cm}
\eqtxt{for all proper prefixes}v\eqtxt{of}w,\diff\,v\geq 0} .
\end{align*}

We have already proven that $X\sub\Pi_\Asf=L(G)$, $Y\sub\Pi_\Bsf$ and
$Z\sub\Pi_\Csf$.  To complete the proof that
$L(G)=\Pi_\Asf=X$, $\Pi_\Bsf=Y$ and $\Pi_\Csf=Z$, we will use
induction on $\Pi$ to prove that
$\Pi_\Asf\sub X$, $\Pi_\Bsf\sub Y$ and $\Pi_\Csf\sub Z$.

We use induction on $\Pi$ to show that:
\begin{enumerate}[\quad(A)]
\item for all $w\in\Pi_\Asf$, $w\in X$;

\item for all $w\in\Pi_\Bsf$, $w\in Y$; and

\item for all $w\in\Pi_\Csf$, $w\in Z$.
\end{enumerate}
Formally, this means that we let the properties $P_\Asf(w)$,
$P_\Bsf(w)$ and $P_\Csf(w)$ be ``$w\in X$'', ``$w\in Y$'' and ``$w\in
Z$'', respectively, and then use the induction principle to prove
that, for all $q\in Q_G$, for all $w\in\Pi_q$,
$P_q(w)$.  But we will actually work more informally.

There are seven productions to consider.
\begin{description}
\item[\quad($\Asf\fun\%$)] We must show that $\%\in X$ (as
  ``$w\in X$'' is the property of part~(A)).  And this holds since
  $\diff\,\% = 0$.

\item[\quad($\Asf\fun\zerosf\Bsf\Asf$)] Suppose $w_1\in\Pi_\Bsf$ and
  $w_2\in\Pi_\Asf$ (as $\zerosf\Bsf\Asf$ is the right-side of the
  production, and $\zerosf$ is in $G$'s alphabet),
  and assume the inductive hypothesis,
  $w_1\in Y$ (as this is the property of part~(B)) and
  $w_2\in X$ (as this is the property of part~(A)).  We
  must show that $\zerosf w_1w_2\in X$, as the production
  shows that $\zerosf w_1w_2\in\Pi_\Asf$.  Because
  $w_1\in Y$ and $w_2\in X$, we have that $\diff\,w_1=1$ and
  $\diff\,w_2=0$.  Thus $\diff(\zerosf w_1w_2) = -1 + 1 + 0 = 0$, showing
  that $\zerosf w_1w_2\in X$.
\end{description}

\begin{description}
\item[\quad($\Bsf\fun\zerosf\Bsf\Bsf$)] Suppose $w_1,w_2\in\Pi_\Bsf$,
  and assume the inductive hypothesis, $w_1,w_2\in Y$.  We must show
  that $\zerosf w_1w_2\in Y$.  Clearly, $\diff(\zerosf w_1w_2) =
  -1 + 1 + 1 = 1$.  So, suppose $v$ is a proper prefix of
  $\zerosf w_1w_2$.  We must show that $\diff\,v\leq 0$.
  There are three cases to consider.
  \begin{itemize}
  \item Suppose $v = \%$.  Then $\diff\,v = 0\leq 0$.

  \item Suppose $v=\zerosf u$, for a proper prefix $u$ of $w_1$.
    Because $w_1\in Y$, we have that $\diff\,u\leq 0$.  Thus
    $\diff\,v = -1 + \diff\,u \leq -1 + 0\leq 0$.

  \item Suppose $v=\zerosf w_1u$, for a proper prefix $u$ of $w_2$.
    Because $w_2\in Y$, we have that $\diff\,u\leq 0$.
    Thus $\diff\,v = -1 + 1 + \diff\,u = \diff\,u\leq 0$.
  \end{itemize}

\item The remaining productions are handled similarly.
\end{description}

\subsection{Notes}

Books on formal language theory typically give short shrift to the
proof of correctness of grammars, carrying out one or two
correctness proofs using induction on the length of strings.  In
contrast, we have introduced and applied elegant techniques for
proving the correctness of grammars.  Of particular note is our principle
of induction on $\Pi$.

%%% Local Variables: 
%%% mode: latex
%%% TeX-master: "book"
%%% End: 
