\section{Simplification of Regular Expressions}
\label{SimplificationOfRegularExpressions}

\index{regular expression!simplification|(}%
\index{simplification!regular expression|(}%
In this section, we give three algorithms---of increasing power, but
decreasing efficiency---for regular expression simplification.  The
first algorithm---weak simplification---is defined via a
straightforward structural recursion, and is sufficient for many
purposes.  The remaining two algorithms---local simplification and
global simplification---are based on a set of simplification rules
that is still incomplete and evolving.

\subsection{Regular Expression Complexity}

To begin with, let's consider how we might measure the
complexity/simplicity of regular expressions.  The most obvious
criterion is size (remember that regular expressions are trees).
But consider this pair of equivalent regular expressions:
\begin{align*}
\alpha &= \mathsf{(00^*11^*)^*} , \eqtxt{and} \\
\beta &= \mathsf{\% + 0(0 + 11^*0)^*11^*} .
\end{align*}
Although the size of $\beta$ ($18$) is strictly greater than the
size of $\alpha$ ($10$), $\beta$ has only one closure inside another
closure, whereas $\alpha$ has two closures inside its outer closure,
and thus there is a sense in which $\beta$ is easier to understand
than $\alpha$.

The standard measure of the closure-related complexity
of a regular expression is its \emph{star-height}: the maximum number
$n\in\nats$ such that there is a path from the root of the regular expression
to one of its leaves that passes through $n$ closures.  But $\alpha$ and
$\beta$ both have star-heights of $2$.  Furthermore, star-height
isn't respected by the ways of forming regular expressions.  E.g.,
if $\gamma_1$ has strictly smaller star-height than $\gamma_2$,
we can't conclude that $\gamma_1\gamma'$ has strictly smaller star-height
than $\gamma_2\gamma'$, as the star height of $\gamma'$ may be
greater than the star-height of $\gamma_2$.

So, we need a better measure of the closure-related complexity of
regular expressions than star-height.  Toward that end, let's define a
\emph{closure complexity} to be a nonempty list $\ns$ of natural
\index{closure complexity}%
\index{regular expression!closure complexity}%
numbers that is (not-necessarily strictly) descending: for all
$i\in[1:|\ns|-1]$, $\ns\,i\geq \ns(i+1)$.  This is a way of
representing nonempty multisets of natural numbers that makes it easy
to define the usual ordering on multisets. We write $\CC$ for the set
\index{regular expression!CC@$\CC$}%
of all closure complexities. E.g., $[3, 2, 2, 1]$ is a closure
complexity, but $[3, 2, 3]$ and $[\,]$ are not.  For all $n\in\nats$,
$[n]$ is a \emph{singleton} closure complexity.  The \emph{union} of
closure complexities $\ns$ and $\ms$ ($\ns\cup\ms$) is the closure
complexity that results from putting $\ns\myconcat\ms$ in descending
order, keeping any duplicate elements.  (Here we are overloading the
term union and the operation $\cup$, but we will never treat closure
complexities as sets, so no confusion should result.)  E.g.,
$[3,2,2,1]\cup[4,2,1,0] = [4,3,2,2,2,1,1,0]$.  The \emph{successor}
$\overline{\ns}$ of a closure complexity $\ns$ is the closure
complexity formed by adding one to each element of $\ns$, maintaining
the order of the elements.  E.g., $\overline{[3,2,2,1]} = [4,3,3,2]$.

It is easy to see that $\cup$ is commutative and associative on $\CC$,
and that the successor operation on $\CC$ preserves union:

\begin{proposition}
\begin{enumerate}[\quad(1)]
\item For all $\ns,\ms\in\CC$, $\ns\cup\ms = \ms\cup\ns$.

\item For all $\ns,\ms,\ls\in\CC$, $(\ns\cup\ms)\cup\ls = \ns\cup(\ms\cup\ls)$.

\item For all $\ns,\ms\in\CC$, $\overline{\ns\cup\ms} =
  \overline{\ns}\cup\overline{\ms}$.
\end{enumerate}
\end{proposition}

\begin{proposition}
\begin{enumerate}[\quad(1)]
\label{CCEquivContext}
\item For all $\ns,\ms\in\CC$, $\overline{\ns} = \overline{\ms}$ iff
  $\ns=\ms$.

\item For all $\ns,\ms,\ls\in\CC$, $\ns\cup\ls = \ms\cup\ls$ iff
  $\ns = \ms$.
\end{enumerate}
\end{proposition}

We define a relation $\ltcc$ on $\CC$ by: for all $\ns,\ms\in\CC$,
$\ns\ltcc\ms$ iff either:
\begin{itemize}
\item $\ms = \ns \myconcat \ls$ for some $\ls\in\CC$; or

\item there is an $i\in\nats - \{0\}$ such that
  \begin{itemize}
  \item $i\leq|\ns|$ and $i\leq|\ms|$,

  \item for all $j\in[1:i-1]$, $\ns\,j = \ms\,j$, and

  \item $\ns\,i < \ms\,i$.
  \end{itemize}
\end{itemize}
In other words, $\ns\ltcc\ms$ iff either $\ms$ consists of the
result of appending a nonempty list at the end of $\ns$, or
$\ns$ and $\ms$ agree up to some point, at which $\ns$'s value
is strictly smaller than $\ms$'s value.  E.g.,
$[2, 2]\ltcc[2, 2, 1]$ and $[2,1,1,0,0]\ltcc[2, 2, 1]$.
We define the relation $\leqcc$ on $\CC$ by: for all $\ns,\ms\in\CC$,
$\ns\leqcc\ms$ iff $\ns\ltcc\ms$ or $\ns=\ms$.

\begin{proposition}
\label{CCLTContext}
\begin{enumerate}[\quad(1)]
\item For all $\ns,\ms\in\CC$, $\overline{\ns} \ltcc \overline{\ms}$ iff
  $\ns\ltcc\ms$.

\item For all $\ns,\ms,\ls\in\CC$, $\ns\cup\ls \ltcc \ms\cup\ls$ iff
  $\ns \ltcc \ms$.

\item For all $\ns,\ms\in\CC$, $\ns\ltcc\ns\cup\ms$.
\end{enumerate}
\end{proposition}

\begin{proposition}
$\ltcc$ is a strict total ordering on $\CC$.
\end{proposition}

\begin{proposition}
\label{LTCCWellFounded}
$\ltcc$ is a well-founded relation on $\CC$.
\end{proposition}

Now we can define the closure complexity of a regular expression.
Define the function $\cc\in\Reg\fun\CC$ by structural recursion:
\begin{align*}
\cc\,\% &= [0]; \\
\cc\,\$ &= [0]; \\
\cc\,a &= [0], \eqtxt{for all}a\in\Sym; \\
\cc({*}(\alpha)) &= \overline{\cc\,\alpha}, \eqtxt{for all}\alpha\in\Reg;\\
\cc(@(\alpha,\beta)) &= \cc\,\alpha\cup\cc\,\beta ,
\eqtxt{for all}\alpha,\beta\in\Reg; \eqtxt{and} \\
\cc({+}(\alpha,\beta)) &= \cc\,\alpha\cup\cc\,\beta,
\eqtxt{for all}\alpha,\beta\in\Reg .
\end{align*}
We say that $\cc\,\alpha$ is \emph{the closure complexity of} $\alpha$.
E.g.,
\begin{align*}
  \cc(\mathsf{(12^*)^*})
  &= \overline{\cc(\mathsf{12^*})} =
  \overline{\cc\,\mathsf{1} \cup \cc(\mathsf{2^*})} =
  \overline{[0] \cup \overline{\cc\,\twosf}} \\
  &= \overline{[0] \cup \overline{[0]}} =
  \overline{[0] \cup [1]} =
  \overline{[1, 0]} =
  [2, 1] .
\end{align*}
In other words, the $\cc\,\alpha$ can be computed by first collecting
together all the paths through $\alpha$ that terminate in leafs, then
counting the numbers of closures visited when following each of these
paths, and finally putting those sums in descending order.

Returning to our initial examples, we have that
$\cc(\mathsf{(00^*11^*)^*}) = [2,2,1,1]$ and $\cc(\% + \mathsf{0(0 +
  11^*0)^*11^*}) = [2,1,1,1,1,0,0,0]$.  Since
$[2,1,1,1,1,0,0,0]\ltcc[2,2,1,1]$, the closure
complexity of $\% + \mathsf{0(0 + 11^*0)^*11^*}$ is strictly smaller
than the closure complexity of $\mathsf{(00^*11^*)^*}$.

\begin{proposition}
For all $\alpha\in\Reg$, $|\cc\,\alpha| = \numLeaves\,\alpha$.
\end{proposition}

\begin{proof}
An easy induction on regular expressions.
\end{proof}

\begin{exercise}
Find regular expressions $\alpha$ and $\beta$ such that
$\cc\,\alpha = \cc\,\beta$ but $\mysize\,\alpha\neq\mysize\,\beta$.
\end{exercise}

In contrast to star-height, closure complexity is compatible with the
ways of forming regular expressions.  In fact, we can prove even
stronger results.

\begin{proposition}
\label{RegCCEquiv}
\begin{enumerate}[\quad(1)]
\item For all $\alpha\in\Reg$, $\cc\,\alpha = \cc\,\beta$
  iff $\cc(\alpha^*) = \cc(\beta^*)$.

\item For all $\alpha,\beta,\gamma\in\Reg$, $\cc\,\alpha = \cc\,\beta$
  iff $\cc(\alpha\gamma) = \cc(\beta\gamma)$.

\item For all $\alpha,\beta,\gamma\in\Reg$, $\cc\,\alpha = \cc\,\beta$
  iff $\cc(\gamma\alpha) = \cc(\gamma\beta)$.

\item For all $\alpha,\beta,\gamma\in\Reg$, $\cc\,\alpha = \cc\,\beta$
  iff $\cc(\alpha + \gamma) = \cc(\beta + \gamma)$.

\item For all $\alpha,\beta,\gamma\in\Reg$, $\cc\,\alpha = \cc\,\beta$
  iff $\cc(\gamma + \alpha) = \cc(\gamma + \beta)$.
\end{enumerate}
\end{proposition}

\begin{proof}
Follows by Proposition~\ref{CCEquivContext}.
\end{proof}

The following proposition says that if we replace a subtree of
a regular expression by a regular expression with the same closure
complexity, then the closure complexity of the
resulting, whole regular expression will be unchanged.

\begin{proposition}
\label{RegCCEquivSubstituteSubtree}

Suppose $\alpha,\beta,\beta'\in\Reg$, $\cc\,\beta = \cc\,\beta'$,
$\pat\in\Path$ is valid for $\alpha$, and $\beta$ is
the subtree of $\alpha$ at position $\pat$.
Let $\alpha'$ be the result of replacing the subtree at
position $\pat$ in $\alpha$ by $\beta'$. Then $\cc\,\alpha =
\cc\,\alpha'$.
\end{proposition}

\begin{proof}
By induction on $\alpha$ using Proposition~\ref{RegCCEquiv}.
\end{proof}

\begin{proposition}
\label{RegCCLTContext}
\begin{enumerate}[\quad(1)]
\item For all $\alpha\in\Reg$, $\cc\,\alpha\ltcc\cc\,\beta$
  iff $\cc(\alpha^*)\ltcc\cc(\beta^*)$.

\item For all $\alpha,\beta,\gamma\in\Reg$, $\cc\,\alpha\ltcc\cc\,\beta$
  iff $\cc(\alpha\gamma)\ltcc\cc(\beta\gamma)$.

\item For all $\alpha,\beta,\gamma\in\Reg$, $\cc\,\alpha\ltcc\cc\,\beta$
  iff $\cc(\gamma\alpha)\ltcc\cc(\gamma\beta)$.

\item For all $\alpha,\beta,\gamma\in\Reg$, $\cc\,\alpha\ltcc\cc\,\beta$
  iff $\cc(\alpha + \gamma)\ltcc\cc(\beta + \gamma)$.

\item For all $\alpha,\beta,\gamma\in\Reg$, $\cc\,\alpha\ltcc\cc\,\beta$
  iff $\cc(\gamma + \alpha)\ltcc\cc(\gamma + \beta)$.
\end{enumerate}
\end{proposition}

\begin{proof}
Follows by Proposition~\ref{CCLTContext}.
\end{proof}

The following proposition says that if we replace a subtree of a regular
expression by a regular expression with strictly smaller closure
complexity, that the resulting, whole regular expression
will have strictly smaller closure complexity than the original
regular expression.

\begin{proposition}
\label{RegCCLTSubstituteSubtree}

Suppose $\alpha,\beta,\beta'\in\Reg$, $\cc\,\beta'\ltcc\cc\,\beta$,
$\pat\in\Path$ is valid for $\alpha$, and $\beta$ is
the subtree of $\alpha$ at position $\pat$.
Let $\alpha'$ be the result of replacing the subtree at
position $\pat$ in $\alpha$ by $\beta'$. Then $\cc\,\alpha'\ltcc
\cc\,\alpha$.
\end{proposition}

\begin{proof}
By induction on $\alpha$, using Proposition~\ref{RegCCLTContext}.
\end{proof}

When judging the relative complexity of regular expressions $\alpha$
and $\beta$, we will first look at how their closure complexities are
related.  And, when their closure complexities are equal, we will
look at how their sizes are related.  To finish explaining how
we will judge the relative complexity of regular expressions, we
need three definitions.

The function
\index{numConcats@$\numConcats$}%
\index{regular expression!numConcats@$\numConcats$}%
\index{regular expression!number of concatenations}%
\begin{gather*}
\numConcats\in\Reg\fun\nats
\end{gather*}
is defined by recursion:
\begin{align*}
\numConcats\,\% &= 0 ; \\
\numConcats\,\$ &= 0 ; \\
\numConcats\,a &= 0, \eqtxt{for all}a\in\Sym ; \\
\numConcats(\alpha^*) &= \numConcats\,\alpha , \eqtxt{for all} \alpha\in\Reg ; \\
\numConcats(\alpha\beta) &= 1 + \numConcats\,\alpha + \numConcats\,\beta ;
\eqtxt{and}\\
\numConcats(\alpha + \beta) &= \numConcats\,\alpha + \numConcats\,\beta .
\end{align*}
Thus $\numConcats\,\alpha$ is the number of concatenations in
$\alpha$, i.e., the number of subtrees of $\alpha$ that are
concatenations, where a given concatenation may occur (and will be
counted) multiple times.  E.g.,
$\numConcats(\mathsf{((01)^*(01))^*}) = 3$.  The function
\index{numSyms@$\numSyms$}%
\index{regular expression!numSyms@$\numSyms$}%
\index{regular expression!number of symbols}%
\begin{gather*}
\numSyms\in\Reg\fun\nats
\end{gather*}
is defined by structural recursion:
\begin{align*}
\numSyms\,\% &= 0 ; \\
\numSyms\,\$ &= 0 ; \\
\numSyms\,a &= 1, \eqtxt{for all}a\in\Sym ; \\
\numSyms(\alpha^*) &= \numSyms\,\alpha , \eqtxt{for all} \alpha\in\Reg ; \\
\numSyms(\alpha\beta) &= \numSyms\,\alpha + \numSyms\,\beta ; \eqtxt{and}\\
\numSyms(\alpha + \beta) &= \numSyms\,\alpha + \numSyms\,\beta .
\end{align*}
Thus $\numSyms\,\alpha$ is the number of occurrences of symbols in $\alpha$,
where a given symbol may occur (and will be counted) more than once.
E.g., $\numSyms(\mathsf{(0^*1)+0}) = 3$.

Finally, we say that a regular expression $\alpha$ is \emph{standardized}
\index{standardized}%
\index{regular expression!standardized}%
iff none of $\alpha$'s subtrees have any of the following forms:
\begin{itemize}
\item $(\beta_1+\beta_2)+\beta_3$ (we can avoid needing parentheses,
  and make a regular expression easier to understand/process from
  left-to-right, by grouping unions to the right);

\item $\beta_1+\beta_2$, where $\beta_1>\beta_2$, or
  $\beta_1+(\beta_2+\beta_3)$, where $\beta_1>\beta_2$ (it's pleasing
  if the regular expressions appear in order (recall that unions
  are greater than all other kinds of regular expressions));

\item $(\beta_1\beta_2)\beta_3$ (we can avoid needing parentheses, and
  make a regular expression easier to understand/process from
  left-to-right, by grouping concatenations to the right); and

\item $\beta^*\beta$, $\beta^*(\beta\gamma)$,
  $(\beta_1\beta_2)^*\beta_1$ or $(\beta_1\beta_2)^*(\beta_1\gamma)$
  (moving closures to the right makes a regular expression easier to
  understand/process from left-to-right).
\end{itemize}
Thus every subtree of a standardized regular expression will be standardized.

Returning to our assessment of regular expression complexity, suppose
that $\alpha$ and $\beta$ are regular expressions generating $\%$.
Then $(\alpha\beta)^*$ and $(\alpha+\beta)^*$ are equivalent, but will
will prefer the latter over the former, because unions are generally
more amenable to understanding and processing than concatenations.
Consequently, when two regular expression have the same closure
complexity and size, we will judge their relative complexity
according to their numbers of concatenations.

Next, consider the regular expressions $\mathsf{0 + 01}$ and
$\mathsf{0(\% + 1)}$.  These regular expressions have the same closure
complexity $[0,0,0]$, size ($5$) and number of concatenations ($1$).
We would like to consider the latter to be simpler than the former,
since in general we would like to prefer $\alpha(\%+\beta)$ over
$\alpha + \alpha\beta$.  And we can base this preference on the fact
that the number of symbols of $\mathsf{0(\% + 1)}$ ($2$) is one less
than the number of symbols of $\mathsf{0 + 01}$.  When regular
expressions have the same closure complexity, size and number of
concatenations, the one with fewer symbols is likely to be easier to
understand and process.  Thus, when regular expressions have
identical closure complexity, size and number of concatenations, we
will use their relative numbers of symbols to judge their relative
complexity.

Finally, when regular expressions have the same closure complexity,
size, number of concatenations, and number of symbols, we will judge
their relative complexity according to whether they are standardized,
thinking that a standardized regular expression is simpler than one
that is not standardized.

We define a relation $\ltsimp$ on $\Reg$ by, for all $\alpha,\beta\in\Reg$,
$\alpha\ltsimp\beta$ iff:
\begin{itemize}
\item $\cc\,\alpha \ltcc \cc\,\beta$; or

\item $\cc\,\alpha = \cc\,\beta$ but $\mysize\,\alpha < \mysize\,\beta$; or

\item $\cc\,\alpha = \cc\,\beta$ and $\mysize\,\alpha = \mysize\,\beta$,
  but $\numConcats\,\alpha < \numConcats\,\beta$; or

\item $\cc\,\alpha = \cc\,\beta$, $\mysize\,\alpha = \mysize\,\beta$
  and $\numConcats\,\alpha = \numConcats\,\beta$, but
  $\numSyms\,\alpha < \numSyms\,\beta$; or

\item $\cc\,\alpha = \cc\,\beta$, $\mysize\,\alpha = \mysize\,\beta$,
  $\numConcats\,\alpha = \numConcats\,\beta$ and $\numSyms\,\alpha =
  \numSyms\,\beta$, but $\alpha$ is standardized and $\beta$ is
  not standardized.
\end{itemize}

We read $\alpha\ltsimp\beta$ as $\alpha$ is \emph{simpler} (less
\emph{complex}) than $\beta$.  We define a relation $\equivsimp$ on
$\Reg$ by, for all $\alpha,\beta\in\Reg$, $\alpha\equivsimp\beta$ iff
$\alpha$ and $\beta$ have the same closure complexity, size, numbers
of concatenations, numbers of symbols, and status of being (or not
being) standardized.  We read $\alpha\equivsimp\beta$ as $\alpha$ and
$\beta$ have the \emph{same complexity}.  Finally, we define a
relation $\leqsimp$ on $\Reg$ by, for all $\alpha,\beta\in\Reg$,
$\alpha\leqsimp\beta$ iff $\alpha\ltsimp\beta$ or
$\alpha\equivsimp\beta$.  We read $\alpha\leqsimp\beta$ as $\alpha$ is
\emph{at least as simple as} (\emph{no more complex than}) $\beta$.

For example, the following regular expressions are equivalent and have
the same complexity:
\begin{displaymath}
\mathsf{1(01 + 10) + (\% + 01)1} \quad\eqtxt{and}\quad
\mathsf{011 + 1(\% + 01 + 10)} .  
\end{displaymath}

\begin{proposition}
\begin{enumerate}[\quad(1)]
\item $\ltsimp$ is transitive.

\item $\equivsimp$ is reflexive on $\Reg$, transitive and symmetric.

\item For all $\alpha,\beta\in\Reg$, exactly one of the following holds:
$\alpha\ltsimp\beta$, $\beta\ltsimp\alpha$ or $\alpha\equivsimp\beta$.

\item $\leqsimp$ is transitive, and, for all $\alpha,\beta\in\Reg$,
$\alpha\equivsimp\beta$ iff $\alpha\leqsimp\beta$ and $\beta\leqsimp\alpha$.
\end{enumerate}
\end{proposition}

The Forlan module \texttt{Reg} defines the abstract type \texttt{cc}
of closure complexities, along with these functions:
\begin{verbatim}
val ccToList  : cc -> int list
val singCC    : int -> cc
val unionCC   : cc * cc -> cc
val succCC    : cc -> cc
val cc        : reg -> cc
val compareCC : cc * cc -> order
\end{verbatim}
The function \texttt{ccToList} is the identity function on closure
complexities: all that changes is the type.  $\mathtt{singCC\,n}$
returns the singleton closure complexity $[n]$, if $n$ is non-negative;
otherwise it issues an error message.  The functions \texttt{unionCC}
and \texttt{succCC} implement the union and successor operations on
closure complexities.  The function \texttt{cc} corresponds to $\cc$, and
\texttt{compareCC} implements $\ltcc$.

Here are some examples of how these functions can be used:
\begin{list}{}
{\setlength{\leftmargin}{\leftmargini}
\setlength{\rightmargin}{0cm}
\setlength{\itemindent}{0cm}
\setlength{\listparindent}{0cm}
\setlength{\itemsep}{0cm}
\setlength{\parsep}{0cm}
\setlength{\labelsep}{0cm}
\setlength{\labelwidth}{0cm}
\catcode`\#=12
\catcode`\$=12
\catcode`\%=12
\catcode`\^=12
\catcode`\_=12
\catcode`\.=12
\catcode`\?=12
\catcode`\!=12
\catcode`\&=12
\ttfamily}
\small
\item[]\textsl{-\ }val\ ns\ =\ Reg.succCC(Reg.unionCC(Reg.singCC\ 1,\ Reg.singCC\ 1));
\item[]\textsl{val\ ns\ =\ -\ :\ Reg.cc}
\item[]\textsl{-\ }Reg.ccToList\ ns;
\item[]\textsl{val\ it\ =\ \symbol{'133}2,2\symbol{'135}\ :\ int\ list}
\item[]\textsl{-\ }val\ ms\ =\ Reg.unionCC(ns,\ Reg.succCC\ ns);
\item[]\textsl{val\ ms\ =\ -\ :\ Reg.cc}
\item[]\textsl{-\ }Reg.ccToList\ ms;
\item[]\textsl{val\ it\ =\ \symbol{'133}3,3,2,2\symbol{'135}\ :\ int\ list}
\item[]\textsl{-\ }Reg.ccToList(Reg.cc(Reg.fromString\ "(00\symbol{'052}11\symbol{'052})\symbol{'052}"));
\item[]\textsl{val\ it\ =\ \symbol{'133}2,2,1,1\symbol{'135}\ :\ int\ list}
\item[]\textsl{-\ }Reg.ccToList(Reg.cc(Reg.fromString\ "%\ +\ 0(0\ +\ 11\symbol{'052}0)\symbol{'052}11\symbol{'052}"));
\item[]\textsl{val\ it\ =\ \symbol{'133}2,1,1,1,1,0,0,0\symbol{'135}\ :\ int\ list}
\item[]\textsl{-\ }Reg.compareCC
\item[]\textsl{=\ }(Reg.cc(Reg.fromString\ "(00\symbol{'052}11\symbol{'052})\symbol{'052}"),
\item[]\textsl{=\ }\ Reg.cc(Reg.fromString\ "%\ +\ 0(0\ +\ 11\symbol{'052}0)\symbol{'052}11\symbol{'052}"));
\item[]\textsl{val\ it\ =\ GREATER\ :\ order}
\item[]\textsl{-\ }Reg.compareCC
\item[]\textsl{=\ }(Reg.cc(Reg.fromString\ "(00\symbol{'052}11\symbol{'052})\symbol{'052}"),
\item[]\textsl{=\ }\ Reg.cc(Reg.fromString\ "(1\symbol{'052}10\symbol{'052}0)\symbol{'052}"));
\item[]\textsl{val\ it\ =\ EQUAL\ :\ order}
\end{list}


The module \texttt{Reg} also includes these functions:
\begin{verbatim}
val numConcats             : reg -> int
val numSyms                : reg -> int
val standardized           : reg -> bool
val compareComplexity      : reg * reg -> order
val compareComplexityTotal : reg * reg -> order
\end{verbatim}
\index{Reg@\texttt{Reg}!compareComplexity@\texttt{compareComplexity}}%
\index{Reg@\texttt{Reg}!compareComplexityTotal@\texttt{compareComplexityTotal}}%
The first two functions implement the functions with the same names.
The function \texttt{standardized} tests whether a regular expression
is standardized, and the function \texttt{compareComplexity} implements
$\leqsimp$/$\equivsimp$. Finally, \texttt{compareComplexityTotal} is like
\texttt{compareComplexity}, but falls back on \texttt{Reg.compare}
(our total ordering on regular expressions) to order regular expressions
with the same complexity.  Thus \texttt{compareComplexityTotal} is
a total ordering.

Here are some examples of how these functions can be used:
\begin{list}{}
{\setlength{\leftmargin}{\leftmargini}
\setlength{\rightmargin}{0cm}
\setlength{\itemindent}{0cm}
\setlength{\listparindent}{0cm}
\setlength{\itemsep}{0cm}
\setlength{\parsep}{0cm}
\setlength{\labelsep}{0cm}
\setlength{\labelwidth}{0cm}
\catcode`\#=12
\catcode`\$=12
\catcode`\%=12
\catcode`\^=12
\catcode`\_=12
\catcode`\.=12
\catcode`\?=12
\catcode`\!=12
\catcode`\&=12
\ttfamily}
\small
\item[]\textsl{-\ }Reg.numConcats(Reg.fromString\ "(01)\symbol{'052}(10)\symbol{'052}");
\item[]\textsl{val\ it\ =\ 3\ :\ int}
\item[]\textsl{-\ }Reg.numSyms(Reg.fromString\ "(01)\symbol{'052}(10)\symbol{'052}");
\item[]\textsl{val\ it\ =\ 4\ :\ int}
\item[]\textsl{-\ }Reg.standardized(Reg.fromString\ "00\symbol{'052}1");
\item[]\textsl{val\ it\ =\ true\ :\ bool}
\item[]\textsl{-\ }Reg.standardized(Reg.fromString\ "00\symbol{'052}0");
\item[]\textsl{val\ it\ =\ false\ :\ bool}
\item[]\textsl{-\ }Reg.compareComplexity
\item[]\textsl{=\ }(Reg.fromString\ "(00\symbol{'052}11\symbol{'052})\symbol{'052}",
\item[]\textsl{=\ }\ Reg.fromString\ "%\ +\ 0(0\ +\ 11\symbol{'052}0)\symbol{'052}11\symbol{'052}");
\item[]\textsl{val\ it\ =\ GREATER\ :\ order}
\item[]\textsl{-\ }Reg.compareComplexity
\item[]\textsl{=\ }(Reg.fromString\ "0\symbol{'052}\symbol{'052}1\symbol{'052}\symbol{'052}",\ \ Reg.fromString\ "(01)\symbol{'052}\symbol{'052}");
\item[]\textsl{val\ it\ =\ GREATER\ :\ order}
\item[]\textsl{-\ }Reg.compareComplexity
\item[]\textsl{=\ }(Reg.fromString\ "(0\symbol{'052}1\symbol{'052})\symbol{'052}",\ \ Reg.fromString\ "(0\symbol{'052}+1\symbol{'052})\symbol{'052}");
\item[]\textsl{val\ it\ =\ GREATER\ :\ order}
\item[]\textsl{-\ }Reg.compareComplexity
\item[]\textsl{=\ }(Reg.fromString\ "0+01",\ \ Reg.fromString\ "0(%+1)");
\item[]\textsl{val\ it\ =\ GREATER\ :\ order}
\item[]\textsl{-\ }Reg.compareComplexity
\item[]\textsl{=\ }(Reg.fromString\ "(01)2",\ \ Reg.fromString\ "012");
\item[]\textsl{val\ it\ =\ GREATER\ :\ order}
\item[]\textsl{-\ }Reg.compareComplexity
\item[]\textsl{=\ }(Reg.fromString\ "1(01+10)+(%+01)1",
\item[]\textsl{=\ }\ Reg.fromString\ "011+1(%+01+10)");
\item[]\textsl{val\ it\ =\ EQUAL\ :\ order}
\end{list}


\subsection{Weak Simplification}

In this subsection, we give our first simplification algorithm: weak
simplification.  We say that a regular expression $\alpha$ is
\emph{weakly simplified}
\index{weakly simplified}%
\index{regular expression!weakly simplified}%
\index{simplification!regular expression!weakly simplified}%
iff $\alpha$ is standardized and none of $\alpha$'s subtrees have
any of the following forms:
\begin{itemize}
\item $\$+\beta$ or $\beta+\$$ (the $\$$ is redundant);

\item $\beta+\beta$ or $\beta+(\beta+\gamma)$ (the duplicate occurrence
  of $\beta$ is redundant);

\item $\%\beta$ or $\beta\%$ (the $\%$ is redundant);

\item $\$\beta$ or $\beta\$$ (both are equivalent to $\$$); and

\item $\%^*$ or $\$^*$ or $(\beta^*)^*$ (the first two can be replaced
  by $\%$, and the extra closure can be omitted in the third case).
\end{itemize}
Thus, if a regular expression $\alpha$ is weakly simplified,
then each of its subtrees will also be weakly simplified.

Weakly simplified regular expressions have some pleasing properties.

\begin{proposition}
\label{WeakSimpProp3}
\begin{enumerate}[\quad(1)]
\item For all $\alpha\in\Reg$, if $\alpha$ is weakly simplified and
  $L(\alpha)=\emptyset$, then $\alpha=\$$.

\item For all $\alpha\in\Reg$, if $\alpha$ is weakly simplified and
  $L(\alpha)=\{\%\}$, then $\alpha=\%$.

\item For all $\alpha\in\Reg$, for all $a\in\Sym$, if $\alpha$ is
  weakly simplified and $L(\alpha)=\{a\}$, then $\alpha=a$.
\end{enumerate}
\end{proposition}

E.g., part~(2) of the proposition says that, if $\alpha$
is weakly simplified and $L(\alpha)$ is the language whose only
string is $\%$, then $\alpha$ is $\%$.

\begin{proof}
The three parts are proved in order, using induction on regular
expressions.

For part~(3), suppose $a\in\Sym$. It will suffice to show that,
for all $\alpha\in\Reg$, if $\alpha$ is weakly simplified and
$L(\alpha)=\{a\}$, then $\alpha=a$.
We will show the concatenation case of the induction.  Suppose
$\alpha,\beta\in\Reg$ and assume the inductive hypothesis:
if $\alpha$ is weakly simplified and $L(\alpha)=\{a\}$,
then $\alpha=a$; and, if $\beta$ is weakly
simplified and $L(\beta)=\{a\}$, then $\beta=a$.
Assume that $\alpha\beta$ is weakly simplified and
$L(\alpha\beta)=\{a\}$.  We must show that $\alpha\beta=a$.  Because
$\alpha\beta$ is weakly simplified, so are $\alpha$ and $\beta$.

Since $L(\alpha)L(\beta)=L(\alpha\beta)=\{a\}$, there are two cases to
consider.
\begin{itemize}
\item Suppose $L(\alpha)=\{a\}$ and $L(\beta)=\{\%\}$.  Since $\beta$
  is weakly simplified and $L(\beta)=\{\%\}$, part~(2) tells us that
  $\beta=\%$.  But this means that $\alpha\beta=\alpha\%$ is not
  weakly simplified after all---contradiction.  Thus we can conclude
  that $\alpha\beta=a$.

\item Suppose $L(\alpha)=\{\%\}$ and $L(\beta)=\{a\}$.  The proof of
  this case is similar to that of the other one.
\end{itemize}
(Note that we didn't use the inductive hypothesis on either $\alpha$ or
$\beta$.)

We use both parts of the inductive hypothesis when proving the union
case. If $L(\alpha)\cup L(\beta) = L(\alpha+\beta)=\{a\}$, then one
possibility is that one of $L(\alpha)$ or $L(\beta)$ is $\emptyset$,
in which case we use part~(1) to get our contradiction. Otherwise,
$L(\alpha) = \{a\} = L(\beta)$, and so the inductive hypothesis tells
us $\alpha = a =\beta$, so that $\alpha+\beta = a + a$, giving us the
contradiction.
\end{proof}

\begin{proposition}
\label{WeakSimpProp2}
For all $\alpha\in\Reg$, if $\alpha$ is weakly simplified, then
$\alphabet(L(\alpha)) = \alphabet\,\alpha$.
\end{proposition}

\begin{proof}
By Proposition~\ref{AlphabetRegMeaning}, it suffices to show that,
for all $\alpha\in\Reg$, if $\alpha$ is weakly simplified, then
$\alphabet\,\alpha\sub\alphabet(L(\alpha))$.  And this
follows by an easy induction on $\alpha$, using
Proposition~\ref{WeakSimpProp3}(2).
\end{proof}

The next proposition says that $\$$ need only be used at the 
top-level of a regular expression.

\begin{proposition}
\label{WeakSimpProp4}
For all $\alpha\in\Reg$, if $\alpha$ is weakly simplified and $\alpha$ has
one or more occurrences of $\$$, then $\alpha=\$$.
\end{proposition}

\begin{proof}
An easy induction on regular expressions.
\end{proof}

Finally, we have that weakly simplified regular expressions with
closures generate infinite languages:

\begin{proposition}
For all $\alpha\in\Reg$, if $\alpha$ is weakly simplified and
$\alpha$ has one or more closures, then $L(\alpha)$ is
infinite.
\end{proposition}

\begin{proof}
An easy induction on regular expressions.
\end{proof}

Next, we see how we can test whether a regular expression is weakly
simplified via a simple structural recursion.  Define
\index{regular expression!weaklySimplified@$\weaklySimplified$}%
$\weaklySimplified\in\Reg\fun\Bool$ by structural recursion, as
follows.  Given a regular expression $\alpha$, it proceeds as follows:
\begin{itemize}
\item Suppose $\alpha$ is $\%$, $\$$ or a symbol.  Then it returns $\true$.

\item Suppose $\alpha$ has the form $\beta^*$.  Then it checks that:
  \begin{itemize}
  \item $\beta$ is weakly simplified (this is done using recursion); and

  \item $\beta$ is neither $\%$, nor $\$$, nor a closure.
  \end{itemize}

\item Suppose $\alpha$ has the form $\alpha_1\,\alpha_2$.  Then it
  checks that:
  \begin{itemize}
  \item $\alpha_1$ and $\alpha_2$ are weakly simplified; and

  \item $\alpha_1$ is neither $\%$ nor $\$$ nor a concatenation; and

  \item $\alpha_2$ is neither $\%$ nor $\$$; and

  \item $\alpha$ has none of the following forms: $\beta^*\beta$,
    $\beta^*(\beta\gamma)$, $(\beta_1\beta_2)^*\beta_1$ or
    $(\beta_1\beta_2)^*\beta_1\gamma$.
  \end{itemize}
  
\item Suppose $\alpha$ has the form $\alpha_1 + \alpha_2$.  Then it
  checks that:
  \begin{itemize}
  \item $\alpha_1$ and $\alpha_2$ are weakly simplified; and

  \item $\alpha_1$ is neither $\$$ nor a union; and

  \item $\alpha_2$ is not $\$$;

  \item if $\alpha_2$ has the form $\beta_1 + \beta_2$, then
    $\alpha_1<\beta_1$; and

  \item if $\alpha_2$ is not a union, then $\alpha_1<\alpha_2$.
  \end{itemize}
\end{itemize}

\begin{proposition}
For all $\alpha\in\Reg$, $\alpha$ is weakly simplified iff
$\weaklySimplified\,\alpha = \true$.
\end{proposition}

\begin{proof}
By induction on regular expressions.
\end{proof}

In preparation for giving our weak simplification algorithm, we need
to define some auxiliary functions. We will also prove some lemmas
that will support our proofs about the weak simplification algorithm.

First, we prove lemmas about $\rightConcat$ and $\rightUnion$ of
Section~\ref{RegularExpressionsAndLanguages}.

\begin{lemma}
\label{RightConcatLem}
If $\alpha,\beta\in\Reg$, then $\rightConcat(\alpha,\beta)$ is
equivalent to $\alpha\beta$, and has the same alphabet, closure
complexity, size, number of concatenations, number of symbols, and
number of leaves as $\alpha\beta$.
\end{lemma}

\begin{proof}
By induction on $\alpha$.
\end{proof}

\begin{lemma}
\label{RightUnionLem}
If $\alpha,\beta\in\Reg$, then $\rightUnion(\alpha,\beta)$ is
equivalent to $\alpha+\beta$, and has the same alphabet, closure
complexity, size, number of concatenations, and number of symbols as
$\alpha+\beta$.
\end{lemma}

\begin{proof}
By induction on $\alpha$.
\end{proof}

We say that a regular expression
$\alpha$ is \emph{almost weakly simplified} iff either:
\begin{itemize}
\item $w\in\{\%,\$\}$; or

\item all elements of $\concatsToList\,\alpha$ are weakly simplified,
  and are not $\%$, $\$$ or concatenations.
\end{itemize}

For example, $\mathsf{0^*0(1+2)^*(1+2)} =
\mathsf{0^*(0((1+2)^*(1+2)))}$ is almost weakly simplified, even
though it's not weakly simplified.  On the other hand:
$\mathsf{(\$+1)1}$ isn't almost weakly simplified, because
$\mathsf{\$+1}$ isn't weakly simplified; $\mathsf{1\%}$ isn't weakly
simplified, because of the location of $\%$; and $\mathsf{(01)1}$
isn't almost weakly simplified, because of the location of the
concatenation $\mathsf{01}$.

Let
\begin{align*}
  \WS &= \setof{\alpha\in\Reg}{\alpha\eqtxtl{is weakly simplified}},
  \eqtxt{and} \\
  \AWS &= \setof{\alpha\in\Reg}{\alpha\eqtxtl{is almost weakly simplified}} .
\end{align*}

In preparation for defining a function
$\shiftClosuresRight\in\AWS\fun\WS$ for shifting closures to the right
in concatenations at the top-level, we need a ``closure shift
metric''. Define $\csm\in\Reg\times\nats\fun\nats$ by structural
recursion on its first argument:
\begin{itemize}
\item for all $n\in\nats$, $\csm(\%, n) = 0$;

\item for all $n\in\nats$, $\csm(\$, n) = 0$;

\item for all $n\in\nats$ and symbols $a$, $\csm(a, n) = 0$;

\item for all $n\in\nats$ and $\alpha\in\Reg$,
  $\csm(\alpha^*, n) = \csm(\alpha, n) + n$;

\item for all $n\in\nats$ and $\alpha,\beta\in\Reg$,
  $\csm(\alpha\beta, n) = \csm(\alpha, \numLeaves\,\beta + n) +
  \csm(\beta, n)$;

\item for all $n\in\nats$ and $\alpha, \beta\in\Reg$,
  $\csm(\alpha+\beta, n) = 0$.
\end{itemize}

Define relations $\ltcsm$, $\equivcsm$ and $\lecsm$
on $\Reg$ by:
\begin{itemize}
\item $\alpha\ltcsm\beta$ iff, for all $n\in\nats$,
  $\numLeaves\,\alpha = \numLeaves\,\beta$ and
  $\csm(\alpha, n) < \csm(\beta, n)$;

\item $\alpha\equivcsm\beta$ iff for all $n\in\nats$,
  $\numLeaves\,\alpha = \numLeaves\,\beta$ and
  $\csm(\alpha, n) = \csm(\beta, n)$;

\item $\alpha\lecsm\beta$ iff $\alpha\ltcsm\beta$ or
  $\alpha\equivcsm\beta$.
\end{itemize}

\begin{lemma}
\label{CSMWF}
$\ltcsm$ is a well-founded relation on $\Reg$.
\end{lemma}

\begin{proof}
By Proposition~\ref{InverseImageWellFounded}, the relation $R$ on
$\Reg$ defined by $\alpha\mathrel{R}\beta$ iff
$\csm(\alpha, 0)<\csm(\beta, 0)$ is well-founded, because $<$ is
well-founded.  But ${\ltcsm}\sub R$, and thus $\ltcsm$ is well-founded
by Proposition~\ref{WellFoundedSubset}.
\end{proof}

The following lemma holds by easy calculations:

\begin{lemma}
\begin{enumerate}[\quad(1)]
\item $\ltcsm$ and $\equivcsm$ are transitive, and $\equivcsm$ is
  symmetric.

\item For all $\alpha, \beta, \gamma\in\ Reg$, if
  $\alpha\ltcsm\beta\equivcsm\gamma$, then $\alpha\ltcsm\gamma$.

\item For all $\alpha, \beta, \gamma\in\ Reg$, if
  $\alpha\equivcsm\beta\ltcsm\gamma$, then $\alpha\ltcsm\gamma$.

\item $\lecsm$ is transitive.

\item For all $\alpha, \beta, \gamma\in\ Reg$, if
  $\alpha\ltcsm\beta\lecsm\gamma$, then $\alpha\lecsm\gamma$.

\item For all $\alpha, \beta, \gamma\in\ Reg$, if
  $\alpha\lecsm\beta\ltcsm\gamma$, then $\alpha\lecsm\gamma$.

\item For all $\alpha, \beta, \gamma\in\ Reg$, if
  $\alpha\equivcsm\beta\lecsm\gamma$, then $\alpha\lecsm\gamma$.

\item For all $\alpha, \beta, \gamma\in\ Reg$, if
  $\alpha\lecsm\beta\equivcsm\gamma$, then $\alpha\lecsm\gamma$.
\end{enumerate}
\end{lemma}

\begin{lemma}
\label{CSMContexts}
Suppose $\gamma,\alpha,\beta\in\Reg$.
\begin{enumerate}[\quad(1)]
\item If $\alpha\ltcsm\beta$, then $\gamma\alpha\ltcsm\gamma\beta$.
\item If $\alpha\equivcsm\beta$, then $\gamma\alpha\equivcsm\gamma\beta$.
\item If $\alpha\ltcsm\beta$, then $\alpha\gamma\ltcsm\beta\gamma$.
\item If $\alpha\equivcsm\beta$, then $\alpha\gamma\equivcsm\beta\gamma$.
\item If $\alpha\ltcsm\beta$, then $\alpha^*\ltcsm\beta^*$.
\item If $\alpha\equivcsm\beta$, then $\alpha^*\equivcsm\beta^*$.
\item If $\alpha\lecsm\beta$, then $\gamma\alpha\lecsm\gamma\beta$.
\item If $\alpha\lecsm\beta$, then $\alpha\gamma\lecsm\beta\gamma$.
\item If $\alpha\lecsm\beta$, then $\alpha^*\lecsm\beta^*$.
\end{enumerate}
\end{lemma}

\begin{proof}
By easy calculations.
\end{proof}

\begin{lemma}
\label{RightConcatCSM}
If $\alpha,\beta\in\Reg$, then
$\rightConcat(\alpha,\beta)\equivcsm\alpha\beta$.
\end{lemma}

\begin{proof}
By structural induction on $\alpha$.
\end{proof}

We define $\shiftClosuresRight\in\AWS\fun\WS$ by well-founded
recursion on the lexicographic relation that first considers size and
then falls back on $\ltcsm$. We want that, for all $\alpha\in\AWS$,
$\shiftClosuresRight\,\alpha$ is equivalent
to $\alpha$, and that it has the same alphabet, closure complexity,
size, number of concatenations and number of symbols as $\alpha$.  But
to prove this, we also need that
\begin{itemize}
\item $\shiftClosuresRight\,\alpha\lecsm\alpha$;

\item if $\alpha\notin\{\%,\$\}$, then $\shiftClosuresRight\,\alpha\not\in
\{\%,\$\}$.
\end{itemize}
Given $\alpha\in\AWS$, $\shiftClosuresRight$ proceeds as follows.  If
$\alpha$ is not a concatenation, than it returns $\alpha$, which will
be weakly simplified and satisfy the remaining properties.  Otherwise,
$\alpha = \alpha_1\alpha_2$ for some $\alpha_1,\alpha_2\in\Reg$.
Since $\alpha$ is almost weakly simplified, then: $\alpha_1$ is weakly
simplified and is either a symbol, a closure or a union; and
$\alpha_2$ is almost weakly simplified and is neither $\%$ nor $\$$.
So $\shiftClosuresRight$ lets
$\alpha'_2\in\WS$ be the result of recursively calling itself on
$\alpha_2$ (this is legal, because
$\alpha_2$ is strictly smaller than $\alpha$).  Then
$\alpha'_2$ is neither $\%$ nor $\$$. Furthermore $\alpha'_2$ is
equivalent to $\alpha_2$, $\alpha'_2\lecsm\alpha_2$, and $\alpha'_2$
has the same alphabet, closure complexity, size, number of
concatenations, and number of symbols as $\alpha_2$.  Thus we have
that $\alpha_1\alpha'_2$ is equivalent $\alpha_1\alpha_2=\alpha$,
$\alpha_1\alpha'_2\lecsm\alpha_1\alpha_2=\alpha$ (by
Lemma~\ref{CSMContexts}), and $\alpha_1\alpha'_2$ has the same
alphabet, closure complexity, size, number of concatenations, and
number of symbols as $\alpha$.
\begin{itemize}
\item Suppose $\alpha_1\alpha'_2$ has the form $\beta^*\beta$, for
  some $\beta\in\Reg$.  An easy calculation shows that
  $\beta\beta^* \ltcsm \beta^*\beta\lecsm\alpha$, so that
  $\beta\beta^* \ltcsm \alpha$.  Furthermore, $\beta\beta^*$ is
  equivalent to $\beta^*\beta$ (and so is equivalent to $\alpha$), and
  has the same alphabet, closure complexity, size, number of
  concatenations, and number of symbols as $\beta^*\beta$ (and so as
  $\alpha$). If $\beta$ is a concatenation, then $\beta\beta^*$ will
  not be almost weakly simplified, so let
  $\gamma = \rightConcat(\beta, \beta^*)$. Then $\gamma$ is almost
  weakly simplified, and by Lemma~\ref{RightConcatCSM}, we have that
  $\gamma\equivcsm\beta\beta^*$, and (Lemma~\ref{RightConcatLem}),
  $\gamma$ is equivalent to $\beta\beta^*$, and $\gamma$ has the same
  alphabet, closure complexity, size, number of concatenations, and
  number of symbols as $\beta\beta^*$. Thus $\gamma\ltcsm\alpha$,
  $\gamma$ is equivalent to $\alpha$, and $\gamma$ has the same
  alphabet, closure complexity, size, number of concatenations, and
  number of symbols as $\alpha$.  Because $\gamma$ has the same size
  as $\alpha$ but $\gamma\ltcsm\alpha$, it is legal for
  $\shiftClosuresRight$ to recursively call itself with $\gamma$ and
  return the result, $\delta$, giving us that $\delta$ is weakly
  simplified, $\delta\lecsm\gamma$, $\delta$ is equivalent to
  $\gamma$, and $\delta$ has the same alphabet, closure complexity,
  size, number of concatenations, and number of symbols as
  $\gamma$. Thus $\delta$ is weakly simplified, $\delta\lecsm\alpha$,
  $\delta$ is equivalent to $\alpha$, and $\delta$ has the same
  alphabet, closure complexity, size, number of concatenations, and
  number of symbols as $\alpha$.

\item Otherwise, suppose $\alpha_1\alpha'_2$ has the form
  $\beta^*\beta\gamma$, for some $\beta,\gamma\in\Reg$. Then
  $\shiftClosuresRight$ returns
  \begin{displaymath}
   \shiftClosuresRight(\beta\beta^*\gamma) . 
  \end{displaymath}
  This is correct for similar reasons to the first case.

\item Otherwise, if $\alpha_1\alpha'_2$ has the form
  $(\beta_1\beta_2)^*\beta_1$, for some
  $\beta_1,\beta_2\in\Reg$, then $\shiftClosuresRight$ returns
  \begin{displaymath}
   \shiftClosuresRight(\beta_1(\rightConcat(\beta_2,\beta_1))^*) . 
  \end{displaymath}
  This is correct for similar reasons to the first case.

\item Otherwise, if $\alpha_1\alpha'_2$ has the form
  $(\beta_1\beta_2)^*\beta_1\gamma$, for some
  $\beta_1,\beta_2,\gamma\in\Reg$, then $\shiftClosuresRight$ returns
  \begin{displaymath}
   \shiftClosuresRight(\beta_1(\rightConcat(\beta_2,\beta_1))^*\gamma) . 
  \end{displaymath}
  This is correct for similar reasons to the first case.

\item Otherwise, $\shiftClosuresRight$ returns $\alpha_1\alpha'_2$,
  which is weakly simplified and satisfies the other properties.
\end{itemize}

\begin{proposition}
\label{ShiftClosuresRightLem}
For all $\alpha\in\AWS$, $\shiftClosuresRight\,\alpha$ is equivalent
to $\alpha$ and has the same alphabet, closure complexity, size,
number of concatenations and number of symbols as $\alpha$.
\end{proposition}

Define a function $\deepClosure\in\WS\fun\WS$ as follows.  For all
$\alpha\in\WS$:
\begin{align*}
  \deepClosure\,\% &= \% , \\
  \deepClosure\,\$ &= \% , \\
  \deepClosure\,({*}(\alpha)) &= \alpha^* , \eqtxt{and} \\
  \deepClosure\,\alpha &= \alpha^*, \eqtxt{if}
  \alpha\not\in\{\%,\$\} \eqtxt{and} \alpha \eqtxtl{is not a closure.}
\end{align*}

\begin{lemma}
\label{DeepClosureLem}
For all $\alpha\in\WS$, $\deepClosure\,\alpha$ is equivalent to
$\alpha^*$, has the same alphabet as $\alpha^*$, has a closure
complexity that is no bigger than that of $\alpha^*$, has a size that
is no bigger than that of $\alpha^*$, has the same number of
concatenations as $\alpha^*$, and has the same number of symbols
$\alpha^*$.
\end{lemma}

Define a function $\deepConcat\in\WS\times\WS\fun\WS$ as follows.  For all
$\alpha,\beta\in\WS$:
\begin{align*}
  \deepConcat(\alpha, \$) &= \$ , \\
  \deepConcat(\$, \alpha) &= \$ , \eqtxt{if} \alpha\neq\$ , \\
  \deepConcat(\alpha, \%) &= \alpha, \eqtxt{if} \alpha\neq\$ , \\
  \deepConcat(\%, \alpha) &= \alpha, \eqtxt{if} \alpha\not\in\{\$,\%\} ,
  \eqtxt{and} \\
  \deepConcat(\alpha,\beta) &=
    \shiftClosuresRight(\rightConcat(\alpha, \beta)), \\
    &{} \hspace*{1cm}\eqtxt{if}\alpha,\beta\not\in\{\$,\%\} .
\end{align*}
To see that the last clause of this definition is proper, suppose
that $\alpha,\beta\in\WS-\{\%,\$\}$.  Thus all the elements of
$\concatsToList\,\alpha$ and $\concatsToList\,\beta$ are weakly
simplified, and are not $\%$, $\$$ or concatenations.  Hence
\begin{displaymath}
\concatsToList(\rightConcat(\alpha,\beta)) =  
\concatsToList\,\alpha \myconcat \concatsToList\,\beta
\end{displaymath}
also has this property, showing that $\rightConcat(\alpha,\beta)$
is almost weakly simplified, which is what $\shiftClosuresRight$
needs to deliver a weakly simplified result.

\begin{lemma}
\label{DeepConcatLem}
For all $\alpha,\beta\in\WS$, $\deepConcat(\alpha,\beta)$
is equivalent to $\alpha\beta$, has an alphabet that is a
subset of the alphabet of $\alpha\beta$, has a closure complexity that
is no bigger than that of $\alpha\beta$, has a size that is no bigger
than that of $\alpha\beta$, has no more concatenations than
$\alpha\beta$, and has no more symbols than $\alpha\beta$.
\end{lemma}

Define a function $\deepUnion\in\WS\times\WS\fun\WS$ as follows.  For all
$\alpha,\beta\in\WS$:
\begin{align*}
  \deepUnion(\alpha, \$) &= \alpha , \\
  \deepUnion(\$, \alpha) &= \alpha , \eqtxt{if} \alpha\neq\$ , \eqtxt{and} \\
  \deepUnion(\alpha, \beta) &=
    \sortUnions(\rightUnion(\alpha, \beta)), \eqtxt{if} \alpha\neq\$
    \eqtxt{and} \beta\neq\$ .
\end{align*}
To see that the last clause of this definition is proper, suppose
$\alpha,\beta\in\WS-\{\$\}$.  Then all the elements of
$\unionsToList(\rightUnion(\alpha,\beta))$ will be weakly simplified,
and won't be $\$$ or unions.  Consequently, $\sortUnions$ will
deliver a weakly simplified result.

\begin{lemma}
\label{DeepUnionLem}
For all $\alpha,\beta\in\WS$, $\deepUnion(\alpha,\beta)$
is equivalent to $\alpha+\beta$, has an alphabet that is a
subset of the alphabet of $\alpha+\beta$, has a closure complexity
that is no bigger than that of $\alpha+\beta$, has a size that is no
bigger than that of $\alpha+\beta$, has no more concatenations than
$\alpha+\beta$, and has no more symbols than $\alpha+\beta$.
\end{lemma}

Now, we can define our weak simplification function/algorithm.
\index{regular expression!weaklySimplify@$\weaklySimplify$}%
Define $\weaklySimplify\in\Reg\fun\WS$ by structural recursion:
\begin{itemize}
\item $\weaklySimplify\,\% = \%$;

\item $\weaklySimplify\,\$ = \$$;

\item $\weaklySimplify\,a = a$, for all $a\in\Sym$;

\item $\weaklySimplify({*}(\alpha))$
  \begin{displaymath}
    {} = \deepClosure(\weaklySimplify\,\alpha) ,
  \end{displaymath}
  for all $\alpha\in\Reg$;

\item $\weaklySimplify(@(\alpha,\beta))$
  \begin{displaymath}
    {} = \deepConcat(\weaklySimplify\,\alpha, \weaklySimplify\,\beta) ,
  \end{displaymath}
  for all $\alpha,\beta\in\Reg$; and

\item $\weaklySimplify({+}(\alpha,\beta))$
  \begin{displaymath}
    {} = \deepUnion(\weaklySimplify\,\alpha, \weaklySimplify\,\beta) ,
  \end{displaymath}
  for all $\alpha,\beta\in\Reg$.
\end{itemize}

\begin{proposition}
\label{WeakSimpProp1}
For all $\alpha\in\Reg$:
\begin{enumerate}[\quad(1)]
\item $\weaklySimplify\,\alpha\approx\alpha$;

\item $\alphabet(\weaklySimplify\,\alpha)\sub\alphabet\,\alpha$;

\item $\cc(\weaklySimplify\,\alpha)\leqcc\cc\,\alpha$;

\item $\mysize(\weaklySimplify\,\alpha)\leq\mysize\,\alpha$;

\item $\numSyms(\weaklySimplify\,\alpha)\leq\numSyms\,\alpha$; and

\item $\numConcats(\weaklySimplify\,\alpha)\leq\numConcats\,\alpha$.
\end{enumerate}
\end{proposition}

\begin{proof}
By induction on regular expressions.
\end{proof}

\begin{exercise}
\label{WeakSimpExercise}
Prove Proposition~\ref{WeakSimpProp1}.
\end{exercise}

\begin{corollary}
\label{WeakSimpProp1Cor}
For all regular expressions $\alpha$,
$\weaklySimplify\,\alpha \leqsimp \alpha$.
\end{corollary}

\begin{proof}
Follows from Proposition~\ref{WeakSimpProp1} and the fact
that weakly simplified regular expressions are standardized.
\end{proof}

\begin{proposition}
\label{WeakSimpProp5}
For all $\alpha\in\Reg$, if $\alpha$ is weakly simplified, then
$\weaklySimplify\,\alpha = \alpha$.
\end{proposition}

\begin{proof}
By induction on regular expressions.
\end{proof}

Using our weak simplification algorithm, we can define an algorithm
for calculating the language generated by a regular expression, when
this language is finite, and for announcing that this language is
infinite, otherwise.  First, we weakly simplify our regular
expression, $\alpha$, and call the resulting regular expression
$\beta$.  If $\beta$ contains no closures, then we compute its meaning
in the usual way.  But, if $\beta$ contains one or more closures, then
its language will be infinite, and thus we can output a message saying
that $L(\alpha)$ is infinite.

The Forlan module \texttt{Reg} defines the following functions relating
to weak simplification:
\begin{verbatim}
val weaklySimplified : reg -> bool
val weaklySimplify   : reg -> reg
val toStrSet         : reg -> str set
\end{verbatim}
\index{Reg@\texttt{Reg}!weaklySimplified@\texttt{weaklySimplified}}%
\index{Reg@\texttt{Reg}!weaklySimplify@\texttt{weaklySimplify}}%
\index{Reg@\texttt{Reg}!toStrSet@\texttt{toStrSet}}%
\texttt{weaklySimplified} and \texttt{weaklySimplify}
implement $\weaklySimplified$ and $\weaklySimplify$,
respectively. Finally, the function \texttt{toStrSet} implements our
algorithm for calculating the language generated by a regular
expression, if that language is finite, and for announcing the this
language is infinite, otherwise.

Here are some examples of how these functions can be used:
\begin{list}{}
{\setlength{\leftmargin}{\leftmargini}
\setlength{\rightmargin}{0cm}
\setlength{\itemindent}{0cm}
\setlength{\listparindent}{0cm}
\setlength{\itemsep}{0cm}
\setlength{\parsep}{0cm}
\setlength{\labelsep}{0cm}
\setlength{\labelwidth}{0cm}
\catcode`\#=12
\catcode`\$=12
\catcode`\%=12
\catcode`\^=12
\catcode`\_=12
\catcode`\.=12
\catcode`\?=12
\catcode`\!=12
\catcode`\&=12
\ttfamily}
\small
\item[]\textsl{-\ }val\ reg\ =\ Reg.input\ "";
\item[]\textsl{@\ }(%\ +\ $0)(%\ +\ 00\symbol{'052}0\ +\ 0\symbol{'052}\symbol{'052})\symbol{'052}
\item[]\textsl{@\ }.
\item[]\textsl{val\ reg\ =\ -\ :\ reg}
\item[]\textsl{-\ }Reg.output("",\ Reg.weaklySimplify\ reg);
\item[]\textsl{(%\ +\ 0\symbol{'052}\ +\ 000\symbol{'052})\symbol{'052}}
\item[]\textsl{val\ it\ =\ ()\ :\ unit}
\item[]\textsl{-\ }Reg.toStrSet\ reg;
\item[]\textsl{language\ is\ infinite}
\item[]
\item[]\textsl{uncaught\ exception\ Error}
\item[]\textsl{-\ }val\ reg'\ =\ Reg.input\ "";
\item[]\textsl{@\ }(1\ +\ %)(2\ +\ $)(3\ +\ %\symbol{'052})(4\ +\ $\symbol{'052})
\item[]\textsl{@\ }.
\item[]\textsl{val\ reg'\ =\ -\ :\ reg}
\item[]\textsl{-\ }StrSet.output("",\ Reg.toStrSet\ reg');
\item[]\textsl{2,\ 12,\ 23,\ 24,\ 123,\ 124,\ 234,\ 1234}
\item[]\textsl{val\ it\ =\ ()\ :\ unit}
\item[]\textsl{-\ }Reg.output("",\ Reg.weaklySimplify\ reg');
\item[]\textsl{(%\ +\ 1)2(%\ +\ 3)(%\ +\ 4)}
\item[]\textsl{val\ it\ =\ ()\ :\ unit}
\item[]\textsl{-\ }Reg.output
\item[]\textsl{=\ }("",
\item[]\textsl{=\ }\ Reg.weaklySimplify(Reg.fromString\ "(00\symbol{'052}11\symbol{'052})\symbol{'052}"));
\item[]\textsl{(00\symbol{'052}11\symbol{'052})\symbol{'052}}
\item[]\textsl{val\ it\ =\ ()\ :\ unit}
\end{list}


\subsection{Local and Global Simplification}

In preparation for the definition of our local and global simplification
algorithms, we must define some auxiliary functions.
First, we show how we can recursively test whether $\%\in L(\alpha)$, for
a regular expression $\alpha$.  We define a function
\index{hasEmp@$\hasEmp$}%
\index{regular expression!hasEmp@$\hasEmp$}%
\index{regular expression!testing for membership of empty string}%
\begin{gather*}
\hasEmp\in\Reg\fun\Bool
\end{gather*}
 by recursion:
\begin{align*}
\hasEmp\,\% &= \true ; \\
\hasEmp\,\$ &= \false ; \\
\hasEmp\,a &= \false, \eqtxt{for all}a\in\Sym ; \\
\hasEmp(\alpha^*) &= \true , \eqtxt{for all} \alpha\in\Reg ; \\
\hasEmp(\alpha\beta) &=
\hasEmp\,\alpha\myand\hasEmp\,\beta , \eqtxt{for all} \alpha,\beta\in\Reg ;
  \eqtxt{and} \\
\hasEmp(\alpha+\beta) &=
\hasEmp\,\alpha\myor\hasEmp\,\beta , \eqtxt{for all} \alpha,\beta\in\Reg .
\end{align*}

\begin{proposition}
\label{HasEmpProp}
For all $\alpha\in\Reg$, $\%\in L(\alpha)$ iff $\hasEmp\,\alpha =
\true$.
\end{proposition}

When interpreting a function (like $\hasEmp$) as an algorithm, we
regard uses of $\myand$ and $\myor$ as ``short circuiting''. I.e., if
\index{short circuiting}%
\index{and@$\myand$}%
\index{or@$\myor$}%
the left argument to $\myand$ evaluates to $\false$, it doesn't bother
to evaluate the right argument, as the overall result will be
$\false$; and if the left argument to $\myor$ evaluates to $\true$, it
doesn't bother to evaluate the right argument, as the overall result
will be $\true$.

\begin{proof}
By induction on regular expressions.
\end{proof}

Next, we show how we can recursively test whether $a\in L(\alpha)$, for
a symbol $a$ and a regular expression $\alpha$.  We define a function
\index{hasSym@$\hasSym$}%
\index{regular expression!hasSym@$\hasSym$}%
\index{regular expression!testing for membership of symbol}%
\begin{gather*}
\hasSym\in\Sym\times\Reg\fun\Bool
\end{gather*}
 by recursion:
\begin{align*}
\hasSym(a, \%) &= \false , \eqtxt{for all} a\in\Sym; \\
\hasSym(a, \$) &= \false , \eqtxt{for all} a\in\Sym; \\
\hasSym(a, b) &= a = b , \eqtxt{for all} a,b\in\Sym; \\
\hasSym(a, \alpha^*) &= \hasSym(a, \alpha) , \eqtxt{for all}
a\in\Sym\eqtxt{and}\alpha\in\Reg ; \\
\hasSym(a, \alpha\beta) &=
(\hasSym(a, \alpha)\myand \hasEmp(\beta)) \myor {} \\
&\quad\;\, (\hasEmp(\alpha)\myand \hasSym(a, \beta)) , \\
&\quad\, \eqtxt{for all} a\in\Sym\eqtxt{and}\alpha,\beta\in\Reg ; \eqtxt{and} \\
\hasSym(a, \alpha+\beta) &=
\hasSym(a, \alpha)\myor\hasSym(a, \beta), \\
&\quad\, \eqtxt{for all} a\in\Sym\eqtxt{and}\alpha,\beta\in\Reg .
\end{align*}

\begin{proposition}
\label{HasSymProp}
For all $a\in\Sym$ and $\alpha\in\Reg$,
$a\in L(\alpha)$ iff $\hasSym(a,\alpha) = \true$.
\end{proposition}

\begin{proof}
By induction on regular expressions, using Proposition~\ref{HasEmpProp}.
\end{proof}

Finally, we define a function/algorithm
\index{obviousSubset@$\obviousSubset$}%
\index{regular expression!obviousSubset@$\obviousSubset$}%
\index{regular expression!conservative subset test}%
\begin{displaymath}
\obviousSubset\in\Reg\times\Reg\fun\Bool
\end{displaymath}
meeting the following specification: for all $\alpha,\beta\in\Reg$,
\begin{displaymath}
\eqtxtr{if} \obviousSubset(\alpha,\beta)=\true,
\eqtxt{then} L(\alpha)\sub L(\beta) .
\end{displaymath}
I.e., this function is a \emph{conservative approximation to subset testing}.
\index{regular expression!conservative approximation to subset testing}%
\index{conservative approximation to subset testing}%
The function that always returns $\false$
would meet this specification, but our function will do much better
than this, and will be reasonably efficient. In
Section~\ref{EquivalenceTestingAndMinimizationOfDFAs},
we will learn of a less efficient algorithm that will provide a complete test
for $L(\alpha)\sub L(\beta)$.

Given $\alpha,\beta\in\Reg$, $\obviousSubset(\alpha,\beta)$ proceeds
as follows.  First, it lets $\alpha'=\weaklySimplify\,\alpha$ and
$\beta'=\weaklySimplify\,\beta$.  Then it returns
$\obviSub(\alpha',\beta')$, where
\begin{displaymath}
\obviSub\in\WS\times\WS\fun\Bool
\end{displaymath}
is the function defined below.

$\obviSub$ is defined by well-founded recursion on the sum of the
sizes of its arguments.  If $\alpha=\beta$, then it returns $\true$;
otherwise, it considers the possible forms of $\alpha$.
\begin{itemize}
\item Suppose $\alpha = \%$. It returns $\hasEmp\,\beta$.

\item Suppose $\alpha = \$$. It returns $\true$.

\item Suppose $\alpha = a$, for some $a\in\Sym$. It returns $\hasSym(a,
  \beta)$.

\item Suppose $\alpha = {\alpha_1}^*$, for some $\alpha_1\in\Reg$.
Here it looks at the form of $\beta$.
\begin{itemize}
\item Suppose $\beta = \%$. It returns $\false$.  (Because
$\alpha$ will be weakly simplified, and so $\alpha$ won't generate
$\{\%\}$.)

\item Suppose $\beta = \$$. It returns $\false$.

\item Suppose $\beta = a$, for some $a\in\Sym$. It returns $\false$.

\item Suppose $\beta$ is a closure.  It returns $\obviSub(\alpha_1, \beta)$.

\item Suppose $\beta=\beta_1\beta_2$, for some $\beta_1,\beta_2\in\Reg$.
If $\hasEmp\,\beta_1=\true$ and $\obviSub(\alpha,\beta_2)$,
then it returns $\true$.
Otherwise, if $\hasEmp\,\beta_2=\true$ and $\obviSub(\alpha,\beta_1)$,
then it returns $\true$.
Otherwise, it returns $\false$ (even though the answer sometimes
should be $\true$).

\item Suppose $\beta = \beta_1 + \beta_2$, for some $\beta_1,\beta_2\in\Reg$.
It returns
\begin{gather*}
\obviSub(\alpha, \beta_1) \myor \obviSub(\alpha, \beta_2)
\end{gather*}
(even though this is $\false$ too often).
\end{itemize}

\item Suppose $\alpha = \alpha_1\alpha_2$, for some $\alpha_1,\alpha_2\in\Reg$.
Here it looks at the form of $\beta$.
\begin{itemize}
\item Suppose $\beta = \%$.  It returns $\false$.  (Because $\alpha$
  is weakly simplified, $\alpha$ won't generate $\{\%\}$.)

\item Suppose $\beta = \$$. It returns $\false$.  (Because $\alpha$ is
  weakly simplified, $\alpha$ won't generate $\emptyset$.)

\item Suppose $\beta = a$, for some $a\in\Sym$. It returns $\false$.
  (Because $\alpha$ is weakly simplified, $\alpha$ won't generate
  $\{a\}$.)

\item Suppose $\beta={\beta_1}^*$, for some $\beta_1\in\Reg$. It returns
\begin{gather*}
\obviSub(\alpha, \beta_1) \\
\myor \\
(\obviSub(\alpha_1, \beta) \myand \obviSub(\alpha_2, \beta))
\end{gather*}
(even though this returns $\false$ too often).

\item Suppose $\beta = \beta_1\beta_2$, for some $\beta_1,\beta_2\in\Reg$.
If $\obviSub(\alpha_1, \beta_1)=\true$ and $\obviSub(\alpha_2,
\beta_2)=\true$, then it returns $\true$.
Otherwise, if $\hasEmp\,\beta_1=\true$ and $\obviSub(\alpha,\beta_2)=\true$,
then it returns $\true$.
Otherwise, if $\hasEmp\,\beta_2=\true$ and $\obviSub(\alpha,\beta_1)=\true$,
then it returns $\true$.
Otherwise, if $\beta_1$ is a closure but $\beta_2$ is not a closure,
then it returns
\begin{gather*}
\obviSub(\alpha_1, \beta_1) \myand \obviSub(\alpha_2, \beta)
\end{gather*}
(even though this returns $\false$ too often).
Otherwise, if $\beta_2$ is a closure but $\beta_1$ is not a closure,
then it returns
\begin{gather*}
\obviSub(\alpha_1, \beta) \myand \obviSub(\alpha_2, \beta_2)
\end{gather*}
(even though this returns $\false$ too often).
Otherwise, if $\beta_1$ and $\beta_2$ are closures, then it
returns
\begin{gather*}
(\obviSub(\alpha_1, \beta_1) \myand \obviSub(\alpha_2, \beta)) \\
\myor \\
(\obviSub(\alpha_1, \beta) \myand \obviSub(\alpha_2, \beta_2))
\end{gather*}
(even though this returns $\false$ too often).
Otherwise, it returns $\false$, even though sometimes we would like
the answer to be $\true$).

\item Suppose $\beta = \beta_1 + \beta_2$, for some $\beta_1,\beta_2\in\Reg$.
It returns
\begin{gather*}
\obviSub(\alpha, \beta_1) \myor \obviSub(\alpha, \beta_2)
\end{gather*}
(even though this is $\false$ too often).
\end{itemize}

\item Suppose $\alpha=\alpha_1+\alpha_2$. It returns
\begin{gather*}
\obviSub(\alpha_1, \beta) \myand \obviSub(\alpha_2, \beta) .
\end{gather*}
\end{itemize}

We say that $\alpha$ is \emph{obviously a subset of} $\beta$ iff
$\obviousSubset(\alpha,\beta)=\true$. On the positive side, we have
that, e.g.,
$\obviousSubset(\mathsf{0^*011^*1},\mathsf{0^*1^*})=\true$.  On the
other hand, $\obviousSubset(\mathsf{(01)^*},\mathsf{(\% + 0)(10)^*(\%
  + 1)})=\false$, even though $L(\mathsf{(01)^*})\sub L(\mathsf{(\% +
  0)(10)^*(\% + 1)})$.

\begin{proposition}
\label{WeakSubProp}
For all $\alpha,\beta\in\Reg$, if $\obviousSubset(\alpha,\beta)=\true$,
then $L(\alpha)\sub L(\beta)$.
\end{proposition}

\begin{proof}
First, we use induction on the sum of the sizes of $\alpha$ and
$\beta$ to show that, for all $\alpha,\beta\in\Reg$, if
$\obviSub(\alpha,\beta)=\true$, then $L(\alpha)\sub L(\beta)$.
The result then follows by Proposition~\ref{WeakSimpProp1}.
\end{proof}

The Forlan module \texttt{Reg} provides the following functions
corresponding to the auxiliary functions $\hasEmp$, $\hasSym$ and
$\obviousSubset$:
\begin{verbatim}
val hasEmp        : reg -> bool
val hasSym        : sym * reg -> bool
val obviousSubset : reg * reg -> bool
\end{verbatim}
\index{Reg@\texttt{Reg}!hasEmp@\texttt{hasEmp}}%
\index{Reg@\texttt{Reg}!hasSym@\texttt{hasSym}}%
\index{Reg@\texttt{Reg}!obviousSubset@\texttt{obviousSubset}}%
Here are some examples of how they can be used:
\begin{list}{}
{\setlength{\leftmargin}{\leftmargini}
\setlength{\rightmargin}{0cm}
\setlength{\itemindent}{0cm}
\setlength{\listparindent}{0cm}
\setlength{\itemsep}{0cm}
\setlength{\parsep}{0cm}
\setlength{\labelsep}{0cm}
\setlength{\labelwidth}{0cm}
\catcode`\#=12
\catcode`\$=12
\catcode`\%=12
\catcode`\^=12
\catcode`\_=12
\catcode`\.=12
\catcode`\?=12
\catcode`\!=12
\catcode`\&=12
\ttfamily}
\small
\item[]\textsl{-\ }Reg.hasEmp(Reg.fromString\ "0\symbol{'052}1\symbol{'052}");
\item[]\textsl{val\ it\ =\ true\ :\ bool}
\item[]\textsl{-\ }Reg.hasEmp(Reg.fromString\ "01\symbol{'052}");
\item[]\textsl{val\ it\ =\ false\ :\ bool}
\item[]\textsl{-\ }Reg.hasSym(Sym.fromString\ "0",\ Reg.fromString\ "0\symbol{'052}1\symbol{'052}");
\item[]\textsl{val\ it\ =\ true\ :\ bool}
\item[]\textsl{-\ }Reg.hasSym(Sym.fromString\ "1",\ Reg.fromString\ "0\symbol{'052}1\symbol{'052}");
\item[]\textsl{val\ it\ =\ true\ :\ bool}
\item[]\textsl{-\ }Reg.hasSym(Sym.fromString\ "0",\ Reg.fromString\ "0\symbol{'052}$");
\item[]\textsl{val\ it\ =\ false\ :\ bool}
\item[]\textsl{-\ }Reg.obviousSubset
\item[]\textsl{=\ }(Reg.fromString\ "(0\ +\ 1)\symbol{'052}",
\item[]\textsl{=\ }\ Reg.fromString\ "0\symbol{'052}(0\ +\ 1)\symbol{'052}1\symbol{'052}");
\item[]\textsl{val\ it\ =\ true\ :\ bool}
\item[]\textsl{-\ }Reg.obviousSubset
\item[]\textsl{=\ }(Reg.fromString\ "0\symbol{'052}(0\ +\ 1)\symbol{'052}1\symbol{'052}",
\item[]\textsl{=\ }\ Reg.fromString\ "(0\ +\ 1)\symbol{'052}");
\item[]\textsl{val\ it\ =\ true\ :\ bool}
\item[]\textsl{-\ }Reg.obviousSubset
\item[]\textsl{=\ }(Reg.fromString\ "0\symbol{'052}011\symbol{'052}1",
\item[]\textsl{=\ }\ Reg.fromString\ "0\symbol{'052}1\symbol{'052}");
\item[]\textsl{val\ it\ =\ true\ :\ bool}
\item[]\textsl{-\ }Reg.obviousSubset
\item[]\textsl{=\ }(Reg.fromString\ "(01\ +\ 011)1\symbol{'052}",
\item[]\textsl{=\ }\ Reg.fromString\ "01\symbol{'052}");
\item[]\textsl{val\ it\ =\ true\ :\ bool}
\item[]\textsl{-\ }Reg.obviousSubset
\item[]\textsl{=\ }(Reg.fromString\ "(01)\symbol{'052}",
\item[]\textsl{=\ }\ Reg.fromString\ "(%\ +\ 0)(10)\symbol{'052}(%\ +\ 1)");
\item[]\textsl{val\ it\ =\ false\ :\ bool}
\end{list}


Our local and global simplification algorithms make use of
simplification rules, which may be applied to arbitrary subtrees of
\index{regular expression!simplification rule}%
\index{regular expression!rule}%
regular expressions. A \emph{simplification rule} (or \emph{rule}, for
short) is a pair $(\alpha, \beta)$, where $\alpha$ and $\beta$ are
like regular expressions except that some of their leaves may be
mathematical variables, and where the variables of $\beta$ are a
subset of the variables of $\alpha$. We typically write $(\alpha, \beta)$
as $\alpha\fun\beta$. For example,
\begin{gather*}
\alpha + \beta\fun \beta+\alpha , \\
\alpha_1\beta + \alpha_2\beta \fun (\alpha_1+\alpha_2)\beta
\end{gather*}
are rules, where the variables of the first rule are $\{\alpha,\beta\}$,
and the variables of the second rule are $\{\alpha_1, \alpha_2, \beta\}$.
But $\%\fun\alpha$ is not a rule.

Suppose $\alpha\fun\beta$ is a rule, where the variables of $\alpha$
are $x_1,\ldots,x_n$.  Suppose $\gamma$ is a regular expression,
$\pat\in\Path$ is valid for $\alpha$, and $\delta$ is the subtree of
$\alpha$ at position $\pat$.  If there are (necessarily unique)
$\lambda_i\in\Reg$, for all $1\leq i\leq n$, such that the result of
substituting the $\lambda_i$ for the $x_i$ in $\alpha$ gives us
$\delta$, then we can \emph{apply} $\alpha\fun\beta$ \emph{to}
\index{simplification rule application}%
\index{regular expression!simplification rule application}%
$\gamma$ \emph{at position} $\pat$ (and we say that the rule is
\emph{applicable to} $\gamma$ at this position), \emph{yielding} the
result of replacing the subtree at position $\pat$ in $\gamma$ by the
result of substituting the $\lambda_i$ for the $x_i$ in $\beta$.  For
example, applying $\alpha+\beta\fun\beta+\alpha$ at position $[1]$ in
$(\zerosf + \onesf)\twosf$ yields $(\onesf + \zerosf)\twosf$.

Some of our rules are \emph{conditional} in that they have
\emph{preconditions} involving their variables. Such a rule may only be
applied if its precondition holds for the instantiation of the variables.
E.g, the rule
\begin{quotation}
  If $\hasEmp\,\alpha$ and $\hasEmp\,\beta$, then
  $(\alpha\beta)^* \fun (\alpha+\beta)^*$.
\end{quotation}
may only be applied to a subtree of the form $(\alpha\beta)^*$ when
both $L(\alpha)$ and $L(\beta)$ include $\%$.

All of our rules will preserve the meaning (language generated by) a
regular expression. If $\alpha\fun\beta$ is a rule (or the conclusion
of a rule in the case of a conditional rule), and we substitute
regular expressions $\lambda_i$ for all the variables $x_i$ of
$\alpha$ (and thus $\beta$), yielding $\alpha'$ and $\beta'$, then,
assuming the precondition holds for the $\lambda_i$ in the case of a
conditional rule, we will have that
$\alpha'\approx\beta'$. Consequently, by Proposition~\ref{RegContext},
we will have that applying a rule to some position of a regular
expression gives us a regular expression with the same meaning.

There are two kinds of rules: structural rules and reduction rules.
When defining global simplification, we'll also treat weak
simplification as a rule, allowing the subtree of a regular expression
at some position to be replaced by its weak simplification.  We begin
by considering the structural rules.

There are nine \emph{structural rules},
\index{structural rule}%
\index{regular expression!structural rule}%
\index{simplification!regular expression!structural rule}%
which preserve the alphabet, closure complexity, size, number of
concatenations and number of symbols of a regular expression:
\begin{enumerate}[\quad(1)]
\item $(\alpha + \beta) + \gamma \fun \alpha + (\beta + \gamma)$.

\item $\alpha + (\beta + \gamma) \fun (\alpha + \beta) + \gamma$.

\item $\alpha(\beta\gamma) \fun (\alpha\beta)\gamma$.

\item $(\alpha\beta)\gamma \fun \alpha(\beta\gamma)$.

\item $\alpha + \beta \fun \beta + \alpha$.

\item $\alpha^*\alpha \fun \alpha\alpha^*$.

\item $\alpha\alpha^* \fun \alpha^*\alpha$.

\item $\alpha(\beta\alpha)^* \fun (\alpha\beta)^*\alpha$.

\item $(\alpha\beta)^*\alpha \fun \alpha(\beta\alpha)^*$.
\end{enumerate}
E.g., if $\alpha\fun\beta$ is one of these rules, then if we
substitute regular expressions $\lambda_i$ for the variables $x_i$ of
$\alpha$ (and thus $\beta$), yielding $\alpha'$ and $\beta'$, we will
have that $\alphabet\,\alpha'\sub\alphabet\,\beta'$.

By Proposition~\ref{RegCCEquivSubstituteSubtree}, and because of how
the alphabet, size, numbers of concatenations, and numbers of symbols
of a regular expression are defined by structural recursion, if we
apply a structural rule to a subtree of a regular expression, this
will preserve the alphabet, closure complexity, size, number of
concatenations and number of symbols of the regular expression
(see Proposition~\ref{RegAlphabetSubtreeSubstitute}).

For even small regular expressions, there may be a very large number
of ways to reorganize them using the structural rules. E.g., suppose
$\alpha_1,\ldots,\alpha_n$ are distinct regular expressions, for
$n\geq 1$. There are $n!$ ways of ordering the $\alpha_i$.  And there
are $(2n)! / (n!)(n+1)!$ (these are the Catalan numbers) binary trees
with exactly $n$ leaves. Consequently, using structural rules (1), (2)
and (5) (without making changes inside the $\alpha_i$), we can
reorganize $\alpha_1+\cdots+\alpha_n$ into $(n!)(2n)! / (n!)(n+1)!$
regular expressions. For $n=16$, this is about $7*10^{20}$.

Because the structural rules preserve the size and alphabet of regular
expressions, it turns out that there are only finitely many regular
expressions that we can transform $\alpha$ into using structural
rules. To see this, suppose $\alpha$ is a regular expression of size
$n$ and with alphabet $\Sigma$. By
Exercise~\ref{RegExactExpOverAlphabetFinite}),
$X = \setof{\alpha\in\Reg}{\alphabet\,\alpha\sub\Sigma \eqtxt{and}
  \size\,\alpha = n}$ is finite.  But all of the structural
reorganizations of $\alpha$ will be elements of $X$.

We can generate $X$ by a sequence of \emph{stages}.  At stage $0$, we
have $\{\alpha\}$. At some stage $n+1$, we consider, in some order,
the regular expressions that we added at stage $n$ (i.e., that were
not added at any earlier stage). For each of these regular
expressions, $\beta$, we add all the regular expressions $\gamma$ that
can be formed by applying a structural rule to one of the subtrees of
$\beta$, subject to the restriction that $\gamma$ has not already been
added at a previous stage or earlier in this stage.  Eventually, there
will be no more regular expressions to add, because $X$ is finite.

There are $26$ \emph{reduction rules}, some of which make use of a
\index{reduction rule}%
\index{regular expression!reduction rule}%
\index{simplification!regular expression!reduction rule}%
conservative approximation $\Sub$ to subset testing. For all of these
rules $\alpha\fun\beta$, if we substitute regular expressions
$\lambda_i$ for the variables $x_i$ of $\alpha$ (and thus $\beta$),
yielding $\alpha'$ and $\beta'$, we will have that
$\alphabet\,\alpha'\sub\alphabet\,\beta'$ and $\beta'\simp\alpha'$,
where $\simp$ is the well-founded relation on $\Reg$ that is defined
below.

We define the relation $\simp$ on $\Reg$ by: for all $\alpha,\beta\in
\Reg$, $\alpha\simp\beta$ iff either:
\begin{itemize}
\item $\cc\,\alpha\ltcc\cc\,\beta$; or

\item $\cc\,\alpha = \cc\,\beta$, but $\mysize\,\alpha < \mysize\,\beta$; or

\item $\cc\,\alpha = \cc\,\beta$ and $\mysize\,\alpha = \mysize\,\beta$,
  but $\numConcats\,\alpha < \numConcats\,\beta$; or

\item $\cc\,\alpha = \cc\,\beta$ and $\mysize\,\alpha =
  \mysize\,\beta$, and $\numConcats\,\alpha = \numConcats\,\beta$, but
  $\numSyms\,\alpha < \numSyms\,\beta$.
\end{itemize}
Note that this is almost the same definition as that of
$\ltsimp$---the difference being that $\simp$ doesn't have the final
step involving standardization. Consequently, ${\simp}\sub{\ltsimp}$.

\begin{proposition}
\label{SimpWF}
$\simp$ is a well-founded relation on $\Reg$.
\end{proposition}

\begin{proof}
Define $f\in\Reg\fun \CC\times\nats\times\nats\times\nats$ by, for all
$\alpha\in\Reg$,
$f\,\alpha = (\cc\,\alpha,\mysize\,\alpha,
\numConcats\,\alpha,\numSyms\,\alpha)$. Then
${\ltcc} \lex {<} \lex {<} \lex {<}$ is a well-founded relation on
$\CC\times\nats\times\nats\times\nats$, by
Propositions~\ref{LTCCWellFounded} and \ref{LexWellFounded}, plus the
fact that $<$ is well-founded on $\nats$
(Proposition~\ref{NonemptyLeastProp}).  Hence
$f^{-1}({\ltcc} \lex {<} \lex {<} \lex {<})$ is a well-founded
relation on $\Reg$, by Proposition~\ref{InverseImageWellFounded}.
Furthermore, this relation is equal to $\simp$, completing the proof.
\end{proof}

\begin{proposition}
\label{SimpTrans}
$\simp$ is transitive.
\end{proposition}

\begin{proposition}
\label{SimpTransLEQSimp}
Suppose $\alpha, \beta, \gamma\in\Reg$.

\begin{enumerate}[\quad(1)]
\item If $\alpha$ and $\beta$ have the same closure complexity, size,
  numbers of concatenations and numbers of symbols, and $\beta\simp\gamma$,
  then $\alpha\simp\gamma$.

\item If $\alpha\simp\beta$, and $\beta$ and $\gamma$ have the same
  closure complexity, size, numbers of concatenations and numbers of
  symbols, then $\alpha\simp\gamma$.

\item If $\alpha\leqsimp\beta\simp\gamma$, then $\alpha\simp\gamma$.
  
\item If $\alpha\simp\beta\leqsimp\gamma$, then $\alpha\simp\gamma$.
\end{enumerate}
\end{proposition}

The following proposition says that if we replace a subtree of a
regular expression by a regular expression that is a
$\simp$-predecessor of that subtree, that the resulting, whole regular
expression will be a $\simp$-predecessor of the original, whole
regular expression.

\begin{proposition}
\label{RegSimpSubtreeSubstitute}

Suppose $\alpha,\beta,\beta'\in\Reg$, $\beta'\simp\beta$,
$\pat\in\Path$ is valid for $\alpha$, and $\beta$ is
the subtree of $\alpha$ at position $\pat$.
Let $\alpha'$ be the result of replacing the subtree at
position $\pat$ in $\alpha$ by $\beta'$. Then $\alpha'\simp\alpha$.
\end{proposition}

Thus if we apply a reduction rule $\alpha\fun\beta$ to some position
of a regular expression $\gamma$, yielding $\gamma'$, we will have
that $\alphabet\,\gamma'\sub\alphabet\,\gamma$ and $\gamma'\simp\gamma$.

Our reduction rules follow.  In the rules, we abbreviate
$\hasEmp\,\alpha=\true$ and $\Sub(\alpha,\beta)=\true$ to
$\hasEmp\,\alpha$ and $\Sub(\alpha,\beta)$, respectively.  Most of
the rules strictly decrease a regular expression's closure complexity
and size.  The exceptions are labeled ``cc'' (for when the closure
complexity strictly decreases, but the size strictly increases),
``concatenations'' (for when the closure complexity and size are
preserved, but the number of concatenations strictly decreases) or
``symbols'' (for when the closure complexity and size normally strictly
decrease, but occasionally they and the number of concatenations stay
they same, but the number of symbols strictly decreases).

\begin{enumerate}[\quad(1)]
%1
\item If $\Sub(\alpha,\beta)$, then $\alpha + \beta \fun \beta$.

%2
\item $\alpha\beta_1 + \alpha\beta_2 \fun \alpha(\beta_1+\beta_2)$.

%3
\item $\alpha_1\beta + \alpha_2\beta \fun (\alpha_1+\alpha_2)\beta$.

%4
\item If $\hasEmp\,\alpha$ and $\Sub(\alpha,\beta^*)$, then
  $\alpha\beta^* \fun \beta^*$.

%5
\item If $\hasEmp\,\beta$ and $\Sub(\beta,\alpha^*)$, then
  $\alpha^*\beta \fun \alpha^*$.

%6
\item If $\Sub(\alpha,\beta^*)$, then $(\alpha+\beta)^* \fun \beta^*$.

%7
\item $(\alpha^* + \beta)^* \fun (\alpha+\beta)^*$.

%8
\item (concatenations) If $\hasEmp\,\alpha$ and $\hasEmp\,\beta$, then
  $(\alpha\beta)^* \fun (\alpha+\beta)^*$.

%9
\item (concatenations) If $\hasEmp\,\alpha$ and $\hasEmp\,\beta$, then
  $(\alpha\beta + \gamma)^* \fun (\alpha+\beta+\gamma)^*$.
 
%10
\item If $\hasEmp\,\alpha$ and $\Sub(\alpha,\beta^*)$, then
  $(\alpha\beta)^* \fun \beta^*$.

%11
\item If $\hasEmp\,\beta$ and $\Sub(\beta,\alpha^*)$, then
  $(\alpha\beta)^* \fun \alpha^*$.

%12
\item If $\hasEmp\,\alpha$ and $\Sub(\alpha,(\beta + \gamma)^*)$, then
  $(\alpha\beta + \gamma)^* \fun (\beta + \gamma)^*$.

%13
\item If $\hasEmp\,\beta$ and $\Sub(\beta,(\alpha + \gamma)^*)$, then
  $(\alpha\beta + \gamma)^* \fun (\alpha + \gamma)^*$.

%14
\item (cc) If $\mynot(\hasEmp\,\alpha)$ and $\cc\,\alpha \cup
  \overline{\cc\,\beta} \ltcc \overline{\overline{\cc\,\beta}}$, then
  $(\alpha\beta^*)^* \fun \% + \alpha(\alpha+\beta)^*$.

%15
\item (cc) If $\mynot(\hasEmp\,\beta)$ and $\overline{\cc\,\alpha} \cup
  \cc\,\beta \ltcc \overline{\overline{\cc\,\alpha}}$, then
  $(\alpha^*\beta)^* \fun \% + (\alpha+\beta)^*\beta$.

%16
\item (cc) If $\mynot(\hasEmp\,\alpha)$ or $\mynot(\hasEmp\,\gamma)$, and
  $\cc\,\alpha \cup \overline{\cc\,\beta} \cup \cc\,\gamma \ltcc
  \overline{\overline{\cc,\beta}}$, then $(\alpha\beta^*\gamma)^*
  \fun \% + \alpha(\beta + \gamma\alpha)^*\gamma$.

%17
\item If $\Sub(\alpha\alpha^*,\beta)$, then $\alpha^*+\beta \fun \% +
  \beta$.

%18
\item If $\hasEmp\,\beta$ and $\Sub(\alpha\alpha\alpha^*, \beta)$,
  then $\alpha^* + \beta \fun \alpha + \beta$.

%19
\item (symbols) If $\alpha\not\in\{\%,\$\}$ and $\Sub(\alpha^n, \beta)$, then
  $\alpha^{n+1}\alpha^* + \beta \fun \alpha^n\alpha^* + \beta$.

%20
\item If $n\geq 2$, $l\geq 0$ and $2n - 1 < m_1 < \cdots < m_l$, then
  $(\alpha^n + \alpha^{n+1} + \cdots + \alpha^{2n - 1} + \alpha^{m_1}
  + \cdots + \alpha^{m_l})^* \fun \% + \alpha^n\alpha^*$.

%21
\item (symbols) If $\alpha\not\in\{\%,\$\}$, then $\alpha+\alpha\beta
  \fun \alpha(\%+\beta)$.

%22
\item (symbols) If $\alpha\not\in\{\%,\$\}$, then $\alpha+\beta\alpha
  \fun (\%+\beta)\alpha$.

%23
\item $\alpha^*(\% + \beta(\alpha+\beta)^*) \fun (\alpha+\beta)^*$.

%24
\item $(\% + (\alpha+\beta)^*\alpha)\beta^* \fun (\alpha+\beta)^*$.

%25
\item If $\Sub(\alpha, \beta^*)$ and $\Sub(\beta, \alpha)$, then
   $\% + \alpha\beta^* \fun \beta^*$.

%26
\item If $\Sub(\beta, \alpha^*)$ and $\Sub(\alpha, \beta)$, then
   $\% + \alpha^* \beta \fun \alpha^*$.
\end{enumerate}

Rules (19)--(20) are abbreviations for infinitely many rules,
for all choices of the natural number parameters.

In rules (14)--(16), the preconditions involving $\cc$ are necessary
and sufficient conditions for the right-hand side to have strictly
smaller closure complexity than the left-hand side.

Consider, e.g., reduction rule (4). Suppose $\alpha, \beta\in\Reg$,
$\hasEmp\,\alpha=\true$ and $\Sub(\alpha,\beta^*)=\true$, so that that
$\%\in L(\alpha)$ and $L(\alpha)\sub L(\beta^*)$.  We need that
$\alpha\beta^* \approx \beta^*$,
$\alphabet(\beta^*)\sub\alphabet(\alpha\beta^*)$ and
$\beta^*\simp\alpha\beta^*$.  The alphabet of $\beta^*$ is clearly a
subset of that of $\alpha\beta^*$.

To obtain $\alpha\beta^* \approx \beta^*$, it will suffice to show
that, for all $A,B\in\Lan$, if $\%\in A$ and $A\sub B^*$, then
$AB^*=B^*$.  Suppose $A,B\in\Lan$, $\%\in A$ and $A\sub B^*$.
We show that $AB^*\sub B^*\sub AB^*$.  Suppose $w\in AB^*$, so
that $w=xy$, for some $x\in A$ and $y\in B^*$.  Since $A\sub B^*$,
it follows that $w=xy\in B^*B^*=B^*$.  Suppose $w\in B^*$.
Then $w=\%w\in AB^*$.

And, to see that $\beta^*\ltcc\alpha\beta^*$, it will suffice to
show that $\cc(\beta^*)\ltcc\cc(\alpha\beta^*)$.  And we
have that
\begin{displaymath}
 \cc(\beta^*) = \overline{\cc\,\beta} \ltcc
 \cc\,\alpha \cup \overline{\cc\,\beta} = \cc(\alpha\beta^*) .
\end{displaymath}

\subsection*{Local Simplification}

\index{regular expression!simplification!local|(}%
\index{regular expression!local simplification|(}%
\index{simplification!regular expression!local|(}%

Suppose $\Sub$ is a conservative approximation to subset testing.
\index{regular expression!locally simplified}%
We say that a regular expression $\alpha$ is \emph{locally simplified with
respect to} $\Sub$:
iff
\begin{itemize}
\item $\alpha$ is weakly simplified, and

\item $\alpha$ can't be transformed by our structural rules (which may
  be applied to subtrees) into a regular expression to which one of
  our reduction rules (which may be applied to subtrees) applies.
\end{itemize}

The \emph{local simplification of} a regular expression $\alpha$
\emph{with respect to} a conservative approximation to subset testing
$\Sub$ proceeds as follows.  It calls its main function with the weak
simplification, $\beta$, of $\alpha$. Then $\beta\leqsimp\alpha$,
$\alphabet\,\beta\sub\alphabet\,\alpha$ and $\beta$ is equivalent to
$\alpha$.

The main function is defined by well-founded recursion on $\simp$.  It
works as follows, when called with a weakly simplified argument,
$\alpha$.
\begin{itemize}
\item It generates the (finite) set $X$ of all regular expressions
  $\weaklySimplify\,\gamma$, such that $\alpha$ can be reorganized
  using the structural rules (allowing applications to subtrees) into
  a regular expression $\beta$, which can be transformed by a single
  application of one of our reduction rules (allowing applications to
  subtrees) into $\gamma$.

\item If $X$ is empty, then it returns $\alpha$.

\item Otherwise, it calls itself recursively on the simplest element,
  $\lambda$, of $X$ (when $X$ doesn't have a unique simplest element,
  the smallest of the simplest elements---in our total ordering on
  regular expressions---is selected).  Because (see
  Proposition~\ref{SimpTransLEQSimp})
  \begin{itemize}
  \item the structural rules (even applied to subtrees) preserve
    closure complexity, size, number of concatenations, and number of
    symbols,

  \item the reduction rules (even applied to subtrees) produce
    $\simp$-predecessors, and

  \item and weak simplification respects $\leqsimp$,
  \end{itemize}
  we have that $\lambda\simp\alpha$ (and so $\lambda\leqsimp\alpha$),
  so that this recursive call is legal.  Furthermore, weak
  simplification, and all of the rules, either preserve or decrease
  (via $\sub$) the alphabet of regular expressions.  Thus
  $\alphabet\,\lambda\sub\alphabet\,\alpha$. Finally, $\lambda$ is
  equivalent to $\alpha$, because all the rules and weak simplification
  preserve equivalence.
\end{itemize}

The algorithm is referred to as ``local'', because at each recursive
call of its main function, $\lambda$ is chosen using the best local
knowledge.  This strategy is reasonably efficient, but there is no
guarantee that another local choice wouldn't result in a simpler
global answer.

We define a function/algorithm
\index{regular expression!locallySimplify@$\locallySimplify$}%
\begin{displaymath}
\locallySimplify\in(\Reg\times\Reg\fun\Bool)\fun\Reg\fun\Reg
\end{displaymath}
by: for all conservative approximations to subset testing $\Sub$, and
$\alpha\in\Reg$, $\locallySimplify\,\Sub\,\alpha$ is the result of
running our local simplification algorithm on $\alpha$, using $\Sub$
as the conservative approximation to subset testing.

\begin{theorem}
For all conservative approximations to subset testing $\Sub$,
and $\alpha\in\Reg$:
\begin{itemize}
\item $\locallySimplify\,\Sub\,\alpha$ is locally simplified with
  respect to $\Sub$;

\item $\locallySimplify\,\Sub\,\alpha$ is equivalent to $\alpha$;

\item $\alphabet(\locallySimplify\,\Sub\,\alpha)\sub\alphabet\,\alpha$;
  and

\item $\locallySimplify\,\Sub\,\alpha\leqsimp\alpha$.
\end{itemize}
\end{theorem}

\begin{proof}
By well-founded induction on $\simp$.
\end{proof}

The Forlan module \texttt{Reg} provides the following functions
relating to local simplification:
\begin{verbatim}
val locallySimplified    :
      (reg * reg -> bool) -> reg -> bool
val locallySimplify      :
      int option * (reg * reg -> bool) -> reg -> bool * reg
val locallySimplifyTrace :
      int option * (reg * reg -> bool) -> reg -> bool * reg
\end{verbatim}
\index{Reg@\texttt{Reg}!locallySimplified@\texttt{locallySimplified}}%
\index{Reg@\texttt{Reg}!locallySimplify@\texttt{locallySimplify}}%
\index{Reg@\texttt{Reg}!locallySimplifyTrace@\texttt{locallySimplifyTrace}}%
The function \texttt{locallySimplified} takes in a conservative
approximation to subset testing $\Sub$ and returns a function that
tests whether a regular expression is $\Sub$-locally simplified.  The
function \texttt{locallySimplifyTrace} implements $\locallySimplify$.
It emits tracing messages explaining its operation, takes in an extra
argument of type \texttt{int~option}, and produces an extra result of
type \texttt{bool}.  If this extra argument is \texttt{NONE}, then it
runs as does $\locallySimplify$, and its boolean result is always
$\true$.  But if it is $\mathtt{SOME}\;n$, for $n\geq 1$, then at each
recursive call of the algorithm's function, no more than $n$ ways of
reorganizing the function's argument will be considered (they are
considered in order, according to the number of applications of
structural rules needed to produce them), and the boolean part of the
result will be $\false$ iff, in the final recursive call, $n$ was not
sufficient to explore all structural reorganizations, so that the
regular expression returned may not be locally simplified with respect
to $\Sub$.  The function \texttt{locallySimplify} works identically,
except it doesn't issue tracing messages.

Here are some examples of how these functions can be used.
\begin{list}{}
{\setlength{\leftmargin}{\leftmargini}
\setlength{\rightmargin}{0cm}
\setlength{\itemindent}{0cm}
\setlength{\listparindent}{0cm}
\setlength{\itemsep}{0cm}
\setlength{\parsep}{0cm}
\setlength{\labelsep}{0cm}
\setlength{\labelwidth}{0cm}
\catcode`\#=12
\catcode`\$=12
\catcode`\%=12
\catcode`\^=12
\catcode`\_=12
\catcode`\.=12
\catcode`\?=12
\catcode`\!=12
\catcode`\&=12
\ttfamily}
\small
\item[]\textsl{-\ }val\ locSimped\ =\ Reg.locallySimplified\ Reg.obviousSubset;
\item[]\textsl{val\ locSimped\ =\ fn\ :\ reg\ ->\ bool}
\item[]\textsl{-\ }locSimped(Reg.fromString\ "(1\ +\ 00\symbol{'052}1)\symbol{'052}00\symbol{'052}");
\item[]\textsl{val\ it\ =\ false\ :\ bool}
\item[]\textsl{-\ }locSimped(Reg.fromString\ "(0\ +\ 1)\symbol{'052}0");
\item[]\textsl{val\ it\ =\ true\ :\ bool}
\item[]\textsl{-\ }fun\ locSimp\ nOpt\ =
\item[]\textsl{=\ }\ \ \ \ \ \ Reg.locallySimplify(nOpt,\ Reg.obviousSubset);
\item[]\textsl{val\ locSimp\ =\ fn\ :\ int\ option\ ->\ reg\ ->\ bool\ \symbol{'052}\ reg}
\item[]\textsl{-\ }locSimp\ NONE\ (Reg.fromString\ "%\ +\ 0\symbol{'052}0(0\ +\ 1)\symbol{'052}\ +\ 1\symbol{'052}1(0\ +\ 1)\symbol{'052}");
\item[]\textsl{val\ it\ =\ (true,-)\ :\ bool\ \symbol{'052}\ reg}
\item[]\textsl{-\ }Reg.output("",\ #2\ it);
\item[]\textsl{(0\ +\ 1)\symbol{'052}}
\item[]\textsl{val\ it\ =\ ()\ :\ unit}
\item[]\textsl{-\ }locSimp\ NONE\ (Reg.fromString\ "%\ +\ 1\symbol{'052}0(0\ +\ 1)\symbol{'052}\ +\ 0\symbol{'052}1(0\ +\ 1)\symbol{'052}");
\item[]\textsl{val\ it\ =\ (true,-)\ :\ bool\ \symbol{'052}\ reg}
\item[]\textsl{-\ }Reg.output("",\ #2\ it);
\item[]\textsl{(0\ +\ 1)\symbol{'052}}
\item[]\textsl{val\ it\ =\ ()\ :\ unit}
\item[]\textsl{-\ }locSimp\ NONE\ (Reg.fromString\ "(1\ +\ 00\symbol{'052}1)\symbol{'052}00\symbol{'052}");
\item[]\textsl{val\ it\ =\ (true,-)\ :\ bool\ \symbol{'052}\ reg}
\item[]\textsl{-\ }Reg.output("",\ #2\ it);
\item[]\textsl{(0\ +\ 1)\symbol{'052}0}
\item[]\textsl{val\ it\ =\ ()\ :\ unit}
\item[]\textsl{-\ }Reg.locallySimplifyTrace
\item[]\textsl{=\ }(NONE,\ Reg.obviousSubset)
\item[]\textsl{=\ }(Reg.fromString\ "1\symbol{'052}(01\symbol{'052}01\symbol{'052})\symbol{'052}");
\item[]\textsl{considered\ all\ 10\ structural\ reorganizations\ of\ 1\symbol{'052}(01\symbol{'052}01\symbol{'052})\symbol{'052}}
\item[]\textsl{1\symbol{'052}(01\symbol{'052}01\symbol{'052})\symbol{'052}\ transformed\ by\ structural\ rule\ 4\ at\ position\ \symbol{'133}2,\ 1\symbol{'135}\ to}
\item[]\textsl{1\symbol{'052}((01\symbol{'052})01\symbol{'052})\symbol{'052}\ transformed\ by\ structural\ rule\ 4\ at\ position\ \symbol{'133}2,\ 1\symbol{'135}}
\item[]\textsl{to\ 1\symbol{'052}(((01\symbol{'052})0)1\symbol{'052})\symbol{'052}\ transformed\ by\ reduction\ rule\ 14\ at\ position}
\item[]\textsl{\symbol{'133}2\symbol{'135}\ to\ 1\symbol{'052}(%\ +\ ((01\symbol{'052})0)((01\symbol{'052})0\ +\ 1)\symbol{'052})\ weakly\ simplifies\ to}
\item[]\textsl{1\symbol{'052}(%\ +\ 01\symbol{'052}0(1\ +\ 01\symbol{'052}0)\symbol{'052})}
\item[]\textsl{considered\ all\ 40\ structural\ reorganizations\ of}
\item[]\textsl{1\symbol{'052}(%\ +\ 01\symbol{'052}0(1\ +\ 01\symbol{'052}0)\symbol{'052})}
\item[]\textsl{1\symbol{'052}(%\ +\ 01\symbol{'052}0(1\ +\ 01\symbol{'052}0)\symbol{'052})\ transformed\ by\ structural\ rule\ 4\ at}
\item[]\textsl{position\ \symbol{'133}2,\ 2,\ 2\symbol{'135}\ to\ 1\symbol{'052}(%\ +\ 0(1\symbol{'052}0)(1\ +\ 01\symbol{'052}0)\symbol{'052})\ transformed\ by}
\item[]\textsl{structural\ rule\ 4\ at\ position\ \symbol{'133}2,\ 2\symbol{'135}\ to\ 1\symbol{'052}(%\ +\ (01\symbol{'052}0)(1\ +\ 01\symbol{'052}0)\symbol{'052})}
\item[]\textsl{transformed\ by\ reduction\ rule\ 23\ at\ position\ \symbol{'133}\symbol{'135}\ to\ (1\ +\ 01\symbol{'052}0)\symbol{'052}}
\item[]\textsl{considered\ all\ 4\ structural\ reorganizations\ of\ (1\ +\ 01\symbol{'052}0)\symbol{'052}}
\item[]\textsl{(1\ +\ 01\symbol{'052}0)\symbol{'052}\ is\ locally\ simplified}
\item[]\textsl{val\ it\ =\ (true,-)\ :\ bool\ \symbol{'052}\ reg}
\end{list}

For even fairly small regular expressions, running through all the
structural reorganizations can take prohibitively long.  So, one
often has to bound the number of such reorganizations, as in:
\begin{list}{}
{\setlength{\leftmargin}{\leftmargini}
\setlength{\rightmargin}{0cm}
\setlength{\itemindent}{0cm}
\setlength{\listparindent}{0cm}
\setlength{\itemsep}{0cm}
\setlength{\parsep}{0cm}
\setlength{\labelsep}{0cm}
\setlength{\labelwidth}{0cm}
\catcode`\#=12
\catcode`\$=12
\catcode`\%=12
\catcode`\^=12
\catcode`\_=12
\catcode`\.=12
\catcode`\?=12
\catcode`\!=12
\catcode`\&=12
\ttfamily}
\small
\item[]\textsl{-\ }val\ reg\ =\ Reg.input\ "";
\item[]\textsl{@\ }1\ +\ (%\ +\ 0\ +\ 2)(%\ +\ 0\ +\ 2)\symbol{'052}1\ +
\item[]\textsl{@\ }(1\ +\ (%\ +\ 0\ +\ 2)(%\ +\ 0\ +\ 2)\symbol{'052}1)
\item[]\textsl{@\ }(%\ +\ 0\ +\ 2\ +\ 1(%\ +\ 0\ +\ 2)\symbol{'052}1)
\item[]\textsl{@\ }(%\ +\ 0\ +\ 2\ +\ 1(%\ +\ 0\ +\ 2)\symbol{'052}1)\symbol{'052}
\item[]\textsl{@\ }.
\item[]\textsl{val\ reg\ =\ -\ :\ reg}
\item[]\textsl{-\ }Reg.equal(Reg.weaklySimplify\ reg,\ reg);
\item[]\textsl{val\ it\ =\ true\ :\ bool}
\item[]\textsl{-\ }val\ (b',\ reg')\ =\ locSimp\ (SOME\ 10)\ reg;
\item[]\textsl{val\ b'\ =\ false\ :\ bool}
\item[]\textsl{val\ reg'\ =\ -\ :\ reg}
\item[]\textsl{-\ }Reg.output("",\ reg');
\item[]\textsl{(0\ +\ 2)\symbol{'052}1(0\ +\ 2\ +\ 1(0\ +\ 2)\symbol{'052}1)\symbol{'052}}
\item[]\textsl{val\ it\ =\ ()\ :\ unit}
\item[]\textsl{-\ }val\ (b'',\ reg'')\ =\ locSimp\ (SOME\ 1000)\ reg';
\item[]\textsl{val\ b''\ =\ true\ :\ bool}
\item[]\textsl{val\ reg''\ =\ -\ :\ reg}
\item[]\textsl{-\ }Reg.output("",\ reg'');
\item[]\textsl{(0\ +\ 2)\symbol{'052}1(0\ +\ 2\ +\ 1(0\ +\ 2)\symbol{'052}1)\symbol{'052}}
\item[]\textsl{val\ it\ =\ ()\ :\ unit}
\end{list}

Note that, in this transcript, \texttt{reg'} turns out to be
locally simplified, despite the fact that \texttt{b'} is \texttt{false}.
\index{regular expression!simplification!local|)}%
\index{regular expression!local simplification|)}%
\index{simplification!regular expression!local|)}%

\subsection*{Global Simplification}

\index{regular expression!simplification!global|(}%
\index{regular expression!global simplification|(}%
\index{simplification!regular expression!global|(}%

Given a conservative approximation to subset testing $\Sub$, and a
\index{regular expression!globally simplified}%
regular expression $\alpha$, we say that $\alpha$ is \emph{globally
  simplified with respect to} $\Sub$ iff no strictly simpler regular
expression can be found by an arbitrary number of applications (to
subtrees) of weak simplification, structural rules and reduction
rules.

The \emph{global simplification of} a regular expression $\alpha$
\emph{with respect to} a conservative approximation to subset testing
$\Sub$ consists of generating the set $X$ of all regular expressions
$\beta$ that can formed from $\alpha$ by an arbitrary number of
applications of weak simplification, the structural rules and the
reduction rules (which may be applied to subtrees). All of the
elements of $X$ will have the same meaning as $\alpha$, and will have
alphabets that are subsets of the alphabet of $\alpha$. Because
\begin{itemize}
\item weak simplification (even applied to subtrees) either preserves
  the closure complexity, size, numbers of concatenations and numbers
  of symbols of a regular expression, or results in a regular
  expression that is a $\simp$-predecessor,

\item the structural rules (even applied to subtrees) preserve the
  closure complexity, size, numbers of concatenations and numbers of
  symbols of a regular expression, and

\item the reduction rules (even applied to subtrees) produce regular
  expressions that are $\simp$-predecessors,
\end{itemize}
by Propositions~\ref{SimpTransLEQSimp} and \ref{SimpTrans}, all the
elements of $X$ either are ${\simp}$-predecessors of $\alpha$ or have
the same closure complexity, size, numbers of concatenations and
numbers of symbols as $\alpha$.

The simplest element of $X$ is then selected (when there isn't a
unique simplest element, the smallest of the simplest elements---in
our total ordering on regular expressions---is selected).  If this
element is a $\simp$-predecessor of $\alpha$, it will be
${\leqsimp}\,\alpha$. Otherwise, it will have the same closure
complexity, size, numbers of concatenations and numbers of symbols as
$\alpha$. And it will be standardized, as weak simplification of a
non-standardized regular expression will standardize it, making it
more simplified. Thus it will be ${\leqsimp}\,\alpha$.  Similarly, it
will be globally simplified with respect to $\Sub$, as otherwise it
wouldn't be the simplest element of $X$.

Of course, this algorithm is much less efficient than the local one,
but by revisiting choices, it is capable of producing simpler answers.

We can generate $X$ by a sequence of \emph{stages}. At stage $0$, we
have $\{\alpha\}$. At some stage $n+1$, we consider, in some order,
the regular expressions that we added at stage $n$. For each of these
regular expressions, $\beta$, we add all the (finitely many) regular
expressions $\gamma$ that can be formed by applying a rule (structural
rule, reduction rule, or weak simplification) to one of the subtrees
of $\beta$, subject to the restriction that $\gamma$ has not already
been added at a previous stage or earlier in this stage.  We want to
show that we are guaranteed to reach a stage where nothing new can be
added. To do this, we will use K\"onig's Lemma
\index{K\"onig's Lemma}%
from graph theory.

A \emph{graph} $G$ consists of a set $V_G$ (or just $V$) of
\emph{vertices}, together with a set $E_G\sub V\times V$ (or just $E$) of
\emph{edges}, with the property that, for all $x, y\in V$,
$(x, x)\not\in E$, and $(x,y)\in E$ iff $(y, x)\in E$.  We say that
$G$ is \emph{infinite} iff $V$ is infinite; otherwise it is
\emph{finite}.  A \emph{path} from $x$ to $y$ in $G$ is a nonempty,
finite sequence of distinct vertices $v_1,\ldots,v_n$ such that
$(v_i, v_{i + 1})\in E$, for all $1\leq i < n$, and where $x = v_1$
and $v_n = y$. The \emph{length} of the path is $n-1$.  (Thus $x$ is a
path from itself to itself.) A graph is a \emph{tree} (different from
our inductively defined trees) iff for all $x,y\in V$, there is a
unique path in $G$ from $x$ to $y$. If we pick an element $r\in V$ as
the \emph{root} of the tree, there is a unique path from $r$ to any
$v\in V$. We say that $x\in V$ is the \emph{parent} of $y\in V$, iff
the path from $r$ to $y$ ends with $x, y$. We say that $y$ is the
\emph{child} of $x$ iff $x$ is the parent of $y$.  An $r$-rooted tree
$G$ is \emph{finitely splitting} iff every $v\in G$ has finitely many
children.  An \emph{infinite branch} of an $r$-rooted tree $G$ is an
infinite sequence of vertices $v_1$, $v_2$, \ldots, where $r = v_1$
and for all $i\in\nats$, $v_{i+1}$ is a child of $v_i$.

\begin{lemma}[K\"onig's Lemma]
If G is an infinite, finitely splitting tree with root $r$,
then $G$ has an infinite branch.
\end{lemma}

We can use K\"onig's lemma to show that:

\begin{lemma}
$X$ is finite.
\end{lemma}

\begin{proof}
As we go through the stages of generating the elements of $X$ from
$\alpha$, we can build an $\alpha$-rooted tree whose vertices are the
regular expressions we add. At stage $0$, we only have the root,
$\alpha$.  At stage $n+1$, we consider, in some order, the regular
expressions $\beta$ that were added at stage $n$, i.e., whose paths
from $\alpha$ have length $n$. For a given $\beta$, we generate all
the (finitely many) regular expressions $\gamma$ that can be formed by
applying one of our rules to a subtree of $\beta$. Each such $\gamma$
becomes a child of $\beta$ (we add $\gamma$ to $V$, and add
$(\beta, \gamma)$ and $(\gamma, \beta)$ to $E$), but only if $\gamma$
is not already in $V$.  If no regular expressions are added at stage
$n+1$, the process terminates, and $X=V$ is finite.

Because at stage $n+1$ we only create children for elements added at
stage $n$, we will have a finitely splitting tree even if the process
of adding new elements were to go on for infinitely many stages---in
which case we would let $V$ and $E$ be the union of the vertices and
edges of all the stages, and we would have an infinite, finitely
splitting tree.

Suppose, toward a contradiction, that $V$ is infinite, and thus $G$ is
infinite. By K\"onig's Lemma, $G$ has an infinite branch, $\beta_1$,
$\beta_2$, \ldots, where $\alpha = \beta_1$. At each step of this
branch, either the closure complexity, size, numbers of
concatenations, and numbers of symbols stays the same (the closure
rules, and sometimes weak simplification) or we go down in the $\simp$
well-founded relation (the reduction rules, and sometimes weak
simplification). Furthermore, the alphabet at each step either stays
the same or becomes a proper subset. (Note that the size may go up,
when the closure complexity goes down.) Because $\simp$ is
well-founded, the set of elements of the branch has a $\simp$-minimal
element, say $\beta_i$. Consequently, all the remaining elements of
the branch must have the same closure complexity, size, numbers of
concatenations, and numbers of symbols as $\beta_i$. And they all have
alphabets that are subsets of the alphabet of $\beta_i$. Because all
the remaining elements of the branch have the same size and alphabets
that are subsets of the same alphabet,
Exercise~\ref{RegExactExpOverAlphabetFinite} tells us that this
infinite set is actually finite, giving us our contradiction.  Thus we
can conclude that $X=V$ is finite.
\end{proof}

We define a function/algorithm
\index{regular expression!globallySimplify@$\globallySimplify$}%
\begin{displaymath}
\globallySimplify\in(\Reg\times\Reg\fun\Bool)\fun\Reg\fun\Reg 
\end{displaymath}
by: for all conservative approximations to subset testing $\Sub$, and
$\alpha\in\Reg$, $\globallySimplify\,\Sub\,\alpha$ is the result
of running our global simplification algorithm on $\alpha$, using
$\Sub$ as our conservative approximation to subset testing.

\begin{theorem}
For all conservative approximations to subset testing $\Sub$, and
$\alpha\in\Reg$:
\begin{itemize}
\item $\globallySimplify\,\Sub\,\alpha$ is globally simplified with
  respect to $\Sub$;

\item $\globallySimplify\,\Sub\,\alpha$ is equivalent to $\alpha$;

\item $\alphabet(\globallySimplify\,\Sub\,\alpha)\sub\alphabet\,\alpha$;
  and

\item $\globallySimplify\,\Sub\,\alpha\leqsimp\alpha$.
\end{itemize}
\end{theorem}

The Forlan module \texttt{Reg} provides the following functions
relating to global simplification:
\begin{verbatim}
val globallySimplified    :
     (reg * reg -> bool) -> reg -> bool
val globallySimplifyTrace :
      int option * (reg * reg -> bool) -> reg -> bool * reg
val globallySimplify      :
      int option * (reg * reg -> bool) -> reg -> bool * reg
\end{verbatim}
\index{Reg@\texttt{Reg}!globallySimplified@\texttt{globallySimplified}}%
\index{Reg@\texttt{Reg}!globallySimplifyTrace@\texttt{globallySimplifyTrace}}%
\index{Reg@\texttt{Reg}!globallySimplify@\texttt{globallySimplify}}%

The function \texttt{globallySimplified} takes in a conservative
approximation to subset testing $\Sub$, and returns a function that
tests whether a regular expression is globally simplified with respect
to $\Sub$.  The function \texttt{globallySimplifyTrace} implements
$\globallySimplify$.  It emits tracing messages explaining its
operation, and takes in an extra argument of type \texttt{int~option},
and produces an extra result of type \texttt{bool}.  If this argument
is \texttt{NONE}, then it runs the same as $\globallySimplify$, and
the boolean result is always $\true$.  But if it is
$\mathtt{SOME}\;n$, for $n\geq 1$, then at most $n$ elements of the
set $X$ are generated (in order according to the number of steps
needed to produce them from the original regular expression), before
picking the simplest one, and the boolean result is $\false$ iff this
$n$ isn't enough to generate all of $X$, and so the returned regular
expression isn't necessarily globally simplified.  The function
\texttt{globallySimplify} works identically, except it doesn't issue
tracing messages.

For even quite small regular expressions,
\texttt{globallySimplified} will fail to run to completion in an
acceptable time-frame, and one will have to bound the size of the set
$X$ in order for \texttt{globallySimplify} and
\texttt{globallySimplifyTrace} to run to completion in an
acceptable time-frame.

Here are some examples of how these functions can be used.
\begin{list}{}
{\setlength{\leftmargin}{\leftmargini}
\setlength{\rightmargin}{0cm}
\setlength{\itemindent}{0cm}
\setlength{\listparindent}{0cm}
\setlength{\itemsep}{0cm}
\setlength{\parsep}{0cm}
\setlength{\labelsep}{0cm}
\setlength{\labelwidth}{0cm}
\catcode`\#=12
\catcode`\$=12
\catcode`\%=12
\catcode`\^=12
\catcode`\_=12
\catcode`\.=12
\catcode`\?=12
\catcode`\!=12
\catcode`\&=12
\ttfamily}
\small
\item[]\textsl{-\ }Reg.size(Reg.fromString\ "(00\symbol{'052}1)\symbol{'052}");
\item[]\textsl{val\ it\ =\ 7\ :\ int}
\item[]\textsl{-\ }Reg.globallySimplifyTrace\ (NONE,\ Reg.obviousSubset)
\item[]\textsl{=\ }(Reg.fromString\ "(00\symbol{'052}1)\symbol{'052}");
\item[]\textsl{considering\ candidates\ with\ explanations\ of\ length\ 0}
\item[]\textsl{simplest\ result\ now:\ (00\symbol{'052}1)\symbol{'052}}
\item[]\textsl{considering\ candidates\ with\ explanations\ of\ length\ 1}
\item[]\textsl{simplest\ result\ now:\ (00\symbol{'052}1)\symbol{'052}\ transformed\ by\ reduction\ rule\ 16\ at}
\item[]\textsl{position\ \symbol{'133}\symbol{'135}\ to\ %\ +\ 0(0\ +\ 10)\symbol{'052}1}
\item[]\textsl{considering\ candidates\ with\ explanations\ of\ length\ 2}
\item[]\textsl{simplest\ result\ now:\ (00\symbol{'052}1)\symbol{'052}\ transformed\ by\ reduction\ rule\ 16\ at}
\item[]\textsl{position\ \symbol{'133}\symbol{'135}\ to\ %\ +\ 0(0\ +\ 10)\symbol{'052}1\ transformed\ by\ reduction\ rule\ 22\ at}
\item[]\textsl{position\ \symbol{'133}2,\ 2,\ 1,\ 1\symbol{'135}\ to\ %\ +\ 0((%\ +\ 1)0)\symbol{'052}1}
\item[]\textsl{considering\ candidates\ with\ explanations\ of\ length\ 3}
\item[]\textsl{considering\ candidates\ with\ explanations\ of\ length\ 4}
\item[]\textsl{considering\ candidates\ with\ explanations\ of\ length\ 5}
\item[]\textsl{considering\ candidates\ with\ explanations\ of\ length\ 6}
\item[]\textsl{considering\ candidates\ with\ explanations\ of\ length\ 7}
\item[]\textsl{search\ completed\ after\ considering\ 36\ candidates\ with\ maximum\ size}
\item[]\textsl{12}
\item[]\textsl{(00\symbol{'052}1)\symbol{'052}\ transformed\ by\ reduction\ rule\ 16\ at\ position\ \symbol{'133}\symbol{'135}\ to}
\item[]\textsl{%\ +\ 0(0\ +\ 10)\symbol{'052}1\ transformed\ by\ reduction\ rule\ 22\ at\ position}
\item[]\textsl{\symbol{'133}2,\ 2,\ 1,\ 1\symbol{'135}\ to\ %\ +\ 0((%\ +\ 1)0)\symbol{'052}1\ is\ globally\ simplified}
\item[]\textsl{val\ it\ =\ (true,-)\ :\ bool\ \symbol{'052}\ reg}
\item[]\textsl{-\ }Reg.size(#2\ it);
\item[]\textsl{val\ it\ =\ 12\ :\ int}
\item[]\textsl{-\ }locSimp\ NONE\ (Reg.fromString\ "(00\symbol{'052}11\symbol{'052})\symbol{'052}");
\item[]\textsl{val\ it\ =\ (true,-)\ :\ bool\ \symbol{'052}\ reg}
\item[]\textsl{-\ }Reg.output("",\ #2\ it);
\item[]\textsl{%\ +\ 00\symbol{'052}1(%\ +\ (0\ +\ 1)\symbol{'052}1)}
\item[]\textsl{val\ it\ =\ ()\ :\ unit}
\item[]\textsl{-\ }Reg.globallySimplify\ (NONE,\ Reg.obviousSubset)
\item[]\textsl{=\ }(Reg.fromString\ "(00\symbol{'052}11\symbol{'052})\symbol{'052}");
\item[]\textsl{val\ it\ =\ (true,-)\ :\ bool\ \symbol{'052}\ reg}
\item[]\textsl{-\ }Reg.output("",\ #2\ it);
\item[]\textsl{%\ +\ 0(0\ +\ 1)\symbol{'052}1}
\item[]\textsl{val\ it\ =\ ()\ :\ unit}
\end{list}


\index{regular expression!simplification!global|)}%
\index{regular expression!global simplification|)}%
\index{simplification!regular expression!global|)}%

\subsection{Notes}

Although books on formal language theory usually study various regular
expression equivalences, we have gone much further, giving three at
least partly novel algorithms for regular expression simplification.
Although many of the simplification and structural rules used in the
simplification algorithms are well-known, some were invented, as was
the concept of closure complexity.

\index{regular expression!simplification|)}%
\index{simplification!regular expression|)}%

%%% Local Variables: 
%%% mode: latex
%%% TeX-master: "book"
%%% End: 
