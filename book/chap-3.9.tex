\section{Empty-string Finite Automata}
\label{EmptyStringFiniteAutomata}

\index{finite automaton!empty-string|(}%
\index{empty-string finite automaton|(}%
\index{EFA|(}%

In this and the following two sections, we will study three
progressively more restricted kinds of finite automata:
\begin{itemize}
\item empty-string finite automata (EFAs);

\item nondeterministic finite automata (NFAs); and

\item deterministic finite automata (DFAs).
\end{itemize}
Every DFA will be an NFA; every NFA will be an EFA; and every EFA will
be an FA.  Thus, $L(M)$ will be well-defined, if $M$ is a DFA, NFA or
EFA.

The more restricted kinds of automata will be easier to process on the
computer than the more general kinds; they will also have nicer
reasoning principles than the more general kinds.  We will give
algorithms for converting the more general kinds of automata into the
more restricted kinds.  Thus even the deterministic finite automata
will accept the same set of languages as the finite automata.  On the
other hand, it will sometimes be easier to find one of the more
general kinds of automata that accepts a given language rather than
one of the more restricted kinds accepting the language.  And, there
are languages where the smallest DFA accepting the language is
exponentially bigger than the smallest FA accepting the language.

\subsection{Definition of EFAs}

An \emph{empty-string finite automaton} (EFA) $M$ is
a finite automaton such that
\begin{gather*}
T_M\sub\setof{q, x\fun r}{q,r\in\Sym\eqtxt{and}x\in\Str\eqtxt{and}|x|\leq 1}.
\end{gather*}
In other words, an FA is an EFA iff every string of every transition
of the FA is either $\%$ or has a single symbol.

For example, $\Asf,\%\fun\Bsf$ and $\Asf,\onesf\fun\Bsf$ are legal EFA
transitions, but $\Asf,\onesf\onesf\fun\Bsf$ is not legal.  We write
\index{EFA@$\EFA$}%
\index{finite automaton!EFA@$\EFA$}%
\index{empty-string finite automaton!EFA@$\EFA$}%
$\EFA$ for the set of all empty-string finite automata.  Thus
$\EFA\subsetneq\FA$.

The following proposition obviously holds.

\begin{proposition}
Suppose $M$ is an EFA.
\begin{itemize}
\item For all $N\in\FA$, if $M\iso N$, then $N$ is an EFA.

\item For all bijections $f$ from $Q_M$ to some set of symbols,
  $\renameStates(M, f)$ is an EFA.

\item $\renameStatesCanonically\,M$ is an EFA.

\item $\simplify\,M$ is an EFA.
\end{itemize}
\end{proposition}

\subsection{Converting FAs to EFAs}

\index{finite automaton!converting FA to EFA}%
\index{empty-string finite automaton!converting FA to EFA}%
If we want to convert an FA into an equivalent EFA, we can proceed as
follows.  Every state of the FA will be a state of the EFA, the start
and accepting states are unchanged, and every transition of the FA
that is a legal EFA transition will be a transition
of the EFA.  If our FA has a transition
\begin{gather*}
p, b_1b_2\cdots b_n\fun r,
\end{gather*}
where $n\geq 2$ and the $b_i$ are symbols, then we replace this
transition with the $n$ transitions
\begin{gather*}
p\tranarr{b_1}q_1, q_1\tranarr{b_2}q_2,\,\ldots, q_{n-1}\tranarr{b_n}r,
\end{gather*}
where $q_1,\,\ldots,q_{n-1}$ are $n-1$ new, non-accepting, states.

For example, we can convert the FA
\begin{center}
\input{chap-3.9-fig1.eepic}
\end{center}
into the EFA
\begin{center}
\input{chap-3.9-fig2.eepic}
\end{center}

We have to be careful how we choose our new states.  The
symbols we choose can't be states of the original machine, and we
can't choose the same symbol twice.  Instead of making a series of
random choices, we will use structured symbols in such a way that
one will be able to look at a resulting EFA and tell what the
original FA was.

First, the algorithm renames each old state $q$ to
$\langle 1,q\rangle$.  Then it can replace a transition
\begin{gather*}
p\tranarr{b_1b_2\cdots b_n}r,
\end{gather*}
where $n\geq 2$ and the $b_i$ are symbols, with the transitions
\begin{gather*}
\langle 1,p\rangle\tranarr{b_1}
\langle 2,\langle p,b_1,b_2\cdots b_n,r\rangle\rangle, \\
\langle 2,\langle p,b_1,b_2\cdots b_n,r\rangle\rangle\tranarr{b_2}
\langle 2,\langle p,b_1b_2,b_3\cdots b_n,r\rangle\rangle, \\
\ldots,\\
\langle 2,\langle p,b_1b_2\cdots b_{n-1},b_n,r\rangle\rangle\tranarr{b_n},
\langle 1,r\rangle.
\end{gather*}

\index{finite automaton!faToEFA@$\faToEFA$}%
\index{empty-string finite automaton!faToEFA@$\faToEFA$}%
We define a function $\faToEFA\in\FA\fun\EFA$ that converts FAs into
EFAs by saying that $\faToEFA\,M$ is the result of running the above
algorithm on input $M$.

\begin{theorem}
For all $M\in\FA$:
\begin{itemize}
\item $\faToEFA\,M\approx M$; and

\item $\alphabet(\faToEFA\,M) = \alphabet\,M$.
\end{itemize}
\end{theorem}

\subsection{Processing EFAs in Forlan}

The Forlan module \texttt{EFA} defines an abstract type \texttt{efa}
\index{EFA@\texttt{EFA}}%
\index{EFA@\texttt{EFA}!efa@\texttt{efa}}%
(in the top-level environment) of empty-string finite automata,
along with various functions for processing EFAs.
Values of type \texttt{efa} are implemented as values of type \texttt{fa}, and
the module EFA provides functions
\begin{verbatim}
val injToFA    : efa -> fa
val projFromFA : fa -> efa
\end{verbatim}
\index{EFA@\texttt{EFA}!injToFA@\texttt{injToFA}}%
\index{EFA@\texttt{EFA}!projFromFA@\texttt{projFromFA}}%
for making a value of type \texttt{efa} have type \texttt{fa}, i.e.,
``injecting'' an \texttt{efa} into type \texttt{fa}, and for
making a value of type \texttt{fa} that is an EFA have type
\texttt{efa}, i.e., ``projecting'' an \texttt{fa} that is an EFA to
type \texttt{efa}.  If one tries to project an \texttt{fa} that is not
an EFA to type \texttt{efa}, an error is signaled.  The functions
\texttt{injToFA} and \texttt{projFromFA} are available in the top-level
environment as \texttt{injEFAToFA} and \texttt{projFAToEFA}, respectively.
\index{empty-string finite automaton!injEFAToFA@\texttt{injEFAToFA}}%
\index{empty-string finite automaton!projFAToEFA@\texttt{projFAToEFA}}%

The module \texttt{EFA} also defines the functions:
\begin{verbatim}
val input  : string -> efa
val fromFA : fa -> efa
\end{verbatim}
\index{EFA@\texttt{EFA}!input@\texttt{input}}%
\index{EFA@\texttt{EFA}!fromFA@\texttt{fromFA}}%
The function \texttt{input} is used to input an EFA, i.e., to input a
value of type \texttt{fa} using \texttt{FA.input}, and then attempt to
project it to type \texttt{efa}.  The function \texttt{fromFA}
corresponds to our conversion function $\faToEFA$, and is available in
the top-level environment with that name:
\begin{verbatim}
val faToEFA : fa -> efa
\end{verbatim}
\index{empty-string finite automaton!faToEFA@\texttt{faToEFA}}%

Finally, most of the functions for processing FAs that were introduced
in previous sections are inherited by \texttt{EFA}:
\begin{verbatim}
val output                  : string * efa -> unit 
val numStates               : efa -> int
val numTransitions          : efa -> int
val equal                   : efa * efa -> bool
val alphabet                : efa -> sym set
val checkLP                 : efa -> lp -> unit
val validLP                 : efa -> lp -> bool
val isomorphism             : efa * efa * sym_rel -> bool
val findIsomorphism         : efa * efa -> sym_rel
val isomorphic              : efa * efa -> bool
val renameStates            : efa * sym_rel -> efa
val renameStatesCanonically : efa -> efa
val processStr              : efa -> sym set * str -> sym set
val accepted                : efa -> str -> bool
val findLP                  : efa -> sym set * str * sym set -> lp
val findAcceptingLP         : efa -> str -> lp
val simplified              : efa -> bool
val simplify                : efa -> efa
\end{verbatim}
\index{EFA@\texttt{EFA}!output@\texttt{output}}%
\index{EFA@\texttt{EFA}!numStates@\texttt{numStates}}%
\index{EFA@\texttt{EFA}!numTransitions@\texttt{numTransitions}}%
\index{EFA@\texttt{EFA}!equal@\texttt{equal}}%
\index{EFA@\texttt{EFA}!alphabet@\texttt{alphabet}}%
\index{EFA@\texttt{EFA}!checkLP@\texttt{checkLP}}%
\index{EFA@\texttt{EFA}!validLP@\texttt{validLP}}%
\index{EFA@\texttt{EFA}!isomorphism@\texttt{isomorphism}}%
\index{EFA@\texttt{EFA}!findIsomorphism@\texttt{findIsomorphism}}%
\index{EFA@\texttt{EFA}!isomorphic@\texttt{isomorphic}}%
\index{EFA@\texttt{EFA}!renameStates@\texttt{renameStates}}%
\index{EFA@\texttt{EFA}!renameStatesCanonically@\texttt{renameStatesCanonically}}%
\index{EFA@\texttt{EFA}!processStr@\texttt{processStr}}%
\index{EFA@\texttt{EFA}!accepted@\texttt{accepted}}%
\index{EFA@\texttt{EFA}!findLP@\texttt{findLP}}%
\index{EFA@\texttt{EFA}!findAcceptingLP@\texttt{findAcceptingLP}}%
\index{EFA@\texttt{EFA}!simplified@\texttt{simplified}}%
\index{EFA@\texttt{EFA}!simplify@\texttt{simplify}}%

Suppose that \texttt{fa} is the finite automaton
\begin{center}
\input{chap-3.9-fig1.eepic}
\end{center}
Here are some example uses of a few of the above functions:
\begin{list}{}
{\setlength{\leftmargin}{\leftmargini}
\setlength{\rightmargin}{0cm}
\setlength{\itemindent}{0cm}
\setlength{\listparindent}{0cm}
\setlength{\itemsep}{0cm}
\setlength{\parsep}{0cm}
\setlength{\labelsep}{0cm}
\setlength{\labelwidth}{0cm}
\catcode`\#=12
\catcode`\$=12
\catcode`\%=12
\catcode`\^=12
\catcode`\_=12
\catcode`\.=12
\catcode`\?=12
\catcode`\!=12
\catcode`\&=12
\ttfamily}
\small
\item[]\textsl{-\ }projFAToEFA\ fa;
\item[]\textsl{invalid\ label\ in\ transition:\ "12"}
\item[]
\item[]\textsl{uncaught\ exception\ Error}
\item[]\textsl{-\ }val\ efa\ =\ faToEFA\ fa;
\item[]\textsl{val\ efa\ =\ -\ :\ efa}
\item[]\textsl{-\ }EFA.output("",\ efa);
\item[]\textsl{\symbol{'173}states\symbol{'175}}
\item[]\textsl{<1,A>,\ <1,B>,\ <2,<A,1,2,B>>,\ <2,<B,3,45,B>>,\ <2,<B,34,5,B>>}
\item[]\textsl{\symbol{'173}start\ state\symbol{'175}\ <1,A>\ \symbol{'173}accepting\ states\symbol{'175}\ <1,B>}
\item[]\textsl{\symbol{'173}transitions\symbol{'175}}
\item[]\textsl{<1,A>,\ 0\ ->\ <1,A>;\ <1,A>,\ 1\ ->\ <2,<A,1,2,B>>;}
\item[]\textsl{<1,B>,\ 3\ ->\ <2,<B,3,45,B>>;\ <2,<A,1,2,B>>,\ 2\ ->\ <1,B>;}
\item[]\textsl{<2,<B,3,45,B>>,\ 4\ ->\ <2,<B,34,5,B>>;\ <2,<B,34,5,B>>,\ 5\ ->\ <1,B>}
\item[]\textsl{val\ it\ =\ ()\ :\ unit}
\item[]\textsl{-\ }val\ efa'\ =\ EFA.renameStatesCanonically\ efa;
\item[]\textsl{val\ efa'\ =\ -\ :\ efa}
\item[]\textsl{-\ }EFA.output("",\ efa');
\item[]\textsl{\symbol{'173}states\symbol{'175}\ A,\ B,\ C,\ D,\ E\ \symbol{'173}start\ state\symbol{'175}\ A\ \symbol{'173}accepting\ states\symbol{'175}\ B}
\item[]\textsl{\symbol{'173}transitions\symbol{'175}}
\item[]\textsl{A,\ 0\ ->\ A;\ A,\ 1\ ->\ C;\ B,\ 3\ ->\ D;\ C,\ 2\ ->\ B;\ D,\ 4\ ->\ E;\ E,\ 5\ ->\ B}
\item[]\textsl{val\ it\ =\ ()\ :\ unit}
\item[]\textsl{-\ }val\ rel\ =\ EFA.findIsomorphism(efa,\ efa');
\item[]\textsl{val\ rel\ =\ -\ :\ sym_rel}
\item[]\textsl{-\ }SymRel.output("",\ rel);
\item[]\textsl{(<1,A>,\ A),\ (<1,B>,\ B),\ (<2,<A,1,2,B>>,\ C),\ (<2,<B,3,45,B>>,\ D),}
\item[]\textsl{(<2,<B,34,5,B>>,\ E)}
\item[]\textsl{val\ it\ =\ ()\ :\ unit}
\item[]\textsl{-\ }LP.output("",\ FA.findAcceptingLP\ fa\ (Str.input\ ""));
\item[]\textsl{@\ }012345
\item[]\textsl{@\ }.
\item[]\textsl{A,\ 0\ =>\ A,\ 12\ =>\ B,\ 345\ =>\ B}
\item[]\textsl{val\ it\ =\ ()\ :\ unit}
\item[]\textsl{-\ }LP.output("",\ EFA.findAcceptingLP\ efa'\ (Str.input\ ""));
\item[]\textsl{@\ }012345
\item[]\textsl{@\ }.
\item[]\textsl{A,\ 0\ =>\ A,\ 1\ =>\ C,\ 2\ =>\ B,\ 3\ =>\ D,\ 4\ =>\ E,\ 5\ =>\ B}
\item[]\textsl{val\ it\ =\ ()\ :\ unit}
\end{list}


\subsection{Notes}

The algorithm for converting FAs to EFAs is obvious, but our use of
structured state names so as to make the resulting EFAs
self-documenting is novel.

\index{finite automaton!empty-string|)}%
\index{empty-string finite automaton|)}%
\index{EFA|)}%

%%% Local Variables: 
%%% mode: latex
%%% TeX-master: "book"
%%% End: 
