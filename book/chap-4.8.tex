\section{Converting Regular Expressions and FA to Grammars}
\label{ConvertingRegularExpressionsAndFAToGrammars}

In this section, we give simple algorithms for converting regular
expressions and finite automata to grammars.  Since we have algorithms
for converting between regular expressions and finite automata, it is
tempting to only define one of these algorithms.  But better results
can be obtained by defining direct conversions.

\subsection{Converting Regular Expressions to Grammars}

Regular expressions are converted to grammars using a recursive
algorithm that makes use of the operations on grammars that were
defined in Section~\ref{ClosurePropertiesOfContextFreeLanguages}.  The
structure of the algorithm is very similar to the structure of our
algorithm for converting regular expressions to finite automata.

The algorithm is implemented in Forlan by the function
\begin{verbatim}
val fromReg : reg -> gram
\end{verbatim}
of the \texttt{Gram} module.  It's available in the top-level
environment with the name \texttt{regToGram}.

Here is how we can convert the regular expression $\mathsf{01 +
10(11)^*}$ to a grammar using Forlan:
\begin{list}{}
{\setlength{\leftmargin}{\leftmargini}
\setlength{\rightmargin}{0cm}
\setlength{\itemindent}{0cm}
\setlength{\listparindent}{0cm}
\setlength{\itemsep}{0cm}
\setlength{\parsep}{0cm}
\setlength{\labelsep}{0cm}
\setlength{\labelwidth}{0cm}
\catcode`\#=12
\catcode`\$=12
\catcode`\%=12
\catcode`\^=12
\catcode`\_=12
\catcode`\.=12
\catcode`\?=12
\catcode`\!=12
\catcode`\&=12
\ttfamily}
\small
\item[]\textsl{-\ }val\ gram\ =\ regToGram(Reg.input\ "");
\item[]\textsl{@\ }01\ +\ 10(11)\symbol{'052}
\item[]\textsl{@\ }.
\item[]\textsl{val\ gram\ =\ -\ :\ gram}
\item[]\textsl{-\ }Gram.output("",\ Gram.renameVariablesCanonically\ gram);
\item[]\textsl{\symbol{'173}variables\symbol{'175}\ A,\ B,\ C,\ D,\ E,\ F\ \symbol{'173}start\ variable\symbol{'175}\ A}
\item[]\textsl{\symbol{'173}productions\symbol{'175}}
\item[]\textsl{A\ ->\ B\ |\ C;\ B\ ->\ 01;\ C\ ->\ DE;\ D\ ->\ 10;\ E\ ->\ %\ |\ FE;\ F\ ->\ 11}
\item[]\textsl{val\ it\ =\ ()\ :\ unit}
\end{list}


\subsection{Converting Finite Automata to Grammars}

We'll explain the process of converting finite automata to grammars
using an example.  Suppose $M$ is the DFA
\begin{center}
  \input{chap-4.8-fig1.eepic}
\end{center}
The variables of our grammar $G$ consist of the states of $M$, and its
start variable is the start state $\Asf$ of $M$.
(If the symbols of the labels of $M$'s transitions conflict with $M$'s states,
we'll have to rename the states of $M$ first.)
We can translate each transition $q, x\fun r$ to a production
$q\fun xr$.  And, since $\Asf$ is an accepting state of
$M$, we add the production $\Asf\fun\%$.
This gives us the grammar
\begin{align*}
  \Asf &\fun \% \mid \mathsf{0B} \mid \mathsf{1A} , \\
  \Bsf &\fun \mathsf{0A} \mid \mathsf{1B} .
\end{align*}

Consider, e.g., the valid labeled path for $M$
\begin{gather*}
\Asf\lparr{\onesf}\Asf\lparr{\zerosf}\Bsf\lparr{\zerosf}\Asf ,
\end{gather*}
which explains why $\mathsf{100}\in L(M)$.  It corresponds to the
valid parse tree for $G$
\begin{center}
\input{chap-4.8-fig2.eepic}
\end{center}
which explains why $\mathsf{100}\in L(G)$.

The Forlan module \texttt{Gram} contains the function
\begin{verbatim}
val fromFA : fa -> gram
\end{verbatim}
which implements our algorithm for converting finite automata to
grammars.  It's available in the top-level environment with the name
texttt{faToGram}.

Suppose \texttt{fa} of type \texttt{fa} is bound to $M$.  Here is how
we can convert $M$ to a grammar using Forlan:
\begin{list}{}
{\setlength{\leftmargin}{\leftmargini}
\setlength{\rightmargin}{0cm}
\setlength{\itemindent}{0cm}
\setlength{\listparindent}{0cm}
\setlength{\itemsep}{0cm}
\setlength{\parsep}{0cm}
\setlength{\labelsep}{0cm}
\setlength{\labelwidth}{0cm}
\catcode`\#=12
\catcode`\$=12
\catcode`\%=12
\catcode`\^=12
\catcode`\_=12
\catcode`\.=12
\catcode`\?=12
\catcode`\!=12
\catcode`\&=12
\ttfamily}
\small
\item[]\textsl{-\ }val\ gram\ =\ faToGram\ fa;
\item[]\textsl{val\ gram\ =\ -\ :\ gram}
\item[]\textsl{-\ }Gram.output("",\ gram);
\item[]\textsl{\symbol{'173}variables\symbol{'175}\ A,\ B\ \symbol{'173}start\ variable\symbol{'175}\ A}
\item[]\textsl{\symbol{'173}productions\symbol{'175}\ A\ ->\ %\ |\ 0B\ |\ 1A;\ B\ ->\ 0A\ |\ 1B}
\item[]\textsl{val\ it\ =\ ()\ :\ unit}
\end{list}


Because of the existence of our conversion functions, we have that
every regular language is a context-free language.
On the other hand, the language $\setof{\zerosf^n\onesf^n}{n\in\nats}$
is context-free, because of the grammar
\begin{gather*}
  \Asf\fun\%\mid\zerosf\Asf\onesf ,
\end{gather*}
but is not regular, as we proved in Section 3.13.

Summarizing, we have:

\begin{theorem}
The regular languages are a proper subset of the context-free
languages: $\RegLan\subsetneq\CFLan$.
\end{theorem}

\subsection{Notes}

The material in this section is standard.

%%% Local Variables: 
%%% mode: latex
%%% TeX-master: "book"
%%% End: 
