\section{Isomorphism of Grammars}
\label{IsomorphismOfGrammars}

In this section, we study grammar isomorphism, i.e., the way in
which grammars can have the same structure, even though they may have
different variables.

\subsection{Definition and Algorithm}

Suppose $G$ is the grammar with variables $\Asf$ and $\Bsf$,
start variable $\Asf$ and productions:
\begin{align*}
  \Asf &\fun \zerosf\Asf\onesf \mid \Bsf , \\
  \Bsf &\fun \% \mid \twosf\Asf .
\end{align*}
And, suppose $H$ is the grammar with variables $\Bsf$ and $\Asf$,
start variable $\Bsf$ and productions:
\begin{align*}
  \Bsf &\fun \zerosf\Bsf\onesf \mid \Asf , \\
  \Asf &\fun \% \mid \twosf\Bsf .
\end{align*}
$H$ can be formed from $G$ by renaming the variables $\Asf$ and $\Bsf$
of $G$ to $\Bsf$ and $\Asf$, respectively.  As a result, we say that
$G$ and $H$ are isomorphic.

Suppose $G$ is as before, but that $H$ is the grammar with variables
$\twosf$ and $\Asf$, start variable $\twosf$ and productions:
\begin{align*}
  \twosf &\fun \zerosf\twosf\onesf \mid \Asf , \\
  \Asf &\fun \% \mid \twosf\twosf .
\end{align*}
Then $H$ can be formed from $G$ by renaming the variables $\Asf$ and
$\Bsf$ to $\twosf$ and $\Asf$, respectively.  But we shouldn't
consider $G$ and $H$ to be isomorphic, since the symbol $\twosf$ is in
both $\alphabet\,G$ and $Q_H$.  In fact, $G$ and $H$ generate
different languages.  A grammar's variables (e.g., $\Asf$) can't be
renamed to elements of the grammar's alphabet (e.g., $\twosf$).

An \emph{isomorphism} $h$ from a grammar $G$ to a grammar $H$ is
a bijection from $Q_G$ to $Q_H$ such that:
\begin{itemize}
\item $h$ turns $G$ into $H$; and

\item $\alphabet\,G\cap Q_H=\emptyset$, i.e., none of the symbols in
  $G$'s alphabet are variables of $H$.
\end{itemize}
We say that $G$ and $H$ are \emph{isomorphic} iff there is an
isomorphism between $G$ and $H$.

As expected, we have that the relation of being isomorphic is
reflexive on $\Gram$, symmetric and transitive, and that isomorphism
implies having the same alphabet and equivalence.

There is an algorithm for finding an isomorphism from one grammar to
another, if one exists, or reporting that there is no such
isomorphism.  It's similar to the algorithm for finding an isomorphism
between finite automata.

The function $\renameVariables$ takes in a pair $(G,f)$, where $G$ is
a grammar and $f$ is a bijection from $Q_G$ to a set of symbols with
the property that $\range\,f\cap\alphabet\,G=\emptyset$, and returns
the grammar produced from $G$ by renaming $G$'s variables using the
bijection $f$.  The resulting grammar will be isomorphic to $G$.

The following function is a special case of $\renameVariables$.
The function $\renameVariablesCanonically\in\Gram\fun\Gram$ renames the
variables of a grammar $G$ to:
\begin{itemize}
\item $\mathsf{A}$, $\mathsf{B}$, etc., when the grammar has no more
  than 26 variables (the smallest variable of $G$ will be renamed to
  $\mathsf{A}$, the next smallest one to $\mathsf{B}$, etc.); or

\item $\mathsf{\langle 1\rangle}$, $\mathsf{\langle 2\rangle}$, etc.,
  otherwise.
\end{itemize}
These variables will actually be surrounded by a uniform number of
extra brackets, if this is needed to make the new grammar's variables
and the original grammar's alphabet be disjoint.

\subsection{Isomorphism Finding/Checking in Forlan}

The Forlan module \texttt{Gram} contains the following functions
for finding and processing isomorphisms in Forlan:
\begin{verbatim}
val isomorphism                : gram * gram * sym_rel -> bool
val findIsomorphism            : gram * gram -> sym_rel
val isomorphic                 : gram * gram -> bool
val renameVariables            : gram * sym_rel -> gram
val renameVariablesCanonically : gram -> gram
\end{verbatim}
The function \texttt{isomorphism} checks whether a relation on symbols
is an isomorphism from one grammar to another.  The function
\texttt{findIsomorphism} tries to find an isomorphism from one grammar
to another; it issues an error message if there isn't one.  The
function \texttt{isomorphic} checks whether two grammars are
isomorphic.  The function \texttt{renameVariables} issues an error
message if the supplied relation isn't a bijection from the set of
variables of the supplied grammar to some set; otherwise, it returns
the result of $\renameVariables$.  And the function
\texttt{renameVariablesCanonically} acts like
$\renameVariablesCanonically$.

Suppose the identifier \texttt{gram} of type \texttt{gram} is bound to the
grammar with variables $\Asf$ and $\Bsf$, start variable $\Asf$ and
productions:
\begin{align*}
\Asf &\fun \zerosf\Asf\onesf \mid \Bsf , \\
\Bsf &\fun \% \mid \twosf\Asf .
\end{align*}
Suppose the identifier \texttt{gram'} of type \texttt{gram} is bound to
the grammar with variables $\Bsf$ and $\Asf$, start variable $\Bsf$
and productions:
\begin{align*}
\Bsf &\fun \zerosf\Bsf\onesf \mid \Asf , \\ \Asf &\fun \% \mid
\twosf\Bsf .
\end{align*}
And, suppose the identifier \texttt{gram''} of type \texttt{gram} is bound
to the grammar with variables $\twosf$ and $\Asf$, start variable
$\twosf$ and productions:
\begin{align*}
\twosf &\fun \zerosf\twosf\onesf \mid \Asf , \\
\Asf &\fun \% \mid \twosf\twosf .
\end{align*}

Here are some examples of how the above functions can be used:
\begin{list}{}
{\setlength{\leftmargin}{\leftmargini}
\setlength{\rightmargin}{0cm}
\setlength{\itemindent}{0cm}
\setlength{\listparindent}{0cm}
\setlength{\itemsep}{0cm}
\setlength{\parsep}{0cm}
\setlength{\labelsep}{0cm}
\setlength{\labelwidth}{0cm}
\catcode`\#=12
\catcode`\$=12
\catcode`\%=12
\catcode`\^=12
\catcode`\_=12
\catcode`\.=12
\catcode`\?=12
\catcode`\!=12
\catcode`\&=12
\ttfamily}
\small
\item[]\textsl{-\ }val\ rel\ =\ Gram.findIsomorphism(gram,\ gram');
\item[]\textsl{val\ rel\ =\ -\ :\ sym_rel}
\item[]\textsl{-\ }SymRel.output("",\ rel);
\item[]\textsl{(A,\ B),\ (B,\ A)}
\item[]\textsl{val\ it\ =\ ()\ :\ unit}
\item[]\textsl{-\ }Gram.isomorphism(gram,\ gram',\ rel);
\item[]\textsl{val\ it\ =\ true\ :\ bool}
\item[]\textsl{-\ }Gram.isomorphic(gram,\ gram'');
\item[]\textsl{val\ it\ =\ false\ :\ bool}
\item[]\textsl{-\ }Gram.isomorphic(gram',\ gram'');
\item[]\textsl{val\ it\ =\ false\ :\ bool}
\item[]\textsl{-\ }val\ gram\ =\ Gram.input\ "";
\item[]\textsl{@\ }\symbol{'173}variables\symbol{'175}\ B,\ C
\item[]\textsl{@\ }\symbol{'173}start\ variable\symbol{'175}\ B
\item[]\textsl{@\ }\symbol{'173}productions\symbol{'175}\ B\ ->\ AC;\ C\ ->\ <A>
\item[]\textsl{@\ }.
\item[]\textsl{val\ gram\ =\ -\ :\ gram}
\item[]\textsl{-\ }SymSet.output("",\ Gram.alphabet\ gram);
\item[]\textsl{A,\ <A>}
\item[]\textsl{val\ it\ =\ ()\ :\ unit}
\item[]\textsl{-\ }Gram.output("",\ Gram.renameVariablesCanonically\ gram);
\item[]\textsl{\symbol{'173}variables\symbol{'175}\ <<A>>,\ <<B>>\ \symbol{'173}start\ variable\symbol{'175}\ <<A>>}
\item[]\textsl{\symbol{'173}productions\symbol{'175}\ <<A>>\ ->\ A<<B>>;\ <<B>>\ ->\ <A>}
\item[]\textsl{val\ it\ =\ ()\ :\ unit}
\end{list}


\subsection{Notes}

Considering grammar isomorphism is non-traditional, but relatively
straightforward.  We were led to doing so mostly because of the need to
support variable renaming.

%%% Local Variables: 
%%% mode: latex
%%% TeX-master: "book"
%%% End: 
