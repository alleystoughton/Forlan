\section{Introduction to Forlan}
\label{IntroductionToForlan}

The Forlan
\index{Forlan|(}%
toolset is an extension of the Standard ML of New Jersey (SML/NJ)
implementation of Standard ML (SML).  It is implemented as a set of
SML modules.  It is used interactively, and users can extend Forlan by
definining SML functions.

Instructions for installing and running Forlan on machines running
\index{Forlan!installing}%
\index{Forlan!running}%
Linux, macOS and Windows can be found on the Forlan website:
\begin{center}
\url{http://alleystoughton.us/forlan}.
\end{center}

A manual for Forlan is available on the Forlan website, and describes
Forlan's modules in considerably more detail than does this book.  See
the manual for instructions for setting system parameters controlling
such things as the search path used for loading files, the line
length used by Forlan's pretty printer, and the number of elements of
a list that the Forlan top-level displays.

In the concrete syntax for describing Forlan objects---automata,
grammars, etc.---\emph{comments} begin with a ``\texttt{\#}'', and run
through the end of the line.  Comments and whitespace may be
arbitrarily inserted into the descriptions of Forlan objects without
changing how the objects will be lexically analyzed and parsed.
For instance,
\begin{verbatim}
ab cd # this is a comment
efg h
\end{verbatim}
describes the Forlan string $\mathsf{abcdefgh}$.
Forlan's input functions prompt with ``\texttt{@}'' when reading from
the standard input, in which case the user signifies end-of-file
by typing a line consisting of a single dot (``\texttt{.}'').

We begin this section by showing how to invoke Forlan, and giving a
quick introduction to the SML core of Forlan.  We then show how
symbols, strings, finite sets of symbols and strings, and finite
relations on symbols can be manipulated using Forlan.

\subsection{Invoking Forlan}

To invoke Forlan, type the command \texttt{forlan} to your shell
(command processor):
\begin{myalltt}
\hspace{\leftmargini}% forlan
\hspace{\leftmargini}\textsl{Forlan Version \(m\) (based on Standard ML of New Jersey Version \(n\))}
\hspace{\leftmargini}\textsl{val it = () : unit}
\hspace{\leftmargini}-
\end{myalltt}
($m$ and $n$ will be the Forlan and SML/NJ versions, respectively.)
The identifier \texttt{it} is normally bound to the value of the most
recently evaluated expression.
Initially, though, its value is the
empty tuple \texttt{()}, the single element of the type \texttt{unit}.
\index{unit@\texttt{unit}}%
\index{ empty tuple@\texttt{()}}%
The value \texttt{()} is used in circumstances when a value is required,
but it makes no difference what that value is.

Forlan's primary prompt is ``\texttt{-}''.
\index{Forlan!primary prompt}%
\index{prompt!primary}%
To exit Forlan, type \emph{CTRL}-\texttt{d}
under Linux and macOS, and \emph{CTRL}-\texttt{z} under Windows.
To interrupt back to the Forlan top-level, type \emph{CTRL}-\texttt{c}.
\index{Forlan!exiting}%
\index{Forlan!interrupting}%

On Windows, you may find it more convenient to invoke Forlan by
double-clicking on the Forlan icon.  On all platforms, a much more
flexible and satisfying way of running Forlan is as a subprocess of
the Emacs text editor.  See the Forlan website for information about
how to do this.

\subsection{The SML Core of Forlan}

This subsection gives a quick introduction to the SML core
of Forlan. Let's begin by using Forlan as a calculator:
\begin{list}{}
{\setlength{\leftmargin}{\leftmargini}
\setlength{\rightmargin}{0cm}
\setlength{\itemindent}{0cm}
\setlength{\listparindent}{0cm}
\setlength{\itemsep}{0cm}
\setlength{\parsep}{0cm}
\setlength{\labelsep}{0cm}
\setlength{\labelwidth}{0cm}
\catcode`\#=12
\catcode`\$=12
\catcode`\%=12
\catcode`\^=12
\catcode`\_=12
\catcode`\.=12
\catcode`\?=12
\catcode`\!=12
\catcode`\&=12
\ttfamily}
\small
\item[]\textsl{-\ }4\ +\ 5;
\item[]\textsl{val\ it\ =\ 9\ :\ int}
\item[]\textsl{-\ }it\ \symbol{'052}\ it;
\item[]\textsl{val\ it\ =\ 81\ :\ int}
\item[]\textsl{-\ }it\ -\ 1;
\item[]\textsl{val\ it\ =\ 80\ :\ int}
\item[]\textsl{-\ }5\ div\ 2;
\item[]\textsl{val\ it\ =\ 2\ :\ int}
\item[]\textsl{-\ }5\ mod\ 2;
\item[]\textsl{val\ it\ =\ 1\ :\ int}
\item[]\textsl{-\ }\symbol{'176}4\ +\ 2;
\item[]\textsl{val\ it\ =\ \symbol{'176}2\ :\ int}
\end{list}

Forlan responds to each expression by printing its value and type
(\texttt{int} is the type of integers),
\index{Standard ML!value}%
\index{Standard ML!type}%
\index{int@\texttt{int}}%
\index{Standard ML!int@\texttt{int}}%
and noting that the expression's value has been bound to
the identifier \texttt{it}.  Expressions must be terminated with
semicolons.
\index{Standard ML!semicolon@\texttt{;}}%
The operators \texttt{div} and \texttt{mod} compute integer division
and remainder, respectively, and negative numbers begin with
\texttt{\tildesym}.

In addition to the type \texttt{int} of integers,
\index{int@\texttt{int}}%
\index{Standard ML!int@\texttt{int}}%
SML has types \texttt{string} and \texttt{bool}, product types
$t_1\mathbin{\mathtt{*}}\cdots\mathbin{\mathtt{*}}t_n$, and list types
$t\;\mathtt{list}$.  \index{Standard ML!product type}%
\index{string@\texttt{string}}%
\index{bool@\texttt{bool}}%
\index{int@\texttt{int}}%
\index{Standard ML!string@\texttt{string}}%
\index{Standard ML!bool@\texttt{bool}}%
\index{Standard ML!int@\texttt{int}}%
\begin{list}{}
{\setlength{\leftmargin}{\leftmargini}
\setlength{\rightmargin}{0cm}
\setlength{\itemindent}{0cm}
\setlength{\listparindent}{0cm}
\setlength{\itemsep}{0cm}
\setlength{\parsep}{0cm}
\setlength{\labelsep}{0cm}
\setlength{\labelwidth}{0cm}
\catcode`\#=12
\catcode`\$=12
\catcode`\%=12
\catcode`\^=12
\catcode`\_=12
\catcode`\.=12
\catcode`\?=12
\catcode`\!=12
\catcode`\&=12
\ttfamily}
\small
\item[]\textsl{-\ }"hello"\ ^\ "\ "\ ^\ "there";
\item[]\textsl{val\ it\ =\ "hello\ there"\ :\ string}
\item[]\textsl{-\ }true\ andalso\ (false\ orelse\ true);
\item[]\textsl{val\ it\ =\ true\ :\ bool}
\item[]\textsl{-\ }if\ 5\ <\ 7\ then\ "hello"\ else\ "bye";
\item[]\textsl{val\ it\ =\ "hello"\ :\ string}
\item[]\textsl{-\ }(3\ +\ 1,\ 4\ =\ 4,\ "a"\ ^\ "b");
\item[]\textsl{val\ it\ =\ (4,true,"ab")\ :\ int\ \symbol{'052}\ bool\ \symbol{'052}\ string}
\item[]\textsl{-\ }#2\ it;
\item[]\textsl{val\ it\ =\ true\ :\ bool}
\item[]\textsl{-\ }\symbol{'133}1,\ 3,\ 5\symbol{'135}\ @\ \symbol{'133}7,\ 9,\ 11\symbol{'135};
\item[]\textsl{val\ it\ =\ \symbol{'133}1,3,5,7,9,11\symbol{'135}\ :\ int\ list}
\item[]\textsl{-\ }rev\ it;
\item[]\textsl{val\ it\ =\ \symbol{'133}11,9,7,5,3,1\symbol{'135}\ :\ int\ list}
\item[]\textsl{-\ }length\ it;
\item[]\textsl{val\ it\ =\ 6\ :\ int}
\item[]\textsl{-\ }null\symbol{'133}\symbol{'135};
\item[]\textsl{val\ it\ =\ true\ :\ bool}
\item[]\textsl{-\ }null\symbol{'133}1,\ 2\symbol{'135};
\item[]\textsl{val\ it\ =\ false\ :\ bool}
\item[]\textsl{-\ }hd\symbol{'133}1,\ 2,\ 3\symbol{'135};
\item[]\textsl{val\ it\ =\ 1\ :\ int}
\item[]\textsl{-\ }tl\symbol{'133}1,\ 2,\ 3\symbol{'135};
\item[]\textsl{val\ it\ =\ \symbol{'133}2,3\symbol{'135}\ :\ int\ list}
\end{list}

The operator \texttt{\uparrowsym} is string
concatenation.  The conjunction \texttt{andalso} evaluates its
left-hand side first, and yields \texttt{false} without evaluating its
right-hand side, if the value of the left-hand side is \texttt{false}.
Similarly, the disjunction \texttt{orelse} evaluates its left-hand
side first, and yields \texttt{true} without evaluating its right-hand
side, if the value of the left-hand side is \texttt{true}.  A
conditional (\texttt{if}-\texttt{then}-\texttt{else}) is evaluated by
first evaluating its boolean expression, and then evaluating its
\texttt{then}-part, if the boolean expression's value is
\texttt{true}, and evaluating its \texttt{else}-part, if its value is
\texttt{false}.  Tuples are evaluated from left to right, and the
function $\texttt{\#}n$ selects the $n$th (starting from $1$) element
of a tuple. The operator \texttt{@} appends lists.   The function
\texttt{rev} reverses a list, the function \texttt{length} computes
the length of a list, and the function \texttt{null} tests
whether a list is empty.  Finally, the functions \texttt{hd} and
\texttt{tl} return the head (first element) and tail (all but the
first element) of a list.

\texttt{nil} and \texttt{::} (pronounced ``cons'', for
``constructor''), which have types \texttt{'a~list} and
\texttt{'a~*~'a~list~->~'a~list}, respectively, are the constructors
for type \texttt{'a~list}.
These constructors are \emph{polymorphic}, having all of the types
that can be formed by instantiating the type variable \texttt{'a} with
a type.  E.g., \texttt{nil} has type \texttt{int~list}, \texttt{bool~list},
\texttt{(int~*~bool)list}, etc.  \texttt{::} is an infix operator,
i.e., one writes $x\mathbin{\mathtt{::}}\xs$ for the list whose
first element is $x$ and remaining elements are those in the list
$\xs$.
\begin{list}{}
{\setlength{\leftmargin}{\leftmargini}
\setlength{\rightmargin}{0cm}
\setlength{\itemindent}{0cm}
\setlength{\listparindent}{0cm}
\setlength{\itemsep}{0cm}
\setlength{\parsep}{0cm}
\setlength{\labelsep}{0cm}
\setlength{\labelwidth}{0cm}
\catcode`\#=12
\catcode`\$=12
\catcode`\%=12
\catcode`\^=12
\catcode`\_=12
\catcode`\.=12
\catcode`\?=12
\catcode`\!=12
\catcode`\&=12
\ttfamily}
\small
\item[]\textsl{-\ }nil;
\item[]\textsl{val\ it\ =\ \symbol{'133}\symbol{'135}\ :\ 'a\ list}
\item[]\textsl{-\ }1\ ::\ nil;
\item[]\textsl{val\ it\ =\ \symbol{'133}1\symbol{'135}\ :\ int\ list}
\item[]\textsl{-\ }1\ ::\ 2\ ::\ nil;
\item[]\textsl{val\ it\ =\ \symbol{'133}1,2\symbol{'135}\ :\ int\ list}
\item[]\textsl{-\ }3\ ::\ \symbol{'133}5,\ 7,\ 9\symbol{'135};
\item[]\textsl{val\ it\ =\ \symbol{'133}3,5,7,9\symbol{'135}\ :\ int\ list}
\end{list}

Lists are implemented as linked-lists, so that doing a cons involves
the creation of a single list node.

SML also has option types $t\;\mathtt{option}$, whose values are
\index{Standard ML!option type}%
built using the type's two constructors: \texttt{NONE}
\index{Standard ML!NONE@\texttt{NONE}}%
of type \texttt{'a\;option}, and \texttt{SOME}
\index{Standard ML!SOME@\texttt{NONE}}%
of type \texttt{'a~->~'a~option}.  This is a predefined
datatype, declared by
\begin{verbatim}
datatype 'a option = NONE | SOME of 'a
\end{verbatim}
E.g., \texttt{NONE}, \texttt{SOME\;1} and
\texttt{SOME\;\tildesym 6} are three of the values of type
\texttt{int\;option}, and \texttt{NONE}, \texttt{SOME\;true} and
\texttt{SOME\;false} are the only values of type
\texttt{bool\;option}.
\begin{list}{}
{\setlength{\leftmargin}{\leftmargini}
\setlength{\rightmargin}{0cm}
\setlength{\itemindent}{0cm}
\setlength{\listparindent}{0cm}
\setlength{\itemsep}{0cm}
\setlength{\parsep}{0cm}
\setlength{\labelsep}{0cm}
\setlength{\labelwidth}{0cm}
\catcode`\#=12
\catcode`\$=12
\catcode`\%=12
\catcode`\^=12
\catcode`\_=12
\catcode`\.=12
\catcode`\?=12
\catcode`\!=12
\catcode`\&=12
\ttfamily}
\small
\item[]\textsl{-\ }NONE;
\item[]\textsl{val\ it\ =\ NONE\ :\ 'a\ option}
\item[]\textsl{-\ }SOME\ 3;
\item[]\textsl{val\ it\ =\ SOME\ 3\ :\ int\ option}
\item[]\textsl{-\ }SOME\ true;
\item[]\textsl{val\ it\ =\ SOME\ true\ :\ bool\ option}
\item[]\textsl{-\ }valOf\ it;
\item[]\textsl{val\ it\ =\ true\ :\ bool}
\end{list}


In addition to the usual operators \texttt{<}, \texttt{<=}, \texttt{>}
and \texttt{>=} for comparing integers, SML offers a function
\texttt{Int.compare} of type \texttt{int~*~int~->~order}, where the
\texttt{order} type contains three elements: \texttt{LESS},
\texttt{EQUAL} and \texttt{GREATER}.  \begin{list}{}
{\setlength{\leftmargin}{\leftmargini}
\setlength{\rightmargin}{0cm}
\setlength{\itemindent}{0cm}
\setlength{\listparindent}{0cm}
\setlength{\itemsep}{0cm}
\setlength{\parsep}{0cm}
\setlength{\labelsep}{0cm}
\setlength{\labelwidth}{0cm}
\catcode`\#=12
\catcode`\$=12
\catcode`\%=12
\catcode`\^=12
\catcode`\_=12
\catcode`\.=12
\catcode`\?=12
\catcode`\!=12
\catcode`\&=12
\ttfamily}
\small
\item[]\textsl{-\ }Int.compare(3,\ 4);
\item[]\textsl{val\ it\ =\ LESS\ :\ order}
\item[]\textsl{-\ }Int.compare(4,\ 4);
\item[]\textsl{val\ it\ =\ EQUAL\ :\ order}
\item[]\textsl{-\ }Int.compare(4,\ 3);
\item[]\textsl{val\ it\ =\ GREATER\ :\ order}
\end{list}


It is possible to bind the value of an expression to an
identifier using a value declaration:
\index{Standard ML!declaration}%
\index{val@\texttt{val}}%
\begin{list}{}
{\setlength{\leftmargin}{\leftmargini}
\setlength{\rightmargin}{0cm}
\setlength{\itemindent}{0cm}
\setlength{\listparindent}{0cm}
\setlength{\itemsep}{0cm}
\setlength{\parsep}{0cm}
\setlength{\labelsep}{0cm}
\setlength{\labelwidth}{0cm}
\catcode`\#=12
\catcode`\$=12
\catcode`\%=12
\catcode`\^=12
\catcode`\_=12
\catcode`\.=12
\catcode`\?=12
\catcode`\!=12
\catcode`\&=12
\ttfamily}
\small
\item[]\textsl{-\ }val\ x\ =\ 3\ +\ 4;
\item[]\textsl{val\ x\ =\ 7\ :\ int}
\item[]\textsl{-\ }val\ y\ =\ x\ +\ 1;
\item[]\textsl{val\ y\ =\ 8\ :\ int}
\item[]\textsl{-\ }val\ x\ =\ 5\ \symbol{'052}\ x;
\item[]\textsl{val\ x\ =\ 35\ :\ int}
\item[]\textsl{-\ }y;
\item[]\textsl{val\ it\ =\ 8\ :\ int}
\end{list}

In the first declaration of \texttt{x}, its right-hand side is first
evaluated, resulting in \texttt{7}, and then \texttt{x} is bound to
this value.  Note that the redeclaration of \texttt{x} doesn't change
the value of the previous declaration of \texttt{x}, it just makes
that declaration inaccessible.

One can use a value declaration to give names to the
components of a tuple, or give a name to the data of a
non-\texttt{NONE} optional value:
\begin{list}{}
{\setlength{\leftmargin}{\leftmargini}
\setlength{\rightmargin}{0cm}
\setlength{\itemindent}{0cm}
\setlength{\listparindent}{0cm}
\setlength{\itemsep}{0cm}
\setlength{\parsep}{0cm}
\setlength{\labelsep}{0cm}
\setlength{\labelwidth}{0cm}
\catcode`\#=12
\catcode`\$=12
\catcode`\%=12
\catcode`\^=12
\catcode`\_=12
\catcode`\.=12
\catcode`\?=12
\catcode`\!=12
\catcode`\&=12
\ttfamily}
\small
\item[]\textsl{-\ }val\ (x,\ y,\ z)\ =\ (3\ +\ 1,\ 4\ =\ 4,\ "a"\ ^\ "b");
\item[]\textsl{val\ x\ =\ 4\ :\ int}
\item[]\textsl{val\ y\ =\ true\ :\ bool}
\item[]\textsl{val\ z\ =\ "ab"\ :\ string}
\item[]\textsl{-\ }val\ SOME\ n\ =\ SOME(4\ \symbol{'052}\ 25);
\item[]\textsl{stdIn:41.5-41.26\ Warning:\ binding\ not\ exhaustive}
\item[]\textsl{\ \ \ \ \ \ \ \ \ \ SOME\ n\ =\ ...}
\item[]\textsl{val\ n\ =\ 100\ :\ int}
\end{list}

This last declaration uses pattern matching: \texttt{SOME(4~*~25)}
is evaluated to $\mathtt{SOME}\;100$, and is then matched against the
pattern \texttt{SOME~n}.  Because the constructors match, the
pattern matching succeeds, and \texttt{n} becomes bound to \texttt{100}.
The warning is because the SML typechecker doesn't know the
expression won't evaluate to \texttt{NONE}.

One can use a \texttt{let} expression to carry out some declarations
in a local environment, evaluate an expression in that environment, and
yield the result of that evaluation:
\begin{list}{}
{\setlength{\leftmargin}{\leftmargini}
\setlength{\rightmargin}{0cm}
\setlength{\itemindent}{0cm}
\setlength{\listparindent}{0cm}
\setlength{\itemsep}{0cm}
\setlength{\parsep}{0cm}
\setlength{\labelsep}{0cm}
\setlength{\labelwidth}{0cm}
\catcode`\#=12
\catcode`\$=12
\catcode`\%=12
\catcode`\^=12
\catcode`\_=12
\catcode`\.=12
\catcode`\?=12
\catcode`\!=12
\catcode`\&=12
\ttfamily}
\small
\item[]\textsl{-\ }val\ x\ =\ 3;
\item[]\textsl{val\ x\ =\ 3\ :\ int}
\item[]\textsl{-\ }val\ z\ =\ 10;
\item[]\textsl{val\ z\ =\ 10\ :\ int}
\item[]\textsl{-\ }let\ val\ x\ =\ 4\ \symbol{'052}\ 5
\item[]\textsl{=\ }\ \ \ \ val\ y\ =\ x\ \symbol{'052}\ z
\item[]\textsl{=\ }in\ (x,\ y,\ x\ +\ y)\ end;
\item[]\textsl{val\ it\ =\ (20,200,220)\ :\ int\ \symbol{'052}\ int\ \symbol{'052}\ int}
\item[]\textsl{-\ }x;
\item[]\textsl{val\ it\ =\ 3\ :\ int}
\end{list}

When a declaration or expression spans more than one line, Forlan
uses its secondary prompt, \texttt{=}, on all of the lines except
\index{Forlan!secondary prompt}%
\index{prompt!secondary}%
for the first one.  Forlan doesn't process a declaration or expression
until it is terminated with a semicolon.
\index{Standard ML!semicolon@\texttt{;}}%

One can declare functions, and apply those functions to arguments:%
\index{Standard ML!function}%
\index{function}%
\begin{list}{}
{\setlength{\leftmargin}{\leftmargini}
\setlength{\rightmargin}{0cm}
\setlength{\itemindent}{0cm}
\setlength{\listparindent}{0cm}
\setlength{\itemsep}{0cm}
\setlength{\parsep}{0cm}
\setlength{\labelsep}{0cm}
\setlength{\labelwidth}{0cm}
\catcode`\#=12
\catcode`\$=12
\catcode`\%=12
\catcode`\^=12
\catcode`\_=12
\catcode`\.=12
\catcode`\?=12
\catcode`\!=12
\catcode`\&=12
\ttfamily}
\small
\item[]\textsl{-\ }fun\ f\ n\ =\ n\ \symbol{'052}\ 2;
\item[]\textsl{val\ f\ =\ fn\ :\ int\ ->\ int}
\item[]\textsl{-\ }f\ 3;
\item[]\textsl{val\ it\ =\ 6\ :\ int}
\item[]\textsl{-\ }f(4\ +\ 5);
\item[]\textsl{val\ it\ =\ 18\ :\ int}
\item[]\textsl{-\ }f\ 4\ +\ 5;
\item[]\textsl{val\ it\ =\ 13\ :\ int}
\item[]\textsl{-\ }fun\ g(x,\ y)\ =\ (x\ ^\ y,\ y\ ^\ x);
\item[]\textsl{val\ g\ =\ fn\ :\ string\ \symbol{'052}\ string\ ->\ string\ \symbol{'052}\ string}
\item[]\textsl{-\ }val\ (u,\ v)\ =\ g("a",\ "b");
\item[]\textsl{val\ u\ =\ "ab"\ :\ string}
\item[]\textsl{val\ v\ =\ "ba"\ :\ string}
\end{list}

The function \texttt{f} doubles its argument.
All function values are printed as \texttt{fn}.
\index{fn@\texttt{fn}}%
A type \texttt{$t_1$\;->\;$t_2$} is the type of all functions taking
\index{Standard ML!function type}%
arguments of type $t_1$ and producing results (if they terminate
without raising exceptions) of type $t_2$.  SML infers the types of
functions.  The function application \texttt{f(4~+~5)} is evaluated as
follows.  First, the argument \texttt{4~+~5} is evaluated, resulting
in \texttt{9}.  Then a local envionment is created in which \texttt{n}
is bound to \texttt{9}, and \texttt{f}'s body is evaluated in that
environment, producing \texttt{18}.  Function application has higher
precedence than operators like \texttt{+}.

Technically, the function \texttt{g} matches its single argument,
which must be a pair, against the pair pattern \texttt{(x,~y)},
binding \texttt{x} and \texttt{y} to the left and right sides of this
argument, and then evaluates its body.  But we can think such a
function as having multiple arguments.
\index{Standard ML!precedence}%
The type operator \texttt{*} has higher precedence
than the operator \texttt{->}.

Except for basic entities like integers and booleans, all values in
SML are represented by pointers, so that passing such a value to a
function, or putting it in a datastructure, only involves copying a
pointer.

Given functions $f$ and $g$ of types \texttt{$t_1$\;->\;$t_2$} and
\texttt{$t_2$\;->\;$t_3$}, respectively, \texttt{$g$\;o\;$f$} is the
composition
\index{Standard ML!composition}%
\index{Standard ML! composition@\texttt{o}}%
of $g$ and $f$, the function of type
\texttt{$t_1$\;->\;$t_3$} that, when given an argument $x$ of type
$t_1$, evaluates the expression \texttt{$g$($f$\;$x$)}.
For example, we have that:
\begin{list}{}
{\setlength{\leftmargin}{\leftmargini}
\setlength{\rightmargin}{0cm}
\setlength{\itemindent}{0cm}
\setlength{\listparindent}{0cm}
\setlength{\itemsep}{0cm}
\setlength{\parsep}{0cm}
\setlength{\labelsep}{0cm}
\setlength{\labelwidth}{0cm}
\catcode`\#=12
\catcode`\$=12
\catcode`\%=12
\catcode`\^=12
\catcode`\_=12
\catcode`\.=12
\catcode`\?=12
\catcode`\!=12
\catcode`\&=12
\ttfamily}
\small
\item[]\textsl{-\ }fun\ f\ x\ =\ x\ >=\ 1\ andalso\ x\ <=\ 10;
\item[]\textsl{val\ f\ =\ fn\ :\ int\ ->\ bool}
\item[]\textsl{-\ }fun\ g\ x\ =\ if\ x\ then\ "inside"\ else\ "outside";
\item[]\textsl{val\ g\ =\ fn\ :\ bool\ ->\ string}
\item[]\textsl{-\ }val\ h\ =\ g\ o\ f;
\item[]\textsl{val\ h\ =\ fn\ :\ int\ ->\ string}
\item[]\textsl{-\ }h\ \symbol{'176}5;
\item[]\textsl{val\ it\ =\ "outside"\ :\ string}
\item[]\textsl{-\ }h\ 6;
\item[]\textsl{val\ it\ =\ "inside"\ :\ string}
\item[]\textsl{-\ }h\ 14;
\item[]\textsl{val\ it\ =\ "outside"\ :\ string}
\end{list}


SML also has anonymous functions, which may also be given names
using value declarations:
\begin{list}{}
{\setlength{\leftmargin}{\leftmargini}
\setlength{\rightmargin}{0cm}
\setlength{\itemindent}{0cm}
\setlength{\listparindent}{0cm}
\setlength{\itemsep}{0cm}
\setlength{\parsep}{0cm}
\setlength{\labelsep}{0cm}
\setlength{\labelwidth}{0cm}
\catcode`\#=12
\catcode`\$=12
\catcode`\%=12
\catcode`\^=12
\catcode`\_=12
\catcode`\.=12
\catcode`\?=12
\catcode`\!=12
\catcode`\&=12
\ttfamily}
\small
\item[]\textsl{-\ }(fn\ x\ =>\ x\ +\ 1)(3\ +\ 4);
\item[]\textsl{val\ it\ =\ 8\ :\ int}
\item[]\textsl{-\ }val\ f\ =\ fn\ x\ =>\ x\ +\ 1;
\item[]\textsl{val\ f\ =\ fn\ :\ int\ ->\ int}
\item[]\textsl{-\ }f(3\ +\ 4);
\item[]\textsl{val\ it\ =\ 8\ :\ int}
\end{list}

The anonymous function \texttt{fn x => x + 1} has type
\texttt{int~->~int} and adds one to its argument.

Functions are data: they may be passed to functions, returned from
functions (a function that returns a function is called
\emph{curried}), be components of tuples or lists, etc.
For example,
\begin{verbatim}
val map : ('a -> 'b) -> 'a list -> 'b list
\end{verbatim}
is a polymorphic, curried function.  The type operator \texttt{->}
associates
\index{Standard ML!associativity}%
to the right, so that \texttt{map}'s type is
\begin{verbatim}
val map : ('a -> 'b) -> ('a list -> 'b list)
\end{verbatim}
\texttt{map} takes in a function
$f$ of type \texttt{'a~->~'b}, and returns a function that when called
with a list of elements of type \texttt{'a}, transforms each element
using $f$, forming a list of elements of type \texttt{'b}.
\begin{list}{}
{\setlength{\leftmargin}{\leftmargini}
\setlength{\rightmargin}{0cm}
\setlength{\itemindent}{0cm}
\setlength{\listparindent}{0cm}
\setlength{\itemsep}{0cm}
\setlength{\parsep}{0cm}
\setlength{\labelsep}{0cm}
\setlength{\labelwidth}{0cm}
\catcode`\#=12
\catcode`\$=12
\catcode`\%=12
\catcode`\^=12
\catcode`\_=12
\catcode`\.=12
\catcode`\?=12
\catcode`\!=12
\catcode`\&=12
\ttfamily}
\small
\item[]\textsl{-\ }val\ f\ =\ map(fn\ x\ =>\ x\ +\ 1);
\item[]\textsl{val\ f\ =\ fn\ :\ int\ list\ ->\ int\ list}
\item[]\textsl{-\ }f\symbol{'133}2,\ 4,\ 6\symbol{'135};
\item[]\textsl{val\ it\ =\ \symbol{'133}3,5,7\symbol{'135}\ :\ int\ list}
\item[]\textsl{-\ }f\symbol{'133}\symbol{'176}2,\ \symbol{'176}1,\ 0\symbol{'135};
\item[]\textsl{val\ it\ =\ \symbol{'133}\symbol{'176}1,0,1\symbol{'135}\ :\ int\ list}
\item[]\textsl{-\ }map\ (fn\ x\ =>\ x\ mod\ 2\ =\ 1)\ \symbol{'133}3,\ 4,\ 5,\ 6,\ 7\symbol{'135};
\item[]\textsl{val\ it\ =\ \symbol{'133}true,false,true,false,true\symbol{'135}\ :\ bool\ list}
\end{list}

In the last use of \texttt{map}, we are using the fact that
function application associates to the left, so that
\index{Standard ML!associativity}%
$f\,x\,y$ means $(f\,x)y$, i.e., apply $f$ to $x$, and
then apply the resulting function to $y$.

The following example shows that local envionments are kept alive as
as long as there are accessible function values referring to them:
\begin{list}{}
{\setlength{\leftmargin}{\leftmargini}
\setlength{\rightmargin}{0cm}
\setlength{\itemindent}{0cm}
\setlength{\listparindent}{0cm}
\setlength{\itemsep}{0cm}
\setlength{\parsep}{0cm}
\setlength{\labelsep}{0cm}
\setlength{\labelwidth}{0cm}
\catcode`\#=12
\catcode`\$=12
\catcode`\%=12
\catcode`\^=12
\catcode`\_=12
\catcode`\.=12
\catcode`\?=12
\catcode`\!=12
\catcode`\&=12
\ttfamily}
\small
\item[]\textsl{-\ }val\ f\ =
\item[]\textsl{=\ }\ \ \ \ \ \ let\ val\ x\ =\ 2\ \symbol{'052}\ 10
\item[]\textsl{=\ }\ \ \ \ \ \ in\ fn\ y\ =>\ y\ \symbol{'052}\ x\ end;
\item[]\textsl{val\ f\ =\ fn\ :\ int\ ->\ int}
\item[]\textsl{-\ }f\ 4;
\item[]\textsl{val\ it\ =\ 80\ :\ int}
\item[]\textsl{-\ }f\ 7;
\item[]\textsl{val\ it\ =\ 140\ :\ int}
\end{list}

If the local environment containing the binding of \texttt{x} was
discarded, then calling \texttt{f} would fail.

It's also possible to declare recursive functions, like the
\index{Standard ML!function!recursive}%
factorial function:
\begin{list}{}
{\setlength{\leftmargin}{\leftmargini}
\setlength{\rightmargin}{0cm}
\setlength{\itemindent}{0cm}
\setlength{\listparindent}{0cm}
\setlength{\itemsep}{0cm}
\setlength{\parsep}{0cm}
\setlength{\labelsep}{0cm}
\setlength{\labelwidth}{0cm}
\catcode`\#=12
\catcode`\$=12
\catcode`\%=12
\catcode`\^=12
\catcode`\_=12
\catcode`\.=12
\catcode`\?=12
\catcode`\!=12
\catcode`\&=12
\ttfamily}
\small
\item[]\textsl{-\ }fun\ fact\ n\ =
\item[]\textsl{=\ }\ \ \ \ \ \ if\ n\ =\ 0
\item[]\textsl{=\ }\ \ \ \ \ \ then\ 1
\item[]\textsl{=\ }\ \ \ \ \ \ else\ n\ \symbol{'052}\ fact(n\ -\ 1);
\item[]\textsl{val\ fact\ =\ fn\ :\ int\ ->\ int}
\item[]\textsl{-\ }fact\ 4;
\item[]\textsl{val\ it\ =\ 24\ :\ int}
\end{list}

One can load the contents of a file into Forlan using the function
\begin{verbatim}
val use : string -> unit
\end{verbatim}
\index{use@\texttt{use}}%
For example, if the file \url{fact.sml} contains the declaration of
the factorial function, then this declaration can be loaded into the
system as follows:
\begin{list}{}
{\setlength{\leftmargin}{\leftmargini}
\setlength{\rightmargin}{0cm}
\setlength{\itemindent}{0cm}
\setlength{\listparindent}{0cm}
\setlength{\itemsep}{0cm}
\setlength{\parsep}{0cm}
\setlength{\labelsep}{0cm}
\setlength{\labelwidth}{0cm}
\catcode`\#=12
\catcode`\$=12
\catcode`\%=12
\catcode`\^=12
\catcode`\_=12
\catcode`\.=12
\catcode`\?=12
\catcode`\!=12
\catcode`\&=12
\ttfamily}
\small
\item[]\textsl{-\ }use\ "fact.sml";
\item[]\textsl{\symbol{'133}opening\ fact.sml\symbol{'135}}
\item[]\textsl{val\ fact\ =\ fn\ :\ int\ ->\ int}
\item[]\textsl{val\ it\ =\ ()\ :\ unit}
\item[]\textsl{-\ }fact\ 4;
\item[]\textsl{val\ it\ =\ 24\ :\ int}
\end{list}


The factorial function can also be defined using pattern matching,
either by using a case expression in the body of the function, or
by using multiple clauses in the function's definition:
\begin{list}{}
{\setlength{\leftmargin}{\leftmargini}
\setlength{\rightmargin}{0cm}
\setlength{\itemindent}{0cm}
\setlength{\listparindent}{0cm}
\setlength{\itemsep}{0cm}
\setlength{\parsep}{0cm}
\setlength{\labelsep}{0cm}
\setlength{\labelwidth}{0cm}
\catcode`\#=12
\catcode`\$=12
\catcode`\%=12
\catcode`\^=12
\catcode`\_=12
\catcode`\.=12
\catcode`\?=12
\catcode`\!=12
\catcode`\&=12
\ttfamily}
\small
\item[]\textsl{-\ }fun\ fact\ n\ =
\item[]\textsl{=\ }\ \ \ \ \ \ case\ n\ of
\item[]\textsl{=\ }\ \ \ \ \ \ \ \ \ \ \ 0\ =>\ 1
\item[]\textsl{=\ }\ \ \ \ \ \ \ \ \ |\ n\ =>\ n\ \symbol{'052}\ fact(n\ -\ 1);
\item[]\textsl{val\ fact\ =\ fn\ :\ int\ ->\ int}
\item[]\textsl{-\ }fact\ 3;
\item[]\textsl{val\ it\ =\ 6\ :\ int}
\item[]\textsl{-\ }fun\ fact\ 0\ =\ 1
\item[]\textsl{=\ }\ \ |\ fact\ n\ =\ n\ \symbol{'052}\ fact(n\ -\ 1);
\item[]\textsl{val\ fact\ =\ fn\ :\ int\ ->\ int}
\item[]\textsl{-\ }fact\ 4;
\item[]\textsl{val\ it\ =\ 24\ :\ int}
\end{list}

The order of the clauses of a case expression or function definition
is significant.  If the clauses of either the case expression or
the function definition were reversed, the function being defined
would never return.

Pattern matching is especially useful when doing list processing.
E.g., we could (inefficiently) define the list reversal function like
this:
\begin{list}{}
{\setlength{\leftmargin}{\leftmargini}
\setlength{\rightmargin}{0cm}
\setlength{\itemindent}{0cm}
\setlength{\listparindent}{0cm}
\setlength{\itemsep}{0cm}
\setlength{\parsep}{0cm}
\setlength{\labelsep}{0cm}
\setlength{\labelwidth}{0cm}
\catcode`\#=12
\catcode`\$=12
\catcode`\%=12
\catcode`\^=12
\catcode`\_=12
\catcode`\.=12
\catcode`\?=12
\catcode`\!=12
\catcode`\&=12
\ttfamily}
\small
\item[]\textsl{-\ }fun\ rev\ nil\ \ \ \ \ \ \ =\ nil
\item[]\textsl{=\ }\ \ |\ rev\ (x\ ::\ xs)\ =\ rev\ xs\ @\ \symbol{'133}x\symbol{'135};
\item[]\textsl{val\ rev\ =\ fn\ :\ 'a\ list\ ->\ 'a\ list}
\end{list}

Calling \texttt{rev} with the empty list will result in the empty
list being returned.  And calling it with a nonempty list will
temporarily bind \texttt{x} to the list's head, bind \texttt{xs} to its
tail, and then evaluate the expression \texttt{rev~xs~@~[x]},
recursively calling \texttt{rev}, and then returning the result of
appending the result of this recursive call and \texttt{[x]}.
Unfortunately, this definition of \texttt{rev} is slow for long lists,
as \texttt{@} rebuilds its left argument. Instead, the official
definition of \texttt{rev} is much more efficient:
\begin{list}{}
{\setlength{\leftmargin}{\leftmargini}
\setlength{\rightmargin}{0cm}
\setlength{\itemindent}{0cm}
\setlength{\listparindent}{0cm}
\setlength{\itemsep}{0cm}
\setlength{\parsep}{0cm}
\setlength{\labelsep}{0cm}
\setlength{\labelwidth}{0cm}
\catcode`\#=12
\catcode`\$=12
\catcode`\%=12
\catcode`\^=12
\catcode`\_=12
\catcode`\.=12
\catcode`\?=12
\catcode`\!=12
\catcode`\&=12
\ttfamily}
\small
\item[]\textsl{-\ }fun\ rev\ xs\ =
\item[]\textsl{=\ }\ \ \ \ \ \ let\ fun\ rv(nil,\ \ \ \ \ vs)\ =\ vs
\item[]\textsl{=\ }\ \ \ \ \ \ \ \ \ \ \ \ |\ rv(u\ ::\ us,\ vs)\ =\ rv(us,\ u\ ::\ vs)
\item[]\textsl{=\ }\ \ \ \ \ \ in\ rv(xs,\ nil)\ end;
\item[]\textsl{val\ rev\ =\ fn\ :\ 'a\ list\ ->\ 'a\ list}
\item[]\textsl{-\ }rev\ \symbol{'133}1,\ 2,\ 3,\ 4\symbol{'135};
\item[]\textsl{val\ it\ =\ \symbol{'133}4,3,2,1\symbol{'135}\ :\ int\ list}
\end{list}

When \texttt{rev} is called with \texttt{[1, 2, 3, 4]}, it
calls the auxiliary function \texttt{rv} with the pair
\texttt{([1, 2, 3, 4], [])}. The recursive calls that
\texttt{rv} makes to itself are a special kind of recursion
called \emph{tail recursion}.
\index{Standard ML!tail recursion}%
\index{tail recursion}%
The SML/NJ compiler generates
code for such calls that simply jumps back to the beginning of \texttt{rv},
only changing its arguments.
We have the following sequence of calls: \texttt{rv([1, 2, 3, 4], [])}
calls \texttt{rv([2, 3, 4], [1])} calls \texttt{rv([3, 4], [2, 1])}
calls \texttt{rv([4], [3, 2, 1])} calls \texttt{rv([], [4, 3, 2, 1])},
which returns \texttt{[4, 2, 2, 1]}, which is returned as the result
of \texttt{rev}.

\index{Standard ML!recursive datatype}%
\index{Standard ML!tree}%
Finally, we can define recursive datatypes, and define functions
by structural recursion on recursive datatypes. E.g., here's how
we can define the datatype of labeled binary trees (both leaves and
nodes (non-leaves) can have labels):
\begin{list}{}
{\setlength{\leftmargin}{\leftmargini}
\setlength{\rightmargin}{0cm}
\setlength{\itemindent}{0cm}
\setlength{\listparindent}{0cm}
\setlength{\itemsep}{0cm}
\setlength{\parsep}{0cm}
\setlength{\labelsep}{0cm}
\setlength{\labelwidth}{0cm}
\catcode`\#=12
\catcode`\$=12
\catcode`\%=12
\catcode`\^=12
\catcode`\_=12
\catcode`\.=12
\catcode`\?=12
\catcode`\!=12
\catcode`\&=12
\ttfamily}
\small
\item[]\textsl{-\ }datatype\ ('a,\ 'b)\ tree\ =
\item[]\textsl{=\ }\ \ \ \ Leaf\ of\ 'b
\item[]\textsl{=\ }\ \ |\ Node\ of\ 'a\ \symbol{'052}\ ('a,\ 'b)\ tree\ \symbol{'052}\ ('a,\ 'b)\ tree;
\item[]\textsl{datatype\ ('a,'b)\ tree}
\item[]\textsl{\ \ =\ Leaf\ of\ 'b\ |\ Node\ of\ 'a\ \symbol{'052}\ ('a,'b)\ tree\ \symbol{'052}\ ('a,'b)\ tree}
\item[]\textsl{-\ }Leaf;
\item[]\textsl{val\ it\ =\ fn\ :\ 'a\ ->\ ('b,'a)\ tree}
\item[]\textsl{-\ }Node;
\item[]\textsl{val\ it\ =\ fn\ :\ 'a\ \symbol{'052}\ ('a,'b)\ tree\ \symbol{'052}\ ('a,'b)\ tree\ ->\ ('a,'b)\ tree}
\item[]\textsl{-\ }val\ tr\ =\ Node(true,\ Node(false,\ Leaf\ 7,\ Leaf\ \symbol{'176}1),\ Leaf\ 8);
\item[]\textsl{val\ tr\ =\ Node\ (true,Node\ (false,Leaf\ 7,Leaf\ \symbol{'176}1),Leaf\ 8)}
\item[]\textsl{\ \ :\ (bool,int)\ tree}
\end{list}

Then we can define a function for reversing a tree, and apply it to
\texttt{tr}:
\begin{list}{}
{\setlength{\leftmargin}{\leftmargini}
\setlength{\rightmargin}{0cm}
\setlength{\itemindent}{0cm}
\setlength{\listparindent}{0cm}
\setlength{\itemsep}{0cm}
\setlength{\parsep}{0cm}
\setlength{\labelsep}{0cm}
\setlength{\labelwidth}{0cm}
\catcode`\#=12
\catcode`\$=12
\catcode`\%=12
\catcode`\^=12
\catcode`\_=12
\catcode`\.=12
\catcode`\?=12
\catcode`\!=12
\catcode`\&=12
\ttfamily}
\small
\item[]\textsl{-\ }fun\ revTree\ (Leaf\ n)\ \ \ \ \ \ \ \ \ \ \ \ =\ Leaf\ n
\item[]\textsl{=\ }\ \ |\ revTree\ (Node(m,\ tr1,\ tr2))\ =
\item[]\textsl{=\ }\ \ \ \ \ \ Node(m,\ revTree\ tr2,\ revTree\ tr1);
\item[]\textsl{val\ revTree\ =\ fn\ :\ ('a,'b)\ tree\ ->\ ('a,'b)\ tree}
\item[]\textsl{-\ }revTree\ tr;
\item[]\textsl{val\ it\ =\ Node\ (true,Leaf\ 8,Node\ (false,Leaf\ \symbol{'176}1,Leaf\ 7))}
\item[]\textsl{\ \ :\ (bool,int)\ tree}
\item[]\textsl{-\ }revTree\ it;
\item[]\textsl{val\ it\ =\ Node\ (true,Node\ (false,Leaf\ 7,Leaf\ \symbol{'176}1),Leaf\ 8)}
\item[]\textsl{\ \ :\ (bool,int)\ tree}
\end{list}

And here is how we can define mutually recursive datatypes, along with
mutually recursive functions over them:
\begin{list}{}
{\setlength{\leftmargin}{\leftmargini}
\setlength{\rightmargin}{0cm}
\setlength{\itemindent}{0cm}
\setlength{\listparindent}{0cm}
\setlength{\itemsep}{0cm}
\setlength{\parsep}{0cm}
\setlength{\labelsep}{0cm}
\setlength{\labelwidth}{0cm}
\catcode`\#=12
\catcode`\$=12
\catcode`\%=12
\catcode`\^=12
\catcode`\_=12
\catcode`\.=12
\catcode`\?=12
\catcode`\!=12
\catcode`\&=12
\ttfamily}
\small
\item[]\textsl{-\ }datatype\ a\ =
\item[]\textsl{=\ }\ \ \ \ A1\ of\ int
\item[]\textsl{=\ }\ \ |\ A2\ of\ b\ \symbol{'052}\ b
\item[]\textsl{=\ }and\ \ \ \ \ \ b\ =
\item[]\textsl{=\ }\ \ \ \ B1\ of\ bool
\item[]\textsl{=\ }\ \ |\ B2\ of\ a\ \symbol{'052}\ a;
\item[]\textsl{datatype\ a\ =\ A1\ of\ int\ |\ A2\ of\ b\ \symbol{'052}\ b}
\item[]\textsl{datatype\ b\ =\ B1\ of\ bool\ |\ B2\ of\ a\ \symbol{'052}\ a}
\item[]\textsl{-\ }fun\ rev_a\ (A1\ n)\ \ \ \ \ =\ A1\ n
\item[]\textsl{=\ }\ \ |\ rev_a\ (A2(x,\ y))\ =\ A2(rev_b\ y,\ rev_b\ x)
\item[]\textsl{=\ }and\ rev_b\ (B1\ b)\ \ \ \ \ =\ B1\ b
\item[]\textsl{=\ }\ \ |\ rev_b\ (B2(x,\ y))\ =\ B2(rev_a\ y,\ rev_a\ x);
\item[]\textsl{val\ rev_a\ =\ fn\ :\ a\ ->\ a}
\item[]\textsl{val\ rev_b\ =\ fn\ :\ b\ ->\ b}
\item[]\textsl{-\ }val\ a\ =\ A2(B2(A1\ 3,\ A1\ 4),\ B1\ true);
\item[]\textsl{val\ a\ =\ A2\ (B2\ (A1\ 3,A1\ 4),B1\ true)\ :\ a}
\item[]\textsl{-\ }rev_a\ a;
\item[]\textsl{val\ it\ =\ A2\ (B1\ true,B2\ (A1\ 4,A1\ 3))\ :\ a}
\end{list}

\index{Standard ML|)}%

\subsection{Symbols}

The Forlan module \texttt{Sym}
\index{Sym@\texttt{Sym}}%
defines the abstract type \texttt{sym}
\index{sym@\texttt{sym}}%
\index{Sym@\texttt{Sym}!sym@\texttt{sym}}%
of Forlan symbols,
\index{symbol}%
as well as some functions for processing symbols, including:
\begin{verbatim}
val input   : string -> sym
val output  : string * sym -> unit
val compare : sym * sym -> order
val equal   : string * string -> bool
\end{verbatim}
\index{Sym@\texttt{Sym}!input@\texttt{input}}%
\index{Sym@\texttt{Sym}!output@\texttt{output}}%
\index{Sym@\texttt{Sym}!compare@\texttt{compare}}%
Symbols are expressed in Forlan's syntax as sequences of symbol
characters, i.e., as $\mathsf{a}$ or $\langle\mathsf{id}\rangle$,
rather than $[\mathsf{a}]$ or $\mathsf{[\,\langle,\,i,\,d,\,\rangle]}$.
The above functions behave as follows:
\begin{itemize}
\item $\mathtt{input}\,\mathit{fil}$ reads a symbol from file
$\mathit{fil}$; if $\mathit{fil} = \texttt{""}$, then the symbol is
read from the standard input;

\item \texttt{output($\mathit{fil}$,\,$a$)} writes the symbol $a$ to
the file $\mathit{fil}$; if $\mathit{fil} = \texttt{""}$, then the
string is written to the standard output;

\item \texttt{compare} implements our total ordering on symbols; and

\item \texttt{equal} tests whether two symbols are equal.
\end{itemize}
All of Forlan's input functions read from the standard input
when called with \texttt{""} instead of a file, and all of Forlan's
output functions write to the standard output when given
\texttt{""} instead of a file.

The type \texttt{sym} is bound in the top-level environment.  On the
other hand, one must write $\mathtt{Sym.}f$ to select the function $f$
of module \texttt{Sym}.
As described above, interactive input is terminated by a
line consisting of a single ``\texttt{.}'' (dot), and Forlan's
input prompt is ``\texttt{@}''.
\index{interactive input}%
\index{Forlan!input prompt}%
\index{prompt!input}%

The module \texttt{Sym} also provides the functions
\begin{verbatim}
val fromString : string -> sym
val toString   : sym -> string
\end{verbatim}
\index{Sym@\texttt{Sym}!fromString@\texttt{fromString}}%
\index{Sym@\texttt{Sym}!toString@\texttt{toString}}%
where \texttt{fromString} is like \texttt{input}, except that it takes
its input from a string, and \texttt{toString} is like
\texttt{output}, except that it writes its output to a string.  These
functions are especially useful when defining functions.  In the
sequel, whenever a module/type has \texttt{input} and \texttt{output}
functions, you may assume that it also has \texttt{fromString} and
\texttt{toString} functions.

Here are some example uses of the functions of \texttt{Sym}:
\begin{list}{}
{\setlength{\leftmargin}{\leftmargini}
\setlength{\rightmargin}{0cm}
\setlength{\itemindent}{0cm}
\setlength{\listparindent}{0cm}
\setlength{\itemsep}{0cm}
\setlength{\parsep}{0cm}
\setlength{\labelsep}{0cm}
\setlength{\labelwidth}{0cm}
\catcode`\#=12
\catcode`\$=12
\catcode`\%=12
\catcode`\^=12
\catcode`\_=12
\catcode`\.=12
\catcode`\?=12
\catcode`\!=12
\catcode`\&=12
\ttfamily}
\small
\item[]\textsl{-\ }val\ a\ =\ Sym.input\ "";
\item[]\textsl{@\ }<i
\item[]\textsl{@\ }d>
\item[]\textsl{@\ }.
\item[]\textsl{val\ a\ =\ -\ :\ sym}
\item[]\textsl{-\ }val\ b\ =\ Sym.fromString\ "<num>";
\item[]\textsl{val\ b\ =\ -\ :\ sym}
\item[]\textsl{-\ }Sym.output("",\ a);
\item[]\textsl{<id>}
\item[]\textsl{val\ it\ =\ ()\ :\ unit}
\item[]\textsl{-\ }Sym.compare(a,\ b);
\item[]\textsl{val\ it\ =\ LESS\ :\ order}
\item[]\textsl{-\ }Sym.equal(a,\ b);
\item[]\textsl{val\ it\ =\ false\ :\ bool}
\item[]\textsl{-\ }Sym.equal(a,\ Sym.fromString\ "<id>");
\item[]\textsl{val\ it\ =\ true\ :\ bool}
\end{list}

Values of abstract types (like \texttt{sym}) are printed
as ``\texttt{-}''.

\subsection{Sets}

The module \texttt{Set}
\index{Set@\texttt{Set}}%
defines the abstract type
\begin{verbatim}
type 'a set
\end{verbatim}
\index{set@\texttt{\primesym a~set}}%
\index{Set@\texttt{Set}!set@\texttt{\primesym a~set}}%
of finite sets
\index{set!finite}%
of elements of type \texttt{\primesym a}.  It is bound in the
top-level environment.  E.g., \texttt{sym\;set} is the type of sets of
symbols.

Each set has an associated total ordering, and some of the functions
of \texttt{Set} take total orderings as arguments.  See the Forlan
manual for the details.  In the book, we won't have to work with
such functions explicitly.

\texttt{Set} provides various constants and functions for
processing sets, but we will only make direct use of a few of them:
\begin{verbatim}
val toList  : 'a set -> 'a list
val size    : 'a set -> int
val empty   : 'a set
val isEmpty : 'a set -> bool
val sing    : 'a -> 'a set
val filter  : ('a -> bool) -> 'a set -> 'a set
val all     : ('a -> bool) -> 'a set -> bool
val exists  : ('a -> bool) -> 'a set -> bool
\end{verbatim}
\index{Set@\texttt{Set}!toList@\texttt{toList}}%
\index{Set@\texttt{Set}!size@\texttt{size}}%
\index{Set@\texttt{Set}!empty@\texttt{empty}}%
\index{Set@\texttt{Set}!notEmpty@\texttt{notEmpty}}%
\index{Set@\texttt{Set}!sing@\texttt{sing}}%
\index{Set@\texttt{Set}!filter@\texttt{filter}}%
\index{Set@\texttt{Set}!all@\texttt{all}}%
\index{Set@\texttt{Set}!exists@\texttt{exists}}%
These values are polymorphic: \texttt{'a} can be \texttt{int},
\texttt{sym}, etc.
The function \texttt{toList} returns the elements of a set, listing
them in ascending order, according to the set's total ordering.
The function \texttt{size} returns the size of a set.
The value \texttt{empty} is the empty set, and the function
\texttt{isEmpty} checks whether a set is empty.
The function \texttt{sing} makes a value $x$ into the singleton
set $\{x\}$. The function \texttt{filter} goes through the elements of
a set, keeping those elements on which the supplied predicate function
returns \texttt{true}. The function \texttt{all} checks whether all
elements of a set satisfy a predicate, whereas the function
\texttt{exists} checks whether at least one element of a set satisfies
the predicate.

\subsection{Sets of Symbols}

The module \texttt{SymSet}
\index{SymSet@\texttt{SymSet}}%
defines various functions for processing
finite sets of symbols (elements of type \texttt{sym\;set};
\index{set@\texttt{\primesym a~set}}%
alphabets), including:
\begin{verbatim}
val input    : string -> sym set
val output   : string * sym set -> unit
val fromList : sym list -> sym set
val memb     : sym * sym set -> bool
val subset   : sym set * sym set -> bool
val equal    : sym set * sym set -> bool
val union    : sym set * sym set -> sym set
val inter    : sym set * sym set -> sym set
val minus    : sym set * sym set -> sym set
val genUnion : sym set list -> sym set
val genInter : sym set list -> sym set
\end{verbatim}
\index{SymSet@\texttt{SymSet}!input@\texttt{input}}%
\index{SymSet@\texttt{SymSet}!output@\texttt{output}}%
\index{SymSet@\texttt{SymSet}!fromList@\texttt{fromList}}%
\index{SymSet@\texttt{SymSet}!memb@\texttt{memb}}%
\index{SymSet@\texttt{SymSet}!subset@\texttt{subset}}%
\index{SymSet@\texttt{SymSet}!equal@\texttt{equal}}%
\index{SymSet@\texttt{SymSet}!union@\texttt{union}}%
\index{SymSet@\texttt{SymSet}!inter@\texttt{inter}}%
\index{SymSet@\texttt{SymSet}!minus@\texttt{minus}}%
\index{SymSet@\texttt{SymSet}!genUnion@\texttt{genUnion}}%
\index{SymSet@\texttt{SymSet}!genInter@\texttt{genInter}}%
The total ordering associated with sets of symbols is our total
ordering on symbols.  Sets of symbols are expressed in Forlan's syntax
as sequences of symbols, separated by commas.

The function \texttt{fromList} returns a set with the same elements of
the list of symbols it is called with.  The function \texttt{memb}
tests whether a symbol is a member (element) of a set of symbols,
\texttt{subset} tests whether a first set of symbols is a subset of a
second one, and \texttt{equal} tests whether two sets of symbols are
equal.  The functions \texttt{union}, \texttt{inter} and
\texttt{minus} compute the union, intersection and difference of two
sets of symbols.  The function \texttt{genUnion} computes the
generalized intersection of a list of sets of symbols $\xss$,
returning the set of all symbols appearing in at least one element of
$\xss$.  And, the function \texttt{genInter} computes the generalized
intersection of a nonempty list of sets of symbols $\xss$, returning
the set of all symbols appearing in all elements of $\xss$.

Here are some example uses of the functions of \texttt{SymSet}:
\begin{list}{}
{\setlength{\leftmargin}{\leftmargini}
\setlength{\rightmargin}{0cm}
\setlength{\itemindent}{0cm}
\setlength{\listparindent}{0cm}
\setlength{\itemsep}{0cm}
\setlength{\parsep}{0cm}
\setlength{\labelsep}{0cm}
\setlength{\labelwidth}{0cm}
\catcode`\#=12
\catcode`\$=12
\catcode`\%=12
\catcode`\^=12
\catcode`\_=12
\catcode`\.=12
\catcode`\?=12
\catcode`\!=12
\catcode`\&=12
\ttfamily}
\small
\item[]\textsl{-\ }val\ bs\ =\ SymSet.input\ "";
\item[]\textsl{@\ }a,\ <id>,\ 0,\ <num>
\item[]\textsl{@\ }.
\item[]\textsl{val\ bs\ =\ -\ :\ sym\ set}
\item[]\textsl{-\ }SymSet.output("",\ bs);
\item[]\textsl{0,\ a,\ <id>,\ <num>}
\item[]\textsl{val\ it\ =\ ()\ :\ unit}
\item[]\textsl{-\ }val\ cs\ =\ SymSet.input\ "";\ 
\item[]\textsl{@\ }a,\ <char>\ 
\item[]\textsl{@\ }.
\item[]\textsl{val\ cs\ =\ -\ :\ sym\ set}
\item[]\textsl{-\ }SymSet.subset(cs,\ bs);
\item[]\textsl{val\ it\ =\ false\ :\ bool}
\item[]\textsl{-\ }SymSet.output("",\ SymSet.union(bs,\ cs));
\item[]\textsl{0,\ a,\ <id>,\ <num>,\ <char>}
\item[]\textsl{val\ it\ =\ ()\ :\ unit}
\item[]\textsl{-\ }SymSet.output("",\ SymSet.inter(bs,\ cs));
\item[]\textsl{a}
\item[]\textsl{val\ it\ =\ ()\ :\ unit}
\item[]\textsl{-\ }SymSet.output("",\ SymSet.minus(bs,\ cs));
\item[]\textsl{0,\ <id>,\ <num>}
\item[]\textsl{val\ it\ =\ ()\ :\ unit}
\item[]\textsl{-\ }val\ ds\ =\ SymSet.fromString\ "<char>,\ <>";
\item[]\textsl{val\ ds\ =\ -\ :\ sym\ set}
\item[]\textsl{-\ }SymSet.output("",\ SymSet.genUnion\symbol{'133}bs,\ cs,\ ds\symbol{'135});
\item[]\textsl{0,\ a,\ <>,\ <id>,\ <num>,\ <char>}
\item[]\textsl{val\ it\ =\ ()\ :\ unit}
\item[]\textsl{-\ }SymSet.output("",\ SymSet.genInter\symbol{'133}bs,\ cs,\ ds\symbol{'135});
\item[]
\item[]\textsl{val\ it\ =\ ()\ :\ unit}
\end{list}


\subsection{Strings}

We will be working with two kinds of strings:
\begin{itemize}
\item SML strings, i.e., elements of type \texttt{string};

\item The strings of formal language theory, which we call
``formal language strings'', when necessary.
\end{itemize}

The module \texttt{Str}
\index{Str@\texttt{Str}}%
defines the type \texttt{str} of formal language
\index{Str@\texttt{Str}!str@\texttt{str}}%
\index{str@\texttt{str}}%
\index{string}%
strings, which is bound in the top-level environment, and is
equal to \texttt{sym\;list}, the type of lists of symbols.
Because strings are lists, we can use SML's list processing functions
on them.
Strings are expressed in Forlan's syntax as either a
single \texttt{\%} or a nonempty sequence of symbols.
\index{string!Forlan syntax}%
\index{Forlan!string syntax}%

The module \texttt{Str} also defines some functions for processing strings,
including:
\begin{verbatim}
val input      : string -> str
val output     : string * str -> unit
val alphabet   : str -> sym set
val compare    : str * str -> order
val equal      : str * str -> bool
val prefix     : str * str -> bool
val suffix     : str * str -> bool
val substr     : str * str -> bool
val power      : str * int -> str
val last       : str -> sym
val allButLast : str -> str
\end{verbatim}
\index{Str@\texttt{Str}!input@\texttt{input}}%
\index{Str@\texttt{Str}!output@\texttt{output}}%
\index{Str@\texttt{Str}!alphabet@\texttt{alphabet}}%
\index{Str@\texttt{Str}!compare@\texttt{compare}}%
\index{Str@\texttt{Str}!prefix@\texttt{prefix}}%
\index{Str@\texttt{Str}!suffix@\texttt{suffix}}%
\index{Str@\texttt{Str}!substr@\texttt{substr}}%
\index{Str@\texttt{Str}!power@\texttt{power}}%
\index{Str@\texttt{Str}!last@\texttt{last}}%
\index{Str@\texttt{Str}!allButLast@\texttt{allButLast}}%

The function \texttt{alphabet} returns the alphabet of a string,
and \texttt{compare} implements our total ordering on
strings.  \texttt{prefix($x$,\;$y$)} tests whether $x$ is a prefix of
$y$, and \texttt{suffix} and \texttt{substring} work similarly.
\texttt{power($x$,\;$n$)} raises $x$ to the power $n$.  And
\texttt{last} and \texttt{allButLast} return the last symbol and all
but the last symbol of a string, respectively.

Here are some example uses of the functions of \texttt{Str}:
\begin{list}{}
{\setlength{\leftmargin}{\leftmargini}
\setlength{\rightmargin}{0cm}
\setlength{\itemindent}{0cm}
\setlength{\listparindent}{0cm}
\setlength{\itemsep}{0cm}
\setlength{\parsep}{0cm}
\setlength{\labelsep}{0cm}
\setlength{\labelwidth}{0cm}
\catcode`\#=12
\catcode`\$=12
\catcode`\%=12
\catcode`\^=12
\catcode`\_=12
\catcode`\.=12
\catcode`\?=12
\catcode`\!=12
\catcode`\&=12
\ttfamily}
\small
\item[]\textsl{-\ }val\ x\ =\ Str.input\ "";
\item[]\textsl{@\ }hello<there>
\item[]\textsl{@\ }.
\item[]\textsl{val\ x\ =\ \symbol{'133}-,-,-,-,-,-\symbol{'135}\ :\ str}
\item[]\textsl{-\ }length\ x;
\item[]\textsl{val\ it\ =\ 6\ :\ int}
\item[]\textsl{-\ }Str.output("",\ x);
\item[]\textsl{hello<there>}
\item[]\textsl{val\ it\ =\ ()\ :\ unit}
\item[]\textsl{-\ }SymSet.output("",\ Str.alphabet\ x);
\item[]\textsl{e,\ h,\ l,\ o,\ <there>}
\item[]\textsl{val\ it\ =\ ()\ :\ unit}
\item[]\textsl{-\ }Str.output("",\ Str.power(x,\ 3));
\item[]\textsl{hello<there>hello<there>hello<there>}
\item[]\textsl{val\ it\ =\ ()\ :\ unit}
\item[]\textsl{-\ }val\ y\ =\ Str.fromString\ "ello";
\item[]\textsl{val\ y\ =\ \symbol{'133}-,-,-,-\symbol{'135}\ :\ str}
\item[]\textsl{-\ }Str.compare(y,\ x);
\item[]\textsl{val\ it\ =\ LESS\ :\ order}
\item[]\textsl{-\ }Str.equal(y,\ x);
\item[]\textsl{val\ it\ =\ false\ :\ bool}
\item[]\textsl{-\ }Str.prefix(y,\ x);
\item[]\textsl{val\ it\ =\ false\ :\ bool}
\item[]\textsl{-\ }Str.substr(y,\ x);
\item[]\textsl{val\ it\ =\ true\ :\ bool}
\item[]\textsl{-\ }val\ z\ =\ Str.fromString\ "h"\ @\ y;
\item[]\textsl{val\ z\ =\ \symbol{'133}-,-,-,-,-\symbol{'135}\ :\ sym\ list}
\item[]\textsl{-\ }Str.prefix(z,\ x);
\item[]\textsl{val\ it\ =\ true\ :\ bool}
\item[]\textsl{-\ }val\ x\ =\ Str.fromString\ "hellothere";
\item[]\textsl{val\ x\ =\ \symbol{'133}-,-,-,-,-,-,-,-,-,-\symbol{'135}\ :\ str}
\item[]\textsl{-\ }null\ x;
\item[]\textsl{val\ it\ =\ false\ :\ bool}
\item[]\textsl{-\ }Sym.output("",\ hd\ x);
\item[]\textsl{h}
\item[]\textsl{val\ it\ =\ ()\ :\ unit}
\item[]\textsl{-\ }Str.output("",\ tl\ x);
\item[]\textsl{ellothere}
\item[]\textsl{val\ it\ =\ ()\ :\ unit}
\item[]\textsl{-\ }Sym.output("",\ Str.last\ x);
\item[]\textsl{e}
\item[]\textsl{val\ it\ =\ ()\ :\ unit}
\item[]\textsl{-\ }Str.output("",\ Str.allButLast\ x);
\item[]\textsl{hellother}
\item[]\textsl{val\ it\ =\ ()\ :\ unit}
\end{list}


\subsection{Sets of Strings}

The module \texttt{StrSet}
\index{StrSet@\texttt{StrSet}}%
defines various functions for processing
finite sets of strings (elements of type \texttt{str\;set};
\index{set@\texttt{\primesym a~set}}%
finite languages),
\index{language}%
including:
\begin{verbatim}
val input    : string -> str set
val output   : string * str set -> unit
val fromList : str list -> str set
val memb     : str * str set -> bool
val subset   : str set * str set -> bool
val equal    : str set * str set -> bool
val union    : str set * str set -> str set
val inter    : str set * str set -> str set
val minus    : str set * str set -> str set
val genUnion : str set list -> str set
val genInter : str set list -> str set
val alphabet : str set -> sym set
\end{verbatim}
\index{StrSet@\texttt{StrSet}!input@\texttt{input}}%
\index{StrSet@\texttt{StrSet}!output@\texttt{output}}%
\index{StrSet@\texttt{StrSet}!fromList@\texttt{fromList}}%
\index{StrSet@\texttt{StrSet}!memb@\texttt{memb}}%
\index{StrSet@\texttt{StrSet}!subset@\texttt{subset}}%
\index{StrSet@\texttt{StrSet}!equal@\texttt{equal}}%
\index{StrSet@\texttt{StrSet}!union@\texttt{union}}%
\index{StrSet@\texttt{StrSet}!inter@\texttt{inter}}%
\index{StrSet@\texttt{StrSet}!minus@\texttt{minus}}%
\index{StrSet@\texttt{StrSet}!genUnion@\texttt{genUnion}}%
\index{StrSet@\texttt{StrSet}!genInter@\texttt{getInter}}%
\index{StrSet@\texttt{StrSet}!alphabet@\texttt{alphabet}}%
The total ordering associated with sets of strings is our
total ordering on strings.
Sets of strings are expressed in Forlan's syntax as sequences of strings,
separated by commas.

Here are some example uses of the functions of \texttt{StrSet}:
\begin{list}{}
{\setlength{\leftmargin}{\leftmargini}
\setlength{\rightmargin}{0cm}
\setlength{\itemindent}{0cm}
\setlength{\listparindent}{0cm}
\setlength{\itemsep}{0cm}
\setlength{\parsep}{0cm}
\setlength{\labelsep}{0cm}
\setlength{\labelwidth}{0cm}
\catcode`\#=12
\catcode`\$=12
\catcode`\%=12
\catcode`\^=12
\catcode`\_=12
\catcode`\.=12
\catcode`\?=12
\catcode`\!=12
\catcode`\&=12
\ttfamily}
\small
\item[]\textsl{-\ }val\ xs\ =\ StrSet.input\ "";
\item[]\textsl{@\ }hello,\ <id><num>,\ %
\item[]\textsl{@\ }.
\item[]\textsl{val\ xs\ =\ -\ :\ str\ set}
\item[]\textsl{-\ }val\ ys\ =\ StrSet.input\ "";
\item[]\textsl{@\ }<id><num>,\ another\ 
\item[]\textsl{@\ }.
\item[]\textsl{val\ ys\ =\ -\ :\ str\ set}
\item[]\textsl{-\ }val\ zs\ =\ StrSet.union(xs,\ ys);
\item[]\textsl{val\ zs\ =\ -\ :\ str\ set}
\item[]\textsl{-\ }Set.size\ zs;
\item[]\textsl{val\ it\ =\ 4\ :\ int}
\item[]\textsl{-\ }StrSet.output("",\ zs);
\item[]\textsl{%,\ <id><num>,\ hello,\ another}
\item[]\textsl{val\ it\ =\ ()\ :\ unit}
\item[]\textsl{-\ }val\ us\ =\ Set.filter\ (fn\ x\ =>\ length\ x\ mod\ 2\ =\ 0)\ zs;
\item[]\textsl{val\ us\ =\ -\ :\ sym\ list\ set}
\item[]\textsl{-\ }StrSet.output("",\ us);
\item[]\textsl{%,\ <id><num>}
\item[]\textsl{val\ it\ =\ ()\ :\ unit}
\item[]\textsl{-\ }SymSet.output("",\ StrSet.alphabet\ zs);
\item[]\textsl{a,\ e,\ h,\ l,\ n,\ o,\ r,\ t,\ <id>,\ <num>}
\item[]\textsl{val\ it\ =\ ()\ :\ unit}
\item[]\textsl{-\ }val\ x\ =\ Str.fromString\ "0123";
\item[]\textsl{val\ x\ =\ \symbol{'133}-,-,-,-\symbol{'135}\ :\ str}
\item[]\textsl{-\ }StrSet.output("",\ StrSet.prefixes\ x);
\item[]\textsl{%,\ 0,\ 01,\ 012,\ 0123}
\item[]\textsl{val\ it\ =\ ()\ :\ unit}
\item[]\textsl{-\ }StrSet.output("",\ StrSet.suffixes\ x);
\item[]\textsl{%,\ 3,\ 23,\ 123,\ 0123}
\item[]\textsl{val\ it\ =\ ()\ :\ unit}
\item[]\textsl{-\ }StrSet.output("",\ StrSet.substrings\ x);
\item[]\textsl{%,\ 0,\ 1,\ 2,\ 3,\ 01,\ 12,\ 23,\ 012,\ 123,\ 0123}
\item[]\textsl{val\ it\ =\ ()\ :\ unit}
\end{list}

In this transcript, \texttt{us} was declared to be all the even-length
elements of \texttt{zs}.

\subsection{Relations on Symbols}

The module \texttt{SymRel}
\index{SymRel@\texttt{SymRel}}%
defines the type \texttt{sym\underscoresym rel}
\index{sym_rel@\texttt{sym\underscoresym rel}}%
\index{SymRel@\texttt{SymRel}!sym_rel@\texttt{sym\underscoresym rel}}%
of finite relations on symbols.
\index{relation}%
It is bound in the top-level environment, and is
equal to \texttt{(sym * sym)set}, i.e., its elements are finite sets
of pairs of symbols.  The total ordering associated with relations
on symbols orders pairs of symbols first according to their
left-hand sides (using the total ordering on symbols), and then
according to their right-hand sides.
Relations on symbols are expressed in Forlan's syntax as sequences of ordered
pairs \texttt{($a$,$b$)} of symbols, separated by commas.
\index{ordered pair}%
\index{ ordered pair@$(\cdot,\cdot)$}%

\texttt{SymRel} also defines various functions for processing finite relations
on symbols, including:
\begin{verbatim}
val input            : string -> sym_rel
val output           : string * sym_rel -> unit
val fromList         : (sym * sym)list -> sym_rel
val memb             : (sym * sym) * sym_rel -> bool
val subset           : sym_rel * sym_rel -> bool
val equal            : sym_rel * sym_rel -> bool
val union            : sym_rel * sym_rel -> sym_rel
val inter            : sym_rel * sym_rel -> sym_rel
val minus            : sym_rel * sym_rel -> sym_rel
val genUnion         : sym_rel list -> sym_rel
val genInter         : sym_rel list -> sym_rel
val domain           : sym_rel -> sym set
val range            : sym_rel -> sym set
val relationFromTo   : sym_rel * sym_set * sym_set -> bool
val reflexive        : sym_rel * sym set -> bool
val symmetric        : sym_rel -> bool
val antisymmetric    : sym_rel -> bool
val transitive       : sym_rel -> bool
val total            : sym_rel -> bool
val inverse          : sym_rel -> sym_rel
val compose          : sym_rel * sym_rel -> sym_rel
val function         : sym_rel -> bool
val functionFromTo   : sym_rel * sym_set * sym_set -> bool
val injection        : sym_rel -> bool
val bijectionFromTo  : sym_rel * sym_set * sym_set -> bool
val applyFunction    : sym_rel -> sym -> sym
val restrictFunction : sym_rel * sym_set -> sym_rel
val updateFunction   : sym_rel * sym * sym -> sym_rel
\end{verbatim}
\index{SymRel@\texttt{SymRel}!input@\texttt{input}}%
\index{SymRel@\texttt{SymRel}!output@\texttt{output}}%
\index{SymRel@\texttt{SymRel}!fromList@\texttt{fromList}}%
\index{SymRel@\texttt{SymRel}!memb@\texttt{memb}}%
\index{SymRel@\texttt{SymRel}!subset@\texttt{subset}}%
\index{SymRel@\texttt{SymRel}!equal@\texttt{equal}}%
\index{SymRel@\texttt{SymRel}!union@\texttt{union}}%
\index{SymRel@\texttt{SymRel}!inter@\texttt{inter}}%
\index{SymRel@\texttt{SymRel}!minus@\texttt{minus}}%
\index{SymRel@\texttt{SymRel}!genUnion@\texttt{genUnion}}%
\index{SymRel@\texttt{SymRel}!genInter@\texttt{genInter}}%
\index{SymRel@\texttt{SymRel}!domain@\texttt{domain}}%
\index{SymRel@\texttt{SymRel}!range@\texttt{range}}%
\index{SymRel@\texttt{SymRel}!relationFromTo@\texttt{relationFromTo}}%
\index{SymRel@\texttt{SymRel}!reflexive@\texttt{reflexive}}%
\index{SymRel@\texttt{SymRel}!symmetric@\texttt{symmetric}}%
\index{SymRel@\texttt{SymRel}!antisymmetric@\texttt{antisymmetric}}%
\index{SymRel@\texttt{SymRel}!transitive@\texttt{transitive}}%
\index{SymRel@\texttt{SymRel}!total@\texttt{total}}%
\index{SymRel@\texttt{SymRel}!inverse@\texttt{inverse}}%
\index{SymRel@\texttt{SymRel}!compose@\texttt{compose}}%
\index{SymRel@\texttt{SymRel}!function@\texttt{function}}%
\index{SymRel@\texttt{SymRel}!functionFromTo@\texttt{functionFromTo}}%
\index{SymRel@\texttt{SymRel}!injection@\texttt{injection}}%
\index{SymRel@\texttt{SymRel}!bijectionFromTo@\texttt{bijectionFromTo}}%
\index{SymRel@\texttt{SymRel}!applyFunction@\texttt{applyFunction}}%
The functions \texttt{domain} and \texttt{range} return the domain
and range, respectively, of a \texttt{relation}.
$\mathtt{relationFromTo}(\rel, \bs, \cs)$ tests whether $\rel$
is a relation from $\bs$ to $\cs$.

$\mathtt{reflexive}(\rel, \bs)$ tests whether $\rel$ is reflexive
on $\bs$.
The functions \texttt{symmetric}, \texttt{antisymmetric} and
\texttt{transitive} test whether a relation is symmetric, antisymmetric
or transitive, respectively.
$\mathtt{total}(\rel, \bs)$ tests whether $\rel$ is total
on $\bs$.

The function \texttt{inverse} computes the inverse of a relation,
and \texttt{compose} composes two relations.

The function \texttt{function} tests whether a relation is a
function.
The function \texttt{applyFunction} is curried.
\index{curried function}%
\index{Standard ML!curried function}%
\index{Standard ML!function!curried}%
Given a relation $\rel$, \texttt{applyFunction} checks that $\rel$ is a
function, issuing an error message, and raising an exception,
otherwise.  If it is a function, it returns a function of type
\texttt{sym -> sym} that, when called with a symbol $a$, will apply
the function $\rel$ to $a$, issuing an error message if $a$ is
not in the domain of $\rel$.
$\mathtt{functionFromTo}(\rel, \bs, \cs)$ tests whether $\rel$
is a function from $\bs$ to $\cs$.
The function \texttt{injection} tests whether a relation is
an injective function.
$\mathtt{bijectionFromTo}(\rel, \bs, \cs)$ tests whether $\rel$
is a bijection from $\bs$ to $\cs$.

$\mathtt{restrictFunction}(\rel, \bs)$ restricts the function
$\rel$ to $\bs$; it issues an error message if $\rel$ is not
a function, or $\bs$ is not a subset of the domain of $\rel$.
And, $\mathtt{updateFunction}(\rel, a, b)$ returns the
updating of the function $\rel$ to send $a$ to $b$; it issues
an error message if $\rel$ isn't a function.

Here is how we can work with total orderings using functions from
\texttt{SymRel}:
\begin{list}{}
{\setlength{\leftmargin}{\leftmargini}
\setlength{\rightmargin}{0cm}
\setlength{\itemindent}{0cm}
\setlength{\listparindent}{0cm}
\setlength{\itemsep}{0cm}
\setlength{\parsep}{0cm}
\setlength{\labelsep}{0cm}
\setlength{\labelwidth}{0cm}
\catcode`\#=12
\catcode`\$=12
\catcode`\%=12
\catcode`\^=12
\catcode`\_=12
\catcode`\.=12
\catcode`\?=12
\catcode`\!=12
\catcode`\&=12
\ttfamily}
\small
\item[]\textsl{-\ }val\ rel\ =\ SymRel.input\ "";
\item[]\textsl{@\ }(0,\ 1),\ (1,\ 2),\ (0,\ 2),\ (0,\ 0),\ (1,\ 1),\ (2,\ 2)
\item[]\textsl{@\ }.
\item[]\textsl{val\ rel\ =\ -\ :\ sym_rel}
\item[]\textsl{-\ }SymRel.output("",\ rel);
\item[]\textsl{(0,\ 0),\ (0,\ 1),\ (0,\ 2),\ (1,\ 1),\ (1,\ 2),\ (2,\ 2)}
\item[]\textsl{val\ it\ =\ ()\ :\ unit}
\item[]\textsl{-\ }SymSet.output("",\ SymRel.domain\ rel);
\item[]\textsl{0,\ 1,\ 2}
\item[]\textsl{val\ it\ =\ ()\ :\ unit}
\item[]\textsl{-\ }SymSet.output("",\ SymRel.range\ rel);
\item[]\textsl{0,\ 1,\ 2}
\item[]\textsl{val\ it\ =\ ()\ :\ unit}
\item[]\textsl{-\ }SymRel.relationFromTo
\item[]\textsl{=\ }(rel,\ SymSet.fromString\ "0,\ 1,\ 2",\ SymSet.fromString\ "0,\ 1,\ 2");
\item[]\textsl{val\ it\ =\ true\ :\ bool}
\item[]\textsl{-\ }SymRel.relationOn(rel,\ SymSet.fromString\ "0,\ 1,\ 2");
\item[]\textsl{val\ it\ =\ true\ :\ bool}
\item[]\textsl{-\ }SymRel.reflexive(rel,\ SymSet.fromString\ "0,\ 1,\ 2");
\item[]\textsl{val\ it\ =\ true\ :\ bool}
\item[]\textsl{-\ }SymRel.symmetric\ rel;
\item[]\textsl{val\ it\ =\ false\ :\ bool}
\item[]\textsl{-\ }SymRel.antisymmetric\ rel;
\item[]\textsl{val\ it\ =\ true\ :\ bool}
\item[]\textsl{-\ }SymRel.transitive\ rel;
\item[]\textsl{val\ it\ =\ true\ :\ bool}
\item[]\textsl{-\ }SymRel.total(rel,\ SymSet.fromString\ "0,\ 1,\ 2");
\item[]\textsl{val\ it\ =\ true\ :\ bool}
\item[]\textsl{-\ }val\ rel'\ =\ SymRel.inverse\ rel;
\item[]\textsl{val\ rel'\ =\ -\ :\ sym_rel}
\item[]\textsl{-\ }SymRel.output("",\ rel');
\item[]\textsl{(0,\ 0),\ (1,\ 0),\ (1,\ 1),\ (2,\ 0),\ (2,\ 1),\ (2,\ 2)}
\item[]\textsl{val\ it\ =\ ()\ :\ unit}
\item[]\textsl{-\ }val\ rel''\ =\ SymRel.compose(rel',\ rel);
\item[]\textsl{val\ rel''\ =\ -\ :\ sym_rel}
\item[]\textsl{-\ }SymRel.output("",\ rel'');
\item[]\textsl{(0,\ 0),\ (0,\ 1),\ (0,\ 2),\ (1,\ 0),\ (1,\ 1),\ (1,\ 2),\ (2,\ 0),\ (2,\ 1),}
\item[]\textsl{(2,\ 2)}
\item[]\textsl{val\ it\ =\ ()\ :\ unit}
\end{list}


And here is how we can work with relations that are functions:
\begin{list}{}
{\setlength{\leftmargin}{\leftmargini}
\setlength{\rightmargin}{0cm}
\setlength{\itemindent}{0cm}
\setlength{\listparindent}{0cm}
\setlength{\itemsep}{0cm}
\setlength{\parsep}{0cm}
\setlength{\labelsep}{0cm}
\setlength{\labelwidth}{0cm}
\catcode`\#=12
\catcode`\$=12
\catcode`\%=12
\catcode`\^=12
\catcode`\_=12
\catcode`\.=12
\catcode`\?=12
\catcode`\!=12
\catcode`\&=12
\ttfamily}
\small
\item[]\textsl{-\ }val\ rel\ =\ SymRel.input\ "";
\item[]\textsl{@\ }(1,\ 2),\ (2,\ 3),\ (3,\ 4)
\item[]\textsl{@\ }.
\item[]\textsl{val\ rel\ =\ -\ :\ sym_rel}
\item[]\textsl{-\ }SymRel.output("",\ rel);
\item[]\textsl{(1,\ 2),\ (2,\ 3),\ (3,\ 4)}
\item[]\textsl{val\ it\ =\ ()\ :\ unit}
\item[]\textsl{-\ }SymSet.output("",\ SymRel.domain\ rel);
\item[]\textsl{1,\ 2,\ 3}
\item[]\textsl{val\ it\ =\ ()\ :\ unit}
\item[]\textsl{-\ }SymSet.output("",\ SymRel.range\ rel);
\item[]\textsl{2,\ 3,\ 4}
\item[]\textsl{val\ it\ =\ ()\ :\ unit}
\item[]\textsl{-\ }SymRel.function\ rel;
\item[]\textsl{val\ it\ =\ true\ :\ bool}
\item[]\textsl{-\ }SymRel.functionFromTo
\item[]\textsl{=\ }(rel,\ SymSet.fromString\ "1,\ 2,\ 3",\ SymSet.fromString\ "2,\ 3,\ 4");
\item[]\textsl{val\ it\ =\ true\ :\ bool}
\item[]\textsl{-\ }SymRel.injection\ rel;
\item[]\textsl{val\ it\ =\ true\ :\ bool}
\item[]\textsl{-\ }SymRel.bijectionFromTo
\item[]\textsl{=\ }(rel,\ SymSet.fromString\ "1,\ 2,\ 3",\ SymSet.fromString\ "2,\ 3,\ 4");
\item[]\textsl{val\ it\ =\ true\ :\ bool}
\item[]\textsl{-\ }val\ f\ =\ SymRel.applyFunction\ rel;
\item[]\textsl{val\ f\ =\ fn\ :\ sym\ ->\ sym}
\item[]\textsl{-\ }Sym.output("",\ f(Sym.fromString\ "1"));
\item[]\textsl{2}
\item[]\textsl{val\ it\ =\ ()\ :\ unit}
\item[]\textsl{-\ }Sym.output("",\ f(Sym.fromString\ "2"));
\item[]\textsl{3}
\item[]\textsl{val\ it\ =\ ()\ :\ unit}
\item[]\textsl{-\ }Sym.output("",\ f(Sym.fromString\ "3"));
\item[]\textsl{4}
\item[]\textsl{val\ it\ =\ ()\ :\ unit}
\item[]\textsl{-\ }Sym.output("",\ f(Sym.fromString\ "4"));
\item[]\textsl{argument\ not\ in\ domain}
\item[]
\item[]\textsl{uncaught\ exception\ Error}
\item[]\textsl{-\ }val\ rel'\ =\ SymRel.input\ "";
\item[]\textsl{@\ }(4,\ 3),\ (3,\ 2),\ (2,\ 1)
\item[]\textsl{@\ }.
\item[]\textsl{val\ rel'\ =\ -\ :\ sym_rel}
\item[]\textsl{-\ }val\ rel''\ =\ SymRel.compose(rel',\ rel);
\item[]\textsl{val\ rel''\ =\ -\ :\ sym_rel}
\item[]\textsl{-\ }SymRel.functionFromTo
\item[]\textsl{=\ }(rel'',\ SymSet.fromString\ "1,\ 2,\ 3",
\item[]\textsl{=\ }\ SymSet.fromString\ "1,\ 2,\ 3");
\item[]\textsl{val\ it\ =\ true\ :\ bool}
\item[]\textsl{-\ }SymRel.output("",\ rel'');
\item[]\textsl{(1,\ 1),\ (2,\ 2),\ (3,\ 3)}
\item[]\textsl{val\ it\ =\ ()\ :\ unit}
\end{list}


\subsection{Notes}

The book and toolset were designed and developed together, which made
it possible to minimize the notational and conceptual distance between
the two.
\index{Forlan|)}%

%%% Local Variables: 
%%% mode: latex
%%% TeX-master: "book"
%%% End: 
