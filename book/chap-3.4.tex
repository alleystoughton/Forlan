\section{Finite Automata and Labeled Paths}
\label{FiniteAutomataAndLabeledPaths}

\index{finite automaton|(}
\index{FA|(}

In this section, we: say what finite automata (FA) are, and show how
they can be processed using Forlan; say what labeled paths are, and
show how they can be processed using Forlan; and use the notion of
labeled path to say what finite automata mean.

\subsection{Finite Automata}

A \emph{finite automaton} (FA) $M$ consists of:
\begin{itemize}
\item a finite set $Q_M$ of symbols (we call the elements of $Q_M$
the \emph{states} of $M$);

\item an element $s_M$ of $Q_M$ (we call $s_M$ the \emph{start state}
of $M$);

\item a subset $A_M$ of $Q_M$ (we call the elements of $A_M$ the
\emph{accepting states} of $M$);

\item a finite subset $T_M$ of $\setof{(q,x,r)}{q,r\in Q_M\eqtxt{and}
x\in\Str}$ (we call the elements of $T_M$ the \emph{transitions} of
$M$, and we often write $(q, x, r)$ as
\begin{gather*}
q\tranarr{x}r
\end{gather*}
or $q,x\fun r$).
\end{itemize}
\index{finite automaton!states}%
\index{finite automaton!start state}%
\index{finite automaton!accepting states}%
\index{finite automaton!transitions}%

We order transitions first by their left-hand sides, then by their
middles, and then by their right-hand sides, using our total orderings
on symbols and strings.  This gives us a total ordering on
transitions.

We often abbreviate $Q_M$, $s_M$, $A_M$ and $T_M$ to $Q$, $s$, $A$ and
$T$, when it's clear which FA we are working with.  Whenever possible,
we will use the mathematical variables $p$, $q$ and $r$ to name
states.  We write $\FA$ for the set of all finite automata, which is a
\index{finite automaton!FA@$\FA$}%
\index{FA@$\FA$}%
countably infinite set.

As an example, we can define an FA $M$ as follows:
\begin{itemize}
\item $Q_M=\{\mathsf{A,B,C}\}$;

\item $s_M=\Asf$;

\item $A_M=\{\Asf, \Csf\}$; and

\item $T_M=\{\mathsf{(A,1,A), (B,11,B), (C,111,C),
  (A,0,B),(A,2,B),(A,0,C),(A,2,C),}\abr\mathsf{(B,0,C), (B,2,C)}\}$.
\end{itemize}

Finite automata are \emph{nondeterministic} machines that take strings as
inputs.  When a machine is run on a given input, it begins in
its start state.

If, after some number of steps, the machine is in state $p$, the
machine's remaining input begins with $x$, and one of the machine's
transitions is $p,x\fun q$, then the machine \emph{may} read $x$ from
its input and switch to state $q$.  If $p, y\fun r$ is also a
transition, and the remaining input begins with $y$, then consuming
$y$ and switching to state $r$ will also be possible, etc.  The case
when $x=\%$, i.e., when we have a $\%$-\emph{transition}, is
interesting: a state switch can happen without reading anything.

If \emph{at least one} execution sequence consumes all of the machine's input
and takes it to one of its accepting states, then we say that the
input is \emph{accepted} by the machine; otherwise, we say that the
input is \emph{rejected}.  The meaning of a machine is the language
consisting of all strings that it accepts.

\index{finite automaton!Forlan syntax}%
Here is how our example FA $M$ can be expressed in Forlan's syntax:
\verbatiminput{3.4-fa}
Since whitespace characters are ignored by Forlan's input routines,
the preceding description of $M$ could have been formatted in many other
ways.  States are separated by commas, and transitions are separated
by semicolons.  The order of states and transitions is irrelevant.

Transitions that only differ in their right-hand states can be
merged into single transition families.  E.g., we can merge
\begin{verbatim}
A, 0 -> B
\end{verbatim}
and
\begin{verbatim}
A, 0 -> C
\end{verbatim}
into the transition family
\begin{verbatim}
A, 0 -> B | C
\end{verbatim}

The Forlan module \texttt{FA} defines an abstract type \texttt{fa} (in
\index{FA@\texttt{FA}}%
\index{FA@\texttt{FA}!fa@\texttt{fa}}%
the top-level environment) of finite automata, as well as a large
number of functions and constants for processing FAs, including:
\begin{verbatim}
val input  : string -> fa
val output : string * fa -> unit 
\end{verbatim}
\index{FA@\texttt{FA}!input@\texttt{input}}%
\index{FA@\texttt{FA}!output@\texttt{output}}%
Remember that it's possible to read input from a file, and to write
output to a file.  During printing, Forlan merges transitions into
transition families whenever possible.

Suppose that our example FA is in the file \texttt{3.4-fa}.
We can input this FA into Forlan, and then output it to the
standard output, as follows:
\begin{list}{}
{\setlength{\leftmargin}{\leftmargini}
\setlength{\rightmargin}{0cm}
\setlength{\itemindent}{0cm}
\setlength{\listparindent}{0cm}
\setlength{\itemsep}{0cm}
\setlength{\parsep}{0cm}
\setlength{\labelsep}{0cm}
\setlength{\labelwidth}{0cm}
\catcode`\#=12
\catcode`\$=12
\catcode`\%=12
\catcode`\^=12
\catcode`\_=12
\catcode`\.=12
\catcode`\?=12
\catcode`\!=12
\catcode`\&=12
\ttfamily}
\small
\item[]\textsl{-\ }val\ fa\ =\ FA.input\ "3.4-fa";
\item[]\textsl{val\ fa\ =\ -\ :\ fa}
\item[]\textsl{-\ }FA.output("",\ fa);
\item[]\textsl{\symbol{'173}states\symbol{'175}\ A,\ B,\ C\ \symbol{'173}start\ state\symbol{'175}\ A\ \symbol{'173}accepting\ states\symbol{'175}\ A,\ C}
\item[]\textsl{\symbol{'173}transitions\symbol{'175}}
\item[]\textsl{A,\ 0\ ->\ B\ |\ C;\ A,\ 1\ ->\ A;\ A,\ 2\ ->\ B\ |\ C;\ B,\ 0\ ->\ C;\ B,\ 2\ ->\ C;}
\item[]\textsl{B,\ 11\ ->\ B;\ C,\ 111\ ->\ C}
\item[]\textsl{val\ it\ =\ ()\ :\ unit}
\end{list}


We also make use of graphical notation for finite automata.
Each of the states of a machine is circled,
and its accepting states are double-circled.  The machine's start state is
pointed to by an arrow coming from ``Start'', and
each transition $p, x\fun q$ is drawn as an arrow from state $p$
to state $q$ that is labeled by the string $x$.  Multiple labeled
arrows from one state to another can be abbreviated to a single
arrow, whose label consists of the comma-separated list of the
labels of the original arrows.

Here is how our FA $M$ can be described graphically:
\begin{center}
\input{chap-3.4-fig1.eepic}
\end{center}

The Java program JForlan, can be used to view and edit finite
automata.  It can be invoked directly, or run via Forlan.  See the
Forlan website for more information.

\index{finite automaton!alphabet}%
\index{finite automaton!alphabet@$\alphabet$}%
We define a function $\alphabet\in\FA\fun\Alp$ by: for all $M\in\FA$,
$\alphabet\,M$ is $\setof{a\in\Sym}{\eqtxt{there are}q,x,r\eqtxt{such
    that} q,x\fun r\in T_M\eqtxt{and}a\in\alphabet\,x}$.  I.e.,
$\alphabet\,M$ is all of the symbols appearing in the strings of $M$'s
transitions.  We say that $\alphabet\,M$ is \emph{the alphabet of}
$M$.  For example, the alphabet of our example FA $M$ is
$\{\mathsf{0,1,2}\}$.

\index{finite automaton!sub-FA}%
We say that an FA $M$ is a \emph{sub-FA} of an FA $N$ iff:
\begin{itemize}
\item $Q_M\sub Q_N$;
\item $s_M = s_N$;
\item $A_M\sub A_N$; and
\item $T_M\sub T_N$.
\end{itemize}
Thus $M=N$ iff $M$ is a sub-FA of $N$ and $N$ is a sub-FA of $M$.

The Forlan module \texttt{FA} contains the functions
\begin{verbatim}
val equal          : fa * fa -> bool
val numStates      : fa -> int
val numTransitions : fa -> int
val alphabet       : fa -> sym set
val sub            : fa * fa -> bool
\end{verbatim}
\index{FA@\texttt{FA}!equal@\texttt{equal}}%
\index{FA@\texttt{FA}!numStates@\texttt{numStates}}%
\index{FA@\texttt{FA}!numTransitions@\texttt{numTransitions}}%
\index{FA@\texttt{FA}!alphabet@\texttt{alphabet}}%
\index{FA@\texttt{FA}!sub@\texttt{sub}}%
The function \texttt{equal} tests whether two FAs are equal, i.e.,
whether they have the same states, start states, accepting states
and transitions.
The functions \texttt{numStates} and \texttt{numTransitions} return
the numbers of states and transitions, respectively, of an FA.
The function \texttt{alphabet} returns the alphabet of an FA.  
And the function \texttt{sub} tests whether a first FA is a sub-FA of
a second FA.

For example, we can continue out Forlan session as follows:
\begin{list}{}
{\setlength{\leftmargin}{\leftmargini}
\setlength{\rightmargin}{0cm}
\setlength{\itemindent}{0cm}
\setlength{\listparindent}{0cm}
\setlength{\itemsep}{0cm}
\setlength{\parsep}{0cm}
\setlength{\labelsep}{0cm}
\setlength{\labelwidth}{0cm}
\catcode`\#=12
\catcode`\$=12
\catcode`\%=12
\catcode`\^=12
\catcode`\_=12
\catcode`\.=12
\catcode`\?=12
\catcode`\!=12
\catcode`\&=12
\ttfamily}
\small
\item[]\textsl{-\ }val\ fa'\ =\ FA.input\ "";
\item[]\textsl{@\ }\symbol{'173}states\symbol{'175}\ A,\ B,\ C
\item[]\textsl{@\ }\symbol{'173}start\ state\symbol{'175}\ A
\item[]\textsl{@\ }\symbol{'173}accepting\ states\symbol{'175}\ C
\item[]\textsl{@\ }\symbol{'173}transitions\symbol{'175}
\item[]\textsl{@\ }A,\ 0\ ->\ B;\ A,\ 2\ ->\ B;\ A,\ 0\ ->\ C;\ A,\ 2\ ->\ C;
\item[]\textsl{@\ }B,\ 0\ ->\ C;\ B,\ 2\ ->\ C
\item[]\textsl{@\ }.
\item[]\textsl{val\ fa'\ =\ -\ :\ fa}
\item[]\textsl{-\ }FA.equal(fa',\ fa);
\item[]\textsl{val\ it\ =\ false\ :\ bool}
\item[]\textsl{-\ }FA.sub(fa',\ fa);
\item[]\textsl{val\ it\ =\ true\ :\ bool}
\item[]\textsl{-\ }FA.sub(fa,\ fa');
\item[]\textsl{val\ it\ =\ false\ :\ bool}
\item[]\textsl{-\ }FA.numStates\ fa;
\item[]\textsl{val\ it\ =\ 3\ :\ int}
\item[]\textsl{-\ }FA.numTransitions\ fa;
\item[]\textsl{val\ it\ =\ 9\ :\ int}
\item[]\textsl{-\ }SymSet.output("",\ FA.alphabet\ fa);
\item[]\textsl{0,\ 1,\ 2}
\item[]\textsl{val\ it\ =\ ()\ :\ unit}
\end{list}


\subsection{Labeled Paths and FA Meaning}

We will formally explain when strings are accepted by finite automata
\index{finite automaton!labeled path|see{labeled path}}%
\index{labeled path}%
\index{labeled path!LP@$\LP$}%
using the notion of a labeled path.  A \emph{labeled path} consists of
a pair $(\xs,q)$, where $\xs\in\List(\Sym\times\Str)$ and $q\in\Sym$,
and the set $\LP$ of labeled paths is
$\List(\Sym\times\Str)\times\Sym$.  Clearly, $\LP$ is countably
infinite.  We typically write
$([(q_1,x_1),(q_2,x_2)\ldots,(q_n,x_n)],q_n+1)\in\LP$ as:
\begin{gather*}
q_1\lparr{x_1}q_2\lparr{x_2}\cdots\,q_n\lparr{x_n}q_{n+1}
\end{gather*}
or
\begin{gather*}
q_1,x_1\Rightarrow q_2,x_2\Rightarrow\cdots\,q_n,x_n\Rightarrow q_{n+1} .
\end{gather*}
This path describes a way of getting from state $q_1$ to state $q_{n+1}$
in some unspecified machine, by reading the strings
$x_1,\,\ldots,x_n$ from the machine's input.  We start out in
state $q_1$, make use of the transition $q_1,x_1\fun q_2$ to read
$x_1$ from the input and switch to state $q_2$, etc.

Let $\lp=(\xs,q)\in\LP$.
We say that:
\begin{itemize}
\item the \emph{start state} of $\lp$ ($\startState\,\lp$) is
  the left-hand side of the first element of $\xs$, if $\xs$ is nonempty,
  and is $q$, if $\xs$ is empty;

\item the \emph{end state} of $\lp$ ($\myendState\,\lp$) is $q$;

\item the \emph{length} of $\lp$ ($|\lp|$) is $|\xs|$; and

\item the \emph{label} of $\lp$ ($\mylabel\,\lp$) is the result of
  concatenating the right-hand sides of $\xs$ ($\%$, if $\xs$ is
  empty).
\end{itemize}
\index{labeled path!startState@$\startState$}%
\index{labeled path!endState@$\myendState$}%
\index{labeled path!length@$\sizedot$}%
\index{labeled path!label@$\mylabel$}%

This defines functions $\startState\in\LP\fun\Sym$,
$\myendState\in\LP\fun\Sym$ and $\mylabel\in\LP\fun\Str$.
For example $\Asf = ([\,],\Asf)$
is a labeled path whose start and end states are both $\Asf$, whose
length is $0$, and whose label is $\%$.  And
\begin{gather*}
\Asf\lparr{\mathsf{0}}\Bsf\lparr{\mathsf{11}}\Bsf\lparr{\twosf}\Csf
\end{gather*}
is a labeled path whose start state is $\Asf$, end state is $\Csf$,
length is $3$, and label is $\mathsf{(0)(11)(2)}=\mathsf{0112}$.

We can join compatible paths together.  Let $\JOIN=
\setof{(\lp_1,\lp_2)\in\LP\times\LP}{\myendState\,\lp_1 = \startState\,\lp_2}$,
and define $\join\in\JOIN\fun\LP$ by: for all $\xs,\ys\in\List(\Sym\times\Str)$
\index{labeled path!join@$\join$}%
and $q,r\in\Sym$, if $((\xs,q),(\ys,r))\in\JOIN$, then
$\join((\xs,q), (\ys,r)) = (\xs\myconcat\ys, r)$.  E.g., if
$\lp_1$ and $\lp_2$ are defined by
\begin{gather*}
\lp_1 = 
\Asf\lparr{\mathsf{0}}\Bsf\lparr{\mathsf{11}}\Bsf , \quad\eqtxt{and}\quad
\lp_2 =
\Bsf\lparr{\mathsf{11}}\Bsf\lparr{\mathsf{2}}\Csf ,
\end{gather*}
then $\join(\lp_1,\lp_2)$ is
\begin{gather*}
\Asf\lparr{\mathsf{0}}\Bsf\lparr{\mathsf{11}}\Bsf
\lparr{\mathsf{11}}\Bsf\lparr{\mathsf{2}}\Csf .
\end{gather*}

\index{finite automaton!labeled path!valid}%
\index{labeled path!valid}%
A labeled path $(\xs,q)\in\LP$ is \emph{valid for} an FA $M$ iff
\begin{itemize}
\item for all $i\in[1:|\xs|-1]$, $\hash{1}(\xs\,i),\hash{2}(\xs\,i)\fun
  \hash{1}(\xs(i+1))$;

\item if $\xs$ is nonempty, then $\hash{1}(\xs\,|\xs|),\hash{2}(\xs\,|\xs|)\fun
  q$; and

\item $q\in Q_M$.
\end{itemize}
(The last of these conditions is redundant whenever $\xs$ is nonempty.)

Recall our example FA $M$:
\begin{center}
\input{chap-3.4-fig1.eepic}
\end{center}
The labeled path $\Asf=([\,],\Asf)$ is valid for $M$,
since $A\in Q_M$.  And
\begin{gather*}
\Asf\lparr{\mathsf{0}}\Bsf\lparr{\mathsf{11}}\Bsf\lparr{\twosf}\Csf
\end{gather*}
is valid for $M$, since $\Asf\tranarr{\zerosf}\Bsf$,
$\Bsf\tranarr{\mathsf{11}}\Bsf$ and $\Bsf\tranarr{\twosf}\Csf$ are in
$T_M$ (and $\Csf\in Q_M$). But the labeled path
\begin{gather*}
\Asf\lparr{\%}\Asf
\end{gather*}
is not valid for $M$, since $\Asf,\%\fun\Asf\not\in T_M$.

\index{finite automaton!accepted by}%
A string $w$ is \emph{accepted by} a finite automaton $M$ iff
there is a labeled path $\lp$ such that
\begin{itemize}
\item $\lp$ is valid for $M$;

\item the label of $\lp$ is $w$;

\item the start state of $\lp$ is the start state of $M$; and

\item the end state of $\lp$ is an accepting state of $M$.
\end{itemize}
For example, $\mathsf{0112}$ is accepted by
$M$ because of the labeled path
\begin{gather*}
\Asf\lparr{\mathsf{0}}\Bsf\lparr{\mathsf{11}}\Bsf\lparr{\twosf}\Csf ,
\end{gather*}
since this labeled path is valid for $M$, is labeled by $\mathsf{0112} =
\mathsf{(0)(11)(2)}$, has a start state ($\Asf$) that is $M$'s start state,
and has an end state ($\Csf$) that is one of $M$'s accepting states.

\index{finite automaton!meaning}%
\index{finite automaton!language accepted by}%
\index{L(@$L(\cdot)$}%
\index{finite automaton!L(@$L(\cdot)$}%}%
Clearly, if $w$ is accepted by $M$, then
$\alphabet\,w\sub\alphabet\,M$.  Thus $\setof{w\in\Str}{w\eqtxt{is
    accepted by}M}\sub(\alphabet\,M)^*$, so we may define the
\emph{language accepted by} a finite automaton $M$ ($L(M)$) to be
\begin{gather*}
\setof{w\in\Str}{w\eqtxt{is accepted by}M}.
\end{gather*}
Furthermore:

\begin{proposition}
Suppose $M$ is a finite automaton.  Then $\alphabet(L(M))\sub\alphabet\,M$.
\end{proposition}

In other words, every symbol of every string that is accepted by $M$
comes from the alphabet of $M$, i.e., appears in the label of one of
$M$'s transitions.

Going back to our example, we have that
\begin{align*}
L(M) &= \{\onesf\}^* \cup {}\\
     &\quad\;\,\mathsf{\{1\}^*\{0,2\}\{11\}^*\{0,2\}\{111\}^*} \cup {} \\
     &\quad\;\,\mathsf{\{1\}^*\{0,2\}\{111\}^*} .
\end{align*}
For example, $\%$, $\mathsf{11}$, $\mathsf{110112111}$ and
$\mathsf{2111111}$ are accepted by $M$.  But $\mathsf{21112}$ and
$\mathsf{2211}$ are not accepted by $M$.

Suppose that $M$ is a sub-FA of $N$.  Then any labeled path that is
valid for $M$ will also be valid for $N$.  Furthermore, we have that:

\begin{proposition}
If $M$ is a sub-FA of $N$, then $L(M)\sub L(N)$.
\end{proposition}

\index{regular expression!equivalence|(}%
\index{ equiv@$\approx$}%
\index{regular expression! equiv@$\approx$}%
We say that finite automata $M$ and $N$ are \emph{equivalent} iff
$L(M) = L(N)$.  In other words, $M$ and $N$ are equivalent iff $M$ and
$N$ accept the same language.  We define a relation $\approx$ on $\FA$
by: $M\approx N$ iff $M$ and $N$ are equivalent.  It is easy to see
that $\approx$ is reflexive on $\FA$, symmetric and transitive.

The Forlan module \texttt{LP} defines an abstract type \texttt{lp} (in the
top-level environment) of labeled paths, as well as various functions
for processing labeled paths, including:
\begin{verbatim}
val input      : string -> lp
val output     : string * lp -> unit
val equal      : lp * lp -> bool
val startState : lp -> sym
val endState   : lp -> sym
val label      : lp -> str
val length     : lp -> int
val join       : lp * lp -> lp
\end{verbatim}
\index{LP@\texttt{LP}}%
\index{LP@\texttt{LP}!input@\texttt{input}}%
\index{LP@\texttt{LP}!output@\texttt{output}}%
\index{LP@\texttt{LP}!startState@\texttt{startState}}%
\index{LP@\texttt{LP}!endState@\texttt{endState}}%
\index{LP@\texttt{LP}!label@\texttt{label}}%
\index{LP@\texttt{LP}!length@\texttt{length}}%
\index{LP@\texttt{LP}!join@\texttt{join}}%
The function \texttt{equal} tests whether two labeled paths are
equal.  The functions \texttt{startState}, \texttt{endState},
\texttt{label} and \texttt{length} return the start state, end
state, label and length, respectively, of a labeled path.  And
the function \texttt{join} joins two compatible paths, and issues
an error message when given paths that are incompatible.

The module \texttt{FA} also defines the functions
\begin{verbatim}
val checkLP : fa -> lp -> unit
val validLP : fa -> lp -> bool
\end{verbatim}
\index{FA@\texttt{FA}!checkLP@\texttt{checkLP}}%
\index{FA@\texttt{FA}!validLP@\texttt{validLP}}%
for checking whether a labeled path is valid in a finite automaton.
These are curried functions---functions that return functions
as their results.
The function \texttt{checkLP} takes in an FA $M$ and returns a function
that checks whether a labeled path $\lp$ is valid for $M$.  When
$\lp$ is not valid for $M$, the function explains why it isn't;
otherwise, it prints nothing. 
And, the function \texttt{validLP} takes in an FA $M$ and returns a function
that tests whether a labeled path $\lp$ is valid for $M$, silently
returning \texttt{true}, if it is, and silently returning \texttt{false},
otherwise.

Here are some examples of labeled path and FA processing
(\texttt{fa} is still our example FA):
\begin{list}{}
{\setlength{\leftmargin}{\leftmargini}
\setlength{\rightmargin}{0cm}
\setlength{\itemindent}{0cm}
\setlength{\listparindent}{0cm}
\setlength{\itemsep}{0cm}
\setlength{\parsep}{0cm}
\setlength{\labelsep}{0cm}
\setlength{\labelwidth}{0cm}
\catcode`\#=12
\catcode`\$=12
\catcode`\%=12
\catcode`\^=12
\catcode`\_=12
\catcode`\.=12
\catcode`\?=12
\catcode`\!=12
\catcode`\&=12
\ttfamily}
\small
\item[]\textsl{-\ }val\ lp\ =\ LP.input\ "";
\item[]\textsl{@\ }A,\ 1\ =>\ A,\ 0\ =>\ B,\ 11\ =>\ B,\ 2\ =>\ C,\ 111\ =>\ C
\item[]\textsl{@\ }.
\item[]\textsl{val\ lp\ =\ -\ :\ lp}
\item[]\textsl{-\ }Sym.output("",\ LP.startState\ lp);
\item[]\textsl{A}
\item[]\textsl{val\ it\ =\ ()\ :\ unit}
\item[]\textsl{-\ }Sym.output("",\ LP.endState\ lp);
\item[]\textsl{C}
\item[]\textsl{val\ it\ =\ ()\ :\ unit}
\item[]\textsl{-\ }LP.length\ lp;
\item[]\textsl{val\ it\ =\ 5\ :\ int}
\item[]\textsl{-\ }Str.output("",\ LP.label\ lp);
\item[]\textsl{10112111}
\item[]\textsl{val\ it\ =\ ()\ :\ unit}
\item[]\textsl{-\ }val\ checkLP\ =\ FA.checkLP\ fa;
\item[]\textsl{val\ checkLP\ =\ fn\ :\ lp\ ->\ unit}
\item[]\textsl{-\ }checkLP\ lp;
\item[]\textsl{val\ it\ =\ ()\ :\ unit}
\item[]\textsl{-\ }val\ lp'\ =\ LP.fromString\ "A";
\item[]\textsl{val\ lp'\ =\ -\ :\ lp}
\item[]\textsl{-\ }LP.length\ lp';
\item[]\textsl{val\ it\ =\ 0\ :\ int}
\item[]\textsl{-\ }Str.output("",\ LP.label\ lp');
\item[]\textsl{%}
\item[]\textsl{val\ it\ =\ ()\ :\ unit}
\item[]\textsl{-\ }checkLP\ lp';
\item[]\textsl{val\ it\ =\ ()\ :\ unit}
\item[]\textsl{-\ }val\ lp''\ =\ LP.input\ "";
\item[]\textsl{@\ }A,\ %\ =>\ A,\ 1\ =>\ B
\item[]\textsl{@\ }.
\item[]\textsl{val\ lp''\ =\ -\ :\ lp}
\item[]\textsl{-\ }checkLP\ lp'';
\item[]\textsl{invalid\ transition:\ "A,\ %\ ->\ A"}
\item[]
\item[]\textsl{uncaught\ exception\ Error}
\item[]\textsl{-\ }val\ lp'''\ =\ LP.input\ "";
\item[]\textsl{@\ }B,\ 2\ =>\ C,\ 34\ =>\ D
\item[]\textsl{@\ }.
\item[]\textsl{val\ lp'''\ =\ -\ :\ lp}
\item[]\textsl{-\ }LP.output("",\ LP.join(lp'',\ lp'''));
\item[]\textsl{A,\ %\ =>\ A,\ 1\ =>\ B,\ 2\ =>\ C,\ 34\ =>\ D}
\item[]\textsl{val\ it\ =\ ()\ :\ unit}
\item[]\textsl{-\ }LP.output("",\ LP.join(lp''',\ lp''));
\item[]\textsl{incompatible\ labeled\ paths}
\item[]
\item[]\textsl{uncaught\ exception\ Error}
\end{list}


\subsection{Design of Finite Automata}

\index{finite automaton!design|(}%
In this subsection, we give two examples of finite automata
design.  First, let's find a finite automaton that accepts the set
of all strings of $\zerosf$'s and $\onesf$'s with an even number of
$\zerosf$'s.  Thus, we should be looking for an FA whose alphabet is
$\{\mathsf{0,1}\}^*$.  It seems reasonable that our machine have two
states: an accepting state $\Asf$ corresponding to the strings of
$\zerosf$'s and $\onesf$'s with an even number of zeros, and a state
$\Bsf$ corresponding to the strings of $\zerosf$'s and $\onesf$'s with
an odd number of zeros.  Processing a $\onesf$ in either state should
cause us to stay in that state, but processing a $\zerosf$ in one of
the states should cause us to switch to the other state.  The above
considerations lead us to the FA
\begin{center}
\input{chap-3.4-fig2.eepic}
\end{center}

For the second example, let's find an FA that accepts the language
$X=\setof{w\in\mathsf{\{0,1,2\}^*}}{\eqtxtr{for all substrings}x
  \eqtxt{of}w,\eqtxt{if}|x|=2,\eqtxt{then}x\in\{\mathsf{01,12,20}\}}$.
We have that $\mathsf{0}$, $\mathsf{01}$, $\mathsf{012}$,
$\mathsf{0120}$, etc., are in $X$, and so are $\mathsf{1}$,
$\mathsf{12}$, $\mathsf{120}$, $\mathsf{1201}$, etc., and
$\mathsf{2}$, $\mathsf{20}$, $\mathsf{201}$, $\mathsf{2012}$, etc.  On
the other hand, no string containing $\mathsf{00}$, $\mathsf{02}$,
$\mathsf{11}$, $\mathsf{10}$, $\mathsf{22}$ or $\mathsf{20}$ is in
$X$.  The above observations suggest that part of our machine should
look like:
\begin{center}
\input{chap-3.4-fig4.eepic}
\end{center}
But how should the machine get started?  The simplest approach is to
make use of $\%$-transitions from the start state, giving us
the FA
\begin{center}
\input{chap-3.4-fig3.eepic}
\end{center}

\begin{exercise}
Let $X=\setof{w\in\{\mathsf{0,1}\}^*}{\mathsf{010}\eqtxt{is not a
    substring of}w}$.  Find a finite automaton $M$ such that $L(M)=X$.
\end{exercise}

\begin{exercise}
Let
\begin{align*}
A &=\{\mathsf{001, 011, 101, 111}\} , \eqtxt{and} \\
B &=\{\,w\in\{\mathsf{0,1}\}^* \mid \eqtxtr{for all}x,y\in\{\mathsf{0,1}\}^*,
\eqtxt{if}w=x\zerosf y, \eqtxtr{then there is a} z\in A \\
&\quad\;\;\; \eqtxt{such that} z\eqtxt{is a prefix of} y\,\}.
\end{align*}
Find a finite automaton $M$ such that $L(M)=B$.  Hint: see the second
example of Subsection~\ref{ProvingTheCorrectnessOfRegularExpressions}.
\end{exercise}

\index{finite automaton!design|)}%

\subsection{Notes}

Finite automata are normally defined via transition functions,
$\delta$, which is simple to do for deterministic finite automata,
but increasingly complicated as one adds degrees of nondeterminism.
Furthermore, this approach means that a deterministic finite automaton
(DFA) is not a nondeterministic finite automaton (NFA), and that an
NFA is not an $\epsilon$-NFA, because the transition function of a DFA
is not one for an NFA, and the transition function for an NFA is not
one of an $\epsilon$-NFA.  And formalizing our FAs, whose transition
labels can be strings of length greater than one, is very messy if
done via transition functions, which probably accounts for why such
machines are not normally considered.  Furthermore, in the standard
approach, to say when strings are accepted by finite automata, one must
first extend the different kinds of transition functions to work on
strings.

In contrast, our approach is very simple.  Instead of transition
functions, we work with finite sets of transitions, enabling us to
define the deterministic finite automata, nondeterministic finite
automata, and nondeterministic finite automata with $\%$-moves (which
we call empty-string finite automata) as restrictions on finite
automata.  Furthermore, using labeled paths to say when---and
how---strings are accepted by finite automata is simple, natural and
diagrammatic.  It's the analogue of using parse trees to say
when---and how---strings are generated by grammars.

\index{finite automaton|)}
\index{FA|)}

%%% Local Variables: 
%%% mode: latex
%%% TeX-master: "book"
%%% End: 
