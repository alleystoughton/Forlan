\section{The Pumping Lemma for Regular Languages}
\label{ThePumpingLemmaForRegularLanguages}

\index{pumping lemma!regular languages|(}%
\index{regular language!pumping lemma|(}%
\index{regular language!showing that languages are non-regular|(}%
In this section we consider techniques for showing that
particular languages are not regular.
Consider the language
\begin{gather*}
L=\setof{\zerosf^n\onesf^n}{n\in\nats} =
\mathsf{\{\%, 01, 0011, 000111,\,\ldots\}} .
\end{gather*}
Intuitively, an automaton would have to have infinitely many states to
accept $L$.  A finite automaton won't be able to keep track of how
many $\zerosf$'s it has seen so far, and thus won't be able to insist
that the correct number of $\onesf$'s follow.  We could turn the
preceding ideas into a direct proof that $L$ is not regular.  Instead,
we will first state a general result, called the Pumping Lemma for
regular languages, for proving that languages are non-regular.  Next,
we will show how the Pumping Lemma can be used to prove that $L$ is
non-regular.  Finally, we will prove the Pumping Lemma.

\begin{lemma}[Pumping Lemma for Regular Languages]
For all regular languages $L$, there is an $n\in\nats-\{0\}$ such that,
for all $z\in\Str$, if $z\in L$ and $|z|\geq n$, then
there are $u,v,w\in\Str$ such that $z=uvw$ and
\begin{enumerate}[\quad(1)]
\item $|uv|\leq n$;

\item $v\neq\%$; and

\item $uv^iw\in L$, for all $i\in\nats$.
\end{enumerate}
\end{lemma}

When we use the Pumping Lemma, we can imagine that we are interacting
with it.  We can give the Pumping Lemma a regular language $L$, and
the lemma will give us back a non-zero natural number $n$ such that
the property of the lemma holds.  We have no control over the value of
$n$; all we know is that $n\geq 1$ and the property of the lemma holds.
We can then give the lemma a string $z$ that is in $L$ and has at
least $n$ symbols. We'll have to choose $z$ as a function of $n$,
because we don't know what $n$ is. The lemma will then break $z$ up
into parts $u$, $v$ and $w$ in such way that (1)--(3) hold.  We have
no control over how $z$ is broken up into these parts; all we know is
that $z=uvw$ and (1)--(3) hold.  (1) says that $uv$ has no more than
$n$ symbols.  (2) says that $v$ is nonempty.  And (3) says that, if we
``pump'' (duplicate) $v$ as many times as we like, the resulting
string will still be in $L$.

Before proving the Pumping Lemma, let's see how it can be used to
prove that $L=\setof{\zerosf^n\onesf^n}{n\in\nats}$ is non-regular.

\begin{proposition}
\label{NonRegularProp1}
$L$ is not regular.
\end{proposition}

\begin{proof}
Suppose, toward a contradiction, that $L$ is regular.
Thus there is an $n\in\nats-\{0\}$ with the property of the Pumping Lemma.
Suppose $z=\zerosf^n\onesf^n$.  Since $z\in L$ and
$|z|=2n\geq n$, it follows that there are $u,v,w\in\Str$ such that
$z=uvw$ and properties (1)--(3) of the lemma hold.  Since
$\zerosf^n\onesf^n=z=uvw$, (1) tells us that
there are $i,j,k\in\nats$ such that
\begin{gather*}
u=\zerosf^i,\quad
v=\zerosf^j,\quad
w=\zerosf^k\onesf^n,\quad
i+j+k=n .
\end{gather*}
By (2), we have that
$j\geq 1$, and thus that $i+k = n - j < n$.  By
(3), we have that
\begin{gather*}
\zerosf^{i+k}\onesf^n=\zerosf^i\zerosf^k\onesf^n=uw=u\%w=uv^0w\in L .
\end{gather*}
Thus $i+k=n$---contradiction.  Thus $L$ is not regular.
\end{proof}

In the preceding proof, we obtained a contradiction by pumping zero
times ($uv^0w$), but pumping two or more times ($uv^2w$, $\ldots$) would
also have worked.  For a case when pumping zeros times is
insufficient, consider $A=\setof{\zerosf^n\onesf^m}{n<m}$.  Given
$n\in\nats-\{0\}$ by the Pumping Lemma, we can let
$z=\zerosf^n\onesf^{n+1}$, obliging the lemma to split $z$ into $uvw$,
in such a way that (1)--(3) hold.  Hence $v$ will consist entirely of
$\zerosf$'s.  Pumping $v$ zero times won't take us outside of $A$.  On
the other hand $uv^2w$ will have at least as many $\zerosf$'s as
$\onesf$'s, giving us the needed contradiction.

Now, let's prove the Pumping Lemma.

\begin{proof}
Suppose $L$ is a regular language.  Thus there is a NFA $M$ such that
$L(M)=L$.  Let $n=|Q_M|$.  Since $Q_M\neq\emptyset$, we have
$n\geq 1$, so that $n\in\nats-\{0\}$.  Suppose $z\in\Str$, $z\in L$
and $|z|\geq n$.  Let $m=|z|$.  Thus $1\leq n\leq|z|=m$.  Since
$z\in L=L(M)$ and $M$ is an NFA, there is a valid labeled path for $M$
\begin{gather*}
q_1\lparr{a_1}q_2\lparr{a_2}\cdots\,q_m\lparr{a_m}q_{m+1} ,
\end{gather*}
that is labeled by $z$ and where
$q_1=s_M$, $q_{m+1}\in A_M$ and $a_i\in\Sym$ for all $1\leq
i\leq m$.  Since $|Q_M|=n$,
not all of the states $q_1,\,\ldots,q_{n+1}$ are
distinct.  Thus, there are $1\leq i<j\leq n+1$ such that
$q_i=q_j$.

Hence, our path looks like:
\begin{gather*}
q_1\lparr{a_1}\cdots\,q_{i-1}\lparr{a_{i-1}}q_i
\lparr{a_i}\cdots\,q_{j-1}\lparr{a_{j-1}}q_j
\lparr{a_j}\cdots\,q_m\lparr{a_m}q_{m+1} .
\end{gather*}
Let
\begin{gather*}
u=a_1\cdots a_{i-1},\quad v=a_i\cdots a_{j-1},\quad
w=a_j\cdots a_m .
\end{gather*}
Then $z=uvw$.  Since $|uv|=j-1$ and $j\leq n+1$,
we have that $|uv|\leq n$.  Since $i<j$, we have that $i\leq j-1$,
and thus that $v\neq\%$.

Finally, since
\begin{gather*}
q_i\in\Delta(\{q_1\},u),\quad q_j\in\Delta(\{q_i\},v),\quad
q_{m+1}\in\Delta(\{q_j\},w)
\end{gather*}
and $q_i=q_j$, we have that
\begin{gather*}
q_j\in\Delta(\{q_1\},u),\quad q_j\in\Delta(\{q_j\},v),\quad
q_{m+1}\in\Delta(\{q_j\},w) .
\end{gather*}
Thus, we have that
$q_{m+1}\in\Delta(\{q_1\},uv^iw)$ for all $i\in\nats$.  But
$q_1=s_M$ and $q_{m+1}\in A_M$, and thus $uv^iw\in L(M)=L$ for all $i\in\nats$.
\end{proof}

Suppose $L'=\setof{w\in\mathsf{\{0,1\}^*}}{w\eqtxt{has an equal number of}
\zerosf\eqtxtr{'s and}\onesf\eqtxtn{'s}}$.
We could show that $L'$ is non-regular using the Pumping Lemma.
But we can also prove this result by using some of the closure
properties of Section~\ref{ClosurePropertiesOfRegularLanguages}
plus the fact that
$L=\setof{\zerosf^n\onesf^n}{n\in\nats}$ is non-regular.

Suppose, toward a contradiction, that $L'$ is regular.  It is easy to
see that $\{\zerosf\}$ and $\{\onesf\}$ are regular (e.g., they are
generated by the regular expressions $\zerosf$ and $\onesf$).  Thus,
by Theorem~\ref{ClosurePropTheorem}, we have that
$\{\zerosf\}^*\{\onesf\}^*$ is regular.  Hence, by
Theorem~\ref{ClosurePropTheorem} again, it follows that
$L=L'\cap\{\zerosf\}^*\{\onesf\}^*$ is regular---contradiction.  Thus
$L'$ is non-regular.

\begin{exercise}
\label{RegPumpEx1}
Let $X = \setof{\zerosf^i\onesf^j}{i,j\in\nats\eqtxt{and}i\neq j}$.
Prove $X$ is non-regular both directly (hint: use factorial) and
via closure properties.
\end{exercise}

As a final example, let $X$ be the least subset of
$\{\mathsf{0,1}\}^*$ such that
\begin{enumerate}[\quad(1)]
\item $\%\in X$; and

\item For all $x,y\in X$, $\zerosf x\onesf y\in X$.
\end{enumerate}
Let's try to prove that $X$ is non-regular, using the Pumping Lemma.
We suppose, toward a contradiction, that $X$ is regular, and give it
to the Pumping Lemma, getting back the $n\in\nats-\{0\}$ with the property
of the lemma, where $X$ has been substituted for $L$.  But then, how
do we go about choosing the $z\in\Str$ such that $z\in X$ and $|z|\geq
n$?  We need to find a string expression $\myexp$ involving the
variable $n$, such that, for all $n\in\nats$, $\myexp\in X$
and $|\myexp|\geq n$.

Because $\%\in X$, we have that
$\zerosf\onesf=\zerosf\%\onesf\%\in X$.  Thus
$\zerosf\onesf\zerosf\onesf=\zerosf\%\onesf(\zerosf\onesf)\in X$.
Generalizing, we can easily prove that, for all $n\in\nats$,
$(\zerosf\onesf)^n\in X$.  Thus we could let $z=(\zerosf\onesf)^n$.
Unfortunately, this won't lead to the needed contradiction, since the
Pumping Lemma may have chosen $n=2$, and may break $z$
up into $u=\%$, $v=\zerosf\onesf$ and $w=(\zerosf\onesf)^{n-1}$---with
the consequence that (1)--(3) hold.

Trying again, we have that $\%\in X$, $\zerosf\onesf\in X$ and
$\zerosf(\zerosf\onesf)\onesf\%=\zerosf\zerosf\onesf\onesf\in X$.
Generalizing, it's easy to prove that, for all $n\in\nats$,
$\zerosf^n\onesf^n\in X$.  Thus, we can let $z=\zerosf^n\onesf^n$, so
that $z\in X$ and $|z|\geq n$.  We can then proceed as in the proof
that $\setof{\zerosf^n\onesf^n}{n\in\nats}$ is non-regular, getting to
the point where we learn that $\zerosf^{i+k}\onesf^n\in X$ and
$i+k<n$.  But an easy induction on $X$ suffices to show that, for all
$w\in X$, $w$ has an equal number of $\zerosf$'s and $\onesf$'s.
Hence $i+k=n$, giving us the needed contradiction.
\index{pumping lemma!regular languages|)}%
\index{regular language!pumping lemma|)}%

\subsection{Experimenting with the Pumping Lemma Using Forlan}

\index{pumping lemma!experimenting with}%
The Forlan module \texttt{LP}
(see Section~\ref{FiniteAutomataAndLabeledPaths}) defines a type and
several functions that implement the idea behind the pumping lemma:
\begin{verbatim}
type pumping_division = lp * lp * lp

val checkPumpingDivision       : pumping_division -> unit
val validPumpingDivision       : pumping_division -> bool
val strsOfValidPumpingDivision :
      pumping_division -> str * str * str
val pumpValidPumpingDivision   : pumping_division * int -> lp
val findValidPumpingDivision   : lp -> pumping_division
\end{verbatim}
\index{LP@\texttt{LP}!pumping_division@\texttt{pumping\_division}}%
\index{LP@\texttt{LP}!checkPumpingDivision@\texttt{checkPumpingDivision}}%
\index{LP@\texttt{LP}!validPumpingDivision@\texttt{validPumpingDivision}}%
\index{LP@\texttt{LP}!strsOfValidPumpingDivision@\texttt{strsOfValidPumpingDivision}}%
\index{LP@\texttt{LP}!pumpValidPumpingDivision@\texttt{pumpValidPumpingDivision}}%
\index{LP@\texttt{LP}!findValidPumpingDivision@\texttt{findValidPumpingDivision}}%
A \emph{pumping division} is a triple $(\lp_1,\lp_2,\lp_3)$,
where $\lp_1,\lp_2,\lp_3\in\LP$.  We say that a pumping division
$(\lp_1,\lp_2,\lp_3)$ is \emph{valid} iff
\begin{itemize}
\item the end state of $\lp_1$ is equal to the start state of $\lp_2$;

\item the start state of $\lp_2$ is equal to the end state of $\lp_2$;

\item the end state of $\lp_2$ is equal to the start state of $\lp_3$; and

\item the label of $\lp_2$ is nonempty.
\end{itemize}
The function \texttt{checkPumpingDivision} checks whether a pumping
division is valid, silently returning \texttt{()}, if it is, and
issuing an error message explaining why it isn't, if it isn't.  The
function \texttt{validPumpingDivision} tests whether a pumping
division is valid.  The function \texttt{strsOfValidPumpingDivision}
returns the triple consisting of the labels of the three components of
a pumping division, in order.  It issues an error message if the pumping
division isn't valid.
The function \texttt{pumpValidPumpingDivision} expects a pair
$(\pd, n)$, where $\pd$ is a valid pumping division and $n\geq 0$.
It issues an error message if $\pd$ isn't valid, or $n$ is negative.
Otherwise, it returns
\begin{gather*}
\join(\hash{1}\,\pd,\join(\lp',\join(\hash{3}\,\pd))) ,
\end{gather*}
where $\lp'$ is the result of joining $\hash{2}\,\pd$ with itself $n$
times (the empty labeled path whose single state is $\hash{2}\,\pd$'s
start/end state, if $n=0$).  Finally, the function
\texttt{findValidPumpingDivision} takes in a labeled path $\lp$, and
tries to find a pumping division $(\lp_1, \lp_2, \lp_3)$ such that:
\begin{itemize}
\item $(\lp_1, \lp_2, \lp_3)$ is valid;

\item $\mathtt{pumpValidPumpingDivision}((\lp_1, \lp_2, \lp_3), 1) = \lp$;
  and

\item there is no repetition of states in the result of joining $\lp_1$ and
  the result of removing the last step of $\lp_2$.
\end{itemize}
\texttt{findValidPumpingDivision} issues an error message if
no such pumping division exists.

For example, suppose the DFA \texttt{dfa} is bound to the
DFA
\begin{center}
\input{chap-3.14-fig1.eepic}
\end{center}
Then we can proceed as follows:
\begin{list}{}
{\setlength{\leftmargin}{\leftmargini}
\setlength{\rightmargin}{0cm}
\setlength{\itemindent}{0cm}
\setlength{\listparindent}{0cm}
\setlength{\itemsep}{0cm}
\setlength{\parsep}{0cm}
\setlength{\labelsep}{0cm}
\setlength{\labelwidth}{0cm}
\catcode`\#=12
\catcode`\$=12
\catcode`\%=12
\catcode`\^=12
\catcode`\_=12
\catcode`\.=12
\catcode`\?=12
\catcode`\!=12
\catcode`\&=12
\ttfamily}
\small
\item[]\textsl{-\ }val\ lp\ =\ DFA.findAcceptingLP\ dfa\ (Str.input\ "");
\item[]\textsl{@\ }001010
\item[]\textsl{@\ }.
\item[]\textsl{val\ lp\ =\ -\ :\ lp}
\item[]\textsl{-\ }LP.output("",\ lp);
\item[]\textsl{A,\ 0\ =>\ B,\ 0\ =>\ C,\ 1\ =>\ A,\ 0\ =>\ B,\ 1\ =>\ B,\ 0\ =>\ C}
\item[]\textsl{val\ it\ =\ ()\ :\ unit}
\item[]\textsl{-\ }val\ pd\ =\ LP.findValidPumpingDivision\ lp;\ 
\item[]\textsl{val\ pd\ =\ (-,-,-)\ :\ LP.pumping_division}
\item[]\textsl{-\ }val\ (lp1,\ lp2,\ lp3)\ =\ pd;
\item[]\textsl{val\ lp1\ =\ -\ :\ lp}
\item[]\textsl{val\ lp2\ =\ -\ :\ lp}
\item[]\textsl{val\ lp3\ =\ -\ :\ lp}
\item[]\textsl{-\ }LP.output("",\ lp1);
\item[]\textsl{A}
\item[]\textsl{val\ it\ =\ ()\ :\ unit}
\item[]\textsl{-\ }LP.output("",\ lp2);
\item[]\textsl{A,\ 0\ =>\ B,\ 0\ =>\ C,\ 1\ =>\ A}
\item[]\textsl{val\ it\ =\ ()\ :\ unit}
\item[]\textsl{-\ }LP.output("",\ lp3);
\item[]\textsl{A,\ 0\ =>\ B,\ 1\ =>\ B,\ 0\ =>\ C}
\item[]\textsl{val\ it\ =\ ()\ :\ unit}
\item[]\textsl{-\ }val\ (u,\ v,\ w)\ =\ LP.strsOfValidPumpingDivision\ pd;
\item[]\textsl{val\ u\ =\ \symbol{'133}\symbol{'135}\ :\ str}
\item[]\textsl{val\ v\ =\ \symbol{'133}-,-,-\symbol{'135}\ :\ str}
\item[]\textsl{val\ w\ =\ \symbol{'133}-,-,-\symbol{'135}\ :\ str}
\item[]\textsl{-\ }(Str.toString\ u,\ Str.toString\ v,\ Str.toString\ v);
\item[]\textsl{val\ it\ =\ ("%","001","001")\ :\ string\ \symbol{'052}\ string\ \symbol{'052}\ string}
\item[]\textsl{-\ }val\ lp'\ =\ LP.pumpValidPumpingDivision(pd,\ 2);
\item[]\textsl{val\ lp'\ =\ -\ :\ lp}
\item[]\textsl{-\ }LP.output("",\ lp');
\item[]\textsl{A,\ 0\ =>\ B,\ 0\ =>\ C,\ 1\ =>\ A,\ 0\ =>\ B,\ 0\ =>\ C,\ 1\ =>\ A,\ 0\ =>\ B,\ 1\ =>}
\item[]\textsl{B,\ 0\ =>\ C}
\item[]\textsl{val\ it\ =\ ()\ :\ unit}
\item[]\textsl{-\ }Str.output("",\ LP.label\ lp');
\item[]\textsl{001001010}
\item[]\textsl{val\ it\ =\ ()\ :\ unit}
\end{list}

\index{regular language!showing that languages are non-regular|)}%

\subsection{Notes}

The Pumping Lemma is usually proved using a DFA accepting the given
regular language.  But because we have described the meaning of
automata via labeled paths, we can do the proof with an NFA, as it has
nothing to do with determinacy.  Forlan's support for experimenting
with the Pumping Lemma is novel.

%%% Local Variables: 
%%% mode: latex
%%% TeX-master: "book"
%%% End: 
