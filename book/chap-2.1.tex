\section{Symbols, Strings, Alphabets and (Formal) Languages}
\label{SymbolsStringsAlphabetsAndFormalLanguages}

In this section, we define the basic notions of the subject: symbols,
strings, alphabets and (formal) languages.
In most presentations of formal language theory, the ``symbols'' that
make up strings are allowed to be arbitrary elements of the
mathematical universe.  This is convenient in some ways, but it means
that, e.g., the collection of all strings is too ``big'' to be a set.
Furthermore, if we were to adopt this convention, we wouldn't be
able to have notation in Forlan for all strings and symbols.  These
considerations lead us to the following definition.

\subsection{Symbols}

\index{symbol|(}%
The set $\Char$ of \emph{symbol characters} consists of the
following $65$ elements:
\begin{itemize}
\item the comma (``$,$'');

\item the \emph{digits} $\mathsf{0}$--$\mathsf{9}$;

\item the \emph{letters} $\mathsf{a}$--$\mathsf{z}$ and
  $\mathsf{A}$--$\mathsf{Z}$; and

\item the angle brackets (``$\langle$'' and ``$\rangle$'').
\end{itemize}
We order $\Char$ as follows:
\begin{displaymath}
  {,} <
  \mathsf{0} < \cdots < \mathsf{9} <
  \mathsf{a} < \cdots \mathsf{z} < 
  \mathsf{A} < \cdots \mathsf{Z} <
  {\langle} < {\rangle} .
\end{displaymath}

The set $\Sym$ of \emph{symbols} is the least subset of
\index{Sym@$\Sym$}%
\index{symbol!Sym@$\Sym$}%
$\List\,\Char$ such that:
\begin{itemize}
\item for all digits and letters $c$, $[c]\in\Sym$; and

\item for all $n\in\nats$ and $x_1,\ldots,x_n\in\{[\,,]\}\cup\Sym$,
  \begin{displaymath}
    [\,\langle\,] \myconcat x_1 \myconcat \cdots \myconcat x_n \myconcat
    [\,\rangle\,]\in\Sym . 
  \end{displaymath}
\end{itemize}
This is an inductive definition (see
Section~\ref{TreesAndInductiveDefinitions}).  $\Sym$ consists of just
those lists of symbol characters that can be built using the above,
two rules.  For example, $[\mathsf{9}]$, $[\,\langle,\,\rangle]$,
$\mathsf{[\,\langle, \,i, \,d, \,\rangle]}$ and
$\mathsf{[\,\langle, \,\langle, \,a, \,,, \,\rangle, \,b, \,\rangle]}$
are symbols.  On the other hand, $[\,\langle,\,\rangle,\,\rangle]$ is
not a symbol.

We can prove by induction that, for all $z\in\Sym$, for all
$x,y\in\List\,\Char$, if $z = x\myconcat y\in\Sym$, then:
\begin{itemize}
\item if $x\in\Sym$, then $y=[\,]$;
\item if $y\in\Sym$, then $x=[\,]$.
\end{itemize}
Thus a symbol never starts or ends with another symbol.

We normally abbreviate a symbol $[c_1,\ldots,c_n]$ to $c_1\cdots c_n$,
so that $\mathsf{9}$, $\langle\,\rangle$, $\langle\mathsf{id}\rangle$
and $\mathsf{\langle\langle a,\rangle b\rangle}$ are symbols.  And if
$x$ and $y$ are elements of $\List\,\Char$, we
typically abbreviate $x\myconcat y$ to
$xy$.

Whenever possible, we will use the mathematical variables $a$, $b$ and
$c$ to name symbols.
\index{a, b, c@$a,b,c$}%
\index{symbol!a, b, c@$a,b,c$}%
We order $\Sym$ first by length, and then lexicographically (in
dictionary order).  So, we have that
\begin{gather*}
\mathsf{0} < \cdots < \mathsf{9} < \mathsf{A} < \cdots < \mathsf{Z}
< \mathsf{a} < \cdots < \mathsf{z} ,
\end{gather*}
and, e.g.,
\begin{gather*}
\mathsf{z} < \mathsf{\langle be\rangle} < \mathsf{\langle by\rangle} <
\mathsf{\langle on\rangle} < \mathsf{\langle can\rangle} <
\mathsf{\langle con\rangle} .
\end{gather*}
\index{symbol!ordering}%
Obviously, $\Sym$ is infinite, but is it countably infinite?
\index{countably infinite}%
The answer is ``yes'', because we can enumerate the symbols in order.
\index{symbol|)}%

\subsection{Strings}

\index{string|(}%
A \emph{string}
\index{list}%
is a list of symbols.
Whenever possible, we will use the mathematical variables $u$,
\index{u, v, w, x, y, z, w@$u, v, w, x, y, z$}%
\index{string!u, v, w, x, y, z, w@$u, v, w, x, y, z$}%
$v$, $w$, $x$, $y$ and $z$ to name strings.

We typically abbreviate the empty string $[\,]$ to $\%$, and
\index{ percent@$\%$}%
\index{string!empty}%
\index{string! percent@$\%$}%
abbreviate $[a_1,\ldots,a_n]$ to $a_1\cdots a_n$, when $n\geq 1$.
For example
$[\zerosf,\langle\zerosf\rangle,\onesf,
\langle\langle,\rangle\rangle]$ is abbreviated to $\mathsf{0\langle
  0\rangle1\langle\langle,\rangle\rangle}$.
We name the empty string by $\%$, instead of following convention and
using $\epsilon$, since this symbol can also be used in Forlan.

We write $\Str$ for $\List\,\Sym$, the set of all strings.
\index{Str@$\Str$}%
\index{string!Str@$\Str$}%
We order
\index{string!ordering}%
$\Str$ first by length and then lexicographically, using our order on
$\Sym$.  Thus, e.g.,
\begin{gather*}
\% < \mathsf{ab} < \mathsf{a\langle be\rangle} < \mathsf{a\langle by\rangle} <
\mathsf{\langle can\rangle\langle be\rangle} < \mathsf{abc} .
\end{gather*}
Since every string can be written as a finite sequence of ASCII
characters, it follows that $\Str$ is countably infinite.
\index{countably infinite}%

Because strings are lists, we have that $|x|$ is the \emph{length}
\index{length!string}%
\index{string!length}%
\index{ size of@$\sizedot$}%
\index{string! size of@$\sizedot$}%
of a string $x$, and that $x\myconcat y$ is the \emph{concatenation}
\index{string!concatenation}%
\index{concatenation!string}%
of strings $x$ and $y$.
We typically abbreviate $x\myconcat y$ to $xy$.
For example:
\begin{itemize}
\item $|\%| = |[\,]| = 0$;

\item $|\mathsf{0\langle 0\rangle1\langle\langle\,,\rangle\rangle}| =
|[\zerosf,\langle\zerosf\rangle,\onesf, \langle\langle\,,\rangle\rangle]| = 4$;
and

\item $(\mathsf{01})(\mathsf{00}) =
\mathsf{[0,1]}\myconcat\mathsf{[0,0]} =
\mathsf{[0,1,0,0]} = \mathsf{0100}$.
\end{itemize}

From our study of lists, we know that:
\begin{itemize}
\item Concatenation is associative: for all $x,y,z\in\Str$,
\index{associative!string concatenation}%
\index{concatenation!string!associative}%
\index{string!concatenation!associative}%
\begin{gather*}
(xy)z = x(yz) .
\end{gather*}

\item $\%$ is the identify for concatenation: for all $x\in\Str$,
\index{identity!string concatenation}%
\index{concatenation!string!identity}%
\index{string!concatenation!identity}%
\begin{gather*}
\%x=x=x\% .
\end{gather*}
\end{itemize}

It is easy to see that, for all $x,y,x',y'\in\Str$:
\begin{itemize}
\item $xy=xy'$ iff $y=y'$; and

\item $xy=x'y$ iff $x=x'$.
\end{itemize}

On the other hand:

\begin{exercise}
Disprove the following statement: for all
$x,y,x',y'\in\Str$, $xy = x'y'$ iff $x=x'$ and $y=y'$.
\end{exercise}

We define the string $x^n$ \emph{formed by raising}
a string $x$ \emph{to the power}
\index{string!power}%
\index{concatenation!string!power}%
\index{string!concatenation!power}%
\index{string!exponentiation}%
\index{concatenation!string!exponentiation}%
\index{string!concatenation!exponentiation}%
\index{ power@$\cdot^\cdot$}%
\index{string! power@$\cdot^\cdot$}%
$n\in\nats$ by recursion
\index{recursion!natural numbers}%
on $n$:
\begin{align*}
x^0       &= \%,\eqtxt{for all}x\in\Str ; \eqtxtl{and} \\
x^{n + 1} &= xx^n,\eqtxt{for all}x\in\Str\eqtxt{and}n\in\nats .
\end{align*}
We assign this operation higher precedence than concatenation, so that
$xx^n$ means $x(x^n)$ in the above definition.
For example, we have that
\begin{gather*}
(\mathsf{ab})^2 = 
(\mathsf{ab})(\mathsf{ab})^1 =
(\mathsf{ab})(\mathsf{ab})(\mathsf{ab})^0 =
(\mathsf{ab})(\mathsf{ab})\% =
\mathsf{abab} .
\end{gather*}

\begin{proposition}
\label{StrPowerProp}
For all $x\in\Str$ and $n,m\in\nats$, $x^{n+m}=x^nx^m$.
\end{proposition}
\index{string!power}%
\index{concatenation!string!power}%
\index{string!concatenation!power}%

\begin{proof}
Suppose $x\in\Str$ and $m\in\nats$.  We use mathematical induction
\index{mathematical induction}%
to show that, for all $n\in\nats$, $x^{n+m} = x^nx^m$.
\begin{description}
\item[\quad(Basis Step)] We have that $x^{0+m}=x^m=\%x^m=x^0x^m$.

\item[\quad(Inductive Step)] Suppose $n\in\nats$, and assume the
inductive hypothesis: $x^{n+m} = x^nx^m$.  We must show
that $x^{(n+1)+m} = x^{n+1}x^m$.  We have that
\begin{alignat*}{2}
x^{(n+1)+m} &= x^{(n+m)+1} \\
            &= xx^{n+m} && \by{definition of $x^{(n+m)+1}$} \\
            &= x(x^nx^m) && \by{inductive hypothesis} \\
            &= (xx^n)x^m \\
            &= x^{n+1}x^m && \by{definition of $x^{n+1}$} .
\end{alignat*}
\end{description}
\end{proof}

Thus, if $x\in\Str$ and $n\in\nats$, then
\begin{alignat*}{2}
x^{n+1} &= xx^n && \by{definition}, \\
\intertext{and}
x^{n+1} &= x^nx^1 = x^nx && \by{Proposition~\ref{StrPowerProp}} .
%\by{Proposition~\ref{StrPowerProp}} .
\end{alignat*}

Next, we consider the prefix, suffix and substring relations on
strings.  Suppose $x$ and $y$ are strings.  We say that:
\begin{itemize}
\item $x$ is a \emph{prefix} of $y$ iff $y=xv$ for some $v\in\Str$;
\index{prefix}%
\index{string!prefix}%

\item $x$ is a \emph{suffix} of $y$ iff $y=ux$ for some $u\in\Str$; and
\index{suffix}%
\index{string!suffix}%

\item $x$ is a \emph{substring} of $y$ iff $y=uxv$ for some $u,v\in\Str$.
\index{substring}%
\index{string!substring}%
\end{itemize}
In other words, $x$ is a prefix of $y$ iff $x$ is an initial part of
$y$, $x$ is a suffix of $y$ iff $x$ is a trailing part of $y$, and $x$
is a substring of $y$ iff $x$ appears in the middle of $y$.  But note
that the strings $u$ and $v$ can be empty in these definitions.  Thus,
e.g., a string $x$ is always a prefix of itself, since $x=x\%$.  A
prefix, suffix or substring of a string other than the string itself
is called \emph{proper}.
\index{proper!prefix}%
\index{proper!suffix}%
\index{proper!substring}%
\index{string!prefix!proper}%
\index{string!suffix!proper}%
\index{string!substring!proper}%
\index{prefix!proper}%
\index{suffix!proper}%
\index{substring!proper}%

For example:
\begin{itemize}
\item $\%$ is a proper prefix, suffix and substring of
$\mathsf{ab}$;

\item $\mathsf{a}$ is a proper prefix and substring of
$\mathsf{ab}$;

\item $\mathsf{b}$ is a proper suffix and substring of
$\mathsf{ab}$; and

\item $\mathsf{ab}$ is a (non-proper) prefix, suffix and
substring of $\mathsf{ab}$.
\end{itemize}

\begin{proposition}
For all $x,y,x',y'\in\Str$, $xy=x'y'$ iff
\begin{itemize}
\item $xu=x'$ and $y=uy'$, for some $u\in\Str$, or

\item $x'u=x$ and $y'=uy$, for some $u\in\Str$.
\end{itemize}
\end{proposition}

\begin{proof}
Straightforward.
\end{proof}

As a consequence of this proposition, we have that:
\begin{itemize}
\item For all $x,x',y'\in\Str$, $x$ is a prefix of $x'y'$ iff
\begin{itemize}
\item $x$ is a prefix of $x'$, or

\item $x=x'u$, for some prefix $u$ of $y'$.
\end{itemize}

\item For all $x,x',y'\in\Str$, $x$ is a suffix of $x'y'$ iff
\begin{itemize}
\item $x$ is a suffix of $y'$, or

\item $x=uy'$, for some suffix $u$ of $x'$.
\end{itemize}

\item For all $x,x',y'\in\Str$, $x$ is a substring of $x'y'$ iff
\begin{itemize}
\item $x$ is a substring of $x'$, or

\item $x=uv$, for some $u,v\in\Str$ such that
  $u$ is a suffix of $x'$ and $v$ is a prefix of $y'$, or

\item $x$ is a substring of $y'$.
\end{itemize}
\end{itemize}

\begin{exercise}
Suppose $a\in\Sym$, $x,y\in\Str$. Prove that:
\begin{enumerate}[(1)]
\item If $a^ix = a^jy$ and neither $x$ nor $y$ begins with $a$,
then $i=j$ and $x=y$;

\item If $xa^i = ya^j$ and neither $x$ nor $y$ ends with $a$,
then $x=y$ and $i=j$.
\end{enumerate}
\end{exercise}

\begin{exercise}
Suppose $a,b\in\Sym$ and $a\neq b$. Disprove the following statement:
for all $i, i', j, j', k, k'\in\nats$, $a^i b^j a^k = a^{i'} b^{j'} a^{k'}$
iff $i = i'$, $j = j'$ and $k = k'$.
\end{exercise}

\index{string|)}%

\subsection{Alphabets}

\index{alphabet|(}%
Having said what symbols and strings are, we now come to alphabets.
An \emph{alphabet}
\index{alphabet}%
is a finite subset of $\Sym$.  We use $\Sigma$
\index{Sigma@$\Sigma$}%
\index{alphabet!Sigma@$\Sigma$}%
(upper case Greek letter sigma) to name alphabets.  For example,
$\emptyset$, $\mathsf{\{0\}}$ and $\mathsf{\{0,1\}}$ are alphabets.
We write $\Alp$ for the set of all alphabets.  $\Alp$ is countably
infinite.
\index{countably infinite}%

We define $\alphabet\in\Str\fun\Alp$
\index{alphabet@$\alphabet$}%
\index{string!alphabet}%
\index{string!alphabet@$\alphabet$}%
by recursion
\index{recursion!string}%
\index{recursion!string!right}%
\index{recursion!string!left}%
on (the length of) strings:
\begin{align*}
\alphabet\,\% &= \emptyset , \\
\alphabet(ax) &= \{a\} \cup \alphabet\,x,
\eqtxt{for all}a\in\Sym\eqtxt{and}x\in\Str .
\end{align*}
I.e., $\alphabet\,w$ consists of all of the symbols occurring in the
string $w$.  E.g., $\alphabet(\mathsf{01101})=\{\zerosf,\onesf\}$.
Because the string $x$ appears on the right side of $ax$ in the rule
$\alphabet(ax) = \{a\} \cup \alphabet\,x$, we call this \emph{right}
recursion.  (Since $\cup$ is associative and commutative, it would
have been equivalent to use \emph{left} recursion, $\alphabet(xa) =
\{a\} \cup \alphabet\,x$.)  We say that $\alphabet\,x$ is the
\emph{alphabet of} $x$.

If $\Sigma$ is an alphabet, then we write $\Sigma^*$
\index{ star@$\cdot^*$}%
\index{alphabet! star@$\cdot^*$}%
for $\List\,\Sigma$.
I.e., $\Sigma^*$ consists of all of the strings that can be built
using the symbols of $\Sigma$.
For example, the elements of $\mathsf{\{0,1\}^*} = \List\,\mathsf{\{0,1\}}$ are:
\begin{gather*}
\%, \mathsf{0}, \mathsf{1}, \mathsf{00}, \mathsf{01}, \mathsf{10},
\mathsf{11},\mathsf{000},\, \ldots
\end{gather*}
\index{alphabet|)}%

\subsection{Languages}

\index{language|(}%
We say that $L$ is a \emph{formal language}
\index{formal language|see{language}}%
(or just \emph{language}) iff $L\sub\Sigma^*$, for some
$\Sigma\in\Alp$.  In other words, a language is a set of strings over
some alphabet.
If $\Sigma\in\Alp$, then we say that $L$ is a
$\Sigma$-\emph{language}
\index{Sigma-language@$\Sigma$-language}%
\index{language!Sigma-language@$\Sigma$-language}%
iff $L\sub\Sigma^*$.

Here are some example languages (all are $\mathsf{\{0,1\}}$-languages):
\begin{itemize}
\item $\emptyset$;

\item $\mathsf{\{0,1\}^*}$;

\item $\mathsf{\{010,1001,1101\}}$;

\item $\setof{\mathsf{0}^n\mathsf{1}^n}{n\in\nats} =
\{\zerosf^0\onesf^0, \zerosf^1\onesf^1, \zerosf^2\onesf^2, \ldots\} =
\mathsf{\{\%, 01, 0011, \ldots\}}$; and

\item $\setof{w\in\mathsf{\{0,1\}^*}}{w\eqtxtl{is a palindrome}}$.
\end{itemize}
(A \emph{palindrome}
\index{palindrome}%
\index{string!palindrome}%
is a string that reads the same backwards and forwards,
i.e., that is equal to its own reversal.)
On the other hand, the set of strings
$X=\mathsf{\{\langle\rangle, \langle 0\rangle, \langle 00\rangle}, \ldots\}$,
is not a language, since it involves infinitely many symbols, i.e.,
since there is no alphabet $\Sigma$ such that $X\sub\Sigma^*$.

Since $\Str$ is countably infinite and every language is a subset
of $\Str$, it follows that every language is countable.
\index{countable}%
Furthermore, $\Sigma^*$ is countably infinite,
\index{countably infinite}%
as long as the alphabet
$\Sigma$ is nonempty ($\emptyset^*=\{\%\}$).

We write $\Lan$
\index{Lan@$\Lan$}%
\index{language!Lan@$\Lan$}%
for the set of all languages.  It turns out that $\Lan$ is
uncountable.
\index{uncountable}%
In fact even
$\powset(\mathsf{\{0\}}^*)$, the set of all
$\mathsf{\{0\}}$-languages, has the same size as $\powset(\nats)$,
and is thus uncountable.

\begin{exercise}
Show that $\powset(\nats)$ has the same size as  $\powset(\mathsf{\{0\}}^*)$.
\end{exercise}

Given a language $L$, we write $\alphabet\,L$
\index{alphabet@$\alphabet$}%
\index{language!alphabet@$\alphabet$}%
for the \emph{alphabet}
\index{alphabet!language}%
\index{language!alphabet}%
\begin{gather*}
\bigcup\setof{\alphabet\,w}{w\in L} .
\end{gather*}
\emph{of} $L$.
I.e., $\alphabet\,L$ consists of all of the symbols occurring in the
strings of $L$.
For example,
\begin{align*}
\alphabet\,\mathsf{\{011,112\}} &=
\bigcup\{\alphabet(\mathsf{011}),\alphabet(\mathsf{112})\} \\
&= \bigcup\{\{\zerosf,\onesf\},\{\onesf,\twosf\}\}
= \{\zerosf,\onesf,\twosf\} .
\end{align*}
Note that, for all languages $L$, $L\sub(\alphabet\,L)^*$.

If $A$ is an infinite subset of $\Sym$ (and so is not an alphabet), we
allow ourselves to write $A^*$
\index{ star@$\cdot^*$}%
for $\List\,A$.
I.e., $A^*$ consists of all of the strings that can be built using the
symbols of $A$.  For example, $\Sym^* = \Str$.
\index{Str@$\Str$}%
\index{Sym@$\Sym$}%
\index{language|)}%

\subsection{Notes}

In a traditional approach to the subject, symbols may be anything,
real numbers, sets, etc.  But such a choice would mean that not all
symbols could be expressed in Forlan's syntax, and would needlessly
complicate the set theoretic foundations of the subject.  By working
with a fixed, countably infinite set of symbols, all symbols can be
expressed in Forlan, and we have that that strings, regular
expressions, etc., are sets, not set-indexed families of sets.

Representing strings as lists of symbols, which in turn are
represented as functions, is nontraditional, but should seem a natural
approach to those with a background in set theory or functional
programming.

%%% Local Variables: 
%%% mode: latex
%%% TeX-master: "book"
%%% End: 
