\section{Proving the Correctness of Finite Automata}
\label{ProvingTheCorrectnessOfFiniteAutomata}

\index{finite automaton!proof of correctness|(}
In this section, we consider techniques for proving the correctness
of finite automata, i.e., for proving that finite automata accept
the languages we want them to.

We begin by defining an indexed family of languages, $\Lambda$.

\subsection{Definition of $\Lambda$}

\begin{proposition}
\label{DeltaProp1}
Suppose $M$ is a finite automaton.
\begin{enumerate}[\quad(1)]
\item For all $q\in Q_M$, ${q}\in\Delta_M(\{q\},\%)$.

\item For all $q,r\in Q_M$ and $w\in\Str$, if $q,w\fun r\in T_M$,
then ${r}\in\Delta_M(\{q\},w)$.

\item For all $p,q,r\in Q_M$ and $x,y\in\Str$, if $q\in\Delta_M(\{p\},x)$
and $r\in\Delta_M(\{q\},y)$, then
${r}\in\Delta_M(\{p\},xy)$.
\end{enumerate}
\end{proposition}

Suppose $M$ is a finite automaton and $q\in Q_M$.
Then we define
\index{ Lambda@$\Lambda_{\cdot,\cdot}$}%
\index{finite automaton! Lambda@$\Lambda_{\cdot,\cdot}$}%
\begin{displaymath}
\Lambda_{M,q} = \setof{w\in\Str}{q\in\Delta_M(\{s_M\},w)}.
\end{displaymath}
In other words, $\Lambda_{M,q}$ is the labels of all of the valid
labeled paths for $M$ that start at state $s_M$ and end at $q$, i.e.,
it's all strings that can take us to state $q$ when processed by $M$.
Clearly, $\Lambda_{M,q}\sub(\alphabet\,M)^*$, for all FAs $M$ and
$q\in Q_M$.  If it's clear which FA we are talking about, we sometimes
abbreviate $\Lambda_{M,q}$ to $\Lambda_q$.

Let our example FA, $M$, be
\begin{center}
\input{chap-3.8-fig1.eepic}
\end{center}
Then:
\begin{itemize}
\item $\mathsf{01101}\in\Lambda_\Asf$, because of the labeled path
  \begin{gather*}
    \mathsf{A\lparr{0}B\lparr{1}B\lparr{1}B\lparr{0}A\lparr{1}A} ,
  \end{gather*}

\item $\mathsf{01100}\in\Lambda_\Bsf$, because of the labeled path
  \begin{gather*}
    \mathsf{A\lparr{0}B\lparr{1}B\lparr{1}B\lparr{0}A\lparr{0}B} .
  \end{gather*}
\end{itemize}

\begin{proposition}
\label{LambdaProp1}
Suppose $M$ is an FA.  Then $L(M) = \bigcup\setof{\Lambda_{M,q}}{q\in
  A_M}$, i.e., for all $w$, $w\in L(M)$ iff $w\in\Lambda_{M,q}$ for
some $q\in A_M$.
\end{proposition}

\begin{proof}
\begin{description}
\item[\quad(only if)] Suppose $w\in L(M)$.  By Proposition~3.5.3,
we have that $\Delta_M({s_M},w)\cap A_M\neq\emptyset$, so that there
is a $q\in A_M$ such that $q\in\Delta_M({s_M},w)$.  Thus
$w\in\Lambda_{M,q}$.

\item[\quad(if)] Suppose $w\in\Lambda_{M,q}$ for some $q\in A_M$.
Thus $q\in\Delta_M({s_M},w)$ and $q\in A_M$, so that
$\Delta_M({s_M},w)\cap A_M\neq\emptyset$.  Hence $w\in L(M)$, by
Proposition~3.5.3.
\end{description}
\end{proof}

\begin{proposition}
\label{LambdaProp2}
Suppose $M$ is a finite automaton.
\begin{enumerate}[\quad(1)]
\item $\%\in\Lambda_{M,s_M}$.

\item For all $q,r\in Q_M$ and $w,x\in\Str$.  If $w\in\Lambda_{M,q}$
  and ${q,x\fun r}\in T_M$, then $wx\in\Lambda_{M,r}$.
\end{enumerate}
\end{proposition}

\begin{proof}
\begin{enumerate}[(1)]
\item By Proposition~\ref{DeltaProp1}(1), we have
  that $s_M\in\Delta(\{s_M\},\%)$, so that $\%\in\Lambda_{M,s_M}$.

\item Suppose $q,r\in Q_M$, $w,x\in\Str$, $w\in\Lambda_{M,q}$ and
  $q,x\fun r\in T_M$.  Thus $q\in\Delta(\{s_M\},w)$.  Because
  ${q,x\fun r}\in T_M$, Proposition~\ref{DeltaProp1}(2) tells us that
  $r\in\Delta(\{q\},x)$.  Hence by Proposition~\ref{DeltaProp1}(3), we
  have that ${r}\in\Delta(\{s_M\},wx)$, so that
  ${wx}\in\Lambda_{M,{r}}$.
\end{enumerate}
\end{proof}

Our main example will be the FA, $M$:
\begin{center}
\input{chap-3.8-fig1.eepic}
\end{center}
Let
\begin{align*}
X &= \setof{w\in\{\mathsf{0,1}\}^*}{w\eqtxt{has an even number of}
\zerosf\eqtxtn{'s}} , \eqtxt{and} \\
Y &= \setof{w\in\{\mathsf{0,1}\}^*}{w\eqtxt{has an odd number of}
\zerosf\eqtxtn{'s}} .
\end{align*}

We want to prove that $L(M) = X$.  Because $A_M=\{\Asf\}$,
Proposition~\ref{LambdaProp1} tells us that $L(M)=\Lambda_{M,\Asf}$.
Thus it will suffice to show that $\Lambda_{M,\Asf} = X$.  But our
approach will also involve showing $\Lambda_{M,\Bsf} = Y$.  We would
cope with more states analogously, having one language per state.

\subsection{Proving that Enough is Accepted}

First, we study techniques for showing that everything we want an
automaton to accept is really accepted.

Since $X,Y\sub\{\mathsf{0,1}\}^*$, to prove that
$X\sub\Lambda_{M,\Asf}$ and $Y\sub\Lambda_{M,\Bsf}$, it will suffice
to use strong string induction to show that, for all
$w\in\mathsf{\{0,1\}^*}$:
\begin{enumerate}[\quad(A)]
\item if $w\in X$, then $w\in\Lambda_{M,\Asf}$; and

\item if $w\in Y$, then $w\in\Lambda_{M,\Bsf}$.
\end{enumerate}

We proceed by strong string induction.  Suppose
$w\in\{\mathsf{0,1}\}^*$, and assume the inductive hypothesis:
for all $x\in\{\mathsf{0,1}\}^*$, if $x$ is a proper substring of
$w$, then:
\begin{enumerate}[\quad(A)]
\item if $x\in X$, then $x\in\Lambda_\Asf$; and

\item if $x\in Y$, then $x\in\Lambda_\Bsf$.
\end{enumerate}
We must prove that:
\begin{enumerate}[\quad(A)]
\item if $w\in X$, then $w\in\Lambda_\Asf$; and

\item if $w\in Y$, then $w\in\Lambda_\Bsf$.
\end{enumerate}
There are two parts to show.
\begin{enumerate}[\quad(A)]
\item Suppose $w\in X$, so that $w$ has an even number of $\zerosf$'s.
  We must show that
  $w\in\Lambda_\Asf$.  There are three cases to consider.
  \begin{itemize}
  \item Suppose $w=\%$.  By Proposition~\ref{LambdaProp2}(1), we have
    that $w=\%\in\Lambda_\Asf$.
  
  \item Suppose $w=x\zerosf$, for some $x\in\{\mathsf{0,1}\}^*$.  Thus
    $x$ has an odd number of $\zerosf$'s, so that $x\in Y$.
    Because $x$ is a proper substring of $w$,  part~(B) of
    the inductive hypothesis tells us that $x\in\Lambda_\Bsf$.
    Furthermore, $\Bsf,\zerosf\fun\Asf\in T$, so that
    ${w=x\zerosf}\in\Lambda_\Asf$, by
    Proposition~\ref{LambdaProp2}(2).
  
  \item Suppose $w=x\onesf$, for some $x\in\{\mathsf{0,1}\}^*$.  Thus
    $x$ has an even number of $\zerosf$'s, so that $x\in X$.
    Because $x$ is a proper
    substring of $w$, part~(A) of the inductive hypothesis tells us
    that $x\in\Lambda_\Asf$.  Furthermore, $\Asf,\onesf\fun\Asf\in
    T$, so that $w=x\onesf\in\Lambda_\Asf$, by
    Proposition~\ref{LambdaProp2}(2).
  \end{itemize}

\item Suppose $w\in Y$, so that $w$ has an odd number of $\zerosf$'s.
  We must show that
  $w\in\Lambda_\Bsf$.  There are three cases to consider.
  \begin{itemize}
  \item Suppose $w=\%$.  But the number of $\zerosf$'s in
    $\%$ is $0$, which is even---contradiction.  Thus
    $w\in\Lambda_\Bsf$.
  
  \item Suppose $w=x\zerosf$, for some $x\in\{\mathsf{0,1}\}^*$.  Thus
    $x$ has an even number of $\zerosf$'s, so that $x\in X$.
    Because $x$ is a proper
    substring of $w$, part (A) of the inductive hypothesis tells us
    that $x\in\Lambda_\Asf$.  Furthermore, $\Asf,\zerosf\fun\Bsf\in
    T$, so that $w=x\zerosf\in\Lambda_\Bsf$, by
    Proposition~\ref{LambdaProp2}(2).
  
  \item Suppose $w=x\onesf$, for some $x\in\{\mathsf{0,1}\}^*$.  Thus
    $x$ has an odd number of $\zerosf$'s, so that $x\in Y$.
    Because $x$ is a proper
    substring of $w$, part (B) of the inductive hypothesis tells us
    that $x\in\Lambda_\Bsf$.  Furthermore, $\Bsf,\onesf\fun\Bsf\in
    T$, so that $w=x\onesf\in\Lambda_\Bsf$, by
    Proposition~\ref{LambdaProp2}(2).
  \end{itemize}
\end{enumerate}

Let $N$ be the finite automaton
\begin{center}
\input{chap-3.8-fig2.eepic}
\end{center}
Here we hope that $\Lambda_{N,\Asf} = {\{\zerosf\}^*}$ and $L(N) =
\Lambda_{N,\Bsf} = \{\zerosf\}^*\{\onesf\onesf\}^*$, but if we try to
prove that
\begin{align*}
  \{\zerosf\}^*&\sub\Lambda_{N,\Asf} , \eqtxt{and}\\
  \{\zerosf\}^*\{\onesf\onesf\}^*&\sub\Lambda_{N,\Bsf}
\end{align*}
using our standard technique, there is a complication related to the
$\%$-transition.

We use strong string induction to show that, for all
$w\in\{\mathsf{0,1}\}^*$:
\begin{enumerate}[\quad(A)]
\item if $w\in\{\zerosf\}^*$, then $w\in\Lambda_\Asf$; \eqtxt{and}

\item if $w\in\{\zerosf\}^*\{\onesf\onesf\}^*$, then
  $w\in\Lambda_\Bsf$.
\end{enumerate}

In part~(B), we assume that $w\in\{\zerosf\}^*\{\onesf\onesf\}^*$, so
that $w=\zerosf^n(\onesf\onesf)^m$ for some $n,m\in\nats$.  We must show that
$w\in\Lambda_\Bsf$.  We consider two cases: $m=0$ and $m\geq 1$.
The second of these is straightforward, so let's focus on the first.
Then $w=\zerosf^n\in\{\zerosf\}^*$.  We want to use part~(A) of
the inductive hypothesis to conclude that $\zerosf^n\in\Lambda_\Asf$,
but there is a problem: $\zerosf^n$ is not a proper substring of
$\zerosf^n=w$.

So, we must consider two subcases, when $n=0$ and $n\geq 1$.
In the first subcase, because $\%\in\Lambda_\Asf$ and $\Asf,\%\fun\Bsf\in T$,
we have that $w=\%=\%\%\in\Lambda_\Bsf$.

In the second subcase, we have that $w=\zerosf^{n-1}\zerosf$.  By
part~(A) of the inductive hypothesis, we have that
$\zerosf^{n-1}\in\Lambda_\Asf$.  Thus, because
$\Asf,\zerosf\fun\Asf\in T$ and $\Asf,\%\fun\Bsf\in T$, we can conclude
$w=\zerosf^n=\zerosf^{n-1}\zerosf\%\in\Lambda_\Bsf$.

Because there are no transitions from $\Bsf$ back to $\Asf$, we
could first prove that, for all $w\in\{\mathsf{0,1}\}^*$,
\begin{enumerate}[\quad(A)]
\item if $w\in\{\zerosf\}^*$, then $w\in\Lambda_\Asf$,
\end{enumerate}
and then use (A) to prove that for all $w\in\{\mathsf{0,1}\}^*$,
\begin{enumerate}[\quad(A)]
\setcounter{enumi}{1}
\item if $w\in\{\zerosf\}^*\{\onesf\onesf\}^*$, then
$w\in\Lambda_\Bsf$.
\end{enumerate}

This works whenever one part of a machine has transitions to
another part, but there are no transitions from that second part
back to the first part, i.e., when the two parts are not mutually
recursive.

In the case of $N$, we could use mathematical induction instead of
strong string induction:
\begin{enumerate}[\quad(A)]
\item for all $n\in\nats$, $\zerosf^n\in\Lambda_\Asf$, and

\item for all $n,m\in\nats$, $\zerosf^n(\onesf\onesf)^m\in\Lambda_\Bsf$
(do induction on $m$, fixing $n$).
\end{enumerate}

\subsection{Proving that Everything Accepted is Wanted}

It's tempting to try to prove that everything accepted by a finite
automaton is wanted using strong string induction, with implications
like
\begin{enumerate}[\quad(A)]
\item if $w\in\Lambda_\Asf$, then $w\in X$.
\end{enumerate}
Unfortunately, this doesn't work when a finite automaton contains
$\%$-transitions.  Instead, we do such proofs using a new induction
principle that we call induction on $\Lambda$.

\begin{theorem}[Principle of Induction on $\Lambda$]
Suppose $M$ is a finite automaton, and $P_q(w)$ is a property
of a $w\in\Lambda_{M,q}$, for all $q\in Q_M$.
If
\begin{itemize}
\item $P_{s_M}(\%)$ and

\item for all $q,r\in Q_M$, $x\in\Str$ and $w\in\Lambda_{M,q}$,\\
if $q, x\fun r\in T_M$ and (\dag) $P_q(w)$, then  $P_r(wx)$,
\end{itemize}
then
\begin{gather*}
\eqtxtr{for all}q\in Q_M,\eqtxt{for all}w\in\Lambda_{M,q}, P_q(w) .
\end{gather*}
\end{theorem}

We refer to (\dag) as the inductive hypothesis.

\begin{proof}
It suffices to show that, for all $\lp\in\LP$, for all $q\in Q_M$,
if $\lp$ is valid for $M$, $\startState\,\lp = s_M$ and
$\myendState\,\lp = q$, then $P_q(\mylabel\,\lp)$.  We prove this by
well-founded induction on the length of $\lp$.
\end{proof}

In the case of our example FA, $M$, we can let $P_\Asf(w)$ and $P_\Bsf(w)$ be
$w\in X$ and $w\in Y$, respectively, where, as before,
\begin{align*}
X &= \setof{w\in\{\mathsf{0,1}\}^*}{w\eqtxt{has an even number of}
\zerosf\eqtxtn{'s}} , \eqtxt{and}\\
Y &= \setof{w\in\{\mathsf{0,1}\}^*}{w\eqtxt{has an odd number of}
\zerosf\eqtxtn{'s}} .
\end{align*}

Then the principle of induction on $\Lambda$ tells us that
\begin{enumerate}[\quad(A)]
\item for all $w\in\Lambda_\Asf$, $w\in X$, and

\item for all $w\in\Lambda_\Bsf$, $w\in Y$,
\end{enumerate}
follows from showing
\begin{description}
\item[\quad(empty string)] $\%\in X$;

\item[\quad($\Asf, \zerosf\fun\Bsf$)] for all $w\in\Lambda_\Asf$, if
  (\dag) $w\in X$, then $w\zerosf\in Y$;

\item[\quad($\Asf, \onesf\fun\Asf$)] for all $w\in\Lambda_\Asf$, if
  (\dag) $w\in X$, then $w\onesf\in X$;

\item[\quad($\Bsf, \zerosf\fun\Asf$)] for all $w\in\Lambda_\Bsf$, if
  (\dag) $w\in Y$, then $w\zerosf\in X$; and

\item[\quad($\Bsf, \onesf\fun\Bsf$)] for all $w\in\Lambda_\Bsf$, if
  (\dag) $w\in Y$, then $w\onesf\in Y$.
\end{description}
We refer to (\dag) as the inductive hypothesis.

In fact, when setting this proof up, instead of explicitly mentioning
$P_\Asf$ and $P_\Bsf$, we can simply say that we are proving
\begin{enumerate}[\quad(A)]
\item for all $w\in\Lambda_\Asf$, $w\in X$, and

\item for all $w\in\Lambda_\Bsf$, $w\in Y$,
\end{enumerate}
by induction on $\Lambda$.

There are five steps to show.
\begin{description}
\item[\quad(empty string)] Because $\%\in\{\mathsf{0,1}\}^*$ and $\%$
has no $\zerosf$'s, we have that $\%\in X$.

\item[\quad($\Asf, \zerosf\fun\Bsf$)] Suppose $w\in\Lambda_\Asf$, and
  assume the inductive hypothesis: $w\in X$.  Hence
  $w\in\{\mathsf{0,1}\}^*$ and $w$ has an even number of $\zerosf$'s.
  Thus $w\zerosf\in\{\mathsf{0,1}\}^*$ and $w\zerosf$ has an odd
  number of $\zerosf$'s, so that $w\zerosf\in Y$.
\end{description}

\begin{description}
\item[\quad($\Asf, \onesf\fun\Asf$)] Suppose $w\in\Lambda_\Asf$, and
  assume the inductive hypothesis: $w\in X$.  Then $w\onesf\in X$.

\item[\quad($\Bsf, \zerosf\fun\Asf$)] Suppose $w\in\Lambda_\Bsf$, and
  assume the inductive hypothesis: $w\in Y$.  Then $w\zerosf\in X$.

\item[\quad($\Bsf, \onesf\fun\Bsf$)] Suppose $w\in\Lambda_\Bsf$, and
  assume the inductive hypothesis: $w\in Y$.  Then $w\onesf\in Y$.
\end{description}

Because of
\begin{enumerate}[\quad(A)]
\item for all $w\in\Lambda_\Asf$, $w\in X$, and

\item for all $w\in\Lambda_\Bsf$, $w\in Y$,
\end{enumerate}
we have that
$\Lambda_\Asf \sub X$ and $\Lambda_\Bsf \sub Y$.

Because $X \sub \Lambda_\Asf$ and $Y \sub \Lambda_\Bsf$, we can
conclude that $L(M) = \Lambda_\Asf = X$ and $\Lambda_\Bsf = Y$.

Consider our second example, $N$, again:
\begin{center}
\input{chap-3.8-fig2.eepic}
\end{center}

We can use induction on $\Lambda$ to prove that
\begin{enumerate}[\quad(A)]
\item for all $w\in\Lambda_\Asf$, $w\in\{\zerosf\}^*$; and

\item for all $w\in\Lambda_\Bsf$, $w\in\{\zerosf\}^*\{\mathsf{11}\}^*$.
\end{enumerate}
Thus $\Lambda_\Asf\sub\{\zerosf\}^*$ and
$\Lambda_\Bsf\sub\{\zerosf\}^*\{\mathsf{11}\}^*$.  Because
$\{\zerosf\}^*\sub\Lambda_\Asf$ and
$\{\zerosf\}^*\{\mathsf{11}\}^*\sub\Lambda_\Bsf$, we can conclude that
$\Lambda_\Asf=\{\zerosf\}^*$ and
$L(N)=\Lambda_\Bsf=\{\zerosf\}^*\{\mathsf{11}\}^*$.

\subsection{Notes}

Books on formal language theory typically give short shrift to the
proof of correctness of finite automata, carrying out one or two
correctness proofs using induction on the length of strings.  In
contrast, we have introduced and applied elegant techniques for
proving the correctness of FAs.  Of particular note is our principle
of induction on $\Lambda$.

\index{finite automaton!proof of correctness|)}

%%% Local Variables: 
%%% mode: latex
%%% TeX-master: "book"
%%% End: 
