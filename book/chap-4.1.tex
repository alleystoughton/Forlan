\section{Grammars, Parse Trees and Context-free Languages}
\label{GrammarsParseTreesAndContextFreeLanguages}

\index{grammars|(}

In this section, we: say what (context-free) grammars are; use the
notion of a parse tree to say what grammars mean; say what it means
for a language to be context-free; and begin to show how grammars can
be processed using Forlan.

\subsection{Grammars}

A \emph{context-free grammar} (or just \emph{grammar}) $G$ consists of:
\begin{itemize}
\item a finite set $Q_G$ of symbols (we call the elements of $Q_G$
the \emph{variables} of $G$);

\item an element $s_G$ of $Q_G$ (we call $s_G$ the \emph{start variable}
of $G$); and

\item a finite subset $P_G$ of $\setof{(q,x)}{q\in Q_G\eqtxt{and}
x\in\Str}$ (we call the elements of $P_G$ the \emph{productions} of
$G$, and we often write $(q, x)$ as $q\fun x$).
\end{itemize}
\index{grammar!variables}%
\index{grammar!start variable}%
\index{grammar!productions}%

In a context where we are only referring to a single grammar, $G$, we
sometimes abbreviate $Q_G$, $s_G$ and $P_G$ to $Q$, $s$ and $P$,
respectively.  Whenever possible, we will use the mathematical
variables $p$, $q$ and $r$ to name variables.  We write $\Gram$ for
\index{grammar!Gram@$\Gram$}%
\index{Gram@$\Gram$}%
the set of all grammars.  Since every grammar can be described by a
finite sequence of ASCII characters, we have that $\Gram$ is countably
infinite.

We give two forms of productions special names:
\begin{itemize}
\item An $\%$-\emph{production} is a production of the form $q\fun\%$.
  \index{grammar! perc@$\%$-production}%

\item A \emph{unit production} for a grammar $G$ is a production of
  the form $q\fun r$, where $r$ is a variable (possibly equal to $q$).
  \index{grammar!unit production}%
\end{itemize}

As an example, we can define a grammar $G$ (of arithmetic expressions) as
follows:
\begin{itemize}
\item $Q_G=\{\mathsf{E}\}$;

\item $s_G=\Esf$; and

\item $P_G=\{\mathsf{E\fun E\plussym E,\;
E\fun E\timessym E,\;
E\fun\openparsym E\closparsym,\;
E\fun\idsym}\}$.
\end{itemize}
E.g., we can read the production $\mathsf{E}\fun\mathsf{E}\plussym\mathsf{E}$
as ``an expression can consist of an expression, followed by
a $\plussym$ symbol, followed by an expression''.

We typically describe a grammar by listing its productions, and
grouping productions with identical left-sides into production
families.  Unless we say otherwise, the grammar's variables are the
left-sides of all of its productions, and its start variable is the
left-side of its first production.
Thus, our grammar $G$ is
\begin{align*}
\Esf &\fun \Esf\plussym\Esf , \\
\Esf &\fun \Esf\timessym\Esf , \\
\Esf &\fun \openparsym\Esf\closparsym , \\
\Esf &\fun \idsym ,
\end{align*}
or
\begin{gather*}
\mathsf{E\fun E\plussym E\mid E\timessym E\mid \openparsym E\closparsym \mid
\idsym} .
\end{gather*}

\index{grammar!Forlan syntax}%
The Forlan syntax for grammars is very similar to the notation used
above.  E.g., our example grammar can be described in Forlan's syntax
by
\begin{verbatim}
{variables} E {start variable} E
{productions}
E -> E<plus>E; E -> E<times>E; E -> <openPar>E<closPar>; E -> <id>
\end{verbatim}
or
\begin{verbatim}
{variables} E {start variable} E
{productions}
E -> E<plus>E | E<times>E | <openPar>E<closPar> | <id>
\end{verbatim}
Production families are separated by semicolons.

The Forlan module \texttt{Gram} defines an abstract type \texttt{gram} (in
\index{Gram@\texttt{Gram}}%
\index{Gram@\texttt{Gram}!gram@\texttt{gram}}%
the top-level environment) of grammars as well as a number of
functions and constants for processing grammars, including:
\begin{verbatim}
val input          : string -> gram
val output         : string * gram -> unit 
val numVariables   : gram -> int
val numProductions : gram -> int
val equal          : gram * gram -> bool
\end{verbatim}
\index{Gram@\texttt{Gram}!input@\texttt{input}}%
\index{Gram@\texttt{Gram}!output@\texttt{output}}%
\index{Gram@\texttt{Gram}!numVariables@\texttt{numVariables}}%
\index{Gram@\texttt{Gram}!numProductions@\texttt{numProductions}}%
\index{Gram@\texttt{Gram}!equal@\texttt{equal}}%
The functions \texttt{numVariables} and \texttt{numProductions} return
the numbers of variables and productions, respectively, of a grammar.
And the function \texttt{equal} tests whether two grammars are equal,
i.e., whether they have the same sets of variables, start variables,
and sets of productions.
During printing, Forlan merges productions into production families
whenever possible.

\index{grammar!alphabet}%
\index{grammar!alphabet@$\alphabet$}%
We define a function $\alphabet\in\Gram\fun\Alp$ by: for all $G\in\Gram$,
$\alphabet\,G$ is
\begin{align*}
&\quad\; \{\,a\in\Sym\mid\eqtxtr{there are}q,x\eqtxt{such that}q\fun x\in P_G
\eqtxt{and} \\
&\hspace{2.6cm} a\in\alphabet\,x \,\} \\
&- Q_G .
\end{align*}
I.e., $\alphabet\,G$ is all of the symbols appearing in the strings of
$G$'s productions that aren't variables.  For example, the alphabet of
our example grammar $G$ is
$\{\mathsf{\plussym,\timessym,\openparsym,\closparsym,\idsym}\}$.

The Forlan module \texttt{Gram} defines a function
\begin{verbatim}
val alphabet : gram -> sym set
\end{verbatim}
\index{Gram@\texttt{Gram}!alphabet@\texttt{alphabet}}%
for calculating the alphabet of a grammar.
E.g., if \texttt{gram} of type \texttt{gram} is bound to our example
grammar $G$, then Forlan will behave as follows:
\begin{list}{}
{\setlength{\leftmargin}{\leftmargini}
\setlength{\rightmargin}{0cm}
\setlength{\itemindent}{0cm}
\setlength{\listparindent}{0cm}
\setlength{\itemsep}{0cm}
\setlength{\parsep}{0cm}
\setlength{\labelsep}{0cm}
\setlength{\labelwidth}{0cm}
\catcode`\#=12
\catcode`\$=12
\catcode`\%=12
\catcode`\^=12
\catcode`\_=12
\catcode`\.=12
\catcode`\?=12
\catcode`\!=12
\catcode`\&=12
\ttfamily}
\small
\item[]\textsl{-\ }val\ bs\ =\ Gram.alphabet\ gram;
\item[]\textsl{val\ bs\ =\ -\ :\ sym\ set}
\item[]\textsl{-\ }SymSet.output("",\ bs);\ \ 
\item[]\textsl{<id>,\ <plus>,\ <times>,\ <closPar>,\ <openPar>}
\item[]\textsl{val\ it\ =\ ()\ :\ unit}
\end{list}


\subsection{Parse Trees and Grammar Meaning}

\index{grammar!parse tree|see{parse tree}}%
\index{parse tree}%
\index{parse!PT@$\PT$}%
We will explain when strings are generated by grammars using the
notion of a parse tree.  The set $\PT$ of \emph{parse trees} is the
least subset of $\Tree(\Sym\cup\{\%\})$ (the set of all
$(\Sym\cup\{\%\})$-trees; see
Section~\ref{TreesAndInductiveDefinitions}) such that:
\begin{enumerate}[\quad(1)]
\item for all $a\in\Sym$ and $\pts\in\List\,\PT$, $(a,\pts)\in\PT$; and

\item for all $a\in\Sym$, $(a,[(\%,[\,])]) = a(\%)\in\PT$.
\end{enumerate}
Note that the $(\Sym\cup\{\%\})$-tree $\% = (\%,[\,])$ is
\emph{not} a parse tree.
It is easy to see that $\PT$ is countably infinite.

For example, $\Asf(\Bsf,\Asf(\%),\Bsf(\zerosf))$, i.e.,
\begin{center}
\input{chap-4.1-fig1.eepic}
\end{center}
is a parse tree.  On the other hand,
although $\Asf(\Bsf,\%,\Bsf)$, i.e.,
\begin{center}
\input{chap-4.1-fig2.eepic}
\end{center}
is a $(\Sym\cup\{\%\})$-tree, it's not a parse tree, since it
can't be formed using rules (1) and (2).

Since the set $\PT$ of parse trees is defined inductively, it gives
rise to an induction principle.
\index{parse tree!induction principle}%

\begin{theorem}[Principle of Induction on Parse Trees]
Suppose $P(\pt)$ is a property of an element $\pt\in\PT$.

If
\begin{enumerate}[\quad(1)]
\item for all $a\in\Sym$ and $\trs\in\List\,\PT$, if $(\dag)$ for all
  $i\in[1:|\trs|]$, $P(\trs\,i)$, then $P((a,\trs))$, and

\item for all $a\in\Sym$, $P(a(\%))$,
\end{enumerate}
then
\begin{gather*}
\eqtxtr{for all}\pt\in\PT,P(\pt) .
\end{gather*}
\end{theorem}
We refer to $(\dag)$ as the inductive hypothesis.

\index{parse tree!yield}%
\index{parse tree!yield@$\yield$}%
We define the yield of a parse tree, as follows.  The function
$\yield\in\PT\fun\Str$ is defined by structural recursion:
\begin{itemize}
\item for all $a\in\Sym$, $\yield\,a=a$;

\item for all $q\in\Sym$, $n\in\nats-\{0\}$ and
$\pt_1,\,\ldots,\pt_n\in\PT$,
$\yield(q(\pt_1,\,\ldots,\pt_n)) =
\yield\,\pt_1\,\cdots\,\yield\,\pt_n$; and

\item for all $q\in\Sym$, $\yield(q(\%))=\%$.
\end{itemize}
We say that $w$ is the \emph{yield of} $\pt$ iff
$w=\yield\,\pt$.

For example, the yield of
\begin{center}
\input{chap-4.1-fig1.eepic}
\end{center}
is
\begin{gather*}
\yield\,\Bsf\,\yield(A(\%))\,\yield(\Bsf(\zerosf)) =
\Bsf\,\%\,\yield\,\zerosf=\Bsf\%\zerosf=\Bsf\zerosf .
\end{gather*}

\index{parse tree!valid}%
\index{parse tree!valid@$\valid_{\cdot}$}%
We say when a parse tree is valid for a grammar $G$ as follows.
Define a function $\valid_G\in\PT\fun\Bool$ by
structural recursion:
\begin{itemize}
\item for all $a\in\Sym$, $\valid_G\,a=a\in
\alphabet\,G\myor a\in Q_G$;

\item for all $q\in\Sym$, $n\in\nats-\{0\}$ and
$\pt_1,\,\ldots,\pt_n\in\PT$,
\begin{align*}
&\quad\; \valid_G(q(\pt_1,\,\ldots,\pt_n)) \\
&= q\fun\rootLabel\,\pt_1\,\cdots\,\rootLabel\,\pt_n\in P_G \myand{} \\
&\quad\;
\valid_G\,\pt_1 \myand\, \cdots\, \myand \valid_G\,\pt_n; \eqtxt{and}
\end{align*}

\item for all $q\in\Sym$, $\valid_G(q(\%))=q\fun\%\in
P_G$.
\end{itemize}
We say that $\pt$ is \emph{valid for} $G$ iff $\valid_G\,\pt=\true$.
We often abbreviate $\valid_G$ to $\valid$.

Suppose $G$ is the grammar
\begin{align*}
\Asf &\fun \Bsf\Asf\Bsf \mid \% , \\
\Bsf &\fun \zerosf 
\end{align*}
(by convention, its variables are $\Asf$ and $\Bsf$ and its
start variable is $\Asf$).
Let's see why the parse tree $\Asf(\Bsf,\Asf(\%),\Bsf(\zerosf))$ is
valid for $G$:
\begin{itemize}
\item Since $\Asf\fun\Bsf\Asf\Bsf\in P_G$ and the concatenation
of the root labels of the sub-trees
$\Bsf$, $\Asf(\%)$ and $\Bsf(\zerosf)$ is $\Bsf\Asf\Bsf$,
the overall tree will be valid for $G$ if these sub-trees are valid
for $G$.

\item The parse tree $\Bsf$ is valid for $G$
since $\Bsf\in Q_G$.

\item Since $\Asf\fun\%\in P_G$, the parse tree
$\Asf(\%)$ is valid for $G$.

\item Since $\Bsf\fun\zerosf\in P_G$ and the root label of the sub-tree
$\zerosf$ is $\zerosf$, the parse tree
$\Bsf(\zerosf)$ will be valid for $G$ if the sub-tree $\zerosf$
is valid for $G$.

\item The sub-tree $\zerosf$ is valid for $G$ since
  $\zerosf\in\alphabet\,G$.
\end{itemize}
Thus, we have that
\begin{center}
\input{chap-4.1-fig1.eepic}
\end{center}
is valid for $G$.

And, if $G$ is our grammar of arithmetic expressions,
\begin{gather*}
\mathsf{E\fun E\plussym E\mid E\timessym E\mid \openparsym E\closparsym \mid
\idsym} ,
\end{gather*}
then the parse tree
\begin{center}
\input{chap-4.1-fig3.eepic}
\end{center}
is valid for $G$.

\begin{proposition}
Suppose $G$ is a grammar. For all parse trees $\pt$, if $\pt$ is valid
for $G$, then $\rootLabel\,\pt\in Q_G\cup\alphabet\,G$ and
$\yield\,\pt \in (Q_G\cup\alphabet\,G)^*$.
\end{proposition}

\begin{proof}
By induction on $\pt$.
\end{proof}

\index{grammar!parsable from}%
Suppose $G$ is a grammar, $w\in\Str$ and $a\in\Sym$.  We say that $w$
\emph{is parsable from} $a$ \emph{using} $G$ iff there is a parse tree
$\pt$ such that:
\begin{itemize}
\item $\pt$ is valid for $G$;

\item $a$ is the root label of $\pt$; and

\item the yield of $\pt$ is $w$.
\end{itemize}
Thus we will have that $w\in(Q_G\cup\alphabet\,G)^*$, and either
$a\in Q_G$ or $[a] = w$.
\index{grammar!generated from}%
We say that a string $w$ \emph{is generated from} a variable $q\in Q_G$
\emph{using} $G$ iff $w\in(\alphabet\,G)^*$ and $w$ is parsable from
$q$.
\index{grammar!meaning}%
\index{grammar!language generated by}%
\index{L(@$L(\cdot)$}%
\index{grammar!L(@$L(\cdot)$}%}%
And, we say that a string $w$ \emph{is generated by} a grammar $G$ iff
$w$ is generated from $s_G$ using $G$.
The \emph{language generated by} a grammar $G$ ($L(G)$) is
\begin{gather*}
\setof{w\in\Str}{w\eqtxt{is generated by}G}.
\end{gather*}

\begin{proposition}
For all grammars $G$, $\alphabet(L(G))\sub\alphabet\,G$.
\end{proposition}

For example, if $G$ is the grammar
\begin{align*}
\Asf &\fun \Bsf\Asf\Bsf \mid \% , \\
\Bsf &\fun \zerosf ,
\end{align*}
then $\zerosf\zerosf$ is generated by $G$,
since $\zerosf\zerosf\in\{\zerosf\}^*=(\alphabet\,G)^*$ and
the parse tree
\begin{center}
\input{chap-4.1-fig4.eepic}
\end{center}
is valid for $G$, has $s_G=\Asf$ as its root label,
and has $\zerosf\zerosf$ as its yield.
And, if $G$ is our grammar of arithmetic expressions,
\begin{gather*}
\mathsf{E\fun E\plussym E\mid E\timessym E\mid \openparsym E\closparsym \mid
\idsym} ,
\end{gather*}
then $\idsym\timessym\idsym\plussym\idsym$ is generated by $G$,
since $\idsym\timessym\idsym\plussym\idsym\in(\alphabet\,G)^*$ and
the parse tree
\begin{center}
\input{chap-4.1-fig3.eepic}
\end{center}
is valid for $G$, has $s_G=\Esf$ as its root label,
and has $\idsym\timessym\idsym\plussym\idsym$ as its yield.

\index{language!context-free|see{context-free language}}%
\index{context-free language}%
A language $L$ is \emph{context-free} iff $L=L(G)$ for some
$G\in\Gram$.  We define
\index{context-free language:CFLan@$\CFLan$}%
\index{language:CFLan@$\CFLan$}%
\begin{align*}
\CFLan &= \setof{L(G)}{G\in\Gram} \\
&= \setof{L\in\Lan}{L\eqtxtl{is context-free}} .
\end{align*}
Since $\{\mathsf{0}^0\}$, $\{\mathsf{0}^1\}$, $\{\mathsf{0}^2\}$,
\ldots, are all context-free languages, we have that $\CFLan$ is
infinite.  But, since $\Gram$ is countably infinite, it follows that
$\CFLan$ is also countably infinite.
Since $\Lan$ is uncountable, it follows that
$\CFLan\subsetneq\Lan$, i.e., there are non-context-free
languages.  Later, we will see that $\RegLan\subsetneq\CFLan$.

\index{grammar!equivalence|(}%
\index{ equiv@$\approx$}%
\index{grammar! equiv@$\approx$}%
We say that grammars $G$ and $H$ are
\emph{equivalent} iff $L(G) = L(H)$.  In other words, $G$
and $H$ are equivalent iff $G$ and $H$ generate the same
language.  We define a relation $\approx$ on $\Gram$ by:
$G\approx H$ iff $G$ and $H$ are equivalent.  It is easy to see
that $\approx$ is reflexive on $\Gram$, symmetric and transitive.

The Forlan module \texttt{PT} defines an abstract type \texttt{pt} of
\index{PT@\texttt{PT}}%
parse trees (in the top-level environment) along with some functions
for processing parse trees:
\begin{verbatim}
val input     : string -> pt
val output    : string * pt -> unit
val height    : pt -> int
val size      : pt -> int
val equal     : pt * pt -> bool
val rootLabel : pt -> sym
val yield     : pt -> str
\end{verbatim}
\index{PT@\texttt{PT}!input@\texttt{input}}%
\index{PT@\texttt{PT}!output@\texttt{output}}%
\index{PT@\texttt{PT}!height@\texttt{height}}%
\index{PT@\texttt{PT}!size@\texttt{size}}%
\index{PT@\texttt{PT}!equal@\texttt{equal}}%
\index{PT@\texttt{PT}!rootLabel@\texttt{rootLabel}}%
\index{PT@\texttt{PT}!yield@\texttt{yield}}%
The Forlan syntax for parse trees is simply the linear syntax that
we've been using in this section.

\index{JForlan}%
The Java program JForlan, can be used to view and edit parse
trees.  It can be invoked directly, or run via
Forlan.  See the Forlan website for more information.

The Forlan module \texttt{Gram} also defines the functions
\begin{verbatim}
val checkPT : gram -> pt -> unit
val validPT : gram -> pt -> bool
\end{verbatim}
\index{Gram@\texttt{Gram}!checkPT@\texttt{checkPT}}%
\index{Gram@\texttt{Gram}!validPT@\texttt{validPT}}%
The function \texttt{checkPT} is used to check whether a parse tree
is valid for a grammar; if the answer is ``no'', it explains why not
and raises an exception; otherwise it simply returns \texttt{()}.
The function \texttt{validPT} checks whether a parse tree is valid
for a grammar, silently returning \texttt{true} if it is, and silently
returning \texttt{false} if it isn't.

Suppose the identifier \texttt{gram} of type \texttt{gram} is bound to the
grammar
\begin{align*}
\Asf &\fun \Bsf\Asf\Bsf \mid \% , \\
\Bsf &\fun \zerosf .
\end{align*}
And, suppose that the identifier
\texttt{gram'} of type \texttt{gram} is bound to our grammar of
arithmetic expressions,
\begin{gather*}
\mathsf{E\fun E\plussym E\mid E\timessym E\mid \openparsym E\closparsym \mid
\idsym} .
\end{gather*}
Here are some examples of how we can process parse trees using Forlan:
\begin{list}{}
{\setlength{\leftmargin}{\leftmargini}
\setlength{\rightmargin}{0cm}
\setlength{\itemindent}{0cm}
\setlength{\listparindent}{0cm}
\setlength{\itemsep}{0cm}
\setlength{\parsep}{0cm}
\setlength{\labelsep}{0cm}
\setlength{\labelwidth}{0cm}
\catcode`\#=12
\catcode`\$=12
\catcode`\%=12
\catcode`\^=12
\catcode`\_=12
\catcode`\.=12
\catcode`\?=12
\catcode`\!=12
\catcode`\&=12
\ttfamily}
\small
\item[]\textsl{-\ }val\ pt\ =\ PT.input\ "";
\item[]\textsl{@\ }A(B,\ A(%),\ B(0))
\item[]\textsl{@\ }.
\item[]\textsl{val\ pt\ =\ -\ :\ pt}
\item[]\textsl{-\ }Sym.output("",\ PT.rootLabel\ pt);
\item[]\textsl{A}
\item[]\textsl{val\ it\ =\ ()\ :\ unit}
\item[]\textsl{-\ }Str.output("",\ PT.yield\ pt);
\item[]\textsl{B0}
\item[]\textsl{val\ it\ =\ ()\ :\ unit}
\item[]\textsl{-\ }Gram.validPT\ gram\ pt;
\item[]\textsl{val\ it\ =\ true\ :\ bool}
\item[]\textsl{-\ }val\ pt'\ =\ PT.input\ "";
\item[]\textsl{@\ }E(E(E(<id>),\ <times>,\ E(<id>)),\ <plus>,\ E(<id>))
\item[]\textsl{@\ }.
\item[]\textsl{val\ pt'\ =\ -\ :\ pt}
\item[]\textsl{-\ }Sym.output("",\ PT.rootLabel\ pt');
\item[]\textsl{E}
\item[]\textsl{val\ it\ =\ ()\ :\ unit}
\item[]\textsl{-\ }Str.output("",\ PT.yield\ pt');
\item[]\textsl{<id><times><id><plus><id>}
\item[]\textsl{val\ it\ =\ ()\ :\ unit}
\item[]\textsl{-\ }Gram.validPT\ gram'\ pt';
\item[]\textsl{val\ it\ =\ true\ :\ bool}
\item[]\textsl{-\ }Gram.checkPT\ gram\ pt';
\item[]\textsl{invalid\ production:\ "E\ ->\ E<plus>E"}
\item[]
\item[]\textsl{uncaught\ exception\ Error}
\item[]\textsl{-\ }Gram.checkPT\ gram'\ pt;
\item[]\textsl{invalid\ production:\ "A\ ->\ BAB"}
\item[]
\item[]\textsl{uncaught\ exception\ Error}
\item[]\textsl{-\ }PT.input\ "";
\item[]\textsl{@\ }A(B,%,B)
\item[]\textsl{@\ }.
\item[]\textsl{line\ 1:\ "%"\ unexpected}
\item[]
\item[]\textsl{uncaught\ exception\ Error}
\end{list}


The Forlan module \texttt{PT} also defines these functions
for processing parse trees:
\begin{verbatim}
val selectPT : pt * int list -> pt option
val update   : pt * int list * pt -> pt
val cons     : sym * pt list option -> pt
val leaf     : sym -> pt
val decons   : pt -> sym * pt list option
\end{verbatim}
\index{PT@\texttt{PT}!selectPT@\texttt{selectPT}}%
\index{PT@\texttt{PT}!update@\texttt{update}}%
\index{PT@\texttt{PT}!cons@\texttt{cons}}%
\index{PT@\texttt{PT}!leaf@\texttt{leaf}}%
\index{PT@\texttt{PT}!decons@\texttt{decons}}%
The function \texttt{selectPT} selects the parse tree at
a given position (path), returning \texttt{SOME} of that tree,
if the path is valid, and returning \texttt{NONE} otherwise
(\texttt{NONE} is returned if the path takes us to a leaf 
labeled $\%$ in the $(\Sym\cup\{\%\})$-tree). The
function \texttt{update} replaces the subtree at a given position
of its first argument by its third argument, issuing an error
message if the supplied path is invalid (we are allowed to replace
a leaf labeled $\%$ by a parse tree). The function \texttt{cons}
builds a parse tree whose root is labeled by the supplied symbol, $a$.
If the second argument is \texttt{NONE}, the parse tree will be $a(\%)$.
If the second argument is $\texttt{SOME}\;\pts$, then $\pts$ will be
its children. $\texttt{leaf}\,a$ is an abbreviation for
$\texttt{cons(}a\texttt{, SOME [])}$. Finally, \texttt{decons}
deconstructs a parse tree into the label $a$ of its root node, paired
with either \texttt{NONE}, if the tree as $a(\%)$, or \texttt{SOME}
of its children.
Here are some examples of how we can use these functions:
\begin{list}{}
{\setlength{\leftmargin}{\leftmargini}
\setlength{\rightmargin}{0cm}
\setlength{\itemindent}{0cm}
\setlength{\listparindent}{0cm}
\setlength{\itemsep}{0cm}
\setlength{\parsep}{0cm}
\setlength{\labelsep}{0cm}
\setlength{\labelwidth}{0cm}
\catcode`\#=12
\catcode`\$=12
\catcode`\%=12
\catcode`\^=12
\catcode`\_=12
\catcode`\.=12
\catcode`\?=12
\catcode`\!=12
\catcode`\&=12
\ttfamily}
\small
\item[]\textsl{-\ }val\ pt\ =\ PT.cons(Sym.fromString\ "A",\ NONE);
\item[]\textsl{val\ pt\ =\ -\ :\ pt}
\item[]\textsl{-\ }PT.output("",\ pt);
\item[]\textsl{A(%)}
\item[]\textsl{val\ it\ =\ ()\ :\ unit}
\item[]\textsl{-\ }val\ pt'\ =\ PT.update(pt,\ \symbol{'133}1\symbol{'135},\ PT.leaf(Sym.fromString\ "B"));
\item[]\textsl{val\ pt'\ =\ -\ :\ pt}
\item[]\textsl{-\ }PT.output("",\ pt');
\item[]\textsl{A(B)}
\item[]\textsl{val\ it\ =\ ()\ :\ unit}
\item[]\textsl{-\ }val\ pt''\ =
\item[]\textsl{=\ }\ \ PT.update
\item[]\textsl{=\ }\ \ (pt',\ \symbol{'133}1\symbol{'135},
\item[]\textsl{=\ }\ \ \ PT.cons(Sym.fromString\ "C",\ SOME\symbol{'133}pt,\ pt\symbol{'135}));
\item[]\textsl{val\ pt''\ =\ -\ :\ pt}
\item[]\textsl{-\ }PT.output("",\ pt'');
\item[]\textsl{A(C(A(%),\ A(%)))}
\item[]\textsl{val\ it\ =\ ()\ :\ unit}
\item[]\textsl{-\ }val\ pt'''\ =\ valOf(PT.selectPT(pt'',\ \symbol{'133}1\symbol{'135}));
\item[]\textsl{val\ pt'''\ =\ -\ :\ pt}
\item[]\textsl{-\ }PT.output("",\ pt''');
\item[]\textsl{C(A(%),\ A(%))}
\item[]\textsl{val\ it\ =\ ()\ :\ unit}
\end{list}


\subsection{Grammar Synthesis}

\label{grammar!synthesis}
We conclude this section with a grammar synthesis example.  Suppose
$X=\setof{\zerosf^n\onesf^m\twosf^m\threesf^n}{n,m\in\nats}$.  The key
to finding a grammar $G$ that generates $X$ is to think of generating
the strings of $X$ from the outside in, in two phases.  In the first
phase, one generates pairs of $\zerosf$'s and $\threesf$'s, and, in
the second phase, one generates pairs of $\onesf$'s and $\twosf$'s.
E.g., a string could be formed in the following stages:
\begin{gather*}
{\mathsf{0\hspace{.6cm}3,}} \\
{\mathsf{00\hspace{.34cm}33,}} \\
{\mathsf{001233.}}
\end{gather*}

This analysis leads us to the grammar
\begin{align*}
\Asf &\fun \zerosf\Asf\threesf , \\
\Asf &\fun \Bsf , \\
\Bsf &\fun \onesf\Bsf\twosf , \\
\Bsf &\fun \% ,
\end{align*}
where $\Asf$ corresponds to the first phase, and $\Bsf$ to the
second phase.
For example, here is how the string $\mathsf{001233}$ may be
parsed using $G$:
\begin{center}
\input{chap-4.1-fig5.eepic}
\end{center}

\begin{exercise}
Let
\begin{displaymath}
X = \setof{\zerosf^i\onesf^j\twosf^k}{i,j,k\in\nats\eqtxt{and}
(i\neq j\eqtxt{or}j\neq k)} .
\end{displaymath}
Find a grammar $G$ such that $L(G)=X$.
\end{exercise}

\subsection{Notes}

Traditionally, the meaning of grammars is defined using derivations,
with parse trees being introduced subsequently.  In contrast, we have
no need for derivations, and find parse trees a much more intuitive
way to define the meaning of grammars.

Pleasingly, parse trees are an instance of the trees introduced in
Section~\ref{TreesAndInductiveDefinitions}.  Thus the terminology,
techniques and results of that section are applicable to them.

\index{grammars|)}

%%% Local Variables: 
%%% mode: latex
%%% TeX-master: "book"
%%% End: 
