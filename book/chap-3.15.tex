\section{Applications of Finite Automata and
Regular Expressions}
\label{ApplicationsOfFiniteAutomataAndRegularExpressions}

In this section we consider three applications of the material from
Chapter 3: searching for regular expressions in files; lexical
analysis; and the design of finite state systems.

\subsection{Representing Character Sets and Files}

Our first two applications involve processing files whose characters
come from some character set, e.g., the ASCII character set.  Although
not every character in a typical character set will be an element of
our set $\Sym$ of symbols, we can \emph{represent} all the characters
of a character set by elements of $\Sym$.  E.g., we might represent
the ASCII characters newline and space by the symbols $\newlinesym$
and $\spacesym$, respectively.

In the following two subsections, we will work with a mostly
unspecified alphabet $\Sigma$ representing some character set.  We
assume that the symbols $\mathsf{0}$--$\mathsf{9}$,
$\mathsf{a}$--$\mathsf{z}$, $\mathsf{A}$--$\mathsf{Z}$, $\spacesym$
and $\newlinesym$ are elements of $\Sigma$.  A \emph{line} is a string
consisting of an element of $(\Sigma-\{\newlinesym\})^*$; and, a
\emph{file} consists of the concatenation of some number of lines,
separated by occurrences of $\newlinesym$.  E.g.,
$\mathsf{0a\newlinesym\newlinesym 6}$ is a file with three lines
($\mathsf{0a}$, $\mathsf{\%}$ and $\mathsf{6}$), and
$\mathsf{\newlinesym}$ is a file with two lines, both consisting of
$\%$.

In what follows, we write:
\begin{itemize}
\item $\any$ for the regular expression $a_1+a_2+\cdots+a_n$, where
  $a_1,a_2,\,\ldots,a_n$ are all of the elements of $\Sigma$ except
  $\newlinesym$, listed in strictly ascending order;

\item $\letter$ for the regular expression
  \begin{gather*}
    \mathsf{a + b + \cdots + z + A + B + \cdots + Z};
  \end{gather*}

\item $\digit$ for the regular expression
  \begin{gather*}
    \mathsf{0 + 1 + \cdots + 9}.
  \end{gather*}
\end{itemize}

\subsection{Searching for Regular Expression in Files}

Given a file and a regular expression $\alpha$ whose alphabet is a
subset of $\Sigma-\{\newlinesym\}$, how can we find all lines of the
file with substrings in $L(\alpha)$?  (E.g., $\alpha$ might be
$\mathsf{\asf(\bsf+\csf)^*\asf}$; then we want to find all lines
containing two $\asf$'s, separated by some number of $\bsf$'s and
$\csf$'s.)

It will be sufficient to find all lines in the file that are elements
of $L(\beta)$, where $\beta = \any^*\,\alpha\,\any^*$.  To do this, we
can first translate $\beta$ to a DFA $M$ with alphabet
$\Sigma-\{\newlinesym\}$.  For each line $w$, we simply check whether
$\delta_M(s_M,w)\in A_M$, selecting the line if it is.  If the file is
short, however, it may be more efficient to convert $\beta$ to an FA
(or EFA or NFA) $N$, and use the algorithm from
Section~\ref{CheckingAcceptanceAndFindingAcceptingPaths} to find all
lines that are accepted by $N$.

\subsection{Lexical Analysis}

A lexical analyzer is the part of a compiler that groups the
characters of a program into lexical items or tokens.  The modern
approach to specifying a lexical analyzer for a programming language
uses regular expressions.  E.g., this is the approach taken by the
lexical analyzer generator Lex.

A lexical analyzer specification consists of a list of regular
expressions $\alpha_1,\alpha_2,\,\ldots,\alpha_n$, together with a
corresponding list of code fragments (in some programming language)
$\code_1,\code_2,\,\ldots,\code_n$ that process elements of
$\Sigma^*$.

For example, we might have
\begin{align*}
\alpha_1 &= \spacesym + \newlinesym , \\
\alpha_2 &= \letter\,(\letter + \digit)^* ,\\
\alpha_3 &= \digit\,\digit^*\,(\% + \Esf\,\digit\,\digit^*) , \\
\alpha_4 &= \any .
\end{align*}
The elements of $L(\alpha_1)$, $L(\alpha_2)$ and $L(\alpha_3)$ are
whitespace characters, identifiers and numerals, respectively.  The
code associated with $\alpha_4$ will probably indicate that an error
has occurred.

A lexical analyzer meets such a specification iff it behaves as
follows.  At each stage of processing its file, the lexical analyzer
should consume the \emph{longest} prefix of the remaining input that
is in the language generated by one of the regular expressions.  It
should then supply the prefix to the code associated with the earliest
regular expression whose language contains the prefix.  However, if
there is no such prefix, or if the prefix is $\%$, then the lexical
analyzer should indicate that an error has occurred.

What happens when we process the file $\mathsf{123Easy\spacesym
  1E2\newlinesym}$ using a lexical analyzer meeting our example
specification?
\begin{itemize}
\item The longest prefix of $\mathsf{123Easy\spacesym 1E2\newlinesym}$
  that is in one of our regular expressions is $\mathsf{123}$.  Since
  this prefix is only in $\alpha_3$, it is consumed from the input and
  supplied to $\code_3$.

\item The remaining input is now $\mathsf{Easy\spacesym
    1E2\newlinesym}$.  The longest prefix of the remaining input that
  is in one of our regular expressions is $\mathsf{Easy}$.  Since this
  prefix is only in $\alpha_2$, it is consumed and supplied to
  $\code_2$.

\item The remaining input is then $\mathsf{\spacesym 1E2\newlinesym}$.
  The longest prefix of the remaining input that is in one of our
  regular expressions is $\spacesym$.  Since this prefix is only in
  $\alpha_1$ and $\alpha_4$, we consume it from the input and supply
  it to the code associated with the earlier of these regular
  expressions: $\code_1$.

\item The remaining input is then $\mathsf{1E2\newlinesym}$.  The
  longest prefix of the remaining input that is in one of our regular
  expressions is $\mathsf{1E2}$.  Since this prefix is only in
  $\alpha_3$, we consume it from the input and supply it to $\code_3$.

\item The remaining input is then $\mathsf{\newlinesym}$.  The longest
  prefix of the remaining input that is in one of our regular
  expressions is $\newlinesym$.  Since this prefix is only in
  $\alpha_1$, we consume it from the input and supply it to the code
  associated with this expression: $\code_1$.

\item The remaining input is now empty, and so the lexical analyzer
  terminates.
\end{itemize}

Now, we consider a simple method for generating a lexical analyzer
that meets a given specification.  More sophisticated methods are
described in compilers courses.

First, we convert the regular expressions $\alpha_1,\,\ldots,\alpha_n$
into DFAs $M_1,\,\ldots,M_n$.  Next we determine which of the states
of the DFAs are dead/live.

Given its remaining input $x$, the lexical analyzer consumes the next
token from $x$ and supplies the token to the appropriate code, as
follows.  First, it initializes the following variables to error
values:
\begin{itemize}
\item a string variable $\acc$, which records the longest prefix of
  the prefix of $x$ that has been processed so far that is accepted by
  one of the DFAs;

\item an integer variable $\mach$, which records the smallest $i$ such
  that $\acc\in L(M_i)$;

\item a string variable $\aft$, consisting of the suffix of $x$ that
  one gets by removing $\acc$.
\end{itemize}

Then, the lexical analyzer enters its main loop, in which it processes
$x$, symbol by symbol, in \emph{each} of the DFAs, keeping track of
what symbols have been processed so far, and what symbols remain to be
processed.
\begin{itemize}
\item If, after processing a symbol, at least one of the DFAs is in an
  accepting state, then the lexical analyzer stores the string that
  has been processed so far in the variable $\acc$, stores the index
  of the first machine to accept this string in the integer variable
  $\mach$, and stores the remaining input in the string variable
  $\aft$.  If there is no remaining input, then the lexical analyzer
  supplies $\acc$ to code $\code_\mach$, and returns; otherwise it
  continues.

\item If, after processing a symbol, none of the DFAs are in accepting
  states, but at least one automaton is in a live state (so that,
  without knowing anything about the remaining input, it's possible
  that an automaton will again enter an accepting state), then the
  lexical analyzer leaves $\acc$, $\mach$ and $\aft$ unchanged.  If
  there is no remaining input, the lexical analyzer supplies $\acc$ to
  $\code_\mach$ (it signals an error if $\acc$ is still set to the
  error value), resets the remaining input to $\aft$, and returns;
  otherwise, it continues.

\item If, after processing a symbol, all of the automata are in dead
  states (and so could never enter accepting states again, no matter
  what the remaining input was), the lexical analyzer supplies string
  $\acc$ to code $\code_\mach$ (it signals an error if $\acc$ is still
  set to the error value), resets the remaining input to $\aft$, and
  returns.
\end{itemize}

Let's see what happens when the file $\mathsf{123Easy\newlinesym}$ is
processed by the lexical analyzer generated from our example
specification.
\begin{itemize}
\item After processing $\onesf$, $M_3$ and $M_4$ are in accepting
  states, and so the lexical analyzer sets $\acc$ to $\onesf$, $\mach$
  to $3$, and $\aft$ to $\mathsf{23Easy\newlinesym}$.  It then
  continues.

\item After processing $\twosf$, so that $\mathsf{12}$ has been
  processed so far, only $M_3$ is in an accepting state, and so the
  lexical analyzer sets $\acc$ to $\mathsf{12}$, $\mach$ to $3$, and
  $\aft$ to $\mathsf{3Easy\newlinesym}$.  It then continues.

\item After processing $\threesf$, so that $\mathsf{123}$ has been
  processed so far, only $M_3$ is in an accepting state, and so the
  lexical analyzer sets $\acc$ to $\mathsf{123}$, $\mach$ to $3$, and
  $\aft$ to $\mathsf{Easy\newlinesym}$.  It then continues.

\item After processing $\Esf$, so that $\mathsf{123E}$ has been
  processed so far, none of the DFAs are in accepting states, but
  $M_3$ is in a live state, since $\mathsf{123E}$ is a prefix of a
  string that is accepted by $M_3$.  Thus the lexical analyzer
  continues, but doesn't change $\acc$, $\mach$ or $\aft$.

\item After processing $\mathsf{a}$, so that $\mathsf{123Ea}$ has been
  processed so far, all of the machines are in dead states, since
  $\mathsf{123Ea}$ isn't a prefix of a string that is accepted by one
  of the DFAs.  Thus the lexical analyzer supplies $\acc=\mathsf{123}$
  to $\code_\mach=\code_3$, and sets the remaining input to
  $\aft=\mathsf{Easy\newlinesym}$.

\item In subsequent steps, the lexical analyzer extracts
  $\mathsf{Easy}$ from the remaining input, and supplies this string
  to code $\code_2$, and extracts $\newlinesym$ from the remaining
  input, and supplies this string to code $\code_1$.
\end{itemize}

\subsection*{Design of Finite State Systems}

\index{finite automaton!design|(}%
\index{finite state system!design|(}%
Deterministic finite automata give us a means to efficiently---both in
terms of time and space---check membership in a regular language.  In
terms of time, a single left-to-right scan of the string is needed.
And we only need enough space to encode the DFA, and to keep track of
what state we are in at each point, as well as what part of the string
remains to be processed.  But if the string to be checked is
supplied, symbol-by-symbol, from our environment, we don't need to
store the string at all.

Consequently, DFAs may be easily and efficiently implemented in both
hardware and software.  One can design DFAs by hand, and test them
using Forlan.  But DFA minimization plus the operations on automata
and regular expressions of
Section~\ref{ClosurePropertiesOfRegularLanguages}, give us an
alternative---and very powerful---way of designing finite state
systems, which we will illustrate with two examples.

As the first example, suppose we wish to find a DFA $M$ such that
$L(M)=X$, where
\begin{displaymath}
X=\setof{w\in\mathsf{\{0,1\}^*}}{w\eqtxtl{has an even
    number of $\zerosf$'s or an odd number of $\onesf$'s}} .  
\end{displaymath}
First, we can note that $X=Y_1\cup Y_2$, where
\begin{align*}
Y_1 &= \setof{w\in\mathsf{\{0,1\}^*}}{w\eqtxtl{has an even
    number of $\zerosf$'s}}, \eqtxt{and} \\
Y_2 &= \setof{w\in\mathsf{\{0,1\}^*}}{w\eqtxtl{has an odd number of
    $\onesf$'s}}.
\end{align*}
Since we have a union operation on EFAs (Forlan
doesn't provide a union operation on DFAs), if we can find EFAs
accepting $Y_1$ and $Y_2$, we can combine them into a EFA that accepts
$X$.  Then we can convert this EFA to a DFA, and then minimize the
DFA.

Let $N_1$ and $N_2$ be the DFAs
\begin{center}
\input{chap-3.15-fig1.eepic}
\end{center}
It is easy to prove that $L(N_1)=Y_1$ and $L(N_2)=Y_2$.  Let $M$
be the DFA
\begin{gather*}
\renameStatesCanonically(\minimize(N)) ,
\end{gather*}
where $N$ is the DFA
\begin{gather*}
\nfaToDFA(\efaToNFA(\union(N_1,N_2))) .
\end{gather*}
\index{union@$\union$}%
\index{empty-string finite automaton!union@$\union$}%
\index{efaToNFA@$\efaToNFA$}%
\index{empty-string finite automaton!efaToNFA@$\efaToNFA$}%
\index{nondeterministic finite automaton!efaToNFA@$\efaToNFA$}%
\index{nfaToDFA@$\nfaToDFA$}%
\index{nondeterministic finite automaton!nfaToDFA@$\nfaToDFA$}%
\index{deterministic finite automaton!nfaToDFA@$\nfaToDFA$}%
\index{minimize@$\minimize$}%
\index{deterministic finite automaton!minimize@$\minimize$}%
\index{renameStatesCanonically@$\renameStatesCanonically$}%
\index{deterministic finite automaton!renameStatesCanonically@$\renameStatesCanonically$}%
Then
\begin{align*}
L(M) &= L(\renameStatesCanonically(\minimize\,N)) \\
&= L(\minimize\,N) \\
&= L(N) \\
&= L(\nfaToDFA(\efaToNFA(\union(N_1,N_2)))) \\
&= L(\efaToNFA(\union(N_1,N_2))) \\
&= L(\union(N_1,N_2)) \\
&= L(N_1)\cup L(N_2) \\
&= Y_1\cup Y_2 \\
&= X ,
\end{align*}
showing that $M$ is correct.

Suppose $M'$ is a DFA that accepts $X$.  Since
$M'\approx N$, we have that
$\minimize(N)$, and thus $M$, has no more states than
$M'$.  Thus $M$ has as few states as is possible.

But how do we figure out what the components of $M$ are, so that,
e.g., we can draw $M$?  In a simple case like this, we could apply the
definitions $\union$, $\efaToNFA$, $\nfaToDFA$,
$\minimize$ and $\renameStatesCanonically$,
and work out the answer.  But, for more complex examples, there would
be far too much detail involved for this to be a practical approach.

Instead, we can use Forlan to compute the answer.  Suppose
\texttt{dfa1} and \texttt{dfa2} of type \texttt{dfa} are
$N_1$ and $N_2$, respectively.  The we can proceed as follows:
\begin{list}{}
{\setlength{\leftmargin}{\leftmargini}
\setlength{\rightmargin}{0cm}
\setlength{\itemindent}{0cm}
\setlength{\listparindent}{0cm}
\setlength{\itemsep}{0cm}
\setlength{\parsep}{0cm}
\setlength{\labelsep}{0cm}
\setlength{\labelwidth}{0cm}
\catcode`\#=12
\catcode`\$=12
\catcode`\%=12
\catcode`\^=12
\catcode`\_=12
\catcode`\.=12
\catcode`\?=12
\catcode`\!=12
\catcode`\&=12
\ttfamily}
\small
\item[]\textsl{-\ }val\ efa\ =\ EFA.union(injDFAToEFA\ dfa1,\ injDFAToEFA\ dfa2);
\item[]\textsl{val\ efa\ =\ -\ :\ efa}
\item[]\textsl{-\ }val\ dfa'\ =\ nfaToDFA(efaToNFA\ efa);
\item[]\textsl{val\ dfa'\ =\ -\ :\ dfa}
\item[]\textsl{-\ }DFA.numStates\ dfa';
\item[]\textsl{val\ it\ =\ 5\ :\ int}
\item[]\textsl{-\ }val\ dfa\ =\ DFA.renameStatesCanonically(DFA.minimize\ dfa');
\item[]\textsl{val\ dfa\ =\ -\ :\ dfa}
\item[]\textsl{-\ }DFA.numStates\ dfa;
\item[]\textsl{val\ it\ =\ 4\ :\ int}
\item[]\textsl{-\ }DFA.output("",\ dfa);
\item[]\textsl{\symbol{'173}states\symbol{'175}\ A,\ B,\ C,\ D\ \symbol{'173}start\ state\symbol{'175}\ D\ \symbol{'173}accepting\ states\symbol{'175}\ A,\ C,\ D}
\item[]\textsl{\symbol{'173}transitions\symbol{'175}}
\item[]\textsl{A,\ 0\ ->\ C;\ A,\ 1\ ->\ D;\ B,\ 0\ ->\ D;\ B,\ 1\ ->\ C;\ C,\ 0\ ->\ A;\ C,\ 1\ ->\ B;}
\item[]\textsl{D,\ 0\ ->\ B;\ D,\ 1\ ->\ A}
\item[]\textsl{val\ it\ =\ ()\ :\ unit}
\end{list}

\index{injDFAToEFA@\texttt{injDFAToEFA}}%
\index{EFA@\texttt{EFA}!union@\texttt{union}}%
\index{efaToNFA@\texttt{efaToNFA}}%
\index{nfaToDFA@\texttt{nfaToDFA}}%
\index{DFA@\texttt{DFA}!inter@\texttt{inter}}%
\index{DFA@\texttt{DFA}!minimize@\texttt{minimize}}%
\index{DFA@\texttt{DFA}!renameStatesCanonically@\texttt{renameStatesCanonically}}%
Thus $M$ is:
\begin{center}
\input{chap-3.15-fig2.eepic}
\end{center}
Of course, this claim assumes that Forlan is correctly
implemented.

We conclude this subsection by considering a second, more involved
example of DFA design.  Given a string $w\in\{\mathsf{0,1}\}^*$, we
say that:
\begin{itemize}
\item $w$ \emph{stutters} iff $aa$ is a substring of $w$, for
\index{stuttering}%
\index{string!stuttering}%
some $a\in\{\mathsf{0,1}\}$;

\item $w$ is \emph{long} iff $|w|\geq 5$.
\end{itemize}
So, e.g., $\mathsf{1001}$ and $\mathsf{10110}$ both stutter, but
$\mathsf{01010}$ and $\mathsf{101}$ don't.  Saying that strings of
length $5$ or more are ``long'' is arbitrary; what follows can be
repeated with different choices of when strings are long.

Let the language $\AllLongStutter$ be
\begin{displaymath}
\setof{w\in\{\mathsf{0,1}\}^*}{\eqtxtr{for all substrings}v\eqtxt{of}w,
\eqtxt{if}v\eqtxt{is long, then}v\eqtxtl{stutters}} .
\end{displaymath}
In other words, a string of $\zerosf$'s and $\onesf$'s is
in $\AllLongStutter$ iff every long substring of this string
stutters.  Since every substring of $\mathsf{0010110}$ of
length five stutters, every long substring of this string stutters,
and thus the string is in $\AllLongStutter$.  On the other
hand, $\mathsf{0010100}$ is not in $\AllLongStutter$, because
$\mathsf{01010}$ is a long, non-stuttering substring of this
string.

Let's consider the problem of finding a DFA that accepts this
language.  One possibility is to reduce this problem to that of
finding a DFA that accepts the complement of $\AllLongStutter$.  Then
we'll be able to use our set difference operation on DFAs to build a
DFA that accepts $\AllLongStutter$, which we can then mimimize.
(We'll also need a DFA accepting $\{\mathsf{0,1}\}^*$.)  To form the
complement of $\AllLongStutter$, we negate the formula in
$\AllLongStutter$'s expression.  Let $\SomeLongNotStutter$ be the
language
\begin{displaymath}
\mtab{
\{\,w\in\{\mathsf{0,1}\}^* \mid \TS\eqtxtr{there is a substring}v\eqtxt{of}w
\eqtxtl{such that}\NL
v\eqtxtl{is long and doesn't stutter}\,\}.}
\end{displaymath}

\begin{lemma}
\label{Stutter1}
$\AllLongStutter = \{\mathsf{0,1}\}^*-\SomeLongNotStutter$.
\end{lemma}

\begin{proof}
Suppose $w\in\AllLongStutter$, so that
$w\in\{\mathsf{0,1}\}^*$ and, for all substrings $v$ of $w$,
if $v$ is long, then $v$ stutters.  Suppose, toward a contradiction,
that $w\in\SomeLongNotStutter$.  Then there is a substring $v$ of
$w$ such that $v$ is long and doesn't stutter---contradiction.
Thus $w\not\in\SomeLongNotStutter$, completing the proof
that $w\in \{\mathsf{0,1}\}^*-\SomeLongNotStutter$.

Suppose $w\in\{\mathsf{0,1}\}^*-\SomeLongNotStutter$, so that
$w\in\{\mathsf{0,1}\}^*$ and $w\not\in
\SomeLongNotStutter$.  To see that $w\in\AllLongStutter$,
suppose $v$ is a substring of $w$ and $v$ is long.  Suppose, toward a
contradiction, that $v$ doesn't stutter.  Then
$w\in\SomeLongNotStutter$---contradiction.  Hence $v$ stutters.
\end{proof}

Next, it's convenient to work bottom-up for a bit.  Let
\begin{align*}
\Long &= \setof{w\in\{\mathsf{0,1}\}^*}{w\eqtxtl{is long}} , \\
\Stutter &= \setof{w\in\{\mathsf{0,1}\}^*}{w\eqtxtl{stutters}} , \\
\NotStutter &= \setof{w\in\{\mathsf{0,1}\}^*}{w\eqtxtl{doesn't stutter}} ,
  \eqtxt{and} \\
\LongAndNotStutter &=
\setof{w\in\{\mathsf{0,1}\}^*}{w\eqtxtl{is long and doesn't stutter}} .
\end{align*}

The following lemma is easy to prove:

\begin{lemma}
\label{Stutter2}
\begin{enumerate}[\quad(1)]
\item $\NotStutter = \{\mathsf{0,1}\}^* - \Stutter$.

\item $\LongAndNotStutter = \Long\cap\NotStutter$.
\end{enumerate}
\end{lemma}

Clearly, we'll be able to find DFAs accepting $\Long$ and $\Stutter$,
respectively.  Thus, we'll be able to use our set difference operation
on DFAs to come up with a DFA that accepts $\NotStutter$.  Then,
we'll be able to use our intersection operation on DFAs to come
up with a DFA that accepts $\LongAndNotStutter$.

What remains is to find a way of converting $\LongAndNotStutter$
to $\SomeLongNotStutter$.  Clearly, the former language is
a subset of the latter one.  But the two languages are not equal,
since an element of the latter language may have the form
$xvy$, where $x,y\in\{\mathsf{0,1}\}^*$ and $v\in\LongAndNotStutter$.
This suggests the following lemma:

\begin{lemma}
\label{Stutter3}
$\SomeLongNotStutter =
\{\mathsf{0,1}\}^*\,\LongAndNotStutter\,\{\mathsf{0,1}\}^*$.
\end{lemma}

\begin{proof}
Suppose $w\in\SomeLongNotStutter$, so that $w\in\{\mathsf{0,1}\}^*$
and there is a substring $v$ of $w$ such that $v$ is long and doesn't
stutter.  Thus $v\in\LongAndNotStutter$, and $w=xvy$ for some
$x,y\in\{\mathsf{0,1}\}^*$.  Hence
$w=xvy\in\{\mathsf{0,1}\}^*\,\LongAndNotStutter\,\{\mathsf{0,1}\}^*$.

Suppose $w\in
\{\mathsf{0,1}\}^*\,\LongAndNotStutter\,\{\mathsf{0,1}\}^*$,
so that $w=xvy$ for some $x,y\in\{\mathsf{0,1}\}^*$ and
$v\in\LongAndNotStutter$.  Hence $v$ is long and doesn't stutter.
Thus $v$ is a long substring of $w$ that doesn't stutter,
showing that $w\in\SomeLongNotStutter$.
\end{proof}

Because of the preceding lemma, we can construct an EFA accepting
$\SomeLongNotStutter$ from a DFA accepting $\{\mathsf{0,1}\}^*$ and
our DFA accepting $\LongAndNotStutter$, using our concatenation
operation on EFAs.  (We haven't given a concatenation operation on
DFAs.)  We can then convert this EFA to a DFA.

Now, let's take the preceding ideas and turn them into reality.
First, we define functions $\regToEFA\in\Reg\fun\EFA$,
$\efaToDFA\in\EFA\fun\DFA$, $\regToDFA\in\Reg\fun\DFA$
and $\minAndRen\in\DFA\fun\DFA$ by:
\index{regToFA@$\regToFA$}%
\index{regular expression!regToFA@$\regToFA$}%
\index{finite automaton!regToFA@$\regToFA$}%
\index{faToEFA@$\faToEFA$}%
\index{finite automaton!faToEFA@$\faToEFA$}%
\index{empty-string finite automaton!faToEFA@$\faToEFA$}%
\index{efaToNFA@$\efaToNFA$}%
\index{empty-string finite automaton!efaToNFA@$\efaToNFA$}%
\index{nondeterministic finite automaton!efaToNFA@$\efaToNFA$}%
\index{nfaToDFA@$\nfaToDFA$}%
\index{nondeterministic finite automaton!nfaToDFA@$\nfaToDFA$}%
\index{deterministic finite automaton!nfaToDFA@$\nfaToDFA$}%
\index{renameStatesCanonically@$\renameStatesCanonically$}%
\index{deterministic finite automaton!renameStatesCanonically@$\renameStatesCanonically$}%
\index{minimize@$\minimize$}%
\index{deterministic finite automaton!minimize@$\minimize$}%
\index{regToEFA@$\regToEFA$}%
\index{regular expression!regToEFA@$\regToEFA$}%
\index{empty-string finite automaton!regToEFA@$\regToEFA$}%
\index{efaToDFA@$\efaToDFA$}%
\index{empty-string finite automaton!efaToDFA@$\efaToDFA$}%
\index{deterministic finite automaton!efaToDFA@$\efaToDFA$}%
\index{regToDFA@$\regToDFA$}%
\index{regular expression!regToDFA@$\regToDFA$}%
\index{deterministic finite automaton!regToDFA@$\regToDFA$}%
\index{minAndRen@$\minAndRen$}%
\index{deterministic finite automaton!minAndRen@$\minAndRen$}%
\begin{align*}
\regToEFA &= \faToEFA \circ \regToFA , \\
\efaToDFA &= \nfaToDFA \circ \efaToNFA , \\
\regToDFA &= \efaToDFA \circ \regToEFA , \eqtxt{and} \\
\minAndRen &= \renameStatesCanonically \circ \minimize .
\end{align*}

\begin{lemma}
\label{Stutter4}
\begin{enumerate}[\quad(1)]
\item For all $\alpha\in\Reg$, $L(\regToEFA(\alpha))=L(\alpha)$.

\item For all $M\in\EFA$, $L(\efaToDFA(M)) = L(M)$.

\item For all $\alpha\in\Reg$, $L(\regToDFA(\alpha)) = L(\alpha)$.

\item For all $M\in\DFA$, $L(\minAndRen(M)) = L(M)$ and,
for all $N\in\DFA$, if $L(N)=L(M)$, then $\minAndRen(M)$ has no more
states than $N$.
\end{enumerate}
\end{lemma}

\begin{proof}
We show the proof of Part~(4), the proofs of the other parts being
even easier.  Suppose $M\in\DFA$.  By Theorem~\ref{Minimization}(1), we
have that
\begin{align*}
L(\minAndRen(M)) &= L(\renameStatesCanonically(\minimize\,M)) \\
                 &= L(\minimize\,M) \\
                 &= L(M) .
\end{align*}
Suppose $N\in\DFA$ and $L(N)=L(M)$.  By Theorem~\ref{Minimization}(4),
$\minimize(M)$ has no more states than $N$.
Hence $\renameStatesCanonically(\minimize(M))$ has no more states than $N$,
showing that $\minAndRen(M)$ has no more states than $N$.
\end{proof}

Let the regular expression $\allStrReg$ be $\mathsf{(0+1)^*}$.
Clearly $L(\allStrReg)=\{\mathsf{0,1}\}^*$.  Let the DFA
$\allStrDFA$ be
\begin{displaymath}
\minAndRen(\regToDFA\,\allStrReg) .
\end{displaymath}

\begin{lemma}
\label{Stutter5}
$L(\allStrDFA)=\{\mathsf{0,1}\}^*$.
\end{lemma}

\begin{proof}
By Lemma~\ref{Stutter4}, we have that
\begin{align*}
L(\allStrDFA) &= L(\minAndRen(\regToDFA\,\allStrReg)) \\
              &= L(\regToDFA\,\allStrReg) \\
              &= L(\allStrReg) \\
              &= \{\mathsf{0,1}\}^* .
\end{align*}
\end{proof}

(Not surprisingly, $\allStrDFA$ will have a single state.)
Let the EFA $\allStrEFA$ be the DFA $\allStrDFA$.  Thus
$L(\allStrEFA)=\{\mathsf{0,1}\}^*$.

Let the regular expression $\longReg$ be
\begin{displaymath}
(\mathsf{0+1})^5(\mathsf{0+1})^* .
\end{displaymath}

\begin{lemma}
\label{Stutter6}

$L(\longReg) = \Long$.
\end{lemma}

\begin{proof}
Since $L(\longReg)=\{\mathsf{0,1}\}^5\{\mathsf{0,1}\}^*$,
it will suffice to show that 
$\{\mathsf{0,1}\}^5\{\mathsf{0,1}\}^* = \Long$.

Suppose $w\in\{\mathsf{0,1}\}^5\{\mathsf{0,1}\}^*$, so that
$w=xy$, for some $x\in\{\mathsf{0,1}\}^5$ and $y\in
\{\mathsf{0,1}\}^*$.  Thus $w=xy\in\{\mathsf{0,1}\}^*$
and $|w|\geq|x|=5$, showing that $w\in\Long$.

Suppose $w\in\Long$, so that $w\in\{\mathsf{0,1}\}^*$ and
$|w|\geq 5$.  Then $w=abcdex$, for some $a,b,c,d,e\in\{\mathsf{0,1}\}$
and $x\in\{\mathsf{0,1}\}^*$.  Hence $w=(abcde)x\in
\{\mathsf{0,1}\}^5\{\mathsf{0,1}\}^*$.
\end{proof}

Let the DFA $\longDFA$ be
\begin{displaymath}
\minAndRen(\regToDFA(\longReg)) .
\end{displaymath}
An easy calculation shows that $L(\longDFA) = \Long$.

Let $\stutterReg$ be the regular expression
\begin{displaymath}
\mathsf{(0+1)^*(00+11)(0 + 1)^*} .
\end{displaymath}

\begin{lemma}
\label{Stutter7}
$L(\stutterReg) = \Stutter$.
\end{lemma}

\begin{proof}
Since $L(\stutterReg)=
\{\mathsf{0,1}\}^*\{\mathsf{00,11}\}\{\mathsf{0,1}\}^*$,
it will suffice to show that
$\{\mathsf{0,1}\}^*\{\mathsf{00,11}\}\{\mathsf{0,1}\}^* = \Stutter$,
and this is easy.
\end{proof}

Let $\stutterDFA$ be the DFA
\begin{displaymath}
\minAndRen(\regToDFA(\stutterReg)) .
\end{displaymath}
An easy calculation shows that $L(\stutterDFA) = \Stutter$.
Let $\notStutterDFA$ be the DFA
\index{minus@$\minus$}%
\index{deterministic finite automaton!minus@$\minus$}%
\begin{displaymath}
\minAndRen(\minus(\allStrDFA, \stutterDFA)) .
\end{displaymath}

\begin{lemma}
\label{Stutter8}
$L(\notStutterDFA) = \NotStutter$.
\end{lemma}

\begin{proof}
Let $M$ be
\begin{displaymath}
\minAndRen(\minus(\allStrDFA, \stutterDFA)) .
\end{displaymath}
By Lemma~\ref{Stutter2}(1), we have that
\begin{align*}
L(\notStutterDFA) &= L(M) \\
&= L(\minus(\allStrDFA, \stutterDFA)) \\
&= L(\allStrDFA) - L(\stutterDFA) \\
&= \{\mathsf{0,1}\}^* - \Stutter \\
&= \NotStutter .
\end{align*}
\end{proof}

Let $\longAndNotStutterDFA$ be the DFA
\begin{displaymath}
\minAndRen(\inter(\longDFA, \notStutterDFA)) .
\end{displaymath}
\index{inter@$\inter$}%
\index{empty-string finite automaton!inter@$\inter$}%

\begin{lemma}
\label{Stutter9}
$L(\longAndNotStutterDFA) = \LongAndNotStutter$.
\end{lemma}

\begin{proof}
Let $M$ be
\begin{displaymath}
\minAndRen(\inter(\longDFA, \notStutterDFA)) .
\end{displaymath}
By Lemma~\ref{Stutter2}(2), we have that
\begin{align*}
L(\longAndNotStutterDFA) &= L(M) \\
&= L(\inter(\longDFA, \notStutterDFA)) \\
&= L(\longDFA)\cap L(\notStutterDFA) \\
&= \Long\cap\NotStutter \\
&= \LongAndNotStutter .
\end{align*}
\end{proof}

Because $\longAndNotStutterDFA$ is an EFA, we can simply let
the EFA $\longAndNotStutterEFA$ be
$\longAndNotStutterDFA$.  Thus we have that
$L(\longAndNotStutterEFA) = \LongAndNotStutter$.

Let $\someLongNotStutterEFA$ be the EFA
\begin{displaymath}
\mtab{
\renameStatesCanonically(\concat(\TS\allStrEFA, \NL
                                 \concat(\TS\longAndNotStutterEFA, \NL
                                             \allStrEFA))).}
\end{displaymath}
\index{renameStatesCanonically@$\renameStatesCanonically$}%
\index{empty-string finite automaton!renameStatesCanonically@$\renameStatesCanonically$}%
\index{concat@$\concat$}%
\index{empty-string finite automaton!concat@$\concat$}%

\begin{lemma}
\label{Stutter10}
$L(\someLongNotStutterEFA) = \SomeLongNotStutter$.
\end{lemma}

\begin{proof}
We have that
\begin{align*}
L(\someLongNotStutterEFA) &= L(\renameStatesCanonically\,M) \\
&= L(M) ,
\end{align*}
where $M$ is
\begin{displaymath}
\concat(\allStrEFA,\concat(\longAndNotStutterEFA,\allStrEFA)) .
\end{displaymath}
And, by Lemma~\ref{Stutter3}, we have that
\begin{align*}
L(M) &= L(\allStrEFA)\,L(\longAndNotStutterEFA)\,L(\allStrEFA) \\
     &= \{\mathsf{0,1}\}^*\,\LongAndNotStutter\,\{\mathsf{0,1}\}^* \\
     &= \SomeLongNotStutter .
\end{align*}
\end{proof}

Let $\someLongNotStutterDFA$ be the DFA
\begin{displaymath}
\minAndRen(\efaToDFA\,\someLongNotStutterEFA) .
\end{displaymath}

\begin{lemma}
\label{Stutter11}
$L(\someLongNotStutterDFA) = \SomeLongNotStutter$.
\end{lemma}

\begin{proof}
Follows by an easy calculation.
\end{proof}

Finally, let $\allLongStutterDFA$ be the DFA
\begin{displaymath}
\minAndRen(\minus(\allStrDFA, \someLongNotStutterDFA)) .
\end{displaymath}

\begin{lemma}
\label{Stutter12}
$L(\allLongStutterDFA) = \AllLongStutter$ and,
for all $N\in\DFA$, if $L(N)=\AllLongStutter$,
then $\allLongStutterDFA$ has no more states than $N$.
\end{lemma}

\begin{proof}
We have that
\begin{displaymath}
L(\allLongStutterDFA) = L(\minAndRen(M)) = L(M) ,
\end{displaymath}
where $M$ is
\begin{displaymath}
\minus(\allStrDFA, \someLongNotStutterDFA) .
\end{displaymath}
Then, by Lemma~\ref{Stutter1}, we have that
\begin{align*}
L(M) &= L(\allStrDFA) - L(\someLongNotStutterDFA) \\
&= \{\mathsf{0,1}\}^* - \SomeLongNotStutter \\
&= \AllLongStutter .
\end{align*}
Suppose $N\in\DFA$ and $L(N)=\AllLongStutter$.  Thus $L(N)=L(M)$, so
that $\allLongStutterDFA$ has no more states than $N$, by
Lemma~\ref{Stutter4}(4).
\end{proof}

The preceding lemma tells us that the DFA $\allLongStutterDFA$
is correct and has as few states as is possible.
To find out what it looks like, though, we'll have to use
Forlan.  First we put the text
\verbatiminput{stutter.sml}

\index{faToEFA@\texttt{faToEFA}}%
\index{regToFA@\texttt{regToFA}}%
\index{nfaToDFA@\texttt{nfaToDFA}}%
\index{efaToNFA@\texttt{efaToNFA}}%
\index{DFA@\texttt{DFA}!renameStatesCanonically@\texttt{renameStatesCanonically}}%
\index{DFA@\texttt{DFA}!minimize@\texttt{minimize}}%
\index{Reg@\texttt{Reg}!fromString@\texttt{fromString}}%
\index{injDFAToEFA@\texttt{injDFAToEFA}}%
\index{Reg@\texttt{Reg}!concat@\texttt{concat}}%
\index{Reg@\texttt{Reg}!power@\texttt{power}}%
\index{Reg@\texttt{Reg}!fromString@\texttt{fromString}}%
\index{DFA@\texttt{DFA}!minus@\texttt{minus}}%
\index{DFA@\texttt{DFA}!inter@\texttt{inter}}%
\index{EFA@\texttt{EFA}!concat@\texttt{concat}}%
\index{EFA@\texttt{EFA}!renameStatesCanonically@\texttt{renameStatesCanonically}}%
in the file \texttt{stutter.sml}.  Then, we proceed as follows
\begin{list}{}
{\setlength{\leftmargin}{\leftmargini}
\setlength{\rightmargin}{0cm}
\setlength{\itemindent}{0cm}
\setlength{\listparindent}{0cm}
\setlength{\itemsep}{0cm}
\setlength{\parsep}{0cm}
\setlength{\labelsep}{0cm}
\setlength{\labelwidth}{0cm}
\catcode`\#=12
\catcode`\$=12
\catcode`\%=12
\catcode`\^=12
\catcode`\_=12
\catcode`\.=12
\catcode`\?=12
\catcode`\!=12
\catcode`\&=12
\ttfamily}
\small
\item[]\textsl{-\ }use\ "stutter.sml";
\item[]\textsl{\symbol{'133}opening\ stutter.sml\symbol{'135}}
\item[]\textsl{val\ regToEFA\ =\ fn\ :\ reg\ ->\ efa}
\item[]\textsl{val\ efaToDFA\ =\ fn\ :\ efa\ ->\ dfa}
\item[]\textsl{val\ regToDFA\ =\ fn\ :\ reg\ ->\ dfa}
\item[]\textsl{val\ minAndRen\ =\ fn\ :\ dfa\ ->\ dfa}
\item[]\textsl{val\ allStrReg\ =\ -\ :\ reg}
\item[]\textsl{val\ allStrDFA\ =\ -\ :\ dfa}
\item[]\textsl{val\ allStrEFA\ =\ -\ :\ efa}
\item[]\textsl{val\ longReg\ =\ -\ :\ reg}
\item[]\textsl{val\ longDFA\ =\ -\ :\ dfa}
\item[]\textsl{val\ stutterReg\ =\ -\ :\ reg}
\item[]\textsl{val\ stutterDFA\ =\ -\ :\ dfa}
\item[]\textsl{val\ notStutterDFA\ =\ -\ :\ dfa}
\item[]\textsl{val\ longAndNotStutterDFA\ =\ -\ :\ dfa}
\item[]\textsl{val\ longAndNotStutterEFA\ =\ -\ :\ efa}
\item[]\textsl{val\ someLongNotStutterEFA'\ =\ -\ :\ efa}
\item[]\textsl{val\ someLongNotStutterEFA\ =\ -\ :\ efa}
\item[]\textsl{val\ someLongNotStutterDFA\ =\ -\ :\ dfa}
\item[]\textsl{val\ allLongStutterDFA\ =\ -\ :\ dfa}
\item[]\textsl{val\ it\ =\ ()\ :\ unit}
\item[]\textsl{-\ }DFA.output("",\ allLongStutterDFA);
\item[]\textsl{\symbol{'173}states\symbol{'175}\ A,\ B,\ C,\ D,\ E,\ F,\ G,\ H,\ I,\ J\ \symbol{'173}start\ state\symbol{'175}\ A}
\item[]\textsl{\symbol{'173}accepting\ states\symbol{'175}\ A,\ B,\ C,\ D,\ E,\ F,\ G,\ H,\ I}
\item[]\textsl{\symbol{'173}transitions\symbol{'175}}
\item[]\textsl{A,\ 0\ ->\ B;\ A,\ 1\ ->\ C;\ B,\ 0\ ->\ B;\ B,\ 1\ ->\ E;\ C,\ 0\ ->\ D;\ C,\ 1\ ->\ C;}
\item[]\textsl{D,\ 0\ ->\ B;\ D,\ 1\ ->\ G;\ E,\ 0\ ->\ F;\ E,\ 1\ ->\ C;\ F,\ 0\ ->\ B;\ F,\ 1\ ->\ I;}
\item[]\textsl{G,\ 0\ ->\ H;\ G,\ 1\ ->\ C;\ H,\ 0\ ->\ B;\ H,\ 1\ ->\ J;\ I,\ 0\ ->\ J;\ I,\ 1\ ->\ C;}
\item[]\textsl{J,\ 0\ ->\ J;\ J,\ 1\ ->\ J}
\item[]\textsl{val\ it\ =\ ()\ :\ unit}
\end{list}

Thus, $\allLongStutterDFA$ is the DFA of Figure~\ref{StuttDFASynExamp}.
\begin{figure}
\begin{center}
\input{chap-3.15-fig3.eepic}
\end{center}
\caption{DFA Accepting $\AllLongStutter$}
\label{StuttDFASynExamp}
\end{figure}
\index{finite state system!synthesis|)}%
\index{finite automaton!synthesis|)}%

\subsection{Notes}

Our treatment of searching for regular expressions in text file is
standard, as is that of lexical analysis.  But our approach to
designing finite state systems depends upon having access to a
toolset, like Forlan, that is embedded in a programming language and
implements our algorithms for manipulating finite automata and regular
expressions.

%%% Local Variables: 
%%% mode: latex
%%% TeX-master: "book"
%%% End: 
