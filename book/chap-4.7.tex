\section{Closure Properties of Context-free Languages}
\label{ClosurePropertiesOfContextFreeLanguages}

\index{context-free language!closure properties|(}%
In this section, we define union, concatenation, closure, reversal,
alphabet-renaming and prefix-closure operations/algorithms on
grammars.  As a result, we will have that the context-free languages
are closed under union, concatenation, closure, reversal,
alphabet-renaming and prefix-, suffix- and substring-closure.

In Section~\ref{ThePumpingLemmaForContextFreeLanguages}, we
will see that the context-free languages aren't closed under
intersection, complementation and set difference.
But we are able to define operations/algorithms for:
\begin{itemize}
\item intersecting a grammar and an empty-string finite automaton; and

\item subtracting a deterministic finite automaton from a grammar.
\end{itemize}
Thus, if $L_1$ is a context-free language, and $L_2$ is a regular
language, we will have that $L_1\cap L_2$ and $L_1-L_2$ are
context-free.

\subsection{Operations on Grammars}

First, we consider some basic grammars and operations on grammars.
The grammar, $\emptyStr$, with variable $\Asf$ and production
$\Asf\fun\%$ generates the language $\{\%\}$.  The grammar,
$\emptySet$, with variable $\Asf$ and no productions generates the
language $\emptyset$.  If $w$ is a string, then the grammar with
variable $\Asf$ and production $\Asf\fun w$ generates the language
$\{w\}$.  Actually, we must be careful to chose a variable that
doesn't occur in $w$. We can do that by adding as many nested
\texttt{<} and \texttt{>} around \texttt{A} as needed (\texttt{A},
\texttt{<A>}, \texttt{<<A>>}, etc.). This defines functions
$\strToGram\in\Str\fun\Gram$ and $\symToGram\in\Sym\fun\Gram$.

Suppose $G_1$ and $G_2$ are grammars.  We can define a grammar $H$
such that $L(H)=L(G_1)\cup L(G_2)$ by unioning together the variables
and productions of $G_1$ and $G_2$, and adding a new start variable
$q$, along with productions
\begin{gather*}
  q\fun s_{G_1}\mid s_{G_2} .
\end{gather*}
For the above to be valid, we need to know that:
\begin{itemize}
\item $Q_{G_1}\cap Q_{G_2}=\emptyset$ and
$q\not\in Q_{G_1}\cup Q_{G_2}$; and

\item $\alphabet\,G_1\cap Q_{G_2}=\emptyset$,
$\alphabet\,G_2\cap Q_{G_1}=\emptyset$ and
$q\not\in\alphabet\,G_1\cup\alphabet\,G_2$.
\end{itemize}
Our official union operation for grammars renames the variables
of $G_1$ and $G_2$, and chooses the start variable $q$,
in a uniform way that makes the preceding properties hold.
This gives us a function
$\union\in\Gram\times\Gram\fun\Gram$.

We do something similar when defining the other closure
operations.  In what follows, though, we'll ignore this issue,
so as to keep things simple.

Suppose $G_1$ and $G_2$ are grammars.  We can define a grammar $H$
such that $L(H)=L(G_1)L(G_2)$ by unioning together the variables and
productions of $G_1$ and $G_2$, and adding a new start variable $q$,
along with production
\begin{gather*}
  q\fun s_{G_1}s_{G_2} .
\end{gather*}
This gives us a function $\concat\in\Gram\times\Gram\fun\Gram$.

Suppose $G$ is a grammar.  We can define a grammar $H$ such that
$L(H)=L(G)^*$ by adding to the variables and productions of $G$ a new
start variable $q$, along with productions
\begin{gather*}
q\fun \%\mid s_Gq .
\end{gather*}
This gives us a function $\closure\in\Gram\fun\Gram$.

Next, we consider reversal and alphabet renaming operations on
grammars.  Given a grammar $G$, we can define a grammar $H$ such that
\index{grammar!reversal}%
\index{reversal!grammar}%
\index{language!reversal}%
\index{reversal!language}%
$L(H)=L(G)^R$ by simply reversing the right-sides of $G$'s
productions.
This gives a function $\rev\in\Gram\fun\Gram$.

Given a grammar $G$ and a bijection $f$ from a set of symbols that is
a superset of $\alphabet\,G$ to some set of symbols, we can define a
grammar $H$ such that $L(H)=L(G)^f$ by renaming the elements of
\index{grammar!alphabet renaming}%
\index{alphabet renaming!grammar}%
\index{language!alphabet renaming}%
\index{alphabet renaming!language}%
$\alphabet\,G$ in the right-sides of $G$'s productions using $f$.
Actually, we may have to rename the variables of $G$ to avoid clashes
with the elements of the renamed alphabet.  Let
$X=\setof{(G,f)}{G\in\Gram\eqtxt{and}f\eqtxtl{is a bijection from a
    set of symbols that}\eqtxtr{is a superset
    of}\alphabet\,G\eqtxtl{to some set of symbols}}$.  Then the above
definition gives us a function
$\renameAlphabet\in X\fun\Gram$.

From Section~\ref{ClosurePropertiesOfRegularLanguages},
we know that if we can define a prefix-closure operation
\index{language!prefix-closure}%
\index{language!suffix-closure}%
\index{language!substring-closure}%
\index{prefix-closure!language}%
\index{suffix-closure!language}%
\index{substring-closure!language}%
\index{grammar!prefix-closure}%
\index{grammar!suffix-closure}%
\index{grammar!substring-closure}%
\index{prefix-closure!grammar}%
\index{suffix-closure!grammar}%
\index{substring-closure!grammar}%
on grammmars, then we can obtain suffix-closure and substring-closure
operations on grammars from the prefix-closure and grammar reversal
operations.

So how can we turn a grammar $G$ into a grammar $H$ such that
$L(H)=L(G)^P$?
We begin by simplifying $G$, producing grammar $G'$.
Thus all of the variables of $G'$ will be useful, unless $G'$ has
a single variable and no productions.  Now, we form the grammmar $H$
from $G'$, as follows.
We make a copy of $G'$, renaming each variable $q$ to
$\langle\onesf,q\rangle$.  (Actually, we may have to rename variables
to avoid clashes with alphabet symbols.)
Next, for each alphabet symbol $a$, we introduce a new variable
$\langle\twosf,a\rangle$, along with productions
$\langle\twosf,a\rangle\fun\%\mid a$.
Next, for each variable $q$ of $G'$, we add a new variable
$\langle\twosf,q\rangle$ that generates all prefixes of what $q$
generated in $G'$.  Suppose we are given a production $q\fun
a_1a_2\cdots a_n$ of $G'$.  If $n=0$, then we replace it with the
production $\langle\twosf,q\rangle\fun\%$.  Otherwise, we replace it
with the productions
\begin{gather*}
\langle\twosf,q\rangle\fun
\langle\twosf,a_1\rangle \mid
f(a_1)\langle\twosf,a_2\rangle \mid \cdots \mid
f(a_1)\,f(a_2)\cdots\langle\twosf,a_n\rangle ,
\end{gather*}
where $f(a) = a$, if $a\in\alphabet\,G'$, and
$f(a)=\langle\onesf,a\rangle$, if $a\in Q_{G'}$.  It's crucial that
$G'$ is simplified; otherwise productions with useless symbols would
be turned into productions that generated strings. Finally, the start
variable of $H$ is $\langle\twosf,s_{G'}\rangle$.

The above definition gives a function $\prefix\in\Gram\fun\Gram$.
For example, the grammar
\begin{gather*}
\Asf\fun \% \mid \zerosf\Asf\onesf
\end{gather*}
is turned into the grammar
\begin{align*}
\langle \twosf,\Asf\rangle &\fun \% \mid
  \langle\twosf,\zerosf\rangle \mid
  \zerosf\langle\twosf,\Asf\rangle \mid
  \zerosf\langle\onesf,A\rangle\langle\twosf,\onesf\rangle , \\
\langle\onesf,A\rangle &\fun \% \mid
  \zerosf\langle\onesf,\Asf\rangle\onesf , \\
\langle\twosf,\zerosf\rangle &\fun \% \mid \zerosf , \\
\langle\twosf,\onesf\rangle &\fun \% \mid \onesf .
\end{align*}

We now consider an algorithm for intersecting a grammar $G$ with an
EFA $M$, resulting in $\simplify\,H$, where the grammar $H$ is defined
as follows.
For all $p\in Q_G$ and $q, r\in Q_M$, $H$ has a variable
$\langle p,q,r\rangle$ that generates
\begin{gather*}
\setof{w\in(\alphabet\,G)^*}{w\in\Pi_{G,p} \eqtxt{and}
 r\in\Delta_M(\{q\},w)} .
\end{gather*}
The remaining variable of $H$ is $\Asf$, which is its start variable.

For each $r\in A_M$, $H$ has a production
\begin{gather*}
  \Asf \fun \langle s_G, s_M, r\rangle .
\end{gather*}
And for each $\%$-production $p\fun\%$ of $G$ and $q,r\in Q_M$, if
$r\in \Delta_M(\{q\}, \%)$,
then $H$ will have the production
\begin{gather*}
  \langle p,q,r\rangle \fun \% .
\end{gather*}

To say what the remaining productions of $H$ are, define
a function
\begin{gather*}
f \in (\alphabet\,G \cup Q_G) \times Q_M \times Q_M\fun
\alphabet\,G\cup Q_H
\end{gather*}
by: for all $a\in\alphabet\,G \cup Q_G$ and $q,r\in Q_M$,
\begin{gather*}
  f(a, q, r) =
  \casesdef{a,}{\eqtxtr{if} a\in\alphabet\,G , \eqtxtl{and}}%
  {\langle a,q,r\rangle,}{\eqtxtr{if} a\in Q_G .}
\end{gather*}
Then, for all $p\in Q_G$, $n\in\nats-\{0\}$,
$a_1,\ldots,a_n\in\Sym$ and $q_1,\ldots,q_{n+1}\in Q_M$, if
\begin{itemize}
\item $p\fun a_1\cdots a_n\in P_G$, and

\item for all $i\in[1:n]$, if $a_i\in\alphabet\,G$, then
  $q_{i+1}\in\Delta_M(\{q_i\},a_i)$,
\end{itemize}
then we let
\begin{gather*}
  \langle p,q_1,q_{n+1}\rangle \fun 
  f(a_1,q_1,q_2) \cdots f(a_n,q_n,q_{n+1})
\end{gather*}
be a production of $H$.

The above definition gives us a function
$\inter\in\Gram\times\Gram\fun\Gram$.  For example, let $G$ be the
grammar
\begin{gather*}
  \Asf \fun \% \mid \mathsf{0A1A} \mid \mathsf{1A0A} ,
\end{gather*}
and $M$ be the EFA
\begin{center}
  \input{chap-4.7-fig1.eepic}
\end{center}
so that $G$ generates all elements of $\{\mathsf{0,1}\}^*$
with an equal number of $\zerosf$'s and $\onesf$'s, and $M$
accepts $\{\zerosf\}^*\{\onesf\}^*$.
Then $\simplify\,H$ is
\begin{align*}
\Asf &\fun \langle\Asf,\Asf,\Bsf\rangle , \\
\langle\Asf,\Asf,\Asf\rangle &\fun \% , \\
\langle\Asf,\Asf,\Bsf\rangle &\fun \% , \\
\langle\Asf,\Asf,\Bsf\rangle &\fun
\zerosf\langle\Asf,\Asf,\Asf\rangle\onesf\langle\Asf,\Bsf,\Bsf\rangle , \\
\langle\Asf,\Asf,\Bsf\rangle &\fun
\zerosf\langle\Asf,\Asf,\Bsf\rangle\onesf\langle\Asf,\Bsf,\Bsf\rangle , \\
\langle\Asf,\Asf,\Bsf\rangle &\fun
\zerosf\langle\Asf,\Bsf,\Bsf\rangle\onesf\langle\Asf,\Bsf,\Bsf\rangle , \\
\langle\Asf,\Bsf,\Bsf\rangle &\fun \% .
\end{align*}
Note that simplification eliminated the variable
$\langle\Asf,\Bsf,\Asf\rangle$.
If we hand simplify further, we can turn this into:
\begin{align*}
\Asf &\fun \langle\Asf,\Asf,\Bsf\rangle , \\
\langle\Asf,\Asf,\Bsf\rangle &\fun
\% \mid \zerosf\langle\Asf,\Asf,\Bsf\rangle\onesf
\end{align*}

To prove that our intersection algorithm is correct, we'll need two
lemmas.

\begin{lemma}
\label{GramEFAInter1}
For $p\in Q_G$, let the property $P_p(w)$, for $w\in\Pi_{G,p}$, be:
\begin{quotation}
\noindent
for all $q,r\in Q_M$, if $r\in\Delta_M(\{q\},w)$, then
$w\in\Pi_{H,\langle p, q, r\rangle}$.
\end{quotation}
Then, for all $p\in Q_G$, for all $w\in\Pi_{G,p}$, $P_p(w)$.
\end{lemma}

\begin{proof}
By induction on $\Pi$. In (2), we use the fact that, if
$n\in\nats-\{0\}$, $q_1,q_{n+1}\in Q_M$,
$w_1,\ldots,w_n\in\Str$ and
$q_{n+1}\in\Delta_M(\{q_1\}, w_1\cdots w_n)$, then
there are $q_2,\ldots,q_n\in Q_M$ such that
$q_{i+1}\in\Delta_M(\{q_i\}, w_i)$, for all $i\in[1:n]$. (This is
true because $M$ is an EFA; if $M$ were an FA, we wouldn't be able to
conclude this.)
\end{proof}

\begin{lemma}
\label{GramEFAInter2}
Let the property $P_\Asf(w)$, for $w\in\Pi_{H,\Asf}$, be
\begin{quotation}
\noindent
$w\in L(G)$ and $w\in L(M)$.  
\end{quotation}
For $p\in Q_G$ and $q,r\in Q_M$,
let the property $P_{\langle p,q,r\rangle}(w)$, for
$w\in\Pi_{H,\langle p,q,r\rangle}$, be
\begin{quotation}
\noindent
$w\in\Pi_{G,p}$ and $r\in\Delta_M(\{q\},w)$.
\end{quotation}
Then:
\begin{enumerate}[(1)]
\item For all $w\in\Pi_{H,\Asf}$, $P_\Asf(w)$.

\item For all $p\in Q_G$ and $q,r\in Q_M$, for all
$w\in\Pi_{H,\langle p,q,r\rangle}$, $P_{\langle p,q,r\rangle}(w)$.
\end{enumerate}
\end{lemma}

\begin{proof}
We proceed by induction on $\Pi$.
\end{proof}

\begin{lemma}
$L(H) = L(G)\cap L(M)$.
\end{lemma}

\begin{proof}
$L(H) \sub L(G)\cap L(M)$ follows by Lemma~\ref{GramEFAInter2}(1).

For the other inclusion, suppose $w\in L(G)\cap L(M)$, so that
$w\in\Pi_{G,s_M}$ and $r\in\Delta_M(\{s_m\},w)$, for some $r\in A_M$. By
Lemma~\ref{GramEFAInter1}, it follows that
$w\in\Pi_{H,\langle s_G, s_M, r\rangle}$. But because $r\in A_M$, we
have that $\Asf\fun\langle s_G, s_M, r\rangle$ is a production of $H$.
Thus $w\in\Pi_{H,\Asf}=L(H)$.
\end{proof}

Finally, we consider a difference operation/algorithm.
Given a grammar $G$ and a DFA $M$, we can define the difference of
$G$ and $M$ to be
\begin{gather*}
\inter(G, \mycomplement(M, \alphabet\,G)) .
\end{gather*}
This is analogous to what we did when defining the difference of
DFAs (see Section~\ref{ClosurePropertiesOfRegularLanguages}).
This definition gives us a function
$\minus\in\Gram\times\DFA\fun\Gram$.

The following theorem summarizes the closure properties for context-free
languages.

\begin{theorem}
Suppose $L,L_1,L_2\in\CFLan$ and $L'\in\RegLan$.
Then:
\begin{enumerate}[\quad(1)]
\item $L_1\cup L_2\in\CFLan$;

\item $L_1L_2\in\CFLan$;

\item $L^*\in\CFLan$;

\item $L^R\in\CFLan$;

\item $L^f\in\CFLan$, where $f$ is a bijection from a set of
symbols that is a superset of $\alphabet\,L$ to some
set of symbols;

\item $L^P\in\CFLan$;

\item $L^S\in\CFLan$;

\item $L^\SSop\in\CFLan$;

\item $L\cap L'\in\CFLan$; and

\item $L-L'\in\CFLan$.
\end{enumerate}
\end{theorem}

\subsection{Operations on Grammars in Forlan}

The Forlan module \texttt{Gram} defines the following constants
and operations on grammars:
\begin{verbatim}
val emptyStr       : gram
val emptySet       : gram
val fromStr        : str -> gram
val fromSym        : sym -> gram
val union          : gram * gram -> gram
val concat         : gram * gram -> gram
val closure        : gram -> gram
val rev            : gram -> gram
val renameAlphabet : gram * sym_rel -> gram
val prefix         : gram -> gram
val inter          : gram * efa -> gram
val minus          : gram * dfa -> gram
\end{verbatim}
\index{Gram@\texttt{Gram}!emptyStr@\texttt{emptyStr}}%
\index{Gram@\texttt{Gram}!emptySet@\texttt{emptySet}}%
\index{Gram@\texttt{Gram}!fromStr@\texttt{fromStr}}%
\index{Gram@\texttt{Gram}!fromSym@\texttt{fromSym}}%
\index{Gram@\texttt{Gram}!union@\texttt{union}}%
\index{Gram@\texttt{Gram}!concat@\texttt{concat}}%
\index{Gram@\texttt{Gram}!closure@\texttt{closure}}%
\index{Gram@\texttt{Gram}!rev@\texttt{rev}}%
\index{Gram@\texttt{Gram}!renameAlphabet@\texttt{renameAlphabet}}%
\index{Gram@\texttt{Gram}!prefix@\texttt{prefix}}%
\index{Gram@\texttt{Gram}!inter@\texttt{inter}}%
\index{Gram@\texttt{Gram}!minus@\texttt{minus}}%
Most of these functions implement the mathematical functions with the
same names. The function \texttt{renameAlphabet} raises an exception
if its second argument isn't a bijection whose domain is a superset of
the alphabet of its first argument.  The functions \texttt{fromStr}
and \texttt{fromSym} correspond to $\strToGram$ and $\symToGram$,
and are also available in the top-level environment with those names
\begin{verbatim}
val strToGram : str -> gram
val symToGram : sym -> gram
\end{verbatim}
\index{strToGram@\texttt{strToGram}}%
\index{symToGram@\texttt{symToGram}}%

For example, we can construct a grammar $G$ such that
$L(G)=\mathsf{\{01\}\cup\{10\}\{11\}^*}$, as follows.
\begin{list}{}
{\setlength{\leftmargin}{\leftmargini}
\setlength{\rightmargin}{0cm}
\setlength{\itemindent}{0cm}
\setlength{\listparindent}{0cm}
\setlength{\itemsep}{0cm}
\setlength{\parsep}{0cm}
\setlength{\labelsep}{0cm}
\setlength{\labelwidth}{0cm}
\catcode`\#=12
\catcode`\$=12
\catcode`\%=12
\catcode`\^=12
\catcode`\_=12
\catcode`\.=12
\catcode`\?=12
\catcode`\!=12
\catcode`\&=12
\ttfamily}
\small
\item[]\textsl{-\ }val\ gram1\ =\ strToGram(Str.fromString\ "01");
\item[]\textsl{val\ gram1\ =\ -\ :\ gram}
\item[]\textsl{-\ }val\ gram2\ =\ strToGram(Str.fromString\ "10");
\item[]\textsl{val\ gram2\ =\ -\ :\ gram}
\item[]\textsl{-\ }val\ gram3\ =\ strToGram(Str.fromString\ "11");
\item[]\textsl{val\ gram3\ =\ -\ :\ gram}
\item[]\textsl{-\ }val\ gram\ =
\item[]\textsl{=\ }Gram.union(gram1,
\item[]\textsl{=\ }\ \ \ \ \ \ \ \ \ \ \ Gram.concat(gram2,
\item[]\textsl{=\ }\ \ \ \ \ \ \ \ \ \ \ \ \ \ \ \ \ \ \ \ \ \ \ Gram.closure\ gram3));
\item[]\textsl{val\ gram\ =\ -\ :\ gram}
\item[]\textsl{-\ }Gram.output("",\ gram);
\item[]\textsl{\symbol{'173}variables\symbol{'175}\ A,\ <1,A>,\ <2,A>,\ <2,<1,A>>,\ <2,<2,A>>,\ <2,<2,<A>>>}
\item[]\textsl{\symbol{'173}start\ variable\symbol{'175}\ A}
\item[]\textsl{\symbol{'173}productions\symbol{'175}}
\item[]\textsl{A\ ->\ <1,A>\ |\ <2,A>;\ <1,A>\ ->\ 01;\ <2,A>\ ->\ <2,<1,A>><2,<2,A>>;}
\item[]\textsl{<2,<1,A>>\ ->\ 10;\ <2,<2,A>>\ ->\ %\ |\ <2,<2,<A>>><2,<2,A>>;}
\item[]\textsl{<2,<2,<A>>>\ ->\ 11}
\item[]\textsl{val\ it\ =\ ()\ :\ unit}
\item[]\textsl{-\ }val\ gram'\ =\ Gram.renameVariablesCanonically\ gram;
\item[]\textsl{val\ gram'\ =\ -\ :\ gram}
\item[]\textsl{-\ }Gram.output("",\ gram');
\item[]\textsl{\symbol{'173}variables\symbol{'175}\ A,\ B,\ C,\ D,\ E,\ F\ \symbol{'173}start\ variable\symbol{'175}\ A}
\item[]\textsl{\symbol{'173}productions\symbol{'175}}
\item[]\textsl{A\ ->\ B\ |\ C;\ B\ ->\ 01;\ C\ ->\ DE;\ D\ ->\ 10;\ E\ ->\ %\ |\ FE;\ F\ ->\ 11}
\item[]\textsl{val\ it\ =\ ()\ :\ unit}
\end{list}

Continuing our Forlan session, the grammar reversal and alphabet
renaming operations can be used as follows:
\begin{list}{}
{\setlength{\leftmargin}{\leftmargini}
\setlength{\rightmargin}{0cm}
\setlength{\itemindent}{0cm}
\setlength{\listparindent}{0cm}
\setlength{\itemsep}{0cm}
\setlength{\parsep}{0cm}
\setlength{\labelsep}{0cm}
\setlength{\labelwidth}{0cm}
\catcode`\#=12
\catcode`\$=12
\catcode`\%=12
\catcode`\^=12
\catcode`\_=12
\catcode`\.=12
\catcode`\?=12
\catcode`\!=12
\catcode`\&=12
\ttfamily}
\small
\item[]\textsl{-\ }val\ gram''\ =\ Gram.rev\ gram';
\item[]\textsl{val\ gram''\ =\ -\ :\ gram}
\item[]\textsl{-\ }Gram.output("",\ gram'');
\item[]\textsl{\symbol{'173}variables\symbol{'175}\ A,\ B,\ C,\ D,\ E,\ F\ \symbol{'173}start\ variable\symbol{'175}\ A}
\item[]\textsl{\symbol{'173}productions\symbol{'175}}
\item[]\textsl{A\ ->\ B\ |\ C;\ B\ ->\ 10;\ C\ ->\ ED;\ D\ ->\ 01;\ E\ ->\ %\ |\ EF;\ F\ ->\ 11}
\item[]\textsl{val\ it\ =\ ()\ :\ unit}
\item[]\textsl{-\ }val\ rel\ =\ SymRel.fromString\ "(0,\ A),\ (1,\ B)";
\item[]\textsl{val\ rel\ =\ -\ :\ sym_rel}
\item[]\textsl{-\ }val\ gram'''\ =\ Gram.renameAlphabet(gram'',\ rel);
\item[]\textsl{val\ gram'''\ =\ -\ :\ gram}
\item[]\textsl{-\ }Gram.output("",\ gram''');
\item[]\textsl{\symbol{'173}variables\symbol{'175}\ <A>,\ <B>,\ <C>,\ <D>,\ <E>,\ <F>\ \symbol{'173}start\ variable\symbol{'175}\ <A>}
\item[]\textsl{\symbol{'173}productions\symbol{'175}}
\item[]\textsl{<A>\ ->\ <B>\ |\ <C>;\ <B>\ ->\ BA;\ <C>\ ->\ <E><D>;\ <D>\ ->\ AB;}
\item[]\textsl{<E>\ ->\ %\ |\ <E><F>;\ <F>\ ->\ BB}
\item[]\textsl{val\ it\ =\ ()\ :\ unit}
\end{list}

And here is an example use of the prefix-closure operation:
\begin{list}{}
{\setlength{\leftmargin}{\leftmargini}
\setlength{\rightmargin}{0cm}
\setlength{\itemindent}{0cm}
\setlength{\listparindent}{0cm}
\setlength{\itemsep}{0cm}
\setlength{\parsep}{0cm}
\setlength{\labelsep}{0cm}
\setlength{\labelwidth}{0cm}
\catcode`\#=12
\catcode`\$=12
\catcode`\%=12
\catcode`\^=12
\catcode`\_=12
\catcode`\.=12
\catcode`\?=12
\catcode`\!=12
\catcode`\&=12
\ttfamily}
\small
\item[]\textsl{-\ }val\ gram\ =\ Gram.input\ "";
\item[]\textsl{@\ }\symbol{'173}variables\symbol{'175}
\item[]\textsl{@\ }A
\item[]\textsl{@\ }\symbol{'173}start\ variable\symbol{'175}
\item[]\textsl{@\ }A
\item[]\textsl{@\ }\symbol{'173}productions\symbol{'175}
\item[]\textsl{@\ }A\ ->\ %\ |\ 0A1
\item[]\textsl{@\ }.
\item[]\textsl{val\ gram\ =\ -\ :\ gram}
\item[]\textsl{-\ }val\ gram'\ =\ Gram.prefix\ gram;
\item[]\textsl{val\ gram'\ =\ -\ :\ gram}
\item[]\textsl{-\ }Gram.output("",\ gram');
\item[]\textsl{\symbol{'173}variables\symbol{'175}\ <1,A>,\ <2,0>,\ <2,1>,\ <2,A>\ \symbol{'173}start\ variable\symbol{'175}\ <2,A>}
\item[]\textsl{\symbol{'173}productions\symbol{'175}}
\item[]\textsl{<1,A>\ ->\ %\ |\ 0<1,A>1;\ <2,0>\ ->\ %\ |\ 0;\ <2,1>\ ->\ %\ |\ 1;}
\item[]\textsl{<2,A>\ ->\ %\ |\ <2,0>\ |\ 0<2,A>\ |\ 0<1,A><2,1>}
\item[]\textsl{val\ it\ =\ ()\ :\ unit}
\item[]\textsl{-\ }fun\ test\ s\ =\ Gram.generated\ gram'\ (Str.fromString\ s);
\item[]\textsl{val\ test\ =\ fn\ :\ string\ ->\ bool}
\item[]\textsl{-\ }test\ "000111";
\item[]\textsl{val\ it\ =\ true\ :\ bool}
\item[]\textsl{-\ }test\ "0001";
\item[]\textsl{val\ it\ =\ true\ :\ bool}
\item[]\textsl{-\ }test\ "0001111";
\item[]\textsl{val\ it\ =\ false\ :\ bool}
\end{list}


Finally, to see how we can use \texttt{Gram.inter} and
\texttt{Gram.minus}, let \texttt{gram} be the grammar
\begin{gather*}
  \Asf \fun \% \mid \mathsf{0A1A} \mid \mathsf{1A0A} ,
\end{gather*}
and \texttt{efa} be the EFA
\begin{center}
  \input{chap-4.7-fig1.eepic}
\end{center}
\begin{list}{}
{\setlength{\leftmargin}{\leftmargini}
\setlength{\rightmargin}{0cm}
\setlength{\itemindent}{0cm}
\setlength{\listparindent}{0cm}
\setlength{\itemsep}{0cm}
\setlength{\parsep}{0cm}
\setlength{\labelsep}{0cm}
\setlength{\labelwidth}{0cm}
\catcode`\#=12
\catcode`\$=12
\catcode`\%=12
\catcode`\^=12
\catcode`\_=12
\catcode`\.=12
\catcode`\?=12
\catcode`\!=12
\catcode`\&=12
\ttfamily}
\small
\item[]\textsl{-\ }val\ gram'\ =\ Gram.inter(gram,\ efa);
\item[]\textsl{val\ gram'\ =\ -\ :\ gram}
\item[]\textsl{-\ }Gram.output("",\ gram');
\item[]\textsl{\symbol{'173}variables\symbol{'175}\ A,\ <A,A,A>,\ <A,A,B>,\ <A,B,B>\ \symbol{'173}start\ variable\symbol{'175}\ A}
\item[]\textsl{\symbol{'173}productions\symbol{'175}}
\item[]\textsl{A\ ->\ <A,A,B>;\ <A,A,A>\ ->\ %;}
\item[]\textsl{<A,A,B>\ ->}
\item[]\textsl
\item[]\textsl{val\ it\ =\ ()\ :\ unit}
\item[]\textsl{-\ }fun\ elimVars(gram,\ nil)\ \ \ \ \ =\ gram
\item[]\textsl{=\ }\ \ |\ elimVars(gram,\ q\ ::\ qs)\ =
\item[]\textsl{=\ }\ \ \ \ \ \ elimVars(Gram.eliminateVariable(gram,\ q),\ qs);
\item[]\textsl{val\ elimVars\ =\ fn\ :\ gram\ \symbol{'052}\ sym\ list\ ->\ gram}
\item[]\textsl{-\ }val\ gram''\ =
\item[]\textsl{=\ }\ \ \ \ \ \ elimVars
\item[]\textsl{=\ }\ \ \ \ \ \ (gram',
\item[]\textsl{=\ }\ \ \ \ \ \ \ \symbol{'133}Sym.fromString\ "<A,A,A>",
\item[]\textsl{=\ }\ \ \ \ \ \ \ \ Sym.fromString\ "<A,B,B>"\symbol{'135});
\item[]\textsl{val\ gram''\ =\ -\ :\ gram}
\item[]\textsl{-\ }val\ gram'''\ =
\item[]\textsl{=\ }\ \ \ \ \ \ Gram.renameVariablesCanonically
\item[]\textsl{=\ }\ \ \ \ \ \ (Gram.restart(Gram.simplify\ gram''));
\item[]\textsl{val\ gram'''\ =\ -\ :\ gram}
\item[]\textsl{-\ }Gram.output("",\ gram''');
\item[]\textsl{\symbol{'173}variables\symbol{'175}\ A\ \symbol{'173}start\ variable\symbol{'175}\ A\ \symbol{'173}productions\symbol{'175}\ A\ ->\ %\ |\ 0A1}
\item[]\textsl{val\ it\ =\ ()\ :\ unit}
\item[]\textsl{-\ }Gram.generated\ gram'''\ (Str.fromString\ "0011");
\item[]\textsl{val\ it\ =\ true\ :\ bool}
\item[]\textsl{-\ }Gram.generated\ gram'''\ (Str.fromString\ "0101");
\item[]\textsl{val\ it\ =\ false\ :\ bool}
\item[]\textsl{-\ }Gram.generated\ gram'''\ (Str.fromString\ "0001");
\item[]\textsl{val\ it\ =\ false\ :\ bool}
\item[]\textsl{-\ }val\ dfa\ =
\item[]\textsl{=\ }\ \ \ \ \ \ DFA.renameStatesCanonically
\item[]\textsl{=\ }\ \ \ \ \ \ (DFA.minimize(nfaToDFA(efaToNFA\ efa)));
\item[]\textsl{val\ dfa\ =\ -\ :\ dfa}
\item[]\textsl{-\ }val\ gram''\ =\ Gram.minus(gram,\ dfa);
\item[]\textsl{val\ gram''\ =\ -\ :\ gram}
\item[]\textsl{-\ }Gram.generated\ gram''\ (Str.fromString\ "0101");
\item[]\textsl{val\ it\ =\ true\ :\ bool}
\item[]\textsl{-\ }Gram.generated\ gram''\ (Str.fromString\ "0011");
\item[]\textsl{val\ it\ =\ false\ :\ bool}
\end{list}


\index{context-free language!closure properties|)}%

\subsection{Notes}

The algorithm for intersecting a grammar with an EFA would normally be given
only indirectly, using push down automata (PDAs): one could convert a grammar
to a PDF, do the intersection there, and convert back to a grammar.  Our direct
algorithm is motivated by this process, but produces grammars that are more
intelligible.

%%% Local Variables: 
%%% mode: latex
%%% TeX-master: "book"
%%% End: 
