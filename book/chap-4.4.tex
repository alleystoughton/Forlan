\section{Simplification of Grammars}
\label{SimplificationOfGrammars}

\index{simplification!grammar|(}%
\index{grammar!simplification|(}%

In this section, we say what it means for a grammar to be simplified,
give a simplification algorithm for grammars, and see how to use this
algorithm in Forlan.

\subsection{Definition and Algorithm}

Suppose $G$ is the grammar
\begin{align*}
\Asf &\fun \mathsf{BB1}, \\
\Bsf &\fun \zerosf\mid\Asf\mid\Csf\Dsf, \\
\Csf &\fun \onesf\twosf , \\
\Dsf &\fun \onesf\Dsf\twosf .
\end{align*}
This grammar is odd for two, related, reasons.  First
$\Dsf$ doesn't generate anything, i.e., there is no parse
tree $\pt$ such that $\pt$ is valid for $G$, the root label of
$\pt$ is $\Dsf$, and the yield of $\pt$ is in $(\alphabet\,G)^* =
\{\mathsf{0,1,2}\}^*$.
Second, there is no valid parse tree that starts at $G$'s
start variable, $\Asf$, has a yield that is in $\{\mathsf{0,1,2}\}^*$
and makes use of $\Csf$.  But if we first removed $\Dsf$, and all productions
involving it, then our objection to $\Csf$ could be simpler: there
would be no parse tree $\pt$ such that $\pt$ is valid for $G$,
the root label of $\pt$ is $\Asf$, and $\Csf$ appears in the yield
of $\pt$.

This example leads us to the following definitions.  Suppose $G$ is a
grammar.  We say that a variable $q$ of $G$ is:
\index{grammar!reachable variable}%
\index{grammar!generating variable}%
\index{grammar!useful variable}%
\begin{itemize}
\item \emph{reachable in} $G$ iff there is a $w\in\Str$ such that
  $w$ is parsable from $s_G$ using $G$, and $a\in\alphabet\,w$;

\item \emph{generating in} $G$ iff there is a $w\in\Str$ such that $q$
  generates $w$ using $G$, i.e., $w$ is parsable from $q$ using $G$,
  and $w\in(\alphabet\,G)^*$;

\item \emph{useful in} $G$ iff $q$ is both reachable and
  generating in $G$.
\end{itemize}
The reader should compare these definitions with the definitions given
in Section~\ref{SimplificationOfFiniteAutomata} of reachable, live and
useful states.

Also note that the standard definition of being useful is stronger
than our definition: there is a parse tree $\pt$ such that $\pt$ is
valid for $G$, $\rootLabel\,\pt=s_G$, $\yield\,\pt\in
(\alphabet\,G)^*$, and $q$ appears in $\pt$.  For example, the
variable $\Csf$ of our example grammar $G$ is useful in our sense, but
not useful in the standard sense.  But as we observed above, $\Csf$
will no longer be reachable (and thus useful) if all productions
involving $\Dsf$ are removed.  In general, we have that, if all
variables of a grammar are useful in our sense, that all variable of
the grammar are useful in the standard sense.

Now, suppose $H$ is the grammar
\begin{align*}
\Asf &\fun \% \mid \zerosf \mid \mathsf{AA} \mid \mathsf{AAA} .
\end{align*}
Here, we have that the productions $\Asf\fun\mathsf{AA}$ and
$\Asf\fun\mathsf{AAA}$ are redundant, although only one of them can be
removed:
\begin{center}
  \input{chap-4.4-fig1.eepic}
\end{center}
Thus any use of $\Asf\fun\mathsf{AA}$ in a parse tree can be replaced
by uses of $\Asf\fun\%$ and $\Asf\fun\mathsf{AAA}$, and any use of
$\Asf\fun\mathsf{AAA}$ in a parse tree can be replaced by two uses
of $\Asf\fun\mathsf{AA}$.

This example leads us to the following definitions.  Given a grammar
$G$ and a finite subset $U$ of $\setof{(q,x)}{q\in Q_G\eqtxt{and}
  x\in\Str}$, we write $G/U$ for the grammar that is identical to $G$
except that its set of productions is $U$.
If $G$ is a grammar and $(q,x)\in P_G$, we say that:
\begin{itemize}
\item $(q,x)$ \emph{is redundant in} $G$ iff $x$ is parsable from $q$
\index{grammar!redundant production}%
 using $H$, where $H=G/(P_G - \{(q,x)\})$; and

\item $(q,x)$ \emph{is irredundant in} $G$ iff $(q,x)$ is not
\index{grammar!irredundant production}%
  redundant in $G$.
\end{itemize}

Now we are able to say when a grammar is simplified.  The reader
should compare this definition with the definition in
Section~\ref{SimplificationOfFiniteAutomata} of when a finite
automaton is simplified.
\index{grammar!simplified}%
A grammar $G$ is \emph{simplified} iff either
\begin{itemize}
\item every variable of $G$ is useful, and every production of $G$
  is irredundant; or

\item $|Q_G|=1$ and $P_G = \emptyset$.
\end{itemize}
The second case is necessary, because otherwise there would be
no simplified grammar generating $\emptyset$.

\begin{proposition}
If $G$ is a simplified grammar, then $\alphabet\,G = \alphabet(L(G))$.
\end{proposition}

\begin{proof}
Suppose $a\in\alphabet\,G$.  We must show that $a\in\alphabet\,w$ for
some $w\in L(G)$.  We have that every variable of $G$ is useful, and
there are $q\in Q_G$ and $x\in\Str$ such that $(q,x)\in P_G$ and
$a\in\alphabet\,x$.  Thus $x$ is parsable from $q$.  Since every
variable occurring in $x$ is generating, we have that $q$ generates a
string $x'$ containing $a$.  Since $q$ is reachable, there is a string
$y$ such that $y$ is parsable from $s_G$, and $q\in\alphabet\,y$.
Since every variable occurring in $y$ is generating, there is a string
$y'$ such that $y'$ is parsable from $s_G$, and $q$ is the only
variable of $\alphabet\,y'$.  Putting these facts together, we have
that $s_G$ generates a string $w$ such that $a\in\alphabet\,w$, i.e.,
$a\in\alphabet\,w$ for some $w\in L(G)$.
\end{proof}

Next, we give an algorithm for removing redundant productions.
Given a grammar $G$, $q\in Q_G$ and $x\in\Str$, we say that
\index{grammar!implicit production}%
$(q,x)$ \emph{is implicit in} $G$ iff $x$ is parsable from $q$ using
$G$.

Given a grammar $G$, we define a function
$\remRedun_G\in\powset\,P_G\times\powset\,P_G\fun\powset\,P_G$ by
well-founded recursion on the size of its second argument.
For $U,V\sub P_G$, $\remRedun(U, V)$ proceeds as follows:
\begin{itemize}
\item If $V=\emptyset$, then it returns $U$.

\item Otherwise, let $v$ be the greatest element of $\setof{(q,x)\in
    V}{\eqtxtr{there are no} p\in\Sym \eqtxt{and} y\in\Str \eqtxt{such
      that} (p,y)\in V \eqtxt{and} |y| > |x|}$, and $V' = V - \{v\}$.
  If $v$ is implicit in $G/(U\cup V')$, then $\remRedun$ returns the
  result of evaluating $\remRedun(U, V')$.  Otherwise, it returns the
  result of evaluating $\remRedun(U \cup \{v\}, V')$.
\end{itemize}

In general, there are multiple---incompatible---ways of removing
redundant productions from a grammar.  $\remRedun$ is defined so as to
favor removing productions whose right-hand sides are longer; and
among productions whose right-hand sides have equal length, to favor
removing productions that are larger in our total ordering on
productions.

Our algorithm for removing redundant productions of a grammar $G$
returns $G/(\remRedun_G(\emptyset,P_G))$.

For example, if we run our algorithm for removing redundant productions
on
\begin{align*}
\Asf &\fun \% \mid \zerosf \mid \mathsf{AA} \mid \mathsf{AAA} ,
\end{align*}
we obtain
\begin{align*}
\Asf &\fun \% \mid \zerosf \mid \mathsf{AA} .
\end{align*}

Our simplification algorithm for grammars proceeds as follows, given
a grammar $G$.
\begin{itemize}
\item First, it determines which variables of $G$ are generating.
If $s_G$ isn't one of these variables, then it returns the
grammar with variable $s_G$ and no productions.

\item Next, it turns $G$ into a grammar $G'$ by deleting all
non-generating variables, and deleting all productions involving such
variables.

\item Then, it determines which variables of $G'$ are reachable.

\item Next, it turns $G'$ into a grammar $G''$ by deleting all
non-reachable variables, and deleting all productions involving such
variables.

\item Finally, it removes redundant productions from $G''$.
\end{itemize}

Suppose $G$, once again, is the grammar
\begin{align*}
\Asf &\fun \mathsf{BB1}, \\
\Bsf &\fun \zerosf\mid\Asf\mid\Csf\Dsf, \\
\Csf &\fun \onesf\twosf , \\
\Dsf &\fun \onesf\Dsf\twosf .
\end{align*}
Here is what happens if we apply our simplification algorithm to $G$.
\begin{itemize}
\item First, we determine which variables are generating. Clearly
    $\Bsf$ and $\Csf$ are.  And, since $\Bsf$ is, it follows that
    $\Asf$ is, because of the production $\Asf\fun\mathsf{BB1}$.  (If
    this production had been $\Asf\fun\mathsf{BD1}$, we wouldn't have
    added $\Asf$ to our set.)

\item Thus, we form $G'$ from $G$ by deleting the variable $\Dsf$, yielding
  the grammar
  \begin{align*}
    \Asf &\fun \mathsf{BB1}, \\
    \Bsf &\fun \zerosf\mid\Asf, \\
    \Csf &\fun \onesf\twosf .
  \end{align*}

\item Next, we determine which variables of $G'$ are reachable.
  Clearly $\Asf$ is, and thus $\Bsf$ is, because of the production
  $\Asf\fun\mathsf{BB1}$.

  Note that, if we carried out the two stages of our simplification
  algorithm in the other order, then $\Csf$ and its production would
  never be deleted.

\item Next, we form $G''$ from $G'$ by deleting the variable $\Csf$,
  yielding the grammar
  \begin{align*}
    \Asf &\fun \mathsf{BB1}, \\
    \Bsf &\fun \zerosf\mid\Asf .
  \end{align*}

\item Finally, we would remove redundant productions from $G''$.
  But $G''$ has no redundant productions, and so we are done.
\end{itemize}

\index{simplify@$\simplify$}%
\index{grammar!simplify@$\simplify$}%
\index{simplification!grammar!simplify@$\simplify$}%
We define a function $\simplify\in\Gram\fun\Gram$ by: for all
$G\in\Gram$, $\simplify\,G$ is the result of running the above
algorithm on $G$.

\begin{theorem}
For all $G\in\Gram$:
\begin{enumerate}[\quad(1)]
\item $\simplify\,G$ is simplified;

\item $\simplify\,G\approx G$; and

\item $\alphabet(\simplify\,G) = \alphabet(L(G)) \sub\alphabet\,G$.
\end{enumerate}
\end{theorem}

Our simplification function/algorithm $\simplify$ gives us an
algorithm for testing whether a grammar is simplified: we apply
$\simplify$ to it, and check that the resulting grammar is equal to
the original one.

\index{grammar!testing that language generated is empty}
\index{testing that language generated by grammar is empty}
Our simplification algorithm gives us an algorithm for testing whether
the language generated by a grammar $G$ is empty. We first simplify $G$,
calling the result $H$. We then test whether $P_H=\emptyset$.  If
the answer is ``yes'', clearly $L(G)=L(H)=\emptyset$. And if the
answer is ``no'', then $s_H$ is useful, and so $H$ (and thus $G$) generates
at least one string.

\subsection{Simplification in Forlan}

The Forlan module \texttt{Gram} defines the functions
\begin{verbatim}
val simplify   : gram -> gram
val simplified : gram -> bool
\end{verbatim}
\index{Gram@\texttt{Gram}!simplify@\texttt{simplify}}%
\index{Gram@\texttt{Gram}!simplified@\texttt{simplified}}%
The function \texttt{simplify} corresponds to $\simplify$, and
\texttt{simplified} tests whether a grammar is simplified.

Suppose \texttt{gram} of type \texttt{gram} is bound to the grammar
\begin{align*}
\Asf &\fun \mathsf{BB1}, \\
\Bsf &\fun \zerosf\mid\Asf\mid\Csf\Dsf, \\
\Csf &\fun \onesf\twosf , \\
\Dsf &\fun \onesf\Dsf\twosf .
\end{align*}
We can simplify our grammar as follows:
\begin{list}{}
{\setlength{\leftmargin}{\leftmargini}
\setlength{\rightmargin}{0cm}
\setlength{\itemindent}{0cm}
\setlength{\listparindent}{0cm}
\setlength{\itemsep}{0cm}
\setlength{\parsep}{0cm}
\setlength{\labelsep}{0cm}
\setlength{\labelwidth}{0cm}
\catcode`\#=12
\catcode`\$=12
\catcode`\%=12
\catcode`\^=12
\catcode`\_=12
\catcode`\.=12
\catcode`\?=12
\catcode`\!=12
\catcode`\&=12
\ttfamily}
\small
\item[]\textsl{-\ }val\ gram'\ =\ Gram.simplify\ gram;
\item[]\textsl{val\ gram'\ =\ -\ :\ gram}
\item[]\textsl{-\ }Gram.output("",\ gram');
\item[]\textsl{\symbol{'173}variables\symbol{'175}\ A,\ B\ \symbol{'173}start\ variable\symbol{'175}\ A}
\item[]\textsl{\symbol{'173}productions\symbol{'175}\ A\ ->\ BB1;\ B\ ->\ 0\ |\ A}
\item[]\textsl{val\ it\ =\ ()\ :\ unit}
\end{list}

And, Suppose \texttt{gram''} of type \texttt{gram} is bound to the grammar
\begin{align*}
\Asf \fun \% \mid \zerosf \mid \mathsf{AA} \mid \mathsf{AAA} \mid
\mathsf{AAAA} .
\end{align*}
We can simplify our grammar as follows:
\begin{list}{}
{\setlength{\leftmargin}{\leftmargini}
\setlength{\rightmargin}{0cm}
\setlength{\itemindent}{0cm}
\setlength{\listparindent}{0cm}
\setlength{\itemsep}{0cm}
\setlength{\parsep}{0cm}
\setlength{\labelsep}{0cm}
\setlength{\labelwidth}{0cm}
\catcode`\#=12
\catcode`\$=12
\catcode`\%=12
\catcode`\^=12
\catcode`\_=12
\catcode`\.=12
\catcode`\?=12
\catcode`\!=12
\catcode`\&=12
\ttfamily}
\small
\item[]\textsl{-\ }val\ gram'''\ =\ Gram.simplify\ gram'';
\item[]\textsl{val\ gram'''\ =\ -\ :\ gram}
\item[]\textsl{-\ }Gram.output("",\ gram''');
\item[]\textsl{\symbol{'173}variables\symbol{'175}\ A\ \symbol{'173}start\ variable\symbol{'175}\ A\ \symbol{'173}productions\symbol{'175}\ A\ ->\ %\ |\ 0\ |\ AA}
\item[]\textsl{val\ it\ =\ ()\ :\ unit}
\end{list}


\subsection{Hand-simplification Operations}

\index{grammar!simplification!hand}%
\index{grammar!hand simplification}%
Given a simplified grammar $G$, there are often ways we can hand-simplify
the grammar further. Below are two examples:
\begin{itemize}
\item Suppose $G$ has a variable $q$ that is not $s_G$, and where no
  production having $q$ as its left-hand side is
  \emph{self-recursive}, i.e., has $q$ as one of the symbols of its
  right-hand side. Let $x_1,\ldots,x_n$ be the right-hand sides of all
  of $q$'s productions. ($n\geq 1$, as $G$ is simplified.)  Then we
  can form an equivalent grammar $G'$ by deleting $q$ and its
  productions from $G$, and transforming each remaining production
  $p\fun y$ of $G$ into all the productions from $p$ that can be
  formed by substituting for each occurrence of $q$ in $y$ some choice
  of $x_i$.

  We refer to this operation as \emph{eliminating} $q$ \emph{from}
\index{grammar!variable elimination}%
  $G$.

\item Suppose there is exactly one production of $G$ involving $s_G$,
  where that production has the form $s_G\fun q$, for some variable
  $q$ of $G$. Then we can form an equivalent grammar $G'$ by deleting
  $s_G$ and $s_G\fun q$ from $G$, and making $q$ be the start variable
  of $G'$.

  We refer to this operation as \emph{restarting} $G$.
\index{grammar!restarting}%
\end{itemize}

The Forlan module \texttt{Gram} has functions corresponding to these
two operations, first simplifying the supplied grammar:
\begin{verbatim}
val eliminateVariable : gram * sym -> gram
val restart           : gram -> gram
\end{verbatim}
\index{Gram@\texttt{Gram}!eliminateVariable@\texttt{eliminateVariable}}%
\index{Gram@\texttt{Gram}!restart@\texttt{restart}}%
Both begin by simplifying the supplied grammar.

For instance, suppose $\texttt{gram}$ is the grammar
\begin{align*}
\Asf &\fun \mathsf{B}, \\
\Bsf &\fun \zerosf \mid \Csf\threesf\Csf , \\
\Csf &\fun \onesf\Bsf\twosf \mid \twosf\Bsf\onesf .
\end{align*}

Then we can proceed as follows:
\begin{list}{}
{\setlength{\leftmargin}{\leftmargini}
\setlength{\rightmargin}{0cm}
\setlength{\itemindent}{0cm}
\setlength{\listparindent}{0cm}
\setlength{\itemsep}{0cm}
\setlength{\parsep}{0cm}
\setlength{\labelsep}{0cm}
\setlength{\labelwidth}{0cm}
\catcode`\#=12
\catcode`\$=12
\catcode`\%=12
\catcode`\^=12
\catcode`\_=12
\catcode`\.=12
\catcode`\?=12
\catcode`\!=12
\catcode`\&=12
\ttfamily}
\small
\item[]\textsl{-\ }val\ gram'\ =\ Gram.eliminateVariable(gram,\ Sym.fromString\ "C");
\item[]\textsl{val\ gram'\ =\ -\ :\ gram}
\item[]\textsl{-\ }Gram.output("",\ gram');
\item[]\textsl{\symbol{'173}variables\symbol{'175}\ A,\ B\ \symbol{'173}start\ variable\symbol{'175}\ A}
\item[]\textsl{\symbol{'173}productions\symbol{'175}}
\item[]\textsl{A\ ->\ B;\ B\ ->\ 0\ |\ 1B231B2\ |\ 1B232B1\ |\ 2B131B2\ |\ 2B132B1}
\item[]\textsl{val\ it\ =\ ()\ :\ unit}
\item[]\textsl{-\ }val\ gram''\ =\ Gram.restart\ gram;
\item[]\textsl{val\ gram''\ =\ -\ :\ gram}
\item[]\textsl{-\ }Gram.output("",\ gram'');
\item[]\textsl{\symbol{'173}variables\symbol{'175}\ B,\ C\ \symbol{'173}start\ variable\symbol{'175}\ B}
\item[]\textsl{\symbol{'173}productions\symbol{'175}\ B\ ->\ 0\ |\ C3C;\ C\ ->\ 1B2\ |\ 2B1}
\item[]\textsl{val\ it\ =\ ()\ :\ unit}
\end{list}


\subsection{Notes}

As described above, our definition of useless variable is weaker than
the standard one.  However, whenever every variable of a grammar is
useful in our sense, it follows that every variable of the grammar is
useful in the standard sense.  Furthermore, our algorithm for removing
useless variables is the standard one.  Requiring that simplified
grammars have no redundant productions is natural, although
non-standard, and our algorithm for removing redundant productions is
straightforward. The algorithms for restarting and eliminating
variables from grammars are non-standard but obvious.

\index{simplification!grammar|)}%
\index{grammar!simplification|)}%

%%% Local Variables: 
%%% mode: latex
%%% TeX-master: "book"
%%% End: 
