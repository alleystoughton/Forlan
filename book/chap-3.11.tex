\section{Deterministic Finite Automata}
\label{DeterministicFiniteAutomata}

\index{finite automaton!deterministic|(}%
\index{deterministic finite automaton|(}%
\index{DFA|(}%

In this section, we study the third of our more restricted kinds of
finite automata: deterministic finite automata.

\subsection{Definition of DFAs}

A \emph{deterministic finite automaton} (DFA) $M$ is a finite
automaton such that:
\begin{itemize}
\item $T_M\sub\setof{q,x\fun r}{q,r\in\Sym\eqtxt{and}x\in\Str\eqtxt{and}|x|=1}$;
  and

\item for all $q\in Q_M$ and $a\in\alphabet\,M$, there is a unique $r\in Q_M$
such that $q,a\fun r\in T_M$.
\end{itemize}
In other words, an FA is a DFA iff it is an NFA and, for every state
$q$ of the automaton and every symbol $a$ of the automaton's alphabet,
there is exactly one state that can be entered from state $q$ by
\index{finite automaton!DFA@$\DFA$}%
\index{deterministic finite automaton!DFA@$\DFA$}%
reading $a$ from the automaton's input.  We write $\DFA$ for the set
of all deterministic finite automata.  Thus
$\DFA\subsetneq\NFA\subsetneq\EFA\subsetneq\FA$.

Let $M$ be the finite automaton
\begin{center}
\input{chap-3.11-fig1.eepic}
\end{center}
It turns out that $L(M)=
\setof{w\in\{\mathsf{0,1}\}^*}{\mathsf{000}\eqtxt{is not a substring
    of}w}$.  Although $M$ is an NFA, it's not a DFA, since
$\zerosf\in\alphabet\,M$ but there is no transition of the form
$\Csf,\zerosf\fun r$.  However, we can make $M$ into a DFA by adding a dead
state $\Dsf$:
\begin{center}
\input{chap-3.11-fig2.eepic}
\end{center}
We will never need more than one dead state in a DFA.

The following proposition obviously holds.

\begin{proposition}
Suppose $M$ is a DFA.
\begin{itemize}
\item For all $N\in\FA$, if $M\iso N$, then $N$ is a DFA.

\item For all bijections $f$ from $Q_M$ to some set of symbols,
$\renameStates(M, f)$ is a DFA.

\item $\renameStatesCanonically\,M$ is a DFA.
\end{itemize}
\end{proposition}

Now, we prove a proposition that doesn't hold for arbitary NFAs.

\begin{proposition}
\label{DetermProp1}

Suppose $M$ is a DFA.  For all $q\in Q_M$ and $w\in(\alphabet\,M)^*$,
$|\Delta_M(\{q\},w)|=1${.}
\end{proposition}

\begin{proof}
An easy left string induction on $w$.
\end{proof}

\index{deterministic finite automaton!transition function}%
\index{deterministic finite automaton!delta@$\delta$}%
Suppose $M$ is a DFA.  Because of Proposition~\ref{DetermProp1}, we
can define \emph{the transition function} $\delta_M$ \emph{for} $M$,
$\delta_M\in Q_M\times(\alphabet\,M)^*\fun Q_M$, by:
\begin{gather*}
\delta_M(q,w) = \eqtxtr{the unique} r\in Q_M\eqtxt{such that}
r\in\Delta_M(\{q\},w) .
\end{gather*}
In other words, $\delta_M(q,w)$ is the unique state $r$ of $M$ that
is the end of a valid labeled path for $M$ that starts at $q$ and
is labeled by $w$.
Thus, for all $q,r\in Q_M$ and $w\in(\alphabet\,M)^*$,
\begin{gather*}
\delta_M(q,w)=r\quad\myiff\quad r\in\Delta_M(\{q\},w) .
\end{gather*}
We sometimes abbreviate $\delta_M(q,w)$ to $\delta(q,w)$.

For example, if $M$ is the DFA
\begin{center}
\input{chap-3.11-fig2.eepic}
\end{center}
then $\delta(\Asf, \%)={\Asf}$, $\delta(\Asf, \mathsf{0100})={\Csf}$
and $\delta(\Bsf, \mathsf{000100})={\Dsf}$.

Having defined the $\delta$ function, we can study its properties.

\begin{proposition}
\label{DetermProp2}
Suppose $M$ is a DFA.
\begin{enumerate}[\quad(1)]
\item For all $q\in Q_M$, $\delta_M(q,\%)={q}$.

\item For all $q\in Q_M$ and $a\in\alphabet\,M$,
$\delta_M(q,a)=\eqtxtr{the unique}r\in Q_M\eqtxt{such that}
{q,a\fun r\in T_M}$.

\item For all $q\in Q_M$ and $x,y\in(\alphabet\,M)^*$,
$\delta_M(q,xy)={\delta_M(\delta_M(q,x),y)}$.
\end{enumerate}
\end{proposition}

Suppose $M$ is a DFA.  By part~(2) of the proposition, we have that,
for all $q,r\in Q_M$ and $a\in\alphabet\,M$,
\begin{gather*}
\delta_M(q,a)=r\quad\myiff\quad q,a\fun r\in T_M .
\end{gather*}

Now we can use the $\delta$ function to explain when a string is
accepted by an FA.

\begin{proposition}
\label{DetermProp3}
Suppose $M$ is a DFA.
$L(M)=\setof{w\in(\alphabet\,M)^*}{\delta_M(s_M,w)\in A_M}$.
\end{proposition}
\begin{proof}
To see that the left-hand side is a subset of the right-hand side,
suppose $w\in L(M)$.  Then $w\in(\alphabet\,M)^*$ and there is a $q\in A_M$
such that $q\in\Delta_M(\{s_M\},w)$.  Thus $\delta_M(s_M,w)=q\in A_M$.

To see that the right-hand side is a subset of the left-hand side,
suppose $w\in(\alphabet\,M)^*$ and $\delta_M(s_M,w)\in A_M$.  Then
$\delta_M(s_M,w)\in\Delta_M(\{s_M\},w)$, and thus $w\in L(M)$.
\end{proof}

The preceding propositions give us an efficient algorithm for checking
whether a string is accepted by a DFA.  For example, suppose
$M$ is the DFA
\begin{center}
\input{chap-3.11-fig2.eepic}
\end{center}
To check whether $\mathsf{0100}$ is accepted by $M$, we need
to determine whether $\delta(\Asf,\mathsf{0100})\in\{\Asf,\Bsf,\Csf\}$.

For instance, we have that:
\begin{align*}
\delta(\Asf,\mathsf{0100})
&= \delta(\delta(\Asf, \zerosf),\mathsf{100}) \\
&= \delta(\Bsf,\mathsf{100}) \\
&= \delta(\delta(\Bsf,\onesf),\mathsf{00}) \\
&= \delta(\Asf,\mathsf{00}) \\
&= \delta(\delta(\Asf,\zerosf),\zerosf) \\
&= \delta(\Bsf,\zerosf) \\
&= \Csf \\
&\in \{\Asf,\Bsf,\Csf\}.
\end{align*}
Thus $\mathsf{0100}$ is accepted by $M$.

It is easy to see that, for all DFAs $M$ and $q\in Q_M$:
\begin{itemize}
\item $q$ is reachable in $M$ iff there is a $w\in(\alphabet\, M)^*$,
  $\delta_M(s_M, w) = q$;
\index{reachable state}%
\index{deterministic finite automaton!reachable state}%
and

\item $q$ is live in $M$ iff there is a $w\in(\alphabet\, M)^*$,
  $\delta_M(q, w) = r$, for some $r\in A_M$.
\index{live state}%
\index{deterministic finite automaton!live state}%
\end{itemize}

\subsection{Proving the Correctness of DFAs}

\index{deterministic finite automaton!proving correctness}%
Since every DFA is an FA, we could prove the correctness of DFAs
using the techniques that we have already studied.
But it turns out that giving a separate proof that enough is accepted
by a DFA is unnecessary---it will follow from the proof that
everything accepted is wanted.

\begin{proposition}
Suppose $M$ is a DFA.  Then, for all $w\in(\alphabet\,M)^*$ and $q\in Q_M$,
\begin{gather*}
w\in\Lambda_{M,q} \quad\myiff\quad \delta_M(s_M,w) = q .
\end{gather*}
\end{proposition}

\begin{proof}
Suppose $w\in(\alphabet\,M)^*$ and $q\in Q_M$.
\begin{description}
\item[\quad(only if)] Suppose $w\in\Lambda_q$.  Then
  $q\in\Delta(\{s\},w)$.  Thus $\delta(s,w) = q$.

\item[\quad(if)] Suppose $\delta(s,w) = q$.  Then
  $q\in\Delta(\{s\},w)$, so that $w\in\Lambda_q$.
\end{description}
\end{proof}

We already know that, if $M$ is an FA, then
$L(M)=\bigcup\setof{\Lambda_q}{q\in A_M}$.

\begin{proposition}
Suppose $M$ is a DFA.
\begin{enumerate}[\quad(1)]
\item $(\alphabet\,M)^* = \bigcup\setof{\Lambda_q}{q\in Q_M}$.

\item For all $q,r\in Q_M$, if $q\neq r$, then
  $\Lambda_q\cap\Lambda_r=\emptyset$.
\end{enumerate}
\end{proposition}

Suppose $M$ is the DFA
\begin{center}
\input{chap-3.11-fig2.eepic}
\end{center}
and let $X=\setof{w\in\{\mathsf{0,1}\}^*}{\mathsf{000}\eqtxt{is not a
    substring of}w}$.  We will show that $L(M)=X$.  Note that, for all
$w\in\{\mathsf{0,1}\}^*$:
\begin{itemize}
\item $w\in X$ iff $\mathsf{000}$ is not a substring of $w$.

\item $w\not\in X$ iff $\mathsf{000}$ is a substring of $w$.
\end{itemize}

First, we use induction on $\Lambda$, to prove that:
\begin{enumerate}[\quad(A)]
\item for all $w\in\Lambda_\Asf$, $w\in X$ and $\zerosf$ is not a
  suffix of $w$;

\item for all $w\in\Lambda_\Bsf$, $w\in X$ and $\zerosf$, but not
  $\zerosf\zerosf$, is a suffix of $w$;

\item for all $w\in\Lambda_\Csf$, $w\in X$ and $\zerosf\zerosf$ is a
  suffix of $w$; and

\item for all $w\in\Lambda_\Dsf$, $w\not\in X$.
\end{enumerate}

There are nine steps (1 + the number of transitions) to show.
\begin{description}
\item[\quad(empty string)] We must show that  $\%\in X$ and
  $\zerosf$ is not a suffix of $\%$.  This follows since $\%$ has
  no $\zerosf$'s.

\item[\quad($\Asf, \zerosf\fun\Bsf$)] Suppose $w\in\Lambda_\Asf$, and
  assume the inductive hypothesis: $w\in X$ and $\zerosf$ is not a
  suffix of $w$.  We must show that $w\zerosf\in X$ and
  $\zerosf$, but not $\zerosf\zerosf$, is a suffix of
  $w\zerosf$. Because $w\in X$ and $\zerosf$ is not a suffix of $w$,
  we have that $w\zerosf\in X$.  Clearly, $\zerosf$ is a suffix of
  $w\zerosf$.  And, since $\zerosf$ is not a suffix of $w$, we have
  that $\zerosf\zerosf$ is not a suffix of $w\zerosf$.

\item[\quad($\Asf, \onesf\fun\Asf$)] Suppose $w\in\Lambda_\Asf$, and
  assume the inductive hypothesis: $w\in X$ and $\zerosf$ is not a
  suffix of $w$.  We must show that $w\onesf\in X$ and $\zerosf$ is
  not a suffix of $w\onesf$.  Since $w\in X$, we have that $w\onesf\in
  X$.  And, $\zerosf$ is not a suffix of $w\onesf$.

\item[\quad($\Bsf, \zerosf\fun\Csf$)] Suppose $w\in\Lambda_\Bsf$, and
  assume the inductive hypothesis: $w\in X$ and $\zerosf$, but not
  $\zerosf\zerosf$, is a suffix of $w$.  We must show that
  $w\zerosf\in X$ and $\zerosf\zerosf$ is a suffix of $w\zerosf$.
  Because $w\in X$ and $\zerosf\zerosf$ is not suffix of $w$, we have
  that $w\zerosf\in X$.  And since $\zerosf$ is a suffix of $w$, it
  follows that $\zerosf\zerosf$ is a suffix of $w\zerosf$.

\item[\quad($\Bsf, \onesf\fun\Asf$)] Suppose $w\in\Lambda_\Bsf$, and
  assume the inductive hypothesis: $w\in X$ and $\zerosf$, but not
  $\zerosf\zerosf$, is a suffix of $w$.  We must show that $w\onesf\in
  X$ and $\zerosf$ is not a suffix of $w\onesf$.  Because $w\in X$, we
  have that $w\onesf\in X$.  And, $\zerosf$ is not a suffix of
  $w\onesf$.

\item[\quad($\Csf, \zerosf\fun\Dsf$)] Suppose $w\in\Lambda_\Csf$, and
  assume the inductive hypothesis: $w\in X$ and $\zerosf\zerosf$ is a
  suffix of $w$.  We must show that $w\zerosf\not\in X$.  Because
  $\zerosf\zerosf$ is a suffix of $w$, we have that $\mathsf{000}$ is
  a suffix of $w\zerosf$.  Thus $w\zerosf\not\in X$.

\item[\quad($\Csf, \onesf\fun\Asf$)] Suppose $w\in\Lambda_\Csf$, and
  assume the inductive hypothesis: $w\in X$ and $\zerosf\zerosf$ is a
  suffix of $w$.  We must show that $w\onesf\in X$ and $\zerosf$ is
  not a suffix of $w\onesf$.  Because $w\in X$, we have that
  $w\onesf\in X$.  And, $\zerosf$ is not a suffix of $w\onesf$.

\item[\quad($\Dsf, \zerosf\fun\Dsf$)] Suppose $w\in\Lambda_\Dsf$, and
  assume the inductive hypothesis: $w\not\in X$.  We must show that
  $w\zerosf\not\in X$.  Because $w\not\in X$, we have that
  $w\zerosf\not\in X$.

\item[\quad($\Dsf, \onesf\fun\Dsf$)] Suppose $w\in\Lambda_\Dsf$, and
  assume the inductive hypothesis: $w\not\in X$.  We must show that
  $w\onesf\not\in X$.   Because $w\not\in X$, we have that
  $w\onesf\not\in X$.
\end{description}

Now, we use the result of our induction on $\Lambda$ to show that
$L(M)=X$.

\begin{description}
\item[\quad($L(M)\sub X$)] Suppose $w\in L(M)$.  Because
  $A_M=\{\Asf,\Bsf,\Csf\}$, we have that $w\in L(M) = \Lambda_\Asf
  \cup \Lambda_\Bsf \cup \Lambda_\Csf$.  Thus, by parts~(A)--(C), we
  have that $w\in X$.

\item[\quad($X\sub L(M)$)] Suppose $w\in X$.  Since
  $X\sub\{\zerosf,\onesf\}^*$, we have that
  $w\in\{\zerosf,\onesf\}^*$.  Suppose, toward a contradiction, that
  $w\not\in L(M)$.  Because $w\not\in L(M) = \Lambda_\Asf \cup
  \Lambda_\Bsf \cup \Lambda_\Csf$ and $w\in\{\zerosf,\onesf\}^* =
  (\alphabet\,M)^* = \Lambda_\Asf \cup \Lambda_\Bsf \cup \Lambda_\Csf
  \cup \Lambda_\Dsf$, we must have that $w\in\Lambda_\Dsf$.  But then
  part~(D) tells us that $w\not\in X$---contradiction.  Thus $w\in
  L(M)$.
\end{description}

For the above approach to work, when proving $L(M)=X$ for a DFA $M$
and language $X$, we simply need that:
\begin{itemize}
\item the property associated with each accepting state implies
  being in $X$; and

\item the property associated with each non-accepting state implies
  not being in $X$.
\end{itemize}

\begin{exercise}
\label{AllGoodDFACorrLem}
Define $\diff\in\{\mathsf{0,1}\}^*\fun\ints$ by:
\index{diff@$\diff$}%
\index{string!diff@$\diff$}%
\index{difference function}%
\index{string!difference function}%
for all $w\in\{\mathsf{0,1}\}^*$,
\begin{displaymath}
\diff\,w =
\eqtxtr{the number of $\mathsf{1}$'s in}w -
\eqtxtr{the number of $\mathsf{0}$'s in}w .
\end{displaymath}
Define $\AllPrefixGood$ to be the language
$\setof{w\in\{\mathsf{0,1}\}^*}{\eqtxtr{for all prefixes} v\eqtxt{of}
  w, |\diff\,v|\leq 2}$.  Find a DFA $\allPrefixGoodDFA$ such that
$L(\allPrefixGoodDFA) = \AllPrefixGood$, and prove that your solution
is correct.
\end{exercise}

\subsection{Simplification of DFAs}

\index{deterministic finite automaton!simplification}%
\index{simplification!deterministic finite automaton}%
Let $M$ be our example DFA
\begin{center}
\input{chap-3.11-fig2.eepic}
\end{center}
Then $M$ is not simplified, since the state $\Dsf$ is dead.  But if we
get rid of $\Dsf$, then we won't have a DFA anymore.  Thus, we will
need:
\begin{itemize}
\item a notion of when a DFA is simplified that is more liberal than
  our standard notion; and

\item a corresponding simplification procedure for DFAs.
\end{itemize}
\index{deterministically simplified}%
\index{deterministic finite automaton!deterministically simplified}%
We say that a DFA $M$ is \emph{deterministically simplified} iff
\begin{itemize}
\item every element of $Q_M$ is reachable; and

\item at most one element of $Q_M$ is dead.
\end{itemize}
For example, both of the following DFAs, which accept $\emptyset$,
are deterministically simplified:
\begin{center}
\input{chap-3.11-fig3.eepic}
\end{center}

We define a simplification algorithm for DFAs that takes in
\begin{itemize}
\item a DFA $M$ and
\item an alphabet $\Sigma$
\end{itemize}
and returns a DFA $N$ such that
\begin{itemize}
\item $N$ is deterministically simplified,

\item $N\approx M$,

\item $\alphabet\,N = \alphabet(L(M))\cup\Sigma$, and

\item if $\Sigma\sub\alphabet(L(M))$, then $|Q_N| \leq |Q_M|$.
\end{itemize}
Thus, the alphabet of $N$ will consist of all symbols that
either appear in strings that are accepted by $M$ or are in $\Sigma$.

The algorithm begins by letting the FA $M'$ be $\simplify\,M$, i.e.,
the result of running our simplification algorithm for FAs on $M$.
$M'$ will have the following properties:
\begin{itemize}
\item $Q_{M'}\sub Q_M$ and $T_{M'}\sub T_M$;

\item $M'$ is simplified;

\item $M'\approx M$;

\item $\alphabet\,M'=\alphabet(L(M'))=\alphabet(L(M))$; and

\item for all $q\in Q_{M'}$ and $a\in\alphabet\,M'$, there is at most
  one $r\in Q_{M'}$ such that $q,a\fun r\in T_{M'}$ (this property
  holds since $M$ is a DFA, $Q_{M'}\sub Q_M$ and $T_{M'}\sub T_M$).
\end{itemize}

Let $\Sigma'=\alphabet\,M'\cup\Sigma=\alphabet(L(M))\cup\Sigma$.  If
$M'$ is a DFA and $\alphabet\,M'=\Sigma'$, the algorithm returns $M'$
as its DFA, $N$. Because $M'$ is simplified, all states of $M'$ are
reachable, and either $M'$ has no dead states, or it consists
of a single dead state (the start state). In either case, $M'$
is deterministically simplified. Because $Q_{M'}\sub Q_M$, we
have $|Q_N| \leq |Q_M|$.

Otherwise, it must turn $M'$ into a DFA whose alphabet is $\Sigma'$.
We have that
\begin{itemize}
\item $\alphabet\,M'\sub\Sigma'$; and

\item for all $q\in Q_{M'}$ and $a\in\Sigma'$, there is at most one
  $r\in Q_{M'}$ such that $q,a\fun r\in T_{M'}$.
\end{itemize}

Since $M'$ is simplified, there are two cases to consider.  If $M'$
has no accepting states, then $s_{M'}$ is the only state of $M'$ and
$M'$ has no transitions.  Thus the algorithm can return the DFA $N$
defined by:
\begin{itemize}
\item $Q_N=Q_{M'}=\{s_{M'}\}$;

\item $s_N=s_{M'}$;

\item $A_N=A_{M'}=\emptyset$; and

\item $T_N=\setof{s_{M'},a\fun s_{M'}}{a\in\Sigma'}$.
\end{itemize}
In this case, we have that $|Q_N| \leq |Q_M|$.

Alternatively, $M'$ has at least one accepting state, so that $M'$ has
no dead states. (Consider the case when $\Sigma\sub\alphabet(L(M))$,
so that $\Sigma' = \alphabet(L(M)) = \alphabet\,M'$.  Suppose, toward
a contradiction, that $Q_{M'} = Q_M$, so that all elements of $Q_M$
are useful.  Then $s_{M'} = s_M$ and $A_{M'} = A_M$.  And
$T_{M'} = T_{M}$, since no transitions of a DFA are redundant. Hence
$M' = M$, so that $M'$ is a DFA with alphabet $\Sigma'$---a
contradiction. Thus $Q_{M'}\subsetneq Q_M$.)

Thus the DFA $N$ returned by the algorithm is defined by:
\begin{itemize}
\item $Q_N = Q_{M'}\cup\{\dead\}$ (we put enough brackets around
$\dead$ so that it's not in $Q_{M'}$);

\item $s_N=s_{M'}$;

\item $A_N=A_{M'}$; and

\item $T_N=T_{M'}\cup T'$, where $T'$ is the set of all transitions
  $q,a\fun\dead$ such that either
\begin{itemize}
\item $q\in Q_{M'}$ and $a\in\Sigma'$, but there is no $r\in Q_{M'}$
  such that $q,a\fun r\in T_{M'}$; or

\item $q=\dead$ and $a\in\Sigma'$.
\end{itemize}
\end{itemize}
(If $\Sigma\sub\alphabet(L(M))$, then $|Q_N|\leq |Q_M|$.)

\index{deterministic finite automaton!determSimplify@$\determSimplify$}
We define a function $\determSimplify\in\DFA\times\Alp\fun\DFA$
by: $\determSimplify(M,\Sigma)$ is the result of
running the above algorithm on $M$ and $\Sigma$.

\begin{theorem}
For all $M\in\DFA$ and $\Sigma\in\Alp$:
\begin{itemize}
\item $\determSimplify(M,\Sigma)$ is deterministically simplified;

\item $\determSimplify(M,\Sigma)\approx M$;

\item $\alphabet(\determSimplify(M,\Sigma)) = \alphabet(L(M))\cup\Sigma$; and

\item if $\Sigma\sub\alphabet(L(M))$, then 
  $|Q_{\determSimplify(M, \Sigma)}| \leq |Q_M|$.
\end{itemize}
\end{theorem}

For example, suppose $M$ is the DFA
\begin{center}
\input{chap-3.11-fig4.eepic}
\end{center}
Then $\determSimplify(M,\{\twosf\})$ is the DFA
\begin{center}
\input{chap-3.11-fig5.eepic}
\end{center}

\begin{exercise}
Find a DFA $M$ and alphabet $\Sigma$ such that
$\determSimplify(M,\Sigma)$ has one more state than does $M$.
\end{exercise}

\subsection{Converting NFAs to DFAs}

\index{nondeterministic finite automaton!converting NFA to DFA}%
\index{deterministic finite automaton!converting NFA to DFA}%
Suppose $M$ is the NFA
\begin{center}
\input{chap-3.11-fig6.eepic}
\end{center}
Our algorithm for converting NFAs to DFAs will convert $M$ into a DFA
$N$ whose states represent the elements of the set
\begin{gather*}
\setof{\Delta_M(\{\Asf\},w)}{w\in\{\zerosf,\onesf\}^*} .
\end{gather*}
For example, one the states of $N$ will be $\langle\Asf,\Bsf\rangle$,
which represents $\{\Asf,\Bsf\}=\Delta_M(\{\Asf\},\mathsf{1})$.
This is the state that our DFA will be in after processing $\onesf$
from the start state.

Before describing our conversion algorithm, we first state a
proposition concerning the $\Delta$ function for NFAs, and say how we
will represent finite sets of symbols as symbols.

\begin{proposition}
\label{NFADeltaProp}
Suppose $M$ is an NFA.
\begin{enumerate}[\quad(1)]
\item For all $P\sub Q_M$, $\Delta_M(P,\%)=P$.

\item For all $P\sub Q_M$ and $a\in\alphabet\,M$,
  \begin{displaymath}
    \Delta_M(P,a)=\setof{r\in Q_M}{p,a\fun r\in T_M,\eqtxt{for some}p\in P}.
  \end{displaymath}

\item For all $P\sub Q_M$ and $x,y\in(\alphabet\,M)^*$,
  \begin{displaymath}
    \Delta_M(P, xy)=\Delta_M(\Delta_M(P,x),y) .
  \end{displaymath}
\end{enumerate}
\end{proposition}

Given a finite set of symbols $P$, we write $\overline{P}$ for
the symbol
\begin{gather*}
\langle a_1,\,\ldots,a_n\rangle,
\end{gather*}
where $a_1,\,\ldots,a_n$ are all of the elements of $P$, in order
according to our total ordering on $\Sym$, and without repetition.  For
example, $\overline{\{\Bsf,\Asf\}}=\langle\Asf,\Bsf\rangle$ and
$\overline{\emptyset}=\langle\rangle$.
It is easy to see that, if $P$ and $R$ are finite sets of symbols, then
$\overline{P}=\overline{R}$ iff $P=R$.

We convert an NFA $M$ into a DFA $N$ as follows.  First,
we generate the least subset $X$ of $\powset\,Q_M$ such that:
\begin{itemize}
\item $\{s_M\}\in X$; and

\item for all $P\in X$ and $a\in\alphabet\,M$,
$\Delta_M(P,a)\in X$.
\end{itemize}
Thus $|X|\leq 2^{|Q_M|}$.  Then we define the DFA $N$ as follows:
\begin{itemize}
\item $Q_N=\setof{\overline{P}}{P\in X}$;

\item $s_N=\overline{\{s_M\}}=\langle s_M\rangle$;

\item $A_N=\setof{\overline{P}}{P\in X\eqtxt{and}P\cap A_M\neq\emptyset}$;
  and

\item $T_N=\setof{(\overline{P},a,
\overline{\Delta_M(P,a)})}{P\in X\eqtxt{and}a\in\alphabet\,M}$.
\end{itemize}
Then $N$ is a DFA with alphabet $\alphabet\,M$ and, for all $P\in X$ and
$a\in\alphabet\,M$, $\delta_N(\overline{P},a)=\overline{\Delta_M(P,a)}$.

Suppose $M$ is the NFA
\begin{center}
\input{chap-3.11-fig6.eepic}
\end{center}
Let's work out what the DFA $N$ is.
\begin{itemize}
\item To begin with, $\{\Asf\}\in X$, so that $\langle\Asf\rangle\in
  Q_N$.  And $\langle\Asf\rangle$ is the start state of $N$.  It is
  not an accepting state, since $\Asf\not\in A_M$.

\item Since $\{\Asf\}\in X$, and $\Delta(\{\Asf\},\zerosf)=\emptyset$,
  we add $\emptyset$ to $X$, $\langle\rangle$ to $Q_N$ and
  $\langle\Asf\rangle,\zerosf\fun\langle\rangle$ to $T_N$.

  Since $\{\Asf\}\in X$, and $\Delta(\{\Asf\},\onesf)=\{\Asf,\Bsf\}$,
  we add $\{\Asf,\Bsf\}$ to $X$, $\langle\Asf,\Bsf\rangle$ to $Q_N$
  and $\langle\Asf\rangle,\onesf\fun\langle\Asf,\Bsf\rangle$ to $T_N$.

\item Since $\emptyset\in X$, $\Delta(\emptyset,\zerosf)=\emptyset$
  and $\emptyset\in X$, we don't have to add anything to $X$ or $Q_N$,
  but we add $\langle\rangle,\zerosf\fun\langle\rangle$ to $T_N$.

  Since $\emptyset\in X$, $\Delta(\emptyset,\onesf)=\emptyset$ and
  $\emptyset\in X$, we don't have to add anything to $X$ or $Q_N$, but
  we add $\langle\rangle,\onesf\fun\langle\rangle$ to $T_N$.

\item Since $\{\Asf,\Bsf\}\in X$,
  $\Delta(\{\Asf,\Bsf\},\zerosf)=\emptyset$ and $\emptyset\in X$, we
  don't have to add anything to $X$ or $Q_N$, but we add
  $\langle\Asf,\Bsf\rangle,\zerosf\fun\langle\rangle$ to $T_N$.

  Since $\{\Asf,\Bsf\}\in X$,
  $\Delta(\{\Asf,\Bsf\},\onesf)=\{\Asf,\Bsf\}\cup\{\Csf\}=
  \{\Asf,\Bsf,\Csf\}$, we add $\{\Asf,\Bsf,\Csf\}$ to $X$,
  $\langle\Asf,\Bsf,\Csf\rangle$ to $Q_N$, and
  $\langle\Asf,\Bsf\rangle,\onesf\fun\langle\Asf,\Bsf,\Csf\rangle$ to
  $T_N$.  Since $\{\Asf,\Bsf,\Csf\}$ contains (the only) one of $M$'s
  accepting states, we add $\langle\Asf,\Bsf,\Csf\rangle$ to $A_N$.

\item Since $\{\Asf,\Bsf,\Csf\}\in X$ and
  $\Delta(\{\Asf,\Bsf,\Csf\},\zerosf)=\emptyset\cup\emptyset\cup\{\Csf\}=
  \{\Csf\}$, we add $\{\Csf\}$ to $X$, $\langle\Csf\rangle$ to $Q_N$
  and $\langle\Asf,\Bsf,\Csf\rangle,\zerosf\fun\langle\Csf\rangle$ to
  $T_N$.  Since $\{\Csf\}$ contains one of $M$'s accepting states, we
  add $\langle\Csf\rangle$ to $A_N$.

  Since $\{\Asf,\Bsf,\Csf\}\in X$,
  $\Delta(\{\Asf,\Bsf,\Csf\},\onesf)=\{\Asf,\Bsf\}\cup\{\Csf\}\cup\emptyset=
  \{\Asf,\Bsf,\Csf\}$ and $\{\Asf,\Bsf,\Csf\}\in X$, we don't have to
  add anything to $X$ or $Q_N$, but we add
  $\langle\Asf,\Bsf,\Csf\rangle,\onesf\fun\langle\Asf,\Bsf,\Csf\rangle$
  to $T_N$.

\item Since $\{\Csf\}\in X$, $\Delta(\{\Csf\},\zerosf)=\{\Csf\}$ and
  $\{\Csf\}\in X$, we don't have to add anything to $X$ or $Q_N$, but
  we add $\langle\Csf\rangle,\zerosf\fun\langle\Csf\rangle$ to $T_N$.

  Since $\{\Csf\}\in X$, $\Delta(\{\Csf\},\onesf)=\emptyset$ and
  $\emptyset\in X$, we don't have to add anything to $X$ or $Q_N$, but
  we add $\langle\Csf\rangle,\onesf\fun\langle\rangle$ to $T_N$.
\end{itemize}
Since there are no more elements to add to $X$, we are done.
Thus, the DFA $N$ is
\begin{center}
\input{chap-3.11-fig7.eepic}
\end{center}

The following two lemmas show why our conversion process is correct.

\begin{lemma}
\label{NFAToDFAConvLemma}
For all $w\in(\alphabet\,M)^*$:
\begin{itemize}
\item $\Delta_M(\{s_M\},w)\in X$; and
\item $\delta_N(s_N,w)=\overline{\Delta_M(\{s_M\},w)}$.
\end{itemize}
\end{lemma}

\begin{proof}
By left string induction.
\begin{description}
\item[\quad(Basis Step)] We have that $\Delta_M(\{s_M\},\%) = \{s_M\}\in X$
and $\delta_N(s_N,\%)= {s_N}= {\overline{\{s_M\}}}=
{\overline{\Delta_M(\{s_M\},\%)}}$.

\item[\quad(Inductive Step)] Suppose $a\in\alphabet\,M$ and
  $w\in(\alphabet\,M)^*$.  Assume the inductive hypothesis:
  $\Delta_M(\{s_M\},w)\in X$ and
  $\delta_N(s_N,w)=\overline{\Delta_M(\{s_M\},w)}$.  Since
  $\Delta_M(\{s_M\},w)\in X$ and $a\in\alphabet\,M$, we have that
  $\Delta_M(\{s_M\},wa) = \Delta_M(\Delta_M(\{s_M\},w),a)\in X$.  Thus
  \begin{alignat*}{2}
    \delta_N(s_N,wa) &= \delta_N(\delta_N(s_N,w),a) \\
    &= \delta_N(\overline{\Delta_M(\{s_M\},w)},a) &&
    \by{ind.\ hyp.} \\
    &= \overline{\Delta_M(\Delta_M(\{s_M\},w),a)} \\
    &= \overline{\Delta_M(\{s_M\},wa)} .
  \end{alignat*}
\end{description}
\end{proof}

\begin{lemma}
$L(N)=L(M)$.
\end{lemma}

\begin{proof}
\begin{description}
\item[\quad($L(M)\sub L(N)$)] Suppose $w\in L(M)$, so that
  $w\in(\alphabet\,M)^*=(\alphabet\,N)^*$ and $\Delta_M(\{s_M\},w)\cap
  A_M\neq\emptyset$.  By Lemma~\ref{NFAToDFAConvLemma}, we have that
  $\Delta_M(\{s_M\},w)\in X$ and
  $\delta_N(s_N,w)=\overline{\Delta_M(\{s_M\},w)}$.  Since
  $\Delta_M(\{s_M\},w)\in X$ and $\Delta_M(\{s_M\},w)\cap
  A_M\neq\emptyset$, it follows that
  $\delta_N(s_N,w)=\overline{\Delta_M(\{s_M\},w)} \in A_N$.  Thus
  $w\in L(N)$.

\item[\quad($L(N)\sub L(M)$)] Suppose $w\in L(N)$, so that
  $w\in(\alphabet\,N)^*=(\alphabet\,M)^*$ and $\delta_N(s_N,w)\in
  A_N$.  By Lemma~\ref{NFAToDFAConvLemma}, we have that
  $\delta_N(s_N,w)= \overline{\Delta_M(\{s_M\},w)}$.  Thus
  $\overline{\Delta_M(\{s_M\},w)}\in A_N$, so that
  $\Delta_M(\{s_M\},w)\cap A_M\neq\emptyset$.  Thus $w\in L(M)$.
\end{description}
\end{proof}

\index{deterministic finite automaton!nfaToDFA@$\nfaToDFA$}%
\index{nondeterministic finite automaton!nfaToDFA@$\nfaToDFA$}%
We define a function $\nfaToDFA\in\NFA\fun\DFA$ by: $\nfaToDFA\,M$ is
the result of running the preceding algorithm with input $M$.

\begin{theorem}
\label{NFAToDFATheorem}
For all $M\in\NFA$:
\begin{itemize}
\item $\nfaToDFA\,M\approx M$; and

\item $\alphabet(\nfaToDFA\,M)=\alphabet\,M$.
\end{itemize}
\end{theorem}

\subsection{Processing DFAs in Forlan}

The Forlan module \texttt{DFA} defines an abstract type \texttt{dfa}
\index{DFA@\texttt{DFA}}%
\index{DFA@\texttt{DFA}!dfa@\texttt{dfa}}%
(in the top-level environment) of deterministic finite automata,
along with various functions for processing DFAs.
Values of type \texttt{dfa} are implemented as values of type \texttt{fa}, and
the module DFA provides the following injection and projection functions
\begin{verbatim}
val injToFA     : dfa -> fa
val injToEFA    : dfa -> efa
val injToNFA    : dfa -> nfa
val projFromFA  : fa -> dfa
val projFromEFA : efa -> dfa
val projFromNFA : nfa -> dfa
\end{verbatim}
\index{DFA@\texttt{DFA}!injToFA@\texttt{injToFA}}%
\index{DFA@\texttt{DFA}!injToEFA@\texttt{injToEFA}}%
\index{DFA@\texttt{DFA}!injToNFA@\texttt{injToNFA}}%
\index{DFA@\texttt{DFA}!projFromFA@\texttt{projFromFA}}%
\index{DFA@\texttt{DFA}!projFromEFA@\texttt{projFromEFA}}%
\index{DFA@\texttt{DFA}!projFromNFA@\texttt{projFromNFA}}%
These functions are available in the top-level environment with the
names \texttt{injDFAToFA}, \texttt{injDFAToEFA}, \texttt{injDFAToNFA},
\texttt{projFAToDFA}, \texttt{projEFAToDFA} and \texttt{projNFAToDFA}.
\index{deterministic finite automaton!injDFAToFA@\texttt{injDFAToFA}}%
\index{deterministic finite automaton!injDFAToEFA@\texttt{injDFAToEFA}}%
\index{deterministic finite automaton!injDFAToNFA@\texttt{injDFAToNFA}}%
\index{deterministic finite automaton!projFAToDFA@\texttt{projFAToDFA}}%
\index{deterministic finite automaton!projEFAToDFA@\texttt{projEFAToDFA}}%
\index{deterministic finite automaton!projNFAToDFA@\texttt{projNFAToDFA}}%

The module \texttt{DFA} also defines the functions:
\begin{verbatim}
val input            : string -> dfa
val determProcStr    : dfa -> sym * str -> sym
val determAccepted   : dfa -> str -> bool
val determSimplified : dfa -> bool
val determSimplify   : dfa * sym set -> dfa
val fromNFA          : nfa -> dfa
\end{verbatim}
\index{DFA@\texttt{DFA}!input@\texttt{input}}%
\index{DFA@\texttt{DFA}!determProcStr@\texttt{determProcStr}}%
\index{DFA@\texttt{DFA}!determAccepted@\texttt{determAccepted}}%
\index{DFA@\texttt{DFA}!determSimplified@\texttt{determSimplified}}%
\index{DFA@\texttt{DFA}!determSimplify@\texttt{determSimplify}}%
\index{DFA@\texttt{DFA}!fromNFA@\texttt{fromNFA}}%
The function \texttt{input} is used to input a DFA.  The function
\texttt{determProcStr} is used to compute $\delta_M(q,w)$ for a DFA
$M$, using the properties of $\delta_M$.  The function
\texttt{determAccepted} uses \texttt{determProcStr} to check whether a
string is accepted by a DFA.  The function \texttt{determSimplified}
tests whether a DFA is deterministically simplified, and
the function \texttt{determSimplify}
corresponds to $\determSimplify$.  The function \texttt{fromNFA}
corresponds to our conversion function $\nfaToDFA$, and is available
in the top-level environment with that name:
\begin{verbatim}
val nfaToDFA : nfa -> dfa
\end{verbatim}
\index{DFA@\texttt{DFA}!nfaToDFA@\texttt{nfaToDFA}}%

Most of the functions for processing FAs that were introduced
in previous sections are inherited by \texttt{DFA}:
\begin{verbatim}
val output                  : string * dfa -> unit 
val numStates               : dfa -> int
val numTransitions          : dfa -> int
val alphabet                : dfa -> sym set
val equal                   : dfa * dfa -> bool
val checkLP                 : dfa -> lp -> unit
val validLP                 : dfa -> lp -> bool
val isomorphism             : dfa * dfa * sym_rel -> bool
val findIsomorphism         : dfa * dfa -> sym_rel
val isomorphic              : dfa * dfa -> bool
val renameStates            : dfa * sym_rel -> dfa
val renameStatesCanonically : dfa -> dfa
val processStr              : dfa -> sym set * str -> sym set
val accepted                : dfa -> str -> bool
val findLP                  : dfa -> sym set * str * sym set -> lp
val findAcceptingLP         : dfa -> str -> lp
\end{verbatim}
\index{DFA@\texttt{DFA}!output@\texttt{output}}%
\index{DFA@\texttt{DFA}!numStates@\texttt{numStates}}%
\index{DFA@\texttt{DFA}!numTransitions@\texttt{numTransitions}}%
\index{DFA@\texttt{DFA}!alphabet@\texttt{alphabet}}%
\index{DFA@\texttt{DFA}!equal@\texttt{equal}}%
\index{DFA@\texttt{DFA}!checkLP@\texttt{checkLP}}%
\index{DFA@\texttt{DFA}!validLP@\texttt{validLP}}%
\index{DFA@\texttt{DFA}!isomorphism@\texttt{isomorphism}}%
\index{DFA@\texttt{DFA}!findIsomorphism@\texttt{findIsomorphism}}%
\index{DFA@\texttt{DFA}!isomorphic@\texttt{isomorphic}}%
\index{DFA@\texttt{DFA}!renameStates@\texttt{renameStates}}%
\index{DFA@\texttt{DFA}!renameStatesCanonically@\texttt{renameStatesCanonically}}%
\index{DFA@\texttt{DFA}!processStr@\texttt{processStr}}%
\index{DFA@\texttt{DFA}!accepted@\texttt{accepted}}%
\index{DFA@\texttt{DFA}!findLP@\texttt{findLP}}%
\index{DFA@\texttt{DFA}!findAcceptingLP@\texttt{findAcceptingLP}}%

Suppose \texttt{dfa} is the DFA
\begin{center}
\input{chap-3.11-fig2.eepic}
\end{center}
We can turn \texttt{dfa} into an equivalent deterministically simplified
DFA whose alphabet is the union of the alphabet of the language
of \texttt{dfa} and $\{\twosf\}$, i.e., whose alphabet is
$\{\zerosf,\onesf,\twosf\}$, as follows:
\begin{list}{}
{\setlength{\leftmargin}{\leftmargini}
\setlength{\rightmargin}{0cm}
\setlength{\itemindent}{0cm}
\setlength{\listparindent}{0cm}
\setlength{\itemsep}{0cm}
\setlength{\parsep}{0cm}
\setlength{\labelsep}{0cm}
\setlength{\labelwidth}{0cm}
\catcode`\#=12
\catcode`\$=12
\catcode`\%=12
\catcode`\^=12
\catcode`\_=12
\catcode`\.=12
\catcode`\?=12
\catcode`\!=12
\catcode`\&=12
\ttfamily}
\small
\item[]\textsl{-\ }val\ dfa'\ =\ DFA.determSimplify(dfa,\ SymSet.input\ "");
\item[]\textsl{@\ }2
\item[]\textsl{@\ }.
\item[]\textsl{val\ dfa'\ =\ -\ :\ dfa}
\item[]\textsl{-\ }DFA.output("",\ dfa');
\item[]\textsl{\symbol{'173}states\symbol{'175}\ A,\ B,\ C,\ <dead>\ \symbol{'173}start\ state\symbol{'175}\ A}
\item[]\textsl{\symbol{'173}accepting\ states\symbol{'175}\ A,\ B,\ C}
\item[]\textsl{\symbol{'173}transitions\symbol{'175}}
\item[]\textsl{A,\ 0\ ->\ B;\ A,\ 1\ ->\ A;\ A,\ 2\ ->\ <dead>;\ B,\ 0\ ->\ C;\ B,\ 1\ ->\ A;}
\item[]\textsl{B,\ 2\ ->\ <dead>;\ C,\ 0\ ->\ <dead>;\ C,\ 1\ ->\ A;\ C,\ 2\ ->\ <dead>;}
\item[]\textsl{<dead>,\ 0\ ->\ <dead>;\ <dead>,\ 1\ ->\ <dead>;\ <dead>,\ 2\ ->\ <dead>}
\item[]\textsl{val\ it\ =\ ()\ :\ unit}
\end{list}

Thus \texttt{dfa'} is
\begin{center}
\input{chap-3.11-fig8.eepic}
\end{center}

Suppose that \texttt{nfa} is the NFA
\begin{center}
\input{chap-3.11-fig6.eepic}
\end{center}
We can convert \texttt{nfa} to a DFA as follows:
\begin{list}{}
{\setlength{\leftmargin}{\leftmargini}
\setlength{\rightmargin}{0cm}
\setlength{\itemindent}{0cm}
\setlength{\listparindent}{0cm}
\setlength{\itemsep}{0cm}
\setlength{\parsep}{0cm}
\setlength{\labelsep}{0cm}
\setlength{\labelwidth}{0cm}
\catcode`\#=12
\catcode`\$=12
\catcode`\%=12
\catcode`\^=12
\catcode`\_=12
\catcode`\.=12
\catcode`\?=12
\catcode`\!=12
\catcode`\&=12
\ttfamily}
\small
\item[]\textsl{-\ }val\ dfa\ =\ nfaToDFA\ nfa;
\item[]\textsl{val\ dfa\ =\ -\ :\ dfa}
\item[]\textsl{-\ }DFA.output("",\ dfa);
\item[]\textsl{\symbol{'173}states\symbol{'175}\ <>,\ <A>,\ <C>,\ <A,B>,\ <A,B,C>\ \symbol{'173}start\ state\symbol{'175}\ <A>}
\item[]\textsl{\symbol{'173}accepting\ states\symbol{'175}\ <C>,\ <A,B,C>}
\item[]\textsl{\symbol{'173}transitions\symbol{'175}}
\item[]\textsl{<>,\ 0\ ->\ <>;\ <>,\ 1\ ->\ <>;\ <A>,\ 0\ ->\ <>;\ <A>,\ 1\ ->\ <A,B>;}
\item[]\textsl{<C>,\ 0\ ->\ <C>;\ <C>,\ 1\ ->\ <>;\ <A,B>,\ 0\ ->\ <>;\ <A,B>,\ 1\ ->\ <A,B,C>;}
\item[]\textsl{<A,B,C>,\ 0\ ->\ <C>;\ <A,B,C>,\ 1\ ->\ <A,B,C>}
\item[]\textsl{val\ it\ =\ ()\ :\ unit}
\end{list}

Thus \texttt{dfa} is
\begin{center}
\input{chap-3.11-fig7.eepic}
\end{center}
And we can see why \texttt{nfa} and \texttt{dfa} accept
$\mathsf{111100}$, as follows:
\begin{list}{}
{\setlength{\leftmargin}{\leftmargini}
\setlength{\rightmargin}{0cm}
\setlength{\itemindent}{0cm}
\setlength{\listparindent}{0cm}
\setlength{\itemsep}{0cm}
\setlength{\parsep}{0cm}
\setlength{\labelsep}{0cm}
\setlength{\labelwidth}{0cm}
\catcode`\#=12
\catcode`\$=12
\catcode`\%=12
\catcode`\^=12
\catcode`\_=12
\catcode`\.=12
\catcode`\?=12
\catcode`\!=12
\catcode`\&=12
\ttfamily}
\small
\item[]\textsl{-\ }LP.output
\item[]\textsl{=\ }("",
\item[]\textsl{=\ }\ NFA.findAcceptingLP\ nfa\ (Str.fromString\ "111100"));
\item[]\textsl{A,\ 1\ =>\ A,\ 1\ =>\ A,\ 1\ =>\ B,\ 1\ =>\ C,\ 0\ =>\ C,\ 0\ =>\ C}
\item[]\textsl{val\ it\ =\ ()\ :\ unit}
\item[]\textsl{-\ }LP.output
\item[]\textsl{=\ }("",
\item[]\textsl{=\ }\ DFA.findAcceptingLP\ dfa\ (Str.fromString\ "111100"));
\item[]\textsl{<A>,\ 1\ =>\ <A,B>,\ 1\ =>\ <A,B,C>,\ 1\ =>\ <A,B,C>,\ 1\ =>\ <A,B,C>,\ 0\ =>}
\item[]\textsl{<C>,\ 0\ =>\ <C>}
\item[]\textsl{val\ it\ =\ ()\ :\ unit}
\end{list}


Finally, we see an example in which an NFA with $4$ states
is converted to a DFA with $2^4=16$ states; there is a state of the
DFA corresponding to every element of the power set of
the set of states of the NFA.  Suppose \texttt{nfa'} is
the NFA
\begin{center}
\input{chap-3.11-fig9.eepic}
\end{center}
Then we can convert \texttt{nfa'} into a DFA, as follows:
\begin{list}{}
{\setlength{\leftmargin}{\leftmargini}
\setlength{\rightmargin}{0cm}
\setlength{\itemindent}{0cm}
\setlength{\listparindent}{0cm}
\setlength{\itemsep}{0cm}
\setlength{\parsep}{0cm}
\setlength{\labelsep}{0cm}
\setlength{\labelwidth}{0cm}
\catcode`\#=12
\catcode`\$=12
\catcode`\%=12
\catcode`\^=12
\catcode`\_=12
\catcode`\.=12
\catcode`\?=12
\catcode`\!=12
\catcode`\&=12
\ttfamily}
\small
\item[]\textsl{-\ }val\ dfa'\ =\ nfaToDFA\ nfa';
\item[]\textsl{val\ dfa'\ =\ -\ :\ dfa}
\item[]\textsl{-\ }DFA.numStates\ dfa';
\item[]\textsl{val\ it\ =\ 16\ :\ int}
\item[]\textsl{-\ }DFA.output("",\ dfa');
\item[]\textsl{\symbol{'173}states\symbol{'175}}
\item[]\textsl{<>,\ <A>,\ <B>,\ <C>,\ <D>,\ <A,B>,\ <A,C>,\ <A,D>,\ <B,C>,\ <B,D>,\ <C,D>,}
\item[]\textsl{<A,B,C>,\ <A,B,D>,\ <A,C,D>,\ <B,C,D>,\ <A,B,C,D>}
\item[]\textsl{\symbol{'173}start\ state\symbol{'175}\ <A>}
\item[]\textsl{\symbol{'173}accepting\ states\symbol{'175}}
\item[]\textsl{<D>,\ <A,D>,\ <B,D>,\ <C,D>,\ <A,B,D>,\ <A,C,D>,\ <B,C,D>,\ <A,B,C,D>}
\item[]\textsl{\symbol{'173}transitions\symbol{'175}}
\item[]\textsl{<>,\ 0\ ->\ <>;\ <>,\ 1\ ->\ <>;\ <>,\ 2\ ->\ <>;\ <A>,\ 0\ ->\ <A>;}
\item[]\textsl{<A>,\ 1\ ->\ <A,B>;\ <A>,\ 2\ ->\ <>;\ <B>,\ 0\ ->\ <C>;\ <B>,\ 1\ ->\ <C>;}
\item[]\textsl{<B>,\ 2\ ->\ <B>;\ <C>,\ 0\ ->\ <D>;\ <C>,\ 1\ ->\ <D>;\ <C>,\ 2\ ->\ <C>;}
\item[]\textsl{<D>,\ 0\ ->\ <>;\ <D>,\ 1\ ->\ <>;\ <D>,\ 2\ ->\ <D>;\ <A,B>,\ 0\ ->\ <A,C>;}
\item[]\textsl{<A,B>,\ 1\ ->\ <A,B,C>;\ <A,B>,\ 2\ ->\ <B>;\ <A,C>,\ 0\ ->\ <A,D>;}
\item[]\textsl{<A,C>,\ 1\ ->\ <A,B,D>;\ <A,C>,\ 2\ ->\ <C>;\ <A,D>,\ 0\ ->\ <A>;}
\item[]\textsl{<A,D>,\ 1\ ->\ <A,B>;\ <A,D>,\ 2\ ->\ <D>;\ <B,C>,\ 0\ ->\ <C,D>;}
\item[]\textsl{<B,C>,\ 1\ ->\ <C,D>;\ <B,C>,\ 2\ ->\ <B,C>;\ <B,D>,\ 0\ ->\ <C>;}
\item[]\textsl{<B,D>,\ 1\ ->\ <C>;\ <B,D>,\ 2\ ->\ <B,D>;\ <C,D>,\ 0\ ->\ <D>;}
\item[]\textsl{<C,D>,\ 1\ ->\ <D>;\ <C,D>,\ 2\ ->\ <C,D>;\ <A,B,C>,\ 0\ ->\ <A,C,D>;}
\item[]\textsl{<A,B,C>,\ 1\ ->\ <A,B,C,D>;\ <A,B,C>,\ 2\ ->\ <B,C>;\ <A,B,D>,\ 0\ ->\ <A,C>;}
\item[]\textsl{<A,B,D>,\ 1\ ->\ <A,B,C>;\ <A,B,D>,\ 2\ ->\ <B,D>;\ <A,C,D>,\ 0\ ->\ <A,D>;}
\item[]\textsl{<A,C,D>,\ 1\ ->\ <A,B,D>;\ <A,C,D>,\ 2\ ->\ <C,D>;\ <B,C,D>,\ 0\ ->\ <C,D>;}
\item[]\textsl{<B,C,D>,\ 1\ ->\ <C,D>;\ <B,C,D>,\ 2\ ->\ <B,C,D>;}
\item[]\textsl{<A,B,C,D>,\ 0\ ->\ <A,C,D>;\ <A,B,C,D>,\ 1\ ->\ <A,B,C,D>;}
\item[]\textsl{<A,B,C,D>,\ 2\ ->\ <B,C,D>}
\item[]\textsl{val\ it\ =\ ()\ :\ unit}
\end{list}

In Section~\ref{EquivalenceTestingAndMinimizationOfDFAs},
we will use Forlan to show that there is no
DFA with fewer than $16$ states that accepts the language accepted
\index{deterministic finite automaton!exponential blowup}%
by \texttt{nfa'} and \texttt{dfa'}.

\subsection{Notes}

In contrast to the standard approach, the transition function $\delta$
for a DFA $M$ is not part of the definition of $M$, but is derived from
the definition.  Our approach to proving the correctness of DFAs,
using induction on $\Lambda$ plus proof by contradiction, is novel,
simple and elegant.  The material on deterministic simplification is
original, but straightforward.  And the algorithm for converting NFAs
to DFAs is standard.

\index{finite automaton!deterministic|)}%
\index{deterministic finite automaton|)}%
\index{DFA|)}%

%%% Local Variables: 
%%% mode: latex
%%% TeX-master: "book"
%%% End: 
