\section{Simplification of Finite Automata}
\label{SimplificationOfFiniteAutomata}

\index{simplification!finite automaton|(}%
\index{finite automaton!simplification|(}%
In this section, we: say what it means for a finite automaton to be
simplified; study an algorithm for simplifying finite automata; and
see how finite automata can be simplified in Forlan.

Suppose $M$ is the finite automaton
\begin{center}
\input{chap-3.7-fig1.eepic}
\end{center}
$M$ is odd for two distinct reasons.  First, there are no valid
labeled paths from the start state to $\Dsf$ and $\Esf$, and so these
states are redundant.  Second, there are no valid labeled paths from
$\Csf$ to an accepting state, and so it is also redundant.  We will
say that $\Csf$ is not ``live'' ($\Csf$ is ``dead''), and that $\Dsf$
and $\Esf$ are not ``reachable''.

Suppose $M$ is a finite automaton.  We say that a state $q\in Q_M$ is:
\begin{itemize}
\item \emph{reachable in} $M$ iff there is a labeled path $\lp$ such that
\index{reachable state}%
\index{finite automaton!reachable state}%
$\lp$ is valid for $M$, the start state of $\lp$ is $s_M$, and
the end state of $\lp$ is $q$;

\item \emph{live in} $M$ iff there is a labeled path $\lp$ such that
\index{live state}%
\index{finite automaton!live state}%
$\lp$ is valid for $M$, the start state of $\lp$ is $q$, and
the end state of $\lp$ is in $A_M$;

\item \emph{dead in} $M$ iff $q$ is not live in $M$; and
\index{dead state}%
\index{finite automaton!dead state}%

\item \emph{useful in} $M$ iff $q$ is both reachable and live in $M$.
\index{useful state}%
\index{finite automaton!useful state}%
\end{itemize}

Let $M$ be our example finite automaton.
The reachable states of $M$ are: $\Asf$, $\Bsf$ and $\Csf$.
The live states of $M$ are: $\Asf$, $\Bsf$, $\Dsf$ and $\Esf$.
And, the useful states of $M$ are: $\Asf$ and $\Bsf$.

There is a simple algorithm for generating the set of reachable states
of a finite automaton $M$.  We generate the least subset $X$ of $Q_M$
such that:
\begin{itemize}
\item $s_M\in X$; and

\item for all $q,r\in Q_M$ and $x\in\Str$, if $q\in X$ and
$(q,x,r)\in T_M$, then $r\in X$.
\end{itemize}
The start state of $M$ is added to $X$, since $s_M$ is always
reachable, by the zero-length labeled path $s_M$.  Then, if $q$ is
reachable, and $(q,x,r)$ is a transition of $M$, then $r$ is clearly
reachable.  Thus all of the elements of $X$ are indeed reachable.
And, it's not hard to show that every reachable state will be added to
$X$.

Similarly, there is a simple algorithm for generating the
set of live states of a finite automaton $M$.  We generate the least
subset $Y$ of $Q_M$ such that:
\begin{itemize}
\item $A_M\sub Y$; and

\item for all $q,r\in Q_M$ and $x\in\Str$, if $r\in Y$ and
$(q,x,r)\in T_M$, then $q\in Y$.
\end{itemize}
This time it's the accepting states of $M$ that initially added
to our set, since each accepting state is trivially live.
Then, if $r$ is live, and $(q,x,r)$ is a transition
of $M$, then $q$ is clearly live.

Thus, we can generate the set of useful states of an FA by generating
the set of reachable states, generating the set of live states, and
intersecting those sets of states.

Now, suppose $N$ is the FA
\begin{center}
  \input{chap-3.7-fig5.eepic}
\end{center}
Here, the transitions $(\Asf,\zerosf,\Bsf)$ and $(\Asf,\onesf,\Bsf)$
are redundant, in the sense that if $N'$ is the result of removing these
transitions from $N$, we still have that $\Bsf\in\Delta_{N'}(\{\Asf\},\zerosf)$
and $\Bsf\in\Delta_{N'}(\{\Asf\},\onesf)$.

Given an FA $M$ and a finite subset $U$ of $\setof{(q,x,r)}{q,r\in
  Q_M\eqtxt{and} x\in\Str}$, we write $M/U$ for the FA that is
identical to $M$ except that its set of transitions is $U$.  If $M$ is
an FA and $(p,x,q)\in T_M$, we say that:
\begin{itemize}
\item $(p,x,q)$ \emph{is redundant in} $M$ iff
  $q\in\Delta_{N}(\{p\},x)$, where $N=M/(T_M-\{(p,x,q)\})$; and

\item $(p,x,q)$ \emph{is irredundant in} $M$ iff $(p,x,q)$ is not
  redundant in $M$.
\end{itemize}

We say that a finite automaton $M$ is \emph{simplified} iff either
\index{finite automaton!simplified}%
\index{simplified!finite automaton}%
\index{simplification!finite automaton!simplified}%
\begin{itemize}
\item every state of $M$ is useful, and every transition of $M$
  is irredundant; or

\item $|Q_M|=1$ and $A_M = T_M = \emptyset$.
\end{itemize}
Thus the FA
\begin{center}
\input{chap-3.7-fig2.eepic}
\end{center}
is simplified, even though its start state is not live, and is
thus not useful.

\begin{proposition}
\label{AlphabetSimplifiedFA}
If $M$ is a simplified finite automaton, then
$\alphabet\,M=\alphabet(L(M))$.
\end{proposition}

We always have that $\alphabet(L(M))\sub\alphabet\,M$.  But, when $M$
is simplified, we also have that $\alphabet\,M\sub\alphabet(L(M))$,
i.e., that every symbol appearing in a string of one of $M$'s
transitions also appears in one of the strings accepted by $M$.

To give our simplification algorithm for finite automta, we need an
auxiliary function for removing redundant transitions from an FA.
Given an FA $M$, $p,q\in Q_M$ and $x\in\Str$, we say that $(p,x,q)$
\emph{is implicit in} $M$ iff $q\in\Delta_M(\{p\},x)$.

Given an FA $M$, we define a function
$\remRedun_M\in\powset\,T_M\times\powset\,T_M\fun\powset\,T_M$ (we
often drop the $M$ when it's clear from the context) by well-founded
recursion on the size of its second argument.  For $U,V\sub T_M$,
$\remRedun(U, V)$ proceeds as follows:
\begin{itemize}
\item If $V=\emptyset$, then it returns $U$.

\item Otherwise, let $v$ be the greatest element of $V$, and $V' = V -
  \{v\}$.  If $v$ is implicit in $M/(U\cup V')$, then $\remRedun$
  returns the result of evaluating $\remRedun(U, V')$.  Otherwise, it
  returns the result of evaluating $\remRedun(U \cup \{v\}, V')$.
\end{itemize}

In general, there are multiple---incompatible---ways of removing
redundant transitions from an FA.  $\remRedun$ is defined so as to
favor removing transitions that are larger in our total ordering on
transitions.

\begin{proposition}
\label{RemRedunFA}
Suppose $M$ is a finite automaton.  For all $U,V\sub T_M$, if all the
elements of $U$ are irredundant in $M/(U\cup V)$, and, for all $p,q\in Q_M$
and $x\in\Str$, $(p,x,q)$ is implicit in $M$ iff $(p,x,q)$ is implicit
in $M/(U\cup V)$, then all the elements of $\remRedun(U,V)$ are
irredundant in $M/\remRedun(U,V)$, and, for all $p,q\in Q_M$ and
$x\in\Str$, $(p,x,q)$ is implicit in $M$ iff $(p,x,q)$ is implicit in
$M/\remRedun(U,V)$.
\end{proposition}

\begin{proof}
By well-founded induction on the size of the second argument to
$\remRedun$.
\end{proof}

Now we can give an algorithm for simplifying finite automata.
\index{simplification!finite automaton!algorithm}%
\index{finite automaton!simplification algorithm}%
We define a function $\simplify\in\FA\fun\FA$ by: $\simplify\,M$ is
\index{simplify@$\simplify$}%
\index{finite automaton!simplify@$\simplify$}%
\index{simplification!finite automaton!simplify@$\simplify$}%
the finite automaton $N$ produced by the following process.
\begin{itemize}
\item First, the useful states are $M$ are determined.

\item If $s_M$ is not useful in $M$, the $N$ is defined by:
\begin{itemize}
\item $Q_N=\{s_M\}$;

\item $s_N = s_M$;

\item $A_N=\emptyset$; and

\item $T_N=\emptyset$.
\end{itemize}

\item And, if $s_M$ is useful in $M$, then $N$ is defined by:
\begin{itemize}
\item $Q_N=\setof{q\in Q_M}{q\eqtxtl{is useful in $M$}}$;

\item $s_N = s_M$;

\item $A_N=A_M\cap Q_N=\setof{q\in A_M}{q\in Q_N}$; and

\item $T_N=\remRedun(\emptyset, \setof{(q, x, r)\in T_M}{q,r\in Q_N})$.
\end{itemize}
\end{itemize}

\begin{proposition}
Suppose $M$ is a finite automaton.
Then:
\begin{enumerate}[\quad(1)]
\item $\simplify\,M$ is simplified;

\item $\simplify\,M\approx M$; and

\item $\alphabet(\simplify\,M) = \alphabet(L(M)) \sub\alphabet\,M$.
\end{enumerate}
\end{proposition}

\begin{proof}
Follows easily using Propositions~\ref{AlphabetSimplifiedFA} and
\ref{RemRedunFA}.
\end{proof}

If $M$ is the finite automaton
\begin{center}
\input{chap-3.7-fig1.eepic}
\end{center}
then $\simplify\,M$ is the finite automaton
\begin{center}
\input{chap-3.7-fig4.eepic}
\end{center}
And if $N$ is the finite automaton
\begin{center}
\input{chap-3.7-fig5.eepic}
\end{center}
then $\simplify\,N$ is the finite automaton
\begin{center}
\input{chap-3.7-fig7.eepic}
\end{center}

The Forlan module \texttt{FA}
\index{FA@\texttt{FA}}%
includes the following functions relating to the simplification of
finite automata:
\begin{verbatim}
val simplify   : fa -> fa
val simplified : fa -> bool
\end{verbatim}
\index{FA@\texttt{FA}!simplified@\texttt{simplified}}%
\index{FA@\texttt{FA}!simplify@\texttt{simplify}}%
The function \texttt{simplify} corresponds to $\simplify$, and
\texttt{simplified} tests whether an FA is simplified.

In the following, suppose \texttt{fa1} is the finite automaton
\begin{center}
\input{chap-3.7-fig1.eepic}
\end{center}
\texttt{fa2} is the finite automaton
\begin{center}
\input{chap-3.7-fig5.eepic}
\end{center}
and \texttt{fa3} is the finita automaton
\begin{center}
\input{chap-3.7-fig6.eepic}
\end{center}
Here are some example uses of \texttt{simplify} and \texttt{simplified}:
\begin{list}{}
{\setlength{\leftmargin}{\leftmargini}
\setlength{\rightmargin}{0cm}
\setlength{\itemindent}{0cm}
\setlength{\listparindent}{0cm}
\setlength{\itemsep}{0cm}
\setlength{\parsep}{0cm}
\setlength{\labelsep}{0cm}
\setlength{\labelwidth}{0cm}
\catcode`\#=12
\catcode`\$=12
\catcode`\%=12
\catcode`\^=12
\catcode`\_=12
\catcode`\.=12
\catcode`\?=12
\catcode`\!=12
\catcode`\&=12
\ttfamily}
\small
\item[]\textsl{-\ }FA.simplified\ fa1;
\item[]\textsl{val\ it\ =\ false\ :\ bool}
\item[]\textsl{-\ }val\ fa1'\ =\ FA.simplify\ fa1;
\item[]\textsl{val\ fa1'\ =\ -\ :\ fa}
\item[]\textsl{-\ }FA.output("",\ fa1');
\item[]\textsl{\symbol{'173}states\symbol{'175}\ A,\ B\ \symbol{'173}start\ state\symbol{'175}\ A\ \symbol{'173}accepting\ states\symbol{'175}\ B}
\item[]\textsl{\symbol{'173}transitions\symbol{'175}\ A,\ %\ ->\ B;\ A,\ 0\ ->\ A;\ B,\ 1\ ->\ B}
\item[]\textsl{val\ it\ =\ ()\ :\ unit}
\item[]\textsl{-\ }FA.simplified\ fa1';
\item[]\textsl{val\ it\ =\ true\ :\ bool}
\item[]\textsl{-\ }val\ fa2'\ =\ FA.simplify\ fa2;
\item[]\textsl{val\ fa2'\ =\ -\ :\ fa}
\item[]\textsl{-\ }FA.output("",\ fa2');
\item[]\textsl{\symbol{'173}states\symbol{'175}\ A,\ B\ \symbol{'173}start\ state\symbol{'175}\ A\ \symbol{'173}accepting\ states\symbol{'175}\ B}
\item[]\textsl{\symbol{'173}transitions\symbol{'175}\ A,\ %\ ->\ B;\ A,\ 0\ ->\ A;\ B,\ 1\ ->\ B}
\item[]\textsl{val\ it\ =\ ()\ :\ unit}
\item[]\textsl{-\ }val\ fa3'\ =\ FA.simplify\ fa3;
\item[]\textsl{val\ fa3'\ =\ -\ :\ fa}
\item[]\textsl{-\ }FA.output("",\ fa3');
\item[]\textsl{\symbol{'173}states\symbol{'175}\ A,\ B,\ C\ \symbol{'173}start\ state\symbol{'175}\ A\ \symbol{'173}accepting\ states\symbol{'175}\ A}
\item[]\textsl{\symbol{'173}transitions\symbol{'175}\ A,\ %\ ->\ B\ |\ C;\ B,\ %\ ->\ A;\ C,\ %\ ->\ A}
\item[]\textsl{val\ it\ =\ ()\ :\ unit}
\end{list}

Thus the simplification of \texttt{fa3} resulted in the removal of
the $\%$-transitions between $\Bsf$ and $\Csf$.

\begin{exercise}
In the simplification of \texttt{fa3}, if transitions had been
considered for removal due to being redundant in other orders,
what FAs could have resulted.
\end{exercise}

\index{simplification!finite automaton|)}%
\index{finite automaton!simplification|)}%

\subsection{Notes}

The removal of useless states is analogous to the standard approach to
ridding grammars of useless variables.  The idea of removing redundant
transitions, though, seems to be novel.

%%% Local Variables: 
%%% mode: latex
%%% TeX-master: "book"
%%% End: 
