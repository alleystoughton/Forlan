\section{A Parsing Algorithm}
\label{AParsingAlgorithm}

\index{grammar!parsing algorithm|(}%
In this section, we consider a simple, fairly inefficient parsing
algorithm that works for all context-free grammars.  In
Section~\ref{AmbiguityOfGrammars}, we consider an efficient parsing
method that works for grammars for languages of operators of varying
precedences and associativities.  Compilers courses cover efficient
algorithms that work for various subsets of the context free grammars.

\subsection{Algorithm}

Suppose $G$ is a grammar, $w\in\Str$ and $a\in\Sym$.  We consider an
algorithm for testing whether $w$ is parsable from $a$ using $G$.
If $w\not\in(Q_G\cup\alphabet\,G)^*$ or $a\not\in Q_G\cup\alphabet\,w$,
then the algorithm returns $\false$.
Otherwise, it proceeds as follows.

Let $A=Q_G\cup\alphabet\,w$ and $B=\setof{x\in\Str}{x\eqtxt{is a substring
of}w}$.
The algorithm generates the least subset $X$ of $A\times B$ such that:
\begin{enumerate}[\quad(1)]
\item For all $a\in\alphabet\,w$, $(a,a)\in X$;

\item For all $q\in Q_G$, if $q\fun\%\in P_G$, then
  $(q,\%)\in X$; and

\item For all $q\in Q_G$, $n\in\nats-\{0\}$, $a_1,\ldots,a_n\in A$ and
  $x_1,\ldots,x_n\in B$, if
  \begin{itemize}
  \item $q\fun a_1\cdots a_n\in P_G$,

  \item for all $i\in[1:n]$, $(a_i,x_i)\in X$, and

  \item $x_1\cdots x_n\in B$,
  \end{itemize}
then $(q,x_1\cdots x_n)\in X$.
\end{enumerate}
Since $A\times B$ is finite, this process terminates.

For example, let $G$ be the grammar
\begin{align*}
\Asf &\fun \Bsf\Csf\mid \Csf\Dsf , \\
\Bsf &\fun \zerosf\mid \Csf\Bsf , \\
\Csf &\fun \onesf\mid \Dsf\Dsf , \\
\Dsf &\fun \zerosf\mid \Bsf\Csf ,
\end{align*}
and let $w=\mathsf{0010}$ and $a=\Asf=s_G$.
We have that:
\begin{itemize}
\item $(\zerosf,\zerosf)\in X$;

\item $(\onesf, \onesf)\in X$;

\item $(\Bsf,\zerosf)\in X$, since $\Bsf\fun\zerosf\in P_G$,
$(\zerosf,\zerosf)\in X$ and $\zerosf\in B$;

\item $(\Csf,\onesf)\in X$, since $\Csf\fun\onesf\in P_G$,
$(\onesf,\onesf)\in X$ and $\onesf\in B$;

\item $(\Dsf,\zerosf)\in X$, since $\Dsf\fun\zerosf\in P_G$,
$(\zerosf,\zerosf)\in X$ and $\zerosf\in B$;

\item $(\Asf,\mathsf{01})\in X$, since $\Asf\fun\mathsf{BC}\in P_G$,
$(\Bsf,\zerosf)\in X$, $(\Csf,\onesf)\in X$ and
$\mathsf{01}\in B$;

\item $(\Asf,\mathsf{10})\in X$, since $\Asf\fun\mathsf{CD}\in P_G$,
$(\Csf,\onesf)\in X$, $(\Dsf,\zerosf)\in X$ and $\mathsf{10}\in B$;

\item $(\Bsf,\mathsf{10})\in X$, since $\Bsf\fun\mathsf{CB}\in P_G$,
$(\Csf,\onesf)\in X$, $(\Bsf,\zerosf)\in X$ and $\mathsf{10}\in B$;

\item $(\Csf,\mathsf{00})\in X$, since $\Csf\fun\mathsf{DD}\in P_G$,
$(\Dsf,\zerosf)\in X$, $(\Dsf,\zerosf)\in X$ and $\mathsf{00}\in B$;

\item $(\Dsf,\mathsf{01})\in X$, since $\Dsf\fun\mathsf{BC}\in P_G$,
$(\Bsf,\zerosf)\in X$, $(\Csf,\onesf)\in X$ and $\mathsf{01}\in B$;

\item $(\Csf,\mathsf{001})\in X$, since $\Csf\fun\mathsf{DD}\in P_G$,
$(\Dsf,\mathsf{0})\in X$, $(\Dsf,\mathsf{01})\in X$ and
$\mathsf{0(01)}\in B$;

\item $(\Csf,\mathsf{010})\in X$, since $\Csf\fun\mathsf{DD}\in P_G$,
$(\Dsf,\mathsf{01})\in X$, $(\Dsf,\mathsf{0})\in X$ and
$\mathsf{(01)0}\in B$;

\item $(\Asf,\mathsf{0010})\in X$, since $\Asf\fun\mathsf{BC}\in P_G$,
$(\Bsf,\mathsf{0})\in X$, $(\Csf,\mathsf{010})\in X$ and
$\mathsf{0(010)}\in B$;

\item $(\Bsf,\mathsf{0010})\in X$, since $\Bsf\fun\mathsf{CB}\in P_G$,
$(\Csf,\mathsf{00})\in X$, $(\Bsf,\mathsf{10})\in X$ and
$\mathsf{(00)(10)}\in B$;

\item $(\Dsf,\mathsf{0010})\in X$, since $\Dsf\fun\mathsf{BC}\in P_G$,
$(\Bsf,\mathsf{0})\in X$, $(\Csf,\mathsf{010})\in X$ and
$\mathsf{0(010)}\in B$;

\item Nothing more can be added to $X$.  To verify this, one must
check that nothing new can be added to $X$ using rule~(3).
\end{itemize}

Back in the general case, we have these lemmas:

\begin{lemma}
For all $(b,x)\in X$, there is a $\pt\in\PT$ such that
\begin{itemize}
\item $\pt$ is valid for $G$,

\item $\rootLabel\,\pt=b$, and

\item $\yield\,\pt=x$.
\end{itemize}
\end{lemma}

\begin{lemma}
For all $\pt\in\PT$, if
\begin{itemize}
\item $\pt$ is valid for $G$,

\item $\rootLabel\,\pt\in A$, and

\item $\yield\,\pt\in B$,
\end{itemize}
then $(\rootLabel\,\pt, \yield\,\pt)\in X$.
\end{lemma}

Because of our lemmas, to determine if $w$ is parsable from $a$, we
just have to check whether $(a, w)\in X$.
In the case of our example grammar, we have that $w=\mathsf{0010}$ is
parsable from $a=\Asf$, since $(\Asf,\mathsf{0010})\in X$.
Hence $\mathsf{0010}\in L(G)$.

Note that any production whose right-hand side contains an element of
$\alphabet\,G-\alphabet\,w$ won't affect the generation of $X$.  Thus
our algorithm ignores such productions.

Furthermore, our parsability algorithm actually generates $X$ in a
sequence of stages. At each point, it has subsets $U$ and $V$ of
$A\times B$. $U$ consists of the older elements of $X$, whereas $V$
consists of the most recently added elements.
\begin{itemize}
\item First, it lets $U=\emptyset$ and sets $V$ to be the union of
  $\setof{(a,a)}{a\in\alphabet\,w}$ and $\setof{(q,\%)}{(q,\%)\in
    P_G}$. It then enters its main loop.

\item At a stage of the loop's iteration, it lets $Y$ be $U\cup V$,
  and then lets $Z$ be the set of all $(q,x_1\cdots x_n)$ such that
  $n\geq 1$ and there are $a_1,\ldots,a_n\in A$ and $i\in[1:n]$ such
  that
  \begin{itemize}
  \item $q\fun a_1\cdots a_n\in P_G$,

  \item $(a_i,x_i)\in V$,

  \item for all $k\in[1:n]-\{i\}$, $(a_k,x_k)\in Y$,

  \item $x_1\cdots x_n\in B$, and

  \item $(q,x_1\cdots x_n)\not\in Y$.
  \end{itemize}
  If $Z\neq\emptyset$, then it sets $U$ to $Y$, and $V$ to $Z$, and
  repeats; Otherwise, the result is $Y$.
\end{itemize}

\index{grammar!minimal parse}%
We say that a parse tree $\pt$ is \emph{a minimal parse} of a string
$w$ \emph{from} a symbol $a$ \emph{using} a grammar $G$ iff $\pt$ is
valid for $G$, $\rootLabel\,\pt = a$ and $\yield\,\pt = w$, and there
is no strictly smaller $\pt'\in\PT$ such that $\pt'$ is valid for $G$,
$\rootLabel\,\pt' = a$ and $\yield\,\pt' = w$.

We can convert our parsability algorithm into a parsing algorithm as
follows.  Given $w\in(Q_G\cup\alphabet\,G)^*$ and
$a\in(Q_G\cup\alphabet\,w)$, we generate our set $X$ as before, but we
annotate each element $(b,x)$ of $X$ with a parse tree $\pt$ such that
\begin{itemize}
\item $\pt$ is valid for $G$,

\item $\rootLabel\,\pt=b$, and

\item $\yield\,\pt=x$,
\end{itemize}
Thus we can return the parse tree labeling $(a, w)$, if this pair is
in $X$, and indicate failure otherwise.

With a little more work, we can arrange that the parse trees returned
by our parsing algorithm are minimally-sized, and this is what the
official version of our parsing algorithm guarantees.  This goal is a
little tricky to achieve, since some pairs will first be labeled by
parse trees that aren't minimally sized. But we keep going as long as
either new pairs are found, or smaller parse trees are found for
existing pairs.

\subsection{Parsing in Forlan}

The Forlan module \texttt{Gram} defines the functions
\begin{verbatim}
val parsable              : gram -> sym * str -> bool
val generatedFromVariable : gram -> sym * str -> bool
val generated             : gram -> str -> bool
\end{verbatim}
\index{Gram@\texttt{Gram}!parsable@\texttt{parsable}}%
\index{Gram@\texttt{Gram}!generatedFromVariable@\texttt{generatedFromVariable}}%
\index{Gram@\texttt{Gram}!generated@\texttt{generated}}%
The function \texttt{parsable} tests whether a string $w$ is parsable
from a symbol $a$ using a grammar $G$.  The function
\texttt{generatedFromVariable} tests whether a string $w$ is generated
from a variable $q$ using a grammar $G$; it issues an error message if
$q$ isn't a variable of $G$.  And the function \texttt{generated}
tests whether a string $w$ is generated by a grammar $G$.

\texttt{Gram} also includes:
\begin{verbatim}
val parse                     : gram -> sym * str -> pt
val parseAlphabetFromVariable : gram -> sym * str -> pt
val parseAlphabet             : gram -> str -> pt
\end{verbatim}
\index{Gram@\texttt{Gram}!parse@\texttt{parse}}%
\index{Gram@\texttt{Gram}!parseAlphabetFromVariable@\texttt{parseAlphabetFromVarable}}%
\index{Gram@\texttt{Gram}!parseAlphabet@\texttt{parseAlphabet}}%
The function \texttt{parse} tries to find a minimal parse of a string
$w$ from a symbol $a$ using a grammar $G$; it issues an error message
if $w\not\in(Q_G\cup\alphabet\,G)^*$, or $a\not\in
Q_G\cup\alphabet\,w$, or such a parse doesn't exist.  The function
\texttt{parseAlphabetFromVariable} tries to find a minimal parse of a
string $w\in(\alphabet\,G)^*$ from a variable $q$ using a grammar $G$;
it issues an error message if $q\not\in Q_G$, or
$w\not\in(\alphabet\,G)^*$, or such a parse doesn't exist.  And the
function \texttt{parseAlphabet} tries to find a minimal parse of a
string $w\in(\alphabet\,G)^*$ from $s_G$ using a grammar $G$; it
issues an error message if $w\not\in(\alphabet\,G)^*$, or such a parse
doesn't exist.

Suppose that \texttt{gram} of type \texttt{gram} is bound to the grammar
\begin{align*}
\Asf &\fun \Bsf\Csf\mid \Csf\Dsf , \\
\Bsf &\fun \zerosf\mid \Csf\Bsf , \\
\Csf &\fun \onesf\mid \Dsf\Dsf , \\
\Dsf &\fun \zerosf\mid \Bsf\Csf .
\end{align*}
We can check whether some strings are generated by this grammar
as follows:
\begin{list}{}
{\setlength{\leftmargin}{\leftmargini}
\setlength{\rightmargin}{0cm}
\setlength{\itemindent}{0cm}
\setlength{\listparindent}{0cm}
\setlength{\itemsep}{0cm}
\setlength{\parsep}{0cm}
\setlength{\labelsep}{0cm}
\setlength{\labelwidth}{0cm}
\catcode`\#=12
\catcode`\$=12
\catcode`\%=12
\catcode`\^=12
\catcode`\_=12
\catcode`\.=12
\catcode`\?=12
\catcode`\!=12
\catcode`\&=12
\ttfamily}
\small
\item[]\textsl{-\ }Gram.generated\ gram\ (Str.fromString\ "0010");
\item[]\textsl{val\ it\ =\ true\ :\ bool}
\item[]\textsl{-\ }Gram.generated\ gram\ (Str.fromString\ "0100");
\item[]\textsl{val\ it\ =\ true\ :\ bool}
\item[]\textsl{-\ }Gram.generated\ gram\ (Str.fromString\ "0101");
\item[]\textsl{val\ it\ =\ false\ :\ bool}
\end{list}

And we can try to find parses of some strings as follows:
\begin{list}{}
{\setlength{\leftmargin}{\leftmargini}
\setlength{\rightmargin}{0cm}
\setlength{\itemindent}{0cm}
\setlength{\listparindent}{0cm}
\setlength{\itemsep}{0cm}
\setlength{\parsep}{0cm}
\setlength{\labelsep}{0cm}
\setlength{\labelwidth}{0cm}
\catcode`\#=12
\catcode`\$=12
\catcode`\%=12
\catcode`\^=12
\catcode`\_=12
\catcode`\.=12
\catcode`\?=12
\catcode`\!=12
\catcode`\&=12
\ttfamily}
\small
\item[]\textsl{-\ }fun\ test\ s\ =
\item[]\textsl{=\ }\ \ \ \ \ \ PT.output
\item[]\textsl{=\ }\ \ \ \ \ \ ("",
\item[]\textsl{=\ }\ \ \ \ \ \ \ Gram.parseAlphabet\ gram\ (Str.fromString\ s));
\item[]\textsl{val\ test\ =\ fn\ :\ string\ ->\ unit}
\item[]\textsl{-\ }test\ "0010";
\item[]\textsl{A(C(D(0),\ D(B(0),\ C(1))),\ D(0))}
\item[]\textsl{val\ it\ =\ ()\ :\ unit}
\item[]\textsl{-\ }test\ "0100";
\item[]\textsl{A(C(D(B(0),\ C(1)),\ D(0)),\ D(0))}
\item[]\textsl{val\ it\ =\ ()\ :\ unit}
\item[]\textsl{-\ }test\ "0101";
\item[]\textsl{no\ such\ parse\ exists}
\item[]
\item[]\textsl{uncaught\ exception\ Error}
\end{list}

But we can also check parsability of strings containing variables, as well
as try to find parses of such strings:
\begin{list}{}
{\setlength{\leftmargin}{\leftmargini}
\setlength{\rightmargin}{0cm}
\setlength{\itemindent}{0cm}
\setlength{\listparindent}{0cm}
\setlength{\itemsep}{0cm}
\setlength{\parsep}{0cm}
\setlength{\labelsep}{0cm}
\setlength{\labelwidth}{0cm}
\catcode`\#=12
\catcode`\$=12
\catcode`\%=12
\catcode`\^=12
\catcode`\_=12
\catcode`\.=12
\catcode`\?=12
\catcode`\!=12
\catcode`\&=12
\ttfamily}
\small
\item[]\textsl{-\ }Gram.parsable\ gram
\item[]\textsl{=\ }(Sym.fromString\ "A",\ Str.fromString\ "0D0C");
\item[]\textsl{val\ it\ =\ true\ :\ bool}
\item[]\textsl{-\ }PT.output
\item[]\textsl{=\ }("",
\item[]\textsl{=\ }\ Gram.parse\ gram
\item[]\textsl{=\ }\ (Sym.fromString\ "A",\ Str.fromString\ "0D0C"));
\item[]\textsl{A(C(D(0),\ D),\ D(B(0),\ C))}
\item[]\textsl{val\ it\ =\ ()\ :\ unit}
\end{list}


\subsection{Notes}

Our parsability and parsing algorithms are straightforward generalizations of
the familiar algorithm for checking whether a grammar in Chomsky Normal Form
generates a string.

\index{grammar!parsing algorithm|)}%

%%% Local Variables: 
%%% mode: latex
%%% TeX-master: "book"
%%% End: 
