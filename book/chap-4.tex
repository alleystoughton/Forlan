\chapter{Context-free Languages}
\label{ContextFreeLanguages}

In this chapter, we study context-free grammars and languages.
Context-free grammars are used to describe the syntax of programming
languages, i.e., to specify parsers of programming languages.
\index{programming language}%
\index{programming language!parser}%
\index{programming language!parsing}%

\index{context-free language}%
A language is called context-free iff it is generated by a
context-free grammar.  It will turn out that the set of all
context-free languages is a proper superset of the set of all regular
\index{regular language}%
languages.  On the other hand, the context-free languages have weaker
closure properties than the regular languages, and we won't be able to
give algorithms for checking grammar equivalence or minimizing the
size of grammars.

\section{Grammars, Parse Trees and Context-free Languages}
\label{GrammarsParseTreesAndContextFreeLanguages}

In this section, we: say what (context-free) grammars are; use the
notion of a parse tree to say what grammars mean; say what it means
for a language to be context-free; and begin to show how grammars can
be processed using Forlan.

\subsection{Grammars}

A \emph{context-free grammar} (or just \emph{grammar}) $G$ consists of:
\begin{itemize}
\item a finite set $Q_G$ of symbols (we call the elements of $Q_G$
the \emph{variables} of $G$);

\item an element $s_G$ of $Q_G$ (we call $s_G$ the \emph{start variable}
of $G$); and

\item a finite subset $P_G$ of $\setof{(q,x)}{q\in Q_G\eqtxt{and}
x\in\Str}$ (we call the elements of $P_G$ the \emph{productions} of
$G$, and we often write $(q, x)$ as $q\fun x$).
\end{itemize}

In a context where we are only referring to a single grammar, $G$, we
sometimes abbreviate $Q_G$, $s_G$ and $P_G$ to $Q$, $s$ and $P$,
respectively.  Whenever possible, we will use the mathematical
variables $p$, $q$ and $r$ to name variables.  We write $\Gram$ for
the set of all grammars.  Since every grammar can be described by a
finite sequence of ASCII characters, we have that $\Gram$ is countably
infinite.

As an example, we can define a grammar $G$ (of arithmetic expressions) as
follows:
\begin{itemize}
\item $Q_G=\{\mathsf{E}\}$;

\item $s_G=\Esf$; and

\item $P_G=\{\mathsf{E\fun E\plussym E,\;
E\fun E\timessym E,\;
E\fun\openparsym E\closparsym,\;
E\fun\idsym}\}$.
\end{itemize}
E.g., we can read the production $\mathsf{E}\fun\mathsf{E}\plussym\mathsf{E}$
as ``an expression can consist of an expression, followed by
a $\plussym$ symbol, followed by an expression''.

We typically describe a grammar by listing its productions, and
grouping productions with identical left-sides into production
families.  Unless we say otherwise, the grammar's variables are the
left-sides of all of its productions, and its start variable is the
left-side of its first production.
Thus, our grammar $G$ is
\begin{align*}
\Esf &\fun \Esf\plussym\Esf , \\
\Esf &\fun \Esf\timessym\Esf , \\
\Esf &\fun \openparsym\Esf\closparsym , \\
\Esf &\fun \idsym ,
\end{align*}
or
\begin{gather*}
\mathsf{E\fun E\plussym E\mid E\timessym E\mid \openparsym E\closparsym \mid
\idsym} .
\end{gather*}

The Forlan syntax for grammars is very similar to the notation used
above.  E.g., our example grammar can be described in Forlan's syntax
by
\begin{verbatim}
{variables} E {start variable} E
{productions}
E -> E<plus>E; E -> E<times>E; E -> <openPar>E<closPar>; E -> <id>
\end{verbatim}
or
\begin{verbatim}
{variables} E {start variable} E
{productions}
E -> E<plus>E | E<times>E | <openPar>E<closPar> | <id>
\end{verbatim}
Production families are separated by semicolons.

The Forlan module \texttt{Gram} defines an abstract type \texttt{gram} (in
the top-level environment) of grammars as well as a number of
functions and constants for processing grammars, including:
\begin{verbatim}
val input          : string -> gram
val output         : string * gram -> unit 
val numVariables   : gram -> int
val numProductions : gram -> int
val equal          : gram * gram -> bool
\end{verbatim}
The functions \texttt{numVariables} and \texttt{numProductions} return
the numbers of variables and productions, respectively, of a grammar.
And the function \texttt{equal} tests whether two grammars are equal,
i.e., whether they have the same sets of variables, start variables,
and sets of productions.
During printing, Forlan merges productions into production families
whenever possible.

The \emph{alphabet of} a grammar $G$ ($\alphabet\,G$) is
\begin{align*}
&\quad\; \{\,a\in\Sym\mid\eqtxtr{there are}q,x\eqtxt{such that}q\fun x\in P_G
\eqtxt{and} \\
&\hspace{2.6cm} a\in\alphabet\,x \,\} \\
&- Q_G .
\end{align*}
I.e., $\alphabet\,G$ is all of the symbols appearing in the strings of
$G$'s productions that aren't variables.
For example, the alphabet of our example grammar $G$ is
$\{\mathsf{\plussym,\timessym,\openparsym,\closparsym,\idsym}\}$.

The Forlan module \texttt{Gram} defines a function
\begin{verbatim}
val alphabet : gram -> sym set
\end{verbatim}
for calculating the alphabet of a grammar.
E.g., if \texttt{gram} of type \texttt{gram} is bound to our example
grammar $G$, then Forlan will behave as follows:
\begin{list}{}
{\setlength{\leftmargin}{\leftmargini}
\setlength{\rightmargin}{0cm}
\setlength{\itemindent}{0cm}
\setlength{\listparindent}{0cm}
\setlength{\itemsep}{0cm}
\setlength{\parsep}{0cm}
\setlength{\labelsep}{0cm}
\setlength{\labelwidth}{0cm}
\catcode`\#=12
\catcode`\$=12
\catcode`\%=12
\catcode`\^=12
\catcode`\_=12
\catcode`\.=12
\catcode`\?=12
\catcode`\!=12
\catcode`\&=12
\ttfamily}
\small
\item[]\textsl{-\ }val\ bs\ =\ Gram.alphabet\ gram;
\item[]\textsl{val\ bs\ =\ -\ :\ sym\ set}
\item[]\textsl{-\ }SymSet.output("",\ bs);\ \ 
\item[]\textsl{<id>,\ <plus>,\ <times>,\ <closPar>,\ <openPar>}
\item[]\textsl{val\ it\ =\ ()\ :\ unit}
\end{list}


\subsection{Parse Trees and Grammar Meaning}

We will explain when strings are generated by grammars using the
notion of a parse tree.  The set $\PT$ of \emph{parse trees} is the
least subset of $\Tree(\Sym\cup\{\%\})$ (the set of all
$(\Sym\cup\{\%\})$-trees; see
Section~\ref{TreesAndInductiveDefinitions}) such that:
\begin{enumerate}[\quad(1)]
\item for all $a\in\Sym$ and $\pts\in\List\,\PT$, $(a,\pts)\in\PT$; and

\item for all $a\in\Sym$, $(a,[(\%,[\,])]) = a(\%)\in\PT$.
\end{enumerate}
Note that the $(\Sym\cup\{\%\})$-tree $\% = (\%,[\,])$ is
\emph{not} a parse tree.
It is easy to see that $\PT$ is countably infinite.

For example, $\Asf(\Bsf,\Asf(\%),\Bsf(\zerosf))$, i.e.,
\begin{center}
\input{chap-4.1-fig1.eepic}
\end{center}
is a parse tree.  On the other hand,
although $\Asf(\Bsf,\%,\Bsf)$, i.e.,
\begin{center}
\input{chap-4.1-fig2.eepic}
\end{center}
is a $(\Sym\cup\{\%\})$-tree, it's not a parse tree, since it
can't be formed using rules (1) and (2).

Since the set $\PT$ of parse trees is defined inductively, it gives
rise to an induction principle.

\begin{theorem}[Principle of Induction on Parse Trees]
Suppose $P(\pt)$ is a property of an element $\pt\in\PT$.

If
\begin{enumerate}[\quad(1)]
\item for all $a\in\Sym$ and $\trs\in\List\,\PT$, if $(\dag)$ for all
  $i\in[1:|\trs|]$, $P(\trs\,i)$, then $P((a,\trs))$, and

\item for all $a\in\Sym$, $P(a(\%))$,
\end{enumerate}
then
\begin{gather*}
\eqtxtr{for all}\pt\in\PT,P(\pt) .
\end{gather*}
\end{theorem}
We refer to $(\dag)$ as the inductive hypothesis.

We define the yield of a parse tree, as follows.  The function
$\yield\in\PT\fun\Str$ is defined by structural recursion:
\begin{itemize}
\item for all $a\in\Sym$, $\yield\,a=a$;

\item for all $q\in\Sym$, $n\in\nats-\{0\}$ and
$\pt_1,\,\ldots,\pt_n\in\PT$,
$\yield(q(\pt_1,\,\ldots,\pt_n)) =
\yield\,\pt_1\,\cdots\,\yield\,\pt_n$; and

\item for all $q\in\Sym$, $\yield(q(\%))=\%$.
\end{itemize}
We say that $w$ is the \emph{yield of} $\pt$ iff
$w=\yield\,\pt$.

For example, the yield of
\begin{center}
\input{chap-4.1-fig1.eepic}
\end{center}
is
\begin{gather*}
\yield\,\Bsf\,\yield(A(\%))\,\yield(\Bsf(\zerosf)) =
\Bsf\,\%\,\yield\,\zerosf=\Bsf\%\zerosf=\Bsf\zerosf .
\end{gather*}

We say when a parse tree is valid for a grammar $G$ as follows.
Define a function $\valid_G\in\PT\fun\Bool$ by
structural recursion:
\begin{itemize}
\item for all $a\in\Sym$, $\valid_G\,a=a\in
\alphabet\,G\myor a\in Q_G$;

\item for all $q\in\Sym$, $n\in\nats-\{0\}$ and
$\pt_1,\,\ldots,\pt_n\in\PT$,
\begin{align*}
&\quad\; \valid_G(q(\pt_1,\,\ldots,\pt_n)) \\
&= q\fun\rootLabel\,\pt_1\,\cdots\,\rootLabel\,\pt_n\in P_G \myand{} \\
&\quad\;
\valid_G\,\pt_1 \myand\, \cdots\, \myand \valid_G\,\pt_n; \eqtxt{and}
\end{align*}

\item for all $q\in\Sym$, $\valid_G(q(\%))=q\fun\%\in
P_G$.
\end{itemize}
We say that $\pt$ is \emph{valid for} $G$ iff $\valid_G\,\pt=\true$.
We often abbreviate $\valid_G$ to $\valid$.

Suppose $G$ is the grammar
\begin{align*}
\Asf &\fun \Bsf\Asf\Bsf \mid \% , \\
\Bsf &\fun \zerosf 
\end{align*}
(by convention, its variables are $\Asf$ and $\Bsf$ and its
start variable is $\Asf$).
Let's see why the parse tree $\Asf(\Bsf,\Asf(\%),\Bsf(\zerosf))$ is
valid for $G$:
\begin{itemize}
\item Since $\Asf\fun\Bsf\Asf\Bsf\in P_G$ and the concatenation
of the root labels of the sub-trees
$\Bsf$, $\Asf(\%)$ and $\Bsf(\zerosf)$ is $\Bsf\Asf\Bsf$,
the overall tree will be valid for $G$ if these sub-trees are valid
for $G$.

\item The parse tree $\Bsf$ is valid for $G$
since $\Bsf\in Q_G$.

\item Since $\Asf\fun\%\in P_G$, the parse tree
$\Asf(\%)$ is valid for $G$.

\item Since $\Bsf\fun\zerosf\in P_G$ and the root label of the sub-tree
$\zerosf$ is $\zerosf$, the parse tree
$\Bsf(\zerosf)$ will be valid for $G$ if the sub-tree $\zerosf$
is valid for $G$.

\item The sub-tree $\zerosf$ is valid for $G$ since
  $\zerosf\in\alphabet\,G$.
\end{itemize}
Thus, we have that
\begin{center}
\input{chap-4.1-fig1.eepic}
\end{center}
is valid for $G$.

And, if $G$ is our grammar of arithmetic expressions,
\begin{gather*}
\mathsf{E\fun E\plussym E\mid E\timessym E\mid \openparsym E\closparsym \mid
\idsym} ,
\end{gather*}
then the parse tree
\begin{center}
\input{chap-4.1-fig3.eepic}
\end{center}
is valid for $G$.

\begin{proposition}
Suppose $G$ is a grammar. For all parse trees $\pt$, if $\pt$ is valid
for $G$, then $\rootLabel\,\pt\in Q_G\cup\alphabet\,G$ and
$\yield\,\pt \in (Q_G\cup\alphabet\,G)^*$.
\end{proposition}

\begin{proof}
By induction on $\pt$.
\end{proof}

Suppose $G$ is a grammar, $w\in\Str$ and $a\in\Sym$.  We say that $w$
\emph{is parsable from} $a$ \emph{using} $G$ iff there is a parse tree
$\pt$ such that:
\begin{itemize}
\item $\pt$ is valid for $G$;

\item $a$ is the root label of $\pt$; and

\item the yield of $\pt$ is $w$.
\end{itemize}
Thus we will have that $w\in(Q_G\cup\alphabet\,G)^*$, and either
$a\in Q_G$ or $[a] = w$.
We say that a string $w$ \emph{is generated from} a variable $q\in Q_G$
\emph{using} $G$ iff $w\in(\alphabet\,G)^*$ and $w$ is parsable from
$q$.
And, we say that a string $w$ \emph{is generated by} a grammar $G$ iff
$w$ is generated from $s_G$ using $G$.
The \emph{language generated by} a grammar $G$ ($L(G)$) is
\begin{gather*}
\setof{w\in\Str}{w\eqtxt{is generated by}G}.
\end{gather*}

\begin{proposition}
For all grammars $G$, $\alphabet(L(G))\sub\alphabet\,G$.
\end{proposition}

For example, if $G$ is the grammar
\begin{align*}
\Asf &\fun \Bsf\Asf\Bsf \mid \% , \\
\Bsf &\fun \zerosf ,
\end{align*}
then $\zerosf\zerosf$ is generated by $G$,
since $\zerosf\zerosf\in\{\zerosf\}^*=(\alphabet\,G)^*$ and
the parse tree
\begin{center}
\input{chap-4.1-fig4.eepic}
\end{center}
is valid for $G$, has $s_G=\Asf$ as its root label,
and has $\zerosf\zerosf$ as its yield.
And, if $G$ is our grammar of arithmetic expressions,
\begin{gather*}
\mathsf{E\fun E\plussym E\mid E\timessym E\mid \openparsym E\closparsym \mid
\idsym} ,
\end{gather*}
then $\idsym\timessym\idsym\plussym\idsym$ is generated by $G$,
since $\idsym\timessym\idsym\plussym\idsym\in(\alphabet\,G)^*$ and
the parse tree
\begin{center}
\input{chap-4.1-fig3.eepic}
\end{center}
is valid for $G$, has $s_G=\Esf$ as its root label,
and has $\idsym\timessym\idsym\plussym\idsym$ as its yield.

A language $L$ is \emph{context-free} iff $L=L(G)$ for some
$G\in\Gram$.  We define
\begin{align*}
\CFLan &= \setof{L(G)}{G\in\Gram} \\
&= \setof{L\in\Lan}{L\eqtxtl{is context-free}} .
\end{align*}
Since $\{\mathsf{0}^0\}$, $\{\mathsf{0}^1\}$, $\{\mathsf{0}^2\}$,
\ldots, are all context-free languages, we have that $\CFLan$ is
infinite.  But, since $\Gram$ is countably infinite, it follows that
$\CFLan$ is also countably infinite.
Since $\Lan$ is uncountable, it follows that
$\CFLan\subsetneq\Lan$, i.e., there are non-context-free
languages.  Later, we will see that $\RegLan\subsetneq\CFLan$.

We say that grammars $G$ and $H$ are
\emph{equivalent} iff $L(G) = L(H)$.  In other words, $G$
and $H$ are equivalent iff $G$ and $H$ generate the same
language.  We define a relation $\approx$ on $\Gram$ by:
$G\approx H$ iff $G$ and $H$ are equivalent.  It is easy to see
that $\approx$ is reflexive on $\Gram$, symmetric and transitive.

The Forlan module \texttt{PT} defines an abstract type \texttt{pt} of
parse trees (in the top-level environment) along with some functions
for processing parse trees:
\begin{verbatim}
val input     : string -> pt
val output    : string * pt -> unit
val height    : pt -> int
val size      : pt -> int
val equal     : pt * pt -> bool
val rootLabel : pt -> sym
val yield     : pt -> str
\end{verbatim}
The Forlan syntax for parse trees is simply the linear syntax that
we've been using in this section.

The Java program JForlan, can be used to view and edit parse
trees.  It can be invoked directly, or run via
Forlan.  See the Forlan website for more information.

The Forlan module \texttt{Gram} also defines the functions
\begin{verbatim}
val checkPT : gram -> pt -> unit
val validPT : gram -> pt -> bool
\end{verbatim}
The function \texttt{checkPT} is used to check whether a parse tree
is valid for a grammar; if the answer is ``no'', it explains why not
and raises an exception; otherwise it simply returns \texttt{()}.
The function \texttt{validPT} checks whether a parse tree is valid
for a grammar, silently returning \texttt{true} if it is, and silently
returning \texttt{false} if it isn't.

Suppose the identifier \texttt{gram} of type \texttt{gram} is bound to the
grammar
\begin{align*}
\Asf &\fun \Bsf\Asf\Bsf \mid \% , \\
\Bsf &\fun \zerosf .
\end{align*}
And, suppose that the identifier
\texttt{gram'} of type \texttt{gram} is bound to our grammar of
arithmetic expressions,
\begin{gather*}
\mathsf{E\fun E\plussym E\mid E\timessym E\mid \openparsym E\closparsym \mid
\idsym} .
\end{gather*}
Here are some examples of how we can process parse trees using Forlan:
\begin{list}{}
{\setlength{\leftmargin}{\leftmargini}
\setlength{\rightmargin}{0cm}
\setlength{\itemindent}{0cm}
\setlength{\listparindent}{0cm}
\setlength{\itemsep}{0cm}
\setlength{\parsep}{0cm}
\setlength{\labelsep}{0cm}
\setlength{\labelwidth}{0cm}
\catcode`\#=12
\catcode`\$=12
\catcode`\%=12
\catcode`\^=12
\catcode`\_=12
\catcode`\.=12
\catcode`\?=12
\catcode`\!=12
\catcode`\&=12
\ttfamily}
\small
\item[]\textsl{-\ }val\ pt\ =\ PT.input\ "";
\item[]\textsl{@\ }A(B,\ A(%),\ B(0))
\item[]\textsl{@\ }.
\item[]\textsl{val\ pt\ =\ -\ :\ pt}
\item[]\textsl{-\ }Sym.output("",\ PT.rootLabel\ pt);
\item[]\textsl{A}
\item[]\textsl{val\ it\ =\ ()\ :\ unit}
\item[]\textsl{-\ }Str.output("",\ PT.yield\ pt);
\item[]\textsl{B0}
\item[]\textsl{val\ it\ =\ ()\ :\ unit}
\item[]\textsl{-\ }Gram.validPT\ gram\ pt;
\item[]\textsl{val\ it\ =\ true\ :\ bool}
\item[]\textsl{-\ }val\ pt'\ =\ PT.input\ "";
\item[]\textsl{@\ }E(E(E(<id>),\ <times>,\ E(<id>)),\ <plus>,\ E(<id>))
\item[]\textsl{@\ }.
\item[]\textsl{val\ pt'\ =\ -\ :\ pt}
\item[]\textsl{-\ }Sym.output("",\ PT.rootLabel\ pt');
\item[]\textsl{E}
\item[]\textsl{val\ it\ =\ ()\ :\ unit}
\item[]\textsl{-\ }Str.output("",\ PT.yield\ pt');
\item[]\textsl{<id><times><id><plus><id>}
\item[]\textsl{val\ it\ =\ ()\ :\ unit}
\item[]\textsl{-\ }Gram.validPT\ gram'\ pt';
\item[]\textsl{val\ it\ =\ true\ :\ bool}
\item[]\textsl{-\ }Gram.checkPT\ gram\ pt';
\item[]\textsl{invalid\ production:\ "E\ ->\ E<plus>E"}
\item[]
\item[]\textsl{uncaught\ exception\ Error}
\item[]\textsl{-\ }Gram.checkPT\ gram'\ pt;
\item[]\textsl{invalid\ production:\ "A\ ->\ BAB"}
\item[]
\item[]\textsl{uncaught\ exception\ Error}
\item[]\textsl{-\ }PT.input\ "";
\item[]\textsl{@\ }A(B,%,B)
\item[]\textsl{@\ }.
\item[]\textsl{line\ 1:\ "%"\ unexpected}
\item[]
\item[]\textsl{uncaught\ exception\ Error}
\end{list}


\subsection{Grammar Synthesis}

We conclude this section with a grammar synthesis example.  Suppose
$X=\setof{\zerosf^n\onesf^m\twosf^m\threesf^n}{n,m\in\nats}$.  The key
to finding a grammar $G$ that generates $X$ is to think of generating
the strings of $X$ from the outside in, in two phases.  In the first
phase, one generates pairs of $\zerosf$'s and $\threesf$'s, and, in
the second phase, one generates pairs of $\onesf$'s and $\twosf$'s.
E.g., a string could be formed in the following stages:
\begin{gather*}
{\mathsf{0\hspace{.6cm}3,}} \\
{\mathsf{00\hspace{.34cm}33,}} \\
{\mathsf{001233.}}
\end{gather*}

This analysis leads us to the grammar
\begin{align*}
\Asf &\fun \zerosf\Asf\threesf , \\
\Asf &\fun \Bsf , \\
\Bsf &\fun \onesf\Bsf\twosf , \\
\Bsf &\fun \% ,
\end{align*}
where $\Asf$ corresponds to the first phase, and $\Bsf$ to the
second phase.
For example, here is how the string $\mathsf{001233}$ may be
parsed using $G$:
\begin{center}
\input{chap-4.1-fig5.eepic}
\end{center}

\begin{exercise}
Let
\begin{displaymath}
X = \setof{\zerosf^i\onesf^j\twosf^k}{i,j,k\in\nats\eqtxt{and}
(i\neq j\eqtxt{or}j\neq k)} .
\end{displaymath}
Find a grammar $G$ such that $L(G)=X$.
\end{exercise}

\subsection{Notes}

Traditionally, the meaning of grammars is defined using derivations,
with parse trees being introduced subsequently.  In contrast, we have
no need for derivations, and find parse trees a much more intuitive
way to define the meaning of grammars.

Pleasingly, parse trees are an instance of the trees introduced in
Section~\ref{TreesAndInductiveDefinitions}.  Thus the terminology,
techniques and results of that section are applicable to them.

%%% Local Variables: 
%%% mode: latex
%%% TeX-master: "book"
%%% End: 

\section{Isomorphism of Grammars}
\label{IsomorphismOfGrammars}

\index{isomorphism!grammar|(}%
\index{grammar!isomorphism|(}%

In this section, we study grammar isomorphism, i.e., the way in
which grammars can have the same structure, even though they may have
different variables.

\subsection{Definition and Algorithm}

Suppose $G$ is the grammar with variables $\Asf$ and $\Bsf$,
start variable $\Asf$ and productions:
\begin{align*}
  \Asf &\fun \zerosf\Asf\onesf \mid \Bsf , \\
  \Bsf &\fun \% \mid \twosf\Asf .
\end{align*}
And, suppose $H$ is the grammar with variables $\Bsf$ and $\Asf$,
start variable $\Bsf$ and productions:
\begin{align*}
  \Bsf &\fun \zerosf\Bsf\onesf \mid \Asf , \\
  \Asf &\fun \% \mid \twosf\Bsf .
\end{align*}
$H$ can be formed from $G$ by renaming the variables $\Asf$ and $\Bsf$
of $G$ to $\Bsf$ and $\Asf$, respectively.  As a result, we say that
$G$ and $H$ are isomorphic.

Suppose $G$ is as before, but that $H$ is the grammar with variables
$\twosf$ and $\Asf$, start variable $\twosf$ and productions:
\begin{align*}
  \twosf &\fun \zerosf\twosf\onesf \mid \Asf , \\
  \Asf &\fun \% \mid \twosf\twosf .
\end{align*}
Then $H$ can be formed from $G$ by renaming the variables $\Asf$ and
$\Bsf$ to $\twosf$ and $\Asf$, respectively.  But we shouldn't
consider $G$ and $H$ to be isomorphic, since the symbol $\twosf$ is in
both $\alphabet\,G$ and $Q_H$.  In fact, $G$ and $H$ generate
different languages.  A grammar's variables (e.g., $\Asf$) can't be
renamed to elements of the grammar's alphabet (e.g., $\twosf$).

An \emph{isomorphism} $h$ from a grammar $G$ to a grammar $H$ is
a bijection from $Q_G$ to $Q_H$ such that:
\begin{itemize}
\item $h$ turns $G$ into $H$; and

\item $\alphabet\,G\cap Q_H=\emptyset$, i.e., none of the symbols in
  $G$'s alphabet are variables of $H$.
\end{itemize}
We say that $G$ and $H$ are \emph{isomorphic} iff there is an
isomorphism between $G$ and $H$.

As expected, we have that the relation of being isomorphic is
reflexive on $\Gram$, symmetric and transitive, and that isomorphism
implies having the same alphabet and equivalence.

There is an algorithm for finding an isomorphism from one grammar to
another, if one exists, or reporting that there is no such
isomorphism.  It's similar to the algorithm for finding an isomorphism
between finite automata of Section~\ref{IsomorphismOfFiniteAutomata}.

\index{grammar!renameVariables@$\renameVariables$}%
The function $\renameVariables$ takes in a pair $(G,f)$, where $G$ is
a grammar and $f$ is a bijection from $Q_G$ to a set of symbols with
the property that $\range\,f\cap\alphabet\,G=\emptyset$, and returns
the grammar produced from $G$ by renaming $G$'s variables using the
bijection $f$.  The resulting grammar will be isomorphic to $G$.

The following function is a special case of $\renameVariables$.
The function $\renameVariablesCanonically\in\Gram\fun\Gram$ renames the
\index{grammar!renameVariablesCanonically@$\renameVariablesCanonically$}%
variables of a grammar $G$ to:
\begin{itemize}
\item $\mathsf{A}$, $\mathsf{B}$, etc., when the grammar has no more
  than 26 variables (the smallest variable of $G$ will be renamed to
  $\mathsf{A}$, the next smallest one to $\mathsf{B}$, etc.); or

\item $\mathsf{\langle 1\rangle}$, $\mathsf{\langle 2\rangle}$, etc.,
  otherwise.
\end{itemize}
These variables will actually be surrounded by a uniform number of
extra brackets, if this is needed to make the new grammar's variables
and the original grammar's alphabet be disjoint.

\subsection{Isomorphism Finding/Checking in Forlan}

The Forlan module \texttt{Gram} contains the following functions
for finding and processing isomorphisms in Forlan:
\begin{verbatim}
val isomorphism                : gram * gram * sym_rel -> bool
val findIsomorphism            : gram * gram -> sym_rel
val isomorphic                 : gram * gram -> bool
val renameVariables            : gram * sym_rel -> gram
val renameVariablesCanonically : gram -> gram
\end{verbatim}
\index{Gram@\texttt{Gram}!isomorphism@\texttt{isomorphism}}%
\index{Gram@\texttt{Gram}!findIsomorphism@\texttt{findIsomorphism}}%
\index{Gram@\texttt{Gram}!isomorphic@\texttt{isomorphic}}%
\index{Gram@\texttt{Gram}!renameVariables@\texttt{renameVariables}}%
\index{Gram@\texttt{Gram}!renameVariablesCanonically@\texttt{renameVariablesCanoically}}%
The function \texttt{isomorphism} checks whether a relation on symbols
is an isomorphism from one grammar to another.  The function
\texttt{findIsomorphism} tries to find an isomorphism from one grammar
to another; it issues an error message if there isn't one.  The
function \texttt{isomorphic} checks whether two grammars are
isomorphic.  The function \texttt{renameVariables} issues an error
message if the supplied relation isn't a bijection from the set of
variables of the supplied grammar to some set; otherwise, it returns
the result of $\renameVariables$.  And the function
\texttt{renameVariablesCanonically} acts like
$\renameVariablesCanonically$.

Suppose the identifier \texttt{gram} of type \texttt{gram} is bound to the
grammar with variables $\Asf$ and $\Bsf$, start variable $\Asf$ and
productions:
\begin{align*}
\Asf &\fun \zerosf\Asf\onesf \mid \Bsf , \\
\Bsf &\fun \% \mid \twosf\Asf .
\end{align*}
Suppose the identifier \texttt{gram'} of type \texttt{gram} is bound to
the grammar with variables $\Bsf$ and $\Asf$, start variable $\Bsf$
and productions:
\begin{align*}
\Bsf &\fun \zerosf\Bsf\onesf \mid \Asf , \\ \Asf &\fun \% \mid
\twosf\Bsf .
\end{align*}
And, suppose the identifier \texttt{gram''} of type \texttt{gram} is bound
to the grammar with variables $\twosf$ and $\Asf$, start variable
$\twosf$ and productions:
\begin{align*}
\twosf &\fun \zerosf\twosf\onesf \mid \Asf , \\
\Asf &\fun \% \mid \twosf\twosf .
\end{align*}

Here are some examples of how the above functions can be used:
\begin{list}{}
{\setlength{\leftmargin}{\leftmargini}
\setlength{\rightmargin}{0cm}
\setlength{\itemindent}{0cm}
\setlength{\listparindent}{0cm}
\setlength{\itemsep}{0cm}
\setlength{\parsep}{0cm}
\setlength{\labelsep}{0cm}
\setlength{\labelwidth}{0cm}
\catcode`\#=12
\catcode`\$=12
\catcode`\%=12
\catcode`\^=12
\catcode`\_=12
\catcode`\.=12
\catcode`\?=12
\catcode`\!=12
\catcode`\&=12
\ttfamily}
\small
\item[]\textsl{-\ }val\ rel\ =\ Gram.findIsomorphism(gram,\ gram');
\item[]\textsl{val\ rel\ =\ -\ :\ sym_rel}
\item[]\textsl{-\ }SymRel.output("",\ rel);
\item[]\textsl{(A,\ B),\ (B,\ A)}
\item[]\textsl{val\ it\ =\ ()\ :\ unit}
\item[]\textsl{-\ }Gram.isomorphism(gram,\ gram',\ rel);
\item[]\textsl{val\ it\ =\ true\ :\ bool}
\item[]\textsl{-\ }Gram.isomorphic(gram,\ gram'');
\item[]\textsl{val\ it\ =\ false\ :\ bool}
\item[]\textsl{-\ }Gram.isomorphic(gram',\ gram'');
\item[]\textsl{val\ it\ =\ false\ :\ bool}
\item[]\textsl{-\ }val\ gram\ =\ Gram.input\ "";
\item[]\textsl{@\ }\symbol{'173}variables\symbol{'175}\ B,\ C
\item[]\textsl{@\ }\symbol{'173}start\ variable\symbol{'175}\ B
\item[]\textsl{@\ }\symbol{'173}productions\symbol{'175}\ B\ ->\ AC;\ C\ ->\ <A>
\item[]\textsl{@\ }.
\item[]\textsl{val\ gram\ =\ -\ :\ gram}
\item[]\textsl{-\ }SymSet.output("",\ Gram.alphabet\ gram);
\item[]\textsl{A,\ <A>}
\item[]\textsl{val\ it\ =\ ()\ :\ unit}
\item[]\textsl{-\ }Gram.output("",\ Gram.renameVariablesCanonically\ gram);
\item[]\textsl{\symbol{'173}variables\symbol{'175}\ <<A>>,\ <<B>>\ \symbol{'173}start\ variable\symbol{'175}\ <<A>>}
\item[]\textsl{\symbol{'173}productions\symbol{'175}\ <<A>>\ ->\ A<<B>>;\ <<B>>\ ->\ <A>}
\item[]\textsl{val\ it\ =\ ()\ :\ unit}
\end{list}


\subsection{Notes}

Considering grammar isomorphism is non-traditional, but relatively
straightforward.  We were led to doing so mostly because of the need to
support variable renaming.

\index{isomorphism!grammar|(}%
\index{grammar!isomorphism|(}%

%%% Local Variables: 
%%% mode: latex
%%% TeX-master: "book"
%%% End: 

\section{A Parsing Algorithm}
\label{AParsingAlgorithm}

\index{grammar!parsing algorithm|(}%
In this section, we consider a simple, fairly inefficient parsing
algorithm that works for all context-free grammars.  In
Section~\ref{AmbiguityOfGrammars}, we consider an efficient parsing
method that works for grammars for languages of operators of varying
precedences and associativities.  Compilers courses cover efficient
algorithms that work for various subsets of the context free grammars.

\subsection{Algorithm}

Suppose $G$ is a grammar, $w\in\Str$ and $a\in\Sym$.  We consider an
algorithm for testing whether $w$ is parsable from $a$ using $G$.
If $w\not\in(Q_G\cup\alphabet\,G)^*$ or $a\not\in Q_G\cup\alphabet\,w$,
then the algorithm returns $\false$.
Otherwise, it proceeds as follows.

Let $A=Q_G\cup\alphabet\,w$ and $B=\setof{x\in\Str}{x\eqtxt{is a substring
of}w}$.
The algorithm generates the least subset $X$ of $A\times B$ such that:
\begin{enumerate}[\quad(1)]
\item For all $a\in\alphabet\,w$, $(a,a)\in X$;

\item For all $q\in Q_G$, if $q\fun\%\in P_G$, then
  $(q,\%)\in X$; and

\item For all $q\in Q_G$, $n\in\nats-\{0\}$, $a_1,\ldots,a_n\in A$ and
  $x_1,\ldots,x_n\in B$, if
  \begin{itemize}
  \item $q\fun a_1\cdots a_n\in P_G$,

  \item for all $i\in[1:n]$, $(a_i,x_i)\in X$, and

  \item $x_1\cdots x_n\in B$,
  \end{itemize}
then $(q,x_1\cdots x_n)\in X$.
\end{enumerate}
Since $A\times B$ is finite, this process terminates.

For example, let $G$ be the grammar
\begin{align*}
\Asf &\fun \Bsf\Csf\mid \Csf\Dsf , \\
\Bsf &\fun \zerosf\mid \Csf\Bsf , \\
\Csf &\fun \onesf\mid \Dsf\Dsf , \\
\Dsf &\fun \zerosf\mid \Bsf\Csf ,
\end{align*}
and let $w=\mathsf{0010}$ and $a=\Asf=s_G$.
We have that:
\begin{itemize}
\item $(\zerosf,\zerosf)\in X$;

\item $(\onesf, \onesf)\in X$;

\item $(\Bsf,\zerosf)\in X$, since $\Bsf\fun\zerosf\in P_G$,
$(\zerosf,\zerosf)\in X$ and $\zerosf\in B$;

\item $(\Csf,\onesf)\in X$, since $\Csf\fun\onesf\in P_G$,
$(\onesf,\onesf)\in X$ and $\onesf\in B$;

\item $(\Dsf,\zerosf)\in X$, since $\Dsf\fun\zerosf\in P_G$,
$(\zerosf,\zerosf)\in X$ and $\zerosf\in B$;

\item $(\Asf,\mathsf{01})\in X$, since $\Asf\fun\mathsf{BC}\in P_G$,
$(\Bsf,\zerosf)\in X$, $(\Csf,\onesf)\in X$ and
$\mathsf{01}\in B$;

\item $(\Asf,\mathsf{10})\in X$, since $\Asf\fun\mathsf{CD}\in P_G$,
$(\Csf,\onesf)\in X$, $(\Dsf,\zerosf)\in X$ and $\mathsf{10}\in B$;

\item $(\Bsf,\mathsf{10})\in X$, since $\Bsf\fun\mathsf{CB}\in P_G$,
$(\Csf,\onesf)\in X$, $(\Bsf,\zerosf)\in X$ and $\mathsf{10}\in B$;

\item $(\Csf,\mathsf{00})\in X$, since $\Csf\fun\mathsf{DD}\in P_G$,
$(\Dsf,\zerosf)\in X$, $(\Dsf,\zerosf)\in X$ and $\mathsf{00}\in B$;

\item $(\Dsf,\mathsf{01})\in X$, since $\Dsf\fun\mathsf{BC}\in P_G$,
$(\Bsf,\zerosf)\in X$, $(\Csf,\onesf)\in X$ and $\mathsf{01}\in B$;

\item $(\Csf,\mathsf{001})\in X$, since $\Csf\fun\mathsf{DD}\in P_G$,
$(\Dsf,\mathsf{0})\in X$, $(\Dsf,\mathsf{01})\in X$ and
$\mathsf{0(01)}\in B$;

\item $(\Csf,\mathsf{010})\in X$, since $\Csf\fun\mathsf{DD}\in P_G$,
$(\Dsf,\mathsf{01})\in X$, $(\Dsf,\mathsf{0})\in X$ and
$\mathsf{(01)0}\in B$;

\item $(\Asf,\mathsf{0010})\in X$, since $\Asf\fun\mathsf{BC}\in P_G$,
$(\Bsf,\mathsf{0})\in X$, $(\Csf,\mathsf{010})\in X$ and
$\mathsf{0(010)}\in B$;

\item $(\Bsf,\mathsf{0010})\in X$, since $\Bsf\fun\mathsf{CB}\in P_G$,
$(\Csf,\mathsf{00})\in X$, $(\Bsf,\mathsf{10})\in X$ and
$\mathsf{(00)(10)}\in B$;

\item $(\Dsf,\mathsf{0010})\in X$, since $\Dsf\fun\mathsf{BC}\in P_G$,
$(\Bsf,\mathsf{0})\in X$, $(\Csf,\mathsf{010})\in X$ and
$\mathsf{0(010)}\in B$;

\item Nothing more can be added to $X$.  To verify this, one must
check that nothing new can be added to $X$ using rule~(3).
\end{itemize}

Back in the general case, we have these lemmas:

\begin{lemma}
For all $(b,x)\in X$, there is a $\pt\in\PT$ such that
\begin{itemize}
\item $\pt$ is valid for $G$,

\item $\rootLabel\,\pt=b$, and

\item $\yield\,\pt=x$.
\end{itemize}
\end{lemma}

\begin{lemma}
For all $\pt\in\PT$, if
\begin{itemize}
\item $\pt$ is valid for $G$,

\item $\rootLabel\,\pt\in A$, and

\item $\yield\,\pt\in B$,
\end{itemize}
then $(\rootLabel\,\pt, \yield\,\pt)\in X$.
\end{lemma}

Because of our lemmas, to determine if $w$ is parsable from $a$, we
just have to check whether $(a, w)\in X$.
In the case of our example grammar, we have that $w=\mathsf{0010}$ is
parsable from $a=\Asf$, since $(\Asf,\mathsf{0010})\in X$.
Hence $\mathsf{0010}\in L(G)$.

Note that any production whose right-hand side contains an element of
$\alphabet\,G-\alphabet\,w$ won't affect the generation of $X$.  Thus
our algorithm ignores such productions.

Furthermore, our parsability algorithm actually generates $X$ in a
sequence of stages. At each point, it has subsets $U$ and $V$ of
$A\times B$. $U$ consists of the older elements of $X$, whereas $V$
consists of the most recently added elements.
\begin{itemize}
\item First, it lets $U=\emptyset$ and sets $V$ to be the union of
  $\setof{(a,a)}{a\in\alphabet\,w}$ and $\setof{(q,\%)}{(q,\%)\in
    P_G}$. It then enters its main loop.

\item At a stage of the loop's iteration, it lets $Y$ be $U\cup V$,
  and then lets $Z$ be the set of all $(q,x_1\cdots x_n)$ such that
  $n\geq 1$ and there are $a_1,\ldots,a_n\in A$ and $i\in[1:n]$ such
  that
  \begin{itemize}
  \item $q\fun a_1\cdots a_n\in P_G$,

  \item $(a_i,x_i)\in V$,

  \item for all $k\in[1:n]-\{i\}$, $(a_k,x_k)\in Y$,

  \item $x_1\cdots x_n\in B$, and

  \item $(q,x_1\cdots x_n)\not\in Y$.
  \end{itemize}
  If $Z\neq\emptyset$, then it sets $U$ to $Y$, and $V$ to $Z$, and
  repeats; Otherwise, the result is $Y$.
\end{itemize}

\index{grammar!minimal parse}%
We say that a parse tree $\pt$ is \emph{a minimal parse} of a string
$w$ \emph{from} a symbol $a$ \emph{using} a grammar $G$ iff $\pt$ is
valid for $G$, $\rootLabel\,\pt = a$ and $\yield\,\pt = w$, and there
is no strictly smaller $\pt'\in\PT$ such that $\pt'$ is valid for $G$,
$\rootLabel\,\pt' = a$ and $\yield\,\pt' = w$.

We can convert our parsability algorithm into a parsing algorithm as
follows.  Given $w\in(Q_G\cup\alphabet\,G)^*$ and
$a\in(Q_G\cup\alphabet\,w)$, we generate our set $X$ as before, but we
annotate each element $(b,x)$ of $X$ with a parse tree $\pt$ such that
\begin{itemize}
\item $\pt$ is valid for $G$,

\item $\rootLabel\,\pt=b$, and

\item $\yield\,\pt=x$,
\end{itemize}
Thus we can return the parse tree labeling $(a, w)$, if this pair is
in $X$, and indicate failure otherwise.

With a little more work, we can arrange that the parse trees returned
by our parsing algorithm are minimally-sized, and this is what the
official version of our parsing algorithm guarantees.  This goal is a
little tricky to achieve, since some pairs will first be labeled by
parse trees that aren't minimally sized. But we keep going as long as
either new pairs are found, or smaller parse trees are found for
existing pairs.

\subsection{Parsing in Forlan}

The Forlan module \texttt{Gram} defines the functions
\begin{verbatim}
val parsable              : gram -> sym * str -> bool
val generatedFromVariable : gram -> sym * str -> bool
val generated             : gram -> str -> bool
\end{verbatim}
\index{Gram@\texttt{Gram}!parsable@\texttt{parsable}}%
\index{Gram@\texttt{Gram}!generatedFromVariable@\texttt{generatedFromVariable}}%
\index{Gram@\texttt{Gram}!generated@\texttt{generated}}%
The function \texttt{parsable} tests whether a string $w$ is parsable
from a symbol $a$ using a grammar $G$.  The function
\texttt{generatedFromVariable} tests whether a string $w$ is generated
from a variable $q$ using a grammar $G$; it issues an error message if
$q$ isn't a variable of $G$.  And the function \texttt{generated}
tests whether a string $w$ is generated by a grammar $G$.

\texttt{Gram} also includes:
\begin{verbatim}
val parse                     : gram -> sym * str -> pt
val parseAlphabetFromVariable : gram -> sym * str -> pt
val parseAlphabet             : gram -> str -> pt
\end{verbatim}
\index{Gram@\texttt{Gram}!parse@\texttt{parse}}%
\index{Gram@\texttt{Gram}!parseAlphabetFromVariable@\texttt{parseAlphabetFromVarable}}%
\index{Gram@\texttt{Gram}!parseAlphabet@\texttt{parseAlphabet}}%
The function \texttt{parse} tries to find a minimal parse of a string
$w$ from a symbol $a$ using a grammar $G$; it issues an error message
if $w\not\in(Q_G\cup\alphabet\,G)^*$, or $a\not\in
Q_G\cup\alphabet\,w$, or such a parse doesn't exist.  The function
\texttt{parseAlphabetFromVariable} tries to find a minimal parse of a
string $w\in(\alphabet\,G)^*$ from a variable $q$ using a grammar $G$;
it issues an error message if $q\not\in Q_G$, or
$w\not\in(\alphabet\,G)^*$, or such a parse doesn't exist.  And the
function \texttt{parseAlphabet} tries to find a minimal parse of a
string $w\in(\alphabet\,G)^*$ from $s_G$ using a grammar $G$; it
issues an error message if $w\not\in(\alphabet\,G)^*$, or such a parse
doesn't exist.

Suppose that \texttt{gram} of type \texttt{gram} is bound to the grammar
\begin{align*}
\Asf &\fun \Bsf\Csf\mid \Csf\Dsf , \\
\Bsf &\fun \zerosf\mid \Csf\Bsf , \\
\Csf &\fun \onesf\mid \Dsf\Dsf , \\
\Dsf &\fun \zerosf\mid \Bsf\Csf .
\end{align*}
We can check whether some strings are generated by this grammar
as follows:
\begin{list}{}
{\setlength{\leftmargin}{\leftmargini}
\setlength{\rightmargin}{0cm}
\setlength{\itemindent}{0cm}
\setlength{\listparindent}{0cm}
\setlength{\itemsep}{0cm}
\setlength{\parsep}{0cm}
\setlength{\labelsep}{0cm}
\setlength{\labelwidth}{0cm}
\catcode`\#=12
\catcode`\$=12
\catcode`\%=12
\catcode`\^=12
\catcode`\_=12
\catcode`\.=12
\catcode`\?=12
\catcode`\!=12
\catcode`\&=12
\ttfamily}
\small
\item[]\textsl{-\ }Gram.generated\ gram\ (Str.fromString\ "0010");
\item[]\textsl{val\ it\ =\ true\ :\ bool}
\item[]\textsl{-\ }Gram.generated\ gram\ (Str.fromString\ "0100");
\item[]\textsl{val\ it\ =\ true\ :\ bool}
\item[]\textsl{-\ }Gram.generated\ gram\ (Str.fromString\ "0101");
\item[]\textsl{val\ it\ =\ false\ :\ bool}
\end{list}

And we can try to find parses of some strings as follows:
\begin{list}{}
{\setlength{\leftmargin}{\leftmargini}
\setlength{\rightmargin}{0cm}
\setlength{\itemindent}{0cm}
\setlength{\listparindent}{0cm}
\setlength{\itemsep}{0cm}
\setlength{\parsep}{0cm}
\setlength{\labelsep}{0cm}
\setlength{\labelwidth}{0cm}
\catcode`\#=12
\catcode`\$=12
\catcode`\%=12
\catcode`\^=12
\catcode`\_=12
\catcode`\.=12
\catcode`\?=12
\catcode`\!=12
\catcode`\&=12
\ttfamily}
\small
\item[]\textsl{-\ }fun\ test\ s\ =
\item[]\textsl{=\ }\ \ \ \ \ \ PT.output
\item[]\textsl{=\ }\ \ \ \ \ \ ("",
\item[]\textsl{=\ }\ \ \ \ \ \ \ Gram.parseAlphabet\ gram\ (Str.fromString\ s));
\item[]\textsl{val\ test\ =\ fn\ :\ string\ ->\ unit}
\item[]\textsl{-\ }test\ "0010";
\item[]\textsl{A(C(D(0),\ D(B(0),\ C(1))),\ D(0))}
\item[]\textsl{val\ it\ =\ ()\ :\ unit}
\item[]\textsl{-\ }test\ "0100";
\item[]\textsl{A(C(D(B(0),\ C(1)),\ D(0)),\ D(0))}
\item[]\textsl{val\ it\ =\ ()\ :\ unit}
\item[]\textsl{-\ }test\ "0101";
\item[]\textsl{no\ such\ parse\ exists}
\item[]
\item[]\textsl{uncaught\ exception\ Error}
\end{list}

But we can also check parsability of strings containing variables, as well
as try to find parses of such strings:
\begin{list}{}
{\setlength{\leftmargin}{\leftmargini}
\setlength{\rightmargin}{0cm}
\setlength{\itemindent}{0cm}
\setlength{\listparindent}{0cm}
\setlength{\itemsep}{0cm}
\setlength{\parsep}{0cm}
\setlength{\labelsep}{0cm}
\setlength{\labelwidth}{0cm}
\catcode`\#=12
\catcode`\$=12
\catcode`\%=12
\catcode`\^=12
\catcode`\_=12
\catcode`\.=12
\catcode`\?=12
\catcode`\!=12
\catcode`\&=12
\ttfamily}
\small
\item[]\textsl{-\ }Gram.parsable\ gram
\item[]\textsl{=\ }(Sym.fromString\ "A",\ Str.fromString\ "0D0C");
\item[]\textsl{val\ it\ =\ true\ :\ bool}
\item[]\textsl{-\ }PT.output
\item[]\textsl{=\ }("",
\item[]\textsl{=\ }\ Gram.parse\ gram
\item[]\textsl{=\ }\ (Sym.fromString\ "A",\ Str.fromString\ "0D0C"));
\item[]\textsl{A(C(D(0),\ D),\ D(B(0),\ C))}
\item[]\textsl{val\ it\ =\ ()\ :\ unit}
\end{list}


\subsection{Notes}

Our parsability and parsing algorithms are straightforward generalizations of
the familiar algorithm for checking whether a grammar in Chomsky Normal Form
generates a string.

\index{grammar!parsing algorithm|)}%

%%% Local Variables: 
%%% mode: latex
%%% TeX-master: "book"
%%% End: 

\section{Simplification of Grammars}
\label{SimplificationOfGrammars}

In this section, we say what it means for a grammar to be simplified,
give a simplification algorithm for grammars, and see how to use this
algorithm in Forlan.

\subsection{Definition and Algorithm}

Suppose $G$ is the grammar
\begin{align*}
\Asf &\fun \mathsf{BB1}, \\
\Bsf &\fun \zerosf\mid\Asf\mid\Csf\Dsf, \\
\Csf &\fun \onesf\twosf , \\
\Dsf &\fun \onesf\Dsf\twosf .
\end{align*}
This grammar is odd for two, related, reasons.  First
$\Dsf$ doesn't generate anything, i.e., there is no parse
tree $\pt$ such that $\pt$ is valid for $G$, the root label of
$\pt$ is $\Dsf$, and the yield of $\pt$ is in $(\alphabet\,G)^* =
\{\mathsf{0,1,2}\}^*$.
Second, there is no valid parse tree that starts at $G$'s
start variable, $\Asf$, has a yield that is in $\{\mathsf{0,1,2}\}^*$
and makes use of $\Csf$.  But if we first removed $\Dsf$, and all productions
involving it, then our objection to $\Csf$ could be simpler: there
would be no parse tree $\pt$ such that $\pt$ is valid for $G$,
the root label of $\pt$ is $\Asf$, and $\Csf$ appears in the yield
of $\pt$.

This example leads us to the following definitions.  Suppose $G$ is a
grammar.  We say that a variable $q$ of $G$ is:
\begin{itemize}
\item \emph{reachable in} $G$ iff there is a $w\in\Str$ such that
  $w$ is parsable from $s_G$ using $G$, and $a\in\alphabet\,w$;

\item \emph{generating in} $G$ iff there is a $w\in\Str$ such that $q$
  generates $w$ using $G$, i.e., $w$ is parsable from $q$ using $G$,
  and $w\in(\alphabet\,G)^*$;

\item \emph{useful in} $G$ iff $q$ is both reachable and
  generating in $G$.
\end{itemize}
The reader should compare these definitions with the definitions given
in Section~\ref{SimplificationOfFiniteAutomata} of reachable, live and
useful states.

Also note that the standard definition of being useful is stronger
than our definition: there is a parse tree $\pt$ such that $\pt$ is
valid for $G$, $\rootLabel\,\pt=s_G$, $\yield\,\pt\in
(\alphabet\,G)^*$, and $q$ appears in $\pt$.  For example, the
variable $\Csf$ of our example grammar $G$ is useful in our sense, but
not useful in the standard sense.  But as we observed above, $\Csf$
will no longer be reachable (and thus useful) if all productions
involving $\Dsf$ are removed.  In general, we have that, if all
variables of a grammar are useful in our sense, that all variable of
the grammar are useful in the standard sense.

Now, suppose $H$ is the grammar
\begin{align*}
\Asf &\fun \% \mid \zerosf \mid \mathsf{AA} \mid \mathsf{AAA} .
\end{align*}
Here, we have that the productions $\Asf\fun\mathsf{AA}$ and
$\Asf\fun\mathsf{AAA}$ are redundant, although only one of them can be
removed:
\begin{center}
  \input{chap-4.4-fig1.eepic}
\end{center}
Thus any use of $\Asf\fun\mathsf{AA}$ in a parse tree can be replaced
by uses of $\Asf\fun\%$ and $\Asf\fun\mathsf{AAA}$, and any use of
$\Asf\fun\mathsf{AAA}$ in a parse tree can be replaced by two uses
of $\Asf\fun\mathsf{AA}$.

This example leads us to the following definitions.  Given a grammar
$G$ and a finite subset $U$ of $\setof{(q,x)}{q\in Q_G\eqtxt{and}
  x\in\Str}$, we write $G/U$ for the grammar that is identical to $G$
except that its set of productions is $U$.
If $G$ is a grammar and $(q,x)\in P_G$, we say that:
\begin{itemize}
\item $(q,x)$ \emph{is redundant in} $G$ iff $x$ is parsable from $q$
 using $H$, where $H=G/(P_G - \{(q,x)\})$; and

\item $(q,x)$ \emph{is irredundant in} $G$ iff $(q,x)$ is not
  redundant in $G$.
\end{itemize}

Now we are able to say when a grammar is simplified.  The reader
should compare this definition with the definition in
Section~\ref{SimplificationOfFiniteAutomata} of when a finite
automaton is simplified.
A grammar $G$ is \emph{simplified} iff either
\begin{itemize}
\item every variable of $G$ is useful, and every production of $G$
  is irredundant; or

\item $|Q_G|=1$ and $P_G = \emptyset$.
\end{itemize}
The second case is necessary, because otherwise there would be
no simplified grammar generating $\emptyset$.

\begin{proposition}
If $G$ is a simplified grammar, then $\alphabet\,G = \alphabet(L(G))$.
\end{proposition}

\begin{proof}
Suppose $a\in\alphabet\,G$.  We must show that $a\in\alphabet\,w$ for
some $w\in L(G)$.  We have that every variable of $G$ is useful, and
there are $q\in Q_G$ and $x\in\Str$ such that $(q,x)\in P_G$ and
$a\in\alphabet\,x$.  Thus $x$ is parsable from $q$.  Since every
variable occurring in $x$ is generating, we have that $q$ generates a
string $x'$ containing $a$.  Since $q$ is reachable, there is a string
$y$ such that $y$ is parsable from $s_G$, and $q\in\alphabet\,y$.
Since every variable occurring in $y$ is generating, there is a string
$y'$ such that $y'$ is parsable from $s_G$, and $q$ is the only
variable of $\alphabet\,y'$.  Putting these facts together, we have
that $s_G$ generates a string $w$ such that $a\in\alphabet\,w$, i.e.,
$a\in\alphabet\,w$ for some $w\in L(G)$.
\end{proof}

Next, we give an algorithm for removing redundant productions.
Given a grammar $G$, $q\in Q_G$ and $x\in\Str$, we say that
$(q,x)$ \emph{is implicit in} $G$ iff $x$ is parsable from $q$ using
$G$.

Given a grammar $G$, we define a function
$\remRedun_G\in\powset\,P_G\times\powset\,P_G\fun\powset\,P_G$ by
well-founded recursion on the size of its second argument.
For $U,V\sub P_G$, $\remRedun(U, V)$ proceeds as follows:
\begin{itemize}
\item If $V=\emptyset$, then it returns $U$.

\item Otherwise, let $v$ be the greatest element of $\setof{(q,x)\in
    V}{\eqtxtr{there are no} p\in\Sym \eqtxt{and} y\in\Str \eqtxt{such
      that} (p,y)\in V \eqtxt{and} |y| > |x|}$, and $V' = V - \{v\}$.
  If $v$ is implicit in $G/(U\cup V')$, then $\remRedun$ returns the
  result of evaluating $\remRedun(U, V')$.  Otherwise, it returns the
  result of evaluating $\remRedun(U \cup \{v\}, V')$.
\end{itemize}

In general, there are multiple---incompatible---ways of removing
redundant productions from a grammar.  $\remRedun$ is defined so as to
favor removing productions whose right-hand sides are longer; and
among productions whose right-hand sides have equal length, to favor
removing productions that are larger in our total ordering on
productions.

Our algorithm for removing redundant productions of a grammar $G$
returns $G/(\remRedun_G(\emptyset,P_G))$.

For example, if we run our algorithm for removing redundant productions
on
\begin{align*}
\Asf &\fun \% \mid \zerosf \mid \mathsf{AA} \mid \mathsf{AAA} ,
\end{align*}
we obtain
\begin{align*}
\Asf &\fun \% \mid \zerosf \mid \mathsf{AA} .
\end{align*}

Our simplification algorithm for grammars proceeds as follows, given
a grammar $G$.
\begin{itemize}
\item First, it determines which variables of $G$ are generating.
If $s_G$ isn't one of these variables, then it returns the
grammar with variable $s_G$ and no productions.

\item Next, it turns $G$ into a grammar $G'$ by deleting all
non-generating variables, and deleting all productions involving such
variables.

\item Then, it determines which variables of $G'$ are reachable.

\item Next, it turns $G'$ into a grammar $G''$ by deleting all
non-reachable variables, and deleting all productions involving such
variables.

\item Finally, it removes redundant productions from $G''$.
\end{itemize}

Suppose $G$, once again, is the grammar
\begin{align*}
\Asf &\fun \mathsf{BB1}, \\
\Bsf &\fun \zerosf\mid\Asf\mid\Csf\Dsf, \\
\Csf &\fun \onesf\twosf , \\
\Dsf &\fun \onesf\Dsf\twosf .
\end{align*}
Here is what happens if we apply our simplification algorithm to $G$.
\begin{itemize}
\item First, we determine which variables are generating. Clearly
    $\Bsf$ and $\Csf$ are.  And, since $\Bsf$ is, it follows that
    $\Asf$ is, because of the production $\Asf\fun\mathsf{BB1}$.  (If
    this production had been $\Asf\fun\mathsf{BD1}$, we wouldn't have
    added $\Asf$ to our set.)

\item Thus, we form $G'$ from $G$ by deleting the variable $\Dsf$, yielding
  the grammar
  \begin{align*}
    \Asf &\fun \mathsf{BB1}, \\
    \Bsf &\fun \zerosf\mid\Asf, \\
    \Csf &\fun \onesf\twosf .
  \end{align*}

\item Next, we determine which variables of $G'$ are reachable.
  Clearly $\Asf$ is, and thus $\Bsf$ is, because of the production
  $\Asf\fun\mathsf{BB1}$.

  Note that, if we carried out the two stages of our simplification
  algorithm in the other order, then $\Csf$ and its production would
  never be deleted.

\item Next, we form $G''$ from $G'$ by deleting the variable $\Csf$,
  yielding the grammar
  \begin{align*}
    \Asf &\fun \mathsf{BB1}, \\
    \Bsf &\fun \zerosf\mid\Asf .
  \end{align*}

\item Finally, we would remove redundant productions from $G''$.
  But $G''$ has no redundant productions, and so we are done.
\end{itemize}

We define a function $\simplify\in\Gram\fun\Gram$ by: for all
$G\in\Gram$, $\simplify\,G$ is the result of running the above
algorithm on $G$.

\begin{theorem}
For all $G\in\Gram$:
\begin{enumerate}[\quad(1)]
\item $\simplify\,G$ is simplified;

\item $\simplify\,G\approx G$; and

\item $\alphabet(\simplify\,G) = \alphabet(L(G)) \sub\alphabet\,G$.
\end{enumerate}
\end{theorem}

Our simplification function/algorithm $\simplify$ gives us an
algorithm for testing whether a grammar is simplified: we apply
$\simplify$ to it, and check that the resulting grammar is equal to
the original one.

\subsection{Simplification in Forlan}

The Forlan module \texttt{Gram} defines the functions
\begin{verbatim}
val simplify   : gram -> gram
val simplified : gram -> bool
\end{verbatim}
The function \texttt{simplify} corresponds to $\simplify$, and
\texttt{simplified} tests whether a grammar is simplified.

Suppose \texttt{gram} of type \texttt{gram} is bound to the grammar
\begin{align*}
\Asf &\fun \mathsf{BB1}, \\
\Bsf &\fun \zerosf\mid\Asf\mid\Csf\Dsf, \\
\Csf &\fun \onesf\twosf , \\
\Dsf &\fun \onesf\Dsf\twosf .
\end{align*}
We can simplify our grammar as follows:
\begin{list}{}
{\setlength{\leftmargin}{\leftmargini}
\setlength{\rightmargin}{0cm}
\setlength{\itemindent}{0cm}
\setlength{\listparindent}{0cm}
\setlength{\itemsep}{0cm}
\setlength{\parsep}{0cm}
\setlength{\labelsep}{0cm}
\setlength{\labelwidth}{0cm}
\catcode`\#=12
\catcode`\$=12
\catcode`\%=12
\catcode`\^=12
\catcode`\_=12
\catcode`\.=12
\catcode`\?=12
\catcode`\!=12
\catcode`\&=12
\ttfamily}
\small
\item[]\textsl{-\ }val\ gram'\ =\ Gram.simplify\ gram;
\item[]\textsl{val\ gram'\ =\ -\ :\ gram}
\item[]\textsl{-\ }Gram.output("",\ gram');
\item[]\textsl{\symbol{'173}variables\symbol{'175}\ A,\ B\ \symbol{'173}start\ variable\symbol{'175}\ A}
\item[]\textsl{\symbol{'173}productions\symbol{'175}\ A\ ->\ BB1;\ B\ ->\ 0\ |\ A}
\item[]\textsl{val\ it\ =\ ()\ :\ unit}
\end{list}

And, Suppose \texttt{gram''} of type \texttt{gram} is bound to the grammar
\begin{align*}
\Asf \fun \% \mid \zerosf \mid \mathsf{AA} \mid \mathsf{AAA} \mid
\mathsf{AAAA} .
\end{align*}
We can simplify our grammar as follows:
\begin{list}{}
{\setlength{\leftmargin}{\leftmargini}
\setlength{\rightmargin}{0cm}
\setlength{\itemindent}{0cm}
\setlength{\listparindent}{0cm}
\setlength{\itemsep}{0cm}
\setlength{\parsep}{0cm}
\setlength{\labelsep}{0cm}
\setlength{\labelwidth}{0cm}
\catcode`\#=12
\catcode`\$=12
\catcode`\%=12
\catcode`\^=12
\catcode`\_=12
\catcode`\.=12
\catcode`\?=12
\catcode`\!=12
\catcode`\&=12
\ttfamily}
\small
\item[]\textsl{-\ }val\ gram'''\ =\ Gram.simplify\ gram'';
\item[]\textsl{val\ gram'''\ =\ -\ :\ gram}
\item[]\textsl{-\ }Gram.output("",\ gram''');
\item[]\textsl{\symbol{'173}variables\symbol{'175}\ A\ \symbol{'173}start\ variable\symbol{'175}\ A\ \symbol{'173}productions\symbol{'175}\ A\ ->\ %\ |\ 0\ |\ AA}
\item[]\textsl{val\ it\ =\ ()\ :\ unit}
\end{list}


\subsection{Hand-simplification Operations}

Given a simplified grammar $G$, there are often ways we can hand-simplify
the grammar further. Below are two examples:
\begin{itemize}
\item Suppose $G$ has a variable $q$ that is not $s_G$, and that
  appears on the left side of exactly one production: $q\fun
  x$. Because $G$ is simplified, it follows that $q$ does not occur in
  $x$ (or $q$ would not be generating).  Then we can form an
  equivalent grammar $G'$ by deleting $q$ and $q\fun x$ from $G$, and
  transforming each remaining production $p\fun y$ into $p\fun z$,
  where $z$ is the result of replacing each occurrence of $q$ in $y$
  by $x$.

  We refer to this operation as \emph{eliminating} $q$ \emph{from}
  $G$.

\item Suppose there is exactly one production of $G$ involving $s_G$,
  where that production has the form $s_G\fun q$, for some variable
  $q$ of $G$. Then we can form an equivalent grammar $G'$ by deleting
  $s_G$ and $s_G\fun q$ from $G$, and making $q$ be the start variable
  of $G'$.

  We refer to this operation as \emph{restarting} $G$.
\end{itemize}

The Forlan module \texttt{Gram} has functions corresponding to these
two operations:
\begin{verbatim}
val eliminateVariable : gram * sym -> gram
val restart           : gram -> gram
\end{verbatim}
Both begin by simplifying the supplied grammar.

For instance, suppose $\texttt{gram}$ is the grammar
\begin{align*}
\Asf &\fun \mathsf{B}, \\
\Bsf &\fun \zerosf \mid \Csf\onesf\Csf , \\
\Csf &\fun \onesf\Bsf\twosf .
\end{align*}

Then we can proceed as follows:
\begin{list}{}
{\setlength{\leftmargin}{\leftmargini}
\setlength{\rightmargin}{0cm}
\setlength{\itemindent}{0cm}
\setlength{\listparindent}{0cm}
\setlength{\itemsep}{0cm}
\setlength{\parsep}{0cm}
\setlength{\labelsep}{0cm}
\setlength{\labelwidth}{0cm}
\catcode`\#=12
\catcode`\$=12
\catcode`\%=12
\catcode`\^=12
\catcode`\_=12
\catcode`\.=12
\catcode`\?=12
\catcode`\!=12
\catcode`\&=12
\ttfamily}
\small
\item[]\textsl{-\ }val\ gram'\ =\ Gram.eliminateVariable(gram,\ Sym.fromString\ "C");
\item[]\textsl{val\ gram'\ =\ -\ :\ gram}
\item[]\textsl{-\ }Gram.output("",\ gram');
\item[]\textsl{\symbol{'173}variables\symbol{'175}\ A,\ B\ \symbol{'173}start\ variable\symbol{'175}\ A}
\item[]\textsl{\symbol{'173}productions\symbol{'175}}
\item[]\textsl{A\ ->\ B;\ B\ ->\ 0\ |\ 1B231B2\ |\ 1B232B1\ |\ 2B131B2\ |\ 2B132B1}
\item[]\textsl{val\ it\ =\ ()\ :\ unit}
\item[]\textsl{-\ }val\ gram''\ =\ Gram.restart\ gram;
\item[]\textsl{val\ gram''\ =\ -\ :\ gram}
\item[]\textsl{-\ }Gram.output("",\ gram'');
\item[]\textsl{\symbol{'173}variables\symbol{'175}\ B,\ C\ \symbol{'173}start\ variable\symbol{'175}\ B}
\item[]\textsl{\symbol{'173}productions\symbol{'175}\ B\ ->\ 0\ |\ C3C;\ C\ ->\ 1B2\ |\ 2B1}
\item[]\textsl{val\ it\ =\ ()\ :\ unit}
\end{list}


\subsection{Notes}

As described above, our definition of useless variable is weaker than
the standard one.  However, whenever every variable of a grammar is
useful in our sense, it follows that every variable of the grammar is
useful in the standard sense.  Furthermore, our algorithm for removing
useless variables is the standard one.  Requiring that simplified
grammars have no redundant productions is natural, although
non-standard, and our algorithm for removing redundant productions is
straightforward.

%%% Local Variables: 
%%% mode: latex
%%% TeX-master: "book"
%%% End: 

\section{Proving the Correctness of Grammars}
\label{ProvingTheCorrectnessOfGrammars}

In this section, we consider techniques for proving the correctness of
grammars, i.e., for proving that grammars generate the languages we want
them to.

\subsection{Preliminaries}

Suppose $G$ is a grammar and $a\in Q_G\cup\alphabet\,G$.  Then
\begin{displaymath}
\Pi_{G,a}=\setof{w\in(\alphabet\,G)^*}{w \eqtxt{is parsable from} a
  \eqtxt{using} G} .  
\end{displaymath}
I.e., $\Pi_{G,a}$ is all strings over $\alphabet\,G$ that are the
yields of valid parse trees for $G$ whose root labels are $a$ If it's
clear which grammar we are talking about, we often abbreviate
$\Pi_{G,a}$ to $\Pi_a$.  Clearly, $\Pi_{G,s_G} = L(G)$.

For example, if $G$ is the grammar
\begin{gather*}
  \Asf \fun \% \mid \zerosf\Asf\onesf
\end{gather*}
then $\Pi_\zerosf=\{\zerosf\}$, $\Pi_\onesf=\{\onesf\}$ and $\Pi_\Asf=
\setof{\zerosf^n\onesf^n}{n\in\nats}= L(G)$.

\begin{proposition}
\label{PiProp}
Suppose $G$ is a grammar.
\begin{enumerate}[\quad(1)]
\item For all $a\in\alphabet\,G$, $\Pi_{G,a}=\{a\}$.

\item For all $q\in Q_G$, if $q\fun\%\in P_G$, then
  ${\%}\in\Pi_{G,q}$.

\item For all $q\in Q_G$, $n\in\nats-\{0\}$, $a_1,\ldots,a_n\in\Sym$
  and $w_1,\ldots,w_n\in\Str$, if $q\fun a_1\cdots a_n\in P_G$ and
  $w_1\in\Pi_{G,a_1}$, \ldots, $w_n\in\Pi_{G,a_n}$, then
  ${w_1\cdots w_n}\in\Pi_{G,q}$.
\end{enumerate}
\end{proposition}

Our main example will be the grammar $G$:
\begin{align*}
\Asf &\fun \% \mid \zerosf\Bsf\Asf \mid \onesf\Csf\Asf , \\
\Bsf &\fun \onesf \mid \zerosf\Bsf\Bsf , \\
\Csf &\fun \zerosf \mid \onesf\Csf\Csf .
\end{align*}
Define $\diff\in\{\mathsf{0,1}\}^*\fun\ints$ by:
for all $w\in\{\mathsf{0,1}\}^*$,
\begin{gather*}
\diff\,w =
\eqtxtr{the number of $\mathsf{1}$'s in}w -
\eqtxtr{the number of $\mathsf{0}$'s in}w .
\end{gather*}
Then: $\diff\,\% = 0$, $\diff\,\mathsf{1} = 1$, $\diff\,\mathsf{0} =
-1$, and, for all $x,y\in\{\mathsf{0,1}\}^*$, $\diff(xy) = \diff\,x +
\diff\,y$.
Let
\begin{align*}
  X &= \setof{w\in\{\mathsf{0,1}\}^*}{\diff\,w = 0} , \\
  Y &= \setof{w\in\{\mathsf{0,1}\}^*}{\diff\,w = 1\eqtxt{and,}\\
    &\quad\hspace{.25cm}
    \eqtxt{for all proper prefixes}v\eqtxt{of}w,\diff\,v\leq 0} , \eqtxt{and}\\
  Z &= \setof{w\in\{\mathsf{0,1}\}^*}{\diff\,w = -1\eqtxt{and,}\\
    &\quad\hspace{.25cm} \eqtxt{for all proper
      prefixes}v\eqtxt{of}w,\diff\,v\geq 0} .
\end{align*}
We will prove that $L(G) = \Pi_{G,\Asf} = X$, $\Pi_{G,\Bsf} = Y$ and
$\Pi_{G,\Csf} = Z$.

\begin{lemma}
\label{GramCorrLem1}
Suppose $x\in\{\mathsf{0,1}\}^*$.
\begin{enumerate}[\quad(1)]
\item If $\diff\,x\geq 1$, then $x=yz$ for some $y,z\in\{\mathsf{0,1}\}^*$
  such that $y\in Y$ and $\diff\,z = \diff\,x - 1$.

\item If $\diff\,x\leq -1$, then $x=yz$ for some $y,z\in\{\mathsf{0,1}\}^*$
  such that $y\in Z$ and $\diff\,z = \diff\,x + 1$.
\end{enumerate}
\end{lemma}

\begin{proof}
We show the proof of (1), the proof of (2) being similar.

Let $y\in\{\mathsf{0,1}\}^*$ be the shortest prefix of $x$ such
that $\diff\,y\geq 1$, and let $z\in\{\mathsf{0,1}\}^*$ be such
that $x=yz$.

Because $\diff\,y\geq 1$, we have that $y\neq\%$.  Thus $y=y'a$
for some $y'\in\{\mathsf{0,1}\}^*$ and $a\in\{\mathsf{0,1}\}$.
By the definition of $y$, we have that $\diff\,y'\leq 0$.
Suppose, toward a contradiction, that $a=\zerosf$.  Since
$\diff\,y' + -1=\diff\,y\geq 1$, we have that $\diff\,y'\geq 2$,
contradicting the definition of $y$.  Thus $a=\onesf$, so
that $y=y'\onesf$.

Because $\diff\,y' + 1 = \diff\,y\geq 1$, we have that $\diff\,y'\geq
0$.  Thus $\diff\,y' = 0$, so that $\diff y = \diff\,y' + 1 = 1$.  By
the definition of $y$, every prefix of $y'$ has a diff that is
$\leq$ $0$.  Thus $y\in Y$.

Finally, because $\diff\,x = \diff\,y + \diff\,z = 1 + \diff\,z$ we have
that $\diff\,z = \diff\,x - 1$.
\end{proof}

\subsection{Proving that Enough is Generated}

First we study techniques for showing that everything we want a
grammar to generate is really generated.

Since $X,Y,Z\sub\{\mathsf{0,1}\}^*$, to prove that
$X\sub\Pi_{G,\Asf}$, $Y\sub\Pi_{G,\Bsf}$ and $Z\sub\Pi_{G,\Csf}$, it
will suffice to use strong string induction to show that, for all
$w\in\mathsf{\{0,1\}^*}$:

\begin{enumerate}[\quad(A)]
\item if $w\in X$, then $w\in\Pi_{G,\Asf}$;

\item if $w\in Y$, then $w\in\Pi_{G,\Bsf}$; and

\item if $w\in Z$, then $w\in\Pi_{G,\Csf}$.
\end{enumerate}

We proceed by strong string induction.  Suppose
$w\in\{\mathsf{0,1}\}^*$, and assume the inductive hypothesis:
for all $x\in\{\mathsf{0,1}\}^*$, if $x$ is a proper substring of
$w$, then:
\begin{enumerate}[\quad(A)]
\item if $x\in X$, then $x\in\Pi_\Asf$;

\item if $x\in Y$, then $x\in\Pi_\Bsf$; and

\item if $x\in Z$, then $x\in\Pi_\Csf$.
\end{enumerate}
We must prove that:
\begin{enumerate}[\quad(A)]
\item if $w\in X$, then $w\in\Pi_\Asf$;

\item if $w\in Y$, then $w\in\Pi_\Bsf$; and

\item if $w\in Z$, then $w\in\Pi_\Csf$.
\end{enumerate}

\begin{enumerate}[\quad(A)]
\item Suppose $w\in X$.  We must show that $w\in\Pi_\Asf$.  There are
  three cases to consider.
  \begin{itemize}
  \item Suppose $w=\%$.  Because $\Asf\fun\%\in P$,
    Proposition~\ref{PiProp}(2) tells us that $w=\%\in\Pi_\Asf$.
  
  \item Suppose $w=\zerosf x$, for some $x\in\{\mathsf{0,1}\}^*$.
    Because $-1 + \diff\,x = \diff\,w = 0$, we have that $\diff\,x =
    1$.  Thus, by Lemma~\ref{GramCorrLem1}(1), we have that $x=yz$,
    for some $y,z\in\{\mathsf{0,1}\}^*$ such that $y\in Y$ and
    $\diff\,z = \diff\,x - 1 = 1 - 1 = 0$.  Thus $w=\zerosf yz$, $y\in
    Y$ and $z\in X$.  By Proposition~\ref{PiProp}(1), we have
    $\zerosf\in\Pi_\zerosf$.  Because $y\in Y$ and $z\in X$ are proper
    substrings of $w$, parts~(B) and (A) of the inductive hypothesis
    tell us that $y\in\Pi_\Bsf$ and $z\in\Pi_\Asf$.  Thus, because
    $\Asf\fun\zerosf\Bsf\Asf\in P$, Proposition~\ref{PiProp}(3) tells
    us that $w=\zerosf yz\in\Pi_\Asf$.
  
  \item Suppose $w=\onesf x$, for some $x\in\{\mathsf{0,1}\}^*$.  The
    proof is analogous to the preceding case.
  \end{itemize}

\item Suppose $w\in Y$.  We must show that $w\in\Pi_\Bsf$.  Because
  $\diff\,w=1$, there are two cases to consider.
  \begin{itemize}
  \item Suppose $w=\onesf x$, for some $x\in\{\mathsf{0,1}\}^*$.
    Because all proper prefixes of $w$ have diffs $\leq$ $0$, we
    have that $x=\%$, so that $w=\onesf$.  Since $\Bsf\fun\onesf\in P$,
    we have that $w=\onesf\in\Pi_\Bsf$.
  
  \item Suppose $w=\zerosf x$, for some $x\in\{\mathsf{0,1}\}^*$.
    Thus $\diff\,x = 2$.  Because $\diff\,x\geq 1$, by
    Lemma~\ref{GramCorrLem1}(1), we have that $x=yz$, for some
    $y,z\in\{\mathsf{0,1}\}^*$ such that $y\in Y$ and $\diff\,z =
    \diff\,x - 1 = 2 - 1 = 1$.  Hence $w=\zerosf yz$.  To finish the
    proof that $z\in Y$, suppose $v$ is a proper prefix of $z$.
    Thus $\zerosf yv$ is a proper prefix of $w$.  Since $w\in Y$, it
    follows that $\diff\,v = \diff(\zerosf yv)\leq 0$, as required.
    Since $y,z\in Y$, part~(B) of the inductive hypothesis tell us
    that $y,z\in\Pi_\Bsf$.  Thus, because $\Bsf\fun\zerosf\Bsf\Bsf\in
    P$ we have that $w=\zerosf yz\in\Pi_\Bsf$.
  \end{itemize}

\item Suppose $w\in Z$.  We must show that $w\in\Pi_\Csf$.  The
  proof is analogous to the proof of part~(B).
\end{enumerate}

Suppose $H$ is the grammar
\begin{gather*}
\Asf \fun \Bsf \mid \zerosf\Asf\threesf , \qquad
\Bsf \fun \% \mid \onesf\Bsf\twosf ,
\end{gather*}
and let
\begin{gather*}
X = \setof{\zerosf^n\onesf^m\twosf^m\threesf^n}{n,m\in\nats} , \qquad
Y = \setof{\onesf^m\twosf^m}{m\in\nats} .
\end{gather*}
We can prove that $X\sub \Pi_{H,\Asf} = L(H)$ and $Y\sub \Pi_{H,\Bsf}$
using the above technique, but the production $\Asf\fun\Bsf$, which is
called a \emph{unit production} because its right side is a single
variable, makes part~(A) tricky.  In
Section~\ref{ProvingTheCorrectnessOfFiniteAutomata}, when considering
techniques for showing the correctness of finite automata, we ran into
a similar problem having to do with $\%$-transitions.

If $w=\zerosf^0\onesf^m\twosf^m\threesf^0 = \onesf^m\twosf^m\in Y$, we
would like to use part~(B) of the inductive hypothesis to conclude
$w\in\Pi_\Bsf$, and then use the fact that $\Asf\fun\Bsf\in P$ to
conclude that $w\in\Pi_\Asf$.  But $w$ is not a proper substring of
itself, and so the inductive hypothesis in not applicable.  Instead,
we must split into cases $m=0$ and $m\geq 1$, using $\Asf\fun\Bsf$ and
$\Bsf\fun\%$, in the first case, and $\Asf\fun\Bsf$ and
$\Bsf\fun\onesf\Bsf\twosf$, as well as the inductive hypothesis on
$\onesf^{m-1}\twosf^{m-1}\in Y$, in the second case.

Because there are no productions from $\Bsf$ back to $\Asf$, we could
also first use strong string induction to prove that, for all
$w\in\mathsf{\{0,1\}^*}$,
\begin{enumerate}[\quad(A)]
\setcounter{enumi}{1}
\item if $w\in Y$, then $w\in\Pi_\Bsf$,
\end{enumerate}
and then use the result of this induction along with
strong string induction to prove that
for all $w\in\mathsf{\{0,1\}^*}$,
\begin{enumerate}[\quad(A)]
\item if $w\in X$, then $w\in\Pi_\Asf$.
\end{enumerate}
This works whenever two parts of a grammar are not mutually recursive.

With this grammar, we could also first use mathematical induction to
prove that, for all $m\in\nats$, $\onesf^m\twosf^m\in\Pi_\Bsf$, and
then use the result of this induction to prove, by mathematical
induction on $n$, that for all $n,m\in\nats$,
$\zerosf^n\onesf^m\twosf^m\threesf^n\in\Pi_\Asf$.

Note that $\%$-productions, i.e., productions of the form $q\fun\%$,
can cause similar problems to those caused by unit productions.  E.g.,
if we have the productions
\begin{displaymath}
  \Asf\fun \Bsf\Csf \quad\eqtxt{and}\quad \Bsf\fun\% ,
\end{displaymath}
then $\Asf\fun\Bsf\Csf$ behaves like a unit production.

\subsection{Proving that Everything Generated is Wanted}

When proving that everything generated by a grammar is wanted, we
could sometimes use strong induction, simply reversing the
implications used when proving that enough is generated.  But this
approach fails for grammars with unit productions, where we would have
to resort to an induction on parse trees.  

In Section~\ref{ProvingTheCorrectnessOfFiniteAutomata}, when
considering techniques for showing the correctness of finite automata,
we ran into a similar problem having to do with $\%$-transitions, and
this led to our introducing the Principle of Induction on $\Lambda$.
Here, we introduce an induction principle called Induction on $\Pi$.

\begin{theorem}[Principle of Induction on $\Pi$]
Suppose $G$ is a grammar, $P_q(w)$ is a property of a
string $w\in\Pi_{G,q}$, for all $q\in Q_G$, and
$P_a(w)$, for $a\in\alphabet\,G$, says ``$w=a$''.
If
\begin{enumerate}[\quad(1)]
\item for all $q\in Q_G$, if $q\fun\%\in P_G$, then $P_q(\%)$, and

\item for all $q\in Q_G$, $n\in\nats-\{0\}$, $a_1,\ldots,a_n\in
  Q_G\cup\alphabet\,G$, and $w_1\in\Pi_{G,a_1}$, \ldots,
  $w_n\in\Pi_{G,a_n}$, if $q\fun a_1\cdots a_n\in P_G$ and (\dag)
  $P_{a_1}(w_1)$, \ldots, $P_{a_n}(w_n)$, then $P_q(w_1\cdots w_n)$,
\end{enumerate}
then
\begin{gather*}
  \eqtxtr{for all}q\in Q_G, \eqtxt{for all}w\in\Pi_{G,q},\,P_q(w).
\end{gather*}
\end{theorem}
We refer to (\dag) as the inductive hypothesis.

\begin{proof}
It suffices to show that, for all $\pt\in\PT$, for all 
$q\in Q_G$ and $w\in(\alphabet\,G)^*$, if
$\pt$ is valid for $G$, $\rootLabel\,\pt = q$ and
$\yield\,\pt = w$, then $P_q(w)$.
We prove this using the principle of induction on parse trees.
\end{proof}

When proving part~(2), we can make use of the fact that, for
$a_i\in\alphabet\,G$, $\Pi_{a_i} = \{a_i\}$, so that $w_i\in\Pi_{a_i}$
will be $a_i$.  Hence it will be unnecessary to assume that
$P_{a_i}(a_i)$, since this says ``$a_i=a_i$'', and so is always true.

Consider, again, our main example grammar $G$:
\begin{align*}
\Asf &\fun \% \mid \zerosf\Bsf\Asf \mid \onesf\Csf\Asf , \\
\Bsf &\fun \onesf \mid \zerosf\Bsf\Bsf , \\
\Csf &\fun \zerosf \mid \onesf\Csf\Csf .
\end{align*}
Let
\begin{align*}
X &= \setof{w\in\{\mathsf{0,1}\}^*}{\diff\,w = 0} , \\
Y &= \setof{w\in\{\mathsf{0,1}\}^*}{\diff\,w = 1\eqtxt{and,}\\
&\quad\hspace{.25cm}
\eqtxt{for all proper prefixes}v\eqtxt{of}w,\diff\,v\leq 0} , \\
Z &= \setof{w\in\{\mathsf{0,1}\}^*}{\diff\,w = -1\eqtxt{and,}\\
&\quad\hspace{.25cm}
\eqtxt{for all proper prefixes}v\eqtxt{of}w,\diff\,v\geq 0} .
\end{align*}

We have already proven that $X\sub\Pi_\Asf=L(G)$, $Y\sub\Pi_\Bsf$ and
$Z\sub\Pi_\Csf$.  To complete the proof that
$L(G)=\Pi_\Asf=X$, $\Pi_\Bsf=Y$ and $\Pi_\Csf=Z$, we will use
induction on $\Pi$ to prove that
$\Pi_\Asf\sub X$, $\Pi_\Bsf\sub Y$ and $\Pi_\Csf\sub Z$.

We use induction on $\Pi$ to show that:
\begin{enumerate}[\quad(A)]
\item for all $w\in\Pi_\Asf$, $w\in X$;

\item for all $w\in\Pi_\Bsf$, $w\in Y$; and

\item for all $w\in\Pi_\Csf$, $w\in Z$.
\end{enumerate}
Formally, this means that we let the properties $P_\Asf(w)$,
$P_\Bsf(w)$ and $P_\Csf(w)$ be ``$w\in X$'', ``$w\in Y$'' and ``$w\in
Z$'', respectively, and then use the induction principle to prove
that, for all $q\in Q_G$, for all $w\in\Pi_q$,
$P_q(w)$.  But we will actually work more informally.

There are seven productions to consider.
\begin{description}
\item[\quad($\Asf\fun\%$)] We must show that $\%\in X$ (as
  ``$w\in X$'' is the property of part~(A)).  And this holds since
  $\diff\,\% = 0$.

\item[\quad($\Asf\fun\zerosf\Bsf\Asf$)] Suppose $w_1\in\Pi_\Bsf$ and
  $w_2\in\Pi_\Asf$ (as $\zerosf\Bsf\Asf$ is the right-side of the
  production, and $\zerosf$ is in $G$'s alphabet),
  and assume the inductive hypothesis,
  $w_1\in Y$ (as this is the property of part~(B)) and
  $w_2\in X$ (as this is the property of part~(A)).  We
  must show that $\zerosf w_1w_2\in X$, as the production
  shows that $\zerosf w_1w_2\in\Pi_\Asf$.  Because
  $w_1\in Y$ and $w_2\in X$, we have that $\diff\,w_1=1$ and
  $\diff\,w_2=0$.  Thus $\diff(\zerosf w_1w_2) = -1 + 1 + 0 = 0$, showing
  that $\zerosf w_1w_2\in X$.
\end{description}

\begin{description}
\item[\quad($\Bsf\fun\zerosf\Bsf\Bsf$)] Suppose $w_1,w_2\in\Pi_\Bsf$,
  and assume the inductive hypothesis, $w_1,w_2\in Y$.  Thus $w_1$ and
  $w_2$ are nonempty. We must show
  that $\zerosf w_1w_2\in Y$.  Clearly, $\diff(\zerosf w_1w_2) =
  -1 + 1 + 1 = 1$.  So, suppose $v$ is a proper prefix of
  $\zerosf w_1w_2$.  We must show that $\diff\,v\leq 0$.
  There are three cases to consider.
  \begin{itemize}
  \item Suppose $v = \%$.  Then $\diff\,v = 0\leq 0$.

  \item Suppose $v=\zerosf u$, for a proper prefix $u$ of $w_1$.
    Because $w_1\in Y$, we have that $\diff\,u\leq 0$.  Thus
    $\diff\,v = -1 + \diff\,u \leq -1 + 0\leq 0$.

  \item Suppose $v=\zerosf w_1u$, for a proper prefix $u$ of $w_2$.
    Because $w_2\in Y$, we have that $\diff\,u\leq 0$.
    Thus $\diff\,v = -1 + 1 + \diff\,u = \diff\,u\leq 0$.
  \end{itemize}

\item The remaining productions are handled similarly.
\end{description}

\begin{exercise}
Let
\begin{displaymath}
  X = \setof{\zerosf^i\onesf^j\twosf^k\threesf^l}{i,j,k,l\in\nats\eqtxt{and}
    i + j = k + l} .
\end{displaymath}
Find a grammar $G$ such that $L(G) = X$, and prove that your answer is
correct.
\end{exercise}

\begin{exercise}
Let
\begin{displaymath}
X = \setof{\zerosf^i\onesf^j\twosf^k}{i,j,k\in\nats\eqtxt{and}
(i\neq j\eqtxt{or}j\neq k)} .
\end{displaymath}
Find a grammar $G$ such that $L(G) = X$, and prove that your answer is
correct.
\end{exercise}

\subsection{Notes}

Books on formal language theory typically give short shrift to the
proof of correctness of grammars, carrying out one or two
correctness proofs using induction on the length of strings.  In
contrast, we have introduced and applied elegant techniques for
proving the correctness of grammars.  Of particular note is our principle
of induction on $\Pi$.

%%% Local Variables: 
%%% mode: latex
%%% TeX-master: "book"
%%% End: 

\section{Ambiguity of Grammars}
\label{AmbiguityOfGrammars}

In this section, we say what it means for a grammar to be ambiguous.
We also give a straightforward method for disambiguating
grammars for languages with operators of various precedences and
associativities, and consider an efficient parsing algorithm for
such disambiguated grammars.

\subsection{Definition}

Suppose $G$ is our grammar of arithmetic expressions:
\begin{gather*}
\mathsf{E\fun E\plussym E\mid E\timessym E\mid \openparsym E\closparsym \mid
\idsym} .
\end{gather*}
Unfortunately,
there are multiple ways of parsing
$\idsym\timessym\idsym\plussym\idsym$ according to this grammar:
\begin{center}
\input{chap-4.6-fig1.eepic}
\end{center}
In $\pt_1$, multiplication has higher precedence than addition; in
$\pt_2$, the situation is reversed.  Because there are multiple ways
of parsing this string, we say that our grammar is ``ambiguous''.

A grammar $G$ is \emph{ambiguous} iff there is a
$w\in(\alphabet\,G)^*$ such that $w$ is the yield of multiple valid
parse trees for $G$ whose root labels are $s_G$; otherwise, $G$ is
\emph{unambiguous}.

Let $G$ be the grammar
\begin{gather*}
\Asf \fun \% \mid \mathsf{0A1A} \mid \mathsf{1A0A}
\end{gather*}
which generates all elements
of $\{\mathsf{0,1}\}^*$ with a $\diff$ of $0$, for the $\diff$
function such that $\diff\,\zerosf = -1$ and $\diff\,\onesf = 1$.
It is ambiguous as, e.g., $\mathsf{0101}$ can be parsed as
$\mathsf{0\%1(01)}$ or $\mathsf{0(10)1\%}$.
But in Section~4.5, we saw another grammar, $H$, for this
language:
\begin{align*}
\Asf &\fun \% \mid \zerosf\Bsf\Asf \mid \onesf\Csf\Asf , \\
\Bsf &\fun \onesf \mid \zerosf\Bsf\Bsf , \\
\Csf &\fun \zerosf \mid \onesf\Csf\Csf ,
\end{align*}
which turns out to be unambiguous.
The reason is that $\Pi_\Bsf$ is all elements of $\{\mathsf{0,1}\}^*$
with a $\diff$ of $1$, but with no proper prefixes with positive
$\diff$'s, and $\Pi_\Csf$ has the corresponding property for
$0$/negative.

\begin{exercise}
Prove that $L(G) = L(H)$.
\end{exercise}

\begin{exercise}
Prove that $H$ is unambiguous.
\end{exercise}

\subsection{Disambiguating Grammars of Operators}

Not every ambiguous grammar can be turned into an equivalent
unambiguous one.  However, we can use a simple technique to
disambiguate our grammar of arithmetic expressions, and this technique
works for many commonly occurring grammars involving operators of
various precedences and associativities.

Since there are two binary operators in our language of arithmetic
expressions, we have to decide:
\begin{itemize}
\item whether multiplication has higher or lower precedence than
  addition; and

\item whether multiplication and addition are left or right
  associative.
\end{itemize}
As usual, we'll make multiplication have higher precedence than
addition, and let addition and multiplication be left associative.

As a first step towards disambiguating our grammar, we can form
a new grammar with the three variables: $\Esf$ (expressions),
$\Tsf$ (terms) and $\Fsf$ (factors), start variable $\Esf$
and productions:
\begin{align*}
  \Esf &\fun \Tsf\mid \Esf\plussym\Esf , \\
  \Tsf &\fun \Fsf\mid \Tsf\timessym\Tsf , \\
  \Fsf &\fun \idsym \mid \openparsym\Esf\closparsym .
\end{align*}
The idea is that the lowest precedence operator ``lives'' at the
highest level of the grammar, that the highest precedence operator
lives at the middle level of the grammar, and that the basic
expressions, including the parenthesized expressions, live at
the lowest level of the grammar.

Now, there is only one way to parse the string
$\idsym\timessym\idsym\plussym\idsym$, since, if we begin
by using the production $\Esf\fun\Tsf$, our yield will only
include a $\plussym$ if this symbol occurs within parentheses.
If we had more levels of precedence in our language, we would simply
add more levels to our grammar.

On the other hand, there are still two ways of parsing the string
$\idsym\plussym\idsym\plussym\idsym$: with left associativity or right
associativity.  To finish disambiguating our grammar, we must break
the symmetry of the right-sides of the productions
\begin{align*}
  \Esf &\fun \Esf\plussym\Esf , \\
  \Tsf &\fun \Tsf\timessym\Tsf ,
\end{align*}
turning one of the $\Esf$'s into $\Tsf$, and one of the $\Tsf$'s into
$\Fsf$.  To make our operators be left associative, we must use
\emph{left recursion}, changing the second $\Esf$ to $\Tsf$, and the
second $\Tsf$ to $\Fsf$; right associativity would result from making
the opposite choices, i.e., using \emph{right recursion}.

Thus, our unambiguous grammar of arithmetic expressions is
\begin{align*}
\Esf &\fun \Tsf \mid \Esf\plussym\Tsf , \\
\Tsf &\fun \Fsf \mid \Tsf\timessym\Fsf , \\
\Fsf &\fun \idsym \mid \openparsym\Esf\closparsym .
\end{align*}
It can be proved that this grammar is indeed unambiguous, and that it
is equivalent to the original grammar.

Now, the only parse of $\idsym\timessym\idsym\plussym\idsym$ is
\begin{center}
\input{chap-4.6-fig2.eepic}
\end{center}
And, the only parse of $\idsym\plussym\idsym\plussym\idsym$
is
\begin{center}
\input{chap-4.6-fig3.eepic}
\end{center}

\subsection{Top-down Parsing}

Top-down parsing is a simple and efficient parsing method for
unambiguous grammars of operators like
\begin{align*}
\Esf &\fun \Tsf \mid \Esf\plussym\Tsf , \\
\Tsf &\fun \Fsf \mid \Tsf\timessym\Fsf , \\
\Fsf &\fun \idsym \mid \openparsym\Esf\closparsym .
\end{align*}

Let $\cal E$, $\cal T$ and $\cal F$ be all of the parse trees that
are valid for our grammar, have yields containing no variables,
and whose root labels are $\mathsf{E}$, $\mathsf{T}$ and $\mathsf{F}$, 
respectively.
Because this grammar has three mutually recursive variables, we
will need three mutually recursive parsing functions,
\begin{align*}
\parE &\in \Str\fun\Option({\cal E}\times\Str) , \\
\parT &\in \Str\fun\Option({\cal T}\times\Str) , \\
\parF &\in \Str\fun\Option({\cal F}\times\Str) ,
\end{align*}
which attempt to parse an element $\pt$ of $\cal E$, $\cal T$ or $\cal
F$ out of a string $w$, returning $\none$ to indicate failure, and
$\some(\pt,y)$, where $y$ is the remainder of $w$, otherwise.

Although most programming languages support mutual recursion, in this
book, we haven't formally justified well-founded mutual recursion.
Instead, we can work with a single recursive function with domain
$\{0,1,2\}\times\Str$, where the $0$, $1$ or $2$ indicates whether
it's $\parE$, $\parT$ or $\parF$, respectively, that is being called.
The range of this function will be $\Option({\cal U}\times\Str)$,
where $\cal U$ is the union of three disjoint sets:
$\{0\}\times{\cal E}$, $\{1\}\times{\cal T}$ and
$\{2\}\times{\cal F}$. E.g., when the function is called with $(0,w)$,
it will either return $\none$ or $\some((0,\pt), z)$, where
$\pt\in\cal E$ and $x\in\Str$. But to keep the notation simple, below,
we'll assume the parsing functions can call each other.

The well-founded ordering we are using allows:
\begin{itemize}
\item $\parE$ to call $\parT$ with strings that are no longer
  than its argument;

\item $\parT$ to call $\parF$ with strings that are no longer
  than its argument; and

\item $\parF$ to call $\parE$ with strings that are strictly shorter
  than its argument.
\end{itemize}

When called with a string $w$, $\parE$ is supposed to
determine whether there is a prefix $x$ of $w$ that is the
yield of an element of $\cal E$.  If there is such an $x$,
then it finds the \emph{longest} prefix $x$ of $w$ with this
property, and returns $\some(\pt,y)$, where $\pt$ is the
element of $\cal E$ whose yield is $x$, and $y$ is such
that $w=xy$.  Otherwise, it returns $\none$.
$\parT$ and $\parF$ have similar specifications.

Given a string $w$, $\parE$ operates as follows.  Because all elements
of $\cal E$ have yields beginning with the yield of an element of
$\cal T$, it starts by evaluating $\parT\,w$.  If this results in
$\none$, it returns $\none$.  Otherwise, it results in $\some(\pt,x)$,
for some $\pt\in{\cal T}$ and $x\in\Str$, in which case $\parE$
returns $\parELoop(E(\pt), x)$, where $\parELoop\in{\cal
  E}\times\Str\fun\Option({\cal E}\times\Str)$ is defined recursively,
as follows.

Given $(\pt, x)\in{\cal E}\times\Str$, $\parELoop$ proceeds as
follows.
\begin{itemize}
\item If $x={\plussym}y$ for some $y$, then $\parELoop$ evaluates
  $\parT\,y$.
  \begin{itemize}
  \item If this results in $\none$, then $\parELoop$ returns $\none$.
  
  \item Otherwise, it results in $\some(\pt',z)$ for some
    $\pt'\in{\cal T}$ and $z\in\Str$, and $\parELoop$ returns
    $\parELoop(E(\pt,\plussym,\pt'),z)$.
  \end{itemize}

\item Otherwise, $\parELoop$ returns $\some(\pt,x)$.
\end{itemize}
The function $\parT$ operates analogously.

Given a string $w$, $\parF$ proceeds as follows.
\begin{itemize}
\item If $w=\idsym x$ for some $x$, then it returns
  $\some(F(\idsym),x)$.

\item Otherwise, if $w=\openparsym x$, then $\parF$ evaluates
  $\parE\,x$.
  \begin{itemize}
  \item If this results in $\none$, it returns $\none$.
      
  \item Otherwise, this results in $\some(\pt,y)$ for some
    $\pt\in{\cal E}$ and $y\in\Str$.
    \begin{itemize}
    \item If $y=\closparsym z$ for some $z$, then $\parF$ returns
      \begin{displaymath}
       \some(F(\openparsym,\pt,\closparsym), z) . 
      \end{displaymath}
    
    \item Otherwise, $\parF$ returns $\none$.
    \end{itemize}
  \end{itemize}

  \item Otherwise $\parF$ returns $\none$.
\end{itemize}

Given a string $w$ to parse, the algorithm evaluates
$\parE\,w$.  If the result of this evaluation is:
\begin{itemize}
\item $\none$, then the algorithm reports failure;

\item $\some(\pt,\%)$, then the algorithm returns $\pt$;

\item $\some(\pt,y)$, where $y\neq\%$, then the algorithm reports failure,
  because not all of the input could be parsed.
\end{itemize}

\subsection{Notes}

The standard approach to doing top-down parsing in the presence of
left recursive productions is to first translate the left recursion to
right recursion, and then restructure the parse trees produced by the
parser.  In contrast, we showed a direct approach to handling left
recursion that works for grammars of operators.

%%% Local Variables: 
%%% mode: latex
%%% TeX-master: "book"
%%% End: 

\section{Closure Properties of Context-free Languages}
\label{ClosurePropertiesOfContextFreeLanguages}

\index{context-free language!closure properties|(}%
In this section, we define union, concatenation, closure, reversal,
alphabet-renaming and prefix-closure operations/algorithms on
grammars.  As a result, we will have that the context-free languages
are closed under union, concatenation, closure, reversal,
alphabet-renaming and prefix-, suffix- and substring-closure.

In Section~\ref{ThePumpingLemmaForContextFreeLanguages}, we
will see that the context-free languages aren't closed under
intersection, complementation and set difference.
But we are able to define operations/algorithms for:
\begin{itemize}
\item intersecting a grammar and an empty-string finite automaton; and

\item subtracting a deterministic finite automaton from a grammar.
\end{itemize}
Thus, if $L_1$ is a context-free language, and $L_2$ is a regular
language, we will have that $L_1\cap L_2$ and $L_1-L_2$ are
context-free.

\subsection{Operations on Grammars}

First, we consider some basic grammars and operations on grammars.
The grammar with variable $\Asf$ and production $\Asf\fun\%$
generates the language $\{\%\}$.
The grammar with variable $\Asf$ and no productions generates 
the language $\emptyset$.
If $w$ is a string, then the grammar with variable $\Asf$ and
production $\Asf\fun w$ generates the language $\{w\}$.
Actually, we must be careful to chose a variable that doesn't occur
in $w$.

Suppose $G_1$ and $G_2$ are grammars.  We can define a grammar $H$
such that $L(H)=L(G_1)\cup L(G_2)$ by unioning together the variables
and productions of $G_1$ and $G_2$, and adding a new start variable
$q$, along with productions
\begin{gather*}
  q\fun s_{G_1}\mid s_{G_2} .
\end{gather*}
For the above to be valid, we need to know that:
\begin{itemize}
\item $Q_{G_1}\cap Q_{G_2}=\emptyset$ and
$q\not\in Q_{G_1}\cup Q_{G_2}$; and

\item $\alphabet\,G_1\cap Q_{G_2}=\emptyset$,
$\alphabet\,G_2\cap Q_{G_1}=\emptyset$ and
$q\not\in\alphabet\,G_1\cup\alphabet\,G_2$.
\end{itemize}
Our official union operation for grammars renames the variables
of $G_1$ and $G_2$, and chooses the start variable $q$,
in a uniform way that makes the preceding properties hold.
We do something similar when defining the other closure
operations.  In what follows, though, we'll ignore this issue,
so as to keep things simple.

Suppose $G_1$ and $G_2$ are grammars.  We can define a grammar $H$
such that $L(H)=L(G_1)L(G_2)$ by unioning together the variables and
productions of $G_1$ and $G_2$, and adding a new start variable $q$,
along with production
\begin{gather*}
  q\fun s_{G_1}s_{G_2} .
\end{gather*}

Suppose $G$ is a grammar.  We can define a grammar $H$ such that
$L(H)=L(G)^*$ by adding to the variables and productions of $G$ a new
start variable $q$, along with productions
\begin{gather*}
q\fun \%\mid s_Gq .
\end{gather*}

Next, we consider reversal and alphabet renaming operations on
grammars.  Given a grammar $G$, we can define a grammar $H$ such that
\index{grammar!reversal}%
\index{reversal!grammar}%
\index{language!reversal}%
\index{reversal!language}%
$L(H)=L(G)^R$ by simply reversing the right-sides of $G$'s
productions.

Given a grammar $G$ and a bijection $f$ from a set of symbols that is
a superset of $\alphabet\,G$ to some set of symbols, we can define a
grammar $H$ such that $L(H)=L(G)^f$ by renaming the elements of
\index{grammar!alphabet renaming}%
\index{alphabet renaming!grammar}%
\index{language!alphabet renaming}%
\index{alphabet renaming!language}%
$\alphabet\,G$ in the right-sides of $G$'s productions using $f$.
Actually, we may have to rename the variables of $G$ to avoid clashes
with the elements of the renamed alphabet.

From Section~\ref{ClosurePropertiesOfRegularLanguages},
we know that if we can define a prefix-closure operation
\index{language!prefix-closure}%
\index{language!suffix-closure}%
\index{language!substring-closure}%
\index{prefix-closure!language}%
\index{suffix-closure!language}%
\index{substring-closure!language}%
\index{grammar!prefix-closure}%
\index{grammar!suffix-closure}%
\index{grammar!substring-closure}%
\index{prefix-closure!grammar}%
\index{suffix-closure!grammar}%
\index{substring-closure!grammar}%
on grammmars, then we can obtain suffix-closure and substring-closure
operations on grammars from the prefix-closure and grammar reversal
operations.

So how can we turn a grammar $G$ into a grammar $H$ such that
$L(H)=L(G)^P$?
We begin by simplifying $G$, producing grammar $G'$.
Thus all of the variables of $G'$ will be useful, unless $G'$ has
a single variable and no productions.  Now, we form the grammmar $H$
from $G'$, as follows.
We make a copy of $G'$, renaming each variable $q$ to
$\langle\onesf,q\rangle$.  (Actually, we may have to rename variables
to avoid clashes with alphabet symbols.)
Next, for each alphabet symbol $a$, we introduce a new variable
$\langle\twosf,a\rangle$, along with productions
$\langle\twosf,a\rangle\fun\%\mid a$.
Next, for each variable $q$ of $G'$, we add a new variable
$\langle\twosf,q\rangle$ that generates all prefixes of what $q$
generated in $G'$.  Suppose we are given a production $q\fun
a_1a_2\cdots a_n$ of $G'$.  If $n=0$, then we replace it with the
production $\langle\twosf,q\rangle\fun\%$.  Otherwise, we replace it
with the productions
\begin{gather*}
\langle\twosf,q\rangle\fun
\langle\twosf,a_1\rangle \mid
f(a_1)\langle\twosf,a_2\rangle \mid \cdots \mid
f(a_1)\,f(a_2)\cdots\langle\twosf,a_n\rangle ,
\end{gather*}
where $f(a) = a$, if $a\in\alphabet\,G'$, and
$f(a)=\langle\onesf,a\rangle$, if $a\in Q_{G'}$.
It's crucial that $G'$ is simplified; otherwise productions with
useless symbols would be turned into productions that generated
strings.
Finally, the start variable of $H$ is $\langle\twosf,s_{G'}\rangle$.

For example, the grammar
\begin{gather*}
\Asf\fun \% \mid \zerosf\Asf\onesf
\end{gather*}
is turned into the grammar
\begin{align*}
\langle \twosf,\Asf\rangle &\fun \% \mid
  \langle\twosf,\zerosf\rangle \mid
  \zerosf\langle\twosf,\Asf\rangle \mid
  \zerosf\langle\onesf,A\rangle\langle\twosf,\onesf\rangle , \\
\langle\onesf,A\rangle &\fun \% \mid
  \zerosf\langle\onesf,\Asf\rangle\onesf , \\
\langle\twosf,\zerosf\rangle &\fun \% \mid \zerosf , \\
\langle\twosf,\onesf\rangle &\fun \% \mid \onesf .
\end{align*}

We now consider an algorithm for intersecting a grammar $G$ with an
EFA $M$, resulting in $\simplify\,H$, where the grammar $H$ is defined
as follows.
For all $p\in Q_G$ and $q, r\in Q_M$, $H$ has a variable
$\langle p,q,r\rangle$ that generates
\begin{gather*}
\setof{w\in(\alphabet\,G)^*}{w\in\Pi_{G,p} \eqtxt{and}
 r\in\Delta(\{q\},w)} .
\end{gather*}
The remaining variable of $H$ is $\Asf$, which is its start variable.

For each $r\in A_M$, $H$ has a production
\begin{gather*}
  \Asf \fun \langle s_G, s_M, r\rangle .
\end{gather*}
And for each $\%$-production $p\fun\%$ of $G$ and $q,r\in Q_M$, if
$r\in \Delta(\{q\}, \%)$,
then $H$ will have the production
\begin{gather*}
  \langle p,q,r\rangle \fun \% .
\end{gather*}

To say what the remaining productions of $H$ are, define
a function
\begin{gather*}
f \in (\alphabet\,G \cup Q_G) \times Q_M \times Q_M\fun
\alphabet\,G\cup Q_H
\end{gather*}
by: for all $a\in\alphabet\,G \cup Q_G$ and $q,r\in Q_M$,
\begin{gather*}
  f(a, q, r) =
  \casesdef{a}{\eqtxtr{if} a\in\alphabet\,G , \eqtxtl{and}}%
  {\langle a,q,r\rangle}{\eqtxtr{if} a\in Q_G .}
\end{gather*}
Then, for all $p\in Q_G$, $n\in\nats-\{0\}$,
$a_1,\ldots,a_n\in\Sym$ and $q_1,\ldots,q_{n+1}\in Q_M$, if
\begin{itemize}
\item $p\fun a_1\cdots a_n\in P_G$, and

\item for all $i\in[1:n]$, if $a_i\in\alphabet\,G$, then
  $q_{i+1}\in\Delta(\{q_i\},a_i)$,
\end{itemize}
then
\begin{gather*}
  \langle p,q_1,q_{n+1}\rangle \fun 
  f(a_1,q_1,q_2) \cdots f(a_n,q_n,q_{n+1})
\end{gather*}
is a production of $H$.

For example, let $G$ be the grammar
\begin{gather*}
  \Asf \fun \% \mid \mathsf{0A1A} \mid \mathsf{1A0A} ,
\end{gather*}
and $M$ be the EFA
\begin{center}
  \input{chap-4.7-fig1.eepic}
\end{center}
so that $G$ generates all elements of $\{\mathsf{0,1}\}^*$
with an equal number of $\zerosf$'s and $\onesf$'s, and $M$
accepts $\{\zerosf\}^*\{\onesf\}^*$.
Then $\simplify\,H$ is
\begin{align*}
\Asf &\fun \langle\Asf,\Asf,\Bsf\rangle , \\
\langle\Asf,\Asf,\Asf\rangle &\fun \% , \\
\langle\Asf,\Asf,\Bsf\rangle &\fun \% , \\
\langle\Asf,\Asf,\Bsf\rangle &\fun
\zerosf\langle\Asf,\Asf,\Asf\rangle\onesf\langle\Asf,\Bsf,\Bsf\rangle , \\
\langle\Asf,\Asf,\Bsf\rangle &\fun
\zerosf\langle\Asf,\Asf,\Bsf\rangle\onesf\langle\Asf,\Bsf,\Bsf\rangle , \\
\langle\Asf,\Asf,\Bsf\rangle &\fun
\zerosf\langle\Asf,\Bsf,\Bsf\rangle\onesf\langle\Asf,\Bsf,\Bsf\rangle , \\
\langle\Asf,\Bsf,\Bsf\rangle &\fun \% .
\end{align*}
Note that simplification eliminated the variable
$\langle\Asf,\Bsf,\Asf\rangle$.

Finally, we consider a difference operation/algorithm.
Given a grammar $G$ and a DFA $M$, we can define the difference of
$G$ and $M$ to be
\begin{gather*}
\inter(G, \mycomplement(M, \alphabet\,G)) .
\end{gather*}
This is analogous to what we did when defining the difference of
DFAs.

The following theorem summarizes the closure properties for context-free
languages.

\begin{theorem}
Suppose $L,L_1,L_2\in\CFLan$ and $L'\in\RegLan$.
Then:
\begin{enumerate}[\quad(1)]
\item $L_1\cup L_2\in\CFLan$;

\item $L_1L_2\in\CFLan$;

\item $L^*\in\CFLan$;

\item $L^R\in\CFLan$;

\item $L^f\in\CFLan$, where $f$ is a bijection from a set of
symbols that is a superset of $\alphabet\,L$ to some
set of symbols;

\item $L^P\in\CFLan$;

\item $L^S\in\CFLan$;

\item $L^\SSop\in\CFLan$;

\item $L\cap L'\in\CFLan$; and

\item $L-L'\in\CFLan$.
\end{enumerate}
\end{theorem}

\subsection{Operations on Grammars in Forlan}

The Forlan module \texttt{Gram} defines the following constants
and operations on grammars:
\begin{verbatim}
val emptyStr       : gram
val emptySet       : gram
val fromStr        : str -> gram
val fromSym        : sym -> gram
val union          : gram * gram -> gram
val concat         : gram * gram -> gram
val closure        : gram -> gram
val rev            : gram -> gram
val renameAlphabet : gram * sym_rel -> gram
val prefix         : gram -> gram
val inter          : gram * efa -> gram
val minus          : gram * dfa -> gram
\end{verbatim}
\index{Gram@\texttt{Gram}!emptyStr@\texttt{emptyStr}}%
\index{Gram@\texttt{Gram}!emptySet@\texttt{emptySet}}%
\index{Gram@\texttt{Gram}!fromStr@\texttt{fromStr}}%
\index{Gram@\texttt{Gram}!fromSym@\texttt{fromSym}}%
\index{Gram@\texttt{Gram}!union@\texttt{union}}%
\index{Gram@\texttt{Gram}!concat@\texttt{concat}}%
\index{Gram@\texttt{Gram}!closure@\texttt{closure}}%
\index{Gram@\texttt{Gram}!rev@\texttt{rev}}%
\index{Gram@\texttt{Gram}!renameAlphabet@\texttt{renameAlphabet}}%
\index{Gram@\texttt{Gram}!prefix@\texttt{prefix}}%
\index{Gram@\texttt{Gram}!inter@\texttt{inter}}%
\index{Gram@\texttt{Gram}!minus@\texttt{minus}}%
The functions \texttt{fromStr} and \texttt{fromSym}
and are also available in the top-level environment with the names
\begin{verbatim}
val strToGram : str -> gram
val symToGram : sym -> gram
\end{verbatim}
\index{strToGram@\texttt{strToGram}}%
\index{symToGram@\texttt{symToGram}}%

For example, we can construct a grammar $G$ such that
$L(G)=\mathsf{\{01\}\cup\{10\}\{11\}^*}$, as follows.
\begin{list}{}
{\setlength{\leftmargin}{\leftmargini}
\setlength{\rightmargin}{0cm}
\setlength{\itemindent}{0cm}
\setlength{\listparindent}{0cm}
\setlength{\itemsep}{0cm}
\setlength{\parsep}{0cm}
\setlength{\labelsep}{0cm}
\setlength{\labelwidth}{0cm}
\catcode`\#=12
\catcode`\$=12
\catcode`\%=12
\catcode`\^=12
\catcode`\_=12
\catcode`\.=12
\catcode`\?=12
\catcode`\!=12
\catcode`\&=12
\ttfamily}
\small
\item[]\textsl{-\ }val\ gram1\ =\ strToGram(Str.fromString\ "01");
\item[]\textsl{val\ gram1\ =\ -\ :\ gram}
\item[]\textsl{-\ }val\ gram2\ =\ strToGram(Str.fromString\ "10");
\item[]\textsl{val\ gram2\ =\ -\ :\ gram}
\item[]\textsl{-\ }val\ gram3\ =\ strToGram(Str.fromString\ "11");
\item[]\textsl{val\ gram3\ =\ -\ :\ gram}
\item[]\textsl{-\ }val\ gram\ =
\item[]\textsl{=\ }Gram.union(gram1,
\item[]\textsl{=\ }\ \ \ \ \ \ \ \ \ \ \ Gram.concat(gram2,
\item[]\textsl{=\ }\ \ \ \ \ \ \ \ \ \ \ \ \ \ \ \ \ \ \ \ \ \ \ Gram.closure\ gram3));
\item[]\textsl{val\ gram\ =\ -\ :\ gram}
\item[]\textsl{-\ }Gram.output("",\ gram);
\item[]\textsl{\symbol{'173}variables\symbol{'175}\ A,\ <1,A>,\ <2,A>,\ <2,<1,A>>,\ <2,<2,A>>,\ <2,<2,<A>>>}
\item[]\textsl{\symbol{'173}start\ variable\symbol{'175}\ A}
\item[]\textsl{\symbol{'173}productions\symbol{'175}}
\item[]\textsl{A\ ->\ <1,A>\ |\ <2,A>;\ <1,A>\ ->\ 01;\ <2,A>\ ->\ <2,<1,A>><2,<2,A>>;}
\item[]\textsl{<2,<1,A>>\ ->\ 10;\ <2,<2,A>>\ ->\ %\ |\ <2,<2,<A>>><2,<2,A>>;}
\item[]\textsl{<2,<2,<A>>>\ ->\ 11}
\item[]\textsl{val\ it\ =\ ()\ :\ unit}
\item[]\textsl{-\ }val\ gram'\ =\ Gram.renameVariablesCanonically\ gram;
\item[]\textsl{val\ gram'\ =\ -\ :\ gram}
\item[]\textsl{-\ }Gram.output("",\ gram');
\item[]\textsl{\symbol{'173}variables\symbol{'175}\ A,\ B,\ C,\ D,\ E,\ F\ \symbol{'173}start\ variable\symbol{'175}\ A}
\item[]\textsl{\symbol{'173}productions\symbol{'175}}
\item[]\textsl{A\ ->\ B\ |\ C;\ B\ ->\ 01;\ C\ ->\ DE;\ D\ ->\ 10;\ E\ ->\ %\ |\ FE;\ F\ ->\ 11}
\item[]\textsl{val\ it\ =\ ()\ :\ unit}
\end{list}

Continuing our Forlan session, the grammar reversal and alphabet
renaming operations can be used as follows:
\begin{list}{}
{\setlength{\leftmargin}{\leftmargini}
\setlength{\rightmargin}{0cm}
\setlength{\itemindent}{0cm}
\setlength{\listparindent}{0cm}
\setlength{\itemsep}{0cm}
\setlength{\parsep}{0cm}
\setlength{\labelsep}{0cm}
\setlength{\labelwidth}{0cm}
\catcode`\#=12
\catcode`\$=12
\catcode`\%=12
\catcode`\^=12
\catcode`\_=12
\catcode`\.=12
\catcode`\?=12
\catcode`\!=12
\catcode`\&=12
\ttfamily}
\small
\item[]\textsl{-\ }val\ gram''\ =\ Gram.rev\ gram';
\item[]\textsl{val\ gram''\ =\ -\ :\ gram}
\item[]\textsl{-\ }Gram.output("",\ gram'');
\item[]\textsl{\symbol{'173}variables\symbol{'175}\ A,\ B,\ C,\ D,\ E,\ F\ \symbol{'173}start\ variable\symbol{'175}\ A}
\item[]\textsl{\symbol{'173}productions\symbol{'175}}
\item[]\textsl{A\ ->\ B\ |\ C;\ B\ ->\ 10;\ C\ ->\ ED;\ D\ ->\ 01;\ E\ ->\ %\ |\ EF;\ F\ ->\ 11}
\item[]\textsl{val\ it\ =\ ()\ :\ unit}
\item[]\textsl{-\ }val\ rel\ =\ SymRel.fromString\ "(0,\ A),\ (1,\ B)";
\item[]\textsl{val\ rel\ =\ -\ :\ sym_rel}
\item[]\textsl{-\ }val\ gram'''\ =\ Gram.renameAlphabet(gram'',\ rel);
\item[]\textsl{val\ gram'''\ =\ -\ :\ gram}
\item[]\textsl{-\ }Gram.output("",\ gram''');
\item[]\textsl{\symbol{'173}variables\symbol{'175}\ <A>,\ <B>,\ <C>,\ <D>,\ <E>,\ <F>\ \symbol{'173}start\ variable\symbol{'175}\ <A>}
\item[]\textsl{\symbol{'173}productions\symbol{'175}}
\item[]\textsl{<A>\ ->\ <B>\ |\ <C>;\ <B>\ ->\ BA;\ <C>\ ->\ <E><D>;\ <D>\ ->\ AB;}
\item[]\textsl{<E>\ ->\ %\ |\ <E><F>;\ <F>\ ->\ BB}
\item[]\textsl{val\ it\ =\ ()\ :\ unit}
\end{list}

And here is an example use of the prefix-closure operation:
\begin{list}{}
{\setlength{\leftmargin}{\leftmargini}
\setlength{\rightmargin}{0cm}
\setlength{\itemindent}{0cm}
\setlength{\listparindent}{0cm}
\setlength{\itemsep}{0cm}
\setlength{\parsep}{0cm}
\setlength{\labelsep}{0cm}
\setlength{\labelwidth}{0cm}
\catcode`\#=12
\catcode`\$=12
\catcode`\%=12
\catcode`\^=12
\catcode`\_=12
\catcode`\.=12
\catcode`\?=12
\catcode`\!=12
\catcode`\&=12
\ttfamily}
\small
\item[]\textsl{-\ }val\ gram\ =\ Gram.input\ "";
\item[]\textsl{@\ }\symbol{'173}variables\symbol{'175}
\item[]\textsl{@\ }A
\item[]\textsl{@\ }\symbol{'173}start\ variable\symbol{'175}
\item[]\textsl{@\ }A
\item[]\textsl{@\ }\symbol{'173}productions\symbol{'175}
\item[]\textsl{@\ }A\ ->\ %\ |\ 0A1
\item[]\textsl{@\ }.
\item[]\textsl{val\ gram\ =\ -\ :\ gram}
\item[]\textsl{-\ }val\ gram'\ =\ Gram.prefix\ gram;
\item[]\textsl{val\ gram'\ =\ -\ :\ gram}
\item[]\textsl{-\ }Gram.output("",\ gram');
\item[]\textsl{\symbol{'173}variables\symbol{'175}\ <1,A>,\ <2,0>,\ <2,1>,\ <2,A>\ \symbol{'173}start\ variable\symbol{'175}\ <2,A>}
\item[]\textsl{\symbol{'173}productions\symbol{'175}}
\item[]\textsl{<1,A>\ ->\ %\ |\ 0<1,A>1;\ <2,0>\ ->\ %\ |\ 0;\ <2,1>\ ->\ %\ |\ 1;}
\item[]\textsl{<2,A>\ ->\ %\ |\ <2,0>\ |\ 0<2,A>\ |\ 0<1,A><2,1>}
\item[]\textsl{val\ it\ =\ ()\ :\ unit}
\item[]\textsl{-\ }fun\ test\ s\ =\ Gram.generated\ gram'\ (Str.fromString\ s);
\item[]\textsl{val\ test\ =\ fn\ :\ string\ ->\ bool}
\item[]\textsl{-\ }test\ "000111";
\item[]\textsl{val\ it\ =\ true\ :\ bool}
\item[]\textsl{-\ }test\ "0001";
\item[]\textsl{val\ it\ =\ true\ :\ bool}
\item[]\textsl{-\ }test\ "0001111";
\item[]\textsl{val\ it\ =\ false\ :\ bool}
\end{list}


Finally, to see how we can use \texttt{Gram.inter} and
\texttt{Gram.minus}, let \texttt{gram} be the grammar
\begin{gather*}
  \Asf \fun \% \mid \mathsf{0A1A} \mid \mathsf{1A0A} ,
\end{gather*}
and \texttt{efa} be the EFA
\begin{center}
  \input{chap-4.7-fig1.eepic}
\end{center}
\begin{list}{}
{\setlength{\leftmargin}{\leftmargini}
\setlength{\rightmargin}{0cm}
\setlength{\itemindent}{0cm}
\setlength{\listparindent}{0cm}
\setlength{\itemsep}{0cm}
\setlength{\parsep}{0cm}
\setlength{\labelsep}{0cm}
\setlength{\labelwidth}{0cm}
\catcode`\#=12
\catcode`\$=12
\catcode`\%=12
\catcode`\^=12
\catcode`\_=12
\catcode`\.=12
\catcode`\?=12
\catcode`\!=12
\catcode`\&=12
\ttfamily}
\small
\item[]\textsl{-\ }val\ gram'\ =\ Gram.inter(gram,\ efa);
\item[]\textsl{val\ gram'\ =\ -\ :\ gram}
\item[]\textsl{-\ }Gram.output("",\ gram');
\item[]\textsl{\symbol{'173}variables\symbol{'175}\ A,\ <A,A,A>,\ <A,A,B>,\ <A,B,B>\ \symbol{'173}start\ variable\symbol{'175}\ A}
\item[]\textsl{\symbol{'173}productions\symbol{'175}}
\item[]\textsl{A\ ->\ <A,A,B>;\ <A,A,A>\ ->\ %;}
\item[]\textsl{<A,A,B>\ ->}
\item[]\textsl
\item[]\textsl{val\ it\ =\ ()\ :\ unit}
\item[]\textsl{-\ }val\ gram''\ =
\item[]\textsl{=\ }\ \ \ \ \ \ Gram.eliminateVariable
\item[]\textsl{=\ }\ \ \ \ \ \ (Gram.eliminateVariable(gram',\ Sym.fromString\ "<A,A,A>"),
\item[]\textsl{=\ }\ \ \ \ \ \ \ Sym.fromString\ "<A,B,B>");
\item[]\textsl{val\ gram''\ =\ -\ :\ gram}
\item[]\textsl{-\ }val\ gram'''\ =
\item[]\textsl{=\ }\ \ \ \ \ \ Gram.renameVariablesCanonically
\item[]\textsl{=\ }\ \ \ \ \ \ (Gram.restart(Gram.simplify\ gram''));
\item[]\textsl{val\ gram'''\ =\ -\ :\ gram}
\item[]\textsl{-\ }Gram.output("",\ gram''');
\item[]\textsl{\symbol{'173}variables\symbol{'175}\ A\ \symbol{'173}start\ variable\symbol{'175}\ A\ \symbol{'173}productions\symbol{'175}\ A\ ->\ %\ |\ 0A1}
\item[]\textsl{val\ it\ =\ ()\ :\ unit}
\item[]\textsl{-\ }Gram.generated\ gram'''\ (Str.fromString\ "0011");
\item[]\textsl{val\ it\ =\ true\ :\ bool}
\item[]\textsl{-\ }Gram.generated\ gram'''\ (Str.fromString\ "0101");
\item[]\textsl{val\ it\ =\ false\ :\ bool}
\item[]\textsl{-\ }Gram.generated\ gram'''\ (Str.fromString\ "0001");
\item[]\textsl{val\ it\ =\ false\ :\ bool}
\item[]\textsl{-\ }val\ dfa\ =
\item[]\textsl{=\ }\ \ \ \ \ \ DFA.renameStatesCanonically
\item[]\textsl{=\ }\ \ \ \ \ \ (DFA.minimize(nfaToDFA(efaToNFA\ efa)));
\item[]\textsl{val\ dfa\ =\ -\ :\ dfa}
\item[]\textsl{-\ }val\ gram''\ =\ Gram.minus(gram,\ dfa);
\item[]\textsl{val\ gram''\ =\ -\ :\ gram}
\item[]\textsl{-\ }Gram.generated\ gram''\ (Str.fromString\ "0101");
\item[]\textsl{val\ it\ =\ true\ :\ bool}
\item[]\textsl{-\ }Gram.generated\ gram''\ (Str.fromString\ "0011");
\item[]\textsl{val\ it\ =\ false\ :\ bool}
\end{list}


\index{context-free language!closure properties|)}%

\subsection{Notes}

The algorithm for intersecting a grammar with an EFA would normally be given
only indirectly, using push down automata (PDAs): one could convert a grammar
to a PDF, do the intersection there, and convert back to a grammar.  Our direct
algorithm is motivated by this process, but produces grammars that are more
intelligible.

%%% Local Variables: 
%%% mode: latex
%%% TeX-master: "book"
%%% End: 

\section{Converting Regular Expressions and FA to Grammars}
\label{ConvertingRegularExpressionsAndFAToGrammars}

In this section, we give simple algorithms for converting regular
expressions and finite automata to grammars.  Since we have algorithms
for converting between regular expressions and finite automata, it is
tempting to only define one of these algorithms.  But better results
can be obtained by defining direct conversions.

\subsection{Converting Regular Expressions to Grammars}

Regular expressions are converted to grammars using a recursive
algorithm that makes use of the operations on grammars that were
defined in Section~\ref{ClosurePropertiesOfContextFreeLanguages}.  The
structure of the algorithm is very similar to the structure of our
algorithm for converting regular expressions to finite automata.

The algorithm is implemented in Forlan by the function
\begin{verbatim}
val fromReg : reg -> gram
\end{verbatim}
of the \texttt{Gram} module.  It's available in the top-level
environment with the name \texttt{regToGram}.

Here is how we can convert the regular expression $\mathsf{01 +
10(11)^*}$ to a grammar using Forlan:
\begin{list}{}
{\setlength{\leftmargin}{\leftmargini}
\setlength{\rightmargin}{0cm}
\setlength{\itemindent}{0cm}
\setlength{\listparindent}{0cm}
\setlength{\itemsep}{0cm}
\setlength{\parsep}{0cm}
\setlength{\labelsep}{0cm}
\setlength{\labelwidth}{0cm}
\catcode`\#=12
\catcode`\$=12
\catcode`\%=12
\catcode`\^=12
\catcode`\_=12
\catcode`\.=12
\catcode`\?=12
\catcode`\!=12
\catcode`\&=12
\ttfamily}
\small
\item[]\textsl{-\ }val\ gram\ =\ regToGram(Reg.input\ "");
\item[]\textsl{@\ }01\ +\ 10(11)\symbol{'052}
\item[]\textsl{@\ }.
\item[]\textsl{val\ gram\ =\ -\ :\ gram}
\item[]\textsl{-\ }Gram.output("",\ Gram.renameVariablesCanonically\ gram);
\item[]\textsl{\symbol{'173}variables\symbol{'175}\ A,\ B,\ C,\ D,\ E,\ F\ \symbol{'173}start\ variable\symbol{'175}\ A}
\item[]\textsl{\symbol{'173}productions\symbol{'175}}
\item[]\textsl{A\ ->\ B\ |\ C;\ B\ ->\ 01;\ C\ ->\ DE;\ D\ ->\ 10;\ E\ ->\ %\ |\ FE;\ F\ ->\ 11}
\item[]\textsl{val\ it\ =\ ()\ :\ unit}
\end{list}


\subsection{Converting Finite Automata to Grammars}

We'll explain the process of converting finite automata to grammars
using an example.  Suppose $M$ is the DFA
\begin{center}
  \input{chap-4.8-fig1.eepic}
\end{center}
The variables of our grammar $G$ consist of the states of $M$, and its
start variable is the start state $\Asf$ of $M$.
(If the symbols of the labels of $M$'s transitions conflict with $M$'s states,
we'll have to rename the states of $M$ first.)
We can translate each transition $q, x\fun r$ to a production
$q\fun xr$.  And, since $\Asf$ is an accepting state of
$M$, we add the production $\Asf\fun\%$.
This gives us the grammar
\begin{align*}
  \Asf &\fun \% \mid \mathsf{0B} \mid \mathsf{1A} , \\
  \Bsf &\fun \mathsf{0A} \mid \mathsf{1B} .
\end{align*}

Consider, e.g., the valid labeled path for $M$
\begin{gather*}
\Asf\lparr{\onesf}\Asf\lparr{\zerosf}\Bsf\lparr{\zerosf}\Asf ,
\end{gather*}
which explains why $\mathsf{100}\in L(M)$.  It corresponds to the
valid parse tree for $G$
\begin{center}
\input{chap-4.8-fig2.eepic}
\end{center}
which explains why $\mathsf{100}\in L(G)$.

The Forlan module \texttt{Gram} contains the function
\begin{verbatim}
val fromFA : fa -> gram
\end{verbatim}
which implements our algorithm for converting finite automata to
grammars.  It's available in the top-level environment with the name
texttt{faToGram}.

Suppose \texttt{fa} of type \texttt{fa} is bound to $M$.  Here is how
we can convert $M$ to a grammar using Forlan:
\begin{list}{}
{\setlength{\leftmargin}{\leftmargini}
\setlength{\rightmargin}{0cm}
\setlength{\itemindent}{0cm}
\setlength{\listparindent}{0cm}
\setlength{\itemsep}{0cm}
\setlength{\parsep}{0cm}
\setlength{\labelsep}{0cm}
\setlength{\labelwidth}{0cm}
\catcode`\#=12
\catcode`\$=12
\catcode`\%=12
\catcode`\^=12
\catcode`\_=12
\catcode`\.=12
\catcode`\?=12
\catcode`\!=12
\catcode`\&=12
\ttfamily}
\small
\item[]\textsl{-\ }val\ gram\ =\ faToGram\ fa;
\item[]\textsl{val\ gram\ =\ -\ :\ gram}
\item[]\textsl{-\ }Gram.output("",\ gram);
\item[]\textsl{\symbol{'173}variables\symbol{'175}\ A,\ B\ \symbol{'173}start\ variable\symbol{'175}\ A}
\item[]\textsl{\symbol{'173}productions\symbol{'175}\ A\ ->\ %\ |\ 0B\ |\ 1A;\ B\ ->\ 0A\ |\ 1B}
\item[]\textsl{val\ it\ =\ ()\ :\ unit}
\end{list}


Because of the existence of our conversion functions, we have that
every regular language is a context-free language.
On the other hand, the language $\setof{\zerosf^n\onesf^n}{n\in\nats}$
is context-free, because of the grammar
\begin{gather*}
  \Asf\fun\%\mid\zerosf\Asf\onesf ,
\end{gather*}
but is not regular, as we proved in Section 3.13.

Summarizing, we have:

\begin{theorem}
The regular languages are a proper subset of the context-free
languages: $\RegLan\subsetneq\CFLan$.
\end{theorem}

\subsection{Notes}

The material in this section is standard.

%%% Local Variables: 
%%% mode: latex
%%% TeX-master: "book"
%%% End: 

\section{Chomsky Normal Form}
\label{ChomskyNormalForm}

In this section, we study a special form of grammars called Chomsky
Normal Form (CNF), which was named after the linguist Noam Chomsky.
Grammars in CNF have very nice formal properties.  In particular,
valid parse trees for grammars in CNF are very close to being binary
trees.

Any grammar that doesn't generate $\%$ can be put in CNF.  And, if $G$
is a grammar that does generate $\%$, it can be turned into a grammar
in CNF that generates $L(G)-\{\%\}$.  In the next section, we will use
this fact when proving the pumping lemma for context-free languages, a
method for showing the certain languages are not context-free.

We will begin by giving an algorithm for turning a grammar $G$ into a
simplified grammar with no productions of the form $q\fun\%$ and
$q\fun r$. This will enable us to give an algorithm that takes in a
grammar $G$, and calculates $L(G)$, when it is finite, and reports
that it is infinite, otherwise.

\subsection{Removing $\%$-Productions}

A $\%$-\emph{production} is a production of the form $q\fun\%$.  We
will show by example how to turn a grammar $G$ into a simplified
grammar with no $\%$-productions that generates $L(G)-\{\%\}$.

Suppose $G$ is the grammar
\begin{align*}
\Asf &\fun \mathsf{0A1} \mid \mathsf{BB} , \\
\Bsf &\fun \% \mid \mathsf{2B} .
\end{align*}

First, we determine which variables $q$ are \emph{nullable} in the
sense that $\%\in\Pi_q$, i.e., that $\%$ is the yield of a valid parse
tree for $G$ whose root label is $q$.
\begin{itemize}
\item Clearly, $\Bsf$ is nullable.

\item Since $\Asf\fun\Bsf\Bsf\in P_G$, it follows that $\Asf$ is
  nullable.
\end{itemize}

Now we use this information to compute the productions of
our new grammar.
\begin{itemize}
\item Since $\Asf$ is nullable, we replace the production
  $\Asf\fun\mathsf{0A1}$ with the productions $\Asf\fun\mathsf{0A1}$
  and $\Asf\fun\mathsf{01}$.  The idea is that this second production
  will make up for the fact that $\Asf$ won't be nullable in the new
  grammar.

\item Since $\Bsf$ is nullable, we replace the production
  $\Asf\fun\mathsf{BB}$ with the productions $\Asf\fun\mathsf{BB}$ and
  $\Asf\fun\mathsf{B}$ (the result of deleting either one of the
  $\Bsf$'s).

\item The production $\Bsf\fun\%$ is deleted.

\item Since $\Bsf$ is nullable, we replace the production
  $\Bsf\fun\mathsf{2B}$ with the productions $\Bsf\fun\mathsf{2B}$ and
  $\Bsf\fun\mathsf{2}$.
\end{itemize}
This give us the grammar
\begin{align*}
\Asf &\fun \mathsf{0A1} \mid \mathsf{01} \mid \mathsf{BB} \mid \Bsf , \\
\Bsf &\fun \mathsf{2B} \mid \twosf .
\end{align*}
In general, we finish by simplifying our new grammar.  The new grammar
of our example is already simplified, however.

\subsection{Removing Unit Productions}

A \emph{unit production} for a grammar $G$ is a production of the
form $q\fun r$, where $r$ is a variable (possibly equal to $q$).  We
now show by example how to turn a grammar $G$ into a simplified
grammar with no $\%$-productions or unit productions that generates
$L(G)-\{\%\}$.

Suppose $G$ is the grammar
\begin{align*}
\Asf &\fun \mathsf{0A1} \mid \mathsf{01} \mid \mathsf{BB} \mid \Bsf , \\
\Bsf &\fun \mathsf{2B} \mid \twosf .
\end{align*}
We begin by applying our algorithm for removing $\%$-productions to
our grammar; the algorithm has no effect in this case.

Our new grammar will have the same variables and start variable as
$G$.  Its set of productions is the set of all $q\fun w$ such that
$q$ is a variable of $G$, $w\in\Str$ doesn't consist of a single variable
of $G$, and there is a variable $r$ such that
\begin{itemize}
\item $r$ is parsable from $q$, and

\item $r\fun w$ is a production of $G$.
\end{itemize}
(Determining whether $r$ is parsable from $q$ is easy, since we are
working with a grammar with no $\%$-productions.)

This process results in the grammar
\begin{align*}
\Asf &\fun \mathsf{0A1} \mid \mathsf{01} \mid \mathsf{BB} \mid
\mathsf{2B} \mid \twosf , \\
\Bsf &\fun \mathsf{2B} \mid \twosf .
\end{align*}
Finally, we simplify our grammar, which gets rid of the production
$\Asf\fun\mathsf{2B}$, giving us the grammar
\begin{align*}
\Asf &\fun \mathsf{0A1} \mid \mathsf{01} \mid \mathsf{BB} \mid \twosf , \\
\Bsf &\fun \mathsf{2B} \mid \twosf .
\end{align*}

\subsection*{Removing $\%$ and Unit Productions in Forlan}

The Forlan module \texttt{Gram} defines the following functions:
\begin{verbatim}
val eliminateEmptyProductions        : gram -> gram
val eliminateEmptyAndUnitProductions : gram -> gram
\end{verbatim}
For example, if \texttt{gram} is the grammar
\begin{align*}
\Asf &\fun \mathsf{0A1} \mid \mathsf{BB} , \\
\Bsf &\fun \% \mid \mathsf{2B} .
\end{align*}
then we can proceed as follows.
\begin{list}{}
{\setlength{\leftmargin}{\leftmargini}
\setlength{\rightmargin}{0cm}
\setlength{\itemindent}{0cm}
\setlength{\listparindent}{0cm}
\setlength{\itemsep}{0cm}
\setlength{\parsep}{0cm}
\setlength{\labelsep}{0cm}
\setlength{\labelwidth}{0cm}
\catcode`\#=12
\catcode`\$=12
\catcode`\%=12
\catcode`\^=12
\catcode`\_=12
\catcode`\.=12
\catcode`\?=12
\catcode`\!=12
\catcode`\&=12
\ttfamily}
\small
\item[]\textsl{-\ }val\ gram'\ =\ Gram.eliminateEmptyProductions\ gram;
\item[]\textsl{val\ gram'\ =\ -\ :\ gram}
\item[]\textsl{-\ }Gram.output("",\ gram');
\item[]\textsl{\symbol{'173}variables\symbol{'175}\ A,\ B\ \symbol{'173}start\ variable\symbol{'175}\ A}
\item[]\textsl{\symbol{'173}productions\symbol{'175}\ A\ ->\ B\ |\ 01\ |\ BB\ |\ 0A1;\ B\ ->\ 2\ |\ 2B}
\item[]\textsl{val\ it\ =\ ()\ :\ unit}
\item[]\textsl{-\ }val\ gram''\ =\ Gram.eliminateEmptyAndUnitProductions\ gram;
\item[]\textsl{val\ gram''\ =\ -\ :\ gram}
\item[]\textsl{-\ }Gram.output("",\ gram'');
\item[]\textsl{\symbol{'173}variables\symbol{'175}\ A,\ B\ \symbol{'173}start\ variable\symbol{'175}\ A}
\item[]\textsl{\symbol{'173}productions\symbol{'175}\ A\ ->\ 2\ |\ 01\ |\ BB\ |\ 0A1;\ B\ ->\ 2\ |\ 2B}
\item[]\textsl{val\ it\ =\ ()\ :\ unit}
\end{list}


\subsection{Generating a Grammar's Language When Finite}

We can now give an algorithm that takes in a grammar $G$ and generates
$L(G)$, when it is finite, and reports that $L(G)$ is infinite,
otherwise. The algorithm begins by letting $G'$ be the result of
eliminating $\%$-productions and unit productions from $G$. Thus $G'$
is simplified and generates $L(G)-\{\%\}$.

If there is recursion in the productions of $G'$---either direct or
mutual---then there is a variable $q$ of $G'$ and a valid parse tree
$\pt$ for $G'$, such that the height of $\pt$ is at least one, $q$ is
the root label of $\pt$, and the yield of $\pt$ has the form $xqy$,
for strings $x$ and $y$, each of whose symbols is in
$\alphabet\,G'\cup Q_{G'}$. (In particular, $q$ may appear in $x$ or
$y$.) Because $G'$ lacks $\%$- and unit-productions, it follows that
$x\neq\%$ or $y\neq\%$. Because each variable of $G'$ is generating,
we can turn $\pt$ into a valid parse tree $\pt'$ whose root label is
$q$, and whose yield has the form $uqv$, for
$u,v\in(\alphabet\,G')^*$, where $u\neq\%$ or $v\neq\%$.

Thus we have that $uqv\in\Pi_{G',q}$, and an easy mathemtical
induction shows that $u^nqv^n\in\Pi_q$ for all $n\in\nats$.  Because
$u\neq\%$ or $v\neq\%$, it follows that there are infinitely many
strings generated from $q$ in $G'$. And, since $q$ is reachable, and
every variable of $G'$ is generating, it follows that $L(G')$, and
thus $L(G)$, is infinite.

Consequently, our algorthim can continue as follows. If the
productions of $G'$ have recursion, then it reports that $L(G)$ is
infinite.  Otherwise, it calculates $L(G')$ from the bottom-up, and
adds $\%$ iff $G$ generates $\%$.

The Forlan module \texttt{Gram} defines the following function:
\begin{verbatim}
val toStrSet : gram -> str set
\end{verbatim}
Suppose \texttt{gram} is the grammar
\begin{align*}
\Asf &\fun \mathsf{BB}, \\
\Bsf &\fun \mathsf{CC}, \\
\Csf &\fun \% \mid \zerosf \mid \onesf ,
\end{align*}
and
\texttt{gram'} is the grammar
\begin{align*}
\Asf &\fun \mathsf{BB}, \\
\Bsf &\fun \mathsf{CC}, \\
\Csf &\fun \% \mid \zerosf \mid \onesf \mid \Asf .
\end{align*}
Then we can proceed as follows:
\begin{list}{}
{\setlength{\leftmargin}{\leftmargini}
\setlength{\rightmargin}{0cm}
\setlength{\itemindent}{0cm}
\setlength{\listparindent}{0cm}
\setlength{\itemsep}{0cm}
\setlength{\parsep}{0cm}
\setlength{\labelsep}{0cm}
\setlength{\labelwidth}{0cm}
\catcode`\#=12
\catcode`\$=12
\catcode`\%=12
\catcode`\^=12
\catcode`\_=12
\catcode`\.=12
\catcode`\?=12
\catcode`\!=12
\catcode`\&=12
\ttfamily}
\small
\item[]\textsl{-\ }StrSet.output("",\ Gram.toStrSet\ gram);
\item[]\textsl{%,\ 0,\ 1,\ 00,\ 01,\ 10,\ 11,\ 000,\ 001,\ 010,\ 011,\ 100,\ 101,\ 110,\ 111,}
\item[]\textsl{0000,\ 0001,\ 0010,\ 0011,\ 0100,\ 0101,\ 0110,\ 0111,\ 1000,\ 1001,\ 1010,}
\item[]\textsl{1011,\ 1100,\ 1101,\ 1110,\ 1111}
\item[]\textsl{val\ it\ =\ ()\ :\ unit}
\item[]\textsl{-\ }StrSet.output("",\ Gram.toStrSet\ gram');
\item[]\textsl{language\ is\ infinite}
\item[]
\item[]\textsl{uncaught\ exception\ Error}
\end{list}
  

Suppose we have a grammar $G$ and a natural number $n$, and we wish to
generate the set of all elements of $L(G)$ of length $n$.  We can start by
creating an EFA $M$ accepting all strings over the alphabet of $G$
with length $n$.  Then, we can intersect $G$ with $M$, and apply
\texttt{Gram.toStrSet} to the resulting grammar.

\subsection{Chomsky Normal Form}

A grammar $G$ is in \emph{Chomsky Normal Form} (CNF) iff
each of its productions has one of the following forms:
\begin{itemize}
\item $q\fun a$, where $a$ is not a variable; and

\item $q\fun pr$, where $p$ and $r$ are variables.
\end{itemize}

We explain by example how a grammar $G$ can be turned into a
simplified grammar in CNF that generates $L(G)-\{\%\}$.
Suppose $G$ is the grammar
\begin{align*}
\Asf &\fun \mathsf{0A1} \mid \mathsf{01} \mid \mathsf{BB} \mid \twosf , \\
\Bsf &\fun \mathsf{2B} \mid \twosf .
\end{align*}
\begin{itemize}
\item We begin by applying our algorithm for removing $\%$-productions
  and unit productions to this grammar.  In this case, it has no
  effect.

\item Since the productions $\Asf\fun\mathsf{BB}$, $\Asf\fun\twosf$
  and $\Bsf\fun\twosf$ are legal CNF productions, we simply transfer
  them to our new grammar.

\item Next we add the variables $\Zero$, $\One$ and $\Two$ to our
  grammar, along with the productions
  \begin{gather*}
    \Zero\fun\zerosf,\quad\One\fun\onesf,\quad\Two\fun\twosf .
  \end{gather*}

\item Now, we can replace the production $\Asf\fun\mathsf{01}$ with
  $\Asf\fun\Zero\One$.  And, we can replace the production
  $\Bsf\fun\mathsf{2B}$ with the production $\Bsf\fun\Two\Bsf$.

\item Finally, we replace the production $\Asf\fun\mathsf{0A1}$ with
  the productions
  \begin{gather*}
    \Asf\fun\Zero\Csf,\quad\Csf\fun\Asf\One ,
  \end{gather*}
  and add $\Csf$ to the set of variables of our new grammar.
\end{itemize}
Summarizing, our new grammar is
\begin{align*}
  \Asf &\fun \mathsf{BB} \mid \twosf \mid \Zero\One \mid
  \Zero\Csf , \\
  \Bsf &\fun \twosf \mid \Two\Bsf , \\
  \Zero &\fun \zerosf , \\
  \One &\fun \onesf , \\
  \Two &\fun \twosf , \\
  \Csf &\fun \Asf\One .
\end{align*}

The official version of our algorithm names variables in a different
way.

\subsection*{Converting to Chomsky Normal Form in Forlan}

The Forlan module \texttt{Gram} defines the following function:
\begin{verbatim}
val chomskyNormalForm : gram -> gram
\end{verbatim}
Suppose \texttt{gram} of type \texttt{gram} is bound to the grammar with
variables $\Asf$ and $\Bsf$, start variable $\Asf$, and productions
\begin{align*}
  \Asf &\fun \mathsf{0A1} \mid \mathsf{BB} , \\
  \Bsf &\fun \% \mid \mathsf{2B} .
\end{align*}
Here is how Forlan can be used to turn this grammar into a CNF
grammar that generates the nonempty strings that are generated by
texttt{gram}:
\begin{list}{}
{\setlength{\leftmargin}{\leftmargini}
\setlength{\rightmargin}{0cm}
\setlength{\itemindent}{0cm}
\setlength{\listparindent}{0cm}
\setlength{\itemsep}{0cm}
\setlength{\parsep}{0cm}
\setlength{\labelsep}{0cm}
\setlength{\labelwidth}{0cm}
\catcode`\#=12
\catcode`\$=12
\catcode`\%=12
\catcode`\^=12
\catcode`\_=12
\catcode`\.=12
\catcode`\?=12
\catcode`\!=12
\catcode`\&=12
\ttfamily}
\small
\item[]\textsl{-\ }val\ gram'\ =\ Gram.chomskyNormalForm\ gram;
\item[]\textsl{val\ gram'\ =\ -\ :\ gram}
\item[]\textsl{-\ }Gram.output("",\ gram');
\item[]\textsl{\symbol{'173}variables\symbol{'175}\ <1,A>,\ <1,B>,\ <2,0>,\ <2,1>,\ <2,2>,\ <3,A1>}
\item[]\textsl{\symbol{'173}start\ variable\symbol{'175}\ <1,A>}
\item[]\textsl{\symbol{'173}productions\symbol{'175}}
\item[]\textsl{<1,A>\ ->\ 2\ |\ <1,B><1,B>\ |\ <2,0><2,1>\ |\ <2,0><3,A1>;}
\item[]\textsl{<1,B>\ ->\ 2\ |\ <2,2><1,B>;\ <2,0>\ ->\ 0;\ <2,1>\ ->\ 1;\ <2,2>\ ->\ 2;}
\item[]\textsl{<3,A1>\ ->\ <1,A><2,1>}
\item[]\textsl{val\ it\ =\ ()\ :\ unit}
\item[]\textsl{-\ }val\ gram''\ =\ Gram.renameVariablesCanonically\ gram';
\item[]\textsl{val\ gram''\ =\ -\ :\ gram}
\item[]\textsl{-\ }Gram.output("",\ gram'');
\item[]\textsl{\symbol{'173}variables\symbol{'175}\ A,\ B,\ C,\ D,\ E,\ F\ \symbol{'173}start\ variable\symbol{'175}\ A}
\item[]\textsl{\symbol{'173}productions\symbol{'175}}
\item[]\textsl{A\ ->\ 2\ |\ BB\ |\ CD\ |\ CF;\ B\ ->\ 2\ |\ EB;\ C\ ->\ 0;\ D\ ->\ 1;\ E\ ->\ 2;}
\item[]\textsl{F\ ->\ AD}
\item[]\textsl{val\ it\ =\ ()\ :\ unit}
\end{list}


\subsection{Notes}

The material in this section is standard.

%%% Local Variables: 
%%% mode: latex
%%% TeX-master: "book"
%%% End: 

\section{The Pumping Lemma for Context-free Languages}
\label{ThePumpingLemmaForContextFreeLanguages}

\index{pumping lemma!context-free languages|(}%
\index{context-free language!pumping lemma|(}%

Consider the language $L =
\setof{\zerosf^n\onesf^n\twosf^n}{n\in\nats}$.  Is $L$ context-free,
i.e., is there a grammar that generates $L$?  Although it's easy to
find a grammar that keeps the $\zerosf$'s and $\onesf$'s matched, or
one that keeps the $\onesf$'s and $\twosf$'s matched, or one that
keeps the $\zerosf$'s and $\twosf$'s matched, it seems that there is
no way to keep all three symbols matched simultaneously.

In this section, we will study the pumping lemma for context-free
languages, which can be used to show that many languages are not
context-free.  We will use the pumping lemma to prove that $L$ is not
context-free, and then we will prove the lemma.  Building on this
result, we'll be able to show that the context-free languages are not
closed under intersection, complementation or set-difference.

\subsection{Statement, Application and Proof of Pumping Lemma}

\begin{lemma}[Pumping Lemma for Context Free Languages]
For all context-free languages $L$, there is a
$n\in\nats-\{0\}$ such that, for all $z\in\Str$, if $z\in L$
and $|z|\geq n$, then there are $u,v,w,x,y\in\Str$ such
that $z=uvwxy$ and
\begin{enumerate}[\quad(1)]
\item $|vwx|\leq n$;

\item $vx\neq\%$; and

\item $uv^iwx^iy\in L$, for all $i\in\nats$.
\end{enumerate}
\end{lemma}

Before proving the pumping lemma, let's see how it can be used to show
that $L=\setof{\zerosf^n\onesf^n\twosf^n}{n\in\nats}$ is not
context-free.  Suppose, toward a contradiction that $L$ is
context-free.  Thus there is an $n\in\nats-\{0\}$ with the property of the
lemma.  Let $z={\zerosf^n\onesf^n\twosf^n}$.
Since $z\in L$ and $|z|=3n\geq n$, we have that there are
$u,v,w,x,y\in\Str$ such that $z=uvwxy$ and
\begin{enumerate}[\quad(1)]
\item $|vwx|\leq n$;

\item $vx\neq\%$; and

\item $uv^iwx^iy\in L$, for all $i\in\nats$.
\end{enumerate}
Since $\zerosf^n\onesf^n\twosf^n=z=uvwxy$, (1) tells us that
$vwx$ doesn't contain both a $\zerosf$ and a $\twosf$.
Thus, $vwx$ has no $\zerosf$'s or $vwx$ has no $\twosf$'s, so that
there are two cases to consider.

Suppose $vwx$ has no $\zerosf$'s.  Thus $vx$ has no $\zerosf$'s.
By (2), we have that $vx$ contains a $\onesf$ or a $\twosf$.
Thus $uwy$:
\begin{itemize}
\item has $n$ $\zerosf$'s;

\item has less than $n$ $\onesf$'s or has less than $n$ $\twosf$'s.
\end{itemize}
But (3) tells us that $uwy=uv^0wx^0y\in L$, so that $uwy$ has an equal
number of $\zerosf$'s, $\onesf$'s and $\twosf$'s---contradiction.
The case where $vwx$ has no $\twosf$'s is similar.
Since we obtained a contradiction in both cases, we have an overall
contradiction.  Thus $L$ is not context-free.

When we prove the pumping lemma for context-free languages, we will
make use of a fact about grammars in Chomsky Normal Form.
Suppose $G$ is a grammar in CNF and that $w\in(\alphabet\,G)^*$
is the yield of a valid parse tree $\pt$ for $G$ whose root label
is a variable.
For instance, if $G$ is the grammar with variable $\Asf$ and
productions $\Asf\fun\Asf\Asf$ and $\Asf\fun\zerosf$, then
$w$ could be $\mathsf{0000}$ and $\pt$ could be the following
tree of height $3$:
\begin{center}
\input{chap-4.10-fig1.eepic}
\end{center}

Generalizing from this example, we can see that if $\pt$ has height
$3$, $|w|$ will never be greater than $4=2^2=2^{3-1}$.  Generalizing
still more, we can prove that, for all parse trees $\pt$, for all
strings $w$, if $w$ is the yield of $\pt$, then $|w|\leq { 2^{k-1}}$.
This can be proved by induction on $\pt$.

\begin{proof}
Suppose $L$ is a context-free language.  By the results of the
preceding section, there is a grammar $G$ in Chomsky Normal Form such
that $L(G)=L-\{\%\}$.  Let $k=|Q_G|$ and $n= 2^k$.  Thus
$n\in\nats-\{0\}$. Suppose $z\in \Str$, $z\in L$ and $|z|\geq n$.
Since $n\geq 1$, we have that $z\neq\%$.  Thus $z\in L-\{\%\}=L(G)$,
so that there is a parse tree $\pt$ such that $\pt$ is valid for $G$,
$\rootLabel\,\pt=s_G$ and $\yield\,\pt=z$.  By our fact about CNF
grammars, we have that the height of $\pt$ is at least $k+1$.  (If
$\pt$'s height were only $k$, then $|z|\leq 2^{k-1}<n$, which is
impossible.)

The rest of the proof can be visualized using the
diagram
\begin{center}
\input{chap-4.10-fig2.eepic}
\end{center}

Let $\pat$ be a valid path for $\pt$ whose length is equal to the
height of $\pt$.  Thus the length of $\pat$ is at least $k+1$, so that
the path visits at least $k+1$ variables, with the consequence that at
least one variable must be visited twice.  Working from the last
variable visited upwards, we look for the first repetition of
variables.  Suppose $q$ is this repeated variable, and let $\pat'$
and $\pat''$ be the initial parts of $\pat$ that take us
to the upper and lower occurrences of $q$, respectively.

Let $\pt'$ and $\pt''$ be the subtrees of $\pt$ at positions
$\pat'$ and $\pat''$, i.e., the positions of the upper and lower
occurrences of $q$, respectively.
Consider the tree formed from $\pt$ by replacing the subtree
at position $\pat'$ by $q$.  This tree has yield
$uqy$, for unique strings $u$ and $y$.

Consider the tree formed from $\pt'$ by replacing the subtree
$\pt''$ by $q$.  More precisely, form the path $\pat'''$ 
by removing $\pat'$ from the beginning of $\pat''$.
Then replace the subtree of $\pt'$ at position $\pat'''$ by
$q$.  This tree has yield $vqx$, for unique strings $v$ and $x$.

Furthermore, since $|\pat|$ is the height of $\pt$, the
length of the path formed by removing $\pat'$ from $\pat$ will be the
height of $\pt'$.  But we know that this length is at most $k+1$,
because, when working upwards through the variables visited by $\pat$,
we stopped as soon as we found a repetition of variables.  Thus the
height of $\pt'$ is at most $k+1$.

Let $w$ be the yield of $\pt''$.  Thus $vwx$ is the yield of $\pt'$,
so that $z=uvwxy$ is the yield of $\pt$.  Because the height of $\pt'$
is at most $k+1$, our fact about valid parse trees of grammars in CNF,
tells us that $|vwx|\leq 2^{(k+1)-1} = 2^k=n$, showing that Part~(1)
holds.

Because $G$ is in CNF, $\pt'$, which has $q$ as its root label, has
two children.  The child whose root node isn't visited by $\pat'''$
will have a non-empty yield, and this yield will be a prefix of $v$,
if this child is the left child, and will be a suffix of $x$, if this
child is the right child.  Thus $vx\neq\%$, showing that Part~(2)
holds.

It remains to show Part~(3), i.e., that $uv^iwx^iy\in L(G)\sub L$, for
all $i\in\nats$.  We define a valid parse tree $\pt_i$ for $G$, with root
label $q$ and yield $v^iwx^i$, by recursion on $i\in\nats$.  We let $\pt_0$ be
$\pt''$.  Then, if $i\in\nats$, we form $\pt_{i+1}$ from $\pt'$ by
replacing the subtree at position $\pat'''$ by $\pt_i$.

Suppose $i\in\nats$.  Then the parse tree formed from $\pt$ by
replacing the subtree at position $\pat'$ by $\pt_i$ is valid for $G$,
has root label $s_G$, and has yield $uv^iwx^iy$, showing that
$uv^iwx^iy\in L(G)$.
\end{proof}

\subsection{Experimenting with the Pumping Lemma Using Forlan}

The Forlan module \texttt{PT} defines a type and several functions
that implement the idea behind the pumping lemma:
\begin{verbatim}
type pumping_division = (pt * int list) * (pt * int list) * pt

val checkPumpingDivision       : pumping_division -> unit
val validPumpingDivision       : pumping_division -> bool
val strsOfValidPumpingDivision :
      pumping_division -> str * str * str * str * str
val pumpValidPumpingDivision   : pumping_division * int -> pt
val findValidPumpingDivision   : pt -> pumping_division
\end{verbatim}
\index{PT@\texttt{PT}!pumping_division@\texttt{pumping\_division}}%
\index{PT@\texttt{PT}!checkPumpingDivision@\texttt{checkPumpingDivision}}%
\index{PT@\texttt{PT}!validPumpingDivision@\texttt{validPumpingDivision}}%
\index{PT@\texttt{PT}!strsOfValidPumpingDivision@\texttt{strsOfValidPumpingDivision}}%
\index{PT@\texttt{PT}!pumpValidPumpingDivision@\texttt{pumpValidPumpingDivision}}%
\index{PT@\texttt{PT}!findValidPumpingDivision@\texttt{findValidPumpingDivision}}%
A \emph{pumping division} is a triple $((\pt_1, \pat_1), (\pt_2,
\pat_2), \pt3)$, where $\pt_1, \pt_2, \pt_3\in\PT$ and $\pat_1,
\pat_2\in\List\,\ints$.  We say that a pumping division $((\pt_1,
\pat_1), (\pt_2, \pat_2), \pt_3)$ is \emph{valid} iff
\begin{itemize}
\item $\pat_1$ is a valid path for $\pt_1$, pointing to a leaf whose
  label isn't $\%$;

\item $\pat_2$ is a valid path for $\pt_2$, pointing to a leaf whose
  label isn't $\%$;

\item the label of the leaf of $\pt_1$ pointed to by $\pat_1$ is equal
  to the root label of $\pt_2$;

\item the label of the leaf of $\pt_2$ pointed to by $\pat_2$ is equal
  to the root label of $\pt_2$;

\item the root label of $\pt_3$ is equal to the root label of $\pt_2$;

\item the yield of $\pt_2$ has at least two symbols;

\item the yield of $\pt_1$ has only one occurrence of the root label
  of $\pt_2$;

\item the yield of $\pt_2$ has only one occurrence of the root label
  of $\pt_2$; and

\item the yield of $\pt_3$ does not contain the root label of $\pt_2$.
\end{itemize}
The function \texttt{checkPumpingDivision} checks whether a pumping division
is valid, silently returning \texttt{()} if it is, and explaining why
it isn't, otherwise.  The function \texttt{validPumpingDivision} tests
whether a pumping division is valid.

When the function \texttt{strsOfValidPumpingDivision} is applied to a
valid pumping division $((\pt_1, \pat_1), (\pt_2, \pat_2), \pt_3)$, it
returns $(u, v, w, x, y)$, where:
\begin{itemize}
\item $u$ is the prefix of $\yield\,\pt_1$ that precedes the unique
  occurrence of the root label of $\pt_2$;

\item $v$ is the prefix of $\yield\,\pt_2$ that precedes the unique
  occurrence of the root label of $\pt_2$;

\item $w = \yield\,\pt_3$;

\item $x$ is the suffix of $\yield\,\pt_2$ that follows the unique
  occurrence of the root label of $\pt_2$; and

\item $y$ is the suffix of $\yield\,\pt_1$ that follows the unique
  occurrence of the root label of $\pt_2$.
\end{itemize}
The function issues an error message if the supplied pumping division
isn't valid.

When the function \texttt{pumpValidPumpingDivision} is applied to the
pair of a
valid pumping division $((\pt_1, \pat_1), (\pt_2, \pat_2), \pt_3)$
and a natural number $n$,
it returns $\update(\pt_1, \pat_1, \pow\,n)$, where the function
$\pow\in\nats\fun\PT$ is defined by:
\begin{align*}
  \pow\,0 &= \pt_3 , \\
  \pow(n + 1) &= \update(\pt_2, \pat_2, \pow\,n) ,
  \eqtxt{for all} n\in\nats .
\end{align*}
The function issues an error message if its first argument isn't valid,
or its second argument is negative.

When the function \texttt{findValidPumpingDivision} is called with a
parse tree $\pt$, it tries to find a valid pumping division $\pd$ such
that
\begin{displaymath}
\mathtt{pumpValidPumpingDivision}(\pd, 1) = \pt .  
\end{displaymath}
It works as follows. First, the leftmost, maximum length path $\pat$
through pt is found. If this path points to $\%$, then an error
message is issued. Otherwise, \texttt{findValidPumpingDivision}
generates the following list of variables paired with prefixes of
$\pat$:
\begin{itemize}
\item the root label of the subtree pointed to by the path consisting
  of all but the last element of $\pat$, paired with that path;

\item the root label of the subtree pointed to by the path consisting
  of all but the last two elements of $\pat$, paired with that path;

\item \ldots;

\item the root label of the subtree pointed to by the path consisting
  of the first element of $\pat$, paired with that path; and

\item the root label of the subtree pointed to by $[\,]$, paired with $[\,]$. 
\end{itemize}
(Of course, the left-hand side of the last of these pairs is the root
label of $\pt$.)
As it works through these pairs, it looks for the first
repetition of variables. If there is no such repetition, it issues an
error message. Otherwise, suppose that:
\begin{itemize}
\item $q$ was the first repeated variable;

\item $\pat_1$ was the path paired with $q$ at the point of the first
  repetition; and

\item $\pat'$ was the path paired with $q$ when it was first seen. 
\end{itemize}
Now, it lets:
\begin{itemize}
\item $\pat_2$ be the result of dropping $\pat_1$ from the beginning of
  $\pat'$;

\item $\pt_1$ be $\update(\pt, \pat_1, q)$;

\item $pt'$ be the subtree of $\pt$ pointed to by $\pat_1$;

\item $pt_2$ be $\update(\pt', \pat_2, q)$;

\item $\pt_3$ be the subtree of $\pt'$ pointed to by $\pat_2$; and

\item $\pd = ((\pt_1, \pat_1), (\pt_2, \pat_2), \pt_3)$.
\end{itemize}
If $\pd$ is a valid pumping division (only the last four conditions of
the definition of validity remain to be checked), it is returned by
\texttt{findValidPumpingDivision}. Otherwise, an error message is
issued.

For example, suppose that \texttt{gram} is bound to the
grammar
\begin{align*}
\Asf &\fun \% \mid \zerosf\Bsf\Asf \mid \onesf\Csf\Asf , \\
\Bsf &\fun \onesf \mid \zerosf\Bsf\Bsf , \\
\Csf &\fun \zerosf \mid \onesf\Csf\Csf .
\end{align*}
Then we can proceed as follows:
\begin{list}{}
{\setlength{\leftmargin}{\leftmargini}
\setlength{\rightmargin}{0cm}
\setlength{\itemindent}{0cm}
\setlength{\listparindent}{0cm}
\setlength{\itemsep}{0cm}
\setlength{\parsep}{0cm}
\setlength{\labelsep}{0cm}
\setlength{\labelwidth}{0cm}
\catcode`\#=12
\catcode`\$=12
\catcode`\%=12
\catcode`\^=12
\catcode`\_=12
\catcode`\.=12
\catcode`\?=12
\catcode`\!=12
\catcode`\&=12
\ttfamily}
\small
\item[]\textsl{-\ }val\ pt\ =\ Gram.parseAlphabet\ gram\ (Str.input\ "");
\item[]\textsl{@\ }1110010010
\item[]\textsl{@\ }.
\item[]\textsl{val\ pt\ =\ -\ :\ pt}
\item[]\textsl{-\ }PT.output("",\ pt);
\item[]\textsl{A}
\item[]\textsl{(1,\ C(1,\ C(1,\ C(0),\ C(0)),\ C(1,\ C(0),\ C(0))),}
\item[]\textsl{\ A(1,\ C(0),\ A(%)))}
\item[]\textsl{val\ it\ =\ ()\ :\ unit}
\item[]\textsl{-\ }val\ pd\ =\ PT.findValidPumpingDivision\ pt;\ 
\item[]\textsl{val\ pd\ =\ ((-,\symbol{'133}2,2\symbol{'135}),(-,\symbol{'133}2\symbol{'135}),-)\ :\ PT.pumping_division}
\item[]\textsl{-\ }val\ ((pt1,\ pat1),\ (pt2,\ pat2),\ pt3)\ =\ pd;
\item[]\textsl{val\ pt1\ =\ -\ :\ pt}
\item[]\textsl{val\ pat1\ =\ \symbol{'133}2,2\symbol{'135}\ :\ int\ list}
\item[]\textsl{val\ pt2\ =\ -\ :\ pt}
\item[]\textsl{val\ pat2\ =\ \symbol{'133}2\symbol{'135}\ :\ int\ list}
\item[]\textsl{val\ pt3\ =\ -\ :\ pt}
\item[]\textsl{-\ }PT.output("",\ pt1);
\item[]\textsl{A(1,\ C(1,\ C,\ C(1,\ C(0),\ C(0))),\ A(1,\ C(0),\ A(%)))}
\item[]\textsl{val\ it\ =\ ()\ :\ unit}
\item[]\textsl{-\ }PT.output("",\ pt2);
\item[]\textsl{C(1,\ C,\ C(0))}
\item[]\textsl{val\ it\ =\ ()\ :\ unit}
\item[]\textsl{-\ }PT.output("",\ pt3);
\item[]\textsl{C(0)}
\item[]\textsl{val\ it\ =\ ()\ :\ unit}
\item[]\textsl{-\ }val\ (u,\ v,\ w,\ x,\ y)\ =\ PT.strsOfValidPumpingDivision\ pd;
\item[]\textsl{val\ u\ =\ \symbol{'133}-,-\symbol{'135}\ :\ str}
\item[]\textsl{val\ v\ =\ \symbol{'133}-\symbol{'135}\ :\ str}
\item[]\textsl{val\ w\ =\ \symbol{'133}-\symbol{'135}\ :\ str}
\item[]\textsl{val\ x\ =\ \symbol{'133}-\symbol{'135}\ :\ str}
\item[]\textsl{val\ y\ =\ \symbol{'133}-,-,-,-,-\symbol{'135}\ :\ str}
\item[]\textsl{-\ }(Str.toString\ u,\ Str.toString\ v,\ Str.toString\ w,
\item[]\textsl{=\ }\ Str.toString\ x,\ Str.toString\ y);
\item[]\textsl{val\ it\ =\ ("11","1","0","0","10010")}
\item[]\textsl{\ \ :\ string\ \symbol{'052}\ string\ \symbol{'052}\ string\ \symbol{'052}\ string\ \symbol{'052}\ string}
\item[]\textsl{-\ }val\ pt'\ =\ PT.pumpValidPumpingDivision(pd,\ 2);
\item[]\textsl{val\ pt'\ =\ -\ :\ pt}
\item[]\textsl{-\ }PT.output("",\ pt');
\item[]\textsl{A}
\item[]\textsl{(1,\ C(1,\ C(1,\ C(1,\ C(0),\ C(0)),\ C(0)),\ C(1,\ C(0),\ C(0))),}
\item[]\textsl{\ A(1,\ C(0),\ A(%)))}
\item[]\textsl{val\ it\ =\ ()\ :\ unit}
\item[]\textsl{-\ }Str.output("",\ PT.yield\ pt');
\item[]\textsl{111100010010}
\item[]\textsl{val\ it\ =\ ()\ :\ unit}
\end{list}


\subsection{Consequences of Pumping Lemma}

\index{context-free language!pumping lemma!consequences}%
\index{pumping lemma!context-free languages!consequences}%
We are now in a position to show that the context-free languages are
\emph{not} closed under either intersection or set difference.
Suppose
\begin{align*}
  L &= \setof{\zerosf^n\onesf^n\twosf^n}{n\in\nats} , \\
  A &= \setof{\zerosf^n\onesf^n\twosf^m}{n,m\in\nats} , \eqtxtl{and} \\
  B &= \setof{\zerosf^n\onesf^m\twosf^m}{n,m\in\nats} .
\end{align*}
As we proved above, $L$ is not context-free.  In contrast, it's easy
to find grammars generating $A$ and $B$, showing that $A$ and $B$ are
context-free.  But $A\cap B=L$, and thus we have a counterexample to
the context-free languages being closed under intersection.

Now, we have that
$\{\zerosf,\onesf,\twosf\}^*-A$ context-free, since it is the
union of the context-free languages
\begin{gather*}
\{\zerosf,\onesf,\twosf\}^*-\{\zerosf\}^*\{\onesf\}^*\{\twosf\}^*
\end{gather*}
and
\begin{gather*}
\setof{\zerosf^{n_1}\onesf^{n_2}\twosf^m}{n_1,n_2,m\in\nats
\eqtxt{and}n_1\neq n_2} ,
\end{gather*}
(the first of these languages is regular), and the context-free languages
are closed under union.
Similarly, we have that $\{\zerosf,\onesf,\twosf\}^*-B$ is
context-free.

Let
\begin{gather*}
C=(\{\zerosf,\onesf,\twosf\}^*-A) \cup (\{\zerosf,\onesf,\twosf\}^*-B) .
\end{gather*}
Thus $C$ is a context-free subset of $\{\zerosf,\onesf,\twosf\}^*$.
Since $A,B\sub\{\zerosf,\onesf,\twosf\}^*$, it is easy to show that
\begin{align*}
A\cap B &= \{\zerosf,\onesf,\twosf\}^*-
((\{\zerosf,\onesf,\twosf\}^*-A)\cup(\{\zerosf,\onesf,\twosf\}^*-B)) \\
&= \{\zerosf,\onesf,\twosf\}^*-C .
\end{align*}
Thus
\begin{gather*}
\{\zerosf,\onesf,\twosf\}^*-C = A\cap B = L
\end{gather*}
is not context-free, giving us a counterexample to the
context-free languages being closed under set difference.
Of course, this is also a counterexample to the context-free
languages being closed under complementation.

\subsection{Notes}

Apart from the subsection on Forlan's support for experimenting with
the pumping lemma, the material in this section is completely
standard.

\index{pumping lemma!context-free languages|)}%
\index{context-free language!pumping lemma|)}%

%%% Local Variables: 
%%% mode: latex
%%% TeX-master: "book"
%%% End: 


%%% Local Variables: 
%%% mode: latex
%%% TeX-master: "book"
%%% End: 
